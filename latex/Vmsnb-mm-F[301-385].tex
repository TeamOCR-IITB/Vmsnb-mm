\documentclass[11pt, openany]{book}
\usepackage[text={4.65in,7.45in}, centering, includefoot]{geometry}
\usepackage[table, x11names]{xcolor}
\usepackage{fontspec,realscripts}
\usepackage{polyglossia}
\setdefaultlanguage{sanskrit}
\setotherlanguage{english}
\setmainfont[Scale=1]{Shobhika}
\newfontfamily\s[Script=Devanagari, Scale=0.9]{Shobhika}
\newfontfamily\regular{Linux Libertine O}
\newfontfamily\en[Language=English, Script=Latin]{Linux Libertine O}
\newfontfamily\ab[Script=Devanagari, Color=purple]{Shobhika-Bold}
\newfontfamily\qt[Script=Devanagari, Scale=1, Color=violet]{Shobhika-Regular}
\newcommand{\devanagarinumeral}[1]{
\devanagaridigits{\number \csname c@#1\endcsname}} % for devanagari page numbers
\XeTeXgenerateactualtext=1 % for searchable pdf
\usepackage{enumerate}
\pagestyle{plain}
\usepackage{fancyhdr}
\pagestyle{fancy}
\renewcommand{\headrulewidth}{0pt}
\usepackage{afterpage}
\usepackage{multirow}
\usepackage{multicol}
\usepackage{wrapfig}
\usepackage{vwcol}
\usepackage{microtype}
\usepackage{amsmath,amsthm, amsfonts,amssymb}
\usepackage{mathtools}% <-- new package for rcases
\usepackage{graphicx}
\usepackage{longtable}
\usepackage{setspace}
\usepackage{footnote}
\usepackage{perpage}
\MakePerPage{footnote}
\usepackage{xspace}
\usepackage{array}
\usepackage{emptypage}
\usepackage{hyperref}% Package for hyperlinks
\hypersetup{colorlinks, citecolor=black, filecolor=black, linkcolor=blue, urlcolor=black}
\begin{document}
{२९२ वीरमित्रोदयस्य श्राद्धप्रकाशे-}{\\
~\\
}{कालजीवनाः भवन्तीति निन्दया एकाहेन कर्त्तव्यताया अप्रशस्त\\
त्वाद्विनद्वयकर्त्तव्यतायामतिप्राशस्त्यम् । महत्सु कर्मसु पूर्वेद्युर\\
ल्पेषु तदहरेव-\\
तथा च गृह्यपरिशिष्टम्-\\
महत्सु पूर्वेद्युस्तदहरल्पेष्विति ।\\
यदा तु पृथक्करणशक्त्या एकस्मिन्नेव वासर एकोपक्रमेण कर्त्त\\
व्यं तदा वैश्वदेविकमिति प्रकृतिभूते दर्शे ।\\
शातातपः ।\\
पृथद्विनेष्वशक्तश्चेदेकस्मिन्पूर्ववासरे ।\\
श्राद्धत्रयं प्रकुर्वीत वैश्वदेवं तु तान्त्रिकम् ॥\\
"अत्र च तन्त्र वा वैश्वदैविक" मिति प्रकृतिभूते दर्शश्राद्धे वैश्व.\\
देविकस्य तन्त्रत्वे विकल्पाभिधानात्तद्विकृतिभूते वृद्धिश्राद्धे
तन्त्रवि\\
कल्पप्राप्तौ विकल्पनिरासेनात्र तन्त्रतैव नियम्यते'' वैश्वदेवन्तु ता.\\
न्त्रिकम्" इति ।\\
मातृपूजने विशेषः-\\
कूर्मपुराणे ।\\
पूर्वं तु मातरः पूज्या भक्त्या वै सगणेश्वरा इति ।\\
मातृणां पूजनं नामानि चोक्तानि -\\
चतुर्विंशतिमते ।\\
तिस्रः पूज्या पितृपक्षे तिस्रः पूज्यास्तु मातृके ।\\
इत्येता मातरः प्रोक्ताः पितृमातृष्वसाष्टमी ॥\\
ब्रह्मण्याद्याः स्मृताः सप्त दुर्गाक्षेत्रगणाधिपाः ।\\
वृद्धौ वृद्धौ सदा पूज्याः पश्चान्नान्दीमुखान पितॄन् ॥\\
गौरी पद्मा शची दक्षा सावित्री विजया जया ।\\
देवसेना स्वधा स्वाहा मातरो लोकमातरः ।\\
एतास्त्वभ्युदये पूज्या आत्मदेवतया सह ॥\\
लोकमातर इति सर्वासां विशेषणमिति मदनरत्नः । पितृपक्षे = पितृवर्गे ।\\
तिस्रो = मातृपितामहीप्रपितामह्यः पूज्याः । तथा मातृके= मातामहवर्गे ।\\
तिस्रो = मातामहीमातुः पितामहीमातुः प्रपितामह्यः । पितृमातृष्वसाष्टमी =\\
पितृष्वसा सप्तमी मातृष्वसाष्टमी च पूज्या इत्यष्टौ मनुष्यमातरः प्रो-\\
क्ता। एताश्च प्रत्यक्षा: प्रत्यक्षमेव कुङ्कुमकुसुमवस्त्राभरणभोजनैः पू-\\


{ वृद्धिश्राद्धनिरूपणम् । २९३\\
जनीयाः । ब्रह्मण्याद्यास्तथा सप्त=ब्राह्मी वैष्णवी माहेश्वरी ऐन्द्री
वाराही\\
कौमारी चामुण्डेत्येताः सप्त देवमातॄः । तथा दुर्गाक्षेत्रगणाधिपान्=
दुर्गां\\
क्षेत्राधिपं गणाधिपं च वृद्ध्यादौ वृद्धिश्राद्धात्प्राक्
षोडशभिरुपचारैः\\
पूजयित्वा पश्चान्नान्दीभुखान्पितॄन् = श्राद्धेन पूजयेदिति ।
आत्मदेवता=स्वकी·\\
यकुलदेवता । तया सहिताः पूज्याः । अन्यत्र स्मृत्यन्तरे ब्रह्माण्या\\
द्यास्त्वष्टौ दर्शिताः । ब्रह्माणी माहेशी कौमारी वैष्णवी च वाराही-\\
न्द्राणी चामुण्डा महालक्ष्मीश्च मातरः प्रोक्ताः ।\\
शातातपोऽप्याह ।\\
गौरी पद्मा शची मेधा सावित्री विजया जया ।\\
देवसेना स्वधा स्वाहा मातरो लोकमातरः ॥\\
धृतिः पुष्टिस्तथा तुष्टिरात्मदेवतया सह ।\\
आभ्योऽर्घे गन्धपुष्पं च धूपं दीपं निवेदयेत् \textbar{}\\
तत्र मातृपूजनं कर्माङ्गश्राद्धेऽपि कार्यमिति षट्त्रिंशन्मत ए-\\
वोक्तम् ।\\
कर्मादिषु च सर्वेषु मातरः सगणाधिपाः ।\\
पूजनीयाः प्रयत्तेन पूजिताः पूजयन्ति ताः \textbar{}\textbar{}\\
आसां च पूजाधिष्ठानं वसोर्धारानिर्माणादिकं च तत्रैवोकम् ।\\
प्रतिमासु च शुभ्रासु लिखित्वा च पटादिषु ।\\
अपि वाक्षतपुखेषु नैवेद्यैश्च पृथग्विधैः ॥\\
अत्र प्रतिमालेखाक्षतपुञ्जानां }{पूर्वं}{ }{पूर्वं}{ फलातिशयार्थम् ।\\
कुड्यलग्ना वसोर्धारा पञ्चधारा घुतेन तु ।\\
कारयेत् सप्त वा धारा नातिनीचा न चोच्छ्रिताः ॥\\
आयुष्याणि च शान्त्वर्थं जप्त्वा तत्र समाहितः ।\\
आयुष्याणि = आनो भद्रा इति ऋग्वेदोक्तानि । तथा यानि आशीः\\
प्रधानानि सुक्तानि, तथाविधमन्त्राधाराणि सामानि च ।\\
एतन्मातृपूजनस्याकरणे भविष्यत्पुराणे दोषो दर्शितः ।\\
अकृत्वा मातृयज्ञं तु यः श्राद्धं परिषेषयेत् ॥\\
तास्तस्य क्रोधसंपन्ना हिंसां कुर्वन्ति दारुणाम् ।\\
आसां स्वरूपं प्रोक्तम्-\\
मार्कण्डेयपुराणे ।\\
हंसयुक्तविमानाग्रे साक्षसूत्रकमण्डलुः ।

{२९४ वीरमित्रोदयस्य श्राद्धप्रकाशे-\\
आयाता ब्रह्मणः शक्तिर्ब्रह्माणी सामिधीयते ॥\\
माहेश्वरी वृषारूढा त्रिशुलवरधारिणी ।\\
महाहिवलया प्राप्ता चन्द्ररेखाविभूषणा \textbar{}\textbar{}\\
कौमारी शक्तिहस्ता च मयूरवरवाहना ।\\
योद्धुमभ्याययौ दैत्यानम्बिका गुहरूपिणी ॥\\
तथैव वैष्णवी शक्तिर्गरुडोपरि संस्थिता ।\\
शङ्खचक्रगदाशार्ङ्गखड्गहस्ताभ्युपाययौ\\
यज्ञवाराहमतुलं रूपं या विभ्रतो हरेः ।\\
शक्तिः साप्याययौ तत्र वारहीं विभ्रती तनुम् ॥\\
वज्रहस्ता तथैवैन्द्री गजराजोपरि स्थिता \textbar{}\\
प्राप्ता सहस्रनयना यथाशक्रस्तथैव सा ॥\\
ब्रह्मपुराणे ।\\
महालक्ष्मीश्च कर्त्तव्या नृत्यमाना कपालिनी ।\\
अत्र गौरीस्वरूपं-\\
कार्त्तिकी तुङ्गखट्वाङ्गत्रिमालाम्बुजधारिणी ।\\
कुष्माण्डा पातु प्रेतस्था दन्तुरा वर्वरंशिरः ?॥\\
सिद्धार्थपृच्छायाम्\\
चन्द्रार्द्धनवचक्षुर्म्यां भास्करेणाविषयेण ? च।\\
विद्योतन्ते करा यस्याः पद्मा सा च प्रकीर्त्तिताः ।\\
कीरक व्यजनीं चैव व्यजनं समिधं तथा ।\\
या बिभर्त्ति कराने सा शची नाम्ना प्रकीर्त्तिता ॥\\
धत्ते या समिधं हस्तैः व्यजनीकरिकावपि ।\\
क्रमतस्तालवृन्तं च सा च मेधाभिधीयते ॥\\
अक्षसूत्रं स्रुवं चैव शक्तिकाञ्च कमण्डलुम् ।\\
कलयन्ति करा यस्याः सावित्री सा प्रकीर्त्तिता ॥\\
कार्मुकं चार्धपात्रं च योगमुद्रां कृपाणिकाम् ।\\
पाणिभिर्या क्रमाद् धत्ते सा दुर्गा परिगीयते ॥\\
योगमुद्रां च चापं च क्रमेणैवार्धभाजनम् ।\\
कृपाणीं चैव या धत्ते सा चामुण्डा प्रकीर्त्तिता ।\\
या कपालं च कौशं च पीयूष पाशमेव च ।\\
आबिभर्ति कराम्भोजैर्विजया सा प्रकीर्त्तिता ॥

{ वृद्धिश्राद्धनिरूपणम् \textbar{} २९५\\
क्रमास्कोशकपाले च पीयूषं च सपाशकम् ।\\
आश्रित्य यत्कराग्राणि विजयन्ते जया च सा ॥\\
तत्वस्रागरसहितायाम् ।\\
दर्पणं पङ्कजं पद्मं पुष्टिक दधतीं स्मरेत् ।\\
पुष्टीं लक्ष्मीं सितां सौम्यां पङ्कजासनसंस्थिताम् \textbar{}\textbar{}\\
बज्रं पद्मं तथा दर्शं पीयूष दधती स्मरेत् ।\\
सा या तुष्टिर्महालक्ष्मीः स्थिता वीरासनोपरि ॥\\
विनायक प्रतिमास्वरूपं मत्स्यपुराणे-\\
स्वदन्तं दक्षिणकरे उत्पन्नं {[} चाक्षसूत्रं {]} तथापरे ।\\
लड्द्रुकं परशु चैव वामतः संप्रकल्पयेत् ।\\
संयुक्तं बुद्धिऋद्धिभ्यामधस्तान्मृलकान्वितम् ।\\
तदेवं विशेषतः कासांचिन्मातॄणां स्वरूपं प्रदर्शितम् ।\\
सामान्याकारेण तु सर्वासां स्वरूप पञ्चरात्र उक्तम् ।\\
पूज्याश्चित्रे तथार्चायां वरदाभयपाणय. ॥ इति ।\\
एवं मातृकापूजनं वसोर्धारापर्यन्तं कृत्वा नान्दीश्राद्धं कार्यम् ।\\
तहुच नवदेवत्यम् ।\\
मातरः प्रथमं पूज्याः पितरस्तदनन्तरम् ।\\
ततो मानामहानां च वृद्धौ श्राद्धत्रयं स्मृतम् ॥\\
इतिमात्स्यात् । कात्यायनवचने-\\
ब्रह्मपुराणम् ।\\
षड्भ्यः पितृभ्यस्तद्नु श्राद्धदानमुपक्रमेत् ।\\
वसिष्ठोको विधिः कृत्स्नो द्रष्टव्योऽत्र निरामिषः ॥\\
इत्यत्र षड्भ्य इति षड्ग्रहणं मातृवर्गस्याप्युपलक्षणं मातरः\\
प्रथममिति वाक्यात् \textbar{} आभिषं= मांसं तद्वर्ज्जित इति । अत्र
नान्दीमु\\
खान्पितॄन्पूजयेदित्यत्र नान्दीमुखाः = स्वपितृपितामहप्रपितामहाः ।\\
अत्र केचिन्नान्दीमुखानां पितॄणां देवतात्वाभिधानाद्येषां नान्दी\\
मुखसंज्ञा विहितास्ति ब्रह्मपुराणे तेषामेव देवतात्वम् । तथा च-\\
ब्रह्मपुराणम् ।

{ पिता पितामहश्चैव तथैव प्रपितामहः ।\\
त्रयो ह्यश्रुमुखा ह्येते पितरः संप्रकीर्त्तिताः ॥\\
तेभ्यः पूर्वतरा ये च प्रजावन्तः सुखैधिताः ।\\
ते तु नान्दीमुखा नाम पितरः परिकीर्तिताः ।

{२९६ वीरमित्रोदयस्य श्राद्धप्रकाशे-\\
अत्र पित्रादीनां त्रयाणामश्रुमुखा इति संज्ञा, प्रपितामहपित्रादीनां\\
त्रयाणां नान्दीमुखा इति संज्ञा, तस्माद्वृद्धिश्राद्धे
वृद्धप्रपितामहादय\\
एव नान्दीमुखास्त एव }{च}{ देवता इत्याहुः \textbar{}\\
तन्न \textbar{} "स्वपितृभ्यः पिता तद्यात्सुतसंस्कारकर्मसु " इति छन्दोग\\
परिशिष्टवचनात्, "पूर्वेद्युमार्तृक श्राद्धं कर्माहे पैतृकं स्मृत" मिति
वसि\\
ष्ठोक्तत्वात्, "एवं प्रदज्ञिणावृत्को वृद्धौ नान्दीमुखान्पितॄन् "
यजेतेति\\
याज्ञवल्क्यवचनात्, नान्दीमुखाः पितरः पितामहाः प्रपितामहश्चेतिका\\
त्यायनवचनात्, नान्दीमुखेभ्यः पितृभ्यः पितामहेभ्यः प्रपितामहेभ्यर्श्च\\
ति गोभिलवचनात्, नान्दीमुखं पितृगणं पूजयेत्प्रयतो गृहीति विष्णुपुरा\\
णात्, तस्मात्पितृभ्यः पूर्वेद्युः करोति पितृभ्य एव तद्यशमिति पूर्वो-\\
दाहृतब्राह्मणाच्च पित्रादित्रिकस्यैव देवतात्वप्रतीतेः । नान्दीमुख\\
संज्ञा परं तेषामपि वाचनिकी । तेन पित्रादीनां त्रयाणामेव वृद्धौ\\
देवतात्वं नान्दीमुखत्वं चेति । अत्र च पित्रादीनां देवतानां
नान्दीमु\\
खत्वस्य गुणत्वात्तद्विशिष्टानामेवोल्लेखः, यथाग्नये दात्र इन्द्राय
प्रदात्रे\\
विष्णवे शिपिविष्टायेत्यत्राग्न्यादीनां देवतानां 'दात्रादिगुणविशिष्टा\\
नामेवोल्लेख इति । प्रपितामहपित्रादीनां तु नान्दीमुखत्वस्य गुणस्त्र\\
भावान्नोल्लेख इत्यग्रे स्फुटम् ।\\
किञ्च ।\\
माता पितामही चैव संपूज्या प्रपितामही ।\\
पित्रादयस्त्रयश्चैव मातुः पित्रादयस्त्रयः \textbar{}\textbar{}\\
एते नवार्चनीयाः स्युः पितरोऽभ्युदये द्विजैः ।\\
इत्याश्वलायनवचने स्पष्टमेव मात्रादीनां पितृपितामहप्रपितामहा-\\
नां मातामहादीनां नवानामभ्युदये देवतात्वाभिधानात् । तथा "मा-

{तृपूर्वान्पितॄन्पूज्य ततो मातामहानपि " इति चतुर्विंशतिमते, तथा-\\
नान्दीमुखा पितर इति कात्यायनवचनेऽपीति कः प्रसङ्ग उत्तर-\\
त्रिकस्य ।\\
किञ्च-\\
नान्दीमुखे विवाहे }{च}{ प्रपितामहपूर्वकम् ।\\
नाम संकीर्तयेद्विद्वानन्यत्र पितृपूर्वकम् ॥\\
इति वृद्धवसिष्ठवचने पितृपितामहप्रपितामहानामेव }{देवतात्वे}{\\
प्रातिलोम्यक्रमो घटते वृद्धौ प्रपितामहपितामहपितर इति व्यु-

{ वृद्धिश्राद्धनिरूपणम् \textbar{} २९७\\
त्क्रमः, आमावास्यादौ तु पितृपूर्वकं पितृपितामहप्रपितामहा इति\\
क्रमः । उत्तरत्रिकस्य तु देवतास्वे प्रपितामहपदं प्रपितामह {[} प्रपिता-\\
मह {]} परं स्यात् पितृपदं च वृद्धप्रपितामहपरं स्यादिति । किञ्च पितृ-\\
पितामहप्रपितामहानामेव देवतात्वे भवति जीवत्पितृकस्याप्यधि\\
कारः । तथा हि । ``येभ्य एव पिता दद्यत्तिभ्यो दद्यात्तु तत्सुतः: "\\
इत्यनेनाधिकारे सिद्धे विष्णूक्तः पितरि जीवति यः श्राद्धं कुर्याद्ये\\
षां पिता कुर्यादित्युक्त्वा त्रिषु जीवत्सु नैव कुर्यादिति निषेधोऽनर्थक\\
एव, त्रयाणां देवतात्वपक्षे तेषां जीवित्वे भवत्यर्थवान्निषेधः । यदा तू-\\
त्तरेषां देवतात्वं तदा त्रयाणां जीवनेऽजीवने वा उत्तरेषामेव प्राप्तत्वा\\
दय निषेधोऽनर्थक एव । किञ्च विष्णूक्तः प्रयोगनियमोऽपि पूर्वेषां\\
देवतात्वे घटते नोत्तरेषां यथा "यस्य पिता प्रेतः स्यात्स पित्रे पिण्डं\\
निधाय पितामहात्पराभ्यां द्वाभ्यां दद्यात्" इत्यत्र पित्रे प्रपितामहाय\\
च पिण्डदानं तयोर्देवतात्वे सङ्गच्छते नान्यथेति किम्भूयो विस्तरेण ।\\
वृद्धप्रपितामहादीनां संज्ञाकरणं तु प्रौष्ठपददीश्राद्ध एव देवतात्वार्थ,\\
तेषामन्यत्र देवतास्वे प्रमाणाभावात् । उक्तं च तत्रैव ।\\
नान्दीमुखानां प्रत्यब्द कन्याराशिगत रवौ ।\\
पौर्णमास्यां तु कर्त्तव्यं वराहवचनं यथा ॥ इति ।\\
अत उदाहृतेषु वचनेषु मातामह्यादित्रिदेवत्यमुक्तम् । चतुर्थम-\\
पि पार्वणं कर्त्तव्यतयोक्तं-\\
चतुर्विंशतिमते ।\\
पूजयेच्च ततः पश्चात्तत्र नाम्दीमुखान् पितॄन् ।\\
मातृपूर्वान्पितॄन् पूज्य ततो मातामहानपि ॥\\
मातामहीस्ततः कचिद्युग्मा भोज्या द्विजातयः ।\\
मातृवर्गे चतुरादियुग्मसंख्याका ब्राह्मणा भोजनीयाः । तथा च\\
मातृवर्गे चत्वारो मातामहीवर्गे चत्वारः । उक्तं च-\\
भविष्यपुराण \textbar{}\\
भोजयेच्च द्विजानष्टो मातृश्राद्धे खगेश्वर \textbar{} इति ।\\
असम्भवे त्वेकैकस्मिन्वर्गे द्वौ द्वावपि भोजनीयौ -\\
तथा च पद्मपुराणे ।\\
युग्मा द्विजातयः पूज्या वस्त्रकार्तस्वरादिभिः ।\\
वो० मि ३८

{२९८ वीरमित्रोदयस्य श्राद्धप्रकाशे-\\
कार्त्तस्वरं = सुवर्णं -\\
छागलेयः ।\\
एकैकस्य तु वर्गस्य द्वौ द्वौ विप्रौ समर्चयेत् ।\\
वैश्वदेवे तथा द्वा च न प्रसज्येत विस्तरः ॥ इति ।\\
एवं चाष्टौ ब्राह्मणाः सम्पद्यन्ते ।\\
भविष्यपुराणे तु नवमोऽध्युक्तः-\\
पूर्वाह्णे भोजयेद्विप्रानष्टौ सर्वं प्रदिक्षणम् ।\\
तथैव नवमं विप्रं चतुरस्त्रे खगोत्तम ॥ इति ।\\
विप्राणां चरणक्षालनार्थं गोमयेन क्रियमाणे चतुरस्रे मण्डले\\
तत्काले यः कश्चिदतिथिरागच्छत् स नवमोऽपि भोजनीय इत्यर्थः ।\\
मातृवर्गे मतामहीवर्गे वा ब्राह्मणालाम पतिपुत्रान्विताश्चतस्रश्चतस्त्रः\\
सुवासिन्यो भोजनीया इत्युक्त वृद्धवशिष्ठेन ।\\
मातृश्राद्धे तु विप्राणामलाभं पूजयेदपि ।\\
पतिपुत्रान्विता भव्या योषितोऽष्टौ सुदान्विताः ॥ इति ।\\
नान्दीश्राद्धस्य वैश्वदेविकपूर्वकत्वमुक्तं वृद्धशातापेन । ``प्रदक्षिणं\\
तु सव्येन भोजयेहेवपूर्वक "मिति । वृद्धेश्राद्धे च सत्यवसुसंज्ञका\\
विश्वदेवा: । अत्र कल्पतरुकारेणाभिहतमाभ्युदयिकप्रस्तावे\\
मार्कण्डेयपुराणे ।\\
वैश्वदेवविहीनं तु केचिदिच्छन्ति मानवाः । इति ।\\
हलायुधस्तु मातृश्राद्धं देवरहितमेव च कार्यमित्याह इति शूलपा\\
णौ । वृद्धिश्राद्धेऽन्येऽपि धर्माविशेषाः प्रतिपाद्यन्ते । तत्राह -\\
शातातपः ।\\
कर्तव्य चाभ्युदयिकं श्राद्धमभ्युदयार्थिना ।\\
सव्येन चोपवीतेन ऋजुदर्भैश्च धीमता ॥\\
पितॄणां रूपमास्थाय देवा ह्यन्नं समश्नुते ।\\
तस्मात्सव्येन दातव्यं वृद्धिश्राद्धे तु नित्यशः ॥\\
यथैवोपचरेहेवांस्तथा वृद्धौ पितॄनपि ।\\
समश्नुते = समश्नुवत इत्यर्थः ।\\
अत्र तिलकार्ये यवविनियोगः कार्यः । तथा च ।\\
विष्णुधर्मोत्तरे ।\\
वृद्धिश्राद्धेषु कर्तव्यास्तिलस्थाने यवास्तथा ॥ इति ।

{ वृद्धिश्राद्धनिरूपणम् । २९६\\
पित्र्यधर्मस्थाने }{धर्मान्तरप्रयोगस्तूक्तो-}{\\
ब्रह्माण्डपुराणे ।\\
स्वाहाशब्दं प्रयुञ्जीत स्वधास्थाने तु बुद्धिमान् ।\\
कुशस्थाने तु दूर्वाः स्युर्मङ्गलस्याभिवृद्धये ॥ इति ।\\
मार्कण्डेयपुराणे ।\\
उदङ्मुखः प्राङ्मुखो वा यजमानः समाहितः ।\\
वृद्धिश्राद्धं प्रकुर्वीत नान्यचक्त्रः कदाचन ॥ इति ।\\
अत्रोदङ्मुखत्वप्राङ्मुखत्वयोर्व्यवस्थोक्ता आश्वलायनगृह्यपरिशिष्टे ।\\
अभ्युदये युग्मा ब्राह्मणाः, अमूला दर्भा, प्राङ्मुखेभ्य उदङ्मु\\
खो दद्यात् उदङ्मुखेभ्यः प्राङ्मुखो द्वौ दर्भों पवित्र इति ।\\
आह - शातातपः ।\\
अपसव्यं न कुर्वीत न कुर्यादप्रदक्षिणम् ।\\
अपसव्येन यो दद्यादृद्धौ किञ्चिदतिक्रमात् ॥\\
न तस्य देवास्तृप्यन्ति पितरश्च यथाविधि ।\\
एतञ्च पित्र्यधर्मविवर्जितदैवधर्मयुक्तं नान्दीश्राद्धमित्थं कार्यम्
।\\
आह - जातूकर्ण्यः ।\\
}{पूर्वेद्युस्तद्दिने}{ वाथ }{दैवपूर्वं}{ निमन्त्रणम् \textbar{}\\
कृत्वा विप्रान्समाहूय }{पूर्वाह्णे}{ नियतः शुचिः ।\\
कृत्वा मण्डलकं तेषां क्षालयेश्चरणांस्ततः ।\\
आचान्तान्कृतसत्कारानासनेषूपवेशयेत् ॥\\
श्लोककात्यायनोऽपि ।\\
अतः परं प्रवक्ष्यामि विशेष इह यो भवेत् ।\\
प्रातरामन्त्रितान्विप्रान्युग्मानुभयतस्तथा ॥\\
उपवेश्य कुशान्दद्यादृजुनैव हि पाणिना ।\\
अस्यार्थः पूर्वेद्युस्तदहर्वा निमन्त्रितान्विप्रान्प्रातरेव
}{पूर्वाह्णे}{ }{एवा-}{\\
हूय }{चरणक्षालनानन्तरमृजुदर्भोपकल्पितासने}{ उपवेशयेत् । उभयत\\
इति = चतुरः पूर्वाभिमुखानेकत उपवेश्य अन्यतश्चतुर उदङ्मुखानु\\
पवेशयेत् ।\\
तथा च भविष्यपुराणे ।\\
}{प्राङ्मुखांश्चतुरश्चैव}{ चतुरश्च उदङ्मुखान् ।\\
निवेश्य }{ऋजुभिर्दर्भैदद्यादासनमादरात्}{ ॥ इति ।\\
·

{३०० }{वीरमित्रोदयस्य}{ श्राद्धप्रकाशे-\\
कल्पतरुकारस्तु । "प्रातरामन्त्रितानिति वचनाद्वैकल्पिकमपि सा-\\
यमामन्त्रणं निवर्त्तते । उभयत = पितृपक्षे वैश्वदेवपक्षे चेत्याह
\textbar{}\\
उभयत = पितृकृत्ये मातामहकृत्ये चेत्यर्थः । दैवे युग्मस्य प्राप्त\\
त्वाद्विशेषासम्भवादिति शूलपाणि । अथवा सर्वेऽपि प्राङ्मुखा एवोप-\\
वेशनयाः ।\\
आह छागलेयः ।\\
सर्वानेव तु तान्विप्रान्प्राङ्मुखानुपवेशयेत् ।\\
ब्रह्मपुराणे ।\\
विप्रान्प्रदक्षिणावर्त्तं प्राङ्मुखानुपवेशयेत् \textbar{}\textbar{}\\
शक्राग्नियमयात्वम्भोऽनिलचन्द्राशिवांशकान् ।\\
समान्प्रशस्तान्सुभगान्पुष्पमालाविभूषितान् ॥\\
दर्द्याहर्भासनं देवान्पितॄनुद्दिश्य तेषु च ।\\
यातु = रक्षोऽधिपः ।\\
वृद्धपाराशरः ।\\
मालत्या शतपत्र्या वा मल्लिकाकुब्जयोरपि ।\\
केतक्या पाटलाया वा देया मालानुलोहिताः ॥

{ तथा ।\\
सुवेषभूषणैस्तत्र सालङ्कारैस्तथा नरैः ।\\
कुङ्कुमाद्यनुलिप्ताङ्गंभाव्यं तु ब्राह्मणैः सह ॥\\
स्त्रियोऽपि स्युस्तथा भूता गीतनृत्यादिहर्षिताः । इति ।\\
दर्भेषु ऋजुत्ववदमूलत्वमप्युक्तमाश्वलायनकारिकायाम् ।\\
ऋजून्दर्भानमूलांस्तु दत्वैषामासनेष्वथ इति ।\\
छान्दोग्यग्रन्थेऽर्द्यपात्र संख्या उक्ता: । "
चत्वार्येवार्घपात्राण्याभ्युदयि\\
के" इति । एतानि मात्रादिश्राद्धत्रये त्रीणि वैश्वदेविके चैकमित्येव\\
विभजनीयानि ।\\
पात्राणा पूरणादीनि दैवेनैव हि कारयेत् ।\\
गोत्रनामभिरामन्त्र्य पितॄनर्घं प्रदापयेत् ॥\\
नात्रापसव्यकरणं न पित्र्यं तीर्थमिष्यते ।\\
ज्येष्ठोत्तरकरान्युग्मान्कराग्राग्रपवित्रकान् ॥\\
कृत्वार्ध्यं सम्प्रदातव्यं नेकैकस्य प्रदीयते ।\\
दैवेन = वैश्वदेविकधर्मेण स च धर्मो दर्भाणामृजुत्वं यवोपादानं\\
"

{ वृद्धिश्राद्धनिरूपणम् \textbar{} ३०१\\
~\\
यज्ञोपवीतित्वं, देवतीर्थं चेति । ज्येष्ठस्य प्रथमोपवेशितस्य उत्तर उपरि\\
करो हस्तो येषां }{प्रथमोपवेशितविप्रस्य}{ करः }{पश्चादुपवेशितविप्रकर-}{\\
स्योपरि यथा भवति तथा कृत्वेत्यर्थः । कराग्राग्रपवित्रकान् =
कराग्रेऽग्रं\\
पवित्रस्य येषां विप्रकरे प्रागग्रं पवित्र धृत्वेत्यर्थ ।
द्वयोर्द्वयोर्हस्तौ\\
संयोज्य तत्र प्रागग्रं पवित्रं स्थापयित्वा
एकैकस्मिन्मातृपित्रादिवर्गे\\
वैश्वदेविके च सकृदेवार्घः प्रदेयो नैकैकस्य विप्रस्य कर इति निष्कृ\\
ष्टोऽर्थः । अर्घपात्रे तिलस्थाने यवप्रक्षेपे तिलोऽसीत्ययं मन्त्रो
यवोऽ-\\
सीत्यूहविशिष्टः प्रयोक्तव्य इत्युक्तमाश्वलायनगृह्यपरिशिष्टे । द्वौ
}{दर्भौ}{\\
पवित्रे सोपयामानि पात्राणि चत्वारि । उपयाम = उपग्रह, }{पात्राधस्ता.}{\\
त्कुशधारणरूपः । शन्नो देवीत्यनुमन्त्रितास्वप्सु }{यवानावपति}{ । ``य\\
वोऽसि सोमदेवत्यो गोसवो देवनिर्मितः । प्रत्नमद्भिः पृक्तः
पुष्ट्याना-\\
न्दीमुखान्लोकान्प्रीणयाहि नः स्वाहेति । विश्वेदेवा इदं वोऽर्ध्यं
नान्दी-\\
मुखाः पितर इति यथालिङ्गमर्थ्यदानं पितरः प्रीयन्तामित्यपां प्रतिग्रह-\\
णं विसर्जन च, एवमुत्तरयोरपि पितामहप्रपितामहयोः, नित्यं चाग्नौ\\
करणं स्वाहाकारेण होमश्च "अतो देवा अवन्तु न" इत्यङ्गुष्ठग्रह इति ।\\
अत्राश्वलायन कारिकापि ।\\
तिलोऽसीति पदस्थाने }{यवोऽसीति}{ पदं वदेत् ।\\
स्वधयेति पदस्थाने }{पुष्ट्या}{ शब्दं वदेदिह ॥ इति ।\\
}{अर्घदानं}{ चावाहनपूर्वकं कार्यम् ।\\
आवाहनप्रकारे-\\
ब्रह्मपुराणे ।\\
}{नान्दीमुखान्पितॄन्भक्त्या}{ साञ्जलिश्च }{समाह्नयेत्}{ ।\\
पठेत्पवित्रं मन्त्रं तु विश्वेदेवास आगत ॥ इति ।\\
अर्घदानं च सत्यवसुसंज्ञका विश्वेदेवाः इदं वोऽर्घं
नान्दीमुखा:\\
पितरः इदं वोऽर्घमित्येवं रूपं कार्यम् आश्वलायनगृह्यवचनात् । 'अपः\\
प्रथमपात्रस्थाः }{स्वाहार्घ्या}{ इति मन्त्रिता" इति प्रयोगपारिजातकार-\\
लिखनात्स्वाहार्घ्या इत्येवं रूपं चार्धदानम् । अर्घदानानन्तरं
गन्धपु-\\
}{ष्पादिकं}{ देयम् ।\\
भविष्यपुराणे ।\\
}{कृत्वा यवैस्तिलार्थं}{ तु }{दद्यादर्घं}{ विधानतः ।\\
गन्धपुष्पादिकं सर्वं कुर्वाद्वीरप्रदक्षिणम् ॥

{३०२ }{वीरमित्रोदयस्य}{ श्राद्धप्रकाशे-\\
ब्रह्मपुराणेऽपि ।

{ अर्धं पुष्पं च धूपं च प्रशस्तमनुलेपनम् \textbar{}\\
वासश्चाप्यहतं शुद्धं देयं च सदृशं समम् ॥\\
अत्र दीपमित्यपि वक्तव्यम् । गन्धमाल्यपुष्पधूपदीपाच्छादनाना\\
प्रदानमित्याश्वलायनसूत्रितत्वात् । इदं च गन्धादिकमेकैकस्य हस्ते द्विः\\
र्द्विर्देयम् ।\\
तथा चाश्वलायनगृह्यपरिशिष्टे ।\\
प्रदक्षिणमुपचारो यवैस्तिलार्थः }{सर्वं}{ द्विर्द्विरिति ।\\
अत्र च सर्वं द्विर्द्विरिति वचनाद्गन्धादिपञ्चकस्यापि
द्विर्द्विर्दान-\\
मित्येव प्रतीयते द्विर्भुंङ्क इत्यत्रेव द्विर्द्विरिति सुचः क्रियाया
आवृत्तौ वि-

हितत्वाद् द्वव्यावृत्तिं विना तु क्रियावृत्तेरनुपपन्नत्वात्तेन
गन्धादिव-\\
द्वासोद्वयमपि देयम् । सममिति वस्त्राविशेषणादपि द्वित्वप्रतीतिरिति\\
केचित् । अन्यच्च -\\
भविष्यत्पुराणे ।\\
रक्तपुष्पतिलांश्चैव अपसव्य च वर्जयेदिति ।\\
इदं च गन्धादिभिरभ्यर्चनं पदार्थानुसमयेन काण्डानुसमयेन\\
वेति बोध्यम् ।\\
गन्धादिदानानन्तरं चाग्नौकिरणं कर्त्तव्यम् ।\\
गृह्यपरिशिष्टे

{ नित्यं चाग्नौ स्वाहाकारेण होमश्चेति ।\\
तत्रैव पाणौ होमः, अग्नये कव्यवाहनाय }{स्वाहा}{ \textbar{} सोमाय पितृ.\\
मते }{स्वाहा}{ इति । तथा पृषदाज्यमिश्र उष्णो हविः सर्वत्र तस्यार्द्धेन\\
द्वे द्वे आहुती जुहुयादिति । सर्वत्रेत्यग्नौकरणद्विजभोजनविकिरणपि.\\
ण्डदानेषु दधिमिश्रमाज्यं पृषदाज्यं तन्मिश्र ओदनो हविः श्राद्धद्रव्यं\\
तस्यौदनस्य पूर्वापरार्द्धे विभागं कल्पयित्वा द्विरवदाय, होमः कार्य -\\
इत्यर्थः । अत्र च कश्चिद्विशेषो-\\
ब्रह्मपुराणे ।\\
रक्षोघ्नीर्जुहुयाद्वह्नौ समिदर्थे महौषधीः ।\\
तिलार्थं तत्र विकिरेत् प्रशस्तांश्च तथा यवान् \textbar{}\textbar{}\\
अत्र }{तिलकार्ये}{ यवविधानवत्समिरकार्ये औषधीविधानं "घृता -\\
काः समिधो हुत्वा दक्षिणाग्नाः समन्त्रका" इत्यनेन प्रकृतिश्राद्धे वि -

{ वृद्धिश्राद्धनिरूपणम् \textbar{} ३०३\\
हितो यः समित्साधनको होमस्त {[}स्य{]} रक्षोऽघ्न्यौषधीनां साधनत्वं वि\\
धीयत इति । रक्षोघ्न्यश्च पलाशशङ्खिनीविष्णुक्रान्ताद्याः । अग्नौकरणान-\\
न्तरं पात्रपरिवेषणीयमन्नमुक्तं -\\
}{भविष्यत्पुराणे}{ ।\\
पृषदाज्येन संयुक्तं दध्यादोदनमादितः ।\\
पायसं च यथाभव्यं मोदकादिरसोत्तरम् ॥\\
मधुरं भोजनं दद्यान्न चाम्लं परिवेषयेत् \textbar{}\\
ब्रह्माण्डपुराणेऽपि ।\\
मङ्गल्यं भक्ष्यभोज्यादि दद्यादन्नं पृथग्विधम् ।\\
गुडमिश्रं खगश्रेष्ठ साज्यं चैवौदनं परम् ॥\\
रसालान्मोदकांश्चैव न चाम्लकटुकादिकम् ।\\
आदिशब्दादुद्रजकतिकादिरससंग्रहः "मधुरं भोजन दद्यात् "\\
इति प्रागूलिखितवचनात् ।\\
ब्रह्मपुराणे ।\\
अन्नं दद्याच्च दैवेन तीर्थेन च जपेत् स्वधाम् ।

{ तथा ।\\
द्राक्षामलकमूलानि यवांश्चाय निवेदयेत् ।\\
मूल= आद्रकादीति कल्पतरुः । सव्यजानुनिपातनं च {[}न{]} कर्त्त\\
व्यमित्याह\\
श्लोककात्यायनः ।\\
निपातो न हि सव्यस्य जानुनो विद्यते क्वचित् ।\\
अत्र च "अतो देवा" इति मन्त्रेणाङ्गुष्ठग्रहणम् । उक्तं चाश्वलायनगृह्ये\\
"अतो देवा अवन्तु न" इत्यङ्गुष्ठग्रहणमिति । भुञ्जानेषु विप्रेषु जपो-\\
ऽपि तत्रैवाभिहितः, पावमानी: शंवती रौद्री चाप्रतिरथं च भाव\\
यीत । शवतीः=शन इद्रानी इत्यादि । रौद्री = रुद्राध्यायादि । अप्रतिरथम्=\\
आशुः शिशान इति ।\\
ब्रह्मपुराणे ।\\
पठेत शाक्रसुक्तं तु स्वस्तिसुक्तं शुभं तथा ।\\
युक्तमश्रुमुखानां तु न पठेत्पितृसहिताम् \textbar{}\\
शाक्रसूक्तम् आशुः शिशान इत्यादि । स्वस्तिसूक्तं = स्वस्तिशब्दयुक्त.\\
स्वस्त्ययनं तार्क्ष्यमित्यादि । शुभं=नराशंसप्रधानं किञ्चिदृग्यजु.प्रभृ

{३०४ }{वीरमित्रोदयस्य}{ श्राद्धप्रकाशे-\\
तिकं, तदपि पठेदिति हेमाद्रौ । तथा पठेच्छकुनिसूक्तं त्विति ब्राह्मे ।\\
शकुनिसूक्त = कनिक्रदजुनुषमित्यादिकम् ।\\
}{श्लोककात्यायनः}{ ।\\
न चाश्नत्सु जपेदत्र कदाचित्पितृसंहिताम् ।\\
अन्य एव जपः कार्यः सोमसामादिकः शुभः ॥ इति ।\\
सोमसामादिकः=सोमसामत्वेनैव प्रसिद्धः । आह-\\
प्रचेताः ।\\
न जपेत्पतृकं जापं न मांसं तत्र दापयेत् ।\\
अत्र अपशब्दाद् भुञ्जानेषु द्विजेषु यः पितृलिङ्गकानां मन्त्राणां
जपः\\
स एव निषिध्यते न पदार्थानुष्ठान करणीभूतानां मन्त्राणां निवृत्तिः ।\\
यस्मिन्कर्माणि यस्य मन्त्रस्य करणता नियम्यते तन्मन्त्रनिवृत्तौ त-\\
त्कर्मसाद्गुण्ये प्रमाणाभावात् । तेन पितृलिङ्गकमन्त्रकरणकं कर्म त-\\
न्मन्त्रणैव कार्य अन्यथा तदक्कृकृत्तमेव स्यादिति ।
इदमेवाभिप्रेत्योक्तं\\
जातूकर्येन -\\
पितृलिङ्गेन मन्त्रेण यत्कर्म मुनिभिः स्मृतम् ।\\
तैनैव तद्विधातव्यममन्त्रमकृतं यतः ॥ इति ।\\
कात्यायन:-\\
सम्पन्नमिति तृप्ताः स्थ प्रश्नस्थाने विधीयते\\
सुसंपन्नमिति प्रोक्ते शेषमन्न निवेदयेत् ॥\\
तृप्तास्थेति प्रश्नस्थाने सम्पन्नमित्येवं रूपः प्रश्नः\\
आह }{श्लोककात्यायनः}{ ।\\
मधुमध्विति यस्तत्र त्रिर्जपोऽशितुमिच्छताम् ।\\
गायत्र्यनन्तरं सोऽत्र मधुमन्त्रविवर्जितः ॥ इति ।\\
अस्यार्थः अशितुमिच्छतां विप्राणां सन्निधो गायत्र्यनन्तर मधुप्रती\\
पाठानन्तरं यो मधुमधुमधुइति त्रिवारं शब्दप्रयोगरूपो जपो विहितः\\
स मधुमन्त्रविवर्जित. = मधुव्याता इति पाठविवर्जितः । गायत्रीपाठोत्तरं\\
मधुमतीस्थाने उपास्मै गायता नर इति पञ्चर्चं अक्षन्नमीमदन्तेति\\
च षष्ठीमृचं श्रावयित्वा ततो मधुमधुमध्विति त्रिर्जपेत् । तथाचोक्त\\
माश्वलायनेन ।\\
मधुव्वाता ऋतायत इति तृचस्थाने उपास्मै गायता नर इति पञ्च,\\
मधुमतीः श्रावयेत् । अक्षन्नमीमदन्तेति च षष्ठीमिति ।

{ वृद्धिश्राद्धनिरूपणम् । ३०५\\
ब्रह्मपुराणे ।\\
}{क्वचित्}{ संपन्नमेतन्मे तान्पृच्छेच्च प्रहर्षितान् ।\\
सुसम्पन्नं च ते बूयुः सर्वं सिद्ध ततः क्षिपेत्
\textbar{}\textbar{}\\
सर्वजातीयसिद्धमन्नमेकपात्रे समुद्धृत्य विकिरं कुर्यादित्यर्थः ।\\
दैवे तु तृप्तिप्रश्ने रोचत इति विशेषो लिखितो -\\
वृद्धवसिष्ठेन ।\\
तृप्तिप्रश्ने तु सम्पन्नं दैवे रोचत इत्यपि ।\\
}{भविष्यत्पुराणे}{ ।\\
एवं भुक्तेषु विप्रेषु दद्यात्पिण्डान्समाहितः ।\\
दध्यक्षतैर्विमिश्रांस्तु बदरैश्च खगाधिप ।\\
अक्षताः= यवाः । "अक्षताश्चं यवां" इतिकोशात् ।\\
विष्णुधर्मोत्तरेऽपि ।-\\
दधिकर्कन्धु संमिश्रान्तथा पिण्डांश्च निर्वपेत् ।\\
याज्ञवल्क्यः ।\\
एवम्प्रदक्षिणावृत्को वृद्धौ नान्दीमुखान्पितॄन् ।\\
यजेत दधिकर्कन्धुमिश्रान् पिण्डान्यवैः क्रियाः ॥\\
{[} अ १ श्राद्धम० श्लो० २५० {]}\\
प्रदक्षिणा आवृदनुष्ठानपद्धतिर्यस्यासौ प्रदक्षिणावृत्कः प्रदक्षिणमु-\\
पचार इति यावत् ।\\
ब्रह्मपुराणेऽपरो विशेषः ।\\
शाल्यन्नं दधिमध्वक्तं बदराणि यवांस्तथा ।\\
मिश्रोकृत्यानुचत्वारि पिण्डान् श्रीफलसन्निभान् \textbar{}\textbar{}\\
दद्यान्नान्द्रीमुखेभ्यश्च पितृभ्यो विधिपूर्वकम् ।\\
कात्यायनः ।

{ सर्वस्मादन्नमुद्धृत्य व्यञ्जनैरुपसिच्य च ।\\
उपसेचनम् = उपरि प्रक्षेपः ।\\
संयोज्य वधकर्कन्धुदाधिभिः प्राङ्मुखस्ततः ।\\
अवनेजनवत्पिण्डान्कृत्वा बिल्वप्रमाणकान् ॥\\
तत्पात्रक्षालनेनाथ. पुनरप्यवनेजयेत् ।\\
पिण्डपात्रस्याधोमुखस्थापनं न कर्त्तव्यमित्याह-\\
वसिष्ठः ।\\
वी० मि १९

{३०६ }{वीरमित्रोदयस्य}{ श्राद्धप्रकाशे-\\
प्राङ्मुखो देवतीर्थेन प्राक्कूलेषु कुशेषु च ।\\
दत्वा पिण्डान्न कुर्वीत पिण्डपात्रमधोमुखम् ॥ इति ।\\
प्राक्कूलाः=प्रागग्राः । पिण्डदानं चोच्छिष्टसन्निधौ न कुर्यात् ।\\
प्रदद्यात्प्राङ्मुखः पिण्डान्वृद्धौ नाम्ना स बाह्यतः । -\\
बाह्यत इति भोजनशालाया बहिर्नतूच्छिष्टसन्निधावित्यर्थः । पि\\
ण्डार्थं गोमयादिना चतुरस्रं मण्डलं कर्त्तव्यमिति }{भविष्यत्पुराणे}{ ।-\\
निर्वपेन्मण्डले वीरे चतुरस्रं विचक्षणः ।\\
पवित्रपाणिराचान्त उपविष्टः समाहितः ॥\\
नामोच्चारणं चात्र प्रथमपिण्ड एव न द्वितीये । यहुकम -\\
चतुर्विंशतिमते ।\\
द्वौ द्वौ चाभ्युदये पिण्डौ एकैकस्मै विनिक्षिपेत् ।\\
एकं नाम्नाऽपरं तुष्णीं दद्यात्पिण्डान्पृथक् पृथक् ॥\\
एकैकस्मै द्वो द्वौ पिण्डौ तत्राद्यं नामगोत्रसहितं दद्यादपरं तूष्णीं\\
दद्यादित्यर्थः । वृद्धिश्राद्धे च पिण्डदानवैकल्पिकमुक्तम्-\/-\\
विष्णुपुराणे ।\\
दध्यक्षतैः सबदरैः प्राङ्मुखोदङ्मुखोऽपि वा ।.\\
देवतीर्थेन वै दद्यात्पिण्डान्कामेन वै नृप ॥\\
कामेन=इच्छाया । इच्छाभावे न दद्यात् । }{भविष्यत्पुराणे}{ तु पिण्डदान -\\
करणाकरणयोर्व्यवस्था कुलधर्मापेक्षिकोक्ता \textbar{}\\
पिण्डनिर्वपणं कुर्यान्न {[} वा कुर्यान् {]} नराधिप ।-\\
वृद्धिश्राद्धे महाबाहो कुलधर्मं निवेक्ष्य वै ॥\\
तेन येषां कुले वृद्धपरम्परया वृद्धिश्राद्धे पिण्डदानानुष्ठानं तैर\\
नुष्ठेयमेव येषां तु कुले नानुष्ठानं तैर्नानुष्ठेयमेवेति । इयन्तु
व्यवस्था\\
निरग्निकानामेव । साग्निकैस्तु सर्वदा सपिण्डकमेव कर्त्तव्यमित्युकं -\\
ब्रह्मपुराणे ।\\
योऽग्नौ तु विद्यमानेऽपि वृद्धौ पिण्डान निर्वपेत् ।\\
पतन्ति पितरस्तस्य नरके स च पच्यते ॥ इति ।\\
आह - कात्यायनः ।\\
अथाग्रेभूमिमासिञ्चेत् सुसुप्रोक्षितमस्त्विति ।\\
शिवा आपः सन्त्विति च युग्मानेवोदकेन च\\
सौमनस्यमस्त्विति च पुष्पदानमनन्तरम् ।\\
अक्षतं चारिष्टं चास्त्वित्यक्षतान्प्रतिपादयेत् ॥.

{ वृद्धिश्राद्धनिरूपणम् । २०७\\
अक्षय्यं च ततः कुर्य्यादैव पूर्वं विधानतः ।\\
षष्ठ्यैव नित्यं तत्कुर्यान्न चतुर्थ्या कदाचन ॥\\
प्रार्थनासु प्रतिप्रोक्ते सर्वास्वेव द्विजोत्तमैः ।\\
पवित्रान्तर्हितान्पिण्डान् सिञ्चेदुत्तानपात्रकृत् ॥\\
प्रार्थनासु प्रोक्षितमित्यादिप्रतिप्रोक्ते अस्तु
सुप्रोक्षितमित्युत्तरिते\\
सत्ति अर्धपात्रसम्बन्धि पवित्राच्छादितान्पिण्डान् "ऊर्जे वहन्ती'' रि-\\
त्यनेन सिञ्चोदित्यर्थः ।\\
युग्मानेव स्वस्तिवाच्याङ्गुष्ठग्रहणं सदा ।\\
}{कृत्वा }{धुर्यस्य विप्रस्य प्रणस्यानुव्रजेत्ततः ॥\\
धुर्यः पङ्किमूर्द्धन्यः \textbar{}\\
ब्रह्मपुराणे ।\\
प्राङ्मुखांस्त्वथ वै दर्भान्दद्यात्क्षीराषनेजनमिति ॥\\
इदमवनेजनं विप्रहस्ते जलस्थाने क्षीरदानमात्रमिति, क्षीरम्फ\\
लातिशयार्थमिति शूलपाणिः ।\\
शातातपः ।\\
नान्दीमुखास्तु पितरस्तृप्यन्तामिति वाचयेत् । अत्रान्योऽपि\\
विशेषः कात्यायनेनोक्तः ।\\
प्रागग्रेषु च दर्भेषु आद्यमामन्त्र्य पूर्ववत् ।\\
अपः क्षिपन् मूलदेशेऽवनेनिश्वेति निस्तिलाः ॥\\
निस्तिला इत्यस्माद्विशेषणात्सयवाः कार्या इत्यर्थः । }{तिलकार्ये}{ यक..\\
विधानात् ।\\
द्वितीयं च तृतीयं च मध्यदेशाप्रदेशयोः ।\\
पूर्वाश्रेषु दर्भेषु दर्भमूले पितॄन् दर्भ मध्यभागे पितामहान् दर्भा\\
प्रदेशे प्रपितामहानित्यर्थः ।\\
मातामहप्रभृतस्तु एतेषामेव वामतः ।\\
उत्तरोत्तरदानेन पिण्डानामुत्तरोत्तरः ।\\
मवेदद्धश्च करणादधरः श्राद्धकर्मसु ।\\
तस्माच्छ्राद्धेषु सर्वेषु वृद्धिमस्तिरेषु च ॥\\
मूलमध्याप्रदेशेषु ईषत्सक्तांश्च निर्वपेत् ।\\
गन्धादी निक्षिपेत्तूष्णीं तत आचमयेद् द्विजान् ॥\\
पिण्डानामुत्तरोत्तरदानेन=प्रागपवर्गप्रदानेन, यजमान उत्तरोत्तरो भवेद\\
धिकेभ्योऽधिको भवेत् । पिण्डानामधःकरणात् = प्रत्यगपवर्गदानात्\\
.

{३०८ }{वीरमित्रोदयस्य}{ श्राद्धप्रकाशे-\\
अधरो=नचिः पापीयान्भवेदित्यर्थः । ईषत्सक्तान् = ईषत्परस्परं लग्नान् ।\\
तूष्णीमिति=पिण्डे गन्धादिदाने मन्त्रनिवृत्तिः । अन्यत्र
वृद्धिश्राद्धाति-\\
रिक्ते आाद्धे यवराहित्यात्तिलसहितो विधि: दक्षिणाप्रवणोदेशो\\
दक्षिणाभिमुखो यजमानः दक्षिणाग्राः कुशाः । वृद्धिश्राद्धे तु यवस-\\
हितो विधिः । प्राक्प्रवणादिर्द्देशः । प्राङ्मुख उदङ्मुखो वा यजमा\\
नः । प्रागग्राः कुशाः । "मातामहप्रभृतींस्तु एतेषामेव वामत" इत्यत्र\\
यजमानस्य प्राङ्मुखत्वे पितृपिण्डानां वामती दक्षिणस्यां दिशि\\
मातामहादिपिण्डाः । तेन प्रादक्षिणोपचारोऽनुगृहीतो भवति ।\\
यदोदङ्मुखो यजमानो ददाति तदा पितृपिण्डानां प्राग्दिगेव माता.\\
महादीनां पिण्डदानम् । तत्रापि पितृपिण्डानां प्राग्दिगेव धामो\\
भागः । एवं }{सत्ति}{ तत्रापि प्रदक्षिणमुपचारोऽनुगृहीतो भवति । अत\\
एव पिण्डदाने पितॄणां ध्यानप्रस्तावे ।\\
आत्माभिमुखमासांना ज्ञानमुद्रा निरायुधाः ।\\
वसवः पितरो ज्ञेया रुद्रास्तत्र पितामहाः ॥\\
पितुः पितामहाः प्रोक्ता आदित्या बर्हिषि स्थिताः ।\\
इति वाक्यात्प्राङ्मुखस्थस्य कर्त्तुराभिमुख्येनोपविष्टानां पितॄणां\\
दक्षिणादिगेव वामभागो भवति । एवमुदङ्मुखस्य कर्तुराभिमुख्ये\\
नोपविष्टानां पूर्वदिगेव वामभागो भवति । तेन पितृपिण्डेभ्यो दक्षि-\\
णदिश्येव मातामहपिण्डदानमिति ।\\
तथा स एव-\/-}

 अक्षय्योदकदानं च अर्घदानवदिष्यते ।

{ षष्ठयैव नित्यं तत्कुर्यान्न चतुर्थ्या कदाचन \textbar{}\textbar{}\\
प्रपितामहसंज्ञाश्च नान्दी मुख्यश्च मातरः ।\\
मातामह्यः पितामह्यः प्रमातामह्य एव च ॥\\
मातामहेभ्यश्च तथा नान्दीवत्क्रेभ्य एव हि ।\\
प्रमातामहसंज्ञेभ्यो भवद्भिश्चे स्वधोच्यताम् ॥\\
अस्तु स्वधेति ते तं च जल्पन्ति प्रहसन्ति च १\\
विश्वेदेवाश्च प्रीयन्तामिति दाता ब्रवीति तान् प्र\\
प्रीता भवन्तु ते तं च वदन्ति मधुराक्षरम्। .\\
त्यमूषु वाजिनमिति पठंस्तांश्च विसर्जयेत् ॥\\
अर्घदानवदिति ज्येष्ठोत्तरकरत्वातिदेशार्शो न तु तन्त्रतानिवृत्यर्थः ।

{ }{वृद्धिश्राद्धनिरूपणम्}{ । ३०९\\
अर्धेऽक्षय्योदके चैव पिण्डदानेऽवनेजने ।\\
तन्त्रस्य विनिवृत्तिः स्यात्स्वधावाचन }{एव}{ च ॥\\
इत्यनेनैव प्राप्तत्वात् ।\\
प्रचेताः ।\\
प्राङ्मुखो देवतीर्थेन वृद्धौ परिचरेत्पितॄन् ।\\
सव्येनैवोपनीतेन क्षिप्रं विप्रविसर्जनम् ॥\\
क्षिप्रमित्यतः पार्वणवत् न सूर्यास्तं प्रतीक्षेतेति । स्वधेस्यत्रापि\\
स्वाहाशब्दः प्रयोज्यः, न स्वधेति । "न स्वधां प्रयुञ्जीत" इति कात्या-\\
यनवचनात् । शूलपाणिस्तु स्वधेति शाखाभेदव्यवस्थितमित्याह ।\\
}{य{[}त{]}त्तु}{ बहुषु कात्यायनाश्वलायनगोभिलाद्युक्तसूत्रेषु बहुषु\\
स्मार्तवचनेषु च स्वधाशब्दस्य निषिद्धत्वाद्विचार्य श्रद्धातव्यमिति ।\\
ब्रह्मपुराणे ।\\
द्राक्षामलकमूलानि यवांश्चाथ निवेदयेत् ।\\
तान्येव दक्षिणार्थं तु दद्याद्विप्रेषु सर्वदा ॥\\
इत्यत्र यवानां तिलार्थेऽपि दानं भक्षणार्थे दक्षिणार्थेऽपि दानं\\
प्रसक्तं तद्यथायोगं भृष्टरूपेण स्वरूपेणैव वा विधेयम् ।
}{दधिबदराक्ष}{\\
तमिश्राः पिण्डा इत्यत्रापि अक्षता = यवाः । ``अक्षताश्च यवाः प्रोक्ता\\
भृष्टा धाना भवन्ति ते" इति वाक्यात् तान्येव दक्षिणार्थमित्यत्र द्रा-\\
क्षामलकादीनां दक्षिणार्थे विधानादनयौपदेशिक्या दक्षिणयातिदे\\
शिक्या दक्षिणाया निवृत्तिः । अन्योऽपि विशेषः ।\\
सङ्ग्रहे-\\
शुभाय प्रथमान्तेन वृद्धौ सङ्कल्पमाचरेत् ।\\
न षष्ठ्या यदि वा कुर्यान्महादोषोऽभिजायते ।\\
तथा ।

{ अनस्मच्छब्दवृद्धानामरूपाणामगोत्रिणाम् ।\\
अनाम्नां चातिलाद्यैश्च नान्दीश्राद्धं न लव्यवत् ।\\
बहचकारिका च ।\\
सम्बन्धनामरूपाणि वर्जयेदत्र कर्मणि । इति ।\\
अत्र च यद्यपि नामगोत्राणां वर्ज़नमुक्तं तत्तु "गोत्रनामभि\\
रामन्त्र्य पितॄनर्घं प्रदापये" दितिकात्यायनवचनेन विरुद्धं, तथापि शा-\\
खाभेदव्यवस्थितं सदविरुद्धमेवेति । प्रयोगपारिजाते तु शुभाय प्रथ-\\
मान्तेन वृद्धौ सङ्कल्पमाचरेत्" इत्युपक्रम्य
अनस्मच्छब्दानामित्युक्तेर्गोत्र-

{११० }{वीरमित्रोदयस्य}{ श्राद्धप्रकाशे-\\
नामादिनिषेधः सङ्कल्पश्राद्धे सपिण्डके तु तन्निषेधो नास्तीत्युक्तम् ।\\
यच्च वृद्धवसिष्ठेनोक्तं ।\\
नान्दीमुखे विवाहे च प्रपितामहपूर्वकम् ।\\
नाम सङ्कीर्त्तयेद्विद्वानन्यत्र पितृपूर्वकम् ॥ इति\\
स्मृत्यर्थसारे च-\\
वृद्धमुख्यास्तु पितरो वृद्धिश्राद्धेषु भुञ्जत इति ।\\
तच्छाखान्तरविषयम् । कात्यायने तु नान्दीमुखाः पितरः पिता.\\
महाः प्रपितामहा इत्युक्तेः पितृभ्यः पितामहेभ्यः प्रपितामहेभ्य इति\\
बहृचपरिशिष्टोक्तेराश्वलायनादीनामानुलोम्येनैव विधानादनुलोम\\
क्रम }{एव}{ ।\\
ननु आश्वलायनव्यतिरिक्तानां सर्वेषामनुलोमक्रम }{एव}{ आश्व-\\
लायनानामेव केवलं प्रातिलोम्यक्रमः "नान्दीमुखे विवाहे च" इति\\
वाक्यात् । इदं हि वृद्धवसिष्ठवाक्यमाश्वलायनविषयकमेव - "वा-\\
सिष्ठं बहृचैरेव " इति होलाधिकरणे वार्त्तिककारैरुकत्वात्, "माता\\
पितामही चैव" इत्याश्वलायनवचनं तु न क्रमपरं किन्तु पदार्थमात्र-\\
परम्। न च वासिष्ठे प्रपितामहपूर्वकालोक्त्या तत् त्रिदेवत्यमात्रे
प्राति-\\
लोम्यविधानमस्तु न मात्रादित्रिके न वा मातामहादित्रिक इति वाच्य-\\
म् । तस्योपलक्षणत्वेन वर्गत्रयेऽपि प्रातिलोम्यविधानस्योचितत्वात् ।\\
तस्मात्सिद्धं नान्दीमुखे विवाहे च आश्वलायनानां प्रतिलोमक्रम\\
इति चेत् ।\\
माता पितामही चैव सम्पूज्या प्रपितामही ।\\
पित्रादयस्त्रयश्चैव मातुः पित्रादयस्त्रयः ।\\
एते नवार्घनीयाः स्युः पितरोऽभ्युदये द्विजैः ॥\\
इत्याश्वलायनाचार्यवचने, }{तथा}{ नान्दीमुखाः पितर इदं वोर्घ्यं\\
पितामहाः प्रपितामहा इति यथालिङ्गमर्घ्यदानं पितरः प्रीयन्तामि\\
त्यपां प्रतिग्रहणं विसर्जनं च, एवमुत्तरयोः पितामहप्रपितामहयोरि-\\
त्याश्वलायनगृह्मपरिशिष्टे,\\
शौनकीयेऽपि -\/-\\
तत्रेदं तेऽर्घ्यमित्येष पितृनामपदादिकः ।\\
पितमहार्थविप्रेभ्यो दत्वार्घं च यथा पुरा ।\\
प्रपितामहशब्दादिमिदं तेऽर्घ्यमितीरयेत् ॥\\
-इत्याश्वलायनशाखाप्रवर्त्तकाश्वलायनादिवाक्येषु प्रतीतो यो-

{ वृद्धिश्राद्धनिरूपणम् । ३११\\
ऽनुलोमक्रमस्तस्यैवाश्वलायनैरन्तरङ्गत्वेनाङ्गी कार्यत्वात् । न वाशिः\\
ष्टोकस्य तस्य बहिरङ्गत्वात् । अतश्चाथर्वणानां शाखाविशेषे प्रा.\\
तिलोम्यक्रमस्य प्रत्यक्षपठितत्वात् । तद्विषयो वासिष्ठोक्तः क्रम इति\\
ध्येयम् ।\\
यच्चोतं वासिष्ठं बहृचैरेवेति होलाकाधिकरणे वार्त्तिककारे,\\
(णोक्तमिति, तदपि, तत्रत्य पूर्वपक्षमूलकं न सिद्धान्तमूलकं । तथा च\\
पूर्वपक्षे वार्त्तिकं तद्यथा गौतमीये गौभिलीये, छान्दोग्यैरेव परिगृ\\
हीते, वासिष्ठं वहृचैः शङ्खलिखितोक्तं वाजसनोयिभिः, आपस्तम्ब-\\
बौधायिनीये तैत्तिरायैरेव प्रतिपन्ने इत्येवं तत्र तत्र गृह्यग्रन्थव्यव\\
स्थाभ्युपगमादिदर्शनाद्विचारयितव्यं, किं तानि तेषामेव प्रमाणानि,\\
उत सर्वाणि सर्वेषामिति ।\\
किं तावत्प्रतिपत्तव्यं व्यवस्थैवेति पाठतः ।\\
नह्यन्यत्र स्थिताल्लिङ्गाल्लिङ्गयन्यत्रानुमीयते ॥\\
''अनुमानाद्व्यवस्था'' इति पूर्वपक्षसुत्रे अनुमानं = लिङ्गम् ''आचा-\\
रात्मकाल्लिङ्गाल्लिङ्गिनो विधिप्रतिषेधौ अनुमीयमानौ तद्विषयादेवा-\\
नुमातव्यावित्यादिना पूर्वपक्षमुक्त्वा ''अपि वा सर्वधर्मः स्यात्''
इत्य-\\
नेन शक्तमात्राधिकारित्वाद्राजा राजसुयेन, वैश्यो वैश्यस्तोमेनेत्या-\\
दिवत्कर्त्तृव्यावर्त्तकविशेषणाभावात्, तत्तदाचाराणामनुवृत्तव्य.\\
त्या कृतिप्राच्यत्वदाक्षिणात्यत्वादिजातिवचनत्वाभावाच गृह्यधर्मस्\\
श्रनिषद्धधर्माणामपि सर्वधर्मत्वमिति हि सिद्धान्ते वार्त्तिकम् । तेन\\
वासिष्ठं बहुचैरेवेति पठितवार्त्तिको नेदं वक्तुं क्षमते । यथेओकं\\
''माता पितामही चैव'' इत्यत्र न क्रमविधिः, किन्तु पदार्थमात्रं त-.\\
दपि न साधु, पाठस्याषि क्रमनियामकत्वमुक्तं '' क्रमेण वा नियम्येत''\\
इत्यत्राधिकरणे । समिधो यजति, तनूनपातं यजतीति क्रमेण पठि\\
तानां यागानामनियमेनानुष्ठानमुत पाठक्रमेणेति श्रुत्यर्थयोरभावा.\\
त्पाठक्रमस्याविधायकत्वाद्विधीनां च पदार्थमात्रपर्यवसानादनियम\\
इति पूर्वपक्षयित्वा राद्धान्तितं ।\\
यथा पाठमनुष्ठानं तथैव प्रतिपत्तितः ।\\
-\/-\/-\/-स्मृतिप्रयोगवेलायां वाक्यैरेव च कर्मणाम् ॥ इति ।\\
यथापाठं क्रियमाणं स्मरणं विहितक्रमं भवतीति पाठक्रमस्य\\
बलवत्वम् । न च प्रतिलोमक्रम औपदेशिक इति वाच्यम् । ''एते न-

{३१२ वीरमित्रोदयस्य श्राद्धप्रकाशे-\\
वार्चनीयाः स्युः पितरोऽभ्युदये द्विजैः'' इत्यभ्युदयपुरस्कारेणानुलो-\\
मक्रमस्याप्यौपदेशिकत्वात् । न हीतोऽधिकमप्यस्ति शृङ्गान्तरमुपदे-\\
शस्य, एवं सति यदा पित्रादित्रिके प्रातिलोम्यक्रमो निरस्तः तदा\\
कैव कथाअपरास्मिन्वर्गद्वये । अत एव-\/- प्रयोगपारिजातकारेण अनु-\\
लोमकमाश्रयणं कृतम् । यच्च ''नान्दीमुखे विवाहे च'' इति विवा-\\
हेऽपि प्रातिलोम्यविधानं तत्तु वत्सगोत्रोद्भवाममुष्य
{[}प्रपौत्रीममुष्य{]}\\
पौत्रीममुष्यपुत्री च वसिष्ठगोत्राद्भवायामुष्य
प्रपौत्रायामुष्यपौत्रायामु-\\
ष्यपुत्रायेत्यादि परिशिष्टोत्केर्भवत्याश्वलायनपरमिति सिद्धो विवाहे\\
प्रतिलोमक्रम इति । न च\\
पुत्रायास्य च पौत्राय नप्त्रेऽस्यामुकगोत्रिणे ।\\
इति कारिकोक्तानुलोमक्रमण स बाधित इति वाच्यम् । परिशि\\
ष्टस्याषत्वेनाधुनिककृतकारिकया बाधायोगात् ।\\
कात्यायनः ।\\
असकृद्यानि कर्माणि क्रियेरन् कर्मकारिभिः ।\\
प्रतिप्रयोगं (१) नैव स्युर्मातरः श्राद्धमेव च ।\\
कर्मावृत्तावपि कुत्र श्राद्धं कार्य कुत्र नेत्युक्तं तेनैव -\\
आधाने होमयोश्चैव वैश्वदेवे तथैव च ।\\
बलिकर्माणि दर्शे च पौर्णमासे तथैव च ॥\\
( २ ) नवयज्ञे च यज्ञज्ञा वदन्त्येवं मनीषिणः ।\\
एकमेव भवेच्छ्राद्धमेतेषु न पृथक् पृथक् ॥\\
एतेषु प्रतिप्रयोगं नावर्त्तते, एतद्भिन्ने तु सोमयागादौ प्रतिप्रयो\\
गमावर्त्तत एव-\/- । क्वचिदादावपि श्राद्धनिषेधः ।\\
नाष्टकासु भवेच्छ्राद्धं न श्राद्धे श्राद्धमिष्यते ।\\
न सोष्यन्तीजातकर्मप्रोषितागतकर्मसु ।\\
विवाहादिः कर्मगणो य उक्तो.\\
गर्भाधानं शुश्रूमो यस्य चान्ते ।\\
विवाहादावेकमेवात्र कुर्यात्\\
छ्राद्धं नादौ कर्मणः कर्मणः स्यात् ॥\\
प्रदोषे श्राद्धमेकं स्यात् गोनिष्क्रमप्रवेशयोः ।

% \begin{center}\rule{0.5\linewidth}{0.5pt}\end{center}

{ ( १ ) नैताः स्युरिति मयूखोदूषतः पाठः ।\\
( २ ) नवयज्ञः आग्रहायणेष्टिरित्यर्थः ।

{ वृद्धिश्राद्धनिरूपणम् । ३१३\\
न श्राद्धं युज्यते कर्त्तुं प्रथमे पुष्टिकर्माणि ॥\\
हलाभियोगादिषु तु षट्त्सु कर्म पृथक् पृथक् ।\\
प्रतिप्रयोगमन्येषामादामेकं तु कारयेत् ।\\
बृहत्पत्रक्षुद्रपशुस्वस्त्यर्थं परिविष्यतोः ।\\
सुर्येन्द्वोः कर्मणी ये तु तयोः श्राद्धं न विद्यते ॥\\
म दशाग्रन्थिके नैव विषवद्दष्टकर्माणि ।\\
कृमिदष्टचिकित्सायां नैव शेषेषु विद्यते ।\\
गणशः क्रियमाणेषु मातृभ्यः पूजन सकृत् ॥\\
सकृदेव भवेच्छ्राद्धमादौ न पृथगादिषु ।\\
यत्र यत्र भवेच्छ्राद्धं तत्र तत्र तु मातरः ॥\\
असकृदिति = प्रतिदिनं प्रतिमासं प्रतिसंवत्सरं च यानि क्रियन्ते वै-\\
श्वदेवबलिकर्मदर्शपूर्णमासश्रावण्यादीनि तेषु प्रथमप्रयोग एव श्राद्धं\\
मातृपूजा च, प्रत्यब्दं येन पशुयागादि क्रियते तेनापि प्रथमप्रयोग\\
एव-\/- श्राद्धं कार्यं । आधाने च ''वसन्ते ब्राह्मणोऽग्नीनादधीत''
इत्या-\\
दिनोक्तं । होमयोः = सायंप्रातर्होमयोः । सोष्यन्त्या आसन्नप्रसवायाः\\
बध्वा: सुखप्रसवार्थे सोष्यन्तीमभ्युक्ष्येत्यादिना होमादिकर्मोक्तं गो-\\
मिलेन तत्र, तथा ब्रीहियवपिष्टेन कुमारजिह्वामार्जनाख्ये जातकर्मणि,\\
तथा प्रोषतागतकर्माणि= प्रवासादेत्य ``प्रोष्यैत्य गृहानुपतिष्ठते''
''पुत्रं दृष्ट्वा\\
जपती ''त्यादिके कुमारमूर्द्धाभिघ्राणादिके च गोभिलोक्ते, न भवती-\\
त्यर्थः । विवाहादिरिति = (१) समनीयचरुहोमगृहप्रवेशयानारोहणचतुष्प\\
थामन्त्रणाक्षभङ्गसमाधानार्थ होमचतुर्थीहोमानामादिशब्देन ग्रहणं,\\
एषु विवाहादिगर्भाधानान्तकर्मसु सकृदेव श्राद्धं कार्यम् । प्रदोष इति ।\\
गोनिःसरणप्रवेशन कर्मणोगोभिलो क्तयोरुभय तन्त्रेणैकं श्राद्धम् । तथा\\
गवां पुष्टिकर्मत्रये गोभिलोक्ते प्रथमे पुष्टिकर्मणि, श्राद्धं न
कर्त्तव्यम् ।\\
हलाभियोगादिषु = हलस्याभिमुखेन योगः । पक्वेक्षु धान्येषु सीतायज्ञः,\\
कृष्टक्षेत्रमध्ये खलयज्ञः, प्रवपनं, प्रस्रवणं, धान्यच्छेदनं, पर्याणं
धान्या-\\
नां गृहगमनं, पषु हलाभियोगादिषट्त्सु कर्मसु एकैककर्मणि पृथक्\\
पृथक् प्रतिप्रयोगं श्राद्धं कार्यम् । अन्येषां
श्रावणीकर्मादीनामादावेकं\\
श्राद्धं कार्यम् । बृहत्पत्रेति = वृहत्पत्रं = हस्त्यश्वादि॥
क्षुद्रपशवः=अजाव्यादयः,\\
तेषां स्वस्त्यर्थम् । परिविष्यतोः=परिविष्यमाणयोः सूर्येन्द्वोर्ये कर्मणी
गो-

% \begin{center}\rule{0.5\linewidth}{0.5pt}\end{center}

{ ( १ ) समशनीयेति श्राद्धतत्वे पाठः ।\\
४० वी० मि०\\


{३१४ वीरमित्रोदयस्य श्राद्धप्रकाशे-\\
भिलप्रोक्ते तयोः श्राद्धं न कार्यम् । यथा गोभिलः ।\\
वृक्ष द्रवेति पञ्चर्व इति प्रकृते द्वितीयया आदित्ये परिविष्यमाणे\\
अक्षततण्डुलान् जुहुयात्, बृहत्पत्रस्वस्त्ययनकाम: तृतीयया चन्द्र-\\
मसमिति तण्डुलान्क्षुद्रपशुस्वस्त्ययनकाम इति ।\\
न दशाग्रन्थिक इति=प्रतिभये ध्वनि वस्त्रदशायां ग्रन्थीन्वयादुपेत्य\\
वसनवतः स्वाहाकारान्ताभिर्माभैषीर्न मरिष्यसीति, विषवता दष्टमद्भि-\\
रभ्युक्षयेत् हतस्तु अत्रिणा कृमिरिति कृमिमन्तं देशमन्द्भिरभ्युक्षये-\\
दिति गोभिलोक्तकर्मत्रये । नैव शेषष्विति। अर्हणीयॠत्विगादीनां पा-\\
द्यार्घविष्टरमधुपर्कदानादिषु श्राद्ध नास्ति । गणश इति । यथा यज-\\
नीयेऽहनि नवयज्ञवास्तुमनोयज्ञाश्च यज्ञेषु समुदायेन क्रियमाणेषु\\
मातृपूजा श्राद्ध च सकृदेव गणादौ न कर्मानुसारेणेति शूलपाणिः ।\\
प्रयोगपारिजाते तु गणश इत्यस्य व्याख्या एवं कृता, देशान्तरगतस्य\\
चिरकालादश्रूयमाणसद्भावस्य मृत इति बुध्या पुत्रादिना कृतौर्द्धदेहि-\\
कस्य पुनरागतस्य यानि पुनर्जातकर्मादीनि संस्कारकर्माणि क्रियन्ते ।\\
तथोपनयनात्प्राक् स्वस्वकाले कथञ्चिदकृतचौलपर्यन्तसंस्कारस्य\\
यानि जातकर्मादीनि उपनयनकाले सम्भूय क्रियन्ते तेषु गणशः\\
सम्भूय क्रियमाणेषु जातिकर्मादिसंस्कारेषु मात्रादिपूजायां नान्दी-\\
श्राद्धस्य च सकृत्तन्त्रेण प्रथमं क्रियमाणस्य संस्कारकर्मण आदौ\\
अनुष्ठानं न पृथगादिषु नावृत्या संस्कारकर्मणामादिष्विति । अङ्ग\\
वङ्गकालिङ्गेषु तीर्थयात्रां विना गमने कर्मनाशाजलस्पर्शादौ च प्रा-\\
यश्चित्ततया पुनः संस्काराणां युगपदनुष्ठानम् । आदौ मातृपूजा आउं\\
च सकृदेवेत्यपरे । यत्र यत्रेति । श्राद्धनिषेधोऽष्टकादौ तत्र मातृपूजा\\
निषेधोऽपीति वचनार्थः । अत्र साग्निरनग्निर्वा वैश्वदेवमादौ कुर्यात् ।\\
उक्तं च-\\
स्मृतिसङ्ग्रहे ।\\
वृद्धावादौ क्षये चान्ते दर्शे मध्ये महालये ।\\
आचान्तेषु च कर्त्तव्यं वैश्वदेवं चतुर्विधम् ॥\\
इति वचनात् ये वा भद्रं दूषयन्ति स्वधाभिरिति कथञ्चिल्लि-\\
ङ्गदर्शनादपि वृद्धिश्राद्धोत्तरं वैश्वदेवनिषेधात् । शिष्टा अपि बहवो\\
वैश्वदेवं चतुर्विधमिति वचनात्कृत्वाभ्युदयिकमनुतिष्ठन्तीति । त-\\
दाचारदर्शनाच्छ्राद्धात्पूर्वमेव कर्त्तव्यमिति ।

{ सामान्यकृष्णपक्षश्राद्धनिरूपणम् । ३१५\\
हेमाद्रौ तु ।\\
शेषमन्नमनुज्ञाप्य वैश्वदेवक्रियां ततः ।\\
श्राद्धाह्नि श्राद्धशेषेण वैश्वदेवं समाचरेत् ॥\\
इति वृद्धिश्राद्धप्रयोगे लिखितचतुर्विंशतिवचनाच्छ्राद्धान्ते कर्त-\\
व्यता यद्यपि प्रतीयते तथाप्यत्र वाक्ये श्राद्धाह्नीति सामान्येन श्रा-\\
द्धग्रहणात्प्रकृतिभूत एवं श्राद्धे वैश्वदेवस्यान्ते कर्त्तव्यता न
वृद्ध्यादौ ।\\
ननु वृद्धिश्राद्धेऽपि विकृतिरूपत्वेन प्राकृतेतिकर्त्तव्यतातिदेशा-\\
दपि वैश्वदेवस्यान्ते कर्त्तव्यता प्राप्तिरपि पुनः पृष्ठलग्नैवेति चेत्
।\\
सत्यम् । यद्यत्र वृद्धावादाविति वचनं न स्यात् । तेनोपदेशप्राबल्या-\\
दप्यादावेव स इति । अत्रार्थे शिष्टाचारोऽप्यनुसन्धेय इति ।\\
वृद्धिश्राद्धे च तदङ्गतिलतर्पणं न कर्त्तव्यमेव ।\\
वृद्धिश्राद्धे सपिण्डे च प्रेतश्राद्ध च मासिके ॥\\
संवत्सरविमोके च न कुर्यात् तिलतर्पणम् ।\\
इति बृहन्नारदीये निषेधात् । नान्दीश्राद्धे ब्राह्मणाभाव आह-\\
वृद्धवशिष्ठः ।\\
मातृश्राद्धे तु विप्राणामलाभे पूजयेदपि ।\\
पतिपुत्रान्विता भव्या योषितोऽष्टौ मुदान्विताः ॥ इति ।\\
पञ्चब्राह्मणपक्षो भविष्ये ।\\
नान्दीमुखान्समुद्दिश्य पितॄन् पञ्चद्विजोत्तमान् ।\\
भोजयेद्विधिवत्प्राज्ञो वृद्धिश्राद्धे प्रदक्षिणम् ॥\\
वृद्धिश्राद्धाकरणे च प्रत्यवायो वृद्धशातातपेनोक्तः ।\\
वृद्धौ न तर्पिता ये वै पितरो गृहमोधिभिः ॥\\
तद्दानमफलं सर्वमासुरो विधिरेव सः ॥ इति । इत्याभ्युदयिकम् ।\\
अथ सामान्यकृष्णपक्षश्राद्धम् ।\\
तत्र कात्यायनः ।\\
अपरपक्षे श्राद्धं कुर्वीतोर्ध्वं वा चतुर्थ्याः ।\\
तश्च सकृदेव न तु प्रतितिथ्यावर्त्तते । ''अश्वयुक्कृष्णपक्षे तु\\
श्राद्धं कार्यं दिने दिने'' इतिवद्वीप्साया अश्रवणात् । वसन्ते ज्योति-\\
ष्टोमवत् । तेन प्रतिपदादिदर्शान्तासु तिथिषु मध्ये यस्यां कस्यां\\
चित्तिथौ श्राद्धं कार्यम् । फलविशेषकामनायां तु पञ्चमप्रभृति यस्यां\\
कस्यां चित्, ततोऽपि विशेषकामनायां दशमीप्रभूति, ततोऽपि

{ ३१६ वीरमित्रोदयस्य श्राद्धप्रकाशे-\\
विशेष कामनायाममावास्यायाम् । अत एव-\/- -\\
कात्यायनः ।\\
ऊर्द्ध्वं वा चतुर्थ्या: ।\\
मनुरपि ।\\
कृष्णपक्षे दशम्यादौ वर्जयित्वा चतुर्द्दशीम् ।\\
निगमोऽपि । {[}अ ३ श्लो० ५७६ {]}\\
अपरपक्षे यदहः सम्पद्यते अमावास्यायां विशेषेण ।\\
एवं च निगमवचनैकवाक्यतया मनुकात्यायनोक्तावपि पक्षौ\\
फलविशेषार्थौ, न तु अनुकल्पभूतौ ज्ञेयौ । एते पक्षाः निरग्निकवि-\\
षयाः । साग्निकस्य तु अमावास्यामेव ''न दर्शेन विना श्रद्धमाहि\\
ताग्नेर्द्विजन्मन'' इति मनुवचनात् । तदा चामावास्याश्राद्धस्य तस्य\\
च तन्त्रेण सिद्धिः । प्रतिपदादिपक्षेषु चतुर्दशीवर्जनं नन्दादिवर्जनं\\
च {[}न{]} कार्यम् ।\\
नभस्यस्यापरे पक्षे श्राद्धं कार्यं दिने दिने ।\\
नैव नन्दादि वर्ज्यं स्यान्नैव वर्ज्या चतुर्दशी ॥\\
इति भाद्रपदापरपक्षे तद्वर्जनविषेधात्, (१) अन्यत्र तद्वर्जनप्रती-\\
तैः ॥ नन्दादिकं च-\\
नारदसंहितायाम् ।\\
न नन्दासु भृगोर्वारे रोहिण्यां च त्रिजन्मसु ।\\
रेवत्यां च मघायां च कुर्यादापरपक्षिकम् ॥\\
नन्दा=प्रतिपत्षष्ठ्येकादश्यः । त्रिजन्मानि=आद्यदश\\
मैकोनविंशानि ।\\
यदपि\\
भानौ भौमे त्रयोदश्यां नन्दाभृगुमघासु च ।\\
पिण्डदानं मृदा स्नानं न कुर्यात्तिलतर्पणम् ॥\\
इति स्मृत्यन्तरे पिण्डदाननिषेधवचनं तदपि, आद्धोपलक्षणा-\\
र्थम् । इति सामान्यकृष्णपक्षश्राद्धनिर्णयः ।\\
अथ महालयश्राद्धम् ।\\
तत्र शाठ्यायनिः ।\\
नभस्यस्यापरे पक्षे तिथिषोडशकस्तु यः ।\\
कन्यागतान्वितश्चेत्स्यात्स कालः श्राद्धकर्माणि ॥\\
अत्र देवता उक्ताश्चतुर्विंशतिमते ।

% \begin{center}\rule{0.5\linewidth}{0.5pt}\end{center}

 (१) सकृन्महालयादावित्यर्थः ।

{ भाद्रापरपक्षीय श्राद्धभेदनिरूपणम् । ३१७\\
क्षयाहं वर्जयित्वा तु स्त्रीणां नास्ति पृथग्विधिः ॥\\
केचिदिच्छन्ति नारीणां पृथक् श्रद्धं द्विजोत्तमाः ।\\
आचार्यगुरुशिष्येभ्यः सखिज्ञातिभ्य एवं-\/- च ॥\\
सर्वेभ्यश्च पितृभ्यश्च तत्पत्नीभ्यस्तथैव च ।\\
पिण्डानेभ्यस्सदा दद्यात्पृथक् भाद्रपदे द्विजः ॥\\
अत्र क्षयाहं वर्जयित्वेत्यनेन षड्दैवत्यमुक्तम् । उत्तरार्द्धेन द्वाद\\
दशदैवत्यमुक्तम् । आचार्येत्यादिना सर्वदैवत्यम् । अन्या अपि देवता-\\
हेमाद्रौ पुराणान्तरे ।\\
उपाध्यायगुरुश्वश्रृपितृव्याचार्य मातुलाः ।\\
श्वशुरभ्रातृतत्पुत्रपुत्रत्विकशिष्यपोषकाः \textbar{}\textbar{}\\
भगिनीस्वामिदुहितृजामातृभगिनीसुताः ।\\
पितरौ पितृपक्षीनां पितुर्मातुश्च या स्वसा ।\\
सखिद्रव्यदशिष्याद्यास्तीर्थे चैव महालये ॥ इति ।\\
पितृपत्न्यः सपत्नमातरः ।\\
धर्मस्तु नवदेवत्यमाह -\\
महालये गयाश्राद्धे वृद्धौ चान्वष्टकासु च ।\\
नवदेवस्य मत्रेष्टं शेषं षट्पुरुषं विदुः ॥ इति ।\\
पार्वणक्रममाह-\\
छागलेयः ।\\
क्षयाहे केवलाः कार्याः वृद्धावादौ प्रकीर्तिताः ।\\
सर्वत्रैव तु मध्यस्था नान्त्याः कार्यास्तु मातरः ॥\\
सर्वत्र = महालयादाविति हेमाद्रिः ।\\
सर्वदेवत्यपक्षे पार्वणैकोदिष्टव्यवस्थोक्ता-\\
जातूकर्ण्येन ।\\
सपिण्डीकरणादूर्ध्वं पित्रोरेव हि पार्वणम् ।\\
पितृव्यभ्रातृमातृणामेकोहिष्टं सदैव हि ॥ इति ।\\
मातरः =लपत्तमातरः ।\\
पित्रोरेव हाति = पितृग्रहणं मातामहस्याप्युपलक्षकम् ।\\
पार्वणैकोद्दिष्टानां पौर्वापर्यमाह -\\
मरीचिः ।\\
बद्येकत्र भवेयातामेकोद्दिष्टं च पार्वणम् ।\\
पार्वणं स्वाभनिर्वर्त्य एकोद्दिष्टं समापयेत् ॥

{३१८ वीर मित्रोदयस्य श्राद्धप्रकाशे-\\
अत्रान्नपाकस्तन्त्रेण ।\\
महालये गयाश्राद्धे गतासूनां क्षयेऽहनि ।\\
तन्त्रेण अपणं कृत्वा श्राद्धं कुर्यात्पृथक् पृथक् ॥\\
इति पुलस्त्योक्तेः ।\\
अत्र विश्वेदेवा धुरिलोचनसंज्ञकाः ।\\
" अपि कन्यागते सूर्ये काम्ये च धुरिलोचनौ" इत्यादिपुराणमत् ।\\
}{इति भाद्रपदपक्षश्राद्धनिर्णयः ।}{\\
अस्मिन्पक्षे भरण्यां श्राद्धमुक्तम् ।\\
मात्स्ये -\\
भरणी पितृपक्षे तु महती परिकीर्त्तिता ।\\
अस्यां येन कृतं श्राद्धं स गयाश्राद्धकृद्भवेत् ।\\
}{इति भरणीश्राद्धम् ।}{\\
अत्रापरपक्षे त्रयोदश्यां श्राद्धं कार्यम् ।\\
अपि जायेत सोऽस्माकं कुले कश्चिन्नरोत्तमः ।\\
प्रावृट्कालेऽसितिपक्षे त्रयोदश्यां समाहितः ।\\
मधुप्लुतेन यः श्राद्धं पायलेन समाचरेत् \textbar{}\textbar{}\\
}{ इति विष्णुक्तेः ।}{\\
तथा मघास्वपि\\
मधुमांसैश्च शाकैश्च पयसा पायसेन च ।\\
एष नो दास्यति श्राद्धं वर्षासु च मघासु च ॥\\
इति वसिष्ठोक्तेः । उभययोगे प्राशस्त्यमुक्तं -\\
}{ विष्णुषर्मोत्तरे ।}{\\
मघायुक्ता च तत्रापि शस्ता राजँस्त्रयोदशी । इति ।\\
प्राच्यास्त्वेतस्मादेव वाक्यात्-\\
प्रौष्ठपद्यामतीतायां मघायुक्तां त्रयोदशीम् ।\\
प्राप्य श्राद्धं हि कर्त्तव्यं मधुना पायसेन च ॥\\
इति शङ्खवचनाच्च मघायुक्तत्रयोदश्यां श्राद्धमाहुः ।\\
त्रयोदशीश्राद्धं चैकवर्गस्य न कार्यं,\\
श्राद्ध नैकस्य वर्गस्य त्रयोदश्यामुपक्रमेत् ।\\
न तृप्तास्तत्र ये यस्य प्रजा हिंसन्ति तस्य ते ॥\\
इति कार्ष्णाञिनिवचनात् । एकवर्गस्प= पितृमातृवर्गस्य । किन्तु
मा\\
तामहवर्गस्यापि कार्यम् । एकवर्गमात्रप्राप्तिश्च भ्रमादाश्वलायनगृ

{ शस्त्रादिहतचतुर्दशी श्राद्धनिरूपणम् । ३१९\\
ह्याद्वेति ज्ञेयम् ।\\
पितृव्यादीनामपि पार्वणमेव कार्यमित्यन्ये ।\\
एतश्चापिण्डकं कार्यम् ।\\
अयनद्वितीये श्राद्धं विषुवद्वितये तथा ।\\
युगादिषु च सर्वासु पिण्डनिर्वपणाइते \textbar{}\\
इति }{पुक्तरस्योक्तेः,}{\\
महालयत्रयोदश्यां पिण्डनिर्वपणं द्विजः ।\\
ससन्तानो नैव कुर्यान्नित्यं ते कवयो विदुः ॥\\
इति बृहत्पराशरोक्तेश्च । अत्र विभागादिपक्षे महालयश्राद्धं तन्त्रेण\\
तदा सपिण्डमेव, त्रयोदशीश्राद्धे पिण्डपर्युदासस्य कुतत्वात् । के-\\
वलत्रयोदशीश्राद्धं त्वपिण्डमेवेति ध्येयम् । एतञ्च पुत्रवद्गुहस्थ.\\
व्यतिरिक्तेन कार्यम् ।\\
"त्रयोदश्यां तु वै श्राद्धं न कुर्यात्पुत्रवान्गृही "तिवचनात् ।\\
यस्तु -\\
कृष्णपक्षे त्रयोदश्यां यः श्राद्धं कुरुते नरः ।\\
पञ्चत्वं तस्य जानीयाज्ज्येष्ठपुत्रस्य निश्चितम् ॥\\
इति ज्योतिर्वृहस्पतिना निषेधः कृतः, स त्रयोदश्यां बहुपुत्रो
बहु'\\
मित्रो दर्शनीयापत्यो युवमारिणस्तु भवन्ती "ति }{आपस्तम्बेन}{ त्रयोद.\\
शीश्राद्धस्य युवमारित्वदोषोक्तेः काम्यश्राद्धविषय उपसंहर्त्तव्यः,\\
एकवर्गयजनविषयो वा सपिण्डश्राद्धविषयो वा पुत्रवद्गृहस्थकर्तृ.\\
कश्राद्धविषयो वेति }{हेमाद्रि }{।\\
अथ }{शस्त्रहतचतुर्दशी निर्णयः}{ ।\\
तत्र }{मरीचिः}{ ।\\
(१) विषसर्पश्वापदादितिर्यग्ब्राह्मणघातिनाम् ।\\
चतुर्दश्यां क्रियाः कार्या अन्येषां तु विगर्हिता ॥ इति ।\\
विषादिभिर्ब्राह्मणान्तैर्घातो येषां ते इति विग्रहः, (ते) न तु तान् ये\\
हन्तीति, विषये [विषादौ] असम्भवात्, "तेषां ये ब्राह्मणैर्हता" इति
।\\
}{ब्रह्मपुराणाच्च । नागरे}{ -\\
अपमृत्युर्भवेद्येषां शस्त्रमृत्युरथापि वा ।\\
उपसर्गमृतानां च विषमृत्युमुपेयुषाम् ।\\
वह्निमा च प्रदग्धानां जलमृत्युमुपेयुषाम् ॥

% \begin{center}\rule{0.5\linewidth}{0.5pt}\end{center}

( १ ) विषशस्त्रश्वापदादितिर्यग्ब्राह्मणघातिनामित्यन्यत्र पाठः ।

{३२० वीरीमत्रोदयस्य श्राद्धमकाशे-\\
सर्पव्याघ्रहतानां च शुङ्गैरुद्वन्धनैरपि ।\\
श्राद्धं तेषां प्रकर्त्तव्यं चतुर्दश्यां नराधिप ।\\
मार्कण्डेयपुराणे -\/-\/-\\
युवानः पितरो यस्य मृता शस्त्रेण वा हताः ।\\
तेन कार्ये चतुर्दश्यां तेषां तृप्तिमभीप्सता ॥ इति ।\\
युवत्वं च षोडशवर्षादूर्ध्वं त्रिंशद्वर्षपर्यन्तमिति श्राद्धकल्पः ।
यक्ष.\\
भूतगुह्यादिभिर्भरणमुपसर्गमरणम् ।\\
प्रचेताः -\\
वृक्षारोहणलोहाद्यैर्विद्युज्ज्वालाविषादिभिः ।\\
नखदंष्ट्रर्विपन्नानां तेषां शस्ता चतुर्द्दशी ॥\\
अत्राविधिमृतानामेव चतुर्द्दश्यामिति नियमः । न तु चतुर्द्दश्या\\
मेव तेषामिति, श्राद्धान्तरविलोपप्रङ्गाद,\\
विषसर्पश्वापदादितिर्यक्ब्राह्मणघातिनाम् ।\\
चतुर्दश्यां क्रिया कार्या अन्येषां तु विगर्हिता ॥ इति ।\\
मरीचिचचनाक्च । स्त्रीणामपि उद्देश्यविशेषणस्याविवक्षित.\\
त्वादित्युपाध्यायाः । क्वचिद्वैधमरणेऽपि चतुर्दश्यां श्राद्धं कार्यं यथाह
-\\
मनुः ।\\
शातिश्रैष्टयं त्रयोदश्यां चतुद्दंश्यां तु सुप्रजाः ।\\
प्रीयन्ते पितरस्तस्य ये च शस्त्रहता रणे ॥\\
शस्त्रमुपलक्षणं प्रायश्चित्तानुगमनाद्यर्थानाम् \textbar{}
जलाग्न्यादिमृते-\\
ऽपवादमाह -\\
शाकटायनः ।\\
जलाग्निभ्यां विपन्नानां संन्यासे चागृहे पथि ।\\
श्राद्धं कुर्वीत तेषां वै वर्जयित्वा चतुर्दशीम् ॥ इति ।\\
एतच्चैकोद्दिष्टं कार्यम् ।\\
चतुर्दश्यां तु यच्छ्राद्धं सपिण्डीकरणात्परम् ।\\
एकोद्दिष्टविधानेन तत्कार्यं शस्त्रघातिनः ॥\\
इति गार्ग्योक्ते:-\\
एतच्चाकृतसपिण्डीकरणस्य न भवति सपिण्डीकरणात्पर,\\
मित्युक्तेः, एतश्च दैवयुक्तं कार्यं ।\\
प्रेतपक्षे चतुर्दश्यामेकोद्दिष्टं विधानतः ।\\
दैवयुकं तु यच्छ्राद्धं पितॄणामक्षयं भवेत् ॥

{ दौहित्र कर्तृकश्राद्धनिरूपणम् । २२१\\
इति पारिजातोक्तवचनात् ।\\
शूलपाणिस्तु तत्रापि पार्वणमेवाह - एकोद्दिष्टवाक्यानां नि.\\
र्मूलत्वात् । अन्ये तु पित्रादीनां पार्वण
भ्रात्रादीनामेकोद्दिष्टमित्या.\\
हुः । तत्तुच्छम् ।\\
समत्वमागतस्यापि पितुः शस्त्रहतस्य वा ।\\
चतुर्दश्यां तु कर्त्तव्यमेकोद्दिष्टं महालये ॥\\
इति भविष्ये पितुरप्येकोद्दिष्टविधेः ।\\
शस्त्रहतस्य चतुर्दश्यां श्राद्धे कृतेऽपि महालये दिनान्तरे पा-\\
र्वणमितरतृप्त्यर्थं कार्यमेव, पितारे शस्त्रहते एकमेकोदिष्टं
}{कार्यं}{,\\
द्वयोः शस्त्रहतयोर्द्वे एकोदिष्टे समानतन्त्रे कार्यं त्रिषु त्रीणि
समा-\\
नतन्त्राणि कार्याणि इति देवस्वामी ।\\
अन्ये तु त्रिषु पार्वणमेव }{कार्यं}{ ।\\
पित्रादयस्त्रयो यस्य शस्त्रैर्यातास्त्वनुक्रमात् ।\\
स भूते पार्वणं कुर्यात् ।\\
इति बृहत्पराशरोक्तेः ।\\
एकस्मिन्वा द्वयोर्वापि विद्युच्छस्त्रेण वा हते ।\\
एकोद्दिष्टं सुतः कुर्यात्त्रयाणां दर्शवत् भवेत् ।\\
इति पृथ्वीचन्द्रोदयोदाहृतवचनाच्चेत्याहु: ।\\
चतुर्दश्यामेव शस्त्रादिना मृतस्य पार्वणमेकोद्दिष्टं वा यथा-\\
चारं कार्यम्, यस्तु चतुर्दश्यां पार्वणनिषेधः स तन्निमित्तस्यैव न\\
क्षयाहनिमितस्येति तदा चतुर्द्देशीनिमित्तकमेकोद्दिष्ट 'पृथग्वा का\\
र्यम् । चतुर्दश्यां विघ्नेनैकोद्दिष्टासम्भवे तत्पक्षे दिनान्तरे
पार्वणं\\
कार्यमित्युक्तं हेमाद्रौ । इति चतुर्दशीश्राद्धम् ।\\
आश्विनशुक्ल प्रतिपदि मातामह श्राद्धमुक्तं\\
हेमाद्रौ ।\\
जातमात्रोऽपि दौहित्रो विद्यमानेऽपि मातुले ।\\
कुर्यान्मातामहश्राद्ध प्रतिपद्याश्विनेऽसिते ॥ इति ।\\
एतच्च जीवत्पितृकेण }{कार्यं}{ सपिण्डञ्चेति दाक्षिणात्याचारः सर्वैः\\
कार्यमविशेषवचनादिति तु युक्तं, दाक्षिणात्यभिन्नशिष्टाचाराच्च ।\\
इति दौहित्रकर्तृकश्राद्धम् ।\\
४१ वी० मि०\\


३२२ वीरमित्रोदयस्य श्राद्धप्रकाशे-

{ अथ नित्यश्राद्धम् ।\\
कौर्मे ।\\
एकं तु भोजयेद् विप्रं पितृनुद्दिश्य सत्तमम् \textbar{}\\
नित्यश्राद्धं तु तहिष्टं पितृयज्ञो गतिप्रदः \textbar{}\textbar{}\\
अशक्तं प्रत्याह-\\
मनुः ।\\
कुर्यादहरहः श्राद्धमन्नाद्येनोदकेन वा ।\\
पयोमूलफलैर्वापि इति । [ अ० ३ श्लो० ८२ ]

{कौर्मे ।\\
उद्धृत्य वा यथाशक्ति किञ्चिदन्नं समाहितः ।\\
वेदतत्वार्थविदुषे द्विजायैवोपपादयेत् ॥\\
अहरहः कर्त्तुमसार्थ्य आह-\\
देवलः ।\\
अनेन विधिना }{श्राद्धं}{ कुर्यात्संवत्सरं द्विजः ।\\
द्विश्चतुर्वा यथाश्राद्धं मासे मास दिने {[} दिने{]} ॥\\
सवत्सरे = तन्मध्ये द्वि. = षट्सु षट्सु मासेषु चतुः =मासत्रये मा.\\
सत्रये । अत्र विशेषमाह -\\
हारीतः ।\\
नित्यश्राद्धमनर्थ्यं स्यात् ।\\
अनर्थ्यं = पिण्डादिवर्जितम् ।\\
प्रचेताः ।\\
नामन्त्रणं न होमं च नाद्वानं न विसर्जनम् ।\\
भविष्योत्तरे-\\
आवाहनस्वधाकारपिण्डाग्नौ करणादिकम् ।\\
ब्रह्मचर्यादिनियमो विश्वेदेवास्तथैव च \textbar{}\textbar{}\\
नित्यश्राद्धे त्यजेदेतदिति, तत्रैव-\\
प्रदद्यादक्षिणां शक्त्या नमस्कारैर्विसर्जयेत् इति ।\\
यत्तु न विसर्जनमित्युक्तं तद् "वाजे वाज" इति मन्त्रेण ।\\
इति नित्यश्राद्धम् ।

{ सांवत्सरिकश्राद्धनिरूपणम् \textbar{} ३२३\\
[ अथ सांवत्सरिक श्राद्धम् । ]

{ तन्नित्यम् ।\\
प्रतिसंवत्सरं }{कार्यं}{ मातापित्रोर्मृतेऽहनि ।\\
पितृव्यस्याप्यपुत्रस्य भ्रातुर्ज्येष्ठस्य चैव हि ॥\\
इति ब्रह्माण्डपुराणोक्तेः । अत्र प्रतिसंवत्सरमिति वीप्सार्थकप्रतिप.\\
दोपादानान्नित्यत्वम् । भ्रातुर्ज्येष्ठस्येति = अत्र
ज्येष्ठग्रहणात्कनिष्ठस्थानाव\\
श्यकम् । अत एवं-\/- -\\
न पुत्रस्य पिता कुर्यान्नानुजस्य तथाग्रजः \textbar{}\\
अपि स्नेहेन कुर्यातां सपिण्डीकरण विना \textbar{}\textbar{}\\
इति सपिण्डीकरणातिरिक्ते श्राद्धे सत्ति स्नेहेऽधिकार उक्तः ।\\
लौगाक्षिरपि ।\\
}{श्राद्धं}{ कुर्यादवश्यं हि प्रमीतपितृकः स्वयम् ।\\
इन्दुक्षये मासि मासि वृद्धौ प्रत्यक्षमेव च ॥ इति ।\\
स्वयमिति = सत्ति सामर्थ्ये । असामर्थ्य तु प्रतिनिधिनापि कार्यम्
\textbar{}\\
असावेतत्त इति यजमानस्य पित्रे ऋत्विगादिः पिण्डान् दद्यादि-\\
ति स्मृतेः ।\\
केचित्तु प्रतिनिधिविधानमिन्दुक्षयादिव्यतिरिक्तश्राद्धपरम्, इह\\
स्वयमित्युपादानादित्याहुः ।\\
अत्र च पार्वणैकोद्दिष्टविधायकानि परस्परविरुद्धानि बहूनि वा-\\
क्यानि दृश्यन्ते तत्रैकोद्दिष्टविधायकानि तावल्लिख्यन्ते ।\\
यमः ।\\
सपिण्डीकरणादूर्ध्वं प्रतिसंवत्सरं सुतः ।\\
एकोद्दिष्टं प्रकुर्वीत पित्रोरन्यत्र पार्वणम् \textbar{}\textbar{}}

{ गार्ग्यः ।\\
कृतेऽपि हि सपिण्डत्वे गणसामान्यतां गते ।\\
प्रतिसंवत्सरं श्राद्धमेकोद्दिष्टं विधीयते ॥\\
लौगाक्षिरपि ।\\
सपिण्डीकरणात्पूर्वमेकोद्दिष्टं सुतः पितुः ।\\
ऊर्ध्वं च }{पार्वणं}{ कुर्यात्प्रत्यब्दमितरेण तु \textbar{}\textbar{}\\
इतरेण= एकोद्दिष्टेन । प्रत्यब्दमितरेणेति = प्रत्यब्दं विनेत्यर्थ इति
के\\
चित् ।\\
एकोद्दिष्टं परित्यज्य }{पार्वणं}{ कुरुते यदि ।\\
अकृतं तद्विजानीयात्स मातृपितृघातकः \textbar{}\textbar{}}

{३२४ वीरमित्रोदयस्य श्राद्धप्रकाशे-\\
इति यमेन पार्वणे दोषमभिधायैकोद्दिष्ट विहितम् ।\\
व्यासः ।\\
}{सपिण्डीकरणादूर्ध्वं}{ यत्र यत्र प्रदीयते ।\\
तत्र तत्र त्रयं }{कार्यं}{ वर्जयित्वा मृतेऽहनि \textbar{}\textbar{}\\
प्रतिसंवत्सरं यत्र मातापित्रोः प्रदीयते ।\\
अदैवं भोजयेच्छ्राद्धं पिण्डमेकं च निर्वपेत् \textbar{}\textbar{}\\
इत्येवमादीनि ।\\
तथा पार्वणविधायकानेि ।\\
जमदग्निः ।\\
आपाद्य सहपिण्डत्वमौरसो विधिवत्सुतः ।\\
कुर्वीत दर्शषच्छ्राद्धं मातापित्रोर्मृतेऽहनि \textbar{}\textbar{}\\
शातातपः ।\\
सपिण्डीकरणं कृत्वा कुर्यात्पार्वणवत्सदा ।\\
प्रतिसंवत्सरं विद्वाञ्छागलेयोदितो विधिः ।\\
इत्यादीनि ।\\
एवं संशये केचिदाहुः ।\\
औरसक्षेत्रजौ पुत्रो विधिना पार्वणेन तु ।\\
प्रत्यब्दमितरे कुर्युरेकोद्दिष्टं सुता दश ॥ इति ।

{तथा-\\
यत्र यत्र प्रदातव्यं सपिण्डीकरणात्परम् \textbar{}\\
पार्वणेन विधानेन देयमग्निमता सदा ॥\\
इति जाबालिमत्स्यपुराणवाक्याभ्यां साग्न्योरौरसक्षेत्रजयो: पार्वण-\\
मितरेषामेकोद्दिष्टमिति व्यवस्थेति । न च सदेत्युपादनालाग्निक-\\
योनित्यं निरनिकयोरौरसक्षेत्रजयोः पाक्षिकं पार्वण, न तु नियमे नै.\\
कोद्दिष्टामेति वाच्यम् । एवं सति विधेर्वैषम्यं स्यात्, एक एवं-\/-
विधिः\\
सान्भिकयोर्नित्यचत्पार्वणंविद्ध्यान्निरग्निकयोस्तु पाक्षिकमिति । किञ्च\\
यदि प्रत्यब्दपुरस्कारेण पार्वणविधिः स्यात्तदा विसेव{[} विरुद्धो {]}\\
भयदर्शनेन पार्वणस्यानियमतः प्राप्ताविदं नियमार्थे स्यात् न च\\
तदस्ति ।\\
न च-\\
ये सपिण्डीकृताः प्रेता न तेषां तु पृथक्क्रिया\\
यस्तु कुर्यात्पृथक् पिण्डं पितृहा सोऽभिजायते ॥ इति ।

{ सांवत्सरिक श्राद्धनिरूपणम् । ३२५\\
तथा ।\\
पार्वणेन विधानेन सांवत्सरिकमिष्यते ।\\
प्रतिसंवत्सरं कार्यं विधिरेष सनातनः ॥ इति ।

{तथा ।\\
सपिण्डीकरणं कृत्वा कुर्यात्पार्वणवत्सदा ।\\
प्रतिसंवत्सरं विद्वानित्येवं मनुरब्रवीत् \textbar{}\textbar{}\\
इति कूर्मपुराणशातातपभविष्यपुराणवाक्येषु प्रतिसंवत्सरपदसत्वा\\
त्कथं न वार्षिकपुरस्कारेण पार्वणविधिरिति वाच्यम् । नह्यत्र\\
प्रतिसंवत्सरपदेन क्षयाहश्राद्धमुच्यते, किन्तु संवत्सरं क्रियमाणं\\
युगमन्वाद्यपि, तथा च पार्वणविधिस्तत्रैव सावकाश इति न\\
क्षयाहमास्कन्दति । या त्वेकोद्दिष्टनिन्दा, सापि साग्निकौरसक्षे\\
त्रजविषया, तेषां पार्वणविधानात् तस्मात्साग्न्योरौरसक्षेत्रजयोः\\
पार्वणमन्येषामेकोद्दिष्टम् । तत्रापि पुत्रिकापुत्रस्यौरस
समत्वात्तस्या.\\
पि साग्ने: पार्वणमिति । न च बृहन्नारदीये " विप्रः क्षयाहपूर्वेद्यु" रि\\
त्युपक्रम्य "अग्न्यभावे तु विप्रस्य पाणौ होमो विधीयत" इत्यनेन\\
निरग्नीनां पाणिरूपविशेषाभिधानेन निरग्नीनामपि क्षयाहे पार्वणवि\\
धिः कल्यतामिति वाच्यम् । साग्निं प्रश्यपि प्रत्यक्षाग्न्यभावदशा\\
यामेतद्विध्युपपत्तेः । अत्र हि विच्छिन्नाग्निरेवाग्न्यभाववानुक्तस्तस्य\\
चाग्निविच्छेददशायामव्यस्त्येव । आधानजन्य संस्कारवत्वादाहिता.\\
शित्वम् । एवं-\\
बहग्नयस्तु ये विप्रा ये चैकाग्नय एष च ।\\
तेषां सपिण्डनादूर्ध्वभेकोद्दिष्टं न पार्वणम् ॥\\
इति भृगुवाक्येऽपि एकाग्निमानित्यनेन विच्छिन्नत्रेताग्निमानर्धा.\\
धान्येवाभिहितोऽजस्त्राग्निवेति सोऽव्याहिताग्निरेवेति ।\\
यदपि,\\
कर्त्तव्यं }{पार्वणं}{ राजन्नैकोदिष्टं कदाचन ।\\
सुबहून्यत्र वाक्यानि मुनिगीतानि वक्षते ॥\\
अल्पानि चैव वाक्यानि एकोद्दिष्टं प्रचक्षते ।\\
तस्माद्वचनसामर्थ्यात्पार्वणं स्यात्मृतानि ॥\\
इति सर्वसाधारणं सुमन्तुवचनं तदपि न्यायोपन्यासेन न श्रुति-\\
मूलकम् । न चात्र दर्शितन्यायस्यावकाशः । यदि हि मनुवाक्यानां

{३२६ वीरमित्रोदयस्य श्राद्धप्रकाशे-\\
परस्परसंवादेनैव प्रामाण्यं स्यात्तदैतत्कथनं युज्यते, न च तदस्ति,\\
मुनिवाक्यानां स्वतन्त्राणामेव श्रुतिकल्पकत्वादिति । न च तस्यैव\\
वाक्यस्य तर्ह्यप्रमाण्यमापद्येतेति वाच्यम् । तस्य पार्वणविधायक\\
वाक्यशेषत्वेनापि प्रमाणत्वात् । एतञ्च साग्न्यौरसक्षेत्रजपुत्रिकापुत्रै\\
रप्यमावास्याप्रेतपक्षयोरेव क्षयाहे }{पार्वणं}{ }{कार्यं}{ ।\\
अमावास्यां क्षयो यस्य प्रेतपक्षेऽथवा भवेत् ।\\
}{पार्वणं}{ तस्य कर्त्तव्यं नैकोद्दिष्टं कदाचन ॥\\
इति तेनैव पार्वणस्मृतौ विहितत्वात् । न चैवं सत्ति तेषामेत-\\
स्कालातिरिक्तकाले श्राद्धाधिकारविध्यभावाच्छ्राद्धमेव कालान्तरे\\
क्षयाहे सत्ति न स्यात् "एकोद्दिष्टं सुता दश" इति वचने एकोद्दि\\
ष्टस्य तान्प्रत्येव विहितत्वात्पार्वणस्य चैतत्कालमात्र विषयत्वादिति\\
वाच्यम् । न ह्यत्र कालो नियम्यते, किन्तु अमावास्याक्षयादिरूपे\\
काले साग्न्योरौरलक्षेत्रजयो: }{पार्वणम्}{ । एवं च तदतिरिक्त काले\\
सामान्यविहितमेतेषामध्येकोद्दिष्टमेव पूर्वोदाहृतवाक्यैरे कोद्दिष्टस्य\\
सामान्यतो विहितत्वात् । तस्मात्साग्न्योरौरस क्षेत्रजयोरमावास्यायां\\
प्रेतपक्षे वा मृतस्य पार्वणम् । अन्येषामेतेषामपि च कालान्तरे
एको.\\
}{द्दिष्टमिति}{ ।\\
हेमाद्रिस्तु ।\\
आपाद्य }{सहपिण्डत्वमौरसो}{ विधिवत्सुतः ।\\
कुर्वीत दर्शवच्छ्राद्धं मातापित्रोः क्षयेऽहनि \textbar{}\textbar{}\\
औरसक्षेत्रजौ पुत्रौ विधिना पार्वणेन तु ।\\
प्रत्यब्दमितरे कुर्युरेकोद्दिष्टं सुता दश ।\\
}{इत्यादिजमदग्निजावालिवाक्यैरौर}{ लक्षेत्रजयोस्तत्समस्य }{पुत्रिका-}{\\
पुत्रस्य च }{पार्वणं}{ तदपि }{साग्नीनाम्}{ ।\\
न पैतृयज्ञियो होमो लौकिकेऽग्नौ विधीयते ।\\
न }{दर्शेन}{ विना श्राद्धमाहिताग्नेर्द्विजन्मनः ॥\\
{[} मनु० }{अ०}{ ३ }{श्लो०}{ २८२ {]}\\
इति मनुवचनेनाहिताग्नेर्दर्शेन विना दर्शोपलक्षितविधि विना\\
श्राद्धाभावकथनात् ।\\
तथा ।\\
अग्निप्रधानं सर्वेषामनुष्ठानं गृहाश्रमे ।\\
तद्योगात्कृतसामर्थ्यात्सर्वत्राहन्ति पार्वणम् ॥

{ सांवत्सरिक श्राद्धनिरूपणम् । ३२७\\
इत्यादिकार्ष्णाजिनिप्रभृतिवाक्यैराहिताग्नेः पार्वणविधानाच्च ।\\
निरग्नीनां तु औरसक्षेत्रजपुत्रिकापुत्राणां विकल्पः । औरसानां
सा-\/-\\
ग्नीनां }{निरग्नीनां}{ च "एकोद्दिष्ट सुता दश" इति नियमादेको दृदि-\\
ष्टमेव । तदुकं -\\
भविष्यपुराणेऽपि ।\\
निरग्नेरौरसस्योक्त मे कोद्दिष्ट मृताहनि ।\\
प्रत्यब्दं }{पार्वणं}{ साग्नेरन्येषां तु न }{पार्वणम्}{ ॥\\
अन्येषां = इतकादीनां दशविधानां न }{पार्वणं}{ किन्तु एकोद्दिष्टमे\\
}{वेति}{ । अमावास्याप्रेतपक्षमृतानां तु पार्वणमेव, "अमावास्यां क्षय\\
इत्यत्र वाक्ये "नैकोद्दिष्टं कदाचन" इति वाक्यशेषपर्यालोचनया\\
एकोद्दिष्टस्य निषिद्धत्वात्, तत्रापि प्रेतपक्षप्रमीतस्य पितुरेष पार्व\\
णम् । अन्येषामेकोद्दिष्टमेव । तथा च हेमाद्वावेव वचनं ।\\
प्रेतपक्षप्रमीतस्य पितुः कुर्वीत }{पार्वणम्}{ ।\\
पितृव्यभ्रातृमातृणामेकोद्दिष्टं न }{पार्वणम्}{ ॥ इत्याह ।\\
मिताक्षरामदनरत्नयोस्तु । औरसक्षेत्रजयोः साग्नेर्निरग्नेश्च पार्वणको\\
द्दिष्टोभयविधिवाक्यपर्यालोचनया विकल्पः । स च "येनास्य\\
पितरो याता" इत्यादिना वंशसमाचाराद्व्यवस्थितः । इति अमावास्या\\
प्रेतपक्षयोर्मृतस्य ।\\
दण्डग्रहणमात्रेण नैव प्रेतो भवेद्यतिः ।\\
अतः सुतेन कर्त्तव्यं }{पार्वणं}{ तस्य सर्वदा \textbar{}\textbar{}\\
इति प्रचेतोवचनात् यतेश्च पार्षणमेवेति ।\\
नव्यास्तु "प्रतिसंवत्सरं }{कार्यं}{" तथा "श्राद्धं कुर्यादवश्यं हि प्र\\
मीतपितृकः स्वयम्" इत्यादिवाक्यैः सामान्यतः क्षया श्राद्धमात्रे\\
विहिते ततश्चाकाडावशात्सर्वप्रकृतिकस्य पार्वणस्य प्रसको "प्र\\
तिसंवत्सरं श्राद्धमेकोद्दिष्टं सुतः पितुः" इत्यादिनातिदेशप्राप्तपा.\\
र्वणबाघपुरःसरमे कोद् दिष्टयोर्विकल्पसिद्धौ "औरसक्षेत्रजौ पुत्रौ\\
विधिना पार्वणेन तु" इत्यस्य, तथा "एकोद्दिष्ट हि कर्त्तव्यमौरसेन\\
मृतेऽहनी "त्यस्य पैठीनसिवाक्यस्य, तथा साग्निनिरभिपुरस्कारेण पार्व\\
णैकोद्दिष्ट विधायकानां वाक्यानां निरर्थकत्वमापद्येत ततश्च विशे\\
षवाक्यानां निरर्थकत्वभयादेव स्वस्ववाक्योपस्थापिता एवं-\/- विशे\\
षाः परस्परविशेषणविशेष्यभावापन्ना एभिर्विधीयन्ते, यथौरसस्य

{३२८ वीरमित्रोदयस्य श्राद्धमकाशे-\\
~\\
पार्वणविधायकमेकोद्दिष्टविधायकं च, अग्निमतश्च पार्वणविधाय-\\
मेव । तथानौरसानामेकोद्दिष्टविधायकमेव । एवं चाग्नि-\\
मद्वाक्यस्यौरसपुरस्कारेण पार्वणविधायकस्य चैकवाक्यतया-\\
ग्निमानौरसः }{पार्वणं}{ कुर्यादित्यर्थः सम्पद्यते, तथौरसपुर-\\
स्कारेणैकोद्दिष्टवाक्यानां निरग्निपुरस्कारेण चैकोद्दिष्टवाक्यानां\\
चैकवाक्यतया निरग्निरौरस एकोद्दिष्टं कुर्यादित्यपरोऽर्थः संपद्यते ।\\
दत्तकादीनां तु सानीनां चैकोद्दिष्टं सुता दशेति वचनादेकोद्दि\\
ष्टमेव, सामान्यप्राप्तविकल्पस्तु अपुत्रमातामहश्राद्धादिषु सावकाश\\
इति न किञ्चिदनुपपन्नमिति । तन्न \textbar{} "ब्रह्मा {[}ब्रह्ण{]} नयस्तु
ये विप्रा" इत्या.\\
दिना निरग्रेरपि पार्वणविधानेन निरग्निरौरसः }{पार्वणं}{ कुर्यादित्यपि\\
वक्तुं शक्यत्वात् । तस्मादविशेषाद्विकल्प एवं-\/- न्याय्यः, स चैच्छिक,\\
प्रमीतपितृकः }{श्राद्धं}{ पर्वकाले यथाविधि ।\\
मृताहनि यथारुच्या वृद्धावभ्युदयक्रिया \textbar{}\textbar{}\\
इति गार्ग्यवचनात् । अत्र सांवत्सरिकाद्रौ पितृपार्वणस्यैव धर्मा\\
तिदेशो न मातामहपार्वणस्य स्वस्यापि विकृतित्वात्, न हि भिक्षुको\\
भिक्षुकान्तरं याचते सत्यन्यस्मिन्नामेक्षुक्र इति न्यायात् । एवं च\\
पितृक्षयाहे पित्रादित्रयाणामेव श्राद्धं मातृक्षयाहे
मात्रादित्रयाणामेव\\
पितृश्राद्धस्यै कोद्दिष्टत्वपक्षे तु मातुरव्ये कोद्दिष्टमेव कार्यम् ।\\
प्रत्यब्दं यो यथा कुर्यात्पुत्रः पित्रे सदा द्विजः ।\\
तथैव मातुः कर्तव्यं }{पार्वणं}{ वान्यदेव च ॥\\
इति कात्यायनवचनात् । अपुत्राणां मातामहादीनां पार्वणमेको\\
द्दिष्टं }{वा-}{ कार्यम् ।\\
पितृव्यभ्रातृमातृणामपुत्राणां तथैव च ।\\
मातामहस्यापुत्रस्य श्राद्धादि पितृवद्भवेत् ॥\\
इतिजातूकर्ण्योक्तेः । मातामहपदं मातामह्या अप्युपलक्षणं दौहित्र.\\
त्वेनाधिकारवाक्येऽधिकार्युकेः ।\\
}{यत्तु}{ -\\
सपिण्डीकरणादूर्ध्वे पित्रोरेव तु }{पार्वणम्}{ ।\\
पितृव्यभ्रातृमातृणामेकोद्दिष्टं सदैव द्दि ॥\\
इति पितृव्यादीना मे कोद्दिष्टविधानं तत्कर्त्रपेक्षया कनिष्ठपितृ\\
व्यादिविषयम् ।\\
पितृव्यभ्रातृमातॄणां ज्येष्ठानां पार्वणं भवेत् ।

{ सांवत्सरिक श्राद्धनिरूपणम् । ३२९\\
एकोद्दिष्टं कनिष्ठानां दम्पत्योः }{पार्वणं}{ मिथः ॥\\
इति चतुर्विंश{[}{[}मनोक्तेः ।{]} मातुलादिश्राद्धं तु एकोद्दिष्टमेव ।\\
मातुःसहोदरो यश्च पितुः सहभवोऽथवा ।\\
तयोश्चैव न कुर्वीत }{पार्वणं}{ पिण्डनाडते \textbar{}\\
यश्च मन्त्रप्रदाता स्याद्यश्च विद्यां प्रयच्छति ।\\
गुरुणामपि कुर्वीत तयार्नैव तु }{पार्वणम्}{ ।\\
}{सपिण्डीकरणादूर्ध्वं}{ यत्र यत्र प्रदीयते \textbar{}\textbar{}\\
}{श्राद्धं}{ भगिन्यै पुत्राय स्वामिने मातुलाय च ।\\
मित्राय गुरवे श्राद्धमेकोद्दिष्टं न }{पार्वणम्}{ ॥\\
इति बृद्धगर्गसुमन्तुवसिष्ठवचनेभ्यः । दम्पत्योस्तु परस्परं पार्वणमेव,\\
पूर्वोदाहृतचतुर्विंशन्मतात् । " भर्तृपत्न्योश्चेव" मिति
स्मृत्यर्थसारोक्तेः\\
सपत्न्योरपि परस्परं }{पार्वणम्}{ । तत्र स्त्रीकर्तृकामावास्यादिपार्वणे\\
विशेषः ।\\
स्वभर्तुप्रभृतित्रिभ्यः स्वपितृभ्यस्तथैव च ।\\
विधवा कारयेच्छ्राद्धम् - इतिवचनात् भर्तृतत्पितृपितामहानां स्व\\
पित्रादीनां त्रयाणां श्राद्धमिति । अत्र च-\\
पितरो यत्र पूज्यन्ते तत्र मातामहा ध्रुवम् ।\\
अविशेषेण कर्त्तव्यं विशेषाशरकं व्रजेत् ॥\\
इति सामान्यवचनात् क्षयाहे यद्यपि मातामहप्रातिस्तथापि -\\
कर्षसमन्वितं मुक्त्वा तथाद्यं श्राद्धषोडशम् ।\\
प्रत्यादिकं च शेषेषु पिण्डाः स्युः षडितिस्थितिः ॥\\
इति कात्यायनेन षट्पिण्डश्राद्धे प्रत्याब्दिक पर्युदासाद्वाध्यते । "पि.\\
तुर्गतस्य देवत्वमौरस्य त्रिपूरुषम्" इति पारस्करेण त्रिपुरुषश्राद्धवि.\\
धानात् त्रिपुरुषत्वविधेः षट्पुरुषतानिवृत्यर्थत्वाच्च । देवत्व गतस्य=
स.\\
पिण्डीकरणेन पितृत्त्वं प्राप्तस्य । संग्रहकारस्तु क्षयाहे मातामहान्सा•\\
क्षान्निषेधति । तथा हि ।\\
याज्ञवल्क्येन कालस्तु अमावास्यादि नोदितः ।\\
अविशेषण पित्र्यस्य तथा मातामहस्य च ॥\\
युगपच्च सत्ति ज्ञेयो वाचनाद्वश्यमाणकात् ।\\
कालभेदेन तन्त्रं स्याद् देशभेदो न चैव हि ॥\\
तस्मात्तन्त्रविधानाच्च यौगपद्यं प्रतीयते ।\\
वी० मि ४२

{३३० वीरमित्रोदयस्य श्राद्धप्रकाशे-\\
अमावास्यादिकालेषु तद्ज्ञेयं न मृतेऽहनि ।\\
अमावास्यादिकालेषु कालकत्वात्सहक्रिया ।\\
मृताहनि तु तद्भेदान युज्येत सहक्रिया ॥ इति ।\\
अत्र न्यायप्रदर्शन यथा तथास्तु "न तद्ज्ञेयं मृतेऽहनी " इति तु स्पष्टो\\
मातामहनिषेधः । त्रिपुरुषत्वविधानादेव च " स्वेन भर्त्रा समं
}{श्राद्धं}{\\
सा- भुङ्गे" इति वचनप्रात्तसपत्नीकत्वस्यापि निवृत्तिः, त्रिपुरुगति,\\
रिक्तदेवतामात्रव्यावृत्यर्थत्वात्तस्य । क्रमेण मृतयोर्मातापित्रो
दैवात्क्ष.\\
याहैक्ये तन्त्रेण पाक कृत्वा मरणक्रमेण श्राद्धं कुर्यात्,
निमित्तक-\\
मस्य नैमित्तिकक्रमनियामकत्वाद । क्रमाज्ञाने तु पूर्वं पितुस्ततो मातु\\
पित्रोः श्राद्ध सम प्राप्ते नवे पर्युषिते तथा ।\\
पितुः पूर्वं सुतः कुर्यादन्यत्रासत्तियोगतः \textbar{}\textbar{}\\
इति कर्ष्णाजनिवचनात् । अन्यत्र=पितृमात्रतिरिक्तस्थले । आसत्तिः=\\
सम्बन्धः, सन्निकृष्टस्य पूर्वं, ततो विप्रकृष्टस्येत्यर्थः । तथा च-\\
ऋष्यशृङ्खः ।\\
भवेद्यदि सपिण्डानां युगपन्मरणं तदा ।\\
सम्बन्धासत्तिमालोच्य तत्क्रमाच्छ्राद्धमाचरेत् ॥ इति ।\\
यत्तु नैकः श्राद्धद्वयं कुर्यात्समानेऽहनि कुत्रचित् " इति प्रचेतोवचनं\\
तन्निमित्तदैवतैक्ये ज्ञेयम् । तथा च-\\
• जावालिः ।\\
}{श्राद्धं}{ कृत्वा तु तस्यैव पुनः }{श्राद्धं}{ न तहिने ।\\
नैमित्तिकं तु कर्त्तव्यं निमित्तानुक्रमोदितम् ॥ इति ।\\
अन्वारोहणे तु विशेषः तथा च-\\
लौगाक्षिः\\
मृताहनि समासेन पिण्डनिर्वपणं पृथक् ।\\
नवश्राद्धं च दम्पत्योरन्वारोहण एवं-\/- तु ॥ इति ।\\
अत्र पृथङ् नवश्राद्धमित्यन्वयः, चस्त्वर्थे । समासेन= संक्षेपेण ।\\
द्विपितृकश्राद्ध इवोभयोद्देशेनैकः पिण्डो देयः । नवश्राद्धे तु
भेदेनेत्य•\\
र्थः । }{यत्तु}{ -\/-\\
या समारोहणं कुर्याद्भर्त्तुश्चित्यां पतिव्रता ।\\
तां मृताइनि सम्प्राप्ते पृथग्पिण्डे नियोजयेत् ॥\\
प्रत्यन्दं च नवश्राद्धं युगपत्तु समापयेत् ॥\\
इति, तद्येषां दत्तकादीनामेकोद्दिष्टमुकं तद्विषयम् । अनेकमातृ.

{ श्राद्धभेदनिरूपणम् । ३३१\\
भिरेकचिश्यारोहणे तु प्रथमं पितुः ततः स्वजनन्याः ततः सपत्न-\\
मातुः प्रत्यासत्तेः ।\\
केचित्तु मृताहश्राद्धं नवश्राद्धं च पित्रोः समासेन समानतन्त्रेण\\
कार्यं पिण्डनिर्वपणं तु पृथक् । पूर्वोदाहृतवचनादिस्याहुः । इति
सांवत्स\\
रिकनिर्णयः ।\\
अथ श्राद्धभेदाः ।\\
विश्वामित्रः ।\\
नित्यं नैमित्तिकं काम्यं वृद्धिश्राद्धं सपिण्डनम् ।\\
}{पार्वणं}{ चेति विशेयं गोष्ठ्यां शुध्यर्थमष्टमम् \textbar{}\textbar{}\\
कर्माङ्गं नवमं प्रोकं दैविकं दशम स्मृतम् ।\\
यात्रास्वेकादशं प्रोक्तं पुष्ट्यर्थ द्वादशं स्मृतम् ॥\\
तत्र नित्यं नैमित्तिकं चाह -\\
कात्यायनः ।\\
अहम्यहनि यत्प्रोक्तं तन्नित्यमिति कीर्त्तितम् ।\\
एकोद्दिष्टं तु यच्छ्राद्धं तत्रैमित्तिकमुच्यते ॥\\
तद्ध्यदैवं कर्त्तव्यमयुग्मानाशयेत् द्विजान् \textbar{}\textbar{}\\
अत्र चैवं "नैमित्तिके कामकाली" इत्येतद्वचनादेकोद्दिष्टेऽपि\\
तयो. प्राप्तौ "अदैव" मितिवचनाश्च विकल्पः । स च शाखाभेदेन\\
व्यवस्थितः । न च क्वापि शाखायामेकोद्दिष्टे दैवविधानं नास्तीति\\
वाच्यम् । आश्वलायनेन नवश्राद्धेषु दैवविधानात् । काम्यमुकं -\\
वसिष्ठेन ।\\
अभिप्रेतार्थसिध्यर्थ काम्यं पार्वणवत्स्मृतम् ।\\
वृद्धिश्राद्धमुक्कं तेनैव-\\
पुत्रजन्मविवाहादौ वृद्धिश्राद्धं प्रकीर्तितम् \textbar{}\textbar{}\\
सपिण्डीकरणमपि तेनैव -\\
नवानीहाईपात्राणि पिण्डश्च परिकर्थिते ।\\
पितृपात्रेषु पिण्डेषु सपिण्डीकरणं तु तत् ॥\\
पार्वणलक्षणमपि तेनैव-\\
प्रतिपर्व भवेद्यस्मात्प्रोच्यते }{पार्वणं}{ तु तत् ।\\
पर्व = अमावास्या, संक्रान्त्याद्यपि\\
चतुर्द्दश्यष्टमी कृष्णा {[}त्व{]}मावास्याथ पूर्णिमा \textbar{}\\
पर्वाण्येतानि राजेन्द्र रविसंक्रमणं तथा ॥

{३२२ वीरमित्रोदयस्य श्राद्धमकाशे-\\
इति }{विष्णुपुराणा}{त् । अत एवं-\/- भविष्यत्पुराणे द्विविधा पार्वणप\\
दनिरुक्तिः -\\
अमावास्यां यत्क्रियते तत्पार्वणमुदाहृतम् ।\\
क्रियते पर्वणि च यत्तत्पार्वणमुदाहृतम् ॥ इति ।\\
युगादिश्राद्धेषु तत्पदप्रवृत्तिस्तद्धर्मकत्वात् ।\\
गोष्ठ्यां यत्क्रियते }{श्राद्धं}{ गोष्ठीश्राद्धं तदुच्यते ।\\
बहूनां विदुषां सम्पत्सुखार्थ पितृतृप्तये \textbar{}\textbar{}\\
बहूनां विदुषाङ्गोष्ठ्यां समुदाये तीर्थयात्रादौ युगपत्कर्त्तव्यतया\\
प्राप्तं श्राद्धं पृथक्पाकाद्यसम्भवेन प्रत्येकं कर्तुमसामर्थ्य
सम्भूय\\
सामग्रीसम्पादने यत्सुखं तदर्थं तल्लिप्सया यन्मिलितैः क्रियते\\
तद्वोष्टीश्राद्धमिति }{शङ्खधर}{ कल्पतरुप्रभृतयः-\\
केचित्तु श्राद्धस्य गोष्ठयां वार्तायां क्रियमाणायां तज्जनितोत्सा\\
हेन यत्क्रियते }{श्राद्धं}{ }{तद्गोष्टीश्राद्ध}{मित्याहुः ।\\
शुध्यर्थमिति तत्प्रोक्तं श्राद्ध पार्वणवत्कृतम् ।\\
यथा प्रायश्चित्ताङ्गविष्णुश्राद्धादि । कर्माङ्गादीनां लक्षणमुक्कं
}{भ}{.\\
}{विष्यत्पुराणे ।}{\\
निषेककाले सोमे च सीमन्तोन्नयने तथा ।\\
ज्ञेयं पुंसवने चैव श्राद्ध कर्माङ्गमेव च ॥\\
एतच्च निषेकादिग्रहणं सकलश्रौतस्मार्त्तकर्मोपलक्षणार्थम् ।\\
"नानिष्ट्वातु पितॄन्श्राद्धे कर्म वैदिकमाचरेत्" }{इतिशातातप}{चचनात् ।\\
"इष्टापूर्त्तादिकं श्राद्धं वृद्धिश्राद्धवदाचरेत्" इति
}{भविष्यत्पुराणाच्च}{ ।\\
देवानुद्दिश्य यच्छ्राद्धं तद्दैविकमिहोच्यते ।\\
हविष्येण विशिष्टेन सप्तम्यादिषु यत्नतः ॥\\
}{वसिष्ठोऽपि ।}{\\
देवानुद्दिश्य क्रियते तद्दैविकमिहोच्यते ।\\
तन्नित्यश्राद्धवत्कुर्याद्द्वादश्यादिषु यत्नतः \textbar{}\textbar{}\\
गच्छेद्देशान्तरं }{यत्तु}{ }{श्राद्धं}{ कुर्यात्त सर्पिषा ।\\
यात्रार्थमिति तत्प्रोकं प्रवेशे च न संशयः ॥\\
}{ गच्छन्}{= तीर्थयात्रार्थ देशान्तरं गच्छन् । देशान्तरग्रहणात्समान\\
देशे यात्राश्राद्धं न भवति । प्रवेशे = पर्यागत्य पुनः स्वगृहप्रवेशे
।\\
तथाच

{ श्राद्धभेदनिरूपणम् । ३३३\\
ब्रह्मपुराणे ।\\
यो यः कश्चित्तीर्थयात्रां तु गच्छेत्\\
सुखयतः सुमनाः सुसमाहितः ।\\
स पूर्वे स्वगृहे कृतोपवासः\\
सम्पूजयेद्विधिवद्भक्तिनम्रो गणेशम् \textbar{}\textbar{}\\
देवान्पितॄन्ब्राह्मणांश्चैव साधून\\
धीमान्प्रीणन्वित {[}नेति{]}शक्त्या प्रयत्नात् ।\\
प्रत्यागतश्चाथ पुनस्तथैव\\
देवान्पितॄन्ब्राह्मणान्पूजयेच्च ॥ इति ।\\
शरीरोपचये श्राद्धमन्त्रोपचय एवं-\/- च ।\\
पुष्ट्यर्थमिति तत्प्रोक्तमौपचायिकमुच्यते ॥\\
शरीरोपचये= तन्निमित्ते रसायनादावित्यर्थः ।\\
}{यत्तु}{ कूर्मपुराणे -\\
अहन्यहनि नित्यं स्यात्काम्यं नैमित्तिकं पुनः ।\\
एकोद्दिष्टं च विशेयं वृद्धिश्राद्धं च }{पार्वणम्}{ ॥\\
एतत्पञ्चविधं }{श्राद्धं}{ मनुना परिकीर्त्तितम् ।\\
इतिकर्मपुराणे पञ्चविधत्वप्रतिपादनं तद्गोष्ठ्यादिश्राद्धानां पार्वणै.\\
कोद्दिष्टान्तर्भावाश्रयणेन ।\\
मत्स्यपुराणे तु,\\
नित्यं नैमित्तिकं काम्यं त्रिविधं श्राद्धमुच्यते ।\\
इतित्रैविध्यमुक्तम् । तत्तत्र तत्रान्तर्भावेनैवेति बोध्यम् ।\\
तत्र नित्यान्याह विष्णुः ।\\
अमावास्या तिस्रोऽष्टकास्तिस्रोऽन्वष्टका मघा प्रौष्टपचूर्ध्वं कृष्ण-\\
त्रयोदशी व्रीहियवपाकौ चेति ।\\
एतांस्तु श्राद्धकालान्वै नित्यानाह प्रजापतिः ।\\
श्राद्धमेतेष्व कुर्वाणो नरकं प्रतिपद्यते ॥ इति ।\\
नैमित्तिकान्याह\\
गालवः । 

{ प्रेतश्राद्धं सपिण्डान्तं संक्रान्तौ महणेषु च ।\\
संवत्सरोदकुम्भं च वृद्धिश्राद्धं निमित्ततः ॥\\
काम्यानपि स एवाह-\/-\\
तिथ्यादिषु च यः }{श्राद्धं}{ मन्वादिषु युगादिषु ।\\
अलभ्येषु च योगेषु तत्काम्यं समुदाहृतम् \textbar{}\textbar{}}

{३३४ वीरमित्रोदयस्य श्राद्धप्रकाशे-\\
युगादीनां नित्यतापि समयप्रकाशे चश्यते । तिथ्यादीत्यादिपदेन\\
नक्षत्रयोगादिग्रहणम् । इदं च त्रिविधमपि पार्वणैकोद्दिष्टभेदेन द्वि\\
विधम् । तदयमर्थः यत्र द्वादशविधमुक्त्वा नित्यादिपुरस्कारकारेण\\
विधिस्तत्र तदन्तर्गतमेव नित्यादि ग्राह्यम् । एवं यत्र पञ्चविधमुक्त्वा\\
नित्यादिपुरस्कारेण विधिस्तत्र तथा, यत्र विभागमनुक्त्वा त्रैविध्यं\\
खोक्त्वा धर्मविधिस्तत्र कात्यायनविष्ण्वाद्युक्तमिति विवेकः । इति
श्राद्धभेदाः ।\\
अथ श्राद्धविकृतिषूहो विचार्यते ।\\
तत्र-\\
अक्रोधनै: शौचपरैः सततं ब्रह्मचारिभिः ।\\
भवितव्यं भवद्भिश्च मया च श्राद्धकर्मणि ॥\\
सर्वायास विनिर्मुकैः कामक्रोधविवर्जितैः ।\\
भवितव्यं भवद्भिर्नः श्वोभूते श्राद्धकर्मणि ॥ इति ।\\
एतौ निमन्त्रणानन्तरं श्राद्धाङ्गभूतनियमश्रावणार्थी मन्त्रौ, प्रकृता-\\
वेकैकस्मिन्निमन्त्रित ब्राह्मणे प्रयुज्यमाना वेकोद्दिष्टे नोह्यौ,
प्रकृतौ बहु-\\
वचनस्य समवेतार्थत्वात् । प्रकृतौ सद्यो निमन्त्रणपक्षे व्रीहिमन्त्रस्य\\
यवेष्विव द्वितीयमन्त्रस्य लोपः ।\\
केचित्तु श्वोभूत इति पदद्वयस्यैव लोपो समर्थत्वाच्छेषस्य तु\\
तेन विनापि वाक्यार्थपर्यवसानात्प्रयोगो भवत्येष । अत एवाग्निर्मा\\
तस्मादेनसो गार्हपत्यः प्रमुञ्चत्वित्यत्रानाहिताग्निकर्तृकेऽपि
पितृयज्ञे\\
गार्हपत्यपदस्यैव लोप इत्याहुः ।\\
विकृतौ तु अद्येत्यूहः । प्रकृतौ सद्यः पक्षे मन्त्राभावस्थार्थिक.\\
त्वात् । नचार्थिकं चोदकः प्रतिदिशतीति न्यायात् । अत एवं-\/- विकृ\\
तौ प्रकृतौ यक्षप्रयोगे लुप्तस्यापि श्रीहिमन्त्रस्योह इत्युक्तमाकरे ।
दर्भ-\\
बटुपक्षेऽर्थाभावान्मन्त्रलोपे प्राप्ते -\\
निधाय }{वा}{ दर्भबटूनासनेषु समाहितः ।\\
प्रेषानुप्रैषसंयुक्तं विधान प्रतिपादयेत् ॥\\
इति हारीतवचनान्मन्त्रप्रयोगः कर्त्तव्य एव । आसनोपवेशने -\\
यमः ।\\
आसध्वमिति तान् ब्रूयादासनं संस्पृशन्नपि ॥ इति ।\\
अत्रासध्वमिति प्रैषपाठस्यासन संस्पर्शसमानकालीनत्वादास-\\
नानां च भिन्नत्वात्प्रतिब्राह्मणमावृत्तेर्बहुवचनस्यासमवेतार्थत्वादेको-

{ विकृतिश्राद्धेषूहविचारः । ३३५\\
दूद्दिष्टादावनूहः \textbar{} आवाहन उशन्तस्त्वेति मन्त्रे पितॄन् हविषे
अत्तव इत्य.\\
श्र पितृपदस्य प्राप्तपितृलोकपरतया प्रकृताविव विकृतावपि समवेता.\\
र्थत्वान्मातृमातामहादिश्राद्धेऽनूहः । एकोद्दिष्टे तु
देवतासन्निधानार्थं\\
सत्यप्यावाहने वचनेन मन्त्रलोपान्नोहप्रसङ्ग इति मदनपारिजाते । आ.\\
वाहनस्यैव लोपा{[}त्{]} नूह इत्यपरे । अत्तव इत्यस्य ब्राह्मणकर्तृकादन '\\
परतया समवेतार्थत्वादामहिरण्यश्राद्धे स्वकिर्त्तवा इत्यूहो यद्यपि\\
न्यायात्प्राप्नोति तथापि तस्माहच नोहे {[}दि{]}ति निषेधादनूह इति के.\\
चित् । तदयुक्तम् । ``आवाहने स्वधाकार मन्त्रा ज{[}ऊ {]} ह्या विसर्जने।\\
इति निरवकाशपौराणवचनेन मन्त्रान्तरे सावकाशोहानषेधवचनव!\\
धस्यैव न्याय्यत्वात् । अन्यथा "बाजे बाज" इति विसर्जनमम्त्रेऽपि\\
स न स्यात् तस्मादूह एवं-\/- म्याथ्यः । "आयन्तु नः पितरः सोम्या.\\
सोऽग्निष्वात्ता इत्यत्र नोहः, जपमात्रस्यादृष्टार्थत्वात् । यज्ञपार्टी
वर्धा.\\
मितिवदग्निष्वात्तादिपरतया परविशेषणत्वाच्च । ``तिलांऽसीति"\\
मन्त्रस्य प्रतिपात्रे तिलावापकरणत्वादेकस्मिंश्च पात्रे पितृबहुत्वान\\
म्वयेन पितृनिति बहुवचनस्यासमवेतार्थस्वात्पितृशब्दस्य च संस्का\\
रवचनत्वादेकोद्दिष्टे नोहः । एकपवित्र एकोद्दिष्टे "पवित्रे स्थो वै\\
ष्णव्यौ विष्णोर्मनसौ पूते स्थ इति पवित्रच्छेदोन्मार्जनमन्त्रौ[नो]
ह्यो,\\
दर्शपूर्णमासा पूर्वप्रयुक्तयोर्मन्त्रयोरुपदेशातिदेशाभावेन
प्रकृतिश्राद्धा\\
र्थत्वाभावेन विकृतावप्राप्तेः । गौस्तु सामान्यतो दर्शनेन प्रकृत्यर्थ\\
स्वमप्युत्प्रेक्ष्यत इति तदुपेक्षणीयमिति मैथिलाः । तन्न ।
प्रागुदाहृतप्रचे.\\
तोवचनेन श्राद्धेऽपि विहितत्वात्, तस्मादूहः । प्रथमे पात्रे संस्रवात्स\\
मानीय पितृभ्यः स्थानमसीति न्युब्जं पात्रं निद्धातीति । अत्रैको\\
दूद्दिष्टे पात्रान्तरीयसंस्रवाभावान्म्युजीकरणं नास्तीति केचित् ।\\
एकस्यापि पात्रस्य न्युब्जीकरणं कार्यं,
पितृस्थानार्थत्वाददृष्टार्थस्वा-\\
बेत्यपरे । अतश्च तन्मते पितृभ्य इत्यत्रै कोद्दिष्टे पित्र इत्यूहः ।
गन्धा.\\
दिदाने -\\
शाठ्यायनः ।\\
एष ते गन्धः, एतत्ते पुष्पम्, एष ते धूपः, एष ते दीपः एतत्त\\
आच्छादनमिति ।\\
अत्र पुष्पादीनां बहुरवेऽपि एकवचननिर्देशो जात्याख्यायाम् ।\\
इदं चः पुष्पमित्युक्त्वा पुष्पाणि च निवेदयेत् ।

{३३६ वीरमित्रोदयस्य श्राद्धप्रकाशे-\\
इतिब्रह्मपुराणवचनात् । चकारो गन्धाद्युपसङ्ग्रहार्थः ।\\
}{यत्तु}{ अत्र श्राद्धतत्वे अत एवं-\/- नवमाध्यायेऽपि सूर्यस्य
चक्षुर्गम.\\
यतादिति मन्त्रः सूर्यस्य चक्षुर्बहुत्वेऽपि एकवचनान्तपदनिर्देशात्सं-\\
सर्गिद्रव्याणामनतोयतेजसां प्रयोग एकवचनान्तः प्रयोग इति वि.\\
चारितमितीत्युक्तम्, तदविचारितमेव रमणीयं गौड़प्रतारणमात्रमि\\
त्युपेक्षणीयम् ।\\
}{यत्तु}{ कामरूपीये एष वो गन्धः, दई वः पुष्पमित्यादि ब्राह्मे च\\
हुवचनान्तात्सर्गवाक्यानुरोधाच्छक्यत्वाच्च सर्वेभ्यस्तन्त्रेण गन्धा\\
दिदानमङ्गीकृत्यैष ते गन्धः, एतत्ते पुष्पमिति शाट्यायनोकैकवचना.\\
न्तमन्त्राणामेकोद्दिष्ट उत्कर्ष इत्युक्तम् । तदसत् । प्रातिपदिकस्य स-\\
मवेतार्थत्वेन प्रकरणवाधायोगात् पाशाधिकरणन्यायेन बिकल्पस्यै.\\
वौचित्यात्, न तु सर्वथास्कर्ष इति ।\\
अग्नौकरणे पाणिहोमे पाणी करिष्य इत्यूहः । ``अग्भ्यभावे तु\\
विप्रस्य पाणावेव जलेऽपि चा" इति मात्स्येऽग्न्यभावे पाणिविधा.\\
नातू, पूर्वपक्षे सोमाभाव इव प्रतीकानाम्, अग्निस्थानापन्नत्वेन पाणे-\\
विकृतित्वात् ।\\
}{यत्तु}{ श्राद्धतत्वे हस्तजलयोरपि अग्नौ करिष्ये इत्येव वक्तव्यं पा.\\
णेः प्रतिनिधित्वादित्युक्तम् । तदयुक्तम् । सादृश्याभावेन प्रतिनिधि-\\
त्वाभावात् ।\\
अग्यभावे तु विप्रस्य पाणी वाथ जलेऽपि }{वा-}{ ।\\
अजाकर्णेऽश्वकर्णे }{वा-}{ गोष्ठे वाथ शिवान्तिके ॥\\
इति मात्स्यप्रत्यक्षवचनविरोधाच' अश्वकर्णादीनां प्रतिनिधित्वा.\\
योगाच्च । बटुपक्षेऽपि जल इत्याद्यूहः । पृथिवी ते पात्रमिति जपे च.\\
टुपक्षे ब्राह्मणस्य मुखे अमृते इत्येतत् पदत्रयस्य लोपः, ब्राह्मणमुख•\\
कार्यकारिणोऽन्यस्य बढावसम्भवात् अनूहः पदान्तरस्य, अमृतं\\
जुहोमस्येतदविकृतमेव, निरधिकरणस्यापि होमस्य सम्भवात् ।\\
ब्राह्मणस्य मुख इत्यस्य बटावित्यूहो वा- ।
ब्राह्मणस्याहवनीयार्थत्वा.\\
वत्स्थानापन्नत्वाइटोः । अत्र वैष्णव्यर्चा यजुषा वेति
प्रसङ्गादृक्यजु\\
बोर्लक्षणमभिधाय गीयमानेषु सामसंज्ञेति जैमिनिना समर्थितत्वात् ।\\
यच्चान्यदुक्तम्-\\
निरङ्गुष्टं च यच्छ्राद्धं बहिर्जानु च यत् कृतम् ।

{ विकृतियाद्वेषूहविचार: । ३३७*\\
बहिर्जानु च यद् भुकं सर्वं भवति चासुरम् ॥\\
इति यमवचने निरङ्गुष्ठश्राद्धे निन्दाश्रुतेर्ब्राह्मणाभावे तस्प्रतिनि.\\
धित्वेन श्राद्धकर्त्रङ्गुष्ठं निवेशयेदिति । तदप्यसत् । "एकैकस्याथ वि.\\
प्रस्य गृहीत्वाङ्गुष्ठमादरात्" इति ब्रह्मपुराणनिन्दाया
ब्राह्मणाङ्गुष्ठा\\
भावविषयत्वात् । अथ ब्राह्मणीयत्वस्याङ्गुष्ठाङ्गत्वादङ्गलोपेऽपि\\
प्रधानलोपस्यान्याय्यत्वात् सन्निधानाद्यजमानाङ्गुष्ठग्रहणं तर्हि\\
बहुपक्षे यावत् किञ्चित् ब्राह्मणाङ्गके व्यापारे जान्वालम्भार्धग्रहण-\\
भोजनादावपि यजमान एवं-\/- सन्निधानाद्यजमानः किं न प्राप्नुयात् ।\\
तस्मादङ्गुलनिवेशनलोपः । नचैवमर्धस्थापि लोपः स्यादिति वाच्यम् ।\\
द्रव्यत्यागस्य ब्राह्मणं विनाप्युपपत्तेः, हस्तप्रक्षेपप्रतिपत्तेस्तु
दृष्टार्थ.\\
त्वाद्यत्र क्वापि कर्तुमुचितत्वात् । अन्तर्जानुकरणमपि यत् किञ्चित्\\
कर्तृकं कर्तुरसामध्ये सर्वकार्येषु प्राप्नुयान्निन्दाश्रवणादित्यास्तां
ता.\\
वत् । अनत्सु जपस्तु बटुपक्षेऽप्यदृष्टार्थत्वाद् रक्षोघ्नत्वाच्च भव\\
त्येव । मधुव्वातेति मधुमतीजपे मधुद्यौरस्तु नः पितेति पितृशब्दो\\
धुविशेषणत्वेना देवतापरत्वादनुद्यः । तृप्तास्थेति प्रश्ने
पङ्किमूर्धन्यप्र.\\
इनपक्षे बहुवचनस्य पूजार्थत्वादेकस्मिन्नप्यविरोधः । सर्वप्रश्नपक्षे\\
एकवचनान्तेनैकस्य द्विवचनान्तेन द्वयोरूहः । तृप्तोऽस्मि, तृप्तौ स्व\\
इति च प्रतिवचनोद्दः । अवनेजनपिण्डदानयोरसौशब्दप्रयोगान्मा.\\
त्रादिश्राद्धे मातरित्यादि प्रयोक्तव्यम् । अत्र पितरो मादयध्वम् ।\\
मीमदन्त पितरः । नमो वः पितरः । गृहान्नः पितर इत्यत्र पितृशब्द.\\
स्य सपिण्डनसंस्कारवचनत्वात्, तन्त्रेण प्रयोगाद्वहुवचनस्य सम\\
वेतार्थत्वान्मात्रेकोद्दिष्टे मातर्यपि पितरिस्यूहः । इत्थं च प्रयोगः ।
अत्र\\
पितर्मादयस्व यथाभागमावृषायस्वेति श्रीदत्तादयः । आवृषायथा इति\\
अनिरुद्धगुणविष्णू \textbar{} अमीमदन् पितः, यथा भागमावृषायिष्ठाः । नमः\\
स्ते पितः, रसाय \textbar{} अघोरः पितास्त्वित्यादि पुल्लिङ्गन्तु नोहां
पुल्लि\\
ङ्गस्यैव पितृशब्दस्य संस्कारवचनत्वात् । एतद्वः पितरो वास\\
इत्यत्र मैथिलाः पितृपदस्य जनकपुरुषशब्दतया मातृश्राद्धे मात्रादि\\
पदेनोह इत्याहुः \textbar{} तन्न । अत्र पितर इत्येतन्मत्रस्य पितृशब्द\\
समानयोगक्षेमत्वादस्य पितृशब्दस्य । श्राद्धविवेकादयोऽप्येवम् ।\\
स्वधावाचने तु पितामहादिपदसमभिव्याहारात् पितृशब्दस्य\\
जनकपुरुषवचनत्वात् । बहुवचनस्यादृष्टार्थत्वात् मातृश्राद्धे मातृ\\
पदोहः कार्यः ।\\
वी० मि० ४३\\
~\\
~\\
•

{३३८ वीरमित्रोदयस्य श्राद्धप्रकाशे-\\
अत्र शूलपाणिः ।\\
अत एव-\/- दैक्षे पशौ एकवचनान्तः पाशमन्त्रो बहुवचनान्तश्च\\
श्रुतः, तत्रैकवचनान्तस्य प्रकृतावर्थवत्वात् द्विपाशिकायां विकृतौ\\
द्विवचनोहः कार्यः बहुवचनान्तस्य तु प्रकृता वनर्थत्वाद्विकृतावपि\\
नोहः, किन्तु बहुवचनान्तस्यैव प्रयोग इत्युक्तं मीमांसायामिति\\
स्वस्य मीमांसाभिज्ञत्वमाविष्कृतवान् । कामरूपीयोऽप्येवं तद्व-\\
योरपि मन्त्रयोर्विकृतावूह इति सिद्धान्तापरिशान विजृम्भितमित्युपे.\\
क्षणीयं मीमांसकैः । आमश्राद्धे केषुचिन्मत्रेषु मरीचिनोहः उक्तः ।\\
तथा च-\\
मरीचिः ।\\
आवाहने स्वधाकारे मन्त्रा ऊह्या विसर्जने ।\\
अन्यकर्मण्यनूह्याः स्युरामश्राद्धे विधिः स्मृतः ॥\\
आवाहने = आवाहनमन्त्रे उशन्तस्त्वा निधीमहीति मन्त्रे पितॄन हविष\\
अत्तव इत्यत्र स्वीकर्तव इत्यूहः । विसर्जने तम्मन्त्रे वाजे बाजे वतेति\\
मन्त्रे तृप्ता यातेत्यत्र तर्थतेति । स्वधाकार इति । स्वघा =
पियहविर्दानं\\
तत् करणं स्वधाकारस्तदङ्गभूतमन्त्रे इदमन्त्र लोपस्करमिति अन्नोत्स\\
र्गमन्त्र इत्यर्थः । अत्रेदमन्त्रमित्यन्नपदस्थाने इदं
धान्यमित्यादिरूपे\\
णोहः कार्यः ।\\
हेमाद्रिस्तु स्वधाकारो = नमो वः पितर इष इति मन्त्रः, तत्रेष इति पद-\\
स्थाने आमायेत्यूह इत्याह । तदयुक्तम् । रसशुस्मादिपदवत् इष इति\\
पद्स्य फलीभूतान्नप्रतिपादकत्वाभावात् । यद्यपि तस्माडचं नोहेदि.\\
ति वचनात् ऋच्यूहो निषिद्धस्तथापि वचनादुक्तास्वृक्षु भवत्येवो.\\
हः । निषेधस्यान्यास्वृक्षु सावकाशत्वात् । द्विपितृकश्राद्धे तु एतत्ते\\
इसौ ये च त्वामत्रान्वित्यादेरेकवचनान्तस्य प्रकृतावूहाभाषाद् द्वयो\\
'श्रचैकवचनस्यासाधुत्वाल्लोप इति केचित् । तन्न । एकस्मिन्नपि द्वौ\\
द्वावुपलक्षयेदिति वचनाद् द्विवचनेनोहः कर्तव्य इति धूर्तमातृदत्तादयः ।\\
वस्तुतस्तु नानेन वचनेनोहो विधीयते किन्तु एतत्ते ऽसौ ये च\\
त्वामत्रान्विति पिण्डान् दद्यादिति वचनेनासौशब्देन सम्बुध्यन्तैक.\\
'वचनान्तदेवतानामविधानात् द्विपितृके चैकैकस्मिन् इति वचनेन\\
तादृशद्वयविधानादेतत्ते कृष्णशर्मन नारायणशर्मंत्रिति प्रयोगे सम्भ\\
बति नोहो लोपो वाऽऽअयितुं युक्तः । पुत्रिकापुत्रस्तु पुत्रिकैव पुत्रः

{ श्राद्धाधिकारिनिरूपणम् । ३३९\\
पुत्रीकृताया }{वा}{ पुत्र इति पक्षयोर्मातरं पितृशब्देन तत्पितरं पिताम-\\
हशब्देन तत्पितरं प्रपितामहशब्देनोद्दिशेत् " पौत्री मातामहस्तेन "\\
इति वचनेन पुत्रिकापितुः पौत्रित्वावधानात् । तथाचेत्थं प्रयोगः, एतत्ते\\
सत रुक्मिणीदेवि ये च त्वामत्रान्विति । एवं प्रेतश्राद्धेषु पितृपद-\\
स्थाने प्रेतशब्दोह इति यथान्यायं द्रष्टव्यं विस्तरमयादुपरस्यते ।\\
इत्यूहविचारः ।\\
अथ श्राद्धाधिकारिणो निरूप्यन्ते ।\\
तत्र मरीचि ।\\
मृते पितरि पुत्रेण क्रिया कार्या विधानतः ।\\
बहब: स्युर्यदा पुत्राः पितुरेकत्र वासिनः ॥\\
सर्वेषां तु मतं कृत्वा ज्येष्ठेनैव तु यत्कृतम् ।\\
द्रव्येण चाविभक्तेन सर्वैरेव कृतं भवेत् ॥\\
पुत्रेणेत्यविशेषात् सर्वेषां पुत्राणामधिकारः ।\\
पुत्रेषु विद्यमानेषु नान्यो वै कारयेव स्वधाम् ।\\
इति ऋष्यशृङ्गवचनाद् द्वादशविधपुत्रसद्भावे नान्यस्याधिकारः ।\\
तेषां चाधिकारक्रमो वक्ष्यते । अत्र पुत्रेणेत्येकवचनमविवक्षितम्,\\
"प्रमीतस्य पितुः पुत्रैः श्राद्धं देयं प्रयक्षत" इति बृहस्पतिवचनातू
।\\
एवं सति सर्वेषां पृथक् श्राद्धानुष्ठान प्राप्तावाई-बहन इति । एकत्र
वासिनः =\\
अविभक्तधनाः । एकत्रवासिन इत्यनेनैव सिद्धेर्द्रव्येण चाविभकेनेति\\
[ यत् ] पृथगुक्तं तद्विभक्तधनेनापि साधारणीकृतेन
द्रव्येणापृथक्क्षा.\\
द्धानुष्ठानप्रदेशनार्थ, अत एवं-\/- -\\
लघुहारीतः ।\\
खपिण्डीकरणान्तानि यानि श्राद्धानि षोडश ।\\
पृथक्पृथक् सुताः कुर्युः पृथग्द्रव्या अपि क्वचित् ।\\
शङ्खलिखितावपि ।\\
यद्येकजाता बहवः पृथक्क्षेत्रा पृथक्धनाः।\\
एकार्पण्डाः पृथक्शौचाः पिण्डस्त्वावर्तते त्रिषु ॥\\
पृथकक्षेत्राः = विजातीयमातृजाता: \textbar{} पृथक् {[}धनाः = विभकाः
\textbar{} एक {]}\\
पिण्डाः सपिण्डीकरणान्तेषु षोडशश्राद्धेषु एकमेव पिण्डं दद्युः, न\\
पुनः प्रतिपुत्रं पृथकूपिण्डदानम् । पृथक्कशौचाः = विजातीयमातृ सम्ब.\\
न्धात् पृथक् शौच भागिनः । पिण्डस्तिबति=सपिण्डता त्रिषु पुरुषेषु भव.\\
तीत्यर्थः ।

{३४० वीरमित्रोदयस्य श्राद्धप्रकाशे-\\
सर्वेषामिति = अयमन्त्र तात्पर्यार्थः ।\\
एकादशाद्याः क्रमशो ज्येष्ठस्य विधिवत् क्रियाः ।\\
कुर्युनैकैकशः श्राद्धमाब्दिकं तु पृथक् पृथक् ॥\\
इति प्रचेतोवचनैकवाक्यतया ज्येष्ठेनैव एकादशाहादिसपिण्डी-\\
करणान्तानि श्राद्धानि कार्याणि । कनिष्ठेन तु तत्र अनुमतिद्रव्यार्पणे\\
एवं-\/- विधेये, एवं च प्रयोगानुष्ठातृत्वरूपं साक्षात्कर्तृत्वं
ज्येष्ठस्यैव,\\
अनुमतत्वरूपमनुग्राहकत्वरूपं वा- कर्तृत्वं कनिष्ठस्य
इत्यभिप्रायेणोक्तं\\
सर्वैरेव कृतं भवेदिति ।\\
एतेन यत्र वैदेश्यादिवशात्कनिष्ठस्यानुमतत्वद्रव्य संश्लेषयोरभावः,\\
तत्र ( १ ) तेन पृथगेव तच्छ्राद्धमनुष्ठेयम् । अन्यथा तस्य प्रत्यवायप.\\
रिहारो न स्यादिति यच्छूलपाणिनोक्तं तश्चिन्त्यम् \textbar{}
तस्यानुमतिद्र-\\
व्यार्पणयोरेवाधिकारस्य वाचनिकत्वान्न तु साक्षादनुष्ठाने, एवं च\\
देशादागत्य तेन ज्येष्ठाय द्रव्यार्पणमात्रं विधेयम्, शक्यत्वात्, ताव.\\
तैव स न प्रत्यवैति । अत एवं-\/- "ज्येष्ठेनैव " इत्यत्र "पृथक् नैव
सुताः\\
कुर्युः" इत्यत्र च एवकारौ कनिष्ठस्य साक्षात् कर्तृत्वनिषेधार्थी, एवं\\
च ज्येष्ठासान्निध्ये कनिष्ठस्य अग्निदातृत्वे तेन दशाहकृत्यमेव कर्त\\
व्यम्, ``यस्त्वग्निदाता प्रेतस्य स दशाहं समापयेत्" इति वचनात् ।\\
न वैकादशाहादिकमपि, ज्येष्ठष्यैव साक्षात् कर्तृत्वबोधनात् । यदि\\
च शास्त्रार्थ भ्रान्त्यादिना कनिष्ठेनैकादशाहादिकमनुष्ठितं तथापि\\
ज्येष्ठेन पुनः कर्तव्यमेव तस्यैव साक्षादनुष्ठानविधानात् । न च पृथ.\\
करणनिषेधात् तेन कथमनुष्ठेयमिति वाच्यम् । कनिष्ठकृतश्राद्ध.\\
स्यानधिकारिकृतत्वेनाकृतकल्पतया तत्र पृथक्करणा - {[}णनिषेधा{]}\\
भावादिति बहवः \textbar{}\\
वस्तुतस्तु "बहवः स्युर्यदा पुत्राः" इत्यादि मरीचिवाक्य\\
एकत्र वासिन इत्यस्य एकदेशावस्थिता इत्यर्थः तथा च बहूनां पु.\\
त्राणा सान्निध्ये ज्येष्ठेन प्रयोगोऽनुष्ठेयः साधारणधनेन कनिष्ठे\\
स्तु अनुमतिद्रव्यार्पणमात्रं विधेयम् । ज्येष्ठासान्निध्ये तु कनिष्ठेन\\
षोडशश्राद्धानि यथाकालं यथा क्रमं संवत्सरपर्यन्तं कर्तव्यानि,\\
पूर्ण संवत्सरे ज्येष्ठासन्निधाने सपिण्डीकरणमपि तदा कर्तव्य\\
मेघ "प्रमीतस्य पितुः पुत्रैः श्राद्धे देयं प्रयत्नतः" इति
बृहस्पतिवाक्ये

% \begin{center}\rule{0.5\linewidth}{0.5pt}\end{center}

( १ ) कनिष्ठेनेत्यर्थः ।

{ श्राद्धाधिकारिनिरूपणम् । ३४१\\
सर्वेषामविशेषेण श्राद्धकर्तृकत्वबोधनात् । बहुपुत्रसान्निध्य एवं-\/-\\
मरीचिना विशेषाभिधानात् ।\\
अत एव-\/- -\\
नवश्राद्धं सपिण्डत्वं श्राद्धान्यपि च षोडश ।\\
एकेनैव तु कार्याणि सविभक्तधनेष्वपि ॥\\
इति दक्षवचने सर्वेषामधिकाराभिप्रायेणाविशेषादेकेनैवेत्युक्तं, न\\
तु ज्येष्ठेनैवेति । अत्र सपिण्ड [च] मिश्रणम्, अतो न
षोडशान्तर्गतसपि•\\
ण्डीकरणश्राद्धेन पौनरुक्त्यम् । एतेन "श्राद्धानि षोडशापाद्य विद.\\
धीत सपिण्डताम्" इत्यपि व्याख्यातम् । अत एवं-\/- रामालान्निध्यव.\\
शाद्भरतेनैतच्छ्राद्धं कृतम् । तदुक्तम्-\\
अयोध्याकाण्डे ।\\
समतीते दशाहे तु कृतशौचो विधानतः ।\\
चक्रे द्वादशिकं श्राद्धं त्रयोदशिकमेव च ॥\\
दशाह इति अशौचकालोपलक्षणम् । द्वादशिक = द्वादशाहेन निर्वृ.\\
प्तम् । यदि पुनरन्तरा ज्येष्ठसान्निध्य तदा तेनैव कनिष्ठकृतावाशे\\
ष्टानि श्राद्धानि कर्तव्यानि । बहुपुत्रसान्निध्ये मरीचिना ज्येष्ठस्यैव\\
कर्तृत्वनियमनात् । अत एव-\/- -\\
प्रेतसंस्कारकर्माणि यानि श्राद्धानि षोडश ।\\
यथाकालं तु कुर्वीत नान्यथा मुच्यते तु सः ॥\\
इति लघुहारीतवाक्ये ज्येष्ठप्रतीक्षया कालातिक्रमो न कर्तव्य इत्य.\\
भिप्रायेण यथाकाल तु कुर्वीतेत्युक्तम् । अन्यथा
विधानसामर्थ्यात्तत्प्राप्ते-\\
र्यथाकालमिति निरर्थकं स्थात् । यदि कनिष्ठोऽग्निमान् ज्येष्ठो निर\\
ग्निस्तदा-\\
एकादशाहं निर्वर्त्य अर्वाक् दर्शाद्यथाविधि ।\\
प्रकुर्वीताग्निमान् पुत्रो मातापित्रोः सपिण्डताम् ॥\\
इतिछन्दोगपरिशिष्टवचनात् कनिष्ठेन दर्शादर्वाक् कते सपिण्डीक\\
रणे ज्येष्ठेन पुनर्नावर्तनीयम् । एवं कनिष्ठस्य ज्येष्ठासन्निधाने
वृद्ध्या-\\
पत्तौ कनिष्ठेन कृतं सपिण्डीकरणं ज्येष्ठेन पुनर्नावर्तनीयम् \textbar{}
"सपि.\\
ण्डीकरणान्तानी"त्यादिलघुहारीतेन पृथक्करणनिषेधात् ।\\
निर्वर्त्तयति यो मोहात् क्रियामन्यनिवर्तिताम् ।\\
विधिघ्नस्तेन भवति पितृहा चोपजायते ॥

{३४२ वीरमित्रोदयस्य श्रद्धमकाशे-\\
तस्मात् प्रेतक्रिया येन केनापि च कृता यदि ।\\
न तां निर्धर्तयेत् प्राज्ञः सतां धर्ममनुस्मरन् ॥\\
इति वायुपुराणवचनाञ्च ।\\
दाक्षिणात्यास्तु ज्येष्ठासन्निधाने कनिष्ठेन कृतं सपिण्डीकरणं ज्ये.\\
ष्ठेन पुनरावर्तनीयं किन्तु प्रेतशब्दोल्लेखो न कार्यं:-\\
कनीयसा कृतं कर्म प्रेतशब्दं विहाय तु ।\\
तज्यायसापि कर्तव्यं सपिण्डीकरणं पुनः ॥\\
इतिवचनादित्याहुः \textbar{}\\
ज्येष्ठश्चात्र वर्तमानानां मध्ये सर्वज्येष्ठः । न तु -\\
ज्येष्ठेन जातमात्रेण पुत्री भवति मानवः ।\\
पितॄणां मातृणां चैव स तस्माल्लुन्धुमर्हति ॥ {[}अ०९ श्लो०१०६{]}\\
इतिमनुतः सर्वपूर्वोत्पन्न एवं-\/- । तस्य विभागप्रकरणीयत्वेन विं.\\
शोद्धारादिप्राप्तये ज्येष्ठस्तुतिमात्रपरत्वात् । ज्येष्ठो
यद्यौद्धत्यादिव\\
शात् तत्र करोति तदा तत्कनीयसा कर्तव्यमेव । श्राद्धकरणस्वरूप.\\
योग्य ज्येष्ठसान्निध्याभावात् । प्रेतत्वविमुक्तये च तदनुष्ठानस्यावश्य.\\
कत्वात् । अन्यथा पतितादिज्येष्ठसान्निध्ये का गतिः । सपिण्डीकर\\
णोत्तरं क्रियमाणेषु अमावास्याश्राद्धादिषु तु अविभक्तानामपृथगेव,\\
विभक्तानां तु पृथगेवाधिकारः । तदाह-\\
बृहस्पतिः ।\\
एकपाकेन वसतां पितृदेवद्विजार्चनम् ।\\
एकं भवेत् विभक्तानां तदेव स्यात् गृहे गृहे ।

{ मनुरपि ।\\
एवं सहवसेयुर्वा पृथग्वा धर्मकाम्यया ।\\
पृथग्विवर्धते धर्मस्तस्माद्धर्म्या पृथक् क्रिया ॥\\
{[}अ० ९ लो० १११ {]}\\
अत्र विभागस्य पृथक्धर्मनिमित्तता प्रतीतेरविभागे धर्मानुष्ठानं\\
पृथङ् नास्तीति प्रतीयते ।\\
यत व्यासवचनं-\\
अर्वाक्संवत्सराज्ज्येष्ठः श्राद्धं कुर्यात् समेत्य वै ।\\
ऊर्ध्द्वं सपिण्डीकरणात् भर्त्रे कुर्युः पृथक् पृथक् ॥\\
इति, तत् -

 एकादशाद्याः क्रमशो ज्येष्ठस्य विधिवत् क्रियाः ।

{ श्राद्धाधिकारिनिरूपणम् । ३४३\\
कुर्युर्नैकैकशः श्राद्धमाब्दिकं तु पृथक् पृथक् ॥\\
इति प्रचेतोवाक्यैकवाक्यतया -\\
अविभक्ता विभक्ता वा- कुर्युः श्राद्धमदैवतम् ।\\
मघासु च ततोऽन्यत्र नाधिकारः पृथग्विना ॥\\
इत्यापस्तम्वस्यैकवाक्यतयाचाब्दिकश्राद्धपरं, विभक्त कर्तव्यामावा•\\
स्थादिश्राद्धपरं वा-, अस्मिंश्च पक्षे प्रचेतोवचने आब्दिकग्रहणं सपि\\
ण्डीकरणोत्तरभाविधाद्धमात्रोपलक्षणार्थमिति ज्ञेयम् ।\\
शूलपाणिस्तु लघुहारीतवाक्ये सपिण्डीकरणान्तानीति विशेषणाद-\\
पिशब्दस्वरसाथ सपिण्डीकरणादूर्ध्वं विभक्तानामविभक्तानामपि\\
पृथक् श्राद्धमिति सिद्ध्यति । अत एवं-\/- व्यासवाक्ये "ऊर्ध्व सपिण्डी\\
करणात सर्वं कुर्युः पृथक्पृथक्" इति विभक्तांविभक साधारण्येन सर्वं\\
इत्युक्तम् । मनुबृहस्पतिभ्यां तु अविभक्तानामपृथग् धर्मः प्रतिपादित.\\
स्तस्मात् पृथग्धर्मानुष्ठाने फलातिशयः प्रतीयत इत्याह ।\\
अथ द्वादशविधपुत्राणां स्वरूपं श्राद्धाधिकारक्रमच निरूप्यते ।\\
तत्र याज्ञवल्क्यः ।\\
औरसो धर्मपर्ताजस्तत्समः पुत्रिकासुतः ।\\
क्षेत्रजः क्षेत्रजातस्तु सगोत्रेणेतरेण वा- \textbar{}\textbar{}\\
गृहे प्रच्छन्न उत्पन्नो मूढजस्तु सुतः स्मृतः ।\\
कानीनः कन्यकाजातो मातामहसुतो मतः ।\\
अक्षतायां क्षतायां वा- जातः पौनर्भवः सुतः ॥\\
दद्यान्माता पिता वायं स पुत्रो दत्तको भवेत् ।\\
क्रीतश्च ताभ्यां विक्रीतः कृत्रिम: स्यात् स्वयंकृतः
\textbar{}\textbar{}\\
दत्तात्मा तु स्वयं दत्तो गर्भे विनः सहोदजः ।\\
उत्सृष्टो गृह्यते यस्तु सोऽपविद्धो भवेत् सुतः ॥\\
पिण्डदौऽशहरश्चैषां पूर्वाभावे परः परः ।\\
{[} अ० २ प्र० ८ श्लो० १२८-१३२ {]}\\
उरलो जात औरसः=स धर्मपत्नीजः । सवर्णा धर्मविवाहोढा धर्म.\\
'पत्नी तस्यां जात औरसः पुत्रो मुख्य इति मिताक्षरा ।\\
मनुरपि -\\
(१) संस्कृतायां सवर्णायाँ स्वयमुत्पादयेत् तु यम् ।

% \begin{center}\rule{0.5\linewidth}{0.5pt}\end{center}

( १ ) स्वक्षेत्रे सस्कृतायां त्विति मुद्रितपुस्तके पाठः ।

{३४४ वीरमित्रोदयस्य श्राद्धप्रकाशे-\\
तमौरसं विजानीयात् पुत्रं प्रथमकल्पिकम् \textbar{}\textbar{}\\
{[} अ० ९ श्लो० १६६ {]}\\
तत्समः = औरससमः । पुत्रिकायाः सुतः पुत्रिकासुतः ।\\
अभ्रातृकां प्रदास्यामि तुभ्यं कन्यामलङ्कृताम् ।\\
अस्यां यो जायते पुत्रः स मे पुत्रो भवेदिति ॥\\
{[}इति{]} वशिष्ठाद्युक्तविधानेन दत्तायामुत्पन्नः, स च "यदपत्यं भ.\\
वेदस्यां तन्मम स्यात् स्वधाकरम्'' इतिसंवित्कृतो मातामहमात्रसम्ब\\
न्धी एकः पुत्रिकापुत्रः । " यदपत्यम्भवेदस्यां तद् द्वयोः स्यात्
स्वधाक.\\
रम्'' इति संविदोत्पन्नोऽपरः । अथवा पुत्रिकैव सुतः पुत्रिकासुतः । स\\
त्रिविधः, औरससमः । औरसक्षेत्रजावुक्त्वाऽऽह -\\
वसिष्ठः ।\\
तृतीयः पुत्रिकैबेति ।\\
पुत्रिकैष तृतीयः पुत्र इत्यर्थः । क्षेत्रजः क्षेत्रजातास्त्विति =
क्षेत्रं = भार्या ।\\
मनुरपि ।\\
यस्तल्पजः प्रमीतस्य क्लीवस्य व्याधितस्य वा- ।\\
स्वधर्मेण नियुक्तायां स पुत्रः क्षेत्रजः स्मृतः ॥\\
{[} अ० ९ श्लो० १६७ {]}\\
तल्प = भार्या \textbar{} स्वधर्मेण = घृताभ्यक्तत्वादिना । तदाह-\\
मनुरेव ।\\
विधवायां नियुक्तस्तु घृताको वाग्यतो निशि ।\\
एकमुत्पादयेत् पुत्रं द्वितीयं न कदाचन \textbar{}\textbar{}\\
{[} अ० ९ श्लो० ६० {]}\\
अयं च अस्यां यदपत्यं जायते तदावयोरित्येवंरूपक्रियाभ्युपगमे\\
बीजिनोऽपि । तदाह-\\
मनुः ।\\
(१)क्रियाभ्युपगमात्वेवं बीजार्थ यत् प्रदीयते ।\\
तस्येह भागिनौ दृष्टौ बीजी क्षेत्रिक एवं-\/- च ॥ {[}अ०९ श्लो० ५३{]}\\
इत्यादि द्विपितृकश्राद्धदेवतानिर्णये तत्प्रपञ्चितम् । बीजिनस्त्व.\\
यं नौरसः "औरसो धर्मपत्नीजः" इति याज्ञवल्क्योक्तेः, किन्तु पौनर्भ\\
वविशेषः । "अक्षतायां क्षतायां वा- जातः पौनर्भवः सुत" इति याज्ञव\\
ल्क्यवचनात् । गृह इत्यादि । गूढजः पुत्रो भर्तृगृहे प्रच्छन्न उत्पन्नो
होना•

% \begin{center}\rule{0.5\linewidth}{0.5pt}\end{center}

( १ ) क्रियाभ्युपगमात्वेतत् इति मुद्रितपुस्तके पाठः ।

{ श्राद्धाधिकारिनिरूपणम् । ३४५\\
धिकजातीयपुरुषजत्व परिहारेण पुरुषविशेषजत्वनिश्चयाभावेऽपि स.\\
वर्णजत्वनिश्चये सत्ति बोद्धव्य इति मिताक्षरा । अयं च क्षेत्रिण एवं-\/-
।\\
तथा च-\\
मनुः ।\\
उत्पाद्यते गृहे यक्ष्य न च ज्ञायेत कस्य सः ।\\
स गृहे गूढ उत्पन्नस्तस्य स्याद्यस्य तल्पजः ॥ इति ।\\
{[} अ० ९ श्लो० १०० {]}\\
कानीन इति । कन्या=अपरिणीता । तथा च-\\
वशिष्ठः ।\\
कानीनः पञ्चमोऽयं पितृगृहे संस्तुता कामादुत्पादयेत् स\\
काननो मातामहपुत्रो भवतत्यिाह ।\\
अथाप्युदाहरन्ति ।\\
अप्रत्ता दुहिता यस्य पुत्रं विन्देत तुल्यतः ।\\
पुत्री मातामहस्तेन दद्यात् पिण्डं हरेद्धनम् ॥\\
तुस्यतः = सजातीयात् ।\\
ब्रह्मपुराणे ।\\
अदत्तायान्तु यो जातः सवर्णेन पितुर्गृहे ।\\
स कानीनः सुतस्तस्य यस्मै सा- दीयते पुनः ॥

{ मनुः ।\\
पितृषेश्मनि कन्या तु यं पुत्रं जनयेद्रहः ।\\
तं कानीन वदेशान्ना वोडुः कन्यासमुद्भवम् ॥\\
{[} अ० ९ मेला० १७२ {]}\\
वोढुः = उत्तरकालं वोढु, "अदत्तायां त्वि" त्यादि ब्रह्मपुराणैकवा\\
क्यात् । तं कन्यासमुद्भवं पुत्रं नाम्ना कानीनं वदेदित्यन्वयः । कन्या.\\
समुद्भवमिति कानीनशब्दस्य योगकथनाय । अत्र मातामहपुत्रत्वषि.\\
धायके याज्ञवल्क्य वशिष्ठवाक्ये अपुत्रमातामहविषये । परिणेतृपुत्र.\\
त्वविधायके तु ब्रह्मपुराणमनुवाक्ये अपुत्रपरिणेतृविषये । उभयोः\\
सपुत्रत्वे वा- उभयोरेवेति सम्प्रदायः ।\\
मिताक्षराकारस्तु काननिः कन्यकायामुत्पन्नः पूर्ववत् सवर्णात्, स.\\
मातामहस्य पुत्रः, यद्यनूढा सा- भवेत् तथा पितृगृह एवं-\/- संस्थिता,\\
अथ ऊढा तदा वोदुरेव पुत्रः "पितृवेश्मनी "त्यादिमनुवाक्यादित्या\\
इ । तश्चिन्त्यम् । "अदत्तायां तु यो जात'' इत्या दित्रह्मपुराणवाक्यवि.\\
वी० मि ४४

{३४६ वीरमित्रोदयस्य श्राद्धप्रकाशे-\\
रोधात् । "अक्षतायां क्षतायां वा- जातः पौनर्भवः सुत" इति याज्ञव\\
ल्क्यवचनेन परिणयोत्तरं परिणेत्रक्षतयोनिकायामुत्पन्नस्य पौनर्भव\\
त्वाभिधानेन कानीनत्वाभावाच्च । अक्षतायामिति । अक्षतायां क्षतायां\\
वा- पुनर्स्वा सवर्णादुत्पन्नः पौनर्भवः ।\\
विष्णुः ।\\
या तु पत्या परित्यक्ता विधवा स्वेच्छ्याथवा ।\\
उत्पादयेत् पुनर्भूत्वा स पौनर्भव उच्यते ॥\\
मनुरपि ।\\
या पत्या वा- परित्यक्ता विधवा वा- स्वयेच्छया ।\\
उत्पादयेत् पुनर्भूत्वा स पौनर्भव उच्यते ॥\\
{[} अ० ९ श्लो० १७५ {]}\\
सा- चेदक्षतयोनिः स्याद्गतप्रत्यागतापि वा- ।\\
पौनर्भवेन भर्त्रा सा- पुनः संस्कारमर्हति ।\\
{[}अ० ९ श्लो० १७६ {]}\\
पुनर्भूत्वा = पुनरन्यस्य भार्याभूत्वा । सा-=स्त्री चेदक्षतयोनिः सती\\
अन्यमाश्रयेत् । तदा तेन पौनर्भवेन पुनर्भूत्वसम्पादकेनान्येन पत्या\\
पुनर्विवाहाण्य संस्कारमर्हति । यदि वा- कौमारं पतिं परित्यज्य अ.\\
व्यमाश्रित्य पुनः पूर्वपति प्रत्यागता भवति, तदापि तेन कुमारेण\\
भर्त्रा पुनर्विवाहाख्यसंस्कारमर्हति । अयं च पुत्रः उत्पादकस्यैव
\textbar{}\\
परित्यागेन मरणेन वा- तस्यां परिणेतुः स्वत्वाभावात् । स्पष्टमाह -\\
कात्यायनः ।\\
क्लीबं विहाय पतितं या पुनर्भजते पतिम् ।\\
तस्यां पौनर्भवो जातो व्यक्तमुत्पादकस्य सः ॥\\
दद्यान्मातेति । अत्र विशेषमाह -\\
मनुः ।\\
माता पिता वा- दद्यातां यमद्भिः पुत्रमापदि ।\\
सहशं प्रीतिसंयुक्तं स ज्ञेयो दत्रिमः सुतः \textbar{}\textbar{}\\
{[}अ० ९ श्लो० १६८ {]}\\
सदृर्शं = सजातीयम् । प्रीतिसंयुक्तं = अविप्रतिपन्नम् "विक्रियं चैव
दाने\\
वान नेयाः स्युरनेिच्छुव" इत्येकवाक्यत्वात् । श्राद्धचिन्तामणौ
प्रीतिसं.\\
युक्ताविति पठित्वा प्रीतिसंयुक्तौ भयाद्युपाधिरहिताविति व्याख्यातम् ।\\
"न स्त्री पुत्रं दद्यात् प्रतिगृह्णीयाद्वान्यत्रानुज्ञान भर्तुरिति
वशिष्ठवा.

{ श्राद्धाधिकारिनिरूपणम् । ३४७\\
क्येन स्त्रिया भर्त्रनुशां विना दानप्रतिग्रहनिषेधः । स जीवद्भर्तृका\\
विषयः । "माता पिता वा-" इत्यनेन मनुयाज्ञवल्क्याभ्यां परस्परनैरपेक्षेण\\
तयोदातृत्वप्रतिपादनात् । जीवति तु भर्तरि स्त्रिया अस्वातन्त्र्याद\\
नुशाग्रहणावश्यकत्वाभिप्रायेण ``अन्यत्रानुज्ञानाद्भर्तुः" इत्युक्तं
वसिष्ठेन ।\\
अत एवं-\/- तेनैव शुक्रशोणितसम्भवः पुरुषो मातापितृनिमित्तकस्तस्य\\
प्रदानविक्रयत्यागेषु मातापितरौ प्रभवत इत्यनेन मातापित्रोस्तुल्य-\\
मेव प्रभुत्वमुक्तम् । एवं-\/-\/-\\
क्षेत्रजादीन् सुतानेतानेकादश यथोदितान् ।\\
पुत्रप्रतिनिधीनाहुः क्रियालोपान् मनीषिणः ॥\\
{[} अ० ९ श्लो० १८० {]}\\
इति मनुवाक्येन स्त्रीपुंससाधारण्येन क्रियालोप हेतोः पुत्रकरणा-\\
वश्यकत्वप्रतिपादनेनापुत्राया विधवाया अपि विनापि भर्त्रनुज्ञां पुत्र.\\
करणं न विरुद्धम् । यस्तु वसिष्ठवाक्ये स्त्रियाः प्रतिग्रहनिषेधः स\\
सभर्तृकाविषय इति प्रागेवोकम् ।\\
वाचस्पतिमिश्रोऽपि पितुर्मातुश्च द्वयोः प्रत्येकमपि दानाधिकारः, इयां\\
स्तु विशेषो यत् सत्ति पितरि तद्नुञ्चानान्मातुर्दातृत्वं असति तु त-.\\
दद्भावेऽपीत्यनेनेदमेवाह ।\\
मनुवाक्ये आपग्रहणादनापदि न देयो, दातुरयं प्रतिषेधः । आपदि\\
प्रतिप्रहीतुः पुत्राभाव इत्यन्ये ।\\
वसिष्ठः ।\\
शुक्रशोणितसम्भवः पुत्रो मातापितृनिमित्तकः तस्य प्रदानविक-\\
यत्यागेषु मातापितरौ प्रभवतः नत्वेकं पुत्रं दद्यात् प्रतिगृह्णीयाद्वा\\
स हि सन्तानाय पूर्वेषां न स्त्री पुत्रं दद्यात्
प्रतिगृह्णीयाद्वान्यत्रानुज्ञा-\\
नाद्भर्तुः । पुत्रं प्रतिग्रहीष्यन् बन्धूनाहूय राजनि वाssवेद्य निवेश-\\
नस्य मध्ये व्याहृतिभिर्दुत्वाऽदूरबान्धवं बन्धुसन्निकृष्टमेव प्रतिगृ.\\
हृयात् । सन्देहे चोत्पत्रे दूरे शुद्रमिव स्थापयेत् । विज्ञायते ह्येकेन\\
बहूस्त्रायत इति । एकं पुत्रमित्युपलक्षणम् \textbar{} अनेक पुत्रसद्भावे
ज्येष्ठो\\
न देयः । ``ज्येष्ठेन जातमात्रेण पुत्री भवति मानवः" इति मनु-\\
वाक्येन तस्यैव पुत्रकार्यकरणे मुख्यत्वाभिधानात् । बधून्नाहूयेति दत्त-\\
कस्य अंशप्राप्त्याद्यर्थे, राजनीत्यपि दृष्टार्थम्, निवेशन=गृहम् ।\\
हुत्वेत्यत्र संख्यानभिधानादनादेशविहिताष्टोत्तरशहस्त्राष्टशतादिरूपा

{३૪૮ वीरमित्रोदयस्य श्राद्धप्रकाशे-\\
संख्या प्राह्या अदूरवान्धवमित्यत्यन्त देशभाषाविप्रकृष्टस्य प्रतिषेध
इति\\
मिताक्षरा । बन्धुसन्निकृष्टं प्रतिग्राह्यबन्धूनां समीपवर्तिनम् । इदमपि\\
दृष्टार्थम् \textbar{} परीक्षितस्यापि कथश्चित ब्राह्मण्यादि सन्देह
उत्पन्ने यावत्त\\
त्रिवृत्ति [स्तावद्] दूरे शुद्रवत् स्थापयेत् तेन न व्यवहरेत् ।
अपुत्रस्य पुत्र\\
करणावश्यकत्वद्योतनाय श्रुत्युपन्यासो विज्ञायत इति श्रूयत इत्यर्थः ।\\
एकेन = अनौरसेनापि पुत्रेण वहून् पित्रादीन् नरकात्त्रायत इति । अत्र\\
दातृप्रतिगृहीतृगोचरो नियमः, क्रीतस्वयंदत्तकृत्रिमापविद्धेष्वपि य\\
थासंभवं द्रष्टव्यो दृष्टार्थत्वात् । स्वयं दत्ते तु होमोऽपि । तत्रापि
प्र\\
तिग्रहसत्वात् "प्रतिग्रहष्विन्" इत्युपक्रम्य वसिष्ठेन होमविधानात् ।\\
अत्र तु स्त्रीशूद्रयोर्न होमस्तयोरविद्यत्वात् । पुत्रं
प्रतिग्रहष्यन्नित्यस्य\\
तु स्वर्गकामादिपदवत् सविद्यपरत्वेनाप्युपपत्तेः । प्रतिग्रहस्तु तयो\\
वंचनादाचाराच ज्ञेयः । अत्र विशेष:-\\
कालिकापुराणे । "औरसः क्षेत्रजचैवेत्याद्युपक्रम्य ।\\
दत्ताश्चापि तनया निजगोत्रेण संस्कृताः ।\\
आयान्ति पुत्रतां सम्यगन्यबीजसमुद्रवाः ॥\\
पितुर्गोत्रेण यः पुत्रः सस्कृतः पृथिवीपते ।\\
आचूडान्तं न पुत्रः स पुत्रतां याति चान्यतः ॥\\
चूड़ाधा यदि संस्कारा निजगोत्रेण वै कृताः।\\
दत्ताद्यास्तनयास्ते स्युरन्यथा दास उच्यते ॥\\
ऊर्ध्वं तु पञ्चमाद्वर्षान्न दत्ताद्याः श्रुता नृप ।\\
गृहीत्वा पञ्चवर्षीयं पुत्रेष्टिं प्रथमं चरेत् ॥\\
पौनर्भवं तु तनय जातमात्रं समानवेत् ।\\
कृत्वा पौनर्भवं स्तोमं जातमात्रस्य तस्य वै ।\\
सर्वोस्तु कुर्यात् संस्कारान् जातकर्मादिकान्नरः ॥\\
कृते पौनर्भवेस्तोमे सुतः पौनर्भवस्ततः ।\\
एकोद्दिष्टं पितुः कुर्यात् न श्राद्धं पार्वणादिकम् ॥\\
क्रीत इति । मातापितृभ्यां मात्रा पित्रा वा विक्रीतः, अत्रापि दत्तक-\\
वदेकं ज्येष्ठं वर्जयित्वा आपदि सवर्णोऽविप्रतिपन्नश्च शेयः । यत्तु -\\
क्रीणीयाद्यस्त्वपत्यार्थे मातापित्रोर्यमन्तिकात् ।\\
स ऋतिकः सुतस्तस्य सरशोऽसडशोपि वा ॥\\
[ अ० ९ श्लो० १७४ ]

{ श्राद्धाधिकारिनिरूपणम् । ३४९\\
इति मनुवाकये अलडश इत्युकं तत् क्रेतुर्गुणैरसदृश इत्यभिप्रायं\\
न तु विजातीयाभिप्रायम्। ``सजातीयेष्वयं प्रोक्तस्तनयेषु मया विधिः"\\
इति याज्ञवल्क्योपसंहाराविरोधात् ।\\
बौधायनः ।\\
मातापित्रोर्हस्तादन्यतरस्य वा- योऽपत्यार्थ गृह्यत स क्रीत इति ।\\
कृत्रिम इति । कृत्रिमस्तु पुत्रः स्वयं पुत्रार्थिना
घनक्षेत्रप्रदर्शनादिप्रलो\\
भनैः पुत्रीकृतः अय च मातापितृरहितः, तत्सद्भावे तत्परतन्त्रत्वात् ।\\
अपविद्धस्तु मानापितृभ्यां त्यक्तः । स्वयंदत्तस्तु स्वात्मानं ददातीति\\
ताभ्यां कृत्रिमस्य भेदः ।\\
मनुः ।\\
( १ ) सदृशं यं प्रकुर्वीत गुणदोषविचक्षणम् ।\\
पुत्रं पुत्रगुणैर्युकं स विज्ञेयस्तु कृत्रिभः ॥\\
{[} अ० ९. श्लो० १६९ {]}\\
गुणदोषविचक्षण= पित्रोरौद्यकरणे दोषस्, तत्करणे गुण-\\
स्य च विवेचकम् । पुत्रगुणैः = पित्रोराराधनादिरूपैः । दत्तात्मेति ।
माता-\\
पितृविहीनस्ताभ्यामकारणात् त्यक्तो वा- तवाहं पुत्रोऽस्मीति स्वमा.\\
त्मानं यो ददाति स स्वयदत्त इत्यर्थः । तथाच-\\
मनुः।\\
मातापितृविहीनो यस्त्यतो वा- स्यादकारणात् ।\\
आत्मानं स्पर्शयेद्यस्तु स्वयं दत्तस्तु स स्मृतः ॥\\
{[} अ० ९ श्लो० १७७ {]}\\
अकारणात् = त्यागकारणं विना । स्पर्शयेत् दद्यात् । गर्भे विन्न इति ।\\
विन्न=परम्परया परिणयसंस्कारवान् । सवर्णात्सम्भूतः परिणयन-\\
काले स्थितो गर्भः सहोढः, तस्माजातः सहोढजः पुत्रः । तथा च-\\
मनुः ।\\
या गर्भिणी संस्क्रियते शाताज्ञातापि वा- सती ।\\
वोदुः स गर्भो भवति सहोढ इति चोच्यते ॥\\
{[} अ० ९ लो० १७३ {]}\\
ज्ञाता गर्भवत्वेन, तथा अज्ञातापि, वोदुरिति विशेषाभिधानान्न-\\
बीजिनो न कानीनवन्मातामहस्यापि । ननु -\\
पाणिग्रहणिका मन्त्राः कन्यास्वेव प्रतिष्ठिताः ।

% \begin{center}\rule{0.5\linewidth}{0.5pt}\end{center}

( १ ) संदृशं तु प्रकुर्याद्यमिति मुद्रितपुस्तके पाठः ।

{३५० वीरमित्रोदयस्य श्राद्धप्रकाशे-\\
नाकन्यासु कचिन्नृणां लुप्तधर्मक्रिया हि ताः \textbar{}\textbar{}\\
{[} अ० ८ श्लो० २२६ {]}\\
इति मनुवचनेन कन्यास्वेव पाणिग्रहाणिक मन्त्रविधानात् कथमन्त्र\\
गर्भिणी संस्क्रियत इत्युक्तम् । कन्यात्वं हि दानेनोपभोगेन }{वा}{ गच्छ-\\
ति "कानीनः कन्यकाजात" इत्यत्र तु कन्यापदमदत्तापरम् । अस्या\\
अनुपभुक्तत्वासम्भवात् । "कन्यामुपयच्छेत" इत्यत्र तु कन्यापदम\\
नुपभुक्ताऽदत्तापरम् । अत }{एव}{ "अनन्यपूर्विकाम्" इत्यत्र याशवल्क्यव.\\
चने दान उपभोगे वा- अन्यः पूर्वो यस्या नास्तीति सर्वनिबन्धूभि\\
व्र्वाख्यातम् । एवं च "कन्यामुपयच्छेत" इत्यत्र कन्यापद्मेतदेक.\\
वाक्यतया अनन्यपूर्विकापरम् । अत }{एव}{ च कुन्त्या अदत्ताया अपि\\
सम्भोगेन कन्यात्वापगममवगत्य "कन्यैव त्वं भविष्यसि " इति\\
तस्यै सूर्येण वरो दत्तः ।\\
मनुस्मृतौ च ।\\
अकन्येति च यः कन्यां ब्रूयाद् द्वेषेण मानवः ।\\
स शतं प्राप्नुयाइण्डं तस्या दोषमदर्शयन् \textbar{}\textbar{}\\
{[} अ० ८ श्लो० २२५ {]}\\
इति वाक्ये अक्षतयोनिकायां कन्याशब्दः स्पष्ट }{एव}{ । एवं च\\
सत्ति सगर्भायाः कथं संस्कार इति चेत् ।\\
अत्रोच्यते । पाणिग्रहणाख्यसंस्कारः भार्यास्वलक्षणः }{कन्या-}{\\
स्वेवोत्पद्यते नाकन्यासु "अनन्यपूर्विकाम्"
}{इत्यनेनान्यपूर्विकायास्तत्र}{\\
पर्युदस्तत्वात् । स च संस्कार: पाणिग्रहणिकमन्त्रैरुत्पद्यते ।\\
पाणिग्रहणिका मन्त्रा नियतं दारलक्षणम् ।\\
तेषां निष्ठा तु विज्ञेया विद्वद्भिः सप्तमे पदे ॥\\
{[} अ० ८ श्लो० २२७{]}\\
इत्यनेन मनुना पाणिग्रहणिकमन्त्राणां दारत्वनिष्पादकत्वाभिधा-\\
नात् । तथा सत्ति ते मन्त्राः कन्यास्वेव प्रयोज्या नाकन्यासु तत्र सं-.\\
स्कारानुत्पत्तेः, अत एवोकं "लुप्तधर्मक्रिया हि ता" इति । }{होमादि-}{\\
रूपः संस्कारस्तु तत्राप्यन्यपूर्विकादावनुवर्तत एवेत्यभिप्रायेणोक्तं\\
"या गर्भिणी संस्क्रियत" इति एवञ्च अन्यपूर्विकायां गर्भिण्यां भा-\\
र्यात्वानुत्पादेन तस्यां विवाहे तत्सम्पादकाः पाणिग्रहणिका मन्त्रा\\
न प्रयोज्या होमादि संस्कारस्त्वनुवर्तत एवेति }{प्रागुक्तम्}{ । गान्ध\\
\textgreater{}}

{ श्राद्धाधिकारिनिरूपणम् । ३५१\\
र्वादिविवाहे तु भार्यात्वसिद्ध्यर्थे मन्त्राः प्रयोज्या एव । अत एव
तेन,\\
गान्धर्वादिविवाहेषु पुनर्वैवाहिको विधिः ।\\
कर्तव्यश्च त्रिभिर्वर्णैः समयेनाझिसाक्षिकः ॥\\
इत्यनेन गान्धर्वादिविवाहेषु सर्वोऽपि वैवाहिको विधिरुक्तः ।\\
उत्सृष्टोगृह्यत इति । अत्र विशेषमाह -\\
मनुः ।\\
मातापितृभ्यामुत्सृष्टं तयोरन्यतरेण वा- ।\\
यं पुत्रं प्रतिगृह्णीयादपविद्धः स उच्यते ।\\
{[} अ० ९ लो० १७१ {]}\\
उत्सृष्टं-त्यक्तं भरणासामर्थ्येन, गण्डजातत्वादिदोषेण वा, न तु\\
पातित्येन, तस्य असंग्राह्यत्वात् । एतेषां च द्वादशपुत्राणां यो[[ग]
नार-\\
दस्मृतौ -\\
औरसः क्षेत्रजश्चैव पुत्रिकापुत्र एवं-\/- च ।\\
कानीनश्च सहोदश्च गूढोत्पन्नस्तथैव च ॥\\
पौनर्भवोऽपविद्धश्च दत्तः क्रीतः कृतस्तथा ।\\
स्वयं चोपागतः पुत्राः द्वादशैते प्रकीर्तिताः \textbar{}\textbar{}\\
इत्यन्यथा क्रमो दृश्यते ।\\
एवं मनुस्मृतावपि ।\\
औरलः क्षेत्रजचैव दत्तः कृत्रिम एवं-\/- च ।\\
मूढोत्पन्नोपविद्धश्च दायादा बान्धवाश्च षट् ॥\\
{[} अ० ९ श्लो० १५९ {]}\\
इति तेषामन्यथा क्रमो दृश्यते । अत्र पुत्रिकापुत्रो नोक्तः । पा-\\
रशवापरनामा शौद्रश्वोक्तः, तलक्षणं चाप्रे वक्ष्यते । स न श्राद्धाद्य-\\
धिकारक्रमतात्पर्यकः किन्तु परिगणनमात्रतात्पर्यकः । तत्र याज्ञव-\\
ल्क्येन "पूर्वाभावे परः परः" इत्यन्यथाक्रमविधानात् ।\\
यद्यपि विष्णुस्मृतौ औरसक्षेत्रजपुत्रि का पुत्र पौनर्भवकार्नानिगूढोत्प\\
न्नसहोढदत्तकक्रीतस्वयमुपगतापविद्धयत्र क्वचनोत्पादितानां प्रथमद्वि.\\
तीयतृतीयादित्वेनाभिधानादेतेषां पूर्वः पूर्वः श्रेयान् स एवं-\/- दायहर\\
इत्युक्तेश्च क्रमोऽभिहित एवं-\/-, दायहरत्वेन च
पिण्डदातृत्वस्यौत्सर्गिक\\
त्वात् पिण्डदाने स एवं-\/- तदनुमतः क्रम इत्युन्नीयते । तथा च विष्णू-\\
तयाज्ञवल्क्योक्तक्रमयोर्विकल्प इति प्रतीयते ।

{३५२ वीरमित्रोदयस्य श्राद्धप्रकाशे-\\
तथापि विज्ञानेश्वरादिप्रामाणिक निबन्धूमिर्याज्ञवल्क्यो क्तक्रमा.\\
ङ्गीकारात् स एवादरणीयः । विष्णुस्मृतौ तु कृत्रिमो नोक्तः, पारशवा\\
परनामा यत्र वचनोत्पादित उक्तः । तल्लक्षणमाह -\\
मनुः ।\\
यं ब्राह्मणस्तु श्रद्रायां कामादुत्पादयेत् सुतम् ।\\
स पारयन्नेव शवस्तस्मात् पारशवः स्मृतः ॥\\
{[} अ० ९ श्लो० १७८ {]}\\
"विनाध्वेष विधिस्मृतः" इति याज्ञवल्क्यदर्शनात् शुद्रायामूढाया.\\
मिति कुल्लूकभट्टः । यत्र वचनोत्पादितस्तु द्वादश इति विष्णुस्मृतिदर्श\\
नाव ऊढायामनूढायां वेति वाचस्पतिमिश्रः । ब्राह्मण इति त्रैवर्णिको पल.\\
क्षणम् । पारयन् पुत्रान्तररहितस्य पितुरुपकुर्वन । शवः शव इव शवः\\
असंपूर्णोपकारकत्वाच्छवव्यपदेशः । अत्र क्षेत्रजादीन् पारशवान्तान्\\
सुतानुक्त्वा ।\\
क्षेत्रजादीन् सुखानेकानेकादश यथोदितान् ।\\
पुत्रप्रतिनिधीनाहुः क्रियालोपान् मनीषिणः \textbar{}\textbar{}\\
{[} अ० ९ लो० १८० {]}\\
इति मनुवचनात् ।\\
ब्राह्मणेन न कर्तव्यं शुद्रस्यैवार्द्धदेहिकम् ।\\
शूद्रेण वा- ब्राह्मणस्य विना पारशवात् क्वचित् ॥\\
इति पारस्करवचनाच्च पूर्वोक्त द्वादशविधपुत्राभावे शूद्रापुत्रस्या\\
पि द्विजातिपितृश्राद्धादावधिकार इति प्रतीयते । यत्तु द्वादश\\
पुत्रानुक्त्वा ।\\
य एतेऽभिहिताः पुत्राः प्रसङ्गादन्यबीजजाः ।\\
यस्य ते बीजतो जाता स्तस्य ते नेतरस्य तु ॥\\
{[} अ० ९ श्लो० १८१ {]}\\
इति मनुवचनं तत् सत्यौरसे पुत्रे पुत्रिकापुत्रे वा- सत्ति ते न कर्त.\\
व्या इत्येवं परम् । "अन्यबीजजा" इति क्षेत्रजादिसकलपुत्रोपलक्ष\\
णार्थम् । अत एवं-\/- स्वर्बाजजातावपि पौनर्भवशौद्रो न कर्तव्यौ ।\\
अत एव

{ वृहष्पतिः ।\\
आज्यं विना यथा तैलं सद्भिः प्रतिनिधिः स्मृतः ।\\
तथैकादशपुत्रास्तु पुत्रिकौरसयोर्विना ॥

 श्राद्धाधिकारिनिरूपणम् । ३५३

{यत्तु-\/-\/-\/-\\
भ्रातॄणामेकजातानामेकश्चेत् पुत्रवान् भवेत् ।\\
सर्वोस्तांस्तेन पुत्रेण पुत्रिणो मनुरब्रवीत् ॥\\
{[} अ० ९ श्लो० १८२ {]}\\
सर्वासामेकपत्नीनामेका चेत् पुत्रिणी भवेत् ।\\
सर्वास्तास्तेन पुत्रेण प्राह पुत्रवती मनुः ॥\\
{[} अ० ९ श्लो० १८३ {]}\\
इति मनुवचनद्वयम् । तत्राद्यवचनं भ्रातृपुत्रस्य पुत्रीकरणसंभवे\\
ऽन्येषां पुत्रीकरणनिषेधार्थ [न] पुनः पुत्ररखप्रतिपादनाय, "तत्सुतो
गो\\
जो बन्धुः" इति याज्ञवल्क्यवचनेन भातृपर्यन्ताभाव एवं-\/-
भ्रातृपुत्रस्या.\\
धिकारप्रतिपादनविरोधात् इति मिताक्षरा । द्वितीयवचनं तु पुत्र\\
त्वप्रतिपादनार्थम् ।\\
विद्ध्यादौरसः पुत्रो जनन्या और्द्धदेहिकम् \textbar{}\\
तदभावे सपतीजः क्षेत्रजाद्यास्तयाहता ॥\\
तेषामभावे तु पतिस्तदभावे सपिण्डकाः ॥\\
इति कात्यायनवचने तस्यापि पुत्रकार्यकरणश्रवणात् ।\\
अन्ये तु\\
अयं पुत्रित्वातिदेशो भातृपुत्रसद्भावे सपत्नीपुत्रसद्भावे च प्रति\\
निधीभूतपुत्रकरण निषेधाद्यर्थः, स्त्रियाः सपिण्डनाद्यर्थश्च ।\\
अपुत्रेण सुतः कार्यो यादृक् तादृक् प्रयत्नतः ।\\
पिण्डोदकक्रियाहेतोर्नाम संकीर्तनस्य च \textbar{}\textbar{}\\
इति वचने पुत्ररहितस्यैव पुत्रीकरणविधानात् । "पतिपुत्रविही\\
नाया स्त्रिया नास्ति सपिण्डनम्" इति वचनेन पुत्रविहीनाया एवं-\/-\\
सपिण्डननिषेधाच्चेति प्राहुः ।\\
वस्तुतस्तु तेन पुत्रेणेतिश्रवणात् अयं पुत्रत्वातिदेशः, पुत्रित्खोप-\\
संहारदर्शनात्, पुत्रिस्वातिदेशश्च । तत्प्रयोजनं च
पुन्नामनरकन्त्राणादि,\\
पुत्रित्व पुरस्कारेण विधिनिषेधप्रवृत्तिश्च । न च तस्य पुत्रीकरणल.\\
म्भवे तत्सद्भावे चाम्यस्य पुत्रीकरणनिषेधः, अपुत्रेणेत्यादिवाक्येन\\
मुख्यपुत्ररहितस्यैव पुत्रीकरणविधानात् । अत एवं-\/- मनुनापि औरसा-\\
दीन द्वादशविधपुत्रानुक्त्वा "क्षेत्रजादीन् सुतानेतान्" इत्यादिना\\
पुत्रप्रतिनिधि [धी] भूतक्षेत्रजादिपुत्रीकरणमुक्तम् । तस्मान्
मनुरूक्रि.\\
वी० मि० ४५

{३५४ वीरमित्रोदयस्य श्राद्धप्रकाशे-\\
यालो पहेतोर्भ्रातृपुत्र सद्भावेऽपि क्षेत्रजादिपुत्रीकरणं न विरुद्धमिति\\
प्रतीमः । अत्र मनुवचने भ्रातृपद सहोदरभ्रातृपरम् । तदाह-\\
बृहस्पतिः ।\\
यद्येकजाता बहवो भ्रातरस्तु सहोदराः ।\\
एकस्यापि सुते जाते सर्वं ते पुत्रिणः स्मृताः ।\\
घहीनामेकपक्षांनामेष एवं-\/- विधिः स्मृतः ॥\\
एका खेत पुत्रिणो तासां सर्वासां पिण्डदस्तु सः ॥ }{इति}{ ।\\
ओर्ध्वदेहिकं चाऽनुपनीतस्यापि पुत्रस्याधिकारः ।\\
न ह्यस्मिन युज्यते कर्म किञ्चिदामोजिबन्धनात् ।\\
माभिव्याहारयेद्रा स्वधानिनयनाइते \textbar{}\textbar{}}

{ {[} अ० २ श्लो० १७१॥-१७२ {]}\\
}{इति}{ मनुवचनात \textbar{}\\
कुर्यादनुपनीतोऽपि श्राद्धमेकस्तु यः सुतः ।\\
पितृयज्ञाहुति पाणो जुहुयाद् ब्राह्मणस्य सः ॥\\
}{इति}{ वृद्धमनुवचनाथ । स च कृतचूडस्यैव । त्रिवर्षाई स्वकृत\\
चूड़स्यापि । तथा च-\\
सुमन्तु ।\\
अनुपेतोऽपि कुर्वीत मन्त्रवत् पैतृमेधिकम् ।\\
यद्यसौ कृतचूडः स्यात् यदि स्याच्च त्रिवत्सरः ॥ इति ।

{यत्तु-\/-\\
कृतचूडस्तु कुर्वीत उदकं पिण्डमेव च ।\\
स्वधाकारं प्रयुञ्जीत वेदोच्चारं न कारयेत् ॥\\
}{इति}{ व्याघ्रवचनं तन्मन्त्रोच्चारणासमर्थपुत्रपरम् ।\\
यस्तु श्राद्धाद्यनुष्ठानासमर्थः पुत्रः, तं प्रत्याह-\\
कात्यायनः ।\\
असंस्कृतेन पत्म्या च ह्यग्निदानं समन्त्रकम् ।\\
कर्तव्यमितरत् सर्वं कारयेद्दन्यमेव हि ॥ }{इति}{ ।\\
तादृशस्याप्यौरस पुत्रस्य सत्व उपनोतेनापि पुत्रिकापुत्रक्षेत्रजादिना\\
औध्वदेहिकादि न कर्तव्यम् । औरसासद्भाव एवं-\/- तेषामधिकारात् ।\\
एवं पुत्रिकापुत्रादिक्षेत्रजादिसमवायेऽपि बोध्यम् । तदेवं द्वाद\\
शविधपुत्राभावे पौत्रस्तदभावे प्रपौत्रः, विष्णुपुराणादौ तथा क्रम-\\
दर्शनात् ।

{ श्राद्धाधिकारिनिरूपणम् । ३५५\\
तथा च विष्णुपुराणम् ।\\
पुत्रः पौत्रः प्रपौत्रो वा- तद्वद्वा भ्रातृसन्ततिः ।\\
सपिण्ड सन्ततिर्वापि क्रियार्हा नृप ! जायते ॥\\
छन्दोगपरिशिष्ट च ।\\
पितामहः पितुः पश्चात् प्रेतत्वं यदि गच्छति ।\\
पौत्रेणैकादशाहादि कर्तव्यं श्राद्धषोडशम् \textbar{}\textbar{}\\
नैतत् पौत्रेण कर्तव्यं पुत्रवांश्वेत् पितामहः ।

{यत्तु ।\\
पौत्रश्च पुत्रिकापुत्रः स्वर्गप्राप्तिकरावुभौ ।\\
रिक्थे च पिण्डदाने च समौ तौ परिकीर्तितो ॥\\
}{इति}{ बृहस्पतिवचनं तत् पुत्रीकृतायाः कन्यायाः पुत्रपरम् ।\\
''अकृता वा- कृता वापि यं विन्देत् सहशात् सुतम् । पौत्री माताम\\
हस्तेन" इत्यनेन मनुवाकयेन तस्य पौत्रित्वाभिधानात् । अतश्च\\
पौत्रगोणपुत्रयोर्विकल्पार्थं तद्वचनं "पुत्रेषु विद्यमानेषु नान्यं वै\\
कारयेत्स्वधाम्" }{इति}{ ऋष्यशृङ्गवचने पुत्रेष्विति श्रवणात् गौणमुख्य\\
पुत्राभाव एवान्येषामधिकारप्रतीतेः । क्षेत्रजादिषु
पुत्रत्वोक्तेश्चेदमेव\\
प्रयोजन यत् पुत्रग्रहणेन तेषामपि ग्रहणमिति ।\\
कालादर्शे तु -\\
औरस पुत्राभावे पौत्रस्तदभावे प्रपौत्रस्तदभावे क्रमेण पुत्रिका\\
पुत्र - क्षेत्रज - दत्तक - क्रीत - कृत्रिम - स्वयं दत्तापविद्धा
इत्युक्तम् । पुत्रः\\
पौत्रः प्रपौत्रश्च पुत्रिकापुत्र एव च " इति स्मृतिसंग्रहवचनं तत्र
प्रमाण-\\
स्वेन लिखितम् । प्रपौत्राभावे पक्षी \textbar{}\\
तथा च शङ्ख. ।\\
पितुः पुत्रेण कर्तव्या पिण्डदानोदकक्रिया \textbar{}\\
पुत्राभावे तु पत्नी स्यात् पत्न्यभावे सहोदरः ।\\
भार्यापिण्ड पतिर्दद्यात् भर्त्रे भार्या तथैव च ॥\\
श्वश्वादेश्च स्नुषा चैव तदभावे सपिण्डकाः ॥ }{इति}{ ।\\
पुत्राभावे विश्यत्र पुत्रपदं पौत्रप्रपौत्रयोरप्युपलक्षणम् ।\\
यत्तु - "न भार्यायाः पतिर्दद्यात् पत्ये भार्या तथैव च" }{इति}{\\
छन्दोगपरिशिष्टच चनं तत् पुत्रपौत्रप्रपौत्रपर्यन्ताधिकारिसद्भावे बोध्यम्
।\\
यद्यपि "अपुत्रस्य तु या पुत्री सापि पिण्डप्रदा भवेत्" इति ऋष्य-

{३५६ वीरमित्रोदयस्य श्राद्धप्रकाशे-\\
शुङ्गवचने पुत्राभावे पुत्र्या अधिकारः प्रतीयते ।\\
अङ्गादङ्गात् सम्भवति पुत्रवद् दुहिता नृणाम् ।\\
तस्यामात्मनि जीवन्त्यां कथमन्यो हरेद्रनम् \textbar{}\textbar{}\\
}{इति}{ मनुवचनेन च पुत्राभावे प्रथमतो दुहितुर्धनाधिकारप्रतिपा.\\
दनेन "गोत्रऋक्थानुगः पिण्ड" इत्यनेन प्रथमतः पिण्डाधिकारः\\
प्रतीयते, तथापि पत्त्यनन्तरं तस्याधिकारो बोद्धव्यः ।\\
पत्नी दुहितरश्चैव पितरौ भ्रातरस्तथा ।\\
तत्सुतो गोत्रजो बन्धुः शिष्यः सब्रह्मचारिणः ॥\\
{[} अ० २ प्र० ८ श्लो० १३५ \textbar{}\\
एषामभावे पूर्वेषां धनभागुत्तरोत्तरः ।\\
स्वर्यातस्य ह्यपुत्रस्य सर्ववर्णेष्वयं विधिः \textbar{}\textbar{}\\
{[} अ० २ प्र० ८ श्लो० १३६ {]}\\
}{इति}{ याज्ञवल्क्यवचने पत्न्यनन्तर दुहितुर्धनाधिकारप्रतिपादनेन\\
श्राद्धेऽपि तथा प्रतीतेः !\\
यत्तु -\\
अपुत्रा स्त्री यथापुत्रं पुत्रवत्यपि भर्तरि ।\\
दद्यात् पिण्डं जलं चैव जलमात्रं तु पुत्रिणी ॥\\
दुहिता पुत्रवत् कुर्यान्मातापित्रोस्तु संस्कृता ।\\
आशौचमुदकं पिण्डमेकोद्दिष्टं सदा तयोः ॥\\
}{इति}{ पठ्यमानं वचनद्वयं तदमूलम् । समूलत्वेऽपि बाल देशान्तरित-\\
पुत्रसद्भावविषयमिति शूलपाणिः । सांवत्सरिक श्राद्धविषयमिति केचित् ।\\
यदपि च "सर्वाभावे स्त्रियः कुर्युः स्वभर्तृणाममन्त्रकम्" }{इति}{ मार्क\\
ण्डेयपुराणवचनम्। "कुलद्वयेऽपि चोच्छन्ने स्त्रीभिः कार्या सपिण्डता"\\
}{इति}{ विष्णुपुराणवचनं तदसवर्णविषयमिति शूलपाणिः ।\\
अपरे तु आसुरादिविवाहोढविषयमिदम् । याज्ञवल्क्यादिभिः प्रपौ.\\
त्रानन्तरं पत्न्या अधिकारप्रतिपादनेन अत्रत्यस्त्रीपदस्य पत्नीव्यक्त\\
[तिरिक्त ] स्त्रीपरत्वात् । आसुरादिविवाहोढायास्तु पत्नीत्वाभावमाह
-\\
शातातपः-\\
धम्यैर्विवाहैरूढा या सा- पक्षी परिकीर्तिता ।\\
सहाधिकारिणी ह्येषा यज्ञादौ धर्मकर्मणि ॥\\
क्रयक्रीता च नारी सा- न पत्नीत्यभिधीयते ।\\
न सा- देवे न सा- पित्र्ये दासीं तां मुनयो जगुः ॥

{ श्राद्धाधिकारिनिरूपणम् । ३५७\\
अत्र क्रयक्रीतेत्युपादानात् धर्म्या }{इति}{ विशेषणाच्चासुरादिवि.\\
वाहिताप्रतीतिरित्याहु: । अपरे तु अपरिणीताविषयमिति । अत्र यद्यपि\\
पत्नी दुहितरचैवेत्यादिवाक्यस्य विभक्तासंसृष्टपरत्वेन अविभक्तस्य\\
संसृष्टस्य च सहोदरस्य पक्षतिः पूर्वं धनाधिकारो वक्ष्यते ।\\
अपुत्रा शयनं भर्तुः पालयन्ती व्रते स्थिता ।\\
पत्न्येव दद्यात् तत्पिण्डं कृत्स्नमेशं लभेत च ॥\\
}{इति}{ मनुवचनमपि विभक्तासंसृष्टमतृधनपरम् । तथापि अविभ\\
कस्य संसृष्टस्य वा- सोदरस्य सत्वे पत्न्या अधिकाराभावेऽपि पत्न्याः\\
श्राद्धाधिकारी वाचनिकः \textbar{} "गोजारक्यानुगः पिण्ड" इत्यस्पौरस\\
र्गिकत्वेन प्रकृते अप्रवृत्तेः । पिनो॑र्धनग्राहित्वेऽपि भ्रातुः
श्राद्धाधि.\\
कारः । कालादर्शे तु पूर्वोक्तापविद्धपर्यन्तपुत्राभावे पत्नी, तदभावे
गू.\\
ढकानीनपौनर्भवाश्च इत्युकम् ।\\
काननिगूढ लहजपुनर्भूतनयाश्च ये ।\\
पत्न्यभावे तु कुर्युस्ते अप्रशस्ता यतः स्मृताः ॥\\
}{इति}{ स्मृतिसमहवचनमत्र प्रमाणत्वेनोकम् \textbar{} पत्न्यभावे कन्या,\\
"अपुत्रस्य तुया पुत्री सापि पिण्डप्रदा भवेत्" }{इति}{ ऋष्यशृङ्गवचना-\\
त् । याज्ञवल्क्येन परम्यनन्तरं तस्या धनाधिकारप्रतिपादनाच्च ।\\
मनुरपि ।\\
यथैवात्मा तथा पुत्रः पुत्रेण दुहिता समा ।\\
तस्यामात्मनि तिष्ठन्त्यां कथमस्या हरेद्धनम् ॥\\
[ अ० ९ श्लो० १३० ]

{ यत्तु-\\
पितुः पुत्रेण कर्तव्या पिण्डदानोदकक्रिया \textbar{}\\
तदभावे तु पत्नी स्वात् तदभावे सहोदरः \textbar{}\textbar{}\\
}{इति}{ शङ्खवचनं, यच्च "भ्रातुर्माता स्वयं चक्रे तद्भार्या चेन्न वि.\\
द्यते" इत्यादिपुराणवचनं तद् विभक्तसंसृष्टविषयमिति केचित् ।\\
अन्ये तु पत्नीभार्यापदे कन्याया अप्युपलक्षके । "पुत्रेण दुहिता\\
समा" इत्यनेन मनुना तस्याः पुत्रतुल्यत्वकथनेन सहोदरापेक्षया च-\\
लवत्वादिति वदन्ति ।\\
तत्र प्रथमतोऽनूढा, तस्याः सगोत्रत्वात, प्रथमं धनाधिकारा-\\
च्च, "गोत्ररिक्थानुगः पिण्ड" इत्यनेन मनुना पिण्डस्य गोत्रानुग.\\
त्वाभिधानात् ।

{३५८ वीररामेत्रोदयस्य श्राद्धमकाशे-\\
दुहिता पुत्रवत् कुर्यात् मातापित्रोस्तु संस्कृता ।\\
अशौचमुदक पिण्डमेकोद्दिष्टं सदा तयोः \textbar{}\textbar{}\\
}{इति}{ भारद्वाजवाक्यमसंस्कृताया मभावे सस्कृताया अप्यधिकार.\\
प्रतिपादकम् । तदभावे दौहित्रः ।\\
दत्तानां चाप्यदत्तानां कन्यानां कुरुते पिता \textbar{}\\
चतुर्थेऽहनि तास्तेषां कुर्वीरन् सुसमाहिताः ॥\\
मातामहानां दौहित्राः कुर्वन्त्यहिने चापरे ।\\
}{इति}{ ब्रह्मपुराणीयपाठक्रमात् ।\\
पौत्रदौहित्रयोर्लोके }{न}{ विशेषस्तु धर्मतः ।\\
तयोर्हि मातापितरौ सम्भूतौ तस्य देहतः ॥\\
{[} अ० ९ लो० १३३ {]}\\
}{इति}{ मनुवचनेन पौत्रदौहित्रयोस्तुल्यविधानाच्च । तेन यथा पु\\
त्राभावे पौत्रस्तथा दुहित्रभावे दौहित्र इति सिद्धम् ।
दुहितुरनन्तरं\\
दौहित्रस्य धनाधिकाराच्च स एवं-\/- पिण्डाधिकारी । }{न}{ च दत्तक\\
न्यादौहित्राभ्यां प्राक् सगोत्रत्वात् सोदराधिकार }{इति}{ वाच्यम् ।\\
पिण्डदानादेर्धनसाध्यतया गोत्रापेक्षया रिक्थस्य बलवत्वेन रि-\\
क्थग्राहिणोर्दुहितृदौहित्रयोर्बलवत्वात् ।\\
दौहित्राभावे सहोदरः-\\
पितुः पुत्रेण कर्तव्या पिण्डदानोदकक्रिया \textbar{}\\
तदभावे तु पत्नी स्यात् तद्द्भावे सहोदरः ॥\\
इतिशङ्खचाक्यात् । अत्र पक्षीपदं दौहित्रपर्यन्तोपलक्षणम् ।\\
ब्रह्मपुराणमपि ।\\
भ्रातुर्भ्राता स्वयं चक्रे तद्भार्या वेश विद्यते ।\\
तस्य भ्रातृसुतश्चक्रे यस्य नास्ति सहोदरः ॥\\
तत्र प्रथमतः कनिष्ठस्याधिकारः, तदभावे ज्येष्ठस्य । "नानुजस्य\\
तथाग्रज" }{इति}{ छन्दोगपरिशिष्टवचनेनानुजे विद्यमाने अग्रजस्य कनिष्ठ\\
भ्रातृश्राद्धनिषेधात् । यद्यप्यनेन ज्येष्ठस्याधिकार एवं-\/- निषिध्यते
।\\
तथापि -\\
पुत्रो भ्राता पिता वापि मातुलो गुरुरेव च ।\\
एते पिण्डप्रदा शेयाः सगोत्राश्चैव बान्धवाः ॥\\
इति प्रचेतोवचनेन भ्रातृत्वेन ज्येष्ठम्भ्रातुरप्यधिकारप्रतिपादनात्

{ श्राद्धाधिकारिनिरूपणम् । ३५९\\
कनिष्ठसद्भाव एवायं निषेध }{इति}{ सम्प्रदायः ।\\
अन्ये तु -\\
}{न}{ जायायाः पतिर्दद्यात् अपुत्राया अपि क्वचित् ।\\
}{न}{ पुत्रस्य पिता चैव नानुजस्य तथाग्रजः ॥\\
इतिछन्दोगपरिशिष्टवचनेन सामान्यतो यद्यपि }{पत्यादोर्निषेधः}{\\
श्रूयते, तथापि -\\
भार्यापिण्ड पतिर्दद्यात् }{भर्त्रे}{ भार्याां तथैव च ।\\
}{श्वश्व्रादेश्च}{ स्नुषा चैव तदभावे सपिण्डकाः ॥\\
}{इति}{ }{शङ्खवचनेन}{ पत्युरधिकारप्रतिपादनात "पुत्रो भ्राता पिता\\
वापी" त्यादिप्रचेतोवचनेन पितुरधिकारप्रतिपादनादाधकार्यन्तर सद्भा\\
वे पतिपित्रोर्निषेधकमस्त छन्दोगपरिशिप्रवाकयम् भ्रात्रधिकारप्र\\
तिपादकानां च वाक्यानां कनिष्ठभ्रातृपरत्वेनाप्युपपते }{"नोनुजस्य}{\\
तथाग्रजः" इत्येतस्यापि सङ्कोचे मानाभावः । यदि च-\\
यदि }{स्नेहेन}{ कुर्वीत सपिण्डीकरण विना \textbar{}\\
गयायां च विशेषेण ज्यायानपि समाचरेत् ॥\\
}{इति}{ छन्दोगपरिशिष्टनाम्ना पठ्यमान वचनं समूलं, तदाऽस्तु ज्याप\\
सोऽपि सपिण्डीकरणातिरिक्तश्राद्वेष्वधिकार }{इति}{ वदन्ति ।\\
}{कनिष्ठबाहुल्ये}{ त- प्रथमं मृतानन्तरस्तदभावे तदनन्तरः सन्नि\\
कर्षनारतस्यात् एवं-\/- ज्येष्ठबाहुल्येऽपि बोध्याममेति वदन्ति । सोद\\
राभावे वैमात्रेयः, तस्याध्ये कपितृजन्यत्वरूप भ्रातृत्व सद्भावेन
भ्रात्र'\\
धिकारप्रतिपादकवाकयेन तस्याप्यांधकारप्रतिपादनात् । }{न}{ चेकोद\\
रजन्यत्वमपि भ्रातृपदार्थान्त तामति वाच्यम् । तथासत्यसहोदरम्य\\
भ्रातृत्वाभावेन परिवदनाप्रसक्तो छन्दोगपाराशष्टिन देशान्तरस्थक्को-\\
बेकवृषणान सहोदरान् इत्युपक्रम्य "परिविन्दन्न दुष्यति" इत्यनेन\\
सोदरस्य परिवेदनप्रनिप्रसवानुपपत्तेः । तत्रापि ज्येष्ठकनिष्ठयोः क्र\\
मशोऽधिकारः पूर्वोक्तयुक्तेः । न च "तस्य भ्रातृसुतश्चक्रं यस्य
नास्ति\\
सहोदर : " }{इति}{ ब्रह्मपुराणवचनात् सहोदराभावे भ्रातृसुतस्यैवाधिका\\
रो न वैमात्रेयस्येति वाच्यम् । "भ्रातुर्भ्राता स्वयं चक्रे तद्भार्या
चेन्न\\
विद्यते" इति तत्पूर्वार्द्धार्थ पर्यालोचनया सोदरपदस्य
भ्रातृपरत्वप्रती.\\
तेः । तदभावे कनिष्ठः सहोदरस्य पुत्रः, तदभावे ज्येष्ठः सहोदरस्य\\
पुत्रः, तदभावे कनिष्ठवेमात्रेयः, तदभावे ज्यष्टवैमात्रेयः इति
क्रमः,\\
•

{३६० वीरमत्रोदयस्य श्राद्धप्रकाशे-\\
सहोदरादितुल्यन्यायत्वात् । तदभावे पिता । ``पुत्रो भूाता पिता वा-\/-\\
पी" त्यादिप्रचेतोवचनोक्कक्रमत्यागे बीजाभावात् । तदभावे माता ज.\\
मनी, "पुत्रो भाता पिता वापी" त्यत्रापिशब्देन मातुः समुच्चयात् ।\\
"पितरो भ्रातरस्तथा" इत्यत्र याज्ञवल्क्यवचने धनाधिकारे तथा\\
दर्शनात् ।\\
असमाप्तवतस्यापि कर्तव्यं ब्रह्मचारिणः ।\\
श्राद्धादि मातापितृभिनं तु तेषां करोति सः ॥\\
इति ब्रह्मपुराणाच्च । ``न च माता न च पिता कुर्यात् पुत्रस्य
पैतृ\\
कम्" इति कात्यायनवचनन्तु भातृपुत्रपर्यन्त सद्भावे द्रष्टव्यम् ।
जनन्य\\
भावेऽपि माता । "सर्वा पितृपत्न्यो मातर }{इति}{ सुमन्तुवाक्येन तस्या\\
अपि मातृत्वस्मरणेन मातृकार्यकारित्वोपपत्तेः । तदभावे स्नुषा ।\\
"}{श्वश्व्रादेश्च}{ स्नुषा चैव " इतिशङ्खवचनात् \textbar{} आदिपदेन
श्वशुरपरि.\\
ग्रहः, न तु श्वश्वा अपि श्वश्ररादिपदार्थः, स्नुषेत्यनुपपत्तेः ।
तदभावे\\
सपिण्डाः, सन्निधितारतम्येन क्रमशोऽधिकारिणः, तदभावे समानो.\\
दका: "सपिण्डसन्ततिर्वापि" इनिवक्ष्यमाणविष्णुपुराणवचनात् । तद.\\
भावे मातामहः ।\\
मातामहानां दौहित्राः कुर्वन्त्यहनि चापरे ।\\
तेऽपि तेषां प्रकुर्वन्ति द्वितीयेऽहनि सर्वदा ॥\\
}{इति}{ ब्रह्मपुराणात् । तदभावे मातुलः । तदभावे भागिनेयः "स्व.\\
स्त्रीयो मातुलस्य" }{इति}{ शातातपवाक्यात्, तदभावे सन्निधिक्रमेण\\
मातामहसपिण्डाः, तदभावे तत्समानोदका: । "मातृपक्षस्य पिण्डेन\\
सम्बद्धा ये जलेन वा-" }{इति}{ विष्णुपुराणवाक्यात् । तदभावे सन्निधिता.\\
रतस्येन स्वबान्धवा, तदभावे पितृबान्धवाः, तदभावे मातृबान्धवाः,\\
"सगोत्राश्चैव बान्धवाः" इत्युदाहृतप्रचेतेावाकयेन बान्धवानामप्यधि.\\
कारप्रतिपादनात् । "तत्सुतो गोत्रजो बन्धुः" }{इति}{ याज्ञवल्क्यवाक्येन\\
बन्धूनां धनाधिकारप्रतिपादनाच्च । ते च -\\
आत्मपितुः श्वसुः पुत्रा आत्ममातु: स्वसुः सुताः ।\\
आत्ममातुलपुत्राश्च विशेया आत्मबान्धवाः ॥\\
पितुः पितृष्वसुः पुत्राः पितुर्मातृष्वसुः सुताः ।\\
पितुर्मातुलपुत्राश्च विज्ञेयाः पितृबान्धवाः \textbar{}\textbar{}\\
मातुः पितुःस्वसुः पुत्रा मातुर्मातु: स्वसुः सुताः ।

{ जीवच्छ्राद्धनिर्णयः । ३६१\\
मातुर्मातुलपुत्राश्च विज्ञेया मातृबान्धवाः \textbar{}\textbar{}\\
तदभावे श्वशुरः, तदभावे जामाता, "जामातुः श्वसुराश्चक्रुः तेषां\\
ते चापि संयता: " }{इति}{ ब्रह्मपुराणात्, तद्भावे मातामहीभ्राता ।\\
"भागिनेयीसुतानां च सर्वेषां त्वपरेऽहनि ।\\
श्राद्धं कार्यं च प्रथमे स्नात्वा }{कृत्वा}{ जलक्रियाम् ॥\\
}{इति}{ ब्रह्मपुराणात् । तदभावे यथाक्रम शिष्यत्विगाचार्याः, "पुत्रा.\\
भावे सपिण्डा मातृसपिण्डा वा- शिष्याश्च दद्युः, तदभावे ऋत्विगा\\
चार्या." }{इति}{ गौतमस्मरणात् । तदभावे सब्रह्मचारिणः, "शिष्यलब्र\\
ह्मचारिणः" }{इति}{ याज्ञवल्क्येन सब्रह्मचारिणो धनाधिकारप्रतिपादनात् ।\\
तदभावे स्वसुहृत्पितृसुहृदौ, ``मित्राणां तदपत्यानाम्" इति
ब्रह्मपुराण.\\
वाक्यात् । सर्वाभावे राजा कारयेत् "सर्वाभावे तु नृपतिः कारयेत्त\\
स्य रिक्थत" }{इति}{ स्कान्दोक्तेः । अधिकारिविशेषेण कियाव्यवस्थोका-\\
विष्णुपुराणे ।\\
पूर्वाः क्रिया मध्यमाश्च तथा चैवोत्तराः क्रियाः ।\\
त्रिप्रकाराः क्रिया होतास्तासां भेदान् शृणुष्व मे ।\\
आदाहादादशाहाच्च मध्ये याः स्युः क्रिया नृप ।\\
ताः पूर्वा, मध्यमा मासि मास्येकोद्दिष्टसंशिताः ॥\\
प्रेते पितृत्वमापत्रे सपिण्डकिरणादनु ।\\
क्रियन्ते याः क्रियाः पुत्रा {[}त्रैः{]} प्रोच्यन्ते ता नृपोत्तराः ॥\\
पितृमातृसपिण्डैश्च समानसलिलैस्तथा ।\\
तत्सङ्घातगतैश्चैव राज्ञा च धनहारिणा ॥\\
पूर्वा मध्याश्च कर्तव्याः पुत्राद्यैरेव चोत्तराः \textbar{}\\
दौहित्रैर्वा नरश्रेष्ठ कार्यास्तत्तनयैस्तथा ॥\\
मृताहनि तु कर्तव्याः स्त्रीणामप्युत्तराः क्रियाः ।\\
{[} }{इति}{ अधिकारिनिर्णयः । {]}\\
अथ जीवच्छ्रादनिर्णय ।\\
तत्र बौधायनः ।\\
अथातो जीवच्छ्राद्धविधिं व्याख्यास्यामः। यस्त्वात्मनः श्रेयसमिच्छ\\
ति अपरपक्षे त्रयोदशीमुपोष्य तस्मिन्नेवाहनि सम्भारानुपकल्पयते, या.\\
न्यौर्ध्वदेहिकानि मृतानां, (१) वस्त्रषट्कं सुवर्णसूचीमङ्कुशं कन्यां
रज्जुं

% \begin{center}\rule{0.5\linewidth}{0.5pt}\end{center}

( १ ) वस्त्रषङ्कं सौवर्णी सूचीमङ्कुशं तान्तव पाशं कथां पलाशवृन्त
मौदुम्बरीमास•\\
न्दीं क्रलशानीति, अन्यान्यपि च । श्वोभूते स्नात्वा -- इति मुद्रित
बौधायनगृह्यसूत्रे पाठः ।\\
वी० मि ४६

{३६२ वीरमित्रोदयस्य श्राद्धप्रकाशे-\\
पालाशवृन्तं कृष्णाजिनमौदुम्बरीमा सन्दीं कलशादीम्यपि, तस्मिन्नेवा.\\
हनि, श्वोभूते ? स्नात्वा मध्याह्ने जले स्थिरवोपोत्थाय ( १ ) पुण्याहं
स्व.\\
स्त्ययनमिति वाचयित्वा वस्त्राङ्गुलीयकं दक्षिणां दद्यात्, सघृतं पायसं\\
दक्षिणामुखोऽश्नीयात्। अथ श्राद्धोक्तविधिनाश्निमुपसमाधाय परिस्ती.\\
योनिमुखान् कृत्वा पक्काज्जुहोति "चत्वारि शृङ्गा" इति
पुरोनुवाकया-\\
मनुष्य त्रिधा हितमिति याज्यथा जुहोति । तत्सवितुर्वरेण्यमिति\\
पुरोनुवाक्यामनूष्य 'योजयित्री सूनृतानाम्' }{इति}{ याज्यया जुहोति ।\\
ये चत्वार }{इति}{ पुरोनुवाक्यामनूच्य द्वे स्रुती }{इति}{ याज्यया जुहो'\\
ति । ( २ ) वसुनि पुरोनुवाक्यामनूष्य यातिरश्चनिपद्यते हमिति\\
याज्यथा जुहोति । अथाज्याहुतीरुपजुहोति, पौरुषेण सुतेनाष्टा\\
दशर्चेन घृतं हुत्वा गायत्र्या अष्टसहस्रमष्टशतमष्टाविंशर्ति वा-\\
जुहुयात् स्विष्टकृत्प्रभृतिसिद्धमाधेनुवरप्रदानात् । धार्य एवाग्निरास.\\
माप्तेः । चक्षुष्पथं गत्वा सूचीमङ्कशं कन्थां रज्जुमिति कृष्णतनवे\\
ह्रस्वाय ब्राह्मणाय दत्वा प्रीयन्तां यमकिङ्करा, }{इति}{ वाचयित्वा\\
व्रीहिषु कलशान् साइयेत् तन्तुनावेष्ट्य जलपूर्णान् पुरुषाकृर्ति\\
}{कृत्वा}{ त्रीणि शीणि, मुखे त्रीणि ग्रीवायामेकविंशतिं शरीरे चतुष्टयं,\\
बाहोर्द्वे द्वे लिङ्गस्यैकं, पादयोः पञ्च पञ्चति प्रीतोस्तु भगवान् यमः,\\
}{इति}{ । तत आसन्दी }{कृत्वा}{ पञ्चगव्येन प्रक्षाल्य पलाशवृन्तैः क्व·\\
ष्णाजिने पुरुषाकृति }{कृत्वा}{ कलशपुरुषे प्राणानभिनिवेश्य वृन्तशरीरे\\
देहमभिनिवेश्य स्वपेत् \textbar{} उदिते सूर्ये कलशैर्देहं
स्वयमेवाभिषेचयेत्\\
पौरुषेण सूक्तेन पञ्चगव्येन शुद्धोदकेन, सायाह्ने सतिलमन्न सर्पि-\\
षाऽश्नीयात् । ब्राह्मणानपि यमकिङ्करतृप्तये भोजयेत् । चतुर्थ्यो\\
यन्त्र दाहः, उदकं पिण्ड चामुकगोत्राय मह्यं पिण्डमामुत्रिकं स्वधेति\\
नमस्कारान्तं }{कृत्वा}{ समापयेत् ।\\
तत्राशौचं दशाहं स्थात् स्वस्य ज्ञातेर्न विद्यते ।\\
एकादश्यामेकोद्दिष्टमिति प्रतिपद्यते । अथाप्युदाहरन्ति ।

% \begin{center}\rule{0.5\linewidth}{0.5pt}\end{center}

{( १ ) पुण्याह स्वस्ति - ऋद्धिम्-}{इति}{ मु० बौ० पाठः ।\\
( २ ) अग्ने नय इति पुरोनुवाक्यामनूच्य या तिरश्चि इति ग्राज्यया
जुहोति ।\\
"}{इति}{ मु० बौ० पाठः ।\\
-

{ जीवनिर्णयः । ३६३\\
आपन्नः स्त्री च शूद्रश्च मन्त्रैर्दग्ध्वा स्वकां तनुम् ।\\
तदहैव क्रियाः सर्वाः कुर्यादित्येव हि श्रुतिः ।\\
स्त्रीणां तुष्णीं समन्त्रक वा-, मासिमास्येवं संवत्सरादूर्ध्वं प्रतिसं\\
वत्सरमाद्वादशाब्दात्, ततो निवृत्तिः, यदा स्वयं }{न}{ शक्नुषात्,\\
तदा पुत्रादयः कुर्युः ।\\
अथाप्युदाहरन्ति ।\\
जीवन्नेवात्मनः श्राद्धं कुर्यादन्येषु सत्स्वपि ।\\
यथाविधि प्रवृत्याशु सपिण्डीकरणाहते ॥ }{इति}{ ।\\
तस्योक्तं कालं }{न}{ विलम्बयेत्, यतोऽनित्यम् जीवनमिति शेष\\
समापयेत् इत्याह- भगवान् बौधायनः ।\\
}{लिङ्गपुराणे ।}{\\
जीवच्छ्राद्धविधिं वक्ष्ये समासाच्छ्रुतिसम्मतम् ।\\
मनवे देवदेवेन कथितं ब्रह्मणा पुरा ॥\\
( १ ) वासिष्ठाय वसिष्ठाय भार्गवाय च साम्प्रतम् ।\\
शृण्वन्तु सर्वभावेन सर्वसिद्धिकरं परम् ॥\\
श्राद्धमार्गक्रमं साक्षात श्रावार्हाणामपि क्रमम् ।\\
विशेषमपि वक्ष्यामि जीवच्छ्राद्वेषु यः स्मृतः ॥\\
पर्वते वा- नदीतीरे बने देवालयेऽपि वा- ।\\
जीवच्छ्राद्धं तु कर्तव्यं मृत्युकाले प्रयत्नतः ।\\
जीवच्छ्राद्धे कृते जीवो जीवश्चैव विमुच्यते ॥\\
कर्म कुर्वत्र कुर्वन् वा- अज्ञानो ज्ञानवानपि ।\\
श्रोत्रियोऽश्रोत्रियो वापि ब्राह्मणः क्षत्रियोऽपि वा- ॥\\
वैश्यो वा- नह्यत्र सन्देहो योगमार्गरतो यथा ।\\
परीक्ष्य भूमिं विधिना गन्धवर्णरसादिभिः \textbar{}\textbar{}\\
शल्यमुद्धृत्य यत्नेन स्थण्डिलं सैकतं भुवि ।\\
मध्यतो हस्तमानेन कुण्डं चैवाप्रतः शुभम् \textbar{}\textbar{}\\
स्थण्डिलं वा- प्रकर्तव्यमिषुमात्रं पुनः पुनः ।\\
उपलिप्य विधानेन चोलियासि निधाय च ॥\\
अन्वाधाय यथाशास्त्र ( २ ) परिसमूह्य सर्वतः ।\\


% \begin{center}\rule{0.5\linewidth}{0.5pt}\end{center}

(१) वसिष्ठाय च शिष्टाय भृगवे भार्गवाय च । इति मुद्रितलिङ्गपुराणे
पाठः ।\\
( २ ) परिगृह्य च सर्वत इति मुद्रितलिङ्गपुराणे पाठः ।

{३६४ वीरमित्रोदयस्य श्राद्धमका शे-\\
~\\
परिस्तीर्य स्वशाखोकं पारम्पर्यक्रमेण तु ॥\\
समाप्याग्निमुखं }{भर्त्रे}{ मन्त्रैरेतैर्यथाक्रमम् ।\\
सम्पूज्य स्थाण्डिले वहाँ होमयेत् समिधादिभिः ॥\\
आदौ }{कृत्वा}{ समिद्धोमं चरुणा च पृथक् पृथक् ।\\
घृतेन च पृथक्पात्रे शोधितेन पृथक् पृथक् ॥\\
जुहुयादात्मनोद्धृत्य तत्वभूतानि सर्वतः ।\\
ॐ भूर्ब्रह्मणे नमः । ॐ भूर्ब्रह्मणे स्वाहा । ॐ भुवः विष्णवे\\
नमः । ॐ भुवः विष्णवे स्वाहा । ॐ स्वः रुद्राय नमः \textbar{}\\
ॐ स्वः रुद्राय स्वाहा । ॐ महः ईश्वराय नमः । ॐमहः, ईश्वराय\\
स्वाहा । ॐ जनः प्रकृतये नमः । ॐ जनः प्रकृतये स्वाहा ।ॐ तपः\\
मुद्गलाय नमः । ॐ तपः मुद्रलाय स्वाहा । ॐ ऋतं पुरुषाय नमः ।\\
ॐ ऋतं पुरुषाय स्वाहा । ॐ सत्यं शिवाय नमः । ॐ सत्यं शिवा\\
य स्वाहा । ॐ शर्व घरां मे गोपाय घ्राणे गन्धं शर्वाय देवाय भूर्नमः ।\\
ॐ शर्व घरां मे गोपाय घ्राणे गन्धं शर्वाय देवाय भूः स्वाहा । ॐ शर्व\\
धरां मे गोपाय घ्राणे गन्धं शस्य देवस्य पत्न्यै भूर्नमः ॐ शर्व धरां\\
मेगोपाय त्राणे गन्धं शर्वस्य देवस्य पत्न्यै भूः स्वाहा । ॐ भव जलं\\
मे गोपाय जिह्वायां रस भवाय देवाय भुवो नमः । ॐ भव जलं मे\\
गोपाय जिह्वायां रसं भवाय देवाय भुवः स्वाहा । ॐ भव जलं मे\\
गोपाय जिह्वायां रसं भवस्य देवस्य पत्न्यै भुवो नमः । ॐ भव जलं मे\\
गोपाय जिह्वायां रसं भवस्य देवस्य पत्यै भुवः स्वाहा । ॐ रुद्राभि\\
मे गोपाय नेत्र रूपं रुद्राय देवाय स्वर्नमः । ॐ रुद्राभि में गोपाय\\
नेत्रे रूपं रुद्राय देवाय स्वः स्वाहा । ॐ रुद्राग्निं मे गोपाय नेत्रे
रूप\\
रुद्रस्य देवस्य परम्बै स्वर्नमः । ॐ रुद्राओं मे गोपाय नेत्रे रूपं
रुद्रस्य\\
देवस्य पत्न्यै स्वः स्वाहा । ॐ उम्र वायुं मे गोपाय त्वचि स्पर्शे
उप्राय\\
देवाय महर्नमः । ॐ उम्र वायुं मे गोपाय त्वचि स्पर्श उन्नाय दे.\\
वाय महः स्वाहा । ॐ उग्र वायुं मे गोपाय त्वचि स्पर्श उप्रस्य देवस्य\\
परम्यै महर्नम । ॐ उम्र वायु मे गोपाय त्वाचे स्पर्श उग्रस्य देवस्य\\
पत्न्यै महः स्वाहा \textbar{} {[} ॐ{]} भीम सुषिर मे गोपाय श्रोत्रे शब्द
भीमाय\\
देवाय जनो नमः । ॐ भीम सुषिरं मे गोपाय श्रोत्रे शब्द भीमाय\\
देवाय जनः स्वाहा । ॐ भीम सुषिरं मे गोपाय भत्रे शब्द भीमस्य\\
देवस्य परम्बै जनो नमः । ॐ भीम सुषिरं मे गोपाय श्रोत्रे शब्द\\


{ जीवच्छ्राद्धनिर्णयः । ३६५\\
भीमस्य देवस्य पत्न्यै जनः स्वाहा । ॐॐ ईश रजो मे गोपाय द्रव्ये\\
तृष्णां ईशाय देवाय तपो नमः । ॐ ईश रजो मे गोपाय द्रव्ये तृष्णां\\
ईशाय देवाय तपः स्वाहा । ॐ ईश रजो मे गोपाय द्रव्ये तृष्णां\\
ईशस्य देवस्य पत्न्यै तपो नमः । ॐ ईश रजो मे गोपाय द्रव्ये तृ.\\
Sणामीशस्य देवस्य पत्न्यै तपः स्वाहा । ॐ महादेव सत्यं मे गोपाय\\
भद्रां धर्मे महादेवाय ऋतं नमः । ॐ महादेव सत्यं मे गोपाय श्रद्धां\\
धर्मे महादेवाय ऋतं स्वाहा । ॐ महादेव सत्यं मे गोपाय श्रद्धां धर्मे\\
महादेवस्य पत्न्यै ऋतं नमः । ॐ महादेव सत्यं मे गोपाय\\
श्रद्धां धर्मे महादेवस्य पत्न्यै ऋतं स्वाहा । ॐ पशुपते पाशं मे गो\\
पाय भोक्तृत्वं भोग्ये पशुपतये देवाय सत्यं नमः । ॐ पशुपते\\
पाशं मे गोपाय भोक्तृत्वं भोग्ये पशुपतये देवाय सत्यं स्वाहा \textbar{}\\
ॐ पशुपते पाशं मे गोपाय भोक्तृत्वं भोग्ये पशुपतेर्देवस्य परन्यै\\
सत्यं नमः । ॐ पशुपते पाशं मे गोपाय भोक्तृत्व भोग्ये\\
पशुपतेर्देवस्य पत्न्यै सत्यं स्वाहा । ॐ शिवाय सत्यं नमः । ॐ शि.\\
वाय सत्यं स्वाहा ।\\
एवं शिवादिहोतव्यं विरिञ्यन्तं च पूर्ववत् ।\\
विरिध्ध्यन्त {[}अध्याद्यं{]}{]} पुराप्रोक्तं सृष्टिमार्गेण सुव्रताः ॥\\
पुनः पशुपतेः पत्नीं तथा पशुपति क्रमात् ।\\
सम्पूज्य पूर्ववन् मन्त्रैहोतग्यं वै क्रमेण च ॥\\
चर्वन्तमाज्यपूर्व च समिदन्तं समाहितः ।\\
ॐ शर्व }{घरां}{ मे छिन्धि, घ्राणे गन्धं छिन्धि, मेऽघं जहि । भूः\\
स्वाहा \textbar{} भुवः स्वाहा \textbar{} स्वः स्वाहा \textbar{} भूर्भुवः
स्वाहा \textbar{}\\
एवं पृथक् पृथक् हुत्वा केवलेन घृतेन च ॥\\
सहस्रं वा- तदर्षे वा- शतमष्टोत्तरं तु वा- ।\\
(१) पशुपत्यन्तमाज्येन शतमष्टोत्तरं पृथक् ॥\\
प्राणादिभ्यश्च जुहुयात् घृतेनैव तु केवलम् ।\\
ॐ प्राणेनिविष्टोऽमृतं जुहोमि शिवो मा विशाप्रदाहाय प्राणाय\\
स्वाहा \textbar{} प्राणाधिपतये रुद्राय वृषान्तकाय स्वाहा । ॐभू. स्वाहा
।

% \begin{center}\rule{0.5\linewidth}{0.5pt}\end{center}

( १ ) विरजा च घृतेनैवेति मुद्रित लिङ्गपुराणे पाठः । विरजा = तत्संज्ञक
दीक्षाम-\\
न्त्रैरिति लिङ्गपुराणठीका \textbar{}}

{३६६ वीरामेत्रोदयस्य श्राद्धप्रकाशे-\\
~\\
ॐ भुवः स्वाहा \textbar{} स्वः स्वाहा \textbar{} भूर्भुवः स्वः स्वाहा ।\\
एवं क्रमेण जुहुयात श्राद्धोकं च यथाक्रमम् ।\\
सप्तमेऽहनि ( १ ) विप्रेन्द्रान् आद्धार्हान विप्र योजयेत् ॥\\
सर्वेषां चैव विप्राणा वस्त्राभरणसंयुतम् ।\\
वाहन शयनं ( २) कांस्यमासनादि च भाजनम् ॥\\
हैमं वै राजतं धेनुं तिलक्षेत्रं च वै गृहम् ।\\
दासीदासगणं चैव दातव्यं दक्षिणापि च ॥\\
पिण्डश पूर्ववदेयं पृथगुक्तप्रकारतः ।\\
ब्राह्मणानां सहस्रं च भोजयेच्च सदक्षिणम् ।\\
एकं वा- योगनिरतं ब्रह्मनिष्ठं जितेन्द्रियम् ॥\\
प्रत्यहं चैव रुद्रस्य महाचरुनिवेदनम् \textbar{}\\
विशेषमेतत् कथितमशेषं श्राद्धचोदितम् \textbar{}\textbar{}\\
मृते कुर्यान्न कुर्याद्वा जीवन मुक्तो यतः स्वयम् ।\\
नित्यं नैमित्तिक यत्तु कुर्याद्वा सन्त्यजेत वा- ॥\\
बान्धवेऽपि मृते तस्य नैषाशौचं विधीयते ।\\
सूतकं च }{न}{ सन्देहः स्नानमात्रेण शुद्ध्यति ।\\
पश्चाजाते कुमारे च स्वक्षेत्रे चात्मनो यदि ॥\\
तस्य }{भर्त्रे}{ प्रकर्तव्यं पुत्रोऽपि ब्रह्मबिद् भवेत् ।\\
कन्यका यदि सञ्जाता पश्चात्तस्य महात्मनः ॥\\
एकपर्णा इवाज्ञेया अपर्णा इव सुव्रता ।\\
भवत्येव }{न}{ संदेहस्तस्याश्चान्वयजा अपि ॥\\
मुच्यन्ते नह्यत्र सन्देहः पितरो नरकादपि ।\\
मुच्यन्ते (३) सर्वकर्माणो मातरः पितरस्तथा ॥\\
काळङ्गते द्विजे भूमौ खनेद्वापि दहेत वा- ।\\
पुत्रकृत्यमशेषं वा- }{कृत्वा}{ दोषो }{न}{ विद्यत ॥\\
कर्मणा चोत्तरेणैव गतिरस्य महात्मनः ।\\
ब्रह्मणा कथितं भर्त्रे मुनीनां भावितात्मनाम् ॥

% \begin{center}\rule{0.5\linewidth}{0.5pt}\end{center}

{( १ ) योगीन्द्रान् श्राद्धाहानपि भोजयेत् । }{इति}{ मु० लि० पु० पाठः\\
( २ ) यानं कास्यताम्रादिभाजनम् । }{इति}{ मु० लि० पाठः ।\\
( ३ ) कर्मणानेनेति मु० कि० पाठः ।\\
१

{ जीवच्छ्राद्धनिर्णयः । ३६७\\
पुनः सनत्कुमाराय कथित तेन धीमता ॥\\
कृष्णद्वैपायनायैव कथितं ब्रह्मसूनुना ।\\
प्रसादात्तस्य देवस्य वेदव्यासस्य धीमतः ॥\\
ज्ञातं मया कृतं चैव नियोगादेव तस्य तु ।\\
एतद्वः कथित सबै रहस्य सर्वसिद्धिदम् ॥\\
नैव दुष्टाय दातव्यं }{न}{ चाभक्ताय सुव्रताः ।\\
अत्र कानिचित पदानि व्याख्यायन्ते \textbar{}\textbar{}\\
वक्ष्ये इति=सुतः श्रोतॄन् प्रत्याह । श्राद्धमार्गेति ।
जीवच्छ्राद्धाद्यधि\\
कारप्रकारम् । विशेष--- देशकालादिरूपम् । मृत्युकाले = मरणकाले समास.\\
\textbar{} जीवनेवेति=सत्वशुद्धिद्वारेण तत्वज्ञानलाभात् मुच्यत एवेत्यर्थः
।\\
अप्रत इति । मध्यस्थण्डिलादच्यांर्चनार्थात् प्राग्दिगेव कुण्ड
स्थण्डिलं\\
वा- होमार्थ कार्यम् । एतैः=वस्थ माणब्रह्मादिमन्त्रैः । स्थण्डिले
ब्रह्मादन्ि\\
सम्पूज्य तैरेव समिधादीन् जुहुयात् । आत्मनोवृत्येति । आत्मस्थानि\\
तत्वानि भूतानि च तत्तन्मन्त्रप्रकाशितानि एकैकशः समुद्धृत्य पृथक्\\
कृतानीति भावयित्वा तां तां देवतामुद्दिश्य जुहुयात् । ततस्तत्तत्तत्वं\\
शुद्धं भावयेत् । अत्र नमोन्तः पूजायां, स्वाहान्तो होमे मन्त्रः । तत्र\\
ब्रह्मादिप्रथमाष्टके तत्तद्देवानां पूजाहोमौ । एवं सृष्टिक्रमेण
द्विचतुर्वि\\
शतिमन्त्रानुत्का संहारे तद्विपरीतताप्रदर्शनार्थे प्रथमाष्टकप्रान्तम.\\
न्त्रमाद्यत्वेनोदाहरति । ॐ शिवाय सत्यामेत्यादि । एवमिति । सृष्टिक्रमेण\\
प्रागुक्तविरिञ्चयादिशिवान्तदेवताष्टकमन्त्रेषु पुनः संहारक्रमेण ॐ\\
शिवाय सत्यं स्वाहेत्यादि ब्रह्मणे स्वाहेत्यन्तम् । तथा पशुपतिप\\
म्यादिशर्वान्तं उक्तक्रमेण सम्पूज्य घृतचतु [ रु] समित्क्रमात्
प्रत्ये.\\
कं होतव्यम् । ॐ शर्वेत्यादिकेषलाज्यहोममन्त्राः । तत्र शर्व धरामिति\\
वाक्यत्रयान्ते भुवः स्वाहेति तृतीयो भूर्भुवः स्वाहेति चतुर्थो मन्त्रः
।\\
एवं-\/- पृथक् पृथक् । अथ भूः स्वाहेत्यादिभिरूहितैः चतुर्भिर्मन्त्रैः
होमः\\
तथा विराञ्चमन्त्रैश्च तथा प्राणे निविष्ट इत्यादि षण्मन्त्रैश्च
श्राद्धोक्तः\\
पितृपितामहप्रपितामहोद्देश्यको होमश्च । एवं सप्ताहं प्रत्यहं हुत्वा स\\
तमे दिने भवादितत्वोद्देशेन विप्रानभ्यर्च्य शर्वादिभ्योऽष्टौ पिण्डा\\
देयाः । एवं कृते जीवच्छ्राद्धे स्वबान्धवे मृते नाशौचं }{न}{ च सुतकम् ।\\
तस्मादुत्पन्नः पुत्रोऽपि पित्रादिना जातकर्मादिना संस्कार्यः स च

{३६८ वीरमित्रोदयस्य श्राद्धप्रकाशे-\\
शानी भवेत् । एवं दुहिता च । तथा चैतत् सन्ततौ योगिनो जाय-\\
त्त }{इति}{ ।\\
आदिपुराणे ।\\
देशकाल धनश्रद्धा व्यवसायसमुच्छ्रये ।\\
जीवते चाप्यजीवाय दद्यात् श्राद्धं स्वयं नरः ॥\\
वाशब्दाच्छ्राद्धकर्त्रन्तराभावे स्वयमेव स्वस्य श्राद्धं कर्तव्यमिति ।\\
तथा ।\\
कृतोपवासः सुस्नातस्त्रयोदश्यां समाहितः ।\\
कर्तारमथ भोकार विष्णुं सर्वेश्वरं यजेत् ।।\\
जले स्थलेऽम्बरे मूर्ती कलशे पुष्करे रवै। ।\\
चन्द्राग्निगुरुगोविप्रमातापितृषु सर्वगम् \textbar{}\textbar{}\\
सदक्षिणाञ्च सतिलास्तिस्रस्तु जलधेनवः ।\\
निवेदयेत् पितृभ्यश्च तदप्रेषु समाहितः ।।\\
एवं विष्णुं सम्पूज्य पितुरुदेशेन तिस्रो जलधेनूर्दद्यात् कैर्मन्त्रै.\\
रित्यपेक्षित आह-\\
सोमायत्वा पितृमते स्वधा नम }{इति}{ ब्रुवन् ।\\
अमये कव्यवाहनाय स्वधा नम }{इति}{ स्मरन् ॥\\
दक्षिणे तु निदध्याच्च तृतीयां दक्षिणायुताम् ।\\
यमायाङ्गिरसे वाथ स्वधा नम }{इति}{ स्मरन् ॥\\
तयोर्मध्ये तु निक्षिप्य विप्रान् पञ्चोपवेशयेत् ।\\
अग्नये कव्यवाहनाय स्वधा नम इतीमं मन्त्रं स्मरन् "दक्षिणेन\\
द्वितीयां निद्ध्यात्" इत्युक्तेः, अर्थात् प्रथमाबा उत्तरतो निधानं
सिद्धं\\
भवति । तृतीयान्तु तयोः पूर्वोक्तयोर्धेन्वोर्मध्ये निक्षिप्य
पञ्चविप्रानु-\\
पवेशयेत् }{इति}{ योजना ।\\
आवाहनादिना पूर्वे विश्वेदेवान् प्रपूज्य च ।\\
वसुभ्यस्त्वामहं विप्र रुद्रेभ्यस्त्वामहं ततः ॥\\
सूर्येभ्यस्त्वामहं विप्र भोजयानीति तान् वदेत् ।\\
आवाहनादिकं }{भर्त्रे}{ कुर्याच्च पितृकर्मवत् \textbar{}\textbar{}\\
सौम्यधेनुस्ततो देया वासवाय द्विजाय तु ।\\
आग्नेयौँ चाथ रौद्राय याम्यां सूर्यद्विजाय तु \textbar{}\textbar{}}

{ जीवछ्राद्धनिर्णयः । ३६९\\
विश्वेभ्यश्चाथ देवेभ्यस्तिलपात्रं निवेदयेत् \textbar{}\\
वासवाय = वसुभ्यस्वामहं भोजयानीत्येचं निमन्त्रिताय । एवं-\/-\/-\\
मप्रेऽपि -\\
स्वस्त्युदकमक्षय्यं जलं दत्वा च तान् द्विजान् ।\\
विसर्जयेत् स्मरन् विष्णुदेवमष्टाक्षरं विभुम् ॥\\
ततः कामकुलेशानं निशि नारायणं स्मरेत् ।\\
एतत् भर्त्रे कृष्णत्रयोदश्यां कृत्वा अपरदिने यत् कर्तव्यं
तदाह-\\
चतुर्दश्यामथो गच्छेत् यथाप्राप्तां सरिद्वराम् ।\\
पूर्वेण विप्रः सौम्बेन राजा वैश्योऽपरेण च ॥\\
दक्षिणेन तथा शूद्रो मार्गेण विकिरन् यवान् ।\\
सौम्येन=उत्तरेण ।\\
वस्त्राणि लोहदण्डांश्च जितं (१) तत }{इति}{ स्मरन् ।\\
दक्षिणाभिमुखो वह्नि ज्वालयेतत्र च स्वयम् ॥\\
पञ्चाशता कुशैर्ब्राह्मीं }{कृत्वा}{ प्रतिकृतिं दहेत् ।\\
तत्र = सरितीरे । स्वयमित्यन्यनिवृत्त्यर्थम् । प्रतिकृति = शरीराकृतिः ।\\
}{कृत्वा}{ श्माशानिकं हौत्रं पूर्णाहुत्यन्तमेव हि ।\\
पूर्णाहुत्यन्तम्माशानिकं हौत्रं }{कृत्वा}{ प्रतिकृतिं संदहेदिति सम्ब\\
न्धः । तत्र चायं प्रकारः, उत्सन्नाग्निना पृष्ठोदिविपक्षोत्पादितेऽझौ\\
स्वगृह्योक्तविधिना पूर्णाहुत्यन्तं कृत्वाग्निप्रदानमन्त्रेणाज्यं हुत्वा
प्रति-\\
कृतिदाहः कार्यः । पृष्ठोदिवि पक्षः कात्यायनेनोक्तः ।\\
अन्वग्निरिति मन्त्रेण प्रामाशिं तु समाहरेत् ।\\
पृष्टोदिवीति चादध्यात् सावित्र्या ज्वालयेदथ \textbar{}\textbar{}\\
तत्सवितुर्वरेण्यं तु विश्वानीति स्मृतोऽपरः ।\\
प्रामाग्निः = लौकिकाभिः । स च श्रोत्रियागारादाहर्तव्यः \textbar{}
सावित्र्या=\\
तत्सवितुर्वरेण्यमित्यनया \textbar{} विश्वानीति च योऽपरो मन्त्रस्तेन च
ज्वा-\\
लयेत् । अकृताग्निपरिग्रहेण तु इत्थं कर्त्तव्यमित्याह-\\
निरग्निरथवा भूमिं यमं रुद्र च संस्मरन् ।\\
हुत्वा प्राजाह के स्थाने पश्चाद् दाहापयेच्च ताम् ॥\\
ताम् = पूर्वोकां कुशप्रतिकृतिम् ।\\
( १ ) त- }{इति}{ संस्मरन् । }{इति}{ मयूखोद्धृत पाठः ।\\
ची० मि ४७

{३७० वीरमित्रोदयस्य श्राद्धप्रकाशे-\\
श्रपयेच्चापरे वही मुद्द्रमिश्रं चरुं ततः ।\\
तिलतण्डुलमिश्रं तु द्वितीयं सपवित्रकम् \textbar{}\textbar{}\\
अपरे = प्रतिकृतिदाहसाधनीभूतादन्यस्मिन् \textbar{} ततत्र चरुश्रवणा\\
र्थमपि पूर्ववदग्निरुत्पादनीयः । सपवित्रकमिति= पवित्रपाणिना कर्तव्यमि\\
त्यर्थः । शुतचर्वभिघारणानन्तरकर्तव्यमाह-\\
मधुक्षीरघृताम्भोभिः पूरयेत्कर्षुकात्रयम् ।\\
तदुपान्ते समुद्रानि पात्राणि त्रीणि पुरयेत् ॥\\
ॐ पृथिव्यै नमस्तुभ्यं }{इति}{ चैकं निवेदयेत् ।\\
ॐ यमाय नमश्चेति द्वितीयं तदनन्तरम् ॥\\
ॐ नमश्चाथ रुद्राय श्मशानपतये तथा ।\\
ततो दीप्तं समिद्धानि भूमौ प्रकृतिदाहकम् ॥\\
क्रव्यादवहिततायै भूम्यै नम }{इति}{ स्मरन् ।\\
क्षीराकं जलकुम्भं तु विकिरेत् तत्प्रशान्तये ॥\\
नाभिमात्रं ततस्तोयं प्रविश्य यमदिङ्मुखः ।\\
सप्तभ्यो यमसंशेभ्यो दद्यात् सप्त जलाञ्जलीन् ॥\\
ॐ नमश्चाथ रुद्राय श्मशानपतये स्मरन् ।\\
अमुकामुकगोत्रैतत्तुभ्यमस्तु तिलोदकम् ॥\\
सूक्ष्मदेहैष ते पिण्डस्त्वर्धपुण्य {[}{[}प{]} सुगन्धिमान् ।\\
धूपो दीपो वलिर्वासस्तवैषा तृप्तिरक्षया ॥\\
दश पिण्डान् ततो दत्वा विष्णुं सौम्यमुखं स्मरन् ।\\
निरुष्मणस्तु तत्तोयं नाभिमात्रं प्रविश्य च ॥\\
प्रक्षिपेत् पूर्णकुम्भेन जलमध्ये पृथक् पृथक् ॥\\
प्रदद्यात् पञ्चपञ्चान्यत् कुम्भांश्चाथ जलाञ्जलीन् ।\\
द्वारोपान्ते गृहे चाथ क्षीरं तोयं च निक्षिपेत् ॥\\
जीवात्र स्वाहि दुग्धं च पिवेदं चाप्यनुस्मरेत् ।\\
याम्योन्मुखेषु दर्भेषु स्वपेत् पश्चादुदङ्मुखः \textbar{}\textbar{}\\
अमावास्यां प्रकुर्याच जीवच्छ्राद्धमतः परम् ।\\
सृतान्नमांसदधिभिः पूरयेत् कर्षुकाजयम \textbar{}\textbar{}\\
कुर्याच मासिकं मालि सपिण्डीकरणं ततः ।\\
,

{ सन्यासाङ्गश्राद्धनिर्णयः । ३७१\\
अशौचान्ते यतः }{भर्त्रे}{ आत्मनो वापरस्य च ॥\\
कुर्यादस्थिरतां ज्ञात्वा भक्त्या रोग्य धनायुषाम् ।\\
}{इति}{ जीवच्छाद्धनिर्णयः ।\\
अथ संन्यासाश्राद्धनिर्णयः ।\\
तत्र संन्यासमधिकृत्य -\\
बौधायनसूत्रे ।\\
}{आदौ}{ अग्निमान् पार्वणविधिनाष्टौ श्राद्धानि कुर्यात् । पूर्णमास्या-\\
ममावास्यायां वा देवश्राद्धम्, ऋषिश्राद्धं, दिव्यश्राद्धं,
मानुषश्राद्धं,\\
भूतश्राद्धं, पितृश्राद्धं, मातृश्राद्धम्, आत्मनः श्राद्धं चेति ।\\
देवश्राद्धे देवतात्रय ब्रह्मविष्णुमहेश्वराः । ऋषिश्राद्धे देवतात्रयं
देव-\\
र्षिब्रह्मर्षिक्षत्रर्षयः । दिव्यश्राद्धे देवतात्र्यं वसुरुद्रादित्याः
\textbar{} मनुष्यश्राद्धे दे\\
वतात्रयं सनकसनन्दनसनातनाः । भूतश्राद्धे देवतात्रयं पृथिव्यादीनि,\\
[भूतानि] चक्षुरादीनि करणानि, चतुर्विधो भूतग्रामः॥ पितृश्राद्धे
देवतात्रयं\\
पितृपितामहप्रपितामहाः, मातामहमातुः पितामहमातुःप्रपितामहाश्च ।\\
मातृश्राद्धे देवतात्रयं मातृपितामहीप्रपितामह्यः । आत्मश्राद्धे
देवतात्रयं,\\
आत्मपितृपितामहाः ।\\
अथातः शौनकप्रोक्तं संन्यासविधि व्याख्यास्यामः । पूर्वेद्युर्नान्दी\\
मुखश्राद्धं कुर्यात् । देवर्षिदिव्यमनुष्यभूतपितृ[मातृ] आत्मनश्च
पृथक्\\
पृथक् पिण्डदानैर्युग्मैर्ब्राह्मणैः पिण्डोदकं दद्यात् । देवश्राद्धे
देवता\\
त्रयं [ ब्रह्मविष्णुमहेश्वराः, पिण्डत्रयं दद्यात् । ऋषिश्राद्धे
देवतात्रय, {]}\\
देवर्षिक्षत्रार्षमनुष्यर्षयः, पिण्डत्रयं दद्यात् । दिव्यश्राद्धे
देवता.\\
त्रयं वसुरुद्रादित्याः पिण्डत्रयं दद्यात् । मनुष्यश्राद्धे देवतात्रय\\
सनकसनन्दनसनातना. पिण्डत्रयं दद्यात् । भूतश्राद्धे देवतात्रयं पृ.\\
थिव्यादीनि भूतानि, चक्षुरादीनि करणानि, चतुर्विधो भूतग्रामः,\\
{[} पिण्डत्रयं दद्यात् । पितृश्राद्धे देवताषट्कम्, पितृपितामहप्रपि\\
तामहाः, मातामहप्रमातामहवृद्धप्रमातामहाञ्च पिण्डषट्कं दद्या-\\
त् मातृश्राद्धे देवतात्रयं मातृपितामहीप्रपितामह्यः {]} पिण्डत्रयं\\
दद्यात् । आत्मश्राद्धे देवतात्रयं आत्मपितृपितामहाः पिण्डत्रयं\\
दद्यात् । नामगोत्रसम्बधात्पिण्डोदकं दद्यात्, अनन्तरं पुण्याहं वाच-\\
येदिति । इति संन्यासाङ्गश्राद्धनिर्णयः ।

{३७२ वीरमित्रोदयस्य श्राद्धप्रकाशे-\\
प्रत्याशं परिवर्द्धतेऽर्थिजनता दैन्यान्धकारापहे\\
श्रीमद्वीर मृगेन्द्रदानजलाधर्यद्वक्त्रचन्द्रोदये ॥\\
राजादोशित मित्रमिश्रविदुषस्तस्योक्तिभिर्निर्मिते\\
ग्रन्थेऽस्मिन् खलु निर्मितः सुरुचिर: श्राद्धप्रकाशोऽगमत् ॥

{\\
}{इति}{ श्रीमत्सकलसामन्तचक्रचूड़ामणिमरीचिमञ्जरीनीराजितचर-\\
णकमलश्रीमन्महाराजाधिराज प्रतापरुद्रतनुजश्रीमन्महाराजम-\\
धुकरसाहसूनुचतुरुदाघवलयवसुन्धराहृदय पुण्डरीकविका.\\
सदिनकरश्रीमन्महाराज श्रीवीरसिंहदेवोद्योजित श्रीहंसप•\\
ण्डितात्मजश्रीपरशुराममिश्रसूनुस कलाविद्यापारावार-\\
पारीणजगहारिद्र्य महागजपारीन्द्र विद्वज्जनजीवातु\\
श्रीमन्मित्रमित्रकृते वीरमित्रोदयामिघनिबन्धे\\
श्राद्धप्रकाशः " समाप्तः ।\\


{ सर्वविधपुस्तक प्राप्तिस्थानम्-\\
}{जयकृष्णदास - हरिदास गुप्त:-}{\\
चौखम्बा संस्कृत सीरिज आफिस,\\
 विद्याविलास प्रेस, बनारस सिटी ।

 शुद्धिपत्रम्

{अ० शु० पृ० पं० अ०\\
निश्रेयसार्थम् नि.श्रेयसार्थम् ३ १८ तण्डुषिद्ध तण्डुषिद्ध '' २९\\
प्रकरणान्तरस्था प्रकरणान्तरन्याया मुर्मुरुः मुर्मुरुः ५५ १४\\
त्वादिति दिति ५ २० स्तरुणैः स्तरुणैः ५६ १४\\
प्रत्युत्तरप्राषा पत्युरप्राधान्यापत्ते देवल देवल '' १६\\
न्यापत्तेः ९ २२ श्रेत्रिय श्रेत्रिय ५९ २४\\
हव्यकव्ययोः हव्यकव्ययोः १० ११ शान्ता शान्ता ६० २८\\
बहीनामित्यस्य बहोनामित्यस्य '' ३२ माह माह ६१ २\\
पिण्डान्वहार्यकं पिण्डान्वाहार्थकं १४ २ च=वं वरवं '' ५\\
प्रमृत्यर्थम् प्रभूतार्थम् '' ७ वेदितो योनि वेदितो योनि ६७ २९\\
घामुष्यायणस्य व्यामुष्यायणस्य २० ६ निरताः ये निरता ये ७० ९\\
प्रतिमहेभ्यः प्रपितामहेभ्यः ' २५ वरणवृत्या वरया वृत्या ७१ १३\\
प्रयितामहेभ्यो प्रपितामहेभ्यो '' ३१ प्रस्त प्रशस्त ७३ २८

{पुरुरवः पुरुरवः २३ २४ तीनातिक्रमेण सीनातिक्रमेण ७८ १०\\
श्राद्धं श्राद्धे २४ १९ भोज्या भोज्याः ,, २१

{निषेधात् । निषेधात् ,, ३२\\
वैश्यदेविके वैश्वदेविके २६ १४\\
य च थे च ,, ३२\\
प्रधान प्रधौन ३१ १४

लक्षणाप्यदोषः । लक्षणायामप्यदोष । , १७

शुक्लानुवृत्तितः शुल्कानुवृत्तित: ३२ ३२

और्धदेहिकम् और्ध्वदेहिकम् ३४ ११ 

{त्वष्टा त्वष्टा ३५ ३१\\
ग्राह्य वर्ज्य ३९ १

{सम्बत्सरं संवत्सरं ५२ १८ दवेल\\
अत्रादिभिर्मास अत्रौषध्यादिभिर्मासं, २३ वृत्रिः ब्रह्म\\
२० ६\\
३१\\
३३\\
२३ २४\\
पुरुरवः\\
श्राद्धे\\
~\\
दुरतः\\
ग्राम्याणा\\
सम्बत्सरं\\
शुद्धिपत्रम् ।\\
गवयं\\
भवेदत्तं\\
पितृर्णां\\
एणे

{\\
५ २०\\
निषेधात् ।\\
वैश्यदेविके\\
य च\\
प्राधान\\
~\\
,\\
दूरत\\
''१\\
प्राम्याण\\
संवत्सरं\\
गावयं\\
भवेद्दतं\\
पितॄणां\\
ऐणे\\
''\\
२४ १९\\
७, ३२\\
~\\
~\\
२६ १४\\
१\\
३४ ११\\
~\\
३९ १\\
~\\
,, ३१ पङ्खोवे\\
ह्मणः\\
दवेल:\\
~\\
~\\
~\\
~\\
वेदिता योनि\\
निरताः ये\\
चरण\\
प्रस्त\\
३२ गोहपस्या\\
२१\\
३१ १४ क्कादयोऽपीति\\
५३ १५\\
''\\
५४\\
१७ व्यक्तिरित\\
श्वक्रीडचोप\\
गणाभ्यतरगः\\
गणगणिका\\
प्रकीर्तित\\
शु●\\
तण्डु ऋसिद्ध\\
मुमुरः\\
तरुगैः\\
देवल'\\
श्रोत्रियं\\
शान्ताः\\
माहू\\
वदिताऽयोनि\\
~\\
७७\\
वरया वृत्या\\
७१ १३\\
प्रशस्त\\
७३ २८\\
तीनातिक्रमणेन सीनातिक्रमणेन ७८ १०\\
भोज्या भोज्याः\\
श्राद्धानिमन्त्रणीय श्राद्धनिमन्त्रणीय ८० २३\\
२१\\
३५\\
स्थूला: भृश\\
स्थूला मृशं\\
८२ ६\\
गार्हपत्या\\
८४ १२\\
कादयोऽपीति\\
८५ १३\\
व्यतिरिक्त\\
२७\\
३९\\
श्वक्रीड़ई खोप\\
૨૮ ૧૪\\
९०\\
९\\
९५ १२\\
२९\\
गणाभ्यन्तरगः\\
गणकगणिका\\
प्रकीर्तितः\\
~\\
पं\\
२९\\
श्थ्\\
५५ १४\\
५६ १५\\
१६\\
५९ २४\\
६० २८\\
६९\\
२\\
वृतिर्ब्रह्म\\
पडावे\\
ब्राह्मणः\\
रिधू\\
वादो\\
ऊर्ध्व\\
पृ०\\
~\\
''\\
६७ ३१\\
''१\\
९६ २\\
५३\\
१०३ ३२\\
१०४ २४\\
१\\
२२ रिघ्र\\
.२ वाहो\\
५ ऊर्ध्व\\
''''\\
१\\
,, १५ णपूर्वकालकृत्यम् णीयब्राह्मणसंख्या १०७\\
,,२० । वैजये\\
वर्जये\\
११३ १८\\
१९\\
३१\\
●६ १५\\
२९

{१०\\
अ०\\
व्यावृतो\\
देशे कस्य\\
तेऽचं\\
श्वरुपिण\\
भुखता\\
शाि\\
दांनं\\
पुराणेः\\
गच्छतिः\\
मृदु\\
}{इति}{ विपुलं\\
षिकां\\
सर्वाः यथो\\
पत्रर्ण\\
स्तन्वेन\\
दार्न\\
शु०\\
पितॄगा\\
दर्भा\\
सौर्षण\\
व्यावृतौ\\
देशिकस्य\\
तेऽर्चा\\
लप एवं-\/-\\
पात्राप्युप\\
विनियोज्या\\
भवन्वि\\
संस्रवाणा\\
इत्पृचा\\
प्राप्नुस्वर्यो\\
दश:

{\\
स्वरूपिण\\
भुजतां\\
शाङ्ग\\
करातेः\\
तृणे }{न}{\\
सर्पिभ्यामिति सर्पिभ्यमिति १६१ १४\\
करोद्धृतः\\
तृणेन\\
दान\\
पुराणे\\
गच्छति\\
मृदु\\
इतिविपुलं\\
षिक्तो\\
सर्वा यथो\\
पत्रोर्ण\\
स्तम्बेन\\
दानं\\
श्राद्धो\\
यानि\\
पृ० पंढ\\
११९ १७\\
पितृणां\\
दर्भा:\\
सौवर्ण\\
अ०\\
स्मृत्यन्तरे\\
१२४ ५ अवनेयनित्य\\
१३४ ५ सव्यनाद्धरणं\\
१३५ २८ निर्वस्य\\
१३७ १४ कश्रिदंस्त्र\\
१४४ १९\\
क्षय्यादक\\
१५० ३१ \textbar{} पवित्राच्छादि\\
१५७\\
श्राद्धा\\
यनि\\
{[} गोमूत्रेण {]} {[} उपलिप्तायां गोमू\\
त्रण {]}\\
१८६\\
२००\\
शुद्धिपत्रम् \textbar{}\\
१६२ १७ नियुञ्जीत\\
१६८ २८ वृवाहे\\
१७० १७\\
१७१ २\\
१७२ ३१\\
१७४ २२\\
"8\\
मानामहानां\\
दर्घा\\
बूयुः\\
दवि\\
१७९ २५\\
वीरे\\
२९\\
चतुरखे\\
३२ इच्छाया\\
"\\
१८० १० पितमहा\\
१८१ १ होलाका\\
२. श्रद्ध\\
"\\
"\\
२९\\
"3\\
तेषुः\\
व्यवस्थिति\\
१४\\
२०८ २३\\
२०९ १५\\
२७\\
}{इति}{ ।\\
९ लोपोष\\
कार्याः\\
पृथक् पृथक्\\
वृद्धया\\
तुस्यतः\\
लोप एवं-\/-\\
पात्राण्युप\\
बधूचा\\
विनियोज्याः २१३ ११ नेका\\
भवन्त्वि\\
29\\
संखवाणां\\
२१८\\
इल्यू चो\\
२३३\\
मेशं\\
प्राङ्मुखेभ्यो २४० २०\\
२४४ ३२ हिने\\
देश:\\
२० मातृ'\\
तयाद्दता\\
विहीनाया\\
•\\
शु०\\
339 "\\
स्मृत्यन्तर\\
अवनेजयस्य २५३ ३२\\
सम्येनोद्धरणं\\
निर्वर्य\\
कश्चिदशस्त्र\\
क्षय्योदके\\
पवित्राच्छादितेषु\\
व्यवस्थित\\
नियुञ्जीत\\
पूर्वाह्णे\\
मातामहानां\\
दद्या\\
बूयुः\\
दधि\\
वीर ?\\
चतुरस्रं\\
इच्छया\\
पितामहा\\
होलका\\
कार्या\\
}{इति}{\\
लोपोऽस\\
पृथक्नव\\
वृध्या\\
तुल्यतः\\
बन्धूना\\
नेता\\
भ्रातृ\\
तयाहताः\\
विहीनायाः\\
मंश\\
हनि\\
पृ० पं०\\
"\\
१९\\
३५६ ३\\
२७० ३१\\
२७४ १८\\
२७७ ९\\
२८८ २१\\
३१\\
39\\
२९१२०\\
२९५ १९\\
३०० १३\\
३०५ ४\\
४\\
२२\\
\$2\\
३०६\\
"" "\\
१९\\
"\\
३१० ३१\\
३११ ६\\
३१५ १३\\
३१७ २१\\
३१९ २८\\
३३४ १८\\
३३९ २५\\
३४१ २८\\
३४५ १६\\
३४७ ३०\\
३५२ १५\\
३५३ ११\\
१५\\
२४\\
""\\
""\\
३५७ ७\\
३५८ ८\\
८

{काशी संस्कृत सीरिज़ प्रन्थमाला ।\\
८० (१) पाणिनिव्याकरणे वादरत्नम् । न्यायव्याकरणाचार्य मीमांसक शिरोमणि
पं०\\
सूर्यनारायणशुक्लेन विरचितम् । ( तत्र च न्यास परिष्कारपरिशिष्ट भेदेन
प्रकरण\\
त्रयम् यत्र पञ्चन्यून सार्धशतसूत्रेषु १४५ महताटोपेन न्यासाविचारिताः
उत्र\\
प्रथमं प्रकरणं मुद्रितं प्रथमो भागः ।\\
{[} व्या० वि १० {]} रु० १-८\\
६०१-४\\
८० (२) पाणिनीव्याकरणे वादरत्नम् । द्वितीयोभागः\\
८१ गणितकौमुदी । [ हिन्दीभाषाटीकासहिता ]
प्रथमापरीक्षापाव्यनिर्द्धारितगणित-\\
संग्रहपुस्तकम् ।\\
{[} गणितविभागे १{]} रु० ०-६\\
छन्दः कौमुदी । [ हिन्दीभाषाटीकासहिता ] प्रथमापरीक्षापाट्य
निर्द्धारितछन्द-\\
संग्रह पुस्तकम् ।\\
{[} छन्दोविभागे ३{]} रु० ०-६\\
३ योगदर्शनम् । महर्षिप्रवरपतञ्जलिप्रणीतम् । तत्र ( १ ) भोजवृत्तिः
{[}२{]} भावा-\\
गणेशवृत्तिः, [३] नागोजीभट्टवृत्तिः, [४] मणिप्रभा, [५]
योगचन्द्रिका,\\
[६] योगसुधाकराख्य टीका षट्कसमेतम् । सटिप्पण । [ योग० विभागे १]
रु० २==०\\
८४ रघुवंशमहाकाव्यम् । श्रीकालिदासविरचितम् । म० म० मल्लिनाथकृत\\
सञ्जीविनीटीका तथा परीक्षोपयोगि सुधाऽऽख्यव्याख्या सहितः । १ से ४ सर्गः\\
मूल्यम् रु० १४-० तथा १ से ५ सर्गः रु० १-८ तथा ६ से १० सर्गः । रु० १-८\\
८९ योगदर्शनम् । पं० श्रीबलदेवमिश्रकृत योगसूत्रप्रदीपिकाख्यव्याख्या
सहितम् ।\\
सटिप्पण । {[} योग वि० २{]} १-०\\
८६ काव्यमीमांसा । राजशेषरविरचिता । साहित्याचार्य पं० श्रीनारायणशाखि\\
खिस्ते कृत काव्यमीमांसा - चन्द्रिका टीका सहित [ १ से ५ अध्याय ]
प्रथमो-\\
{[} काव्यविभाग १३{]} रु०८\\
भागः ।\\
८७ भागानन्दनाटकम् \textbar{} श्रीहर्षदेवप्रणीतम् ।
काशीविश्वविद्यालयाध्यापकेन एम० पु०\\
साहित्याचार्य पण्डित श्री बलदेव उपाध्यायेन स्वप्रणीतया
भावार्थदीपिकाख्यया\\
' व्याख्यया समलङ्कृत्य वृहत् भूमिका हिन्दीभाषानुवादादिभिः सनाथीकृत्य\\
सम्पादितम् ।\\
{[} नाटक विभागे १{]}रु० १-४\\
८८ मेघदुतकाव्यम् । महाकवि श्रीकालीदास विरचितम् । मल्लिनाथकृत
सञ्जीविन्या,\\
चारित्रवर्द्धनाचार्य विरचित चारित्रवर्द्धिन्या तथा साहित्याचार्य पं०
श्री नारायण-\\
शास्त्री खिस्तेकृत भावप्रबोधिनीव्याख्या टिप्पण्या च सहितम् \textbar{}
{[} काव्य०४ {]} *०१-८\\
८९ जागदीशीव्यधिकरणम् । न्यायाचार्य श्रीशिवदत्तमिश्रविरचित
गंगाक्यव्याख्या\\
टिप्पणी सहितम् ।\\
( न्यायविभागे २ ) ३० २=०\\
९० काव्यकल्पलतावृत्तिः । श्रीअमरचन्द्रयतिनिर्मिता अरिसिंहकृतसूत्रसहिता
।\\
( अलङ्कारविभागे ४ ) ६० १-४\\
९१. वैयाकरणसिद्धान्त चन्द्रिका । श्रीरामाश्रमप्रणीता ।
श्रीसदानन्दकृत-सुबोधिन्या,\\
श्रीलो केशकरकृत - तत्वदीपिकया व्याख्यया च सहिता । पं०
श्रीनवकिशोरशास्त्रिणा-\\
निर्मितया चक्रधराख्य महत्या टिप्पण्या अव्ययार्थमालया
लिङ्गानुशासनप्रक्रियया\\
उणादिकोषेण च सहिता । (व्या० वि० ११) पूर्वार्द्धम् । रु०२-८ उत्तरार्द्ध
\textbar{} रु० २-८\\
९२ त्रिपुरा रहस्यम् ( महात्म्यखण्डम् ) सामन्ययोगशास्त्राचार्य
श्रीमुकुन्दलाल\\
शास्त्रिणा संशोधितम् । साहित्याचार्य खिस्ते- इत्युपाख्य पं०
श्रीनारायणशास्त्रिणा\\
निबद्धाम्यां भूमिकाऽध्यायानुक्रमणिकाभ्यां च सहितम्। (पुराणेतिहास वि०१)
रु० ५-०\\
९३ आपस्तम्बधर्मसूत्रम् । श्रीमद्धरदत्तमिश्र विरचितया उज्ज्वलाख्यया
वृत्या\\
संवलितम् ।\\
( कर्मकाण्डविभागे ७ ) रु० ४-०\\
९४ अवच्छेदकत्वनिरुक्तिः । श्रीजगदीशतलङ्कारकृताः । न्यायाचार्य
श्रीशिवदत्त-\\
मिश्रविरचित गंगाख्यव्याख्या टिप्पणी सहितः । ( न्याय वि० १३) ६० १-४

{ काशी संस्कृत सीरिज़ प्रन्थमाला\\
~\\
९५ संस्कारदीपकः । म० म० पण्डित श्री नित्यानन्दपन्त पर्वतीयविरचितः ।
संपूर्णः ।\\
( कर्मकाण्ड वि० ८ ) रु० ८--०\\
९६ वर्षकृत्यदीपकः । कालनिर्णयन्त्रतोद्यापनसहितः । म० म० पं०
नित्यानन्दपन्त\\
पर्वतीय विरचितः ।\\
( कर्मकाण्ड वि० ९) रु० ३-८\\
९७ श्रौतसूत्रम् । श्रीमन्महर्षि - लाट्यायनप्रणीतमग्निष्टोमान्तम् ।
(कर्म० वि० १०) रु० २-०\\
९८ नलचम्पू: अथवा दमयन्तीकथा । महाकविश्रीत्रिविक्रम भट्ट विरचिता ।
विषमपद-\\
प्रकाशाख्यव्याख्यया सहिता । भावबोधिनी टिप्पणी सहिता । (का० वि०१५)
रु०१-४\\
९९ श्रीब्रह्मसूत्रम् । श्रीभगवन्निम्बार्क महामुनीन्द्रविरचित
वेदान्तपारिजात सौरभा-\\
ख्यसूत्रवाक्याथन श्रीश्रीनिवासाचार्यचरणप्रणीत श्रीवेदान्तकौस्तुभभाष्येन
च\\
सनाथीकृतम् । ( श्रीनिम्बार्कभाष्यम्)\\
( वेदान्त वि० १०) रु०३-०\\
१०० वाग्वल्लभः । सर्वथाऽपि नवीनोऽपूर्वः प्रौढः परमोपयोगितया
नियतमुपादेयतमश्छ-\\
न्दोनिबन्धः, श्रीमता दैवज्ञाप्रेसरेणागममार्मिकेण कविपुङ्गवेन
दुःखभञ्जनविदुषा\\
विरचितः, तत्सुतेन बहुशास्त्रपारगेण कविचक्रवर्तिना महामहोपाध्यायेन
देवीप्रसाद\\
पण्डितप्रवरेण कृतया वरवर्णिन्या टीकयोपस्कृतः । (छन्दः शास्त्र वि० ४ )
६०२-८\\
१०१ सिद्धान्तलक्षणम् । श्रीजगदीशतर्कालङ्कारकृतम् । न्यायाचार्य
श्रीशिवदसमिश्र-\\
विरचित गंगाख्यव्याख्या टिप्पणी सहितम् । ( न्याय विभागे ४) ) रु० १ -८\\
१०२ वेदभाष्यभूमिकासंग्रहः । ( सायणाचार्यविरचितानां
स्ववेदभाष्यभूमिकानां\\
{[} वेद० वि० ५ {]} रु० २=८\\
१०३ माधवीयधातुवृत्तिः । श्रीमत्सायणाचार्यविरचिता । ( व्या० वि० १२) रु०
५--०\\
१०४ बौधायनधर्मसूत्रम् \textbar{} श्रीगोविन्दस्वामिप्रणीत विवरण समेतम्
\textbar{} {[}कर्म०वि० ११{]} रु० ४०,\\
१०५ ताण्डयमहाब्राह्मणम् । सायणाचार्यविरचितभाष्यसहितम् । (प्रथमोभागः )\\
( वेद वि० ६) रु० १०\\
संग्रह: ) ।\\
१०६ न्यायमञ्जरी । जयन्त भट्टकृता । न्याय - व्याकरणाचार्यण पं०
सूर्यनारायणशास्त्रिणा\\
कृतया टिप्पण्या समेता । द्वितीयोभागः ।\\
{[} म्या० वि० १५ {]} ६०३-०\\
१०७ शारदातिलकम् । श्रीमद्वाववभट्टकृत पदार्थादर्शटीकासहितम् ।\\
( तन्त्रशा० वि० १) ६०५-०\\
१०८ मन्त्रार्थदीपिका । म० म० श्रीशत्रुघ्नविरचिता । ( वेद० वि० ७ ) रु०
२०\\
१०९ शब्दशक्तिप्रकाशिका । श्रीमज्जगदीशतर्कालङ्कारविनिर्मिता ।
श्रीकृष्णकान्त-\\
विद्यावागीशकृतया कृष्णकान्तीटीकया श्रीमद्रामभद्रसिद्धान्तवागीशविरचितया\\
रामभद्रीटीकया च समलङ्कृता । न्यायाचार्य काव्यतीर्थ पं०
ढुण्ढिराजशास्त्रि.\\
कृतया छात्रोपयुक्तया विषमस्थल टिप्पण्या मुलका रिकार्थेन च सहिता ।\\
(न्या वि० १६) रु० ४-८\\
११० योगदर्शनम् ( पातअलदर्शनम्) भगवत्पतञ्जलिरचित, राघवानन्दसरस्वतीकृत-\\
"पातअकरहस्याण्य'' टिप्पनीयुकथा द्वादशदर्शनकानन पञ्चाननवाचस्पतिमिश्रवि.\\
रचितया "तत्त्ववैशारद्याव्याख्यया भूषितेन विज्ञानभिक्षुनिर्मित
"योगवात्तिक" -\\
समुद्भासितेन मधुपुरीयकापिल
मठस्थस्वामिहरिहरानन्दारण्यक्कृतभास्वतीवृत्या\\
सहितेन भगवच्छ्रीकृष्णद्वैपायनव्यासदेवोपज्ञ-"सांख्यप्रवचन"
माध्येणोद्योतिः\\
तम्, प्रदेशविशेषेषु श्रीमन्माध्वसम्प्रदायाचार्य्यं -
दार्शनिकसार्वभौम-साहित्य.\\
दर्शनाद्याचार्य - तकरत्न - न्यायरत्न- गोस्वामिदामोदरशास्त्रिणा विहितया
टिप्पन्या\\
``पातञ्जलप्रभाश्नामिकथा भूमिकथा च संवलितम् एतदीयसंशोधनसम्पादन-\\
कर्मीकृतं च ।\\
( योगशास्त्रविभागे ३) रु० ५-\/-०\\
प्राप्तिस्थानम् - चौखम्बा संस्कृत पुस्तकालय, बनारस सिटी ।\\
~\\
·

\end{document}