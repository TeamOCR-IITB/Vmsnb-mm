\documentclass[11pt, openany]{book}
\usepackage[text={4.65in,7.45in}, centering, includefoot]{geometry}
\usepackage[table, x11names]{xcolor}
\usepackage{fontspec,realscripts}
\usepackage{polyglossia}
\setdefaultlanguage{sanskrit}
\setotherlanguage{english}
\setmainfont[Scale=1]{Shobhika}
\newfontfamily\s[Script=Devanagari, Scale=0.9]{Shobhika}
\newfontfamily\regular{Linux Libertine O}
\newfontfamily\en[Language=English, Script=Latin]{Linux Libertine O}
\newfontfamily\ab[Script=Devanagari, Color=purple]{Shobhika-Bold}
\newfontfamily\qt[Script=Devanagari, Scale=1, Color=violet]{Shobhika-Regular}
\newcommand{\devanagarinumeral}[1]{
\devanagaridigits{\number \csname c@#1\endcsname}} % for devanagari page numbers
\XeTeXgenerateactualtext=1 % for searchable pdf
\usepackage{enumerate}
\pagestyle{plain}
\usepackage{fancyhdr}
\pagestyle{fancy}
\renewcommand{\headrulewidth}{0pt}
\usepackage{afterpage}
\usepackage{multirow}
\usepackage{multicol}
\usepackage{wrapfig}
\usepackage{vwcol}
\usepackage{microtype}
\usepackage{amsmath,amsthm, amsfonts,amssymb}
\usepackage{mathtools}% <-- new package for rcases
\usepackage{graphicx}
\usepackage{longtable}
\usepackage{setspace}
\usepackage{footnote}
\usepackage{perpage}
\MakePerPage{footnote}
\usepackage{xspace}
\usepackage{array}
\usepackage{emptypage}
\usepackage{hyperref}% Package for hyperlinks
\hypersetup{colorlinks, citecolor=black, filecolor=black, linkcolor=blue, urlcolor=black}
\begin{document}
{९२ }{ वीरमित्रोदयस्य श्राद्धप्रकाशे-}{\\
द्यूतवृत्तिः द्यूतोपजीवी । पूर्वन्तु कौतुकात् इतकर्तोक्तः, इदन्तु
वृत्यर्थ-\\
मिति भेदः ।\\
पुत्राचार्य उको नारदेन,\\
पुत्राचार्य: स विज्ञेयो ग्रामे यो बालपाठकः ।\\
पुत्रादवासविद्यो वा पुत्राचार्य: स उच्यते ॥ इति ।\\
पुत्राचार्य: = भ्रामरी च निगद्यते । अपस्मारी भ्रामरीति मेधातिथिः,\\
रोगसाहचर्यात् । श्वित्रि - श्वेतकुष्ठवान् \textbar{} पिशुनः
परमर्मप्रकाशकः ।\\
वेदनिन्दकः वेदकुत्सनकर्ता । हस्तिगोऽश्षोष्ट्रदमकः = एतेषां विनेता गति\\
शिक्षायेतेति यावत् । नक्षत्रजीवी = ज्योतिषिकः । स्रोतसां भेदक:-
सेतुभेद\\
नेन श्रीह्याद्यर्थे प्रवाहनेता । तेषां स्रोतसां आवरणम् आच्छादन\\
तस्मिन्त्रैरन्तर्येण युक्तः । गृहसंवेशकः वार्द्धकिवृत्युपर्जावी
\textbar{} दृत्यः =\\
प्रेध्यः । वृक्षारोपकः मुख्येन वृक्षारोपणकर्ता । धर्मार्थ
वृक्षारोपणस्य\\
विहितत्वात् । खकांडी = श्वभिः क्रीडनकर्ता । श्वभिर्जीवति स\\
श्वजीवी । कन्यादूषकः = कन्याया योनिविदारणकर्ता । वृषलवृत्तिः = वृषलाः\\
शूद्राः तेषां सेवादिरूपा वृत्तिर्यस्य सः । क्वचिद् वृषलपुत्रेति\\
पाठ: । तदा वृषला एव पुत्रा यस्य स । गणाना याजकः = गणा.\\
नामहर्गणानां द्वादशादीनां याजकः । क्लीबः = असत्वः । नित्ययाचनकः\\
नित्यं याचापर इत्यर्थः \textbar{} कृषिजीवी स्वयं कृषिकर्ता \textbar{}
अस्वय कर्तृका\\
यास्तस्याः गौतमेन ब्राह्मणमुख्यवृत्तित्वेनाभिधानात् । श्लीपदी -
श्लीपद:\\
पादोच्छूनताख्यो रोगस्तद्वान् । निन्दितः = विनापि दोष सतां दृष्यः\\
और काडरमा मेषास्तैः क्रयविक्रयादिकर्ता तद्वृत्युपजीवीति या\\
वत् । माहिषिकोऽप्येवमिति मेधातिथिः ।\\
अन्ये तु -\\
महिषी तुज्यते भार्या या चैव व्यभिचारिणी ।\\
तस्यां यो जायते गर्भः स वै माहिषिकः स्मृतः ॥\\
इति देवलोको ग्राह्य इत्याहु: । अयं च स्वभर्तुः सकाशाज्जातः,\\
अन्यस्य कुण्डत्वात् ।\\
ब्रह्माण्डपुराणे तु\\
महिषीत्युच्यते भार्या या चैव व्यभिचारिणी ।\\
तस्यां यः क्षमते दोषं स वै माहिषिकः स्मृतः ॥

{ }{ वर्ज्यब्राह्मणनिरूपणम् । ९३}{\\
इति उक्तः । परपूर्वापतिः = परपूर्वा पुनर्भूस्तस्याः पतिः । एतान् वि.\\
गर्हिताचारान् उभयत्र दैवे पित्र्ये चेत्यर्थः ।\\
यमोऽपि ।\\
अश्राद्धेया द्विजाश्चान्ये तान्मे निगदतः शृणु ।\\
येभ्यो दन्त न देवानां न पितॄणां च कर्मकृत् ॥\\
काणा: कुब्जाश्च षण्ढाश्च कृतज्ञा गुरुतल्पगाः 1\\
ब्रह्मन्नाश्च सुरापाश्च स्तेना गोना चिकित्सकाः \textbar{}\textbar{}\\
राष्ट्रकामास्तपोन्मत्ताः पशुविक्रयिणश्च ये ।\\
(१) ग्रामकूटास्तुलाकूटाः शिल्पिनो ग्रामयाजकाः ॥\\
वृषलीभि प्रपीताश्च श्रेणीराजन्ययाजकाः ।\\
राजभृत्यान्धबाधन्मूकखल्वाटपङ्गचः \textbar{}\textbar{}\\
(२) कल्पोपजीविनश्चैव ब्रह्मविक्रयिणस्तथा ।\\
दण्डध्वजाश्च ये विप्रा ग्रामकृत्यकराश्च ये ॥\\
आगारदाहिनश्चैव गरदानलदाहकाः ।\\
कुण्डाशिनो देवलकाः परदाराभिमर्षकाः\\
इयावदन्ता कुनखिनः शिवत्रिणः कुष्ठिनश्च ये ।\\
वणिजो मधुहन्तारो हस्त्यश्वमका द्विजाः ॥\\
कन्यानां दूषकाञ्श्चव ब्राह्मणानां च दूषकाः \textbar{}\\
सूचकाः पोषकाश्चैव कितवाश्च कुशीलवाः \textbar{}\textbar{}\\
समयानां च भेत्तारः प्रदाने ये च वारकाः ।\\
अजाविका माहिषिकाः सर्वविक्रयिणश्च ये ।\\
वैष्णवीषु च ये सक्ताः शलाकादाहिनश्च ये ।\\
धनुष्कर्ता द्यूतवृत्तिः मित्रध्रुक् शठ एव च ॥\\
इषुकर्ता तथा वर्ज्य यञ्चाग्रेदिधिषूपतिः ।\\
पाण्डुरोगी गण्डमाली यक्ष्मी च भ्रामरी तथा ॥\\
पिशुनः कूटसाक्षी च दीर्घरोगी वृथाश्रमी \textbar{}\\
प्रव्रज्योपनिवृत्तश्च वृथा प्रव्रजितश्च यः ॥\\
यस्तु प्रव्रजिताजातः प्रव्रज्यावसितश्च यः ।

% \begin{center}\rule{0.5\linewidth}{0.5pt}\end{center}

{\\
( १ ) मानकूटा इति कमलाकरोद्धृतः पाठः ।\\
२) कन्योपजीविन इति युकम् ।

{९४ }{ वीरमित्रोदयस्य श्राद्धमकाशे-}{\\
तावुभौ ब्रह्मचाण्डालावाह वैवस्वतो यमः \textbar{}\textbar{}\\
राज्ञः प्रेष्यकरो यश्च ग्रामस्य नगरस्य च ।\\
समुद्रयायी वान्ताशी केशविक्रयिणश्च ये ।\\
अवकीर्णी च वीरघ्नो गुरुः पितृदूषकः ।\\
गोविक्रयी च दुर्वालः पूगानां चैव याजकः \textbar{}\textbar{}\\
मद्यपश्च कदर्यश्च सह पित्रा विवादकृत् ।\\
दाण्डिको बन्धकीभर्ता त्यक्तात्मा दारदूषकः ॥\\
सद्भिश्च निन्दिताचारः स्वकर्मपरिवर्जितः ।\\
परिवित्तिः परिवेत्ता भृत्याचार्यो निराकृतिः ॥\\
शूद्राचार्य: सुताचार्यः शुद्रशिष्यश्च नास्तिकः ।\\
दुष्टस्तु दारकाचार्यो मानकृचैलिकस्तथा ॥\\
चौरा वार्धुषिका दुष्टाः परस्वानां च नाशकाः ।\\
चतुराश्रमबाह्याश्च ये चान्ये पङ्किदूषकाः \textbar{}\textbar{}\\
इत्येतैर्लक्षणैर्युकांस्तान् विप्रान्न नियोजयेत् ।\\
राष्ट्रकामः = पौरोहित्यार्थे परराष्ट्रं वशीकर्तुं कामयते सः । अलीकव्य-\\
वहारेण ग्रामद्रव्यं यो भक्षयति स ग्रामकूटः \textbar{} तुला कूट:-तुलासु
कप\\
टकर्ता । वृषल्या असकृच्चुम्बितो दृषलीप्रपीतः । श्रेणीयाजकः = स्वर्ण-\\
कारादिपङ्कियाजकः । ब्रह्म =वेदः । दण्डध्वजाः =अपराधे सति राज्ञ। कृत.\\
चिह्नाः । मधुहन्ता=मध्वर्ये मक्षिकाहन्ता । सूचकः, परदोषस्येति शेषः ।\\
पोषकाः = परापवादकथासु दोषपोषकाः । सूचकपदसमभिव्याहारात् ।\\
समया. शिष्टकृता राजकृता वा नियमाः । वैष्णवीषु च ये सक्ताः = वेश्या\\
नुरक्ता इति यावत्, अथवा वैष्णवीषु = इन्द्रजालादिमायालु । शलाका-\\
दाहिन. ब्रह्मचिकित्सायां लोहशलाकया दाहकाः । कल्पतरौ तु भला.\\
जादाहिन इति पाठः, तदा लाजादाहो लाजाहोमस्तदुपलक्षितो वि.\\
वाहो यैर्न कृत इत्यर्थः । वृथाश्रमीच्या आश्रमान्तरे स्थित्वा आश्रमा•\\
न्तरधर्मवान् सः, वृथा प्रयासकर्ता वा । प्रव्रज्योपनिवृत्तः प्रव्रज्या
त्या\\
गलङ्कल्पस्ततो निवृत्तः । वृथा प्रब्रजितः = वैराग्यमन्तरेण प्रव्रजितः ।\\
प्रव्रज्यावसितः व्यस्तां स्वीकृत्य त्यजति सः ।\\
वान्ताशी चोको विष्णुना\\
देशं गोत्रं कुलं विद्यामन्नार्य यो निवेदयेत् \textbar{}\\
वैवस्वतेषु धर्मेषु वान्ताग्री स प्रकीर्तितः ॥

{ }{ वर्ज्यब्राह्मणनिरूपणम् । ९५}{\\
सर्वशा वयमित्येवमभिमानरता नराः ।\\
वान्ताशिनः परित्याज्याः श्राद्धे दाने च लम्पदाः ॥ इति ।\\
वीरनः = वालघ्नः \textbar{}\\
कदर्यथोको वृद्धगौतमेन,\\
आत्मनं धर्मकृत्यं च पुत्र दारांश्च पीडयेत् ।\\
मोहान्धः प्रचिनोत्यर्थान् स कदर्य इति स्मृतः ॥ इति ।\\
दण्डेऽधिकृतो दाण्डिकः । बन्धकी = पुंश्चली, तस्या भर्ता । त्यतात्मा आ.\\
हमघातार्थ कृतोद्यमः । दारदूषकः स्वयं पुरुषान्तरसंयोजनेन च यः\\
कुलस्त्रीणां दूषकः । दारका: अकृतोपनयनास्तेषामाचार्यो दारकाचार्य ।\\
सुमन्तुरप्यपायानाह ।\\
तस्करकितवाऽजपालगणगणिकाशूद्रप्रेष्यागम्यागामिपरिवेतृपरि\\
विन्तिपर्याहितपर्याधातृपौनर्भवान्धबधिरचारणक्कीबावकीर्णिवार्धुषि\\
कगरदायिकूटसाक्षिनास्तिकवृषलीपतिअद्भुतादोपसृष्टान्झिसोमविक्र-\\
यिविक्रेतृ पौस्तिककथककुण्डाशीकुण्डगोलकयन्त्रकारकाण्डपृष्ठदुध\\
मंचण्डविद्ध शिश्नदेवलकषण्डारूढपतितप्रायोस्थित कुनखिकिला-\\
सिश्यावद न्तवणिशिल्पवादित्र नृत्यगीतवाद्योपजीविमूल्यसांवत्ल-\\
रिकमहापथिका मकुट्टद्दीनातिरिक्ताङ्गविरागवास सञ्चापाङ्केयाः ।\\
गणादीनां त्रयाणां प्रेषयः सेवकः । पर्याधातृपर्याहितौ चोक्तौ । चारणो\\
बन्दिविशेषः । अहुतादः = विना पञ्चयज्ञादिक भोक्ता \textbar{} पौस्तिको =
पुस्तक\\
विक्रेता । पुस्तकलेखनकर्मोपजीवी वा । कथक - बृहत्कथादीनां कथानां\\
वक्ता । (१) काण्डपृष्ठाश्चोक्ता हारीतादिभे\\
शूद्रापुत्राः स्वयं दत्ता ये चैते क्रांतकाः सुताः ।\\
ते सर्वे मनुना प्रोक्ताः काण्डपृष्ठा न संशयः ॥\\
स्वकुलं पृष्ठतः कृत्वा यो वै परकुलं व्रजेत् ।\\
तेन दुश्चरितेनासौ काण्डपृष्ठो न संशयः \textbar{}\textbar{}\\
नारदेन तु ब्राह्मणस्यापदि गतायामापद्वत्तित्यागमभिधायोक्तम्-\\
तस्यामेव तु यो वृत्तौ ब्राह्मणो रमते रसात् ।\\
काण्डपृष्ठश्च्युतो मार्गात् सोऽपाङ्केयः प्रकीर्तित ॥ इत्युक्तः ।

% \begin{center}\rule{0.5\linewidth}{0.5pt}\end{center}

{\\
( १ ) स्वशाखां त्यक्त्वा परशाखयोपनीतस्तदध्यायी च काण्डपृष्ठ इति\\
कमलाकरः ।

{९६ }{ वीरमित्रोदयस्य श्राद्धप्रकाशे-}{\\
देवेलस्त्वन्यथाह,\\
वेश्यापतिः कृष्णपृष्ठः काण्डपृष्ठो भवेदिति ।\\
हेमाद्रौ तु काण्डस्पृष्ट इति पाठः स च "काण्डस्पृष्टः शस्त्रजांवी "\\
इत्यनुशासनात् प्रसिद्ध इत्युक्तम् । चण्डः = छिन्न मेहनचर्मेति हेमाद्रिः
।\\
विद्धशिश्नः = शिश्नमले छिद्रं कृत्वा तत्र सुवर्णघण्टिका घटितमुक्ताफ-\\
लादिबन्धनकर्तेति स एव । प्रायोऽनशनं तत उत्थितो निवृतः प्रायो-\\
त्थितः । किलासी सिध्मरोगी ।\\
शङ्खलिखितौ, घाण्डिको देवलकः पुरोहितो नक्षत्रादेशवृत्तिः ब्रह्मपु.\\
रुष इति । अपाङ्केया इति वक्ष्यमाणेन सम्बन्ध: ।\\
घाण्टिकचोक्त उशनसा ।\\
राज्ञः प्रबोधसमये घण्टा शिल्पस्तु घाण्टिकः ॥ इति ।\\
महतीं घण्टां वादयन् यः प्रतिगृहं भिक्षते स वा घाण्टिकः ।\\
ब्रह्मपुरुष म्यो जीवन् मुक्तवेषेण लोकान् प्रतारयन् द्रव्यमर्जयति सः ।\\
शातातपः,\\
अग्निष्टोमादिभिर्यज्ञैर्ये यजन्त्यल्पदक्षिणै ।\\
तेषामनं न भोक्तव्यमपाङ्केयाः (१) प्रकीर्तिताः ॥\\
आवेष्टिक पौंश्चल वार्धुषिकाहितुण्डिकप्रत्यवसितभृतकाध्याप.\\
काध्यापिततैलिक सूचकनियामककुशीलवादीन् दैवे पित्र्ये च वर्जयेत् ।\\
नास्ति कलीयकदर्योरभ्रकाक्रीडितानृतवादिनो जपहोमसन्ध्याविव.\\
र्जिताः । स्त्रौवयौन मौखसङ्करसङ्कीर्णाऽपुण्याब्राह्मणवृत्तिप्लवबन्धो.\\
पजीविवृषलीपतिशूद्रग्रामयाचकतस्करागारदाहिगरदाः सोमवि\\
क्रयिगायक नर्तकधनुःशरयोजककुण्डगोलकनराशंसकपिलदेवघा\\
ण्टिकनक्षत्रजीविमहोदधिगामिदीर्घरोगिमहापथिकाः पतिताः पति-\\
तपुरोहिताश्चानिवर्तमानाः पङ्किदूषकाः इति । मरणफलकं ग्रीवावेष्टनं\\
भावे कस्तदर्थ यो यज्ञे परिक्रीयते स आवेष्टिकः । पश्वलः पुंश्चली\\
शुल्कोपजीवी । आहितुण्डिकः = सर्पक्रीड़ी। प्रत्यवसितः = आश्रमच्युतः ।\\
नियामक = कपोतवाहः परान्नियमयतीति वा । आक्रीडी = क्रीडाशीलः ।\\
नराशंसो = मनुष्यस्तावकः । कपिलो=अतिकपिलवर्णः । दीर्घरोगी = आचेकि\\
त्स्यव्याधिः । महापथिकः=अतिदूरदेशादागतः । कल्पतरौ तु दीर्घमहाप-

% \begin{center}\rule{0.5\linewidth}{0.5pt}\end{center}

{\\
( १ ) अपाङ्कास्ते प्रकीर्तिता इति कमलाकरोत. पाठः ।

{ }{ वर्ज्यब्राह्मणनिरूपणम् \textbar{} ९७}{\\
थिक इति पाठः, तदैकमेव पदम् । दीर्घमहापन्था=मरणम् - तत्र कृतोद्यम\\
इत्यर्थः । अनिवर्तमानः=प्रायश्चित्तमनिच्छमानः \textbar{}\\
आपस्तम्बः,\\
शिवत्री शिपिविष्टः परतल्पगाम्यायुधी यः पुत्रः शूद्रोत्पत्रो ब्राह्म•\\
ण्याम्, इत्येते श्राद्धे भुञ्जानाः पङ्किदूषका भवन्ति ।\\
शिपिविष्टः = दुश्चर्मा । य. पुत्रः शूद्रोस्रो ब्राह्मण्यामिति ।
असवर्णपरिग्रहे\\
ब्राह्मण्यां पुत्रमनुत्पाद्य शूद्रायामुत्पादित इति कपािति कल्पतरुः ।\\
हेमाद्रिस्तु शूद्रसमाद् ब्राह्मणात् सत्कुलप्रसुतायां सद्वृत्तायां
ब्राह्म-\\
ण्यामुत्पादित इत्याह ।\\
वायुपुराणे,\\
यस्तिष्ठेद्वायुसक्षस्तु चातुराश्रम्यबाह्यतः ।\\
अयतिर्मोक्षवादी च उभौ तौ पङ्गिदूषकौ \textbar{}\textbar{}\\
उप्रेण तपसा युक्तः श्चिञवाली ? बहुश्रुतः ।\\
अनाश्रमी तपस्तेपे तं विप्रं न निमन्त्रयेत् ॥\\
तथैौपपत्तिकः शाठो नास्तिको वेदनिन्दकः ।\\
ध्यानिनं ये च निन्दन्ति सर्वे ते पदूषकाः ॥\\
वृथा मुण्डांश्च जटिलान् सर्वकार्पटिकांस्तथा ।\\
निर्घृणान् भिन्नवृत्तांश्च सर्वभक्षांश्च वर्जयेत् ॥\\
प्राह वेदान् (१) वेदभृतो वेदान् यश्चोपजीवति ।\\
उभौ तौ नाईतः श्राद्धं पुत्रिकापतिरेव च \textbar{}\textbar{}\\
वृथा दारांश्च यो गच्छेत् यो यजेताध्वरैर्वृथा ।\\
नाईतस्तावपि श्राद्धं द्विजो यश्चैव नास्ति कः ॥\\
आत्मार्थे यः पचेदनं न देवातिथिकारणात् ।\\
नाईत्यसावपि श्राद्धं पतितो ब्रह्मराक्षसः ॥\\
स्त्रियो रक्ताम्बरा येषां परिवादरताश्च ये ।\\
अर्थकामरता ये च न तान् श्राद्धेषु भोजयेत् \textbar{}\textbar{}\\
सन्ति वेदविरोधेन केचिद् विज्ञानमानिनः ।\\
अयज्ञपतयो नाम ते ( २ ) ध्वसन्ति यथा रजः ॥\\
मुण्डान् जटिलकाषायान् श्राद्धे यत्नेन वर्जयेत् ।\\
औपपत्तिक = उपपत्तिस्तर्कस्तन्मात्रव्यवहारी हैतुक इति यावत् ।

% \begin{center}\rule{0.5\linewidth}{0.5pt}\end{center}

{\\
( १ ) वेदविद इति वायुपुराणे पाठः ।\\
( २ ) ते श्राद्धेषु यथा रज इति वायुपुराणे पाठः ।\\
१३ वी० मिश्र

{९८ }{वीरमित्रोदयस्य श्राद्धप्रकाशे-}{\\
शाठोऽत्र शाठ्यधर्माभिरतः । वृथा सुण्डा: अविहितमुण्डन कर्तारः । वृथा\\
जटिला = अविहितजटावन्तः । इया दारांश्च यो गच्छेदिति । वृथा निषिद्धदिवसे
।\\
रक्ताम्बरधरा=आर्तववती ।\\
महाभारते,\\
ऋणहर्ता च यो राजन् यश्च वार्बुषिको द्विजः ।\\
प्राणविक्रयवृत्तिश्च राजन्नार्हन्ति केतनम् \textbar{}\textbar{}\\
ऋणहर्ता ऋणं गृहीत्वा यो दातुं नेच्छति सः । प्राणविक्रयवृत्तिः =\\
प्राणसंशयापादकवृत्तिः ।\\
मत्स्यपुराणे-\/-\/-\\
कृतघ्नाम्नास्तिकांस्तद्वद् म्लेच्छदेशनिवासिनः ।\\
त्रिशङ्कन् बर्बरानाम्ध्रांश्चीनद्रविडकुडुणान् (१) ॥\\
कर्णाटकांस्तथा कीरान् (२) कलिङ्गांश्च विवर्जयेत् ।\\
अयं स म्लेच्छादिदेशातिरिक्तदेशवासिषु कर्तृषु सति संभवे\\
म्लेच्छादिदेशवासिद्विजनिषेधो न तु तत्तद्देशे, अन्यथा तस्मिन् देशे\\
श्राद्धलोपापत्तेः ।\\
देवल:-\\
गोभर्तृविश्वस्तान्नदप्रवजित बन्धुमित्रघातका मातृपितृपुत्रदारा\\
ग्रिहोत्रस्यागिन: यज्ञोपहन्ता वृषलीपतिः सोमविक्रयी वात्यो नि\\
ष्क्रियश्चेति पतिताः । जारोपपतिकुण्डगोलकभर्तृदिधिषूपतिमूढभृत.\\
काध्यापकायाज्ययाजक ब्रह्म धर्म दुष्टद्रव्यविक्रयिकदर्यवर्णसंभेदकाना\\
र्य भ्रष्ट शौचाऽधन्यदेव लकवार्धुषिक
गोत्रभित्परिवित्तिपरिवेत्तृकृष्णपृ.\\
ष्ठनिष्कृतिनिराकृत्यवकीर्णिम्लेच्छावरेटकरणागारदाहिनः षड्विधा:\\
क्लीबाश्चेति उपपातकिनः । अनपत्यकूटोपसाक्षिपृष्ठोपघातिस्त्रीजित से.\\
तुभेदकतालाव चरणखङ्गोपजीविधर्मपाठक नित्ययाचकप्रायश्चित्तवृत्ति\\
धूर्तसाधनिकमृगयुकितवनास्तिकपिशुनशबरवणिकबन्दिपौनर्भवारम\\
म्भरिशुक्तिसमुद्रयायिकृत्याभिचारशीलतैलिकशास्त्रि कवैद्यराजभृत्य-\\
ककन्यादूषक भ्रूणघ्नक्रूरकुहक मित्रक्षुग्दत्तापव्ययिसमयभेदकबाकूदण्ड.\\
पुरुषाशक्यशिल्पिक\\
हस्त्यारोहकञ्चबन्धकाश्चेति पातनीयकाः। अष्टभि\\
महारोगैरभिभूता विकलेन्द्रिया हीनाङ्गा अधिकाङ्गाश्चेति पङ्किदूषकाः।\\
उन्मादस्त्वग्दोषो राजयक्ष्मा श्वासो जलोदरः प्रमेहो भगन्दरमश्म-

% \begin{center}\rule{0.5\linewidth}{0.5pt}\end{center}

{\\
( १ ) कौडणानिति कमलाकरोद्धृतः पाठः ।\\
( २ ) तथाभीरानिति कमलाकरोद्धृतः पाठः \textbar{}}

{ वर्ज्यब्राह्मणनिरूपणम् \textbar{} ९९\\
~\\
रीत्यष्टौ महारोगाः । जडान्धकाणबधिरकुणि इति विकलेन्द्रिया । उभय-\\
भागकेदा दुष्टव्रणाः । पापिष्ठतमाश्चेति ।\\
एते पञ्चविधाः प्रोक्ता वर्जनीया नराधमाः ।\\
स्वसंज्ञालक्षणास्ते स्युः विशेषश्चात्र दृश्यते ॥\\
एते दुर्ब्राह्मणाः सर्वे क्रमशः समुदाहृताः ।\\
कर्मणा योनितश्चैव देहदोषैश्च कुत्सिताः \textbar{}\textbar{}\\
( १ ) एतेषां कर्मदोषेण पतिता ये नराधमाः ।\\
यान्ति ते निरयान् घोरान् त्यक्ताः सद्भिरिदैव च ॥\\
योनिदोषेण ये दुष्टा ये च दोषैः शरीरजैः ।\\
इहैव वर्जनं तेषां भवेदनपराधिनाम् ॥\\
कृतघ्नः पिशुनः कूरो नास्तिकः कुहकः शठः ।\\
मित्रध्रुक् चेति सर्वेषां विशेषा निरयालयाः ॥\\
सर्वेषूत्तरभोज्याः स्युरदानार्हाश्च कर्मसु ।\\
ब्रह्मभावान्निरस्ताइच पापदोषवशानुगाः ॥\\
कुण्डगोलकभर्ती = यस्तयोः पुत्रत्वेन स्वीकर्ता । वर्णसंभेदकाऽध\\
न्यावरेटकरणकृष्णपृष्ठाश्चोक्ता देवलेन-\\
निकृष्टोत्कृष्टयोर्मध्ये यो वर्णेष्वनवग्रहः ।\\
आचरत्यपराचारं वर्णसंभेदकस्तु सः ॥ इति ।\\
एकाकी व्यसनाक्रान्तो ऽधन्य इत्युच्यते बुधैः ।\\
वेश्यापति: कृष्णपृष्ठः काण्डपृष्ठोऽथवा भवेत् ।\\
द्वितीयस्य पितुर्योऽन्नं भुक्त्वा परिणतो द्विजः ।\\
अवरेट इति शेयः शुद्धधर्मा स चेष्यते ।मैं\\
भवेत् करणसंज्ञश्च यः क्रयव्यवहारवान् । इति ।\\
द्वितीयस्येति = पितुः सकाशाद् द्वितीयस्य तन्मातुर्निरोधकस्य पुरुषा\\
न्तरस्येत्यर्थः । पृष्ठोपघाती = पृष्ठमैथुन कर्ता, परोक्षापकरणशीलो वा,
चमरी\\
पुच्छच्छेत्ता वा । तालावचरण: =तालोपजीवी । धर्मपाठकः=अधर्मशीलेभ्यो\\
धर्मस्याध्यापकः । साधनिकः -तुरगादिसाधनेष्वधिकृतः । आत्मम्भरिः\\
पित्रादेरभरणेन स्वोदरमात्र पोषकः । कुहक =दाम्भिकः । दत्तापव्ययी=\\
धर्मार्थ प्रतिगृह्याऽसद्व्ययकर्ता ।

% \begin{center}\rule{0.5\linewidth}{0.5pt}\end{center}

१ ) एतेषा = कर्मयोनिदेहदोषाणा मध्य इत्यर्थ. ।

१૦૦ वीर मित्रोदयस्य श्राद्धप्रकाशे-

{कूर्मपुराणे ।\\
बुद्धश्रावकनिर्मन्थाः पञ्चरात्रविदो जनाः ।\\
कार्पाटिकाः पाशुपताः पाषण्डा ये च तद्विधाः ॥\\
यस्याश्नन्ति हवींष्येते दुरात्मानस्तु तामसाः ।\\
न तस्य तद् भवेत् श्राद्धं प्रेत्य चेह फलप्रदम् ॥ इति ।\\
पञ्चरात्रञ्च विरुद्धाचारयुतं पाञ्चरात्रं, न तु नारदपञ्चरात्रादि\\
तस्याविगीतमहाजनपरिग्रहात् ।\\
कश्यपः ।\\
दारविगणहन्तूंश्च व्यङ्गान्नक्षत्र सूचकान् ।\\
वर्जयेद् ब्राह्मणानेतान् सर्वकर्मसु यत्नतः ।\\
नागर खण्डे ।\\
अन ये च निर्दिष्टास्तानेतान् श्रृणु वमि ते ।\\
हीनाङ्गानधिकाङ्गांश्च सर्वभक्षान्निराकृतीन् ॥\\
श्यावदन्तान् वृथावेदान् वेदविक्रयकारकान् ।\\
वेदविप्लावकान् वापि वेदशास्त्रविवर्जितान् ॥\\
कुनखान् योगसंयुक्तान् द्विर्ननान् परहिंसकान् ।\\
जनापवादसंयुक्तान् नास्तिकाननृतानपि ॥\\
वार्धुषिकान् विकर्मस्थान् शौचाचा रविवर्जितान् ।\\
अतिदीर्घान् कृशान् वापि स्थूलानप्यतिलोमशान् ॥\\
निर्लोमान् वर्जयेत् श्राद्धेय इच्छेत पितृतर्पणम् ।\\
परदाररताश्चैव तथा यो वृषलीपतिः ॥\\
शठो मलिम्लुचो दम्भी राजभृच्छ्रन्यवृत्तयः ।\\
सगोत्रायां च सम्भूतस्तथैकप्रवरासु च ॥\\
कनिष्ठप्राक्कृताधानाः कृतोद्वाहास्त्वपाङ्कयः ।\\
प्रागूदीक्षितो यः कनिष्ठः स त्याज्योऽग्रज संयुतः ।\\
मातापितृगुरुत्यागी तथैव गुरुतल्पगः ॥\\
निर्दोषां यस्त्यजेत् पक्ष कृतोद्वाहश्च कर्षकः ।\\
शिल्पजीवी प्रमादी च पण्यजीवी घृतायुधः ।\\
एतान् विवर्जयेच्छ्राद्धे येषां न ज्ञायते कुलम् ॥ इति ।\\
सौरपुराणे,\\
अङ्गवङ्गकलिङ्गांश्च सौराष्ट्रान् गुर्जरांस्तथा \textbar{}\\
आभीरान कोणांश्चैष द्रविडान् दक्षिणापथान् ॥

{ }{ वर्ज्यब्राह्मणनिरूपणम् \textbar{} १०१}{\\
आवन्त्यान्मागधांश्चैव ब्राह्मणांस्तान् विवर्जयेत् ।\\
अपाङ्केयेभ्य एव दत्तं न केवलमफलम् । अपि तु तत् पङ्क्तयुप-\\
विष्टानामपि दत्तमफलमिति ।\\
तथा च मनु,\\
अपाङ्कयो यावतः पाङ्कयान् भुञ्जानाननुपश्यति ।\\
तावतो न फलं तत्र दाता प्राप्नोति बालिशः ॥

{तथा,\\
यावतः संस्पृशेदङ्गैर्ब्राह्मणान् शूद्रयाजकः ।\\
तावतां न भवेद्दातु फलं दानस्य पैतृकम् ॥\\
केषां चिदपाङ्केयानां संख्याविशेषेण सहपङ्कयुपविष्टकत्व माह\\
स एव-

{ वीक्ष्यान्धो नवतेः काणः षष्टेः श्वित्री शतस्य तु ।\\
पापरोगी सहस्रस्य दातुर्नाशयते फलम् ॥\\
अत्र चान्धस्य वीक्षणासम्भवात् वीक्ष्येत्यनेन संनिधानमात्र\\
लक्ष्यते तेन संनिधानमेव नवत्यादेः फलनाशकमिति ज्ञेयम् । सत्रि\\
धिश्च यावान् देशश्चक्षुष्मतो दृष्टिगोचरस्तावति देशेऽवस्थितत्वम् ।\\
एतच काणादौ संख्यापचये दोषलाघवं प्रायश्चित्तविशेषार्थमिति\\
मेधातिथिः । क्वचिदपवादमाह वशिष्ठः । अथाप्युदाहरन्ति ।\\
अथ चेत् मन्त्रविधुतः शारीरैः पङ्किदूषणैः ।\\
अदृष्यं तं मनुः प्राह पङ्किपावन एव सः ॥ इति ।\\
मन्त्रविदा नवत्या मध्यस्थेन युक्त इत्यर्थः । शारीरा अपि दोषाः\\
शिवत्र्यादिव्यतिरिक्ताः । हौनाधिकाङ्गुलित्वादय इति हेमाद्रिः
\textbar{}\\
अत्र विशेषमाह कश्यपः ।\\
काणादीन भोजयेदेवे श्राद्धे दाने तु वर्जयेत् ।\\
दैवे वैश्वदेवे यदि श्राद्धे भोजयेत्तदेत्यर्थः । दाने तु वर्जयेदेव ।\\
यत्तु वशिष्ठनोक्तम्-\\
(१) विद्वत्सु चेत्वविद्वांसो यस्य राष्ट्रेषु भुञ्जते ।\\
(२) तान्यनावृष्टिमिच्छन्ति (३) महदू चा जायते भयम् ॥ इतेि ।\\
तत्, यत्र विद्वसु सत्सु श्राद्धे, अविद्वांसः श्रोत्रियाद्यसमिश्रिता\\
भुञ्जत इति व्याख्येयम् ।

% \begin{center}\rule{0.5\linewidth}{0.5pt}\end{center}

{\\
१) विद्वद्भोज्यान्यविद्वांसो येषु राष्ट्रेषु भुञ्जते इति पु०
मुद्रितवशिष्ठस्मृतौ पाठ. \textbar{}\\
( २ ) तदन्नं नाशमायाति महम्चापि भयं भवत् इति पाठान्तरम् ।\\
( ३ ) तान्यनावृष्टिमृच्छन्तीति पाठान्तरम् ।

{१०२ }{ वीरमित्रोदयस्य श्राद्धप्रकाशे-}{\\
नन्विदं विहितनिषिद्धस्वरूपज्ञानं न परीक्षणमन्तरेणोपपद्यते\\
तत्र परीक्षणं विश्वामित्रेण निषिद्धम् ।\\
न ब्राह्मणं परीक्षेत कदाचिदपि बुद्धिमान् ।\\
दातून परीक्ष्य दत्तानि नयन्ति नरकं ध्रुवम् ॥\\
भविष्यपुराणेऽपि । आदित्य उवाच ।\\
एवमेव न सन्देहो यथा वदलि खेचर \textbar{}\\
ममाप्येतन्मतं वीर ब्राह्मणं न परीक्षयेत् ॥\\
तत् कथं परीक्षणव्यतिरेकेण विहितज्ञानं कार्यमिति । सत्यम् ।\\
निषेधस्यान्नदान पर्गक्षापरत्वेनाप्युपपत्तेः । श्राद्धे परीक्षायां बा.\\
धकाभावात् ।\\
अत एव विष्णुधर्मोत्तरे,\\
अन्नदाने न कर्तव्यं पात्रावेक्षणमेव तु ।\\
अनं सर्वत्र दातव्यं धर्मकामेन वै द्विज \textbar{}\\
सदोषेऽपि तु निर्दोषं सगुणेऽपि गुणावहम् ।\\
तस्मात् सर्वप्रयत्नेन देयमनं सदैव तु \textbar{}\textbar{}\\
पित्र्ये कर्मणि तु प्राशः परीक्षेत प्रयत्नतः ।\\
बिष्णुरपि ।\\
दैवे कर्मणि ब्राह्मणं न परीक्षेत यत्नात् परीक्षेत पित्र्ये ।\\
केचित्तु स्वयं परीक्षणस्यायं निषेधोऽन्यद्वारा परीक्षणे न दोष\\
इति वदन्ति ।\\
परीक्षण प्रकार उतो मत्स्यपुराणे ।\\
शीलं संवलता ज्ञेयं शौचं सयवहारतः ।\\
प्रशा सङ्कथनाद् ज्ञेया त्रिभिः पात्रं परीक्ष्यते ॥ इति ।\\
इय च श्राद्धीयब्राह्मणपरीक्षा अतिथिव्यतिरिक्तस्य ।\\
न परीक्षेत चारित्रं न विद्यां न कुलं तथा ।\\
न शीलं न च देशादीनतिथेरागतस्य हि ॥\\
कुरूपं वा सुरूपं वा कुचैलं वा सुवाससम् ।\\
विद्यावन्तमविद्यं वा सगुणं वाथ निर्गुणम् ॥\\
मन्येत विष्णुमेवैनं साक्षान्नारायणं हरिम् ।\\
अतिथि समनुप्राप्तं विचिकित्सेन कर्हि चित् ॥\\
इति नृसिंहपुराणे तत्परीक्षणनिषेधात् ।\\
गुणागुणविचारेण धमन्ते तेऽवमानिताः ।\\
निर्दहत्याशु गृहिणं ताइशेष्वमानना \textbar{}\textbar{}}

{ }{वर्ज्यब्राह्मणनिरूपणम् \textbar{} १०३}{\\
अतोऽतिथेर्न कर्तव्या कापि चर्चा कदाचन ॥\\
इति वायुपुराणे दोषोक्तेश्च । न चायं परीक्षानिषेधः श्राद्धादन्यत्र\\
श्राद्धेऽतिथेर्निमन्त्रणाभावेनाभोज्यत्वादिति वाच्यम् ।\\
काले तत्रातिथि प्राप्तमन्नकामं द्विजोत्तमम् ।\\
ब्राह्मणैरभ्यनुज्ञातः कामं तमपि भोजयेत् ॥\\
योगिनो विविधैरूपैर्भवन्तीत्युपकारिणः ।\\
भ्रमन्तः पृथिवीमेतामविज्ञातस्वरूपिणः ॥\\
तस्मादभ्यर्चयेत् प्राप्तं श्राद्धकालेऽतिथि बुधः ।\\
श्राद्धक्रियाफलं हन्ति द्विजेन्द्रोऽपूजितोऽतिथिः ॥\\
इति वाराहे तस्य भोज्यताविधानात्,\\
अतिथिर्यस्य नाश्नाति तच्छ्राद्धं न प्रशस्यते ।\\
इत्यभोज्यत्वे शातातपेन दोषोक्तेश्च । अत्राविज्ञातस्वरूपिण एवा.\\
चनीयत्वाभिधानात् यत् केनचिदुच्यते प्रागुदाहृतविष्णुवचनाद.\\
तिथेरपि श्राद्धपरीक्षणं कर्तव्यमेव, यत्तु तस्यापररीक्ष्यत्वं
तच्छ्राद्धा-\\
तिरिक्तविषयमिति, तत्तिरस्कृतं वेदितव्यम् ।\\
अविज्ञातं द्विज श्राद्धे न परीक्षेत् सदा बुधः ।\\
इति वायुपुराणेऽतिथिं प्रक्रम्योक्तेश्च । ब्राह्मणैरभ्यनुज्ञात इति
वाक्ये\\
ब्राह्मणपदस्य निमन्त्रितार्थकत्वात् तदनुज्ञायामेवातिथिर्भोज्य इति\\
भावः । ब्राह्मणातिरिक्तानामतिथिसमानधर्मिणान्तु श्राद्धोत्तरकालं\\
भोजनम् । तदपि विकल्पेन । तथाच मनुः-\\
यदि त्वतिथिधर्मेण क्षत्रियो गृहमाव्रजेत् ।\\
भुक्तवत्सु च विप्रेषु काम तमपि भोजयेत् ।।\\
अत्र काममित्युपादानात् सत्यामिच्छायामिति गम्यते न ब्राह्म\\
णातिथिवनियमेनेति । ब्राह्मणोऽतिथिश्च ब्राह्मणपङ्कावुपवेशनीयः ।\\
न हि विद्यादयस्तस्मिन् पूज्यताहेतवः स्मृताः ।\\
केवलेनातिथित्वेन स भवेत् पङ्गिपावनः ॥\\
इति वायुपुराणे पङ्किपावनत्वोकेः ।\\
यत्तु -\\
शेषान् वित्तानुसारेण भोजयेदन्यवेश्मनि ॥\\
इति पुराणवचनं तच्छ्राद्धीय वेश्मन्यवकाशाभाव इति हेमाद्रिः ।\\
यतिरूपातिथिस्तु श्राद्धपङ्कोवेवोपवेशनीयः ।\\
श्राद्धकाले यति प्राप्त पितृस्थानेषु भोजयेत् ।

{१०४ }{वीरमित्रोदयस्य श्राद्धप्रकाशे-}{\\
इति बृहस्पतिना स्थानविशेषोतेः । पितृस्थानेष्विति बहुवचनात्\\
पित्रादिस्थानेव्वित्यर्थः । इदं च पितृस्थाने उपवेशनस्थानकल्पना.\\
तः पूर्वमागतस्य, पश्चादागतस्य त्वतिथिषत्पङ्को भोजनमात्रं देय\\
मिति हेमाद्रिः । अत्र च यतिस्त्रिदण्डी शेयः ।\\
शिखिभ्यो धातुरक्तेभ्यस्त्रिदण्डिभ्यः प्रदापयेत् ।\\
इति ब्रह्मवैवर्तात् । शिखावन्तो गैरिकारक्तवसनास्त्रिदण्डाश्च तेभ्यः\\
प्रयत्नेन दद्यादिति श्राद्धं प्रक्रम्य बौधायनवचनाच \textbar{}\\
यस्तु,\\
मुण्डान् जटिलकाषायान् आद्धकाले विवर्जयेत् ।\\
इति निषेधः स त्रिदण्डिव्यतिरिक्तपरः ।\\
न च सवतेंन-\/-\\
अष्टौ भिक्षाः समादाय दश द्वादश वा यतिः ।\\
अखिला शोधयेत्तास्तु ततोऽश्नीयाद् द्विजोत्तमः ॥\\
इति भैक्ष्यभोजनविधानात् कथं यतेः श्राद्धनियोग इति वाच्यम् ।\\
असक्तोऽनुग्रहार्थ वा यतिरेकान्नभुग्भवेत् ।\\
इति काष्र्ष्णाजिनिनैक भिक्षाया अपि विधानात् नियोगोपपत्तेः ।\\
नन्वेवमपि न पतेः श्राद्धे प्राप्ति:, श्रद्धस्य मधुमांससाध्यत्वात्\\
यतेश्च तन्निषेधादिति चेन्न । तद्रहिते श्राद्धे तन्नियोगस्थ सुवचत्वात्,\\
निषेधमुल्लङ्घ प्रवृत्तयतिविषयत्वेनापि नियोगसम्भवाश्चेति । ननु\\
कस्मान् मधुमांसवत्यपि श्राद्धे यतेर्मधुमांसव्यतिरेकेण भोजनं न न\\
भवति ब्रह्मचारिण इवेति चेत्, विधायकाभावात् । ब्रह्मचारिणस्तु,\\
ब्रह्मचर्ये स्थितो नैकमन्नमद्यादनापदि ।\\
ह्मणः काममश्नीयाच्छ्राद्धे व्रतमपीडयन् ॥\\
इति याज्ञवल्क्येन तथाविधानादिति हेमाद्रिः ।\\
इति श्रीमत्सकलसामन्त चक्र चूडामणिमरीचिमञ्जरीनीराजितचरण.\\
कमलश्रीमन्महाराजप्रतापरुद्रतनूजश्रीमन्महाराजमधुकरसाहस्\\
नुचतुरुदधिवलय वसुन्धरा हृदय\\
पुण्डरी कृषि काशदिनकर श्री.\\
मन्महाराजाधिराजश्री वीरसिंह देवोद्योजित श्रीहंसपण्डिता.\\
त्मजश्रीपरशुराममिश्रसूनुस कलविद्यापारावारपारीण.\\
धुरीणजगद्दारिभ्रमहागजपारीन्द्र विद्वज्जनजीवा.\\
तुश्रीमन्मित्रमिश्रकृते वीरमित्रोदयाभिधनिबन्धे\\
श्राद्धप्रकाशे ब्राह्मणनिरूपणम् ।

{ }{ निमन्त्रणप्रकारः । १०५ }{\\
अथ निमन्त्रणम् \textbar{}\\
तच्चाहमत्र विश्वेदेवादिस्थाने भविष्यामीत्येवं ब्राह्मणस्वी-\\
कारफलको यः वक्ष्यमाणप्रयोगवाक्योच्चारणपूर्वको व्यापार: ।\\
इद च श्राद्धपूर्वदिने प्रदोषान्ते कार्यम् ।\\
"प्रार्थयीत प्रदोषान्ते भुक्तानशयितान् द्विजान्"\\
इति यमोक्ते । भुक्तान् इति च निमन्त्रणोत्तर भोजनव्यावृत्तिमात्रं\\
क्रियते न तु कृतभोजनत्वं वास्तवमपेक्षितमनुपयोगात् । उक्तकाले\\
तदसम्भवे निमन्त्रणं परेद्युरपि कार्यम्,\\
असम्भवे परेद्युर्वा ब्राह्मणांस्तानिमन्त्रयेत् \textbar{}\\
इति देवलोक्ते । इद च स्वयमेव कार्ये "दाता विप्रानिमन्त्रयेत्"\\
इति तस्यैवोक्तेः । तदसम्भवे वा स्वयं प्रेषितेन सवर्णेन । "सवर्ण\\
प्रेषयेदाप्तं द्विजानामुपमन्त्रणे " इति प्रचेतःस्मरणात्,\\
अभोज्यं ब्राह्मणस्यानं क्षत्रियाद्यैर्निमन्त्रितम् ।\\
इति हेमाद्रिधृतनिषेधवचनाच । तत्रापि पुत्रो मुख्यः "यथैवात्मा\\
तथा पुत्र" इति स्मरणात् । तस्याप्यसम्भवे भ्रातृशिष्यादिरिति शे\\
यम् । "स्वयं शिष्योऽथवा सुत" इति बृहस्पतिवचनात् । अत्र च पाठ\\
क्रमो न विवक्षितः सुतस्यान्तरङ्गत्वादिति हेमाद्रिः ।\\
निमन्त्रणप्रकारश्च प्रचेतसा उक्तः-\\
कृतापसव्यः पूर्वेद्युः पितृपूर्व निमन्त्रयेत् \textbar{}\\
भवद्भिः पितृकार्य नः सम्पाद्यं वः प्रसीदत ॥\\
सव्येन वैश्वदेवार्थ प्रणिपत्य निमन्त्रयेत् ॥ इति ।\\
भवद्भिरित्यादि च ब्राह्मणोन्मुखीकरणमन्त्रो न तु निमन्त्रणमन्त्र\\
इति हेमाद्रिः । निमन्त्रणं तु अमुकस्य श्राद्धे त्या क्षणः क्रियता-\\
मित्येवं कार्यम् ।\\
ब्राह्मणानां गृहं गत्वा तान् प्रार्थ्य विनयान्वितः ।\\
अमुकस्य त्वया श्राद्धे क्षणो वै क्रियतामिति ॥\\
(१) वदनभ्युपगच्छेयुर्विप्राश्चैवतथेति च ।\\
भूयोऽपि व्याहरेत् कर्ता तान् प्राप्नोतु भवानिति ॥\\
द्विजस्तु प्राप्नवानीति विधिरेष निमन्त्रणे \textbar{}\textbar{}\\
इति नागरखण्डोक्तेः । एवं च ब्राह्मणैरपि सर्वैः सहैवोन्तथेत्युच्चार्य

% \begin{center}\rule{0.5\linewidth}{0.5pt}\end{center}

{( १ ) वदेदिति मयूखे पाठः ।\\
१४ वी० मि‍

{१०६ वीरमित्रोदयस्य श्राद्धप्रकाशे-}{\\
यजमानेन च प्राप्नोतु भवान् इत्युक्ते प्राप्नवानीति च वक्तव्यमिति\\
हेमाद्रिप्रभृतयः \textbar{}\\
केचित्तु,\\
दक्षिणं जानुमालभ्य त्वं मयाऽत्र निमन्त्रितः ।\\
इति निमन्त्रणवाक्यमित्याहु' \textbar{}\\
मैथिलास्तु त्वामहमामन्त्रये इति वदन्ति ।\\
शूलपाणिस्तु त्वां निमन्त्रये इति प्रयोगवाक्यमित्याह ।\\
अन्येतु उभयविधप्रयोगदर्शनात विकल्प एवेत्याहुः \textbar{}\\
तञ्च निमन्त्रण ब्राह्मणेन दक्षिणं जानुमालभ्य कार्यम्, उदा-\\
हृतमत्स्यवाक्यात् । यत्तु प्रागुदाहृतप्रचेतोवचने प्रणिपत्येत्युक्तम्,\\
तत्, शूद्रविषयम् ।\\
दक्षिणं चरणं विप्रः सव्य वै क्षत्रियस्तथा ।\\
पादावादाय वैश्यो द्वौ शूद्रः प्रणतिपूर्वकम् ॥\\
इति आदित्यपुराणात् । निमन्त्रणस्य क्वचिदपवाहो मार्कण्डेयपुराणे-\\
भिक्षार्थमागतान् विप्रान् काले संयमिनो यतीन् ।\\
भोजयेत् प्रणिपाताद्यैः प्रसाद्य यतमानसः । इति ।\\
सयमिनो = ब्रह्मचारिणः ।\\
अथ निमन्त्रणीयब्राह्मणसङ्ख्या \textbar{}\\
तत्र याज्ञवल्क्यः,\\
दैवे युग्मान् यथाशक्ति पित्रेऽयुग्मांस्तथैव च ।\\
युग्मान्=समसङ्ख्याकान् ।\\
गौतमः,\\
नवावरान् भोजयेत्, अयुजो वा यथोत्साहमूर्ध्वं त्रिभ्यो गुणव-\\
न्तमिति ।\\
नवभ्यः नावरा अधिकसङ्ख्याकाः । नवाद्येक सङ्ख्याका इति तु\\
हेमाद्रिः । इयं च सङ्ख्या प्रत्येकं पित्रादित्रिके सम्बध्यते
प्रतिप्रधानं\\
गुणावृत्तेन्यय्यत्वात् । नवावरेष्वपि समव्यावृत्यर्थमयुज इत्युक्तम् ।\\
यथोत्साहमिति । उत्साहः =शक्तिः । ऊर्ध्व=नवभ्योऽपि । यत्र तु ब्राह्मणा\\
अनेके न लभ्यन्त तत्र त्रयाणां पित्रादीनामर्थ एको यदि भोज्यस्तदा\\
गुणवानेवेत्याह । त्रिभ्यो गुणवन्तमित्यादिना । इदं च नवावरत्वं वैश्व-\\
देविकेऽपि ज्ञेयम् । समसङ्ख्यत्वं परं तत्राधिकं ज्ञेयं स्मृत्यन्तरात् ।\\
एवं मातामहेष्वपि अनुसन्धेयम् । तेषामपि च वैश्वदेविकस्य तन्त्र-

{ }{ निमन्त्रणपूर्वकाल कृत्यम् \textbar{} १०७}{\\
त्वपक्षे भेदपक्षे वा गणनया तावन्तो ब्राह्मणा अनुसन्धेयाः ।\\
मातामहानामप्येवं तन्त्रं वा वैश्वदेविकम् ।\\
इति वचनात् ।\\
अशक्तं प्रत्याह विष्णुपुराणे ।\\
देवानामेकमेकं वा पितॄणां च नियोजयेत् ॥ इति ।\\
यदा तु अत्यन्तमशक्तेन एक एव ब्राह्मणः प्राप्यते तदा तेषां\\
वर्गद्वयस्थाने एकमेव उपवेश्य देवस्थाने देवतां वा दार्भ बटुं वा\\
स्थापयेत् ।\\
एकेनापि हि विप्रेण षटूपिण्डं श्राद्धमाचरेत् ।\\
इति देवलोके ।\\
यद्येकं भोजयेच्छ्राद्धे दैवं तत्र कथं भवेत् ।\\
अनं पात्रे समुद्धृत्य सर्वस्य प्राकृतस्य च ॥\\
देवतायतने कृत्वा यथाविधि प्रवर्तयेत् ।\\
इति वृद्धवशिष्ठवचनात् । तदभावे कुशमयं स्थापयित्वा निमन्त्रयेदि\\
ति समुद्रकरधृतभविष्यचचनाच्च । पित्रे विषमसङ्ख्या एव ब्राह्मणा भवन्ती\\
त्युक्तम् । यत्तु आश्वलायनेन एकैकस्य द्वौ द्वौ त्रीस्त्रीन् वा वृद्धौ
फलभूय-\\
स्त्वं नत्वेवैकं सर्वेषां पिण्डैव्र्याख्यातं काममनाद्य इति पित्र्ये
सम-\\
सङ्ख्यकब्राह्मणविधानं कृतं तद्वृद्धिश्राद्धविषयमिति कल्पतरुप्रभृतयः ।\\
नत्वेवैकं सर्वेषामर्थे एकं न कुर्यादित्यर्थः । पिण्डैर्व्याख्यातं = यथा
सर्वे.\\
षामर्थे एकपिण्डो न भवति तद्वदित्यर्थः । काममनाथ = आद्य सपि\\
ण्डीकरणं तद्भिने कामं इच्छया एकमपि भोजयेदित्यर्थः । अनाद्ये=अ-\\
श्राभावे वा, अथवा अनाये - आमश्राद्ध इत्यर्थः । यदा चैकोऽपि भोज्यते\\
तदा प्रकारविशेषो बृहस्पतिनोक्तः-\\
यद्यकें भोजयेत् श्राद्धे स्वल्पत्वात् प्रकृतस्य च ।\\
स्तोकं स्तोकं समुद्धृत्य तेभ्योऽन्नं विनिवेदयेत् ॥\\
तेभ्यः = पितृभ्यः । उक्तेषु पक्षेषु विस्तरपक्षं सतक्रिया विधायकत्वेन\\
अनुकल्पपक्षप्रशंसार्थ निन्दति बृहस्पतिः ।\\
एकैकमथवा द्वौ त्रीन् दैवे पित्र्ये च भोजयेत् ।\\
सक्रियादेशकालादि न सम्पद्येत विस्तरे ॥ इति ।\\
अयं च निषेधो नादृष्टार्थः । हेतुनिर्देशात् तेन यो ब्राह्मणवा\\
हुल्येऽपि सक्रियादि सम्पादनसमर्थः तस्य गौतमीयोक्ताः\\
पक्षाः ज्ञेयाः ।

१०८  वीरमित्रोदयस्य श्राद्धप्रकाश-

{अथ निमन्त्रणपूर्वकालकृत्यम् ।\\
तत्रोशनाः-\\
गोमयेनोदकैश्च भूमिमार्जनं भाण्डशौच कृत्वा स्वः कर्तास्मीति\\
ब्राह्मणान् निमन्त्रयेत् । अत्र गोमयादिग्रहणं सकलशुद्धिसाधनद्र\\
ब्योपलक्षणम् ।\\
देवलः -\/-\\
श्वः कर्ताऽस्मीति निश्चित्य दाता विप्रान् निमन्त्रयेत् \textbar{}\\
निरामिषं सकृद् भुक्त्वा सर्वसुतजने गृहे ॥\\
असम्भवे परेधुर्वा ब्राह्मणांस्तान्नि मन्त्रयेत् \textbar{}\\
सुप्तेत्यत्र भुक्तेति मैथिलानां पाठः ॥\\
अत्र निश्चित्येत्यभिधानात् यत्र तीर्थश्राद्धादौ निश्चयाभावस्तत्र\\
न निरामिषसकृद्भोजनमङ्गम्, स्फ्याश्लिष्टज्याधिकरणानुरोधा\\
दिति । अत्र वचनान्नापूर्व भोजनं विधीयते किं तु रागतः प्राप्तभो\\
जनानुवादेन निरामिषत्वसक्कत्वरूपगुणमात्रं विधीयते अतश्चोपवा\\
सदिने न भोजनमिति गौडाः ।\\
अत्र च निमन्त्रणात् पूर्वे श्वः श्राद्धं करिष्यामीति संकल्पय\\
ब्राह्मणान् निमन्त्रयेदिति पैठीनसिवचनात् सङ्कल्पः कार्यः । स\\
चाचारादेवं कार्यः, देशकालौ संकीर्त्य अमुकामुकगोत्रनामकाना•\\
मस्मत् पित्रादीनां सदैवं सपिण्डं पार्वणश्राद्धद्वयं करिष्ये इति । एवं\\
सङ्कल्प्य निमन्त्रयेत् । तच्च पितृपूर्वकं कार्य "पितृपूर्व
निमन्त्रयेत्"\\
इति प्रचेतः स्मरणात् ।\\
यस्तु -\\
उपवीती ततो भूत्वा देवतार्थान् द्विजोत्तमान् ।\\
अपसव्येन पित्र्येऽथ स्वयं शिष्योऽथवा सुत. ॥\\
इति बृहस्पतिवाक्ये अथ शब्दो न स क्रमपरः, अथशब्दस्य\\
पाठक्रमवत्वेन श्रुतिक्रमापक्षया दुर्बलत्वात् ।\\
अन्येतु अथशब्दानुरोधात् विकल्प इत्याहु: ।\\
निमन्त्रणोत्तरं च नियमश्रावणमुक्तं मात्स्ये -\\
एवं निमन्त्र्य नियमान् श्रावयेत्पतृकान् बुधः ।\\
अक्रोधनैः शोचपरैः सततं ब्रह्मचारिभिः ।\\
भवितव्यं भवद्भिश्च मया च श्राद्धकारिणा ॥ इति ।

{ }{ निमन्त्रितानां नियमाः । १०९}{\\
अत्र निमन्त्रणीयब्राह्मणसमपिगमनादिनियमश्रावणान्तं प्रतिब्रा\\
ह्मणमनुसन्धेयमिति हेमाद्रिः । अङ्गीकृतनिमन्त्रणेन तु "आमन्त्रितो-\\
जपेद्दोग्नीम्" इति भृगुस्मृते: ``आब्रह्मन् ब्राह्मणो
ब्रह्मवर्चसीत्यादि\\
योगक्षेमो नः कल्पताम्" इत्यन्तानि यजूषि जप्तव्यानि । तानि जप्यप्र.\\
करणे वक्ष्यन्ते ।\\
अथ निमन्त्रितनियमा.\\
तत्र मनुः-\\
कोततस्तु यथान्यायं हव्यकव्ये द्विजोत्तमः ।\\
कथं चिदप्यतिक्रामन् पापः शुकरतां व्रजेत् ॥\\
केतितो= निमन्त्रितः । स्वीकृत्य निमन्त्रण यदि कामादिनातिक्रा\\
मेत्तदा सुकरयोनिप्रदं पापं प्राप्नुयादित्यर्थ' ।\\
केचित्तु प्रार्थ्यमानः सन् यदि अतिक्रामेत् नेच्छेन्निमन्त्रणमिति -\\
व्याचख्युः ।\\
ननु लिप्सया श्राद्धभोजने प्रवृत्तिर्न विधितः, तत्र विध्यपरा-\\
धाभावे कुतो दोष इति चेत् । न । ऋतुगमनवदुपपत्तेः । यथा हि\\
ऋतौ भार्यागमने रागतः प्रवृत्तिसम्मवेऽपि अगमने दोषश्रवणात्\\
क्रोधादिनाऽगच्छन् प्रत्यवतीति कल्प्यते, एवमिहापि सत्यपि श्राद्ध-\\
भोजनस्य रागतः प्राप्तत्वे उक्तवक्ष्यमाणनिन्दावचनैर्निमन्त्रणमति\\
क्रामन् प्रत्यवैतति कल्प्यते ।\\
यतः,\\
आमन्त्रितस्तु यो विप्रो भोक्तुमन्यत्र गच्छति ।\\
नरकाणां शतं गत्वा चाण्डालेष्वभिजायते ॥\\
अनिन्द्यामन्त्रणमवश्यमङ्गीकर्तव्यमित्याह देवल:-\/-\\
कर्मप्रतिश्रवस्तेषामनिन्द्यामन्त्रणे कृते ।\\
अनिन्द्येनामन्त्रितानां काममस्त्वित्येवं प्रतिअवोऽङ्गीकार एव\\
युक्त इत्यर्थः । इदं च शक्तविषयम्, अशक्तं प्रति विध्यप्रवृत्तेः । तथा\\
च ज्वराद्यभिभवेन भोक्तुमसामर्थ्य प्रत्याख्यानं कुर्वतोऽपि न दोषः ।\\
अनेनैवाभिप्रायेणाह गौतमः 1\\
अनिन्दितेनामन्त्रिते शक्तेन न प्रत्याख्यानं कर्तव्यमिति ।\\
शकेन =भोजन समर्थेनेत्यर्थः ।\\
एवं ब्राह्मण चतुर्मुखं कृत्वा देवताः पितृभ्यः सह तदनं समुपा•

{११० }{ वीरमित्रोदयस्य श्राद्धप्रकाशे-}{\\
इनन्ति तस्मात् स न व्यतिक्रामेदिति यमवाक्यमपि अनिन्द्यामन्त्रित.\\
शक्तविषयमेव व्याख्येयम् ।\\
कूर्मपुराणे,\\
आमन्त्रितो ब्राह्मणो वै योऽन्यस्मिन् कुरुते क्षणम् ।\\
स याति नरकं घोरं सुकरत्वं प्रयाति च ॥\\
मरस्यपुराणे,\\
आमन्त्रितास्तु गुणिनो निर्धनेनापि च द्विजा. ।\\
नान्यमिष्टान्नलोभेन तमतिक्रमयन्ति हि ॥\\
निमन्त्रितास्तु येनादौ तस्माद् गृह्णन्ति नान्यतः \textbar{}\\
अन्यस्य पुरुषस्य मिष्टं यदनं तल्लोभेनेत्यर्थः ॥\\
अत्र गुणिनेत्युपादानात् पूर्वमविदितदोषस्यामन्त्रणे स्वीकृते पश्चा\\
होषे विदिते प्रत्याख्यानं कुर्वन् नापराध्यतीति सूचितम् । कात्या\\
यनादिवचनेष्वपि निमन्त्रणङ्कर्तुरनिन्द्यताविशेषेण निन्द्यकृतामन्त्र.\\
णातिक्रमण न दोषायेत्येवमेवार्थे गमयति । तदेतत् सर्वमाढ्य विषयम्-\\
विद्यमानधनो विद्वान् भोज्याने न निमन्त्रितः ।\\
कथं चिद्ध्यतिक्रामन् पापः सुकरतां व्रजेत् ॥\\
इति षड्त्रिशन्मतेऽभिधानात् । अतोऽत्यन्तनिर्धनो बहुदक्षिणा-\\
दिलोभेन पूर्वनिमन्त्रणमतिक्रामन्नपि न दुष्यतीति गम्यते ।\\
केचित्तु सधनस्य दोषाधिक्यज्ञापनार्थमिदं वचनमिति व्या\\
चण्युः ।\\
निमन्त्रणं गृहीत्वान्यदनं न प्रतिगृह्णीयादित्याह कात्यायनः ।\\
आमन्त्रितोऽन्यदनं न प्रातेगृह्णीयात् ।\\
यच्छ्राद्धार्थ निमन्त्रितस्तदर्थादनादन्यत् श्राद्धव्यतिरेकेणापि येन\\
केन चिद्दीयमानमनं तदहर्न प्रतिग्राह्यमिति ।\\
यस्तु स्वीकृतनिमन्त्रणः केन चिन्निमित्तेन कुतुपादिकालातिपत्ति\\
करोति तस्य दोष आदित्यपुराणे उक्तः ।\\
आमन्त्रितश्चिरं नैव कुर्याद्विप्रः कदा च न ।\\
देवतानां पितॄणां च दातुरन्यत्र चैव हि ॥\\
चिरकारी भवेद्रोग्धा पच्यते नरकाग्निना ।\\
द्रोग्ध्रा-द्रोहकारी । देवतादीना द्रोहकारी भवेदित्यर्थः ।\\
याज्ञवल्क्यः,\\
तैयापि संवतैर्भाव्यं मनोवाक्कायकर्मभिः ।

 निमन्त्रितानां नियमाः । १११

{देवलः\\
तस्मात् दोषान् परित्यज्य त्रीनेतानपरानपि ।\\
ब्रह्मचारी शुचिर्भूत्वा श्राद्धं भुञ्जीत शक्तिमान् ॥\\
त्रीन् = स्त्री सम्भोगाऽन्यप्रतिग्रहपुनर्भोजनाख्यानू ।\\
यमः,\\
आमन्त्रितस्तु यः श्राद्धे वृषल्या सह मोदते ।\\
भवन्ति पितरस्तस्य तं मासं शुक्रभोजनाः ।\\
अत्र वृषलीशब्दः स्त्रीमात्रोपलक्षणार्थः । सामान्यत एव ब्रह्मच.\\
र्यस्य विधानात् । वृषं भर्तारं लाति स्वीकुरुते इति व्युत्पत्या
ब्राह्म-\\
ण्यपि वृषल्येवेति हेमाद्रौ व्याख्यातम् । शुद्रीगमने दोषाधिक्यप\\
राण्येतानि वचनानीत्यन्ये । अत्र च मैथुनं ऋतुमत्यामपि स्वभार्या.\\
यां न कर्तव्यम् । तथा नियुक्तेन देवतादिनापि न कर्तव्यम् ।\\
ऋतुकाले नियुक्तो यो नैवगच्छेत् स्त्रियं क्वचित् ।\\
तत्र गच्छन् समाप्तोति ह्यनिष्टं फलमेव तु ॥\\
इति निमन्त्रितं प्रक्रम्य वृद्धमनुकेः ।\\
कूर्मपुराणे -

{निमन्त्रितस्तु यो विप्रो ह्यध्वानं याति दुर्मतिः ।\\
भवन्ति पितरस्तस्य तं मासं मलभोजनाः ॥}{\\
यमोशनसौ-\/-\/-}

{आमन्त्रितस्तु यः श्राद्धे भारमुद्रद्दति द्विजः ।\\
पितरस्तस्य तं मासं भवन्ति स्वेदभोजनाः ॥}{\\
उशनाः-\\
}{आमन्त्रितस्तु यः श्राद्धे द्यूतं संसेवते द्विजः ।\\
भवन्ति पितरस्तस्य तं मासं मलभोजनाः ॥\\
आमन्त्रितस्तु यः श्राद्धे आयासं कुरुते द्विजः ।\\
भवन्ति पितरस्तस्य तं मासं पित्तभोजनाः ॥}{\\
यमः-\\
अहिंसा सत्यमक्रोधो दूरे चागमनक्रिया \textbar{}\\
अभारोद्वहनं क्षान्तिः श्राद्धस्योपासनाविधिः ॥\\
दूरे= सीम्नः परस्तात् । न गन्तव्यमित्यर्थः ।\\
ब्रह्माण्डपुराणेऽपि -\/-\/-\\
न सीमानमतिक्रामेत् श्राद्धार्थ वै निमन्त्रितः ।

{११२ }{ बीरमित्रोदयस्य श्राद्धप्रकाशे-}{\\
पर्यटन् सीममध्ये तु कदाचित्र प्रदुष्यति ॥\\
श्राद्धभुक् पुनर्भोजनादि न कुर्यादित्याह यमः -\\
पुनर्भोजनमध्वानं भाराध्ययनमैथुनम् ।\\
सन्ध्यां प्रतिग्रहं होमं श्राद्धभुग् वर्जयेत् सदा ॥\\
सदेति, निमन्त्रणक्षणमारभ्य श्राद्धाहोरात्रपर्यन्तम् । सन्ध्या\\
होमादौ विशेषो भविष्यतपुराणे -\/-\\
दशकृत्वः पिवेदापो गायत्र्या श्रद्धभुग् द्विजः ।\\
ततः सन्ध्यामुपासीत जपेच्च जुहुयादपि ॥ इति ।\\
अयं निमन्त्रितनियमः पूर्वदिने श्रद्धदिने वा निमन्त्रणमारभ्य\\
भवति निमन्त्रणप्रयुक्तत्वान्नियमजातस्येत्ति दिक् । इति निमन्त्रित
ब्राह्मण\\
नियमाः ।\\
अथ कर्तृनियमाः ।

{तत्र देवल:-\/-\/-\\
अक्रोधो निर्वृतः स्वस्थः श्रद्धावानत्वरः शुचिः ।\\
समाहितमनाः श्राद्धक्रियायामसकृद्भवेत् ॥\\
अक्रोधः क्रोधशुन्यः । उपलक्षणमेतनमात्सर्यादीनाम् । निर्वृत्तः-\\
सुप्रसन्नः । स्वस्थः =अव्याकुलीकृतचित्तः । असकृत् =श्राद्धसमाप्ति-\\
पर्यन्तम् ।\\
विष्णु:-\\
कोपं परिहरेत् नाश्रुपातयेत् न त्वरां कुर्यात् ।\\
मत्र यस्यामङ्गादिलोपः सम्भाव्यते तादृशी त्वरा निषिध्यते ।\\
प्रयोगप्राशुभावस्तु विध्यनुमत एवेति न निषिध्यते ।\\
पैठीनसिः-\/-\\
श्राद्धे सत्यं चाक्रोध च शौचं च त्वरांच प्रशंसति ।\\
वाराहपुराणे -\/-\/-\\
दन्तकाष्ठं व विसृजेत् ब्रह्मचारी शुचिर्भवेत् ।\\
दन्तप्रक्षालनार्थे दन्तकाष्ठं नादद्यादित्यर्थः । ब्रह्मचर्ये
चाष्टप्रका\\
रमैथुनवर्जनम् । तदाह गोभिलः ।\\
स्मरणं कीर्तन केलिः प्रेक्षणं गुह्यमाषणम् ।\\
सङ्कल्पोऽध्यवसायश्च क्रियानिर्वृतिरेव च ॥\\
एतन्मैथुनमष्टाङ्गं प्रवदन्ति मनीषिणः ।\\
विपरीतं ब्रह्मचर्यमेतदेवाष्टलक्षणम् ।

{ }{ कर्तृनियमाः । ११३}{\\
केलिः = स्त्रिया सह क्रीडा, आफलप्राप्तेः सुरतसम्पादनं क्रियानिर्वृ\\
तिः । अष्टाङ्गम् = अष्टप्रकारम् \textbar{}\\
व्यास:-\\
श्राद्धे यज्ञे च नियमे नाद्यात् प्रोषितभर्तृका ।\\
श्राद्धकर्तुर्निषेधोऽय न तु भोक्तुः कदाचन ॥\\
दन्तकाष्ठनिषेधे सति कर्तव्यं विशेषमाइ व्यासः -\/-\\
अलाभे दन्तकाष्ठानां निषिद्धायां तथा तिथौ ।\\
अपां द्वादशगण्डूषैर्विदध्याइन्तधावनम् ॥ इति ।\\
आदित्यपुराणे-\\
तदहस्तु शुचिर्भूत्वाऽक्रोधनोऽत्वरितो भवेत् ।\\
अप्रमत्तः सत्यवादी यजमानोऽथ वर्जयेत् ।\\
अध्वानं मैथुनं चैव श्रम स्वाध्यायमेव च \textbar{}\textbar{}\\
तदहः = श्राद्धीयेऽहनि । अध्वशब्देन लक्षणया सीमातः परस्तात्\\
गमनं कथ्यते । श्रमो=भारवाहनादिजन्य. केश: । स्वाध्यायशब्देनाध्य.\\
यनाध्यापने विवक्षिते ।\\
वृद्धमनुः,\\
निमन्त्रय विप्रांस्तदहर्वजयेन्मैथुनं क्षुरम् \textbar{}\\
प्रमत्ततां च स्वाध्यायं क्रोधाशौचं तथानृतम् \textbar{}\textbar{}\\
क्षुर=क्षुरकर्म ।\\
जावाल:-\\
ताम्बूलं दन्तकाष्ठं च स्नेह स्नानमभोजनम् ।\\
रत्यौषधपरानानि श्राद्धकर्ता तु वर्जयेत् ॥\\
स्नेहस्नानम्=अभ्यङ्गस्नानम् । अभोजनम् = उपवासः । आवश्यकोपवासे\\
प्राप्ते पितृसे वित्तमाघ्रायोपवसेत् ।\\
तथा च श्रुतिः "अवधेयमेव तत्रैव प्राशिनं तन्त्रैवाऽप्राशितम्'' इति ।\\
निमन्त्रित ब्राह्मणपरित्यागे दोषमाह नारायणः \textbar{}\\
केतनं कारयित्वा तु निवारयति दुर्मतिः ।\\
ब्रह्मवध्यमवाप्नोति शूद्रयोनौ च जायते ॥\\
एतस्मिन्नेनास प्राप्ते ब्राह्मणो नियतः शुचिः ।\\
यतिचान्द्रायणं कृत्वा तस्मात् पापात् प्रमुच्यते ॥\\
अयं च निषेधोऽदुष्टब्राह्मणत्याग इति बोध्यम् ।\\
यतिचान्द्रायणं नाम चान्द्रायणविशेषः । श्राद्धकर्त्रा च निम\\
१५ वी० मि

{११४ वीरमित्रोदयस्य श्राद्धप्रकाशे-\\
न्त्रणमारभ्याऽऽश्राद्धसमाप्तेराहारो वर्जनीय इत्याहापस्तम्बः,
'आरम्भे\\
वाभोजनमासमापनादिति' ।\\
गुरुतरकार्यव्यासङ्गादिना स्वयं श्राद्धं कर्तुमशक्नुवन् यदि\\
कदाचित पुत्रादिना श्राद्धं कारयेत् तदानेन श्राद्धाधिकारिणा च\\
नियमा अनुष्ठेया इत्युक्त वाराहपुराणे ।\\
न शक्नोति स्वयं कर्तुं यदा ह्यनवकाशतः ।\\
श्राद्ध शिष्येण पुत्रेण तदान्येनाऽपि कारयेत् ॥\\
नियमानाचरेत् सोऽपि विहितांश्च वसुन्धरे ।\\
यजमानोऽपि तान् सर्वान्नाचरेत्सुसमाहितः ॥\\
ब्रह्मचर्यादिभिर्भूमि ! नियमैः श्राद्धमक्षयम् ।\\
अन्यथा क्रियमाणं तु मोघमेव न संशयः ॥\\
भूमीति सम्बोधनम् । एते च नियमाः श्राद्धभुक्तान्नपाकान्तं\\
कर्तव्याः । तदाह लौगाक्षः -\/-\/-\\
विनीतः प्रार्थयन् भक्त्या विप्रानामन्त्र्य यत्नतः ।\\
श्राद्धं भुक्त्वान्नपाकान्तं नियमानाचरेत्ततः । इति श्राद्धकर्तृनियमाः
।\\
अधोभयनियमाः ।\\
तत्रादित्यपुराणे-\/-\/-\\
तां निशां ब्रह्मचारी स्यात् श्राद्धकृत् श्राद्धिकैः सह ।\\
अन्यथा वर्तमानौ तौ स्यातां निरयगामिनौ ॥\\
पद्मपुराणे ।\\
पुनर्भोजनमध्वानं भारमायासमैथुनम् ।\\
श्राद्धकृत् श्राद्धक् चैव सर्वमेतद्विवर्जयेत् \textbar{}\textbar{}\\
स्मयं च कलहं चैव दिवा स्वापं तथैव च ।

{हारीतः ।\\
आमन्त्रिता आमन्त्रयिता च शुचयस्तां रात्रि वसेयुः ।\\
शुचयो = मैथुन कामक्रोधादिरहिताः । मनुर्थमञ्च\\
निमन्त्रितो द्विजः पित्र्ये नियतात्मा भवेत्सदा ।\\
न च च्छन्दांस्यधीयीत यस्य श्राद्धं च तद् भवेत् ॥\\
यस्य च तत् कर्तव्यं श्राद्धं भवेत्सोऽपि नियतात्मा भवेदित्यर्थः ।\\
विष्णुपुराणे ।\\
ततः क्रोधव्यवायादीनायासं च द्विजैः सह ।\\
बज़मानो न कुर्वीत दोषस्तत्र महानयम् \textbar{}\textbar{}}

{ }{कर्तृभोक्तृनियमाः । ११५}{\\
अयम् = नियमाकरणरूपो महान् दोष इत्यर्थः । तत्र = क्रोधव्यवाया.\\
दौ, आयासे वा ।\\
श्राद्धे नियुक्तो भुक्त्वा वा भोजयित्वा नियुज्य च ।\\
व्यवायी रेतसो गर्ते मज्जयत्यात्मनः पितॄन् \textbar{}\textbar{}\\
नियुक्तो निमन्त्रितो ब्राह्मणः श्राद्धभोजनात् प्राकू श्राद्धभोजनो\\
त्तरकाल वा व्यवायी मैथुनकर्ता यदि भवति तदा आत्मनः स्वस्य\\
पितॄन् रेतोगर्ते मजयेत् । एवं श्राद्धकर्ताऽपि नियुज्य निमन्त्र्य
ब्राह्मण\\
भोजनात् पूर्व भोजयित्वा (१) ब्राह्मणभोजनानन्तरमपि वा भुक्तान्नपा-\\
नपरिणामावधि यदि व्यवायी स्यात् तदा सोऽपि तमेव दोषं\\
प्राप्नुयात् ।\\
प्रचेताः -\\
स्यादनपरिणामान्तं ब्रह्मचर्य द्वयोस्ततः ।\\
अन्नपरिणामान्तं =भुक्तान्नपरिणामान्तम् । तदाह बृहस्पति' ।\\
भुक्तान्नपरिणामान्तं नियमान्न विवर्जयेत् ।\\
निषिद्धं कुर्वतां दोषस्तद्वद्वैधमकुर्वताम् ॥ इति ।\\
वैधं विहितम् । वायुपुराणे ।\\
श्राद्धदाता च भोक्ता च मैथुनं यदि गच्छतः ।\\
पितरस्तु तयोर्मासं रेतोऽश्नन्ति न संशयः \textbar{}\textbar{}\\
उभयोर्नियमानुष्ठाने हेतुर्हांरीतेनोक. ।\\
पूर्वेद्युरामन्त्रितान् विप्रान् पितरः संविशन्ति वै\\
यजमानं च तां रात्रिं वसेयुर्नियतास्ततः ॥\\
कात्यायनः ।\\
तदहः शुचिरक्रोधनो त्वरितो ऽप्रमत्तः सत्यवादी स्यात् अध्व.\\
मैथुनश्रमस्वाध्यायांश्च वर्जयेत् आवाहनादि वाग्यत ओपस्पर्श\\
आमन्त्रिताश्चैवम् ।\\
नात्,\\
आवाहनादि = आवाहनप्रभृति \textbar{} ओपस्पर्शनात = उपस्पर्शनपर्यन्तम् ।
श्रा.\\
द्धकर्ता वाग्यतो मौनी स्यात् । उपस्पर्शन शब्देन कात्यायनसूत्रोक्तं\\
विसर्जनान्ते जलस्पर्शनं विवक्षितम् । आमन्त्रिताश्चैवमिति । शुचि\\
स्वादियुक्ता आवाहनादि उदस्पर्शनपर्यन्तं वाग्यताश्च भवेयुरित्यर्थः ।\\
इति उभयनियमाः ।

% \begin{center}\rule{0.5\linewidth}{0.5pt}\end{center}

( १ ) एतच भोजयित्वेत्यस्यैव विवरणं बोध्यम् ।

{११६ }{ वीर मित्रोदयस्य श्राद्धप्रकाशे-}{\\
अथ निमन्त्रितब्राह्मणानां श्राद्धभोजने नियमा ।\\
तत्र विष्णुः ।\\
अश्नीयुर्ब्राह्मणा न सोपानत्का न पीठोपनिहितपादाः । उपानहा=\\
चर्मपादुके । पीठात् बहिः कृतपादा अश्नीयुरित्यर्थः ।\\
प्रभासखण्डे ।\\
यश्च फूत्कारवद् भुङ्क्ते यश्च पाणितले द्विजः ।\\
न तदश्नन्ति पितरो यश्च वायुं समुत्सृजेत् ॥\\
फुफूत्कारवत = फूत्कारादिशब्दवत् । यश्च पाणितल इति पाणितले ग्रास\\
गृहीत्वा न भोक्तव्यम्, तेनाङ्गुल्यत्रैर्मासग्रहणं कर्तव्यमिति विवक्षितम्
।\\
वायुसमुत्सर्जनम्=अपानवायूत्सर्जनम् । भोजने अन्ननिन्दादि न कर्तव्य\\
मित्याह प्रचेताः ।\\
पीत्वापोशानमश्नीयात् पात्रे दत्तमगर्हितम् ।\\
सर्वेन्द्रियाणां चापल्यं न कुर्यात् पाणिपादयोः ॥\\
अत्युष्णं सर्वमन्नं स्याद् भुञ्जीरंस्ते च वाग्यताः ।\\
न च द्विजातयो ब्रूयुर्दात्रा पृष्ठा हविर्गुणान् \textbar{}\textbar{}\\
सर्वेन्द्रियाणां चापल्यं =भोजनार्थव्यापारातिरिक्तव्यापारः, इन्द्रि\\
यग्रहणेन पादयोर्ग्रहणे सिद्धे पुनस्तद्ग्रहणं पाणिपादचापल्यमाधिक\\
दोषजनकमिति ज्ञापनार्थम् । अत्युष्णम् = ईषदु णम् । अत्युष्णस्य भोक्तु.\\
मशक्यत्वात् । वाग्यताः=वाग्व्यापाररहिताः । तेन शब्दोचारण न क.\\
र्तव्यमित्यभिप्रेयते । दात्रेत्युपलक्षणम् अन्येनापि पृष्टा हविर्गुणान्\\
श्राद्धीयानगुणा ब्रूयुः । वाग्यतानामप्रसक्तं हविर्गुणवचनं न निषेध्य.\\
मिति वेदभिनयादिनापि हविर्गुणप्रकाशनं न कर्तव्यमित्येतदर्थत्वाद\\
स्य वचनस्य । ततश्च ब्रूयुरिति लक्षणया प्रकाशयेयुरिति व्याख्येयम् ।\\
अत्रिरपि ।\\
अंकारेणापि यो ब्रूयाद्धस्ताद्वापि गुणान् वदेत् ।\\
इति वर्जनीयाधिकारे हस्तग्रहणं कुर्वन् इममेवार्थमभ्यसूचयत् ।\\
देवलः ।\\
स्वन्नपानकशीतोदं ददन्द्रयो ह्यवलोकितः ।\\
वक्तव्ये कारणे संज्ञां कुर्वन भुञ्जीत पाणिना ॥\\
अस्य श्लोकस्यार्थ, हेमाद्रौ इत्थं विवृतः, अपेक्षानुसारेणान्न\\
पानशीतलवार्यादिदातृभिरपेक्षां ज्ञातुमवलोकितः क्षुत्पिपासालक्षणे

{निमन्त्रित ब्राह्मणानां भोजननियमाः । ११७}{\\
अपेक्षादेः कारणे वक्तव्ये पाणिना संज्ञासङ्केतमपेक्षादिसुचकं कुर्वन्\\
भुञ्जीतेति ।\\
मनुः ।\\
यावदन्नं भवत्युष्णं यावदश्नन्ति वाग्यताः ।\\
तावदश्नन्ति पितरो यावन्त्रोका हविर्गुणाः ॥\\
( अ० ३ श्लो० २३७ )

{अत्रि ।\\
हुङ्कारेणापि यो ब्रूयात् हस्ताद्वाऽपि गुणान् वदेत् ।\\
भूतलाबोद्धरेत् पात्रं मुञ्चेद्धस्तेन वा पिवन् ॥\\
प्रौढपादो बहिःकक्षो बहिर्जानुकरोऽथवा ।\\
अङ्गुष्ठेन विनाश्नाति मुखशब्देन वा पुनः ॥\\
पीत्वाऽवशिष्टतोयानि पुनरुद्धृत्य वा पिबेत् ।\\
खादितार्थान् पुनः खादेन मोदकानि फलानि वा ॥\\
मुखेन वा धमेदन्नं निष्ठीवेद्भाजनेऽपि वा ।\\
इत्थमश्नन् द्विजः श्राद्ध हत्वा गच्छत्यधोगतिम् ॥\\
पात्रमित्यस्य मुञ्चेदित्यत्राप्यन्वयः । प्रौढपादस्तु -\\
आसनारूढपादस्तु प्रौढपादः स उच्यते ॥\\
इति भविष्यपुराणेऽभिहितः । बहिःको = बहिर्भूतकक्षद्वयः ।\\
वायुपुराणेऽपि ।\\
यावन्न स्तूयते चान्नं यावदोष्ण न मुञ्चति ।\\
तावदश्नन्ति पितरो यावदश्नन्ति वाग्यताः ॥\\
प्रभासखण्डे ।\\
रसा यत्र प्रशस्यन्ते भोक्तारो बन्धुगोत्रिणः ।\\
राजवार्तादिसंक्रन्दो रक्षःश्राद्धस्य लक्षणम् ॥\\
भोकारो=बन्धुगोत्रिणः । पित्रादिस्थाने भोकार इत्यर्थः । श्राद्ध-\\
शेषभोजनं तु तेषां विहितमेव । रसशंसादि, रक्षः श्राद्धस्य लक्षण\\
सूचकम, तत् आद्धं पितृतृप्तिकरं न भवतीत्यर्थः । हविर्गुणप्रशंसा-\\
निषेधस्तु श्राद्धसमाप्तेः पूर्वमेवेत्यभिप्रेत्याह वृद्धवशिष्ठ,\\
श्राद्धावसाने कर्तव्या द्विजैरन्नगुणस्तुतिः \textbar{}\\
निगमः,\\
मानपानादिकं श्राद्धे वारयेन्मुखतः कचित् ।\\
अनिष्टत्वाद बहुत्वाद्वा वारणं हस्तसंज्ञया \textbar{}\textbar{}}

{११८ }{ वीरमित्रोदयस्य श्राद्धप्रकाशे-}{\\
अनिष्टत्वात्=अनपेक्षितत्वात् । अपेक्षितस्याऽपि वा पात्रस्थस्य\\
बहुत्वात् यदा अन्नादि वारयेत् तदा न मुखतः - न शब्दप्रयोगेण,\\
किन्तु हस्तसंज्ञया हस्तसंकेतेनेत्यर्थः ।\\
शलिखितौ ।\\
ब्राह्मणा अन्नगुणदोषौ नाभिवदेयुः, नातृप्तं ब्रूयुरन्योन्यं न प्र\\
शंसेयुरन्नपान न प्रभूतमिति वदेयुरन्यत्र हस्तसंज्ञया । पात्रे प्रभूत\\
मन्नमस्त्यन्यन्न परिवेष्यमिति भोक्तृभिर्त वक्तव्यं किं तु हस्तसंकेतेन\\
सुचनीयमित्यर्थः । अपेक्षित चावश्यं याचनीयमेवेत्याह वृद्धशातातपं ।\\
अपेक्षितं याचितव्यं श्राद्धार्थमुपकल्पितम् ।\\
न याचते द्विजो मूढः स भवेत् पितृघातकः ॥\\
मनुः ।\\
यद्वेष्टितशिरा भुङ्गे यद् भुङ्गे दक्षिणामुखः ।\\
सोपानत्कश्च यद् भुते तद्वै रक्षांसि भुञ्जते ॥\\
( अ० ३ श्लो० २३८ )\\
वेष्टनमुष्णीषादिना । अत्र दक्षिणमुखताया निषेधात् वचना\\
स्तरविहितोदङ्मुखताया असम्भवे प्रतिनिधित्वेन दिगन्तराभिमुख.\\
तेति अवसीयते, निषिद्धम्य प्रतिनिधित्वानुपपत्तेः ।\\
पात्रोद्धरणे दोष उक्तो वाराहपुराणे ।\\
उद्धरेद्यदि पात्रं तु ब्राह्मणो ज्ञानवर्जितः ।\\
हरन्ति राक्षसास्तस्य भुञ्जतोऽन्नं च सुन्दरि ॥\\
भुञ्जानैः बुद्धिपूर्वमितरेतरस्पर्शो न कर्तव्य इत्युक्त वायुपुराणे ।\\
श्राद्धे नियुक्ता ये विप्रा दम्भं क्रोधं च चापलम् ।\\
अन्योन्य स्पर्शनं कामाद्वर्जयेयुर्मदंं तथा ॥\\
प्रमादादन्योन्यस्पर्शे कि कर्तव्यमित्यपेक्षित आह शङ्खः ।\\
श्राद्धपङ्कौ तु भुञ्जानो ब्राह्मणो ब्राह्मण स्पृशेत् ।\\
तद्नमत्यजन् शुक्ला गायत्र्यष्टशतं जपेत् ॥ ब्राह्मण भुञ्जानम् ।\\
वशिष्ठः ।\\
नियुक्तस्तु यदा श्राद्धे दैवे वा मांसमुत्सृजेत् ।\\
यावन्ति पशुरोमाणि तावन्नरकमृच्छति ॥

{यमः\\
नियुतचैव यः श्राद्धे यत्किचित परिवर्जयेत् ।\\
पितरस्तस्य तं मासं नैराश्यं प्रतिपादरे ॥ नैराश्यम्=अनशनम् ॥

  निमन्त्रितब्राह्मणानां भोजननियमाः । ११९

{प्रचेता. ।\\
न स्पृशेद्वामहस्तेन भुञ्जानोऽन्नं कदाचन ।\\
न पादौ च शिरो वस्ति न पदा भाजनं स्पृशेत् \textbar{}\textbar{}\\
वस्ति = मूत्रपुरीषस्थानम् ।

{वसिष्टः ।\\
उभयोर्हस्तयोर्मुक्तं पितृभ्योऽन्नं निवेदितम् ।\\
तदन्तरं प्रतीक्षन्ते ह्यसुरा दुष्टचेतसः ॥\\
तस्मादशून्यहस्तेन कुर्यादन्नमुपागतम् ।\\
भाजनं वा समालभ्य तिष्ठेदोच्छेषणाद् द्विजः ॥\\
उभयोर्ब्राह्मणहस्तयोरन्यतरेणापि यदा तदनमनधिष्ठितं भवति\\
तदन्तरालमसुराः सर्वदा प्रतीक्षन्ते, लब्धान्तराञ्च तद्नमपहरन्ती\\
स्वर्थ: । तस्मात् पितृभ्यो निवेदितमनं भोजनसमाप्तिपर्यन्तं धामह\\
स्तेनाशुन्यं कुर्यात् कण्डूयनाद्यर्थे वामहस्तव्यापारसमये तु दक्षिण\\
हस्तेन भाजनं समालभ्य वर्तेत इति तात्पर्यार्थ इति हेमाद्री व्याख्या
\textbar{}\\
अन्ये त्वेवं व्याचक्षते अन्नमुभयोर्हस्तयोरन्यतरेणापि यथा मुक्तं न\\
भवति तथा कर्तव्यम् । ततश्च भाजनस्थेऽन्ने दक्षिणस्य व्यावृतो\\
तेनैवाशममुक्तं भवति, दक्षिणव्यावृत्यमावे तु उक्तप्रचेतोवाक्येन\\
वामेन हस्तेनान्नस्पर्शस्य निषेधाद् वामेन भाजनं समालभ्य वर्तते ।\\
क्वचित् उभयोः शाखयोर्मुकमिति पाठः । तत्र शाखयोर्हस्तयो-\\
रित्येवार्थ ।\\
जातूकर्ण्योऽपि,

{भाजने परिशिष्टानं हस्तेन ब्राह्मणः स्पृशेत् ।\\
रक्षोभ्यस्त्रायते यस्मात् धरणीयं प्रयत्नतः ॥\\
यस्मात् स्पृशन् ब्राह्मणो रक्षोभ्यः श्राद्धानं त्रायते तस्मादित्यर्थः
।\\
"ब्राह्मणो हि रक्षसामपहन्ता" इति शतपथश्रुतेः । देवतोद्देशेनान\\
त्यागात् प्राकू न परिविष्टानं हस्तेन स्प्रष्टव्यमित्याहात्रिः ।\\
असङ्कल्पितमन्नाद्यं पाणिभ्यां य उपस्पृशेत् ।\\
अभोज्यं तद् भवेदन्नं पितॄणां नोपतिष्ठते ॥\\
असंकल्पितं = देव तोद्देशेनात्यक्तम् । 

{निगमः\\
मांसापूपफलेक्ष्वादि दन्तच्छेदं न भक्षयेत् ।\\
ग्रासशेषं न पात्रेऽस्येत् पीतशेषं तु नो पिबेत् ॥\\
~\\
-

{१२० }{ वीरमित्रोदयस्य श्राद्धप्रकाशे-}{\\
हस्तेन मांसादि धृत्वा स्वल्पं दन्तैश्छित्वा न भक्षयेत् । नास्येत्य\\
न निक्षिपेत ।\\
प्रचेताः -\\
दन्तच्छेदं हस्तपानं वर्जयेश्चातिभोजनम् ।\\
हस्तपानं= हस्तेन जलादिपानम्, न कुर्यादित्यर्थः ॥\\
बहूचपरिशिष्टे,\\
यच पाणितले दत्तं यच्चान्नमुपकल्पितम् ।\\
एकीभावेन भोक्तव्यं पृथक्भावो न विद्यते ॥\\
पाणितले दत्तमग्नौकरणान्नम्, उपकल्पितं भोजनपात्रेषु पित्राद्यु.\\
देशेन निहितं, तत् उभयमपि एकीकृत्य मिश्रयित्वा भोक्तव्यम् ।\\
जमदग्निः । न च्छिन्द्युर्नावशेषयेयुः ।\\
छेदनं=दन्तच्छेदनम् । नाव शेषयेयुः = तृप्तेः प्राक् क्रोधादिनानं न
त्यजे.\\
युरित्यर्थः ।\\
नदाह सुमन्तुः ।\\
आतृप्तेर्भोजनं तेषां कामतो नावशेषणम् ।\\
तृप्तौ जातायां तु किश्चिदवशेषणयिम् ।\\
तदाह जमदग्निः \textbar{}\\
अन्यत् पुनरुत्स्रष्टव्यं तस्यासंस्कृतप्रमीतानां भागधेयत्वात्\\
अन्यद् - दध्यादिभ्यो निरविशेषभोज्येभ्यः ।\\
भोजनपात्रेण जलं न पिबेदित्याह । शातातप ।\\
अर्धभुक्ते तु यो विप्रस्तस्मिन् पात्रे जलं पिबेत् ।\\
यद् भुक्तं तत् पितॄणां तु शेषं विद्यादथासुरम् ॥\\
इति श्राद्धभोक्तृनियमा ।\\
अथ भोजयितृधर्माः ।\\
तत्र याज्ञवल्क्यः । ( अ० १ श्रा० प्र० श्लो० २४० )\\
अन्नमिष्ट हविष्यं च दद्यादक्रोधनोऽत्वरः ।\\
इष्टं ब्राह्मणानां, स्वस्य, उद्देश्य पित्रादीनां च, वाक्चापल्यराईना.\\
नामपि ब्राह्मणानां हस्तसंकेतादिना यद्यदिष्टं
जानयात्तत्तद्दद्यादित्यर्थः।\\
अपेक्षिताऽप्रदाने दोषमाह -\\
वृद्धशातातपः ।\\
अपेक्षितं यो न दद्यात् श्राद्धार्थमुपकल्पितम् ।\\
अवः कृच्छासु घोरासु तिर्यग्योनिषु जायते ॥

{ }{ भोजयितृनियमाः । १२१}{\\
मनुः । ( अ० ३ श्लो० २३३ )\\
हर्षयेत् ब्राह्मणांस्तुष्टो भोजयेश्वाशन शनैः ।\\
अन्नाद्येनासक्कश्ञ्चैनान् गुणैश्च परिचोदयेत् ॥\\
भोजयेचाशन शनैरिति = मन्दं मन्दं भोक्तव्यमिति ब्राह्मणान् प्रेर\\
येदित्यर्थः ।\\
तथा,\\
यद्यद्रोचेत विप्रेभ्यस्तत्तद्दद्यादमत्सरः ।\\
ब्रह्मोद्याश्च कथाः कुर्यात् पितॄणामेतदीप्सितम् ॥\\
ब्रह्मोद्याः = कश्चिदेकाकी चरतीत्याद्याः । अथवा ब्रह्मप्रधाना मन्त्रा\\
र्थवादादय इति हेमाद्रौ । ब्राह्मणान् हविर्गुणदोषौ न पृच्छेदिश्याह-\/-\\
राज्ञः ।\\
श्रद्धे नियुक्तान् भुञ्जानान् न पृच्छेल्लवणादिषु ।\\
उच्छिष्टाः पितरो यान्ति पृच्छतो नात्र संशयः ॥\\
दातुः पतति बाहुर्वै जिह्वा भोक्तुस्तु भिद्यते ।\\
एवं च लवणादिन्यूनाधिकभावस्य ज्ञानार्थ न ब्राह्मणाः प्रष्टव्याः,\\
किन्तु ब्राह्मणकृतहस्तसङ्केतादिना जानीयादित्यर्थः । न च लवणा.\\
दि भुञ्जानेभ्यो दद्यात् । "दातुः पतति" इत्यादिना दोषश्रवणात् ।\\
स्मृत्यर्थसारे-\\
ओदन स्पपायस भैश्यपानादिकं च बहु परिवेष्यम् । हिङ्गुशुण्ठी.\\
पिप्पलीमरीचानि द्रव्यसंस्कारार्थानि आद्धे स्युः प्रत्यक्षेण न भक्ष-\\
णीयानीति ।\\
यमः-\\
निराकारेण यद् भुकं परिविष्टं समन्युना ।\\
दुरात्मना च यद् भुतं तद्वै रक्षांसि गच्छति ॥\\
अवेदव्रतचारित्रास्त्रिभिर्वर्णैर्द्विजातयः ।\\
मन्त्रवत् परिविष्यन्ते तद्वै रक्षांसि गच्छति ॥\\
विधिहीनममृष्टानं मन्त्रहीनमदक्षिणम् ।\\
अश्रद्धया हुतं दत्तं तद्वै रक्षांसि गच्छति ॥

{मनुः-\\
नास्त्रमापातयेज्जातु न कुप्येनानृतं वदेत् ।\\
न पादेन स्पृशेदन्नं न चैतदवधूनयेत् ॥\\
(अ० ३ श्लो० २२९)

१६ वी० मिन

{१२२ }{ वीरमित्रोदयस्य श्राद्धप्रकाशे-}{\\
अस्रं = रोदनम् । नानृतं वदेदिति पुरुषार्थतया निषिद्धस्याप्यनृत.\\
वदनस्य कर्मार्थोऽयं निषेधः । एवं न पादेन स्पृशेदत्रमित्येवमा.\\
दिग्वपि निषेधेषु बोद्धव्यम् ।\\
ब्रह्माण्डपुराणे,\\
न चानुपात येज्जातु न (१) शुक्ताङ्गिरमीरयेत् ।\\
न चोद्वीक्षेत भुञ्जानान् न च कुर्वीत मत्सरम् ॥\\
शुकां = शोकवतीम् । न चोद्वीक्षेत अनवरतमिति शेषः ।\\
समः-\/-\\
इष्टं निवेदितं दत्तं भुक्तं जप्तं तपः श्रुतम् ।\\
यातुधानाः प्रलुम्पन्ति शौचभ्रष्टं द्विजन्मनः ॥\\
तथा क्रोधेन यद्दत्तं भुक्तं यत्त्वरया पुनः ।\\
उभयं तद्विलुम्पन्ति यातुधानाः सराक्षसाः ॥\\
पितॄनावाहयित्वा तु नायुक्तप्रभवो भवेत् ।\\
तस्मानियम्य वाचं च क्रोधं च श्राद्धमाचरेत् ॥\\
न क्रोधं कस्य चित् कुर्यात् कस्मिश्चिदपि कारणे ।\\
अक्रुद्धपरिविष्टं हि श्राद्धे प्रीणयते पितॄन \textbar{}\textbar{}\\
शौचभ्रष्ट = शौचरहितेन कृतं वृथा भवतीत्यर्थः । अयुतप्रभवः = अयुकं\\
नियमायोगः यथेष्टाचरणं तस्य प्रभव उत्पत्तिर्यस्मादिति स\\
तथोक्तः । कस्मिंश्चिदपि कारणे= क्रोध कारणे सतीत्यर्थः ।\\
विष्णुः ।\\
नान्त्रमासनमारोपयेन्त्र पदा स्पृशेत्, नावक्षुतं कुर्यात् । आसन-\\
ग्रहणं आधारोपलक्षणार्थम् \textbar{} अन्नम्-अन्नपात्रम् \textbar{}
यन्त्रादिषु नारोपये•\\
दित्यर्थः । नावक्षुतं कुर्यात्=अनोपरि क्षुतं न कुर्यादित्यर्थः ।\\
कात्यायनः ।\\
श्रद्धान्वितः श्राद्धं कुर्वीत शाकेनाऽपि ।\\
देवलः ।\\
नाश्रु वा पातयेच्छ्राद्धे न जल्पेश इसेन्मिथः ।\\
म विभ्रमेन संकुभ्येनोद्विजेऽचात्र कर्हि चित् \textbar{}\textbar{}\\
प्राप्ते हि कारणे श्राद्धे नैव क्रोधं समुच्चरेत् ।\\
आभितः स्विगात्रो वा न तिष्ठेत् पितृसन्निधौ ॥

% \begin{center}\rule{0.5\linewidth}{0.5pt}\end{center}

{\\
(१) शुष्कामिति कमलाकरोद्धृतः पाठः ।

{ }{प्राचीनावीतयज्ञोपवतिविचारः । १२३}{\\
न चात्र श्येनकाकादीन् पक्षिणः प्रतिषेधयेत् ।\\
तद्रूपाः पितरस्ते हि समायान्तीति वैदिकम् ॥\\
क्रोधं न समुच्चरेश कुर्यात् । आश्रितः भित्तिस्तम्भाद्याश्रित्याव.\\
स्थितः । श्येनका कादिप्रतिषेधनिषेधस्तीर्थश्राद्धविषयः, अन्यथा 'क्र.\\
व्यादाः पक्षिणः श्राद्धं नेक्षेरन्' इत्यादिवचनविरोधः । तीर्थश्राद्धे व\\
काकादिनिवारणं न कर्तव्यमिति वचनान्तरवशेन प्रमितमिति हेमाद्रौ ।\\
इति भोजयितृनियमाः । इति श्राद्धीयनियमाः ।\\
अथ प्राचीनावीत यज्ञोपवीतविचार. \textbar{}\\
तत्र श्राद्धे यत्तावदचत्यागादि प्रधानं, तत् "प्राचीनावीतं पितॄणाम्"\\
इति श्रुतेः, प्राचीनावीतिनैव कार्यम्, तदङ्गभूतास्तु पदार्था द्विविधाः,\\
विहिता भविहिताञ्च द्विविधा अपि पुनस्त्रिविधाः, पितृमात्रसम्ब\\
न्धिनो, देवमात्र सम्बन्धिन, उभयसम्बन्धिनश्चेति । तत्र ये तावद्वि-\\
हितद्द्रव्याद्यर्थाक्षिप्ताः क्रयादयस्तेषां श्राद्धाङ्गत्वाभावान कोऽपि\\
श्राद्धकतो नियमः । " नित्योदकी नित्ययज्ञोपवीती" इत्यादिना परं\\
तत्र यज्ञोपवीतं प्राप्यते । एवं येषु केवलदेवसम्बन्धिष्वपि पदार्थेषु\\
यज्ञोपवीतमेव "यज्ञोपवीतिना कार्य दैवं कर्म विजानता" इत्यादि-\\
वचनैः प्राप्तं तेष्वपि न पितृदेवत्यश्राद्धाङ्गत्वमात्रेण
प्राचीनावीतप्र\\
सक्तिः । येषु तु केवलपितृसम्बन्धिषु न साक्षाद्यज्ञोपवीतप्रसक्तिः,\\
येषु तु केवलपितृलम्बन्धिषु न साक्षाद्यज्ञोपवीतविधानं तेषु प्रयो.\\
गविधिबलात् प्रधानद्वारा प्राप्तं प्राचीनावीतमेव कार्यम् ।\\
यत्तु दैवपितृसाधारणं यथा पाकप्रोक्षणादि, तत्र यद्यपि सामा•\\
म्यतो देवसम्बन्धित्वेन यज्ञोपवीतमपि प्राप्यते तथापि प्रधानधर्मः\\
प्राचीनावीतमेव युकः । वस्तुतस्तु पताडशस्य न दैवत्वम्, अन्यला\\
पेक्षत्वेन "दैवं कर्म विजानता" इत्यत्र तद्धितोत्पत्ययोगात् ।\\
नन्वेवं निरामिषसकृद्भोजनेऽपि प्राचीनावीतं प्रसज्येत, तस्य\\
पित्र्यत्वात्, नचेष्टापत्तिः, शिष्टाचारविरोधादिति चेत् ।\\
अत्र केचित् । रागतः प्राप्तभोजनानुवादेन गुणमात्रं विधीयते\\
म भोजनमपि, अतश्च भोजने न तत् प्रसक्तिरिति । तन्न । निरामि.\\
बत्व सकृत्वा द्यनेकगुणोपादानात् वाक्यभेदापत्तेर्भोजनस्यैव वि.\\
घेयत्वात् । तस्मादेषं वर्णनीयं निरामिषसकद्भोजनस्य सहड्रोज

{१२४ }{वीरमित्रोदयस्य श्राद्धप्रकाशे-}{\\
नत्वसामान्यादेकभक्तविकारत्वम् । अत एव एकभक्त शब्दप्र.\\
योगस्तत्र श्रीदत्तादीनाम् । एकभकस्य च नित्योदकीत्यनेन यज्ञोपवी.\\
तानाङ्गकत्वात्, अत्राप्यतिदेशेन तदेव प्राप्यते । न चौपदेशिकेनाति\\
देशे कस्य बाघ इति वाच्यम् । प्रयोगविध्याश्रितस्यातिदेशिकापेक्षया\\
दुर्बलत्वाद, तस्य प्रयोगबहिर्भूतत्वाच । तथा च पाके न तत् प्रस.\\
तिः । तस्य प्रायशः पत्न्यादिकर्तृत्वेन तत्र प्राचीनावीतासम्भवात् ।\\
यजमानकर्तृके पाके प्राचीनावीतविधाने नित्यानित्यसंयोगविरो.\\
धापन्तेरिति हेमाद्रिः ।\\
वस्तुतस्तु पाके यजमानस्यैव मुख्यत्वात् पत्म्यादेस्तद्भावे प्र.\\
तिनिधित्वस्य वक्ष्यमाणत्वान्न नित्यानित्यसंयोगप्रसक्तिः, अन्यथा\\
पुत्रकर्तृकौ ऽर्ध्वदेहिकेऽपि न प्राचीनावीतत्वप्रसङ्गः तत्रापि तदद्भावे\\
पत्न्यादे: कर्तृत्वात्, तस्माद्यदि तत्र यज्ञोपवीताचारस्तदा स एव\\
शरणम्, नोचेत् प्राचीनावीतमेव साधु \textbar{} तथा श्राद्धाङ्गभृतयोः
स्ना\\
नाचमनयोरपि न प्राचीनावतिम् । अन्यथा तत्र तस्मिन्ननुष्ठीयमाने\\
भाद्धविकृतिभूतायां प्रेतक्रियायामपि प्राचीनावीतप्राप्तौ प्रयोगवि\\
घिबलात्तदङ्गस्नाने तत्प्राप्तेः, तत्र पुनः "एकवस्त्राः प्राचीनावीतिनः"\\
इत्यादिना तद्विभ्यानर्थक्यप्रसङ्गात् । अङ्गान्तरपरिसंख्यायें स\\
इति चेत् । शेषपरिसंख्यापेक्षया शेषिपरिसंख्याया विधे.\\
यगतत्वेन प्रधानाङ्गबाधाभावेन च वक्तुमुचितत्वात् । तस्मात्\\
प्रेतक्रियायां प्राचीनावीतविधानान्तदतिरिक्तश्राद्धाद्यङ्गभूतस्ना.\\
नाचमनयोर्यज्ञोपवीतमेवेति सिद्धम् । एवं प्राणायामेऽपि यज्ञोप\\
वीतं कार्यम्, आचारात, प्रयोगबहिर्भूतत्वाच्च । श्राद्धसंकल्पपूर्वभा.\\
विजपस्तु यद्यपि प्रयोगबहिर्भूतत्वादुपवीतिन एव प्राप्तस्तथापि\\
अपसव्यं ततः कृत्वा मन्त्रं जप्त्वा च वैष्णवम् ।\\
गायत्रीं प्रणवञ्चापि ततः श्राद्धमुपक्रमेत् ॥\\
इति प्रचेतोवचनात प्राचीनावीतिनैध कार्यम् । परिवेषणं तु पितृ-\\
पात्रेष्वपि उपवीतिनैव कार्यम् ।\\
अपसव्येन मस्त्वन्नं ब्राह्मणेभ्यः प्रयच्छति ।\\
विष्ठामश्नन्ति पितरस्ते च सर्वे द्विजोत्तमाः ॥\\
इति कार्ष्णाजिनिस्मरणात् । अतिस्तु सर्वे यज्ञोपवीतेनैव कार्यमिति\\
हेमाद्रिः । पथमन्यत्रापि सर्वत्र ।

{ }{ यजमानजप्यानि । १२५}{\\
(१) यज्ञोपवीतिना सम्यगपसव्यमतन्द्रिणा ।\\
पित्र्यमानिधनात् कार्य विधिविधर्मपाणिना ॥\\
( अ० ३ श्लो० ३७९ )\\
इति मनुना कर्तव्यपदार्थष्वेव प्राचीनावीतविधानात् । अत एव\\
ब्राह्मणकर्तृकेष्वपि पदार्थेषु न प्राचीनावीतम् । श्राद्धकर्तारं
प्रत्येव\\
प्राचीनावीतावीघप्रवृत्तेः । एव श्राद्धशेषभोजनेऽपि न तत् प्रयोग\\
बहिर्भूतत्वादिति दिक् ।\\
अथ सामान्यतः श्राद्धेयपदार्थाः ।\\
कात्यायनः.\\
दक्षिणं पातयेज्जानु देवान् परिचरन् सदा ।\\
पातयेदितरं जानुं पितॄन् परिचरन् सदा ॥\\
प्रदक्षिण तु देवानां पितृणामप्रदक्षिणम् ।

{शातातपः,\\
उदङ्मुखेन देवानां पितॄणा दक्षिणामुखः \textbar{}\\
देवानामृजवो दर्भाः पितॄणां द्विगुणाः स्मृताः ॥\\
अथ जप्यानि ।\\
तत्र तावद्यजमानजध्यानि ।\\
तत्र पद्मपुराणे ।\\
स्वाध्यायं श्रावयेत् पैत्र्ये पुराणानि खिलानि च ।\\
ब्रह्मविष्ण्वर्करुद्राणा स्तोत्राणि विविधानि च ॥\\
खिलानि = श्रीसूक्त महानास्तिकादीनि । एतानि भुञ्जानान् ब्राह्मणान्\\
आवयेत् ।\\
मनु\\
स्वाध्यायं श्रावयेत् पैत्र्ये धर्मशास्त्राणि चैव हि ।\\
आख्यानानीतिहासांत्र पुराणानि खिलानि च ।\\
( अ० ३ श्लो० २३२ )\\
आख्यानानि = लौपर्ण मैत्रावरुणपारिनुवादीनि बाहुच्ये पठ्यन्ते । इति-\\
हासा = महाभारतादयः । अयं च जप उपवीतिना कार्यः ।\\
तथा च ब्रह्माण्डपुराणे ।

% \begin{center}\rule{0.5\linewidth}{0.5pt}\end{center}

{\\
(१) प्राचीनावीतिनेति मनुस्मृतौ पाठः ।

{१२६ }{ वीरमित्रोदयस्य श्राद्धंकाशे-}{\\
कुशपाणि: कुशासन उपवीतो जपेत्ततः ।\\
वेदोकानि पवित्राणि पुराणानि खिलानि च ।\\
अयं च जपो ब्राह्मणानां पुरतः कार्यः ।\\
यमः,\\
स्वाध्यायं श्रावयेत् सम्यक् धर्मशास्त्राणि चासकृत् ।\\
इतिहासांश्च विविधान् कीर्तयेत्तेषु चामतः ॥\\
हर्षयेद् ब्राह्मणान् दृष्टो भोजयेदशनं शनैः ।\\
प्रचेताः ।\\
भुखानेषु तु विप्रेषु ऋग्यजुःसामलक्षणम् \textbar{}\\
जपेदभिमुखो भूत्वा पित्र्यं चैव विशेषतः \textbar{}\textbar{}\\
पित्र्य= पितृदेवत्यम् ।\\
बौधायनः ।\\
मध्वृचोऽथ पवित्राणि श्रावयेदाशयेच्छनैः ।\\
मयुचः मधुमत्य ऋचः । पवित्राणि=पुरुषसुक्तादि ।\\
निगमः ।\\
भुञ्जत्सु जपेत् पवित्रमन्त्रान् ऋग्यजुःसामेतिहासपुराणरक्षोघ्नीः\\
पावमानीरुदीरतामवरमध्वनवतीश्च ।\\
पवित्रमन्त्रान् = द्वादशाष्टाक्षरप्रभृतीन् । रक्षोघ्नी. = 'कृणुष्व पाजः
प्र.\\
सितिं न पृथ्वीम्' इत्याद्याः पञ्चदशर्च:, 'रक्षोहणं वाजिनमाजिघ\\
म इति पञ्चविंशतिः, इन्द्रासोमा तपतं रक्ष उब्जतम्' इति पञ्चवि\\
शतिः, "अग्ने हंसिन् न्यत्रिणम्'' इति नव । पात्रमान्य: = पुनन्तु मा\\
पितर' इत्याद्याः षोडशर्च: । 'तरत्समन्दीति वर्गः । 'पवस्व विश्व-\\
वर्षण' इति द्वात्रिंशत् ऋच: । 'त्वं सोमासीति द्वात्रिंशदित्येताः ।\\
'उदीरतामबर' इत्याद्याश्चतुर्दश । मधुमस्यः = 'मधुव्वाता ऋतायत' इ.\\
त्याद्याः, 'तद्वां नरासनयेदंसमित्येवं प्रकाराश्च । अजवत्यः पितुं तु\\
स्तोषम्' इत्याद्या एकादशर्चः ।\\
जपेदित्यनुतौ । शङ्खलिखितो -\\
विविधं धर्मशास्त्रमप्रतिरथं मध्ये गायत्रीमनुभाव्य इति ।\\
अप्रतिरथं = साम इति माध्यकारः । 'आशुः शिशान' इत्यादि द्वाद\\
शर्वमित्यम्ये ।\\
विष्णुः ।\\
ततस्त्वदत्सु ब्राह्मणेषु 'यन्मे प्रकामा महोरात्रैर्यद्वः क्रव्यादिति

{ }{ यजमानजप्यानि । १२७}{\\
जपेत्' इतिहासपुराणे धर्मशास्त्राणि च ।\\
सौरपुराणे ।\\
धर्मशास्त्रपुराणे च तथाथर्वशिरस्तथा ।\\
ऐन्द्रं च पौरुषं सकं भावयेद्ब्राह्मणांस्ततः \textbar{}\textbar{}\\
पाद्द्ममात्स्यप्रभासखण्डेषु ।\\
इन्द्रेश सोमसुतानि पावमानश्च शक्तितः ।\\
तत्रैन्द्राणि='सुरूपकृत्नुमृतय' इत्यादिस्तत्रयं प्रत्येकं दशर्चम्,\\
'इन्द्रमिद्वाथिन' इत्यादिसुक्तत्रयं च ।\\
ऐशानसूकानि= 'इमा रुद्राय सबसे कपर्दिने' इत्येकादशचे, 'कद्रुद्राय\\
प्रचेतस' इति नवर्चे, 'आरोपितर्मरुतां सुम्नमेतु' इति पञ्चदशर्चम्'\\
'इमा रुद्राय स्थिरधन्वने गिर' इति चतुर्ऋचम् ।\\
सौम्यानि='स्वादिष्ठया' इत्यादीनि चत्वारि सूक्तानि ।\\
याज्ञवल्क्यः,\\
मातृप्तेस्तु पवित्राणि जत्वा पूर्वजपं तथा ।\\
(अ० १ श्रद्धप्र० श्लो० २४० )\\
पूर्वजेपं सव्याहृतिकां गायत्रीमित्यादिपूर्वोक्तम् ।\\
तथा बहुचामू-\\
अग्निमीळं इति नवर्चे सूक्त 'वायवा याहि दर्शते' ते 'अश्विना य.\\
ज्वरीरिष' इति द्वादशर्चम् \textbar{} 'गायन्ति त्वा गायत्रिण' इति
'इन्द्रं वि.\\
श्वा अवीवृधन्' इत्यष्टौ ऋच: । 'अस्य वामस्य पलितस्य होतुः' इति\\
द्विपञ्चाशडचम् । सूकम्=आरण्यकं विदामघवन् विदागातुमिति\\
खण्डम् ।\\
साङ्ख्यायनीयानां तु 'न वा उ देवाः क्षुधमिद्वधं ददुः" इति नवचें\\
सुकम् । 'अग्निमीळे पुरोहितम्' इत्यादीनि एकादशस्कानि ।\\
अथ यजुर्वेदिनां जप्यानि ।\\
'अत्र पितर' इति यजु', 'नमो वः पितर' इति यजुः, 'स्मान्तम्',\\
'मधुग्वाता ऋतायत' इति तिस्रः 'पुनन्तु मा पितर इत्यनुवाकः 'त्वं\\
सोमप्रचिकित' इति चैषा पित्र्या संहिता, एतान् जपन् पितृन् प्रीणा.\\
तीति । स्मान्तं = 'वयं तेषां वसिष्ठा भूयास्म इत्येतदन्तं यजुः ।\\
बौधायनः ।\\
रक्षोमानि च सामानि स्वधावन्ति यजूंषि च ।

{१२८ }{ वीरमित्रोदयस्य श्राद्धप्रकाशे-}{\\
रक्षोघ्नानि = रक्षोहनन लिङ्गानि च देववतसंज्ञानि सामानि । स्वधाव.\\
न्ति = स्वधाशब्दयुक्तानि 'पितृभ्यः स्वधायिभ्यः स्वधानमः' इत्येवं प्र.\\
काराणि यजूंषि ।\\
मात्स्यपाद्द्मप्रभास खण्डेषु ।\\
तथैव शान्तिकाध्यायं मधुब्राह्मणमेव च ।\\
मण्डलब्राह्मणं तद्वद् प्रांतिकारि च यत् पुनः ॥\\
विप्राणामात्मनश्चापि तत्सर्वे समुदीरयेत् ।\\
भारताध्ययन कार्य पितॄणां परमं प्रियम् ॥\\
'शन्नो वातः पवताम्' इत्यादिशान्ति काध्यायः । 'इयं पृथिवी'\\
इत्यादिमधुब्राह्मणम् \textbar{} 'यदेतन्मण्डलं तपाते' इत्यादि
मण्डलब्राह्मणम्।\\
तथा । भुञ्जानानां च विप्राणां आत्मनश्चापि यत् प्रीतिकारि=प्रीतिकर\\
मितिहासाऽऽख्यानादि वीणावेणुध्वम्यादिकम् । तद्वत् = वेदजपवत् \textbar{}\\
ब्रह्मपुराणे ।\\
वीणावंशध्वनि चाथ विप्रेभ्यः संनिवेदयेत् ।\\
अन्यान्यपि पवित्राणि शिष्टाचारात् जप्यानि ।\\
तद्यथा तैत्तिरीयाणां 'दिवो वाजिष्ण उतवा पृथिव्या' इत्यादि\\
विष्णवे इत्यन्तानि यजूंषि, अग्न उदधियात इषुर्युवानामित्यादि वन्यः\\
पञ्चम इत्यन्तानि च । 'रक्षोहणो व्वलगहनो वैष्णवानखनामीत्यनुवा.\\
कः । 'सोमाय पितृमते पुरोडाश षड्कपालं निर्वपेदित्यनुवाकः॥ 'उश-\\
न्तस्त्वा हवामह उशन्तः समिधीमहीत्यनुवाकः । असौ वा आदि.\\
त्यो अस्मिन् लोक आसीदित्यनुवाकः । 'ध्रुवासि घरुणास्मृतेत्यनुवाकः\\
'इन्द्रो वृत्रं हत्वेत्यनुवाकः । 'वैश्वदवेन वा प्रजापतिः प्रजा असृजता\\
वरुण प्रयासर्वरुणपाशाद मुञ्चतेत्यनुवाकद्वयम् । 'अयं वा वयः पवत\\
इत्यनुवाकजयम् \textbar{} 'ऋचां प्राचीमहर्तादिमुच्यत इत्यनुवाकः । 'अमृ\\
तोपस्तरणमसीत्यादयः पञ्चानुवाकाः \textbar{} 'ब्रह्ममेतु माम्'
इत्यनुवाक•\\
त्र्यम् \textbar{} 'अणोरणीयान्' इत्यनुवाकः 'मेघां म इन्द्रो ददातु मेधां
देवी\\
सरस्वतीत्यादयश्चत्वारोऽनुवाकाः 'नकञ्चनचशतौ प्रत्याचक्षतेत्यादि\\
य एवं वेदेत्युपनिषदित्यन्तम् ।\\
वाजसनेयिनां तु । सप्रणवां सव्याहृतिकां गायत्रीं द्विः सकद्वा\\
पठित्वा 'अग्नये कव्यवाहनाय स्वाहा' इत्याद्याः षट्कण्डिकाः 'सुराव.\\
न्तमित्याद्या 'सप्तदश, अव्याजास्त्वित्याद्या नवेति पित्र्यमन्त्रान् 'आ.

{ }{ यजमानजप्यानि । १२९}{\\
शुः शिशान' इत्यादि सप्तदशर्चमप्रतिरथम् 'यज्जाग्रत' इत्यादि षट्क.\\
बम 'शिवसंकल्पः' 'प्रजापतिर्वै भूतानि' इत्यादिपिण्डपितृयज्ञब्राह्मणम्,\\
'पञ्चैव महायशा' इत्यादिब्रह्मयज्ञब्राह्मण 'इन्द्रस्य वै पत्त्रा'
इत्यादिसु.\\
राहोमब्राह्मणमिति ।\\
मैत्रायणीयानां तु-\\
'इषेत्वा सुभूतायवीयवदेवो वः सविता' इत्यादयः पञ्चानुवाकाः ।\\
कठानां तु 'सोमाय पितृमन्त्राज्यं पितृभ्यो बर्हिसद्द्भ्यः' इत्य\\
नुवाका: । 'उशन्तस्त्वा हवामह' इत्यनुवाका: । 'न प्राक्कृत्यात्पितृ\\
यज्ञ' इत्याद्यनुवाकाः ।\\
अथ छन्दोगानां जप्यानि ।\\
तत्र गोभिलः ।\\
अश्नत्सु जपेत् व्याइतिपूर्वी सावित्रीं तस्यां चैव गायनं पिडयां\\
च संहितां माधुच्छन्दसीं वस्वर्लोके महीयते दन्तं चास्याशयं\\
भवति ।\\
अश्नत्सु=भुञ्जानेषु \textbar{} सावित्री = गायत्री, सवितृदेवत्वात् ।
व्याहृत.\\
यश्च महर्जन इति द्वे त्यक्त्वा इतराः पञ्च जप्याः । ताश्च प्राणाया\\
मपूर्विकाः । प्राणायामपूर्वकं सत्यान्तं कृत्वा गायत्रीं सप्रणवां
सन्या.\\
इतिकां पठेदिति वरतन्तुस्मरणात् । अभूः, ॐ भुवः, ॐ स्वः ॐ तपः,\\
ॐ सत्यम् इति पञ्च सत्यान्तं कृत्वा, ॐ भूर्भुवः स्वः, इति प्रणवव्याहृति\\
पूर्वी गायत्रीं जपेदित्यर्थः । तस्यां च गायत्र्यां गायत्र= साम गायेत्
।\\
'तत् सवितुर्वरेणि' उमिति प्रस्तावः, आ इति निधनं, यहा उविश्पतिः\\
सनादग्ने अक्षन्नभीमदन्तह्यमित्रिपृष्टः मक्रां समुद्रः कनिक्रदन्तीति
द्वे\\
एषा पित्र्या संहिता \textbar{} माधुच्छन्दसं = 'इदं ह्यन्वोजसा' इति
प्रथमोत्तमे द्वे 'त्वा-\\
मिदेोते । 'स पूर्व्यो महोना' मिति । 'पुरां भिन्दुरिति । 'उप प्रक्षे
मधुमती-\\
ति' 'पवस्व सोमेति । 'सुरूपानुमिति च प्रचेताः पुरुषव्रतानि ज्येष्ठसा\\
मानि विविधानि च । पुरुषव्रतानि=पुरुषसूक्ते गीयमानानि पञ्चसामानि,\\
तत्र प्रथमस्य प्रस्तावः 'सहस्रशीर्षा पुरुषः, निधनं इड् इडा ।
द्वितीयस्य\\
प्रस्तावः 'त्रिपादूर्ध्व उदैत्पुरुषः, निधनं ऊ । तृतीयस्य प्रस्ताव:
'पुरुष\\
एवेदं सर्व निधनं ई । चतुर्थस्य प्रस्तावः एतावानस्य महिमा, निधनं\\
इट् इडा । पञ्चमप्रस्तावः 'ततो विराडजायत, निधनं ई । छन्दोगाना.\\
मेव पुष्पग्रन्थे दर्शितानि । 'इदं विष्णुः' 'प्रक्षस्य वृष्णो'
'प्रकाव्यमु\\
१७ वी० मि‍

{१३० वीरमित्रोदयस्य श्राद्धप्रकाशे-}{\\
शने वब्रुवाण' इति 'बाराहमन्त्यं, पुरुषवते चैषा वैष्णवीसंहिता, एतान्\\
प्रयुञ्जन् विष्णुं प्रीणाति । अन्यान्यपि समाचारात् सामानि गेयानि ।\\
शार्दूलशाखिनां तु-\/-\/-\\
स पूर्थ्यो महोनामिति प्रस्तावः स पूर्थ्यो महोनां, निधनं मधु\\
ब्युताः पुरां भिन्दुर्युवा कविरिति । मारुतं प्रस्तावः पुरा
भिन्दुर्युवा\\
कविः, निधनं पुरुष्टुताः हो इडा । उप प्रक्षे मधुमति क्षियन्त इति वाचः\\
साम प्रस्तावः उवा, उप प्रक्षे मधुमति क्षियन्तः उषा निधन, षानान्हो.\\
बाचेत्यादिसतखण्डानि \textbar{}\\
कौथुमशास्त्रिनां च ।\\
यद्वा 'उविश्पनिरित्यादीनि पञ्चदश सामानि, 'असौ वा आदित्यो\\
देवमध्वित्यध्याय: ।\\
राणायनीयशाखिनां ।\\
महानाम्नीसामप्रस्तावः विदामघवन् विदा, निधनं एवाहिदेवाः \textbar{}\\
मथाथर्ववेदिनां जप्यानि ।\\
'इन्द्रस्य बाहू' इत्यप्रतिरथं सुतम् । 'प्राणायाम इत्यादीनि त्रीणि\\
प्राणसूकानि । 'सहस्रबाहुः पुरुष' इति पुरुषसुक्तम् । 'कालोभ्वो वहतु\\
सप्तरश्मिः' इति कालसूक्तम् । उपनिषदमध्यात्मकम् । प्राणाग्निहोत्र-\\
महोपनिषदम् ।\\
अथ सप्तार्चिर्मन्त्रः ।\\
विष्णुधर्मोत्तरभविष्यत् पुराणयोः ।\\
पाप्मापहं पावनीय अश्वमेधसमं तथा ।\\
मन्त्रं वक्ष्याम्यहं तस्मादमृतं ब्रह्मनिर्मितम् ॥\\
देवताम्यः पितृभ्यश्च महायोगिभ्य एव च ।\\
नमः स्वधायै स्वाहायै नित्यमेव नमोनमः ॥\\
आद्यावसाने श्राद्धस्य त्रिरावर्त जपेत्सदा ।\\
अश्वमेधफलं येतद् द्विजैः सत्कृत्य पूजितम् ॥\\
पिण्डनिर्वेपणे चापि जपेदेनं समाहितः ।\\
पितरस्तृतिमायान्ति राक्षसाः प्रद्रवन्ति च ॥\\
पितॄंध त्रिषु लोकेषु मन्त्रोऽयं तारयत्युत ।\\
पठ्यमानः सदा श्राद्धे नियतैर्ब्रह्मवादिभिः ॥\\
राज्यकामो जपेदेन सदा मन्त्रमतन्द्रितः ।

{ }{ यजमानजप्यानि । १३२}{\\
वीर्य सर्वार्थ शौर्यादिश्रीरायुष्यविवर्धनम् ॥\\
प्रीयन्ते पितरोऽनेन जप्येन नियमेन च ।\\
चतुर्भिश्च चतुर्मिश्च द्वाभ्यां पञ्चभिरेव च ॥\\
हूयते च पुनर्द्वाभ्यां लमे विष्णुः प्रसीदतु ।\\
यस्य स्मृत्या च नामोक्त्या तपोयज्ञक्रियादिषु ।\\
न्यूनं सम्पूर्णतां याति सद्यो वन्दे तमच्युतम् ॥\\
आदिमध्यावसानेषु श्राद्धस्य नियमः शुचिः ।\\
जपन् विष्ण्वाह्वयं मन्त्रं विष्णुलोकं समश्नुते ॥\\
न्यूनं चैषातिरिक्तं च यत् किञ्चित् कर्मणो भवेत् ।\\
सर्वे यथावदेव स्यात् पितंश्चैव समुद्धरेत् \textbar{}\textbar{}\\
"ओं श्रावय" इति चत्वारि अक्षराणि, "अस्तु श्रौषट्" इति च\\
स्वारि ``यज" इति द्वे, ``येयजामहे " इति पञ्च, "वौषट्" इति द्वे, ए.\\
तैर्यो हूयते स विष्णुः प्रसीदत्वित्यर्थः ।\\
अथ सप्तार्चिस्तोत्रम् ।\\
प्रभासखण्डब्रह्माण्डपुराणयोः ।\\
सप्तार्चिषं प्रवक्ष्यामि सर्वकामप्रदं शुभम् ।\\
अमूर्तीनां समूर्तीनां पितॄणां दीप्ततेजसाम् \textbar{}\textbar{}\\
नमस्यामि सदा तेषां ध्यायिनां योगचक्षुषाम् ।\\
इन्द्रादीनां च नेतारो दक्षमारांचयांस्तथा ॥\\
सप्तर्षीणां पितॄणां च तान्नमस्यामि कामदान् ।\\
मन्वादीनां च नेतारः सूर्याचन्द्रमसोस्तथा ॥\\
तान्नमस्यामि सर्वान् वै पितॄन स्वर्णद्वेषु च ।\\
नक्षत्राणां ग्रहाणां च वाय्वग्निपितरस्तथा \textbar{}\textbar{}\\
द्यावापृथिव्योश्च सदा नमस्ये तान् पितामहान् ।\\
देवर्षीणां च नेतारः सर्वलोकनमस्कृताः ॥\\
त्रातारो ये च भूनानां नमस्ये तान् कृताञ्जलिः ।\\
प्रजापतेर्गयां वहेः सोमाय च यमाय च\\
योगेश्वरेभ्यश्च सदा नमस्ये तान् कृताञ्जलिः ।\\
पितृगणेभ्यः सर्वेभ्यो नमो लोकेषु सप्तसु ॥\\
स्वयम्भुवे नमस्तुभ्यं ब्रह्मणे लोकचक्षुषे ।\\
एतदुक्तं च सप्तार्चिर्ब्रह्मर्षिगणपूजितम् \textbar{}\textbar{}}

{१३२ }{ वीरमित्रोदयस्य श्राद्धप्रकाशे-}{\\
पवित्र परमं ह्येतत् श्रीमद् रक्षोविनाशनम् ।\\
एतेन विधिना युक्तस्त्रीन् वरान् लभते नरः ॥\\
अनमायुः सुतांश्चैव ददते पितरो भुवि ।\\
भक्त्या परमया युक्तः श्रद्दधानो जितेन्द्रियः ॥\\
सप्तार्चिषं जपेद्यस्तु नित्यमेव समाहितः ।\\
सप्तद्वीपसमुद्रायां पृथिव्यामेकराड् भवेत् \textbar{}\textbar{}\\
अथ पितृस्तव ।\\
मार्कण्डेयपुराणे ।\\
ब्रह्मा चाह रुचि विप्रं श्रुत्वा तस्याभिवाञ्छितम् ।\\
प्रजापतिस्त्वं भविता स्रष्टव्या भवता प्रजाः ॥\\
सृष्ट्रा प्रजाः सुतान् विप्र समुत्पाद्य क्रियास्तथा \textbar{}\\
कृताकृताधिकारस्त्वं ततः सिद्धिमवाप्स्यसि \textbar{}\textbar{}\\
स त्वं यथोक्तं पितृभिः कुरु दारोपसङ्ग्रहम् ।\\
कामं चेममनुध्याय कुरु तत् पितृपूजनम् \textbar{}\textbar{}\\
एवं च तुष्टाः पितरः प्रदास्यन्ति तवेप्सितम् ।\\
पत्नी सुतांश्च सन्तुष्टाः किं न दधुः पितामहाः ॥\\
मार्कण्डेय उवाच ।\\
इत्यृषे वचन श्रुत्वा ब्रह्मणोऽव्यक्तजन्मनः ।\\
नद्या विविक्ते पुलिने चकार पितृतर्पणम् \textbar{}\textbar{}\\
तुष्टाव च पितॄन् विप्रस्तवैरेभिरथाऽऽदृतः ।\\
एकाग्रप्रयतो भूत्वा भक्तिनम्रात्मको रुचिः \textbar{}\textbar{}\\
रुचिरुवाच ।\\
नमस्येऽहं पितॄन् भक्त्या ये वसन्त्यधिदेवताः ।\\
देवैरपि हि तर्प्यन्ते ये श्राद्धेषु स्वधोत्तरैः ॥\\
नमस्येऽहं पितॄन् स्वर्गे ये तर्प्यन्ते महर्षिभिः ।\\
श्राद्धैर्मनोमयैर्भक्त्या भुक्तिमुक्तिमभीप्सुभिः ॥\\
नमस्येऽहं पितॄन् स्वर्ग्यान् ते च सन्तर्पयन्ति तान् ।\\
श्राद्धेषु दिव्यैः सकलैरुपहारैरनुत्तमैः ॥\\
नमस्येऽहं पितॄन् भक्त्या येऽर्यन्ते गुह्यकैर्दिथि ।\\
तन्मयत्वेन बाञ्छद्भिर्वृद्धिमात्यन्तिकीं शुभाम् ॥\\


{ }{ यजमानजप्यानि । १३३}{\\
नमस्येऽहं पितॄन् मर्त्यैरर्च्यन्ते भुवि ये सदा ।\\
श्राद्धेषु श्रद्धयामीष्टलोकपुष्टिप्रदायिनः ॥\\
नमस्येऽहं पितॄन भक्त्याऽभ्यचर्यन्ते भुवि ये सदा ।\\
धन्बैः श्राद्धैस्तथाहारैस्त पोनिर्धूत किल्विषैः ॥\\
नमस्येऽहं पितॄन्विप्रैर्नैष्ठिकव्रतचारिभिः ।\\
ये संयतात्मभिर्नित्यं सन्तर्प्यन्ते समाधिभिः ॥\\
नमस्येऽहं पितॄन् श्राद्धे राजन्यास्तर्पयन्ति यान् ।\\
कल्पैरशेषैर्विधिवत् लोकद्वयफलप्रदान् \textbar{}\textbar{}\\
नमस्येऽहं पितॄन् वैश्यैरर्च्यन्ते भुवि ये सदा ।\\
स्वकर्मनिरतैर्नित्यं पुष्पधूपान्नवारिभिः ॥\\
नमस्येऽहं पितॄन् श्राद्धे ये शूद्रैरपि भक्तितः ।\\
सन्तर्प्यन्ते जगत्यत्र नाम्नाख्याताः सुकालिनः \textbar{}\textbar{}\\
नमस्येऽहं पितॄन् श्राद्धे पाताले ये महासुरैः ।\\
सन्तर्प्यन्ते सदाहारस्स्यक्तदम्भमदैः सदा ॥\\
नमस्येऽहं पितॄन् श्राद्धे ह्यर्थ्यन्ते ये रसातले ।\\
भोगैर शेषैर्विविधैर्नागैः कामानभीप्सुभिः ॥\\
नमस्येऽहं पितॄन् श्राद्धे सर्वैः सन्तर्पिताः सदा ।\\
स्तुत्वैवं विविधैर्मन्त्रैर्भोगसम्पत्समन्वितैः ॥\\
पितॄन् नमस्ये निवसन्ति साक्षात् ये देवलोकेषु महीतले वा ।\\
तथान्तरिक्षे व सुरादिपूज्यास्ते संप्रतीच्छन्तु मयोपनीतम्
\textbar{}\textbar{}\\
पितॄन् नमस्ये परमाणुभूता ये वै विमानेषु वसन्त्यमूर्ताः ।\\
यजन्ति यानस्तमला मनोभिर्योगीस्वराः क्लेशविमुक्तिहेतोः ॥\\
पितृनमस्ये दिवि ये च मूर्ताः स्वधाभुज. काम्यफलाभिसन्धौ ।\\
प्रदानशक्ताः सकलेप्सितानां विमुक्तिदा येऽनभिसंहितेषु ॥\\
तृप्यन्तु तस्मिन् पितरः समस्ता इच्छावतां ये प्रदिशन्ति कामान् ।\\
सुरस्वमिन्द्रत्व मनोधिकत्वं वस्वात्मजान् क्ष्मामबलागृहाणि ॥\\
सूर्यस्य ये रश्मिषु चन्द्रविम्बे शुक्ले विमाने च सदा वसन्ति ।\\
तृप्यन्तु तेऽस्मिन् पितरोऽन्नतोयैर्गन्धादिना तुष्टिमतो व्रजन्तु ॥\\
येषां हुतेऽग्नौ हविषापि तृप्तिर्ये भुञ्जते विप्रशरीरसंस्थाः ।\\
वे पिण्डदानेन मुदं प्रयान्ति तृप्यन्तु तेऽस्मिन् पितरोऽन्नतोयैः ॥\\
ये खङ्गमांसेन सुरैरभीष्टैः कृष्णैस्तिलैर्दैत्यमहोरगैश्च ।

{१३४ }{ वीरमित्रोदयस्य श्राद्धप्रकाशे-}{\\
कालेन शाकेन महर्षिवर्यैः सम्प्रीणितास्ते मुदमत्र यान्तु
\textbar{}\textbar{}\\
कव्यान्यशेषाणि च यान्यभीष्टान्यतीव तेषाममरार्चितानाम् ।\\
तेषां तु सान्निध्यमिहास्तु पुष्पगन्धाम्बुभोज्येषु मयाऽऽहृतेषु ॥\\
दिने दिने ये प्रतिगृह्णतेऽचं मासानुभोज्या भुवि येऽष्टकासु ।\\
ये वत्सरान्तेऽभ्युदयेषु पूज्याः प्रयान्तु ते मे पितरोऽद्य तृप्तिम् ॥\\
पूज्या द्विजानां कुमुदेन्दुभासो ये क्षत्रियाणां च नवार्कवर्णाः ।\\
तथा विशां ये ह्यनलावभासा नीलांनिभाः शूद्रजनस्य ये च ॥\\
तस्मिन् समस्ता मम पुष्पगन्धधूपानतोयादिनिवेदनेन ।\\
तथाझिहोमेन च यान्तु तृप्ति सदा पितृभ्यः प्रणतोस्मि तेभ्यः
\textbar{}\textbar{}\\
ये देवपूर्वाण्यपि तृप्तिहेतोरश्नन्ति कव्यानि शुभाहृतानि ।\\
तृप्तास्तु ये भूतिसृजो भवन्ति तृप्यन्तु तेऽस्मिन् प्रणतोऽस्मि तेभ्यः ॥\\
रक्षांसि भूतान्यसुरांस्तथोग्रान् निःसारयन्तस्त्वशिवं प्रजानाम् ।\\
आद्याः सुराणाममरैश्च पूज्याः तुष्यन्तु तेन प्रणतोऽस्मि तेभ्यः
\textbar{}\textbar{}\\
अभिष्वाचा बर्हिषद आज्यपाः सोमपास्तथा ।\\
व्रजन्तु तृतिं श्राद्धेऽस्मिन् पितरस्तर्पिता मया ॥\\
अग्निष्वात्ताः पितृगणाः प्राचीं रक्षन्तु मे दिशम् ।\\
तथा वर्हिषदः पान्तु याम्यां ये पितरस्तथा ॥\\
प्रतीचीमाज्यपास्तद्दुदीचीमपि सोमपाः ।\\
रक्षोभूतपिशाचेभ्यः तथैवासुरदोषतः ॥\\
सर्वतश्चाधिपस्तेषां यमो रक्षा करोतु मे ।\\
विश्वो विश्वभुगाराध्यो धन्यो धर्मः सनातनः ॥\\
भूतिदो भूतिकृत् भूतिः पितॄणां ये गणा नव \textbar{}\\
कल्याणः कल्यतां कर्ता कल्यः कल्यतराश्रयः ॥\\
कल्यताहेतुरनघः षडित्येते गणाः स्मृताः ।\\
वसे वरेण्यो वरदः स्तुष्टिदः पुष्टिदस्तथा ।\\
विश्वपाता तथा धाता सप्त चैते तथा गणाः ।\\
महामहा महावेजा मतिमांश्च फलप्रदः ।\\
गणाः पञ्च तथैवैत पितॄणां पापनाशनाः \textbar{}\textbar{}\\
सुखदो धनदग्धान्यो धर्मदोऽन्यञ्च भूतिदः ।\\
पितॄणां कथ्यते चैव तथा गणचतुष्टयम् ॥

{ }{ यजमानजप्यानि । १३५}{\\
एकत्रिंशत् पितृगणा यैर्व्याप्तमखिलं जगत् ।\\
ते मेऽत्र तुष्टाः तुष्यन्तु यच्छन्तु व सदा हितम् \textbar{}\textbar{}\\
मार्कण्डेय उवाच ।\\
एवं च स्तुवतस्तस्य तेजसां राशिरुत्थितः ।\\
प्रादुर्बभूव सहसा गगनव्याप्तिकारकः ॥\\
तं दृष्ट्रा सुमहन्तेजः समाच्छाद्य स्थितं जगत् ।\\
जानुभ्यामवनीं गत्वा रुचिस्तोत्रमिदं जगौ ॥\\
मूर्तिभाजाममूर्तीनां पितॄणाममितौजसाम् ।\\
नमस्यामि सदा तेषां ध्यायिनां दिव्यचक्षुषाम् ॥\\
इन्द्रादीनां च नेतारो दक्षमारीचयोस्तथा ।\\
सप्तर्षीणां तथान्येषां तान्नमस्यामि कामदान् ॥\\
मन्वादीनां च नेतृश्व सूर्याचन्द्रमसोस्तथा ।\\
तान्नमस्याम्यहं सर्वान् पितरश्चार्णवेषु च \textbar{}\textbar{}\\
नक्षत्राणां ग्रहाणां च वाय्वग्निनभसां तथा ।\\
द्यावापृथिव्योश्च तथा नमस्यामि कृताञ्जलिः ॥\\
देवर्षीणां च नेतृश्व सर्वदेवनमस्कृतान् ।\\
अभयस्य सदा दातॄन् नमस्येऽह कृताञ्जलिः ॥\\
प्रजापतेः कश्यपाय सोमाय वरुणाय च ।\\
योगेश्वरेभ्यश्च तथा नमस्यामि कृताञ्जलिः ॥\\
नमो गणेभ्यः सप्तभ्यस्तथा लोकेषु सप्तसु ।\\
स्वयम्भुवे नमस्यामि ब्रह्मणे योगचक्षुषे ॥\\
यदाधाराः पितृगणा योगमूर्तिधरा हि ते ।\\
नमस्यामि ततः सोमं पितरं जगतामहम् ॥\\
अग्निरूपांस्तथैवान्यान् नमस्यापि पितृनहम् ।\\
अग्नीषोममयं विश्वं यत एतदशेषतः ॥\\
ये तु तेजोमयाश्चैते सोमसूर्याग्निमूर्तयः ।\\
जगत्स्वरूपिणश्चैव तथा ब्रह्मस्वरूपिणः ॥\\
तेभ्योऽखिलेभ्यो योगिभ्यः पितृभ्यो यतमानसः ।\\
नमोनमो नमस्ते मे प्रसीदन्तु स्वधाभुजः ॥\\
मार्कण्डेय उवाच ।\\
एव स्तुता ततस्तेन तेजसो मुनिसत्तम ।\\
निश्चक्रमुस्ते पितरो भासयन्तो दिशो दश ॥\\


{१३६ }{वीरमित्रोदयस्य श्राद्धप्रकाशे-\\
}{~\\
निवेदितं च यत्तेन गन्धमाल्यानुलेपनम्\\
तद्भूषितानथ स तान् दडशे पुरतः स्थितान् ॥\\
प्रणिपत्य पुनर्भक्त्या पुनरेव कृताञ्जलिः \textbar{}\\
नमस्तुभ्यं नमस्तुभ्यमत्याह पृथगाहृतः ॥\\
ततः प्रसन्नाः पितरस्तमूचुर्मुनिसत्तमम् ।\\
वरं वृणीष्वेति स तानुवाचानतकन्धरः \textbar{}\textbar{}\\
साम्प्रतं सर्गकर्तृत्वमादिष्टं ब्रह्मणा मम ।\\
सोऽहं पत्नीमभीप्लामि धन्यां दिव्यां प्रजावतीम् ॥\\
पितर ऊचुः ।\\
अद्यैव सद्यः पक्षी से भविष्यति मनोरमा ।\\
तस्यां च पुत्रो भविता रुचिरो मुनिसत्तमः \textbar{}\textbar{}\\
मन्वन्तराधिपो धीमांस्त्वत्रान्नैवोपलक्षितः ।\\
रुचेरौव्य इति ख्यानि प्रयास्पति जगत्त्रये ॥\\
तस्यापि बहवः पुत्रा भविष्यन्ति महात्मनः ।\\
महाबला महावीर्याः पृथिवीपरिपालकाः ॥\\
त्व च प्रजापतिर्भूत्वा प्रजाः सृष्टा चतुर्विधाः ।\\
क्षीणाधिकारी धर्मज्ञः ततः सिद्धिमवाप्स्यसि ॥\\
स्तोत्रेणानेन हि रुचे योऽस्मांस्तोध्यति भक्तितः ।\\
तस्य तुष्टा षय भोगान् दास्यामो शानमुत्तमम् ॥\\
शरीरारोग्यमैश्वर्ये पुत्रपौत्रादिकं तथा ।\\
वाञ्छद्भिः सतत स्तव्याः स्तोत्रेणानेन यत्नतः ॥\\
श्राद्धे च य इमं भक्त्या अस्मत्प्रीतिकरस्तवम् ।\\
पठिष्यति द्विजाग्याणां भुञ्जतां पुरतः स्थितः \textbar{}\textbar{}\\
स्तोत्र श्रवणसम्प्रीत्या सन्निधाने परे कृते ।\\
अस्माकमक्षयं श्राद्धं तद्भविष्यत्यसंशयः ॥\\
यद्यप्यश्रोत्रियं श्राद्धं यद्यप्युपहतं भवेत् ।\\
अन्यायोपात्तवित्तेन यदि वा कृतमन्यथा ॥\\
अश्राद्धा रूप हुने रुपहारैस्तथा कृतम् ।\\
अकालेऽप्यथवाऽदेशे विधिहीनमथापि वा ॥\\
अश्रद्दधानैः पुरुषैर्दम्ममाश्रित्य यत् कृतम् ।\\
अस्माकं जायते तृप्तिस्तथाप्येतदुदीरणात् ।

{ }{ यजमानजप्यानि । १३७}{\\
यत्रैतत् पठ्यते श्राद्धे स्तोत्रमस्मत्सुखावहम् \textbar{}\\
अस्माकं जायते तृप्तिस्तत्र द्वादशवार्षिकी \textbar{}\textbar{}\\
हेमन्ते द्वादशाब्दानि तृप्तिमेतत् प्रयच्छति ।\\
शिशिरे द्विगुणान्दानि तृप्तिं स्तोत्रमिदं श्रुतम् ॥\\
वसन्ते षोडशसमास्तृप्तये श्राद्धकर्मणि ।\\
ग्रीष्मे च षोडशैवैतत् पठितं तृप्तिकारकम् \textbar{}\textbar{}\\
विकलेऽपि कृते श्राद्धे स्तोत्रेणानेन (१) शोधिते ।\\
वर्षासु तृप्तिरस्माकमक्षया जायते (२) स्तुते ॥\\
शरत्कालेऽपि पठितं श्राद्धकाले प्रयच्छति ।\\
अस्माकमेतत्पुरुषैस्तृप्तिः पञ्चदशाब्दिकी ॥\\
यस्मिन् गृहेऽपि लिखितं एतत्तिष्ठति नित्यदा ।\\
सन्निधानं कृते श्राद्धे तत्रास्माकं भविष्यति ॥\\
तस्मादेतस्वया श्राद्धे विप्राणां भुञ्जता पुरः ।\\
श्रावणीयं महाभाग ! अस्माकं प्रीतिहेतुकम् ॥ इति ।\\
मत्र ब्रह्मवैवर्तादिपुराणोक्तानि गयाप्रशंसावचनान्यपि शिष्टाचा.\\
रात् पठनीयानि ।\\
गयायां धर्मपृष्ठे च सरसि ब्रह्मणस्तथा ।\\
गयाशीर्षे वटे चैव पितॄणां दत्तमक्षयम् \textbar{}\textbar{}\\
गयायां पितृरूपेण स्वयमेव जनार्दनः ।\\
तं दृष्टा पुण्डरीकाक्षं मुच्यते च ऋणत्रयात् \textbar{}\textbar{}\\
शमीपत्रप्रमाणेन पिण्डं दद्यात् गयाशिरे ।\\
उद्धरेत् सप्तगोत्राणि कुलमेकोत्तरं शतम् ॥\\
भूमिष्ठास्तु दिवं यान्ति स्वर्गस्था मोक्षगामिनः ।\\
स्वर्गपातालमर्त्येषु नास्ति तीर्थ गयासमम् \textbar{}\textbar{}\\
पितरो यान्ति देवत्वं दत्ते पिण्डे गयाशिरे । इति ॥\\
मत्स्यपुराणे ।\\
सप्तम्याधा दशार्णेषु मृगाः कालञ्जरे गिरौ ।\\
चक्रवाकाः सरोद्वीपे हंसा: सरसि मानसे ॥

% \begin{center}\rule{0.5\linewidth}{0.5pt}\end{center}

{( १ ) साधिते. इति मार्कण्डेयपुराणे पाठः ।\\
( १ ) रुचे \textbar{} इति मा० पु० पाठः ।\\
१८ वी० मि०\\


{१३८ }{वीरमित्रोदयस्य श्राद्धप्रकाशे-}{\\
तेऽपि जाताः कुरुक्षेत्रे ब्राह्मणा वेदपारगाः ।\\
प्रस्थिता दीर्घमध्वानं यूय तेभ्योऽवसीदत ॥ इति ।\\
समग्र सप्तव्याधाख्यान पठनासमर्थस्तु तत्संग्रहरूपमिदं श्लोक\\
द्वयं पठेदिति हेमाद्रौ ।\\
नारदीयपुराणे ।\\
आख्यानानि पितॄणां च श्राद्धग्वक्षय्यतृप्तये ।\\
गाथाश्च पितृभिर्गीता भुञ्जानान् श्रावयेद् द्विजान् ॥\\
पितृगाथाश्च काश्चित् प्रदश्यन्ते ।\\
आह विष्णुः ।\\
अथ पितृगीते गाधे भवतः ।\\
अपि जायेत सोऽस्माकं कुले कश्चिन्नरोत्तमः ।\\
प्रावृट्कालेऽसिते पक्षे त्रयोदश्यां समाहितः ॥\\
मधूत्कटे तु यः श्राद्धं पायसेन समाचरेत् ।\\
कार्तिकं सकलं वापि प्राक्छाये कुञ्जरस्य वा ॥ इति ।\\
याज्ञवल्क्यः ।\\
कुलेऽस्माकं स तन्तुः स्यात् यो नो दद्याज्जलाञ्जलिम् ।\\
नदीषु बहुतोयासु शीतलासु विशेषतः ॥\\
अपि जायेत सोऽस्माकं कुले कश्चिन्नरोत्तमः ।\\
गयाशीर्षे घंटे आद्धं यो नः कुर्यात् समाहितः ॥ इति ।\\
तथा बृहस्पतिः ।\\
काङ्क्षन्ति पितरः पुत्रान्नरकापातभीरवः ।\\
गयां यास्यति यः कश्चिन् सोऽस्मान् सन्तारयिष्यति \textbar{}\textbar{}\\
करिष्यति वृषोत्सर्गमिष्टापूर्ते तथैव च ।\\
पालविष्यति गार्हस्थ्यं श्राद्धं दास्यति चान्वहम् \textbar{}\textbar{}\\
वायुपुराणब्रह्माण्डपुराणयो ।\\
अत्र गाथा: पितृगीताः कीर्तयन्ति पुराविदः ।\\
तास्तेऽहं सम्प्रवस्यामि यथावच निबोधत ॥\\
अपि नः स कुले जायेत् यो नो दद्यात् त्रयोदशीम् ।\\
पायसं मधुसर्पिर्थ्यो छायायां कुञ्जरस्य च ।\\
अजेन सर्वलोहन वर्षासु च मघासु च ।\\
एष्टव्या बहवः पुत्रा यद्येकोऽपि गयां व्रजेत् ॥\\
गौरीं वाप्युद्धहेत् भार्यौ नलिं वा वृषमुत्सृजेत् । इति ॥

{ श्राद्धदेशाने रूपणम् १३९\\
उक्तसर्वप्रकारजपासम्भवे प्रकारान्तरमुक्तं मत्स्यपुराणे-\\
अभावे सर्वविद्यानां गायत्रीजपमारभेत् ॥ इति । इति जप्यानि ।\\
अथ श्राद्धदेशाः ।

{तत्र विष्णुधर्मोतरे ।\\
दक्षिणाप्रवणे देशे तीर्थादौ वा गृहेऽथवा ।\\
भुसंस्कारादिसंयुके श्राद्धं कुर्यात् प्रयत्नतः ॥\\
तीर्थ - ऋषिदेवतासेव्यं जलम् \textbar{} आदिशब्देन ऋषिसेविताश्रमप\\
रिग्रहः । भूसंस्कारो=गोमयोपलेपादिः \textbar{} आदिपदात् केशाद्यपसारण\\
परिग्रहः ।\\
तदुकं तत्रैव ।\\
गोमयेनोपलिप्तेषु विविक्तेषु गृद्देषु च ।\\
कुर्याच्छ्राद्धमथैतेषु नित्यमेव यथाविधि ॥\\
विविक्तेषु = पवित्रेषु ।\\
याज्ञवल्क्यः ।\\
(१) परिधिते शुचौ देशे दक्षिणाप्रवणे तथा ।\\
( अ० १ श्र० श्लो० २२७)\\
परिश्रिते = परितश्छादिते । शुचौ गोमयादिनोपलिप्ते । दक्षिणाप्रवणे =द.\\
क्षिणोपनते देशे । स्वतो दक्षिणाप्रणत्वासम्भवे तु देशस्य यत्नतो\\
दक्षिणाप्रवणत्वं कार्यम् ।\\
तथा च मनुः ॥\\
शुद्धिं देशं विविकं तु गोमयेनोपलिप्य च ।\\
दक्षिणाप्रवणं चैव प्रयत्नेनोपपादयेत् \textbar{}\textbar{} (अ० ३ श्लो०
२०६)

{यमः ।\\
रुक्षं कृमिहतं क्लिश्वं सङ्कीर्णानिष्टगन्धिकम् ।\\
देशं त्वनिष्टशब्दं च वर्जयेच्छ्राद्धकर्माणि \textbar{}\textbar{}\\
रूक्षम् = अस्निग्धम् । क्लिनम् = लकर्दमम् । सङ्कीर्णम् =
पदार्थान्तरैरा\\
कीर्णम् । अनिष्टगन्धिकं पूतिगन्धिकम् \textbar{} अनिष्टशब्दम् =
अश्राव्यशब्दम् ।\\
मार्कण्डेयः ।\\
बर्ज्या जन्तुमयी रूक्षा क्षितिः प्लुष्टा तथाग्निना ।\\
अनिष्टदुष्टशब्दोना दुर्गन्धा श्राद्धकर्माणि ॥

% \begin{center}\rule{0.5\linewidth}{0.5pt}\end{center}

( १ ) परिस्तृते इति मुद्रितयाज्ञवल्क्ये पाठः ।

{१४० }{वीरमित्रोदयस्य श्राद्धप्रकाशे-}{\\
अग्निनाप्लुष्ठा=दग्धा । उप्रा=भयजनिका \textbar{}\\
शङ्खः ।\\
गोगजाश्वादिपृष्ठेषु कृत्रिमायां तथा भुवि ।\\
म कुर्याच्छ्राद्धमेतेषु परक्यासु च भूमिषु ॥\\
कृत्रिमायां =वेदिकादौ । परक्यासु = परपरिगृह्णतासु । ताश्च गृहगो\\
ठारामादयो न पुनस्तीर्थादयः ।\\
तथा चादिपुराणम् ।\\
अटवी पर्वताः पुण्या नदीतीराणि यानि च ।\\
सर्वाण्यस्वामिकान्याहुर्न हि तेषु परिग्रहः \textbar{}\textbar{}\\
वनानि गिरयो नद्यस्तीर्थान्यायतनानि च ।\\
देवखाताच गर्ताश्च न स्वामी तेषु विद्यते ॥\\
तीर्थक्षेत्रविशेषेषु कृतं श्राद्धमतिशयफलप्रदं भवतीत्याह -\\
देवल ।\\
श्राद्धस्य पूजितो देशो गया गङ्गा सरस्वती ।\\
कुरुक्षेत्रं प्रयागश्च नैमिषं पुष्कराणि च ॥\\
नदीतटेषु तीर्थेषु शैलेषु पुलिनेषु च ।\\
विविक्तेष्वेव तुष्यन्ति दत्तेनेह पितामहाः ॥

{व्यासः ।\\
पुष्करेष्वक्षयं श्राद्धं जपहोमतपांसि च ।\\
महोदधौ प्रयागे च काश्यां च कुरुजाङ्गले ॥\\
शङ्खः ।\\
गङ्गायमुनयोस्तीरे पयोष्ण्यमरकण्टके \textbar{}\\
नर्मदा बाहुदातीरे भृगुतुङ्गे हिमालये \textbar{}\textbar{}\\
गङ्गाद्वारे प्रयागे च नैमिषे पुष्करे तथा ।\\
सन्निहत्यां गयायां च दत्तमक्षय्यतां व्रजेद ॥

{ब्रह्माण्डपुराणे ।\\
नदीसमुद्रतीरे वा हृदे गोष्ठेऽथ पर्वते ।\\
समुद्रगानदीतीरे सिन्धुसागरसङ्गमे ॥\\
मधोर्वा सङ्गमे शस्ते शालग्रामशिलान्तिके ।\\
पुष्करे वा कुरुक्षेत्रे प्रयागे नैमिषे तथा ॥\\
शालग्रामे च गोकर्णे गयायां च विशेषतः ।\\
क्षेत्रेष्वेतेषु यः श्राद्धं पितृभक्तिसमन्वितः ॥

{ }{ श्राद्धदेशनिरूपणम् १४१}{\\
करोति विधिवन् मर्त्यः कृतकृत्यो विधीयते ।

{बृहपत्तिः ।\\
काङ्क्षन्ति पितरः पुत्रान्नरकापात भीरवः \textbar{}\\
गयां यास्यति यः कश्चित् सोऽस्मान् सन्तारयिष्यति ॥\\
करिष्यति वृषोत्सर्गमिष्टापूर्ते तथैव च ।\\
पालयिष्यति वृद्धत्वे श्राद्धं दास्यति चान्वहम् \textbar{}\textbar{}\\
गयायां धर्मपृष्ठे च सदसि ब्रह्मणस्तथा ।\\
गयाशीर्षे वटे चैव पितॄणां दत्तमक्षयम् ॥\\
धर्मारण्यं धर्मपृष्ठं धेनुकारण्यमेव च ।\\
दृष्ट्वैतानि पितॄंस्तर्प्य वंशान् विंशतिमुद्धरेत् \textbar{}\textbar{}\\
विष्णुरपि ।\\
अथ पुष्करेष्वक्षयं श्राद्धं जपहोमनपांसि च ।\\
पुष्करे स्नातमात्रस्तु सर्वपापेभ्यः पूतो भवति । एवमेव गयाशीर्षे\\
घंटे, अमरकण्टके पर्वते, यत्र क्वचन नर्मदातीरे, यमुनातीरे, गङ्गायां,\\
विशेषतो गङ्गाद्वारे, प्रयागे च गङ्गासागरसङ्गमे, कुशावर्ते, बिल्वके,\\
मीलपर्वते, कनखले, कुब्जाम्रे, भृगुतुङ्गे केदारे, महालये लालतिका\\
यां, सुगन्धायां, शाकम्भर्यो, फल्गुतीर्थे महागङ्गायां, तन्दुलिकाश्रमे,\\
कुमाराधारायां, प्रभासे, यत्र क्वचन सरस्वत्यां, विशेषतो नैमिषारण्ये,\\
चाराणस्याम्, अगस्त्याश्रमे; कण्वाश्रमे, कौशिक्यां, सरयूतीरे, शो-\\
णस्य ज्योतीरथ्याश्च सङ्गमे, श्रीपर्वते, कालोदके, उत्तरमानसे, मत\\
अवाप्यां, सप्तचौं, विष्णुपदे, स्वर्गप्रदेशे, गोदावर्या, गोमत्यां
क्षेत्रव\\
त्यां, विपाशायां, वितस्तायां, शतद्रतीरे, चन्द्रभागायाम्, पैरावत्यां,\\
सिन्धोस्तीरे, दक्षिणे पञ्चनदे, औजसे एवमादिष्वन्येषु तीर्थेषु लरि•\\
द्धारासु सङ्गमेषु, प्रभवेषु, पुलिनेषु, निकुञ्जेषु, प्रस्रवणेषु
वनेषूपवनेषु,\\
गोमयेनोपलिप्तेषु गृहेषु, मनोशेषु च ।\\
अत्रापि पितृगता गाथा भवन्ति ।\\
कुलेऽस्माकं स जन्तुः स्याद्यो नो दद्याज्जलाञ्जलिम् ॥\\
नदीषु बहुतोयासु शीतलासु विशेषतः ।\\
अपि जायेत सोऽस्माकं कुले कश्चिनरोत्तमः ॥\\
गयाशीर्षे घटे श्राद्धं यो न कुर्यात् समाहितः ।\\
एष्टव्या बहवः पुत्रा यद्येकोऽपि गयां व्रजेत् ।\\
यजेत वाश्वमेधेन नीलं वा वृषमुत्सृजेत् ॥\\
•

{१४२ }{ वीरमित्रोदयस्य श्राद्धप्रकाशे-}{\\
प्रभवेषु नदीनामुत्पत्ति प्रदेशेषु, सरिद्धारासु सङ्गमेषु चेति
प्रस्तुश्य\\
प्रभवेष्वित्यभिधानात् । पुलिनं = नदीतोयोत्थितः प्रदेशः । निकुजो = ल.\\
तादिपरिवेष्टितः प्रदेशः । प्रसवणं-निर्झरः । उपवनं गृहवाटिका \textbar{}
मनोश=\\
रमणीयम् ।\\
वायुपुराणे ।\\
गयायां धर्मपृष्ठे च सरासे ब्रह्मणस्तथा ।\\
( १ ) गयागृध्रे वटे चैव श्राद्धं दत्तं महाफलम् ॥\\
भरतस्याश्रमे पुण्ये नित्यं पुण्यत मैर्वृते ।\\
मतङ्गस्य पदं तत्र दृस्यते सर्वमानुषैः \textbar{}\textbar{}\\
ख्यापितं धर्मसर्वस्वं लोकस्यास्य निदर्शनम् ।\\
( २ ) यच्चम्पकवनं पुण्यं पुण्यकृद्भिर्निषेवितम् ॥\\
यस्मिन् पाण्डुविशल्येति तीर्थ सद्योनिदर्शनम् ।\\
तृतीयायां तथा पादे (३) निश्वीरायाश्च मण्डले ।\\
महाइदे च कौशिक्यां दत्तं श्राद्धं महाफलम् ।\\
मुण्डपृष्ठे पदं न्यस्तं महादेवेन धीमता ॥\\
बहून् वर्षगणांस्तप्त्वा तपस्तीव्रं सुदुस्करम् ।\\
अल्पेनाप्यत्र कालेन नरो धर्मपरायणः ॥\\
पाप्मानमुत्सृजत्याशु जीर्णो त्वचमिवोरगः।\\
नाम्ना कनकनन्देति तीर्थे जगति विश्रुतम् ॥\\
उदीच्यां मुण्डपृष्ठस्य ( ४ ) ब्रह्मर्षिगणसेवितम् ।\\
तत्र स्नात्वा दिवं यान्ति स्वशरीरण मानवाः ॥\\
दत्तं चापि तथाश्राद्धमक्षयं समुदाहृतम् ।\\
स्नात्वा ऋणत्रयं तत्र निष्क्रीणाति नरोत्तमः ॥\\
मानसे सरसि स्नात्वा श्राद्धं निर्वर्तयेत्ततः ।\\
उत्तरं मानसं गत्वा सिद्धिं प्राप्नोत्यनुत्तमाम् ॥\\
तस्मिन्निवर्तयेच्छ्राद्धं यथाशक्ति यथाबलम् \textbar{}\\
कामान् स लभते दिव्यान् मोक्षोपायांश्च कृत्स्नशः ॥

% \begin{center}\rule{0.5\linewidth}{0.5pt}\end{center}

{(१) गयायां गृद्धकूटे चेति मुद्रितवायुपुराणे पाठः ।\\
(२) एवं पश्ञ्चवनमिति वा० पु० पाठः ।\\
(३) निःस्खरे पावमण्डले इति वा० पु० पाठः ।\\
(४) देवपीति वा० पु० पाठः ।

{ श्राद्धदेशनिरूपणम् १४३\\


{मत्स्यपुराणे ।\\
नन्दाथ ललिता तद्वत् तीर्थ मायापुरी तथा ।\\
तथा चित्रपदं नाम ततः केदारमुत्तमम् \textbar{}\textbar{}\\
गङ्गासागरमित्याहुः सर्वतीर्थमयं शुभम् ।\\
तीर्थं ब्रह्मसरस्तद्वच्छतद्सलिलोद्गमे ॥\\
तथा ।\\
कृतशौच महापुण्यं सर्वपापनिषूदनम् ।\\
यत्रास्ते नरसिंहस्तु स्वयमेव जनार्दनः \textbar{}\textbar{}\\
तीर्थमिक्षुमती नाम पितॄणां वल्लभं सदा ।\\
सङ्गमे यत्र तिष्ठन्ति गङ्गायाः पितरः सदा ॥

{तथा ।\\
यमुना देविका काली चन्द्रभागा दृषद्वती ।\\
नदी धेनुमती पुण्या पारा वेत्रवती तथा ॥\\
नीलकण्ठमितिख्यातं पितॄन् तीर्थ द्विजोत्तमाः ।\\
तथा भद्रसरः पुण्यं सरो मानसमेव च ॥\\
मन्दाकिनी तथाऽच्छोदा विपाशा च सरस्वती ।\\
तीर्थं मित्रपदं तद्वद् वैद्यनाथं महाफलम् \textbar{}\textbar{}\\
शिप्रानदी महाकालं तथा कालखरं शुभम् ।\\
वंशोद्भेदं हरोन्द्भेदं गर्भोद्भवं महाफलम् ॥\\
भद्रेश्वरं विश्नुपदं नर्मदाद्वारमेव च ।\\
गयापिण्डप्रदानेन सामान्याहुर्महर्षयः ॥

{तथा ।\\
ॐ कार पितृतीर्थ च कावेरी कपिलोदकम् ।\\
संभेदं चण्डवे गायास्तथैवामरकण्टकम् \textbar{}\textbar{}\\
कुरुक्षेत्राच्छतगुणमस्मिन् स्नानादिकं भवेत् ।\\
शुक्लतीर्थं च विख्यातं तीर्थे सोमेश्वरं परम् ॥\\
सर्वव्याधिहरं पुण्यं फलं कोटिशताधिकम् ।

{तथा ।\\
कायावरोहणं नाम तथा चर्मण्वती नदी ।\\
गोमती वरणा तद्वत्तीर्थमोशनसं परम् \textbar{}\textbar{}\\
भैरव भृगुतीर्थे च गौरीतीर्थमनुत्तमम् ।\\
तीर्थं वैनायकं नाम वस्त्वेश्वरमतः परम् \textbar{}\textbar{}\\


{१४४ }{वीरमित्रोदयस्य श्राद्धप्रकाशे-}{\\
परं तथा पापहरं पुण्यं प्रतपती नदी ।\\
मूलतापी पयोष्णी च पयोष्णीसङ्गमस्तथा ॥\\
महाबोधिः पाटला व नगतीर्थमवन्तिका ।\\
तथा वेणानदी पुण्या महाशालं तथैव च ॥\\
महारुद्रो महालिङ्ग दशार्णा च नदी शुभा ।\\
शतरुद्रा शताद्वा च तथा विष्णुपदं परम् ॥\\
अङ्गारवाहिका तद्वन्नदौ तौ शोणघर्घरौ ।\\
कालिका च नदी पुण्या पितराधनदी शुभा ॥\\
पतानि पितृतीर्थानि शस्यन्ते स्नानदानयोः ।\\
श्राद्धमेतेषु यद्दन्तं तदनन्तफलं स्मृतम् ॥\\
द्रोणी वादनदी वारा सरःक्षीरनदी तथा ।\\
द्वारकाकृष्णतीर्थं च तथार्बुदसरस्वती \textbar{}\textbar{}\\
नदीं मणिमती नाम तथा च गिरिकर्णिका ।\\
धूतपापा तथा तीर्थ समुद्रो दक्षिणस्तथा ॥\\
गोकर्णौ गजकर्णश्च तथा च पुरुषोत्तमः ।\\
एतेषु पितृतीर्थेषु श्राद्धमानन्त्यमुच्यते ॥\\
तीर्थ मधुकरं नाम स्वयमेव जनार्दनः ।\\
यत्र शार्ङ्गिधरो विष्णुर्मेखलायामवस्थितः \textbar{}\textbar{}\\
तथा मन्दोदरीतीर्थ यत्र चम्पा नदी शुभा ।\\
तथा सामलनामानं महाशालवती तथा ॥\\
पयोष्ण्या दक्षिणे तीर्थे देवदेवेश्वरं तथा ।\\
कोटेश्वरं तथा देव रेणुकायाः समीपतः ॥\\
वक्रकोट तथा पुण्यं तीर्थ नाम जलेश्वरम् ।\\
अर्जुन त्रिपुरेशं च सिद्धेश्वरमतः परम् ॥\\
श्रीशैलं शाङ्कर तीर्थ नारसिंहमतः परम् ।\\
महेन्द्रं च तथा पुण्यं तथा श्रीरङ्गसंज्ञितम् \textbar{}\textbar{}\\
एतेष्वपि सदा श्राद्धमनन्तफलमश्नुते ।\\
दर्शनादपि पुण्यानि सर्वपापहराणि वै ॥\\
तुङ्गभद्रा नदी पुण्या तथा भीमरथी सरित् ।\\
भीमेश्वरं कृष्णवेणा कावेरी वञ्जुला नदी \textbar{}\textbar{}\\
नदी गोवरी नाम त्रिसन्ध्यार्थिमुत्तमम् ।

{ }{ श्राद्धदेशनिरूपणम् १४५}{\\
तीर्थे त्रैयम्बकं नाम सर्वतीर्यैर्नमस्कृतम् ॥\\
यत्रास्ते भगवानीशः स्वयमेव त्रिलोचनः ।\\
श्राद्धमेतेषु सर्वेषु दत्तं कोटिगुणं भवेत् ॥\\
स्मरणादपि पापानि व्रजन्ति शतधा द्विजाः ।\\
श्रीपर्णी च नदी पुण्या व्यासतीर्थमनुत्तमम् ॥\\
तथा मत्स्यनदीकारा शिवधारा तथैव च ।\\
भवतीर्थ तु विख्यातं पस्पातीर्थं च शाश्वतम् ॥\\
पुण्यं रामेश्वरं तद्वदेलापुरमलम्पुरम् ।\\
अङ्गारभूतं विख्यातमामर्दकमलम्बुसम् ॥\\
भाम्रातकेश्वरं चैव तद्वदेकाम्रकं परम् ।\\
गोवर्द्धनं हरिश्चन्द्रं पुरश्चन्द्र पृथूदकम् \textbar{}\textbar{}\\
सहस्राक्षं हिरण्याक्षं तथा च कदली नदी ।\\
रामाधिवासोऽपि तथा तथा सौमित्रिसङ्गतम् ॥\\
इन्द्रकील तथा नादं तथा च प्रियमेलकम् ।\\
एतान्यपि सदा श्राद्धे प्रशस्तान्यधिकानि च ।\\
एतेषु सर्वदेवानां सान्निध्यं पठ्यते यतः \textbar{}\textbar{}\\
दानमेतेषु सर्वेषु भवेत् कोटिशताधिकम् ।\\
बाहुदा च नदी पुण्या तथा सिद्धिवनं शुभम् ॥\\
तीर्थ पाशुपतं नाम तटी पर्वतका तथा ।\\
श्राद्धमेतेषु सर्वेषु दत्तं कोटिशतोत्तरम् \textbar{}\textbar{}\\
तथैव पितृतीर्थ तु यत्र गोदावरी नदी ।\\
पुरलिङ्गसहस्रेण सव्येतरजलावहा \textbar{}\textbar{}\\
जामदग्न्यस्य तत्तीर्थं रामायतनमुत्तमम् ।\\
प्रतीकस्य भयाद्भिन्ना यत्र गोदावरी नदी ॥\\
तीर्थं तद्धव्यकव्यानामप्सरोयुगसंयुतम् ।\\
श्राद्धाग्निकार्ये दानं च तत्र कोटिशताधिकम् ॥\\
तथा सहस्रलिङ्ग च राघवेश्वरमुत्तमम् ।\\
चन्द्रफेना नदी पुण्या यत्रेन्द्रः खदितः पुरा ॥\\
निहत्य नमुचिं शत्रु तपसा स्वर्गमाप्तवान् ।\\
तत्र दत्तं नरैः श्राद्धमनन्तफलदं भवेत् \textbar{}\textbar{}\\
तीर्थन्तु पुष्करं नाम शालग्रामं तथैव च ।\\
१९ वी० मि\\


{१४६ वीरमित्रोदयस्य श्राद्धप्रकाशे-}{\\
शोणापातश्च विख्यातो यत्र वैश्वानरालयः ॥\\
तीर्थ सारस्वतं नाम स्वामितीर्थं तथैव च ।\\
मणिदरा नदी पुण्या कौशिका चन्द्रिका तथा ॥\\
विदर्भा वायुवेगा च पयोष्णी प्राङ्मुखी तथा ।\\
कावेरी चोत्तराका च तथा जालन्धरो गिरिः \textbar{}\textbar{}\\
एतेषु श्राद्धतीर्थेषु श्राद्धमानन्त्यमुच्यते ।\\
लोहदण्डं तथा तीर्थ चित्रकूटं तथैव च ॥\\
विन्ध्ययोगश्च गङ्गायास्तथा नन्दातट शुभम् ।\\
कुब्जाम्रञ्च तथा तीर्थ उर्वशीपुलिने तथा ॥\\
संसारमोचनं तीर्थ तथैव ऋणमोचनम् ।\\
एतेषु पितृतीर्थेषु श्राद्धमानन्त्यमुच्यते ॥\\
अट्टहास तथा तीर्थ गौतमेश्वरमेव च ।\\
तथावशिष्टतीर्थे तु हारीतस्य ततः परम् ॥\\
ब्रह्मावर्ते कुशावर्ते हयतीर्थ तथैव च ।\\
पिण्डारकं च विख्यातं शङ्खोद्वारं तथैव च ॥\\
खण्डेश्वरं बिल्वकं च नीलपर्वतमेव च ।\\
तथा च वदतिीर्थ रामतीर्थ तथैव च ॥\\
जयन्तं विजयं चैव शुक्रतार्थ तथैव च ।\\
श्रीपतेश्च तथा तीर्थ तीर्थ रैवतकं तथा ॥\\
तथैव शारदातीर्थ भद्रकालेश्वरं तथा ।\\
वैकुण्ठतीर्थ च तथा भीमेश्वरपुरं तथा ॥\\
अश्वतीर्थेति विख्यातमनन्तं श्राद्धदानयोः ।\\
तीर्थं वेदशिरो नाम तत्रैवौद्यवती नदी ॥\\
तीर्थं वस्त्रपदं नाम छागलिङ्गं तथैव च ।\\
एषु श्राद्धप्रदातारः प्रयान्ति परम ( पदम् \textbar{}\textbar{}\\
तीर्थ मातृगृहं नाम करवीरपुरं तथा ।\\
सप्तगोदावरीतीर्थ सर्वतीर्थेष्वनुत्तमम् \textbar{}\textbar{}\\
तत्र श्राद्धं प्रदातव्यमनन्तफलमीप्सुभिः ।

{तथा।\\
एषु तीर्थेषु यच्छ्राद्धं तत्कोटिगुणमिष्यते ।\\
तस्मात्तत्र प्रयत्नेन तीर्थे श्राद्धं समाचरेत् \textbar{}\textbar{}}

{ श्राद्धे विहितनिषिद्धदेशनिरूपणम् । १४७}{\\
कुर्याच्छ्राद्धमयैतेषु नित्यमेव यथाविधि ।\\
प्राग्दक्षिणां दिशं गत्वा सर्वकामचिकीर्षया ।\\
अत्र च श्राद्धे प्रशस्तत्वेन कीर्तितानां नानादेशीयतीर्थानां स्व.\\
रूपाणि तत्तद्देशवासिभ्योऽधिगन्तव्यानि । अत्र च तीर्थादिदेशेषु\\
आनन्त्यादिश्रवणाद् गुणफलसम्बन्धपरा एव तद्विधयो न तु तीर्था.\\
दर्दानामङ्गत्वपराः । अतश्च तदभावेऽपि न श्राद्धवैगुण्यमिति ध्येयम् ।\\
अथ निषिद्धदेशा \textbar{}\\
वायुपुराणे ।\\
त्रिशङ्कोर्वर्जयेद्देशं सर्व द्वादशयोजनम् ।\\
उत्तरेण महानद्या दक्षिणेन तु कीकटम् \textbar{}\textbar{}\\
देशस्त्रैशङ्कवो नाम श्राद्धकर्माणि वर्जितः ।\\
कारस्कराः कलिङ्गाश्च सिन्धोरुत्तरमेव च ।\\
प्रनष्टाश्रमवर्णाश्च देशा वर्ज्याः प्रयत्नतः ॥\\
विष्णु. ।\\
न म्लेच्छविषये श्राद्धं कुर्यात् । न च गच्छेत् ।\\
तथा -\\
चातुर्वर्ण्यव्यवस्थानं यस्मिन् देशे न विद्यते ।\\
तं म्लेच्छदेशं जानीयादार्यावर्तमतः परम् (१) ॥\\
ब्रह्मपुराणे ।\\
परकीयगृहे यस्तु स्वान् पितॄंस्तर्पयेज्जडः ।\\
तद्भूमिस्वामिनस्तस्य हरन्ति पितरो बलात् ॥\\
अग्रभागं ततस्तेभ्यो दद्यान्मूल्यं च जीवताम् ।\\
गृह इति भूमिभागोपलक्षणम् \textbar{} "पारक्यभूमिभाग" इति यमव-\\
चनात् । स्वीयभूमेरभावे कथं श्राद्धं कर्तव्यमित्याकाङ्क्षायामुच्यते ।\\
अप्रभागमिति । श्राद्धीयद्रव्यस्य प्रथममेकदेशमुद्धृत्य भूमिस्वामिपितृ-\\
भ्यो दद्यात् । जीवतामिति - जीवत्सु भूमिस्वामिपितृषु मूल्यं दद्यादिति\\
शूलपाण्यादयः ।\\
अन्ये चकारो वार्ये स्वामिपितृभ्यो अग्रभागं मूल्यं वा दद्या.\\
दित्याहुः ।\\
इदं च पितृभ्योऽग्रदानं पितृरीत्या कार्यमिति मैथिलाः । देवरीत्येति\\
उपायकारः ।\\
(१) आर्यदेशस्ततः पर इति दानखण्डहेमाद्र। पाठः ।

{१४८ वीरमित्रोदयस्य श्राद्धप्रकाशे-\\
अन्ये तु एतस्य श्राद्धाङ्गत्वात् पार्वणादिश्राद्धे पितृरीत्या नान्दी-\\
मुझे तु देवरीत्येत्याहुः ।\\
इदं च श्रद्धारम्भात् प्राकू कार्यम् । परिवेषणात् प्राक्कार्यमिति\\
उपायकारः ।\\
यमः ।\\
अटव्यः पर्वताः पुण्या नदीतीराणि यानि च ।\\
सर्वाण्यस्वामिकान्याहुर्न हि तेषु परिग्रहः \textbar{}\textbar{}\\
वनानि गिरयो नद्यस्तीर्थान्यायतनानि च ।\\
देवखाताश्च गर्ताश्च न स्वाम्यं तेषु विद्यते । इति श्राद्धदेशाः ।\\
अथ श्राद्धदेशादपासनीयानि द्रव्याणि ।\\
तत्र देवलः ।\\
हीनाङ्गः पतितः कुष्ठी व्रणी पुलकसनास्तिकौ ।\\
कुक्कुटा: सूकरा: श्वानो वर्ज्या: श्राद्धेषु दूरतः\\
वीभत्सुमशुचिं ननं मत्त धूर्ते रजस्वलाम् ।\\
नीलकाषायवसनं छिन्नकर्ण विवर्जयेत् ।\\
शस्त्र कालायसं सीसं मलिनाम्बरवाससम् ॥\\
अन्नं पर्युषितं वापि श्राद्धेषु परिवर्जयेत् ।\\
पुरकसो =म्लेच्छविशेषः । नास्तिकः परलोकत्रासशून्यः \textbar{} सूकर=\\
विड्वराहः । बीभत्सु = जुगुप्सितः । छिन्नकर्णः = सन्धितकर्णः । इतरस्य\\
हीनाङ्गत्वेनैव निषेधात् । शस्त्रम् क्षुरकादि । कालायसम् = लोहम् ।
सीसं=\\
नागाख्यो धातुः । कालायसादिवत् पर्युषितमन्नमपि श्राद्धदेशस\\
भिहितं न कार्यम् ।\\
विष्णुः ।\\
न हीनाङ्गाधिकाङ्गाः श्राद्धं पश्येयुः, न शूद्रा न पतिता न महारो\\
गिणः, अयं च प्रकरणात श्राद्धाङ्गभूतो निषेधो न हीनाङ्गादीनां पुरु.\\
बार्थः, फलकल्पनापत्तेः ।\\
आपस्तम्बः । श्वभिरपपात्रैश्च श्राद्धदर्शनं परिरक्षेत् । खभिरिति व.\\
हुवचनात् ग्राम सुकरादीनां तादृशानां ग्रहणमिति तद्भाष्यम् । अपपा\\
त्रैः =अपष्टपात्रगुणैः । परिरक्षेत्= निवारयेत् ।\\
उशनाः ।\\
विद्वराहमार्जारकुक्कुटन कुलशूद्ररजस्वला शुद्रमतारा दूरम

{ }{ श्रद्धस्थानादपासनीयपदार्थनिरूपणम् । १४९}{\\
पनेतव्याः, श्राद्धप्रदेशादिति शेषः ।\\
मनुः ।\\
चाण्डालश्च वराहश्च कुक्कुटश्च तथैव च ।\\
रजस्वला च षण्डश्च नेक्षेरनश्नतो द्विजान् ॥\\
( अ० ३ श्लो० २३९)\\
होमे प्रदाने भोज्ये च यदभिरभिवीक्षितम् ।\\
दैवे हविषि पित्र्ये वा (१) तद्गच्छत्यसुरान् हविः ॥\\
( अ० ३ श्लो० २४० )\\
ईक्षणमत्र सन्निधानस्याप्युपलक्षणम् \textbar{} होमे = अग्निहोत्रादौ शा\\
त्यादिहोमे वा \textbar{} प्रदाने = मवादिदाने \textbar{} भोज्ये=
अदृष्टार्थे ब्राह्मणभोजने ।\\
दैवे हविषि-दर्शपौर्णमासादौ । पित्र्ये= श्राद्धे \textbar{}\\
यद्यपि च श्राद्धप्रकरणं तथापि वाक्येन तद् बाधित्वा होमादावय\\
प्रतिषेधो यथा सन्तर्दने (२) ।\\
अन्ये तु प्रकरणादेवं व्याचक्षते । होमे = अग्नीकरण होमे । प्रदाने\\
पिण्डप्रदाने । भोज्ये=भोज्योपकल्पनदेशे । दैवे = विश्वेदेवसम्बन्धिनि
\textbar{}\\
बाराद्वादस्यस्मिन्नपि स्थाने यदश्नादिकं तदित्यर्थः ।

{ पुनः स एव ।\\
घ्राणेन शुकरो हन्ति पक्षवातेन कुक्कुटः \textbar{}\\
श्वा तु दृष्टिनिपातेन स्पर्शेनावरवर्णजः ॥\\
( म० अ० ३ श्लो० २४१)\\
(३) खओ वा यदि वाऽश्रोत्रः कारुः प्रेभ्योऽपि वा भवेत् ।\\
हीनातिरिकगात्रो वा तमप्यपनयेत्ततः ॥\\
( म० अ० ३ श्लो० २४२ )\\
अवरवर्णजः = शूद्रः, तस्य द्विजातिश्राद्धे श्राद्धोपकरणस्पर्शप्रतिषे.\\
घोऽयं, नात्मार्थे । प्रेष्यः = परिचारकः ।\\
हेमाद्रौ स्मृत्यन्तरे ।\\
कुक्कुटो विवराहश्च काकः श्वापि विडालकः ।\\
(१) तद् गच्छत्ययथातथमिति मनुस्मृतौ पाठः ।\\
(२) पूर्वमी० अ० ३ पा० ३ अषि० १६ सू० २४-३१ ।\\
(३) खन्जो वा यदि वा काणो दातुः प्रेष्योऽपि वा भवेत् । इति मनुस्मृतौ पाठः
।

{१५० वीरमित्रोदयस्य श्राद्धप्रकाशे-\\
वृषलीपतिश्च वृषलः षण्डो (१) नारी रजस्वला ॥\\
एतानि श्राद्धकाले तु परिवर्ज्यानि नित्यशः ।\\
कुक्कुटः पक्षवातेन हन्ति श्राद्धमसंवृतम् ॥\\
घ्राणेन विड्वराहश्च वायसश्च रुतेन तु ।\\
श्वा तु दृष्टिनिपातेन मार्जारः श्रवणेन तु \textbar{}\textbar{}\\
वृषर्लापतिश्च दानेन चक्षुर्भ्यां वृषलस्तथा ।\\
छायया हन्ति वै षण्ड. स्पर्शेन च रजस्वला ॥\\
खञ्जः काणः कुणि: श्वित्री कारुः प्रेष्यकरो भवेत् ।\\
ऊनाङ्गो वातिरिक्ताङ्गस्तमाशु निनयेत्ततः ॥\\
विडालो= मार्जारः । वृषलः= शुद्रः \textbar{} असंवृतम् = अनावृतप्रदेशम् ।
श्रव.\\
णेन = शब्दश्रवणेन \textbar{} दानेन=पात्रीकृतः सन्नित्यर्थः ।\\
व्याघ्रपात ।\\
मद्यपः स्वैरिणी यावत् स्वैरिणीपतिरेव च ।\\
नैव श्राद्धेऽभिवीक्षेरन् ।\\
आवापः पाकं कर्तु तण्डुलादीनां पठि (पिठ )रादौ प्रक्षेपः\\
तत्प्रभृतिभोजननिष्पत्तिपर्यन्तं यावत् पाकस्थाने भोजनस्थाने\\
ऽन्यत्र वा अन्नादिकं किमपि न वीक्षेरन्नित्यर्थः । मद्यपादिग्रहणं\\
अप्रशस्तानामुपलक्षणम् ।\\
गौतमः ।\\
चाण्डालपतितावक्षणे दुष्टं, तस्मात् परिवृते दद्यात् तिलैर्वा\\
विकिरेत्, पडिपावनो वा शमयेत् ।\\
यतः दुष्टं दुष्यति अतः प्रच्छन्ने श्राद्धं कुर्यादित्यर्थः ।\\
बृहस्पतिः ।\\
श्वपाकषण्डपतितश्वानसूकरकुक्कुटान् ।\\
रजस्वलां च चाण्डालान् श्राद्धे कुर्याच्च रक्षणम् ॥\\
परिश्रितेषु दद्याच्च तिलैर्वा विकिरेन्महीम् ।\\
निनयेत् पङ्किमूर्धन्यस्तं दोषं पङ्गिपावनः \textbar{}\textbar{}\\
श्वपाको= निषादः \textbar{} षण्डो = नपुंसकः । एतान् बहिष्कृत्वा रक्षणं\\
कुर्यादित्यर्थः । निनयेत्=अपहरेत् ।

% \begin{center}\rule{0.5\linewidth}{0.5pt}\end{center}

( १ ) षण्डोऽवीरा रजस्वला । इति कमलाकरादुद्धृतेः पाठः ।

{ श्राद्धस्थानादपासनीय पदार्थनिरूपणम् । १५१\\
विष्णुः ।\\
संवृतं च श्राद्धं कुर्यात् न रजस्वलां पश्येश श्वानं न विड्वराहं\\
न ग्रामकुक्कुटं प्रयत्नात् श्राद्धमजस्य दर्शयेत् ।\\
श्राद्धं= श्राद्धस्थान् पदार्थान् । अजः = कृष्ण च्छागः ।\\
ब्रह्मपुराणे ।\\
नग्नादयो न पश्येयुः श्राद्धमेतत् कदा च न \textbar{}\\
गच्छन्त्येतैस्तु दृष्टानि न पितॄन् न पितामहान् ॥\\
दृष्टानि हवींषीति शेषः । नग्ना = वेदपरित्यागिनः \textbar{} आदिशब्दात्\\
तत्कर्मानुष्ठानत्यागिनो गृह्यन्ते ।\\
तथा च तत्रैवोकं ।\\
सर्वेषामेव भूतानां त्रयीसंवरणं यतः ।\\
ये वै त्यजन्ति तां मोहात्ते वै नझा इति स्मृताः ॥\\
बौद्धभावकनिर्ग्रन्थशाकजीवककापिलाः ।\\
येऽधर्माननुवर्तन्ते ते वै नझादयो जनाः ॥\\
बौद्धा:= सौगताः । श्राषकाः = श्वेतपटाः । निर्मन्था - जैना: । शाफा:-कौ•\\
लाः । जीवका=बाईरूपत्याः, चार्वाका इति यावत् । कपिलो = लोकायतैक.\\
देशी तेन प्रणीता: कापिलाः। तान् अधर्मान्=अधर्मपरान् । येऽनुवर्तन्त\\
इत्यर्थः ।\\
वायुपुराणे ।\\
वृथाजटी वृथामुण्डी वृथाननोऽपि यो नरः ।\\
महापातकिनो ये च ये च ननादयो नराः ॥\\
ब्रह्ममांश्च कृतघ्नांश्च नास्तिकान् गुरुतल्पगान् ।\\
दस्यूंश्चैव नृशंसांश्च दर्शनेन विवर्जयेत् ॥\\
ये चान्ये पापकर्माणः सर्वास्तानपि वर्जयेत् ।\\
देवतानामृषीणां च पापवादरताश्च ये \textbar{}\textbar{}\\
असुरान् यातुधानांश्च दृष्टमेभिर्वजत्यधः ।\\
अपुमानपविद्धश्च कुक्कुटो ग्रामशूकरः ॥\\
श्वा चैव हन्ति श्राद्धानि दर्शनादेव सर्वशः ।\\
नष्टं भूकृमिभिर्दृष्टं दीर्घरोगिभिरेव च ॥\\
पतितैर्मलिनैश्चैव न द्रष्टव्यं कदा च न ।

{१५२ वीरमित्रोदयस्य श्राद्धप्रकाशे-\\
नागरखण्डे ।\\
घरटोलूखलोत्थे च तथा शब्दे व्यवस्थितम् ।\\
शूद्रस्य वा विशेषेण तच्छ्राद्धं व्यर्थतां व्रजेत \textbar{}\textbar{}\\
व्यवस्थितं = कृतम् । इति अपासनीयानि ।\\
अथ श्राद्धोपकरणानि ।\\
तत्र कुशस्वरूपमाह हारीतः ।\\
अच्छिनाप्रान् सपवित्रान् समूलान् कोमलान् शुभान् ।\\
पितृदेवजपार्थ च समादद्यात् कुशान् द्विजः ॥ इति ।\\
अत्र च विशेषतः कुशलक्षणं तत्प्रतिनिधयस्तदाहरणप्रकार.\\
आहिक प्रकाशोको शेयः । पिण्डास्तरणकुशेषु विशेषो -\\
वायुपुराणे ।\\
रक्षिमात्रा: कुशाः शस्ताः पितृतीर्थेन संहनाः ।\\
उपमूलं तथा लूनाः पिण्डसंस्तरणे मताः ॥ इति ।\\
रत्निः = बद्धमुष्टिः करः । तिलाश्चोका-\\
भारते ।\\
वर्द्धमानतिलश्राद्धमक्षय्यं मनुरब्रवीत् ।\\
सर्वकामैः स यजते यस्तिलैर्यजते पितॄन् ॥ इति ।\\
मात्स्ये ।\\
तिलान्कृष्णानतिश्लक्ष्णानिति ।\\
इदं च न तिलान्तरप्रतिषेधार्थे किं तु कृष्णतिलप्रशंसार्थम् ।\\
गौराः कृष्णास्तथारण्यास्तथैव विविधास्तिलाः \textbar{}\textbar{}\\
पितॄणां तृप्तये सृष्टा इत्याह भगवान् मनुः । इति ब्रह्मपुराणात् ।\\
चाराहोऽपेि ।\\
जर्तिलास्तु निलाः प्रोक्ताः कृष्णवर्णा वनोद्भवाः ।\\
जर्तिलाश्चैव ते ज्ञेया अकृष्टोत्पादिताश्च ये ।\\
देवलः ।\\
तिलांश्च विकिरेत्तत्र परितो बन्धयेदजान् ।\\
असुरोपहत द्रव्यं तिलैः शुद्ध्यत्यजेन वा ॥ इति ।\\
अजेनेत्येकवचनं जात्यपेक्षया । विधौ अजानिति बहुवचनोपा-\\
दानात् ।

{श्राद्धोपकरणनिरूपणम् । १५३\\
यवाश्चोक्काथम त्कारखण्डे ।\\
श्राद्धेषु वैश्वदेवानि यानि कर्माणि कानि चित् ।\\
यवैरेव विधीयन्ते तानि श्रोत्रियपुङ्गवैः । ॥\\
कृष्णाजिनं च नागरखण्डे ।\\
सन्निधापायिता यश्च श्राद्धे कृष्णाजिनं नरः ।\\
प्राप्स्यन्ति पितरस्तस्य तृतिमाकल्पकालिकीम् ॥\\
आस्तीर्य दक्षिणाग्रीवमेतदुत्तरलोमकम् ।\\
सर्वान् श्राद्धस्य सम्भारानस्योपरि निवेशयेत् ॥\\
रजतञ्च नन्दिपुराणे ।\\
रजतेन समायुक्तं यद्यत् श्राद्धेषु किञ्चन ।\\
तत्तदक्षय्यतां याति रहस्यं पितृसम्मतम् ॥\\
अलाभे सति रुप्याणां नामापि परिकर्तियेत् ।\\
रूप्यहस्तेन दातव्यं यत् किञ्च पितृदेवतम् ॥\\
रजतं दक्षिणां दद्याच्छ्राद्धकर्मणि चैव हि ।\\
रजतस्य हस्ते धारणं च तर्जन्याम \textbar{}\\
तर्जन्यां रजतं धृत्वा पितृभ्यो यत् प्रदापयेत् ।\\
इति बराहपुराणोक्तेः ।\\
सुवर्णन्तु महाभारते ।\\
दश पूर्वान् दर्शवान्यान् तथा सन्तारयन्ति ते ।\\
सुवर्णे ये प्रयच्छन्ति इत्येवं पितरोऽब्रुवन् ॥\\
वर्णविशेषस्तु विष्णुधर्मोत्तरे ।\\
जाम्बूनदं तद्देवानामिन्द्रगोपकसन्निभम् ।\\
पितॄणां चन्द्ररश्म्याभ दैत्यानामसुरोपमम् ॥\\
अर्धपात्राणि= अर्धप्रकरणे -\\
कात्यायनः ।\\
सौवर्णराजतौ दुम्बरखड्गमणिमयानां पात्राणामन्यतमेषु यानि वा\\
विद्यन्ते पत्रपुटेषु वा ।\\
हारीतः । कांस्यपार्णराजतताम्रपात्राण्य र्घोदकधारणार्थानि सर्वाण्यु\\
पकल्प्यानि ।\\
बैजवापः ।\\
खादिरौदुम्बराण्यर्धपात्राणि श्राद्धकर्माणि ।\\
२० वी० मि०

१५४ वीरमित्रोदयस्य श्राद्धप्रकाशे -

{\\
(१) आप्याश्ममृत्मयानि स्युरपि पर्णपुटास्तथा ॥\\
मन्त्र विशेषो ब्रह्मपुराणे ।\\
स्रस्तभाण्डानि वर्ज्यानि पितृदैवतकर्मणि ।\\
सौवर्णता म्ररूप्याइमस्फाटिक शङ्खशुक्तयः \textbar{}\\
भिन्नान्यपि प्रयोज्यानि पात्राणि पितृकर्मणि ॥\\
मत्स्यपुराणे ।\\
पात्रं वनस्पतिमयं तथा पर्णमयं पुनः ।\\
जलजं वापि कुर्वीत तथा सागरसम्भवम् ॥\\
सागरसम्भवम् = युक्त्यादि ।\\
ब्रह्मपुराणे ।\\
दत्वा हेममये पात्रे रूपवान् स्यात् स मानवः ।\\
दत्वा रत्नमये पात्रे सर्वरत्नाधिपो भवेत् ॥\\
पालाशे ब्रह्मवर्चस्वी अश्वत्थे राज्यमाप्नुयात् ।\\
पात्र मौदुम्बर कृत्वा सर्वभूताधिपो भवेत् ॥\\
दत्वा न्यग्रोधपात्रे तु प्रजां पुष्टि श्रियं लभेत् ।\\
रक्षोघ्ने काश्मरीपात्रे दत्वा पुण्यं लभेद्यशः \textbar{}\textbar{}\\
सौभाग्यं चाथमाधूके फल्गुपात्रे च सम्पदः ।\\
श्वेतार्क मन्दारमये दत्वा च मतिमान् भवेत् ॥\\
बिल्वपात्र धनं बुद्धिं दीर्घमायुरवाप्नुयात् ।\\
अब्जपत्र पुढे दत्त्वा मुनीनां वल्लभो भवेत् \textbar{}\textbar{}\\
सिक्ते मधुघृताभ्यां च यथा सम्भवमेव च ।\\
काश्मरी=गम्भारीः \textbar{} फल्गु = काकोदुम्बरिका \textbar{}\\
ब्रह्मवैवर्ते ।\\
पलाशफल्गुन्यग्रोधप्लक्षाश्वत्थविकङ्कताः ।\\
उदुम्बरस्तथा बिल्वश्चन्दना यज्ञियाश्च ये ॥\\
सरलो देवदारुश्च शालोऽथ खदिरस्तथा ।\\
एते ह्यर्घादिपात्राणां योनयः परिकीर्तिताः ॥

% \begin{center}\rule{0.5\linewidth}{0.5pt}\end{center}

{\\
(१) अत्र 'अपाश्ममृ-मयानीति, आदर्शपुस्तके, 'अप्यश्ममृन्मयानीति, श्राद्ध\\
काशिकायाम्, अथाश्ममृन्मयानीति, श्राद्धमयूख, पाठ उपलभ्यते, अस्माभिः
पुनः\\
"जलजं वापि कुर्वीतेति मत्स्यपुराणवचनात्, अकार
पकारघटितादर्शपुस्तकपाठानुगुण्या•\\
पच 'आप्याश्ममृन्मयानीत्येव पाठो युक्त इति मत्वा स एव सन्निवेशित इति ।

 श्राद्धोपकरणनिरूपणम् । १५५

{तत्रैव-\\
राजतं रजताकं वा पितॄणां पात्रमुच्यते ।\\
पुनस्तत्रैव ।\\
वर्षत्यजस्त्रं तत्रैव पर्जन्यो वेणुपात्रतः ।\\
एषां पात्राणामेकस्मिन् प्रयोगे एकजातीयानि ग्राह्याणि । एषाम•\\
न्यतम इति कात्यायनोक्ते । इत्यर्घपात्राणि ।\\
अथ पाकपात्राणि ।\\
नागरखण्डे \textbar{}\\
शुचीनि पितृकार्यार्थ पिठराणि प्रकल्पयेत् ।\\
सौवर्णान्यथ रौप्याणि कांस्यताम्रोद्भवानि च ॥\\
मार्तिकान्यपि भव्यानि नूतनानि दृढानि च ।\\
न कदाचित्पचेदन्नमयः स्थालीषु पैतृकम् ॥\\
अयसो दर्शनादेव पितरो विद्रवन्ति हि ।\\
अपैतृकं ह्यमङ्गल्यं लोहमाहुर्मनीषिणः ॥\\
दैवेषु चैव पित्र्येषु गर्हितं सर्वकर्मसु ।\\
कालायसं विशेषेण निन्दितं पितृकर्मणि ॥\\
फलानां चैव शाकानां छेदनार्थानि यानि तु ।\\
महानसेऽपि शस्त्राणां तेषामेव हि सन्निधिः ॥\\
दृश्यते नेतरच्छस्तं शस्त्रमात्रस्य दर्शनम् ।\\
अयःशङ्कुमयं पीठं प्रदेयं नोपवेशने ॥\\
आदित्यपुराणे पात्राण्युपक्रम्य -\\
अच्छिद्रेषु विलेपेषु तथानुपहतेषु च ।\\
नायसेषु न भिन्नेषु दूषितेष्विव कर्हि चित् \textbar{}\textbar{}\\
पूर्व कृतोपभोगेषु मृन्मयेषु न कुत्रचित् ।\\
पक्कान्नस्थापनार्थे तु शस्यन्ते दारुजान्यपि ॥\\
पात्रेषु फलविशेषश्चमत्कारखण्डे ।\\
स्वर्णभाण्डेषु कुर्वीत पितॄणां पाकमुत्तमम्\\
तेन सौभाग्यमतुलं लभते चाक्षयां श्रियम् ॥\\
राजतेषु हि पात्रेषु कुर्वन् पाक हि पैतृकम् \textbar{}\\
पितॄणां कुरुते प्रीतिं यावदाभूतसम्प्लवम् \textbar{}\textbar{}\\
पचमानस्तु भाण्डेषु भक्त्या ताम्रमयेषु च ।

{१५६ वीरमित्रादयस्य श्राद्धप्रकाशे- }{\\
समुद्धरति वै घोरात् पितॄन् दुःखमहार्णवात् ॥ इति पाक-

{पात्राणि ।\\
अथ भोजनपात्राणि ।

{अत्रि ।\\
भोजने हैमरौप्याणि दैवे पित्र्ये यथाक्रमम् ।\\
हारीत. ।\\
रजत कांस्यपर्णताम्रपात्राणि ब्राह्मणभोजनार्थानि महान्ति कार्याणि ।\\
महान्ति = भोज्यान्न पर्याप्तानि । पर्णमू=पलाशः ।\\
'पलाशेभ्यो विना न स्युरन्यपात्राणि भोजने'\\
इत्यत्रिवचनात् ।\\
वाराहपुराणे ।\\
सौवर्णानीह पात्राणि सम्पाद्यानि प्रयत्ततः ।\\
तदलाभे तु रौप्याणि कांस्यानि तदसम्भवे ॥\\
पलाशपर्णजानि स्युः श्राद्धे तु द्विजभोजने ।\\
अन्यान्यपि च पात्राणि दारुजान्यपि जानता ॥\\
यथोपपत्रं कार्याणि मृन्मयानि न तु क्वचित् ।\\
नायसानि प्रकुर्वीत पैतलानि न हि क्वचित् ।\\
नवसीसमयानीह शस्यन्ते त्रपुजान्यपि ।\\
सुवर्णादिविधेरेवाऽऽयसादिनिवृत्तौ पुनस्तनिषेधः, प्रतिनिधित्वे.\\
मापि तन्निवृत्यर्थ: ।\\
अङ्गिराः ।\\
न जातीकुसुमानि दद्यात् न कदलीपत्रमिति ।\\
अत्रापवादः स्मृतिसङ्ग्रहे ।\\
कदली चूतपनसजम्बूपत्रार्कचम्पकाः ।\\
अलाभे मुख्यपात्राणां ब्राह्याः स्युः पितृकर्मणि ॥ इति भोजन 

{पात्रणि ।\\
अथ परिवेषणपात्राणि ।\\
कूर्मपुराणे ।\\
काञ्चनेन तु पात्रेण राजतौदुम्बरेण च ।\\
दत्तमक्षयतां याति खड्गेन च विशेषतः \textbar{}\textbar{}\\
परिवेषणं प्रक्रम्य विष्णु ।

{ श्राद्धोपकरणनिरूपणम् । १५७\\
हस्ते न घृतव्यञ्जनादि ।\\
पात्राभावे आह-\\
वृद्धशातातपः ।\\
हस्तदत्ताश्च ये स्नेहा लवणं व्यञ्जनानेि च ।\\
दातारं नोपतिष्ठन्ति भोक्ता भुञ्जति किल्विषम् \textbar{}\textbar{}\\
तस्मादन्तरितं देयं पर्णेनैव तृणेन वा ।\\
प्रदद्यान्न तु हस्तेन नायसेन कदाचन ॥ इति ।\\
पात्रव्यवस्थामाह पैठीनसिः-\\
दैवे सौवर्णानि पित्र्येषु राजतानि परिवेषणपात्राणीति ।\\
मार्कण्डेये ।\\
नापवित्रेण हस्तेन नैकेन न विना कुशम् \textbar{}\\
नायसे नायसेनैव श्राद्धं तु परिवेषयेत् ॥\\
~\\
अपवित्रेण= दुर्लेप संसर्गादिनेति हेमाद्रि, सुवर्णरजतान्यतरही नेने.\\
स्यपरे । नैकेन = किन्तु वामोपगृहीतेनेति हेमाद्रि ;
दर्व्यादिशून्येनेत्यपरे ।\\
एकत्राऽऽयस इति सप्तमी । इति परिवेषणपात्राणि ।\\
अथ गन्धा: ।

{भगवतीपुराणे ।\\
यञ्चन्दनेन गन्धेन सुश्लक्ष्णेन सुगन्धिना ।\\
अनुलिम्पति वै विप्रान् स कदाचिन्न तप्यते ॥\\
तेनैवागुरुमिश्रेण परं सौभाग्यमश्नुते ।\\
अगुरोश्चापि लेपेन भुङ्क्ते भोगाननुत्तमान् \textbar{}\textbar{}\\
यः श्रीखण्डं सकर्पूरं पितृभ्यः प्रतिपादयेत् ।\\
यशः प्राप्नोति विपुलं स पृथिव्यां महामतिः ॥\\
यः कुङ्कुमसमोपेतं ददाति मलयोद्भवम् ।\\
केवलं कुङ्कुमं वापि रूपवान् स प्रजायते ॥\\
दद्यान् मलयज गन्धं यस्तु कस्तूरिकायुतम् ।\\
कस्तूरी केवलां वापि स प्राप्नोति महाभियम् ॥\\
यो यक्ष कर्दमें दन्ते श्राद्धेषु श्रद्धयान्वितः ।\\
स भूपतित्वमासाद्य महेन्द्र इव मोदते ॥\\
यस्तु श्राद्धे ददेद्गन्धान्नानाकुसुमवासितान् ।\\
स सर्वकामसंयुक्तः स्वर्गे वैमानिको भवेत् ।

{१५८ वीरमित्रोदयस्य श्राद्धप्रकाशे-\\
यक्षकर्दमलक्षणम् \textbar{}\\
कर्पूरा गुरुकङ्कोलदर्पकुङ्कुमचन्दनैः ।\\
यक्षकर्दम इत्युक्तो गन्धः स्वर्गेऽपि दुर्लभः ॥\\
इति विष्णुधर्मोत्तरोत्तमनुसन्धेयम् ।\\
ब्रह्मपुराणे ।\\
श्वेतचन्दनकर्पूरकुङ्कुमानि शुभानि च ।\\
विलेपनार्थे दद्यात्तु यञ्चान्यत् पितृवल्लभम् \textbar{}\textbar{}\\
कुष्टं माँसी बालकं च त्वक्कुष्टी जातिपत्रकम् ।\\
नालिकोशीरमुस्तं च ग्रन्थिपर्णे च तुम्बुरम् \textbar{}\\
मुरा चेत्येवमादीनि गन्धयोग्यानि पैतृके ॥ इति ।\\
कुष्टीः = गन्धद्रव्यभेदः । कुष्टं=कुठ इति प्रसिद्धम् । मांसी = जटामांसी
\textbar{}\\
बाल=सुगन्धिवाला इति प्रसिद्धम् । वक्=दारचीणीति प्रसिद्धा ।\\
नालिक=तज इति प्रसिद्धम् ।\\
मरीचिः ।\\
कर्पूरकुङ्कुमोपेतं सुगन्धिसितचन्दनम् ।\\
दैविकेऽप्यथवा पित्र्ये गन्धदानं प्रशस्यते ॥ इति ।\\
अथ वर्ज्यगन्धाः \textbar{}\\
नरसिंहे ।\\
श्राद्धेषु विनियोक्तव्या न गन्धा जीर्णदारुजाः ।\\
कल्कीभावं समासाद्य न च पर्युषिताः कचित् ॥\\
न विगन्धाश्च दातव्या भुक्तशेषा विशेषतः ।\\
ब्रह्मपुराणे ।\\
पूतीकां मृगनाभि च रोचनां रक्तचन्दनम् ।\\
कालीयकं जोगकं च तुतुष्कं चापि वर्जयेत् ॥\\
पूर्तीका = सुगन्धितृणविशेषः, करञ्जो वेति हेमाद्रिः । कस्तूर्या विहि\\
तप्रतिषिद्धत्वाद्विकल्पः । तुतुष्क=सिह्नकम् ।\\
अथ पुष्पाणि ।\\
वायुपुराणे ।\\
पितृभ्यो यस्तु माल्यानि सुगन्धनि च दापयेत् ।\\
सदा दाता श्रिया युक्तः सोऽपि याति दिवाकरम् ॥

{ }{ श्राद्धोपकरणनिरूपणम् । १५९}{\\
तथा ।\\
यस्तु श्राद्धे द्विजाग्र्याणां पुष्पाणि प्रतिपादयेत् ।\\
सुगन्धीनि मनोज्ञानि तस्य स्यादक्षयं फलम् ॥\\
अतः पत्रैश्च पुष्पैश्च मञ्जरीभिरथापि वा ।\\
सुकुमारैः किशलयैनवदूर्वाङ्कुरैरपि ॥\\
न प्रसुनैर्विना पूजा कृता पुण्यतमा भवेत् ।\\
ब्रह्मपुराणे ।\\
शुक्लाः सुमनसः श्रेष्ठास्तथा पद्मोत्पलानि च ।\\
गन्धरूपोपपन्नानि यानि चान्यानि कृत्स्नशः ॥\\
नन्दिपुराणे ।\\
पुष्पजातिर्यदा सृष्टा तदा प्राक्शतपत्रिका \textbar{}\\
सृष्टा तेन च मुख्या स्यात् श्राद्धकर्मणि सर्वथा ॥\\
मार्कण्डेयपुराणे ।\\
जात्यश्च सर्वा दातव्या मल्लिका श्वेतयूथिका \textbar{}\\
जलोद्भवानि सर्वाणि कुसुमानि च चम्पकम् ॥\\
जात्यो = मालत्यः । मल्लिका=मुकुरपुष्पाणि ।\\
स्मृत्यन्तरे तु 'जातीदर्शनमात्रेण निराशाः पितरो गता' इति जा\\
त्यो निषिद्धा अतो विकल्पः । पीतजातीविषयो निषेध इति केचित्\\
तत्र ; जात्यः सर्वा इत्युक्तेः । जलोद्भवानि रक्तान्यपि देयानि "जल-\\
जानि रक्तान्यपि दद्यातु" इति विष्णूतेः ।\\
ब्रह्मपुराणे ।\\
शाकमारण्यकं चैव दद्यात् पुष्पाण्यमूनि वै ।\\
जातीचम्पकलोध्राश्च मलिका वाणवर्वरी ॥\\
चूताशोकाटरुषं च तुलसी तिलक तथा ।\\
पाटलीं शतपत्रं च गन्धनेपालिकामपि ॥\\
कुब्जकं तगरं चैव मृगमारं च केतकीम्\\
यूथिकामतिशुक्ल च श्राद्धयोग्यानि भो द्विजाः ॥\\
कमल कुमुदं पद्मं पुण्डरीकं च यत्नतः ।\\
इन्दीवर कोकनदं कल्हारं च नियोजयेत् ॥ इति ।\\
लोधो= गालवः । वाण=आर्तवः । वर्वरी=कवरीति हेमाद्रिः । तिलक.=

{१६० वीरमित्रोदयस्य श्राद्धप्रकाशे-\\
तगरः । गन्धनेपालिका=वनमल्लिका \textbar{} दुजक्म्=अजकम् \textbar{}
तगरो=गन्धत.\\
गर: । मृगमारः = श्रावेणिकापुष्पमिति हेमाद्रिः । अतिशुक्लं = माधवील-\\
तापुष्पम् ।\\
यस्तु विष्णुना सर्षपसुरसाकूष्माण्डानि वर्जयेदिति तुलस्या नि\\
षेध उक्तः स शाकप्रायपाठाच्छाकत्वेनेत्युक्तं द्रव्यनिर्णये ।\\
अथ निषिद्धानि ।\\
वायुब्रह्माण्डपुराणयोः ।\\
जपादिसुमना भाण्डी रूपिका सकुरुण्टिका ।\\
पुष्पाणि वर्जनीयानि श्राद्धकर्मणि नित्यशः ॥\\
यानि गन्धाइपेतानि उग्रगन्धीनि यानि च ।\\
वर्जनीयानि पुष्पाणि भूतिमन्विच्छता सदा ॥\\
जपादीत्यादिशब्देन करवीरादि । सुमना=जाति. । भाण्डी=मञ्जिष्ठा \textbar{}\\
ऋषिका=अर्कः । कुरुण्टिका = पीताम्लातकपुष्पम् ।\\
शङ्खः ।\\
उग्रगन्धीन्यगन्धीनि चैत्यवृक्षोद्भवानि च ।\\
पुष्पाणि वर्जनीयानि रक्तवर्णानि यानि च ॥\\
जलोद्भवानि देयानि रक्तान्यपि विशेषतः ॥ इति ।\\
चैत्यवृक्ष :- श्मशानवृक्षः ।\\
मात्स्ये ।\\
पद्मबिल्वार्कधन्तरपारिभद्वाटरूषकाः ।\\
न देयाः पितृकार्येषु पय आजाविकं तथा ॥ इति ।\\
अत्र पद्मं = स्थलजम् \textbar{} जलजस्य विहितत्वादिति हेमाद्रिः । पारि\\
भद्रो =रकस्तबको मन्दारः ।\\
विष्णुः ।\\
सितानि सुगन्धीनि कण्टकितान्यपि दद्यात् । अनेनानैवंविधानि\\
कण्टकिजातानि न देयानीति गम्यते ।\\
अथ धूपाः ।\\
शङ्खः ।\\
धूपार्थे गुग्गुलुं दद्यात् घृतयुक्तं मधूत्कटम् ।\\
चन्दनं च तथा दद्यात् कर्पूरं कुङ्कुम शुभम् ॥

{ श्राद्धोपकरणनिरूपणम् । १६१\\
ब्रह्मपुराणे ।\\
चन्दनागुरुणी चोभे तथैवोशीरपद्मकम् ।\\
तुरुष्कं गुग्गुलुं चैव घृताक्तं युगपद्दहेत् ॥\\
ब्रह्मवैवर्ते=धूपं प्रक्रम्य ।\\
चन्दनागुरुणी चैव तमालोशीरपत्रकम् ।\\
वायुपुराणे ।\\
गुग्गुलवादींस्तथा धूपान् पितृभ्यो यः प्रयच्छति ।\\
संयुज्य मधुसर्पिभ्य सोऽग्निष्टोमफलं लभेत् ॥\\
अथ निषिद्धधुपा. \textbar{}\\
विष्णुः ।\\
जीवजं सर्वे न धूपायें, तैलघृते च दद्यादिति ।\\
जीवजं=नखकस्तूरिकादि । तैलघृते केवले न दद्यात् संयुज्य\\
मधुसर्पिभ्यांमिति पूर्वोदाहृतवायुपुराणात्,\\
घृतं न केवलं दद्याद् दुष्टं वा तृणगुग्गुलुम् ।\\
इति मदनरत्नधृतवचनाच्च । तृणगुग्गुलुः = सर्जरस इति तत्रैव.\\
व्याख्या \textbar{}\\
अथ दीपाः ।\\
स्कान्दे ।\\
स्थाप्याः प्रतिद्विजं दर्दापाः इवेतसूत्रजवर्तयः ।\\
गव्येण माहिषेणापि घृतेन भृतभाजनाः ॥\\
अथवा तिलतैलेन पूरिता विमलार्चिषः ।\\
पितॄनुद्दिश्य दातव्याः प्रत्येकं ते यथाविधि ॥ इतेि ।\\
तैलं घृताभावे । घृताभावे तु यो दीप तिलतैलप्रवर्तितम्,\\
इति पाद्मोक्तः । तैलाभावे तत्रैव,\\
अभावे तिलतैलस्य स्नेहै: प्राण्यङ्गवर्जितैरिति ।\\
दीपं दद्यादिति शेषः । स्नेहाः = एरण्डकुसुम्भातसीबी\\
जोद्भवा, न वसादयः ।\\
घृतेन दीपो दातव्यस्त्वथवा औषधीरसैः ।\\
वसामेदोद्भवं दीपं प्रयत्नेन विवर्जयेत् ॥\\
इति शङ्खोक्तेः ।\\
बी० मि‍ २१

१६२ वीरमित्रोदयस्य श्राद्धप्रकाशे -

{कलिकापुराणे ।\\
दीपिकां धातुसंयुक्तां सालदारुमयमिपि ।\\
अलाभे मृन्मय वापि मानाभ्यामधिसस्थिताम् ।\\
यो ददाति पितृभ्यस्तु तस्य पुण्यफलं श्रृणु ।\\
यो धृपदहनं पात्रं पात्रमारार्त्तिकस्य च ॥\\
दद्यात् पितृभ्यः प्रयतस्तस्य स्वर्गेऽक्षया गतिः । इति ।\\
मानं = तैलधारणस्थलम् ।\\
अधाच्छादनम् ।

{ब्रह्मपुराणे ।\\
अनङ्गलग्नं यद्वस्त्रं सम्भवे (१) तु युगं शुभमिति ।

{पद्मपुराणे ।\\
संपूज्य गन्धपुष्पाद्यैर्दद्यादाच्छादनं ततः ।\\
अधौतं सदशं स्थूलमच्छिद्रममलीमसम् \textbar{}\textbar{}\\
तस्याभावे तु देयं स्यात् सवर्णैः क्षालितं च यत् ।\\
प्रदेयं पितृकार्येषु कारुधौतं न कर्हि चित्र \textbar{}\textbar{}\\
अधौतस्य दानं मण्डराहित्ये ।\\
स्मृत्यन्तरे ।\\
कौशेयं क्षौमकार्पासं दुकूलमहतं तथा ।\\
आवेष्टनानि यो दद्यात् कामानाप्नोति पुष्कलान् ॥\\
कौशेयं = कृमिकोशोत्थम् । क्षौमम्=आतसम् । अहतलक्षणमाह-\\
प्रचेताः ।\\
ईषद्धौतं नवं श्वेतं सदशं यन्त्र धारितम् ।\\
अहतं तद्विजानीयात् सर्वकर्मस्वपावृतम् ॥ इति ।\\
ईषद्धौतं = स्वेनेति शेषः ।\\
तथाच वृद्धमनुः ।\\
स्वयं धौतेन कर्तव्या: क्रिया धर्म्या विपश्चिता ।\\
न निर्णेजकधौतेन नोपयुक्तेन वा क्वचित् ।\\
स्वयं ग्रहणादेव निर्णेजकनिवृत्तावपि पुनर्निर्णेजक प्रतिषेधो ऽन्ये\\
नापि ब्राह्मणादिना शुद्धिः कार्येत्येव मर्था[न्तरपरत्वात् ]
\textbar{} "तस्याभावे\\
तु देयं स्यात् सवर्णैः क्षालितं च यत्" इति मत्स्यपुराणैकवाक्यत्वाच्च ।\\
( १ ) तद् युग शुभमिति श्राद्धतत्वे पाठः ।

{ श्राद्धोपकरणनिरूपणम् \textbar{} १६३}

{भगवतीपुराणे ।\\
अधरीयोत्तरीयार्थे उदिश्येकैकमादरात् ।\\
वासोयुगं प्रदातव्यं पितृकृत्ये विपश्चितैः ॥\\
निष्क्रयो वा यथाशक्ति वस्त्रालाभे प्रदीयते ।\\
स्कन्दपुराणे ।\\
महाधनानि वासांसि पितृभ्यो यः प्रयच्छति ।\\
धनधान्यसमृद्धोऽसौ सुवेषश्चैव जायते ॥\\
रूपवान् सुभगः श्रीमान् वनितानां च वल्लभः ।\\
आयुरारोग्यसंपन्नः कीर्ति विन्दति चामलाम् ॥\\
चन्द्रिकाजालशुभ्राणि यः क्षौमाणि प्रयच्छति ।\\
स चान्द्रमसमासाद्य लोकं दीव्यति देववत् ॥\\
दत्वा श्रौमाणि शोणानि सूर्यलोक समश्नुते ।\\
पीतानि तानि दत्वा वै याति लोकं मधुद्विषः ॥\\
चित्राणि दत्वा माहेन्द्रे लोके नित्यं महीयते ।\\
पट्टसूत्रमये दत्त्वा वाससी पितृतत्परः \textbar{}\\
रूपसौभाग्यसंपन्नो राजराजो भवेदिह \textbar{}\textbar{}\\
कौशेयान्यपि वासांसि पितृभ्यो यः प्रयच्छति ।\\
स नाकपृष्ठे रमते दिव्यैर्भोगैः शतं समाः ।\\
कार्पाससुत्रजं वासः सुसूक्ष्म चातिशोभनम् ॥\\
यो ददाति पितॄणां वै सोऽनन्तसुखमाप्नुयात् ।\\
मञ्जिष्ठाद्यैः शुभैः र रक्षितं च मनोहरम् \textbar{}\textbar{}\\
प्रदाय पितृदेवेभ्वः परमामृद्धिमृच्छति ॥\\
विष्णुधर्मोत्तरे \textbar{}\\
यः कम्बुकं तथोष्णीषं पितृभ्यः प्रतिपादयेत् ।\\
द्वन्द्वोद्भवानि दुःखानि स कदाचिन्न पश्यति ॥\\
ददाति यः प्रसन्नात्मा पटान् कम्बु {[}ञ्चु{]} कबन्धनान् ।\\
विमुक्तः सर्वपापेभ्यः सोऽग्यां विदन्ति सम्पदम् ॥\\
स्त्रीणां श्राद्धेषु सिन्दूरं दधुश्चण्डातकानि च ।\\
निमन्त्रितास्वः स्त्रीभ्यो ये ते स्युः सौभाग्यसंयुताः ॥\\
चण्डातकं = स्त्रीपरिधेयवस्त्रविशेषः । ``अर्धोरुकं विलासिन्या\\
वासश्चण्डातकं विदुः" इति स्मरणात् ।\\


{१६४ वीरमित्रोदयस्य श्राद्धप्रकाशे-\\
तथा ।

{ वस्त्रस्थापनभाण्डानि सवस्त्राणि प्रयच्छति ।\\
यः पितृभ्यः स सम्पद्भिः सर्वाभिरपि पूर्यते ॥\\
शीतातपसमुद्भूतां पीडां वारयितुं क्षमान् ॥\\
यः प्रावारानतिश्लक्ष्णान् विशालान् सुदृढान् नवान् \textbar{}\textbar{}\\
दद्यात् पितृभ्यस्तस्येह द्वन्द्वदुःख न विद्यते ।\\
ऊर्णायुलोमरचितान् विविधांश्च महापटान् \textbar{}\textbar{}\\
विचित्रान् विविधै गगैर्वातप्रावरणांचितान् ।\\
प्रयच्छति पितॄणां य. स सदाऽऽरोग्यवान् भवेत् ॥\\
वायुपुराणे,\\
वस्त्रं हि सर्वदेवत्यं सर्वदेवैरभिष्टुतम् ।\\
वस्त्राभावे क्रिया न स्युर्यशदानादिकाः क्वचित \textbar{}\\
तस्मात् वस्त्राणि देयानि श्राद्धकाले विशेषत ॥\\
ब्रह्मवैवर्ते विशेषः ।\\
कौशेयक्षौमपत्रोर्णान् तथा प्रावारकम्बलान् ।\\
अजिनं रौरवं यत् स्यात् ऊर्णिकं मृगलोमकम् ॥\\
दत्वा ह्येतानि विप्रेभ्यो भोजयित्वा यथाविधि ।\\
प्राप्नोति श्रद्दधानस्य वाजपेयस्य यत् फलम् ॥\\
बह्यो नार्यः सुरूपाश्च पुत्रा भृत्याश्च किङ्कगः ।\\
वशे तिष्ठन्ति भूतानि वपुर्विन्दत्यनामयम् \textbar{}\textbar{}\\
अलक्ष्मी नाशयत्याशु तमः सूर्योदये यथा ।\\
विभ्राजते विमानाने नक्षत्रोग्विव चन्द्रमाः \textbar{}\textbar{}\\
नित्यश्राद्धेषु यो दद्यात् वस्त्रं पितृपरायणः ।\\
सर्वान् कामानवाप्नोति राज्यं स्वर्गे तथैव च ॥\\
अथ निषिद्धवस्त्राणि ।\\
आदित्य पुराणे ।\\
न कृष्णवर्ण दातव्यं वासः कार्पाससम्भवम् ।\\
पितृभ्यो नापि मलिनं नोपयुक्तं कदा च न \textbar{}\textbar{}\\
न छतं नापदशं न धौत कारुणापि च ।\\
ब्रह्मपुराणे,\\
कार्पासं नैव दातव्यं पितृम्य' काममंशुकम् ।

{ श्राद्धोपकरणानिरूपणम् । १६५\\
कृष्णं चापि प्रदातव्यमन्यत् कार्पाससम्भवात् ॥\\
नामापि न ग्रहीतव्यं नीलीरक्तस्य वासलः ।\\
दर्शनात कीर्तनात् भीत्या निराशाः पितरो गताः ॥\\
अथ यज्ञोपवीतम् ।\\
आदित्यपुराणे ।\\
पितॄन् सत्कृत्य वासोभिर्दद्याद्यज्ञोपवतिकम् ।\\
यज्ञोपवीतदानेन विना श्राद्धं तु निष्फलम् ॥\\
तस्मात् यज्ञोपवीतस्य दानमावश्यकं स्मृतम् ॥ इति ।\\
वायुपुराणे ।\\
उपवीतं तु यो दद्यात् श्राद्धकर्मणि धर्मवित् ।\\
पावनं सर्वविप्राणां ब्रह्मदानस्य तत् फलम् ॥ इति ।\\
अत्र शूद्रकर्तृकश्राद्धे षज्ञोपवीतस्य श्राद्धाङ्गत्वेन देयत्वावगमा-\\
ज्ञानं भवत्येवेति -- हेमाद्रिः । एवं स्त्र्याद्युद्देश्यकश्राद्धेऽपि ।\\
भविष्यपुराणे -\\
दद्याद्यशोपर्वातानि पितॄणां प्रीतये सदा ।\\
श्रद्धावान् धार्मिकस्तेन जायते ब्रह्मवर्चसी ।\\
हेमाद्रौ चमत्कारखण्डे \textbar{}\\
सितसूक्ष्मेण सूत्रेण रचितं मन्त्रपूर्वकम् \textbar{}\\
उपवीतं ददत् श्राद्धं मेघावानभिजायते ॥\\
चामीकरमयं दिव्यं पितृणामुपवीतकम् \textbar{}\\
दत्वा चामीकरमयैर्विमानैर्दिवि दीव्यति ॥\\
राजतान्युपर्वातानि पितॄणां ददतः सदा ।\\
आयुः प्रज्ञा च तेजश्च यशश्चैवाभिवर्धते ॥\\
अत्र च यज्ञोपवीतस्य प्राधान्यावगतेः प्रधान्येन दानम् ।\\
वस्त्राभावेऽपि दातव्यमुपवीतं विजानता ।\\
पितृणां वस्त्रदानस्य फलं तेनाप्नुतेऽखिलम् ॥\\
इति ब्रह्मवैवर्ते वस्त्राभावे यज्ञोपवीतस्मरणाच्च प्रतिनिधित्वेनापि ।\\
अथ दण्डयोगपट्टी \textbar{}\\
विष्णुधर्मोतरे-\\
दण्डान् भाद्धेषु यो दद्यात् पितृप्रीत्यै महामनाः।\\
कदाचित्तं न बाधन्त आपदः श्वापदोद्भवाः ।\\


{१६६ वीरमित्रोदयस्य श्राद्धमका शे-}{\\
दण्डांश्च योगपट्टांश्च योगिभ्यो यः प्रयच्छति ।\\
योगिनामुपयुक्तानि वस्तून्यन्यानि यानि च \textbar{}\textbar{}\\
कामैस्तमभिवर्धन्ते पितरो योगवित्तमाः ।\\
पालाशान् वैणवान् वापि यस्तु दण्डान् यथोचितम् ॥\\
व्रतिभ्यो वा गृहस्थेभ्यो यतिभ्यः परितुष्टये ।\\
ददाति योगपट्टांश्च पट्टसूत्रादिनिर्मितान् ॥\\
स योगिनां कुले भूत्वा योगिराजः प्रजायते ।\\
आदित्यपुराणे विशेष. -\\
वैणवान् सुदृढान् दण्डान् ऋजून् सन्नतपर्वणः ।\\
पितृप्रीतिकृते दद्यान्नभय जातु विन्दति ॥\\
आयसेन जनित्रेण मूलदेशे परिष्कृतान् ।\\
मृत्कुशानां यथाकाम खननेषु क्षमान् भृशम् ॥\\
केवलान् वाथ दण्डान् वा वः श्राद्धे प्रतिपादयेत् ।\\
तस्य श्रद्धां च मेधां च शौचमास्तिक्यमेव च ॥\\
इह जन्मनि चान्यत्र प्रयच्छन्ति पितामहाः ॥ इति ।\\
शालङ्कायनः ।\\
प्रदाय वैणवीं यष्टि नूतनां सुदृढामृजुम् \textbar{}\\
श्लक्ष्णामनुल्वणग्रन्थि द्विजाय श्राद्धभोजिने ॥\\
विजयी जायते नित्यं न पश्यति पराजयम् ।\\
तावद्भवति कल्माषी सर्वे तरति कल्मषम् ॥ इति ।\\
अथ कमण्डल्वादि ।\\
अग्निपुराणे ।\\
श्राद्धे कमण्डलून् दद्यात् जलेनापूर्य यक्षतः ।\\
सर्वकामैः स सम्पूर्णश्चिरं स्वर्गेऽभिमोदते ॥\\
प्रभासखण्डे ।\\
चक्रवद्धं तु यो दद्यात् श्राद्धकाले कमण्डलुम्।\\
काञ्चनेन विमानेन किङ्किणीजालमालिना \textbar{}\textbar{}\\
वसते चिररात्राय सुवृष्णे मेरुमूर्धनि ।\\
वाराहपुराणे ।\\
यः काञ्चनमयं दिव्यं प्रयच्छति कमण्डलुम् ।\\
पितृभ्यः स चिरं भोगैर्मोदते काञ्चनालये ॥

{ श्राद्धोपकरणनिरूपणम् । १६७\\
यो ददाति पितॄणां हि राजतं वै कमण्डलुम्\\
सम्पन्नः सकलैर्भोगैः स राजा धार्मिको भवेत् ॥\\
कमण्डलुं ताम्रमयं श्राद्धेषु प्रददाति यः ।\\
स महत्या श्रिया युक्तः कुले महाते जायते ॥\\
काष्ठेन रचितं यस्तु नारिकेलमथापि वा ।\\
दद्यात् कमण्डलुं श्राद्धे स श्रीमानभिजायते ॥\\
चर्मणा निर्मितं वापि पात्रं नानाविधं तु यः ।\\
प्रतिपादयति श्राद्धे स मुखी जायते चिरम् ॥\\
यो मृत्तिकाविरचितान् श्राद्धेषु च घटान् नवान् ।\\
प्रयच्छति महामेधाः स दुःख नैव विन्दति ॥\\
स्कन्दपुराणे ।\\
यस्तडागान् तथाऽऽरामान् चापीकूपान् प्रपास्तथा ।\\
उत्सृजेत् पितृतृप्त्यर्थ ब्रह्मलोक स गच्छति ॥\\
मणिकानम्भसा पूर्णान् प्रदद्याद्वा गलन्तिकाम् ।\\
प्रदद्यात् करका वापि यदि वा करपत्रिकाः ॥\\
श्राद्धकाले यथाशक्ति सोऽक्षयं विन्दते सुखम् ।\\
करपत्रिका = जलपात्रविशेषः ॥\\
वायुपुराणे ।\\
दत्वा पवित्रं योगिभ्यो जन्तुवारणमभ्भस \textbar{}\\
श्राद्धे निष्कसहस्रस्य फलं प्राप्नोति मानवः ॥ इति ।\\
पूयते जलमनेनेति पवित्रम् ॥\\
अत्र छत्रम् \textbar{}\\
ब्रह्मवर्ते ।\\
छत्रं शतशलाकं यः सितवस्त्रोपशोभितम् ।\\
पितॄणां प्रयतो दद्यात् सोऽपि राजा भवेदिह ॥\\
मयूरपिच्छबहुभिर्निर्मितं रुचिराकृतिम् ।\\
छत्रं ददाति यस्तस्य विहारो नन्दने वने ॥\\
यः प्रदद्याल्लघुच्छत्रं रम्यमातपवारणम् ।\\
श्राद्धकाले स मनुजो न क्वचित् परितप्यते ॥\\
वायुपुराणे ।\\
( १ ) श्रेष्ठच्छत्रं च यो दद्यात् पुष्पमाल्यादिशोभितम् ।\\
प्रासादो ह्युत्तमो भूत्वा गच्छन्तमनुगच्छति ॥

% \begin{center}\rule{0.5\linewidth}{0.5pt}\end{center}

( १ ) पूर्णशय्यान्तु यो दद्यात् पुष्पमालाविभूषितामिति मुद्रित
वायुपुराणे पाठः ।

१६८ वीरमित्रोदयस्य श्राद्धप्रकाशे-

{ अथोपानतूपादुके ।\\
ब्रह्मपुराणे ।\\
उपानद्युगलं यस्तु श्राद्धकर्मणि धर्मवित् ।\\
एकैकस्मै द्विजाग्रन्थाय पित्रर्थे सम्प्रयच्छति ॥\\
पितॄणां तत् परे लोके विमानमुपतिष्ठते ।\\
दातापि स्वर्गमाप्नोति सुयुक्तं वडवारथैः ॥\\
सौरपुराणे ।\\
निर्माय सुदृढे दद्याददुर्गन्धेन चर्मणा ।\\
न न्यूने नातिरिक्ते च पादयोः सुसुखे मृदु \textbar{}\textbar{}\\
उपानहाँ ब्राह्मणेभ्यः पितॄणां सुखहेतवे ।\\
प्रीयन्ते पितरस्तस्य प्रीता यच्छन्ति वाञ्छ्रितम् ॥\\
नन्दिपुराणे ।\\
यः पादुके प्रदद्यात्तु पितृतृप्त्यर्थमादरात् ।\\
तस्य पुण्येषु लोकेषु भवेदप्रतिघा गति \textbar{}\textbar{}\\
अप्रतिघा=अप्रत्यूहा \textbar{}\\
हेमाद्रो चमत्कारखण्डे ।\\
धात्वादिनिर्मितं दद्यात् पितृभ्यः पादुकायुगम् ।\\
यस्तस्य देवलोकेषु गतिर्वैमानिकी भवेत् ॥\\
गजदन्तकृते यस्तु पादुके सम्प्रयच्छति ।\\
स वै चित्राणि यानानि लभते प्रेत्य चैव हि ।\\
यः पादुके प्रयच्छेत सारदारुमये शुभे ॥\\
पितृभ्य: सोपि मेधावी सुखमत्यन्तमश्नुते ।\\
अथ आसनानि ।\\
ब्रह्मपुराण \textbar{}\\
आसनानि च रम्याणि पितृभ्यो यः प्रयच्छति ।\\
स आस्ते सुचिरं काल त्रिदशैरभिपूजितः \textbar{}\textbar{}\\
देवीपुराणेः ।\\
पीठान्यतिमनोज्ञानि पितृणां प्रददाति यः ।\\
तस्य पीठेश्वरी नित्यं वरान् यच्छति वाञ्छितान् ॥\\
हेमाद्रौ चमत्कारखण्डे विशेषः,\\
चामीकरमयं श्राद्धेष्वासनं य प्रदापयेत् ।\\
तस्यासनं मेरुपीठे समीपे परमेष्ठिनः ॥

{ श्राद्धोपकरणनिरूपणम् । ५६९\\
यः पितॄणां सुघटितं दद्याद्राजतमासनम् ।\\
स स्वर्गे सुखमासीनः क्रीडते कालमक्षयम् ॥\\
येन ताम्रमयं दत्तमासनं पितृकर्मणि ।\\
स वै दिव्यासनारूढो न हि प्रच्यवते चिरम् ॥\\
प्रदद्यादासनं यस्तु निर्मितं सारदारुभिः ।\\
तस्य नाकं गतस्योच्चैः कथ्यते भव्यमासनम् ॥\\
विष्णुधर्मोत्तरे विशेषः ।\\
यस्तु भद्रासनं चारु पितॄणां प्रतिपादयेत् ।\\
सबै सिंहासनासीनः शोभते नरराडिव ।\\
यस्त्वासनं वस्त्रमयं हंसपिच्छैः सुसम्भृतम् ।\\
प्रयच्छति महीपालास्तमासनिमुपासते \textbar{}\textbar{}\\
तूलैः पूर्ण वरं वस्त्रमासनं यो निवेदयेत् ।\\
पितॄणामादरादेनं प्रत्यासीदन्ति सम्पदः \textbar{}\textbar{}\\
पितॄनुद्दिश्य योगिभ्यो दत्वा कुशमयीर्वृषीः ।\\
सर्वसङ्गविनिर्मुको विमुक्तात्मा स जायते ।।\\
वेत्रासनानि चित्राणि पितृभ्यः प्रतिपादयेत् ।\\
( १ ) नीरोगः पुरुषः श्रीमान् पुरुषः सम्प्रदापयेत् ॥\\
यस्त्वासनोपयोगार्थ प्रदद्यात् कम्बलान् नवान् ।\\
अष्टाङ्गयोगसंयुक्तसिद्धिं तस्योपजायते ॥\\
}{अष्टाङ्ग}{योगसंयुक्तस्य}{ या सिद्धि: सेत्यर्थः ।\\
यस्तृणैर्मृदुभिः ऋक्ष्णैर्निर्माय दृढमासनम् ।\\
दद्यात् श्राद्धेषु तस्याशु स्थिराः स्युः सर्वतः श्रियः ॥\\
अथ शय्यादि ।\\
कूर्मपुराणे ।\\
सारदारुमयीं शय्यां लक्ष्णामास्तरणान्विताम् ।\\
दत्वा सुमनसां लोके दिव्यान् भोगानवाप्नुयात् ।\\
ब्रह्माण्डपुराणे-\\
शय्यामास्तरणोपेतामुत्तरच्छदसंयुताम् ।\\
उपधानेन संयुक्तां पितॄनुद्दिश्य यो ददेत् ॥

% \begin{center}\rule{0.5\linewidth}{0.5pt}\end{center}

{(१) नीरोमः सुभगः श्रीमान् पुरुषः सम्प्रजायते । इति पाठो\\
०मि० २२\\


१७० वीरमित्रोदयस्य श्राद्धप्रकाशे-

{\\
मोदन्ते पितरस्तस्य सुखिनः शाश्वतीः समाः ।\\
दातापि स्वर्गमासाद्य विमान दिव्यमास्थितः ॥\\
सेव्यते सुरनारीभिर्गीयमानश्च किन्नरैः ।\\
विष्णुधर्मोत्तरे ।\\
आन्दोलकं सास्तरणं सोपधानप्रसाधनम् ।\\
धातुजाभिः सुरस्याभिः शृङ्खलाभिश्च संयुतम् ॥\\
मत्तवारणशोभाढ्यं प्रथित मृदुभिः पटैः ।\\
ददाति पितृकार्येषु यो हि श्रद्धापरायणः \textbar{}\textbar{}\\
गन्धर्वाप्सरसां लोके गयमानो निरन्तरम् ।\\
स भुङ्क्ते विविधान् भोगान् त्रिदशैरपि दुर्लभान् ॥\\
चमस्कारखण्डे ।\\
श्राद्धे शय्यां प्रवच्छेद्यः सूर्यलोके स राजते ।\\
गजदन्तमयं दिव्य श्राद्धे दत्वा तु मञ्चकम् ।\\
गत्वा चान्द्रमसं लोकं शरदामयुतं वसेत् ॥\\
पत्रमयैः पट्टेग्रंथितां च ददाति यः ।\\
शय्यां पितृभ्यो मेधावी देवीलाकें स गच्छतिः ॥\\
कर्पास सूत्रजैः पट्टेः सुदृढां यः प्रयच्छति ।\\
चन्द्रस्य भवने सोऽपि कामान् भुङ्गे यथेप्सितान् ॥\\
कृतां शणमये पट्टे: सूत्र जैर्वा ऽविजैरपि ॥\\
दत्वा जन्मान्तरे जातः स्त्रियो विन्दति सुन्दरीः \textbar{}\\
अविजैः = अविलोमकृतपट्टेः ।\\
हंसपिच्छमयीं तूष्णी पितृभ्यः प्रददाति यः ।\\
गन्धर्वाप्सरसां लोके मोदते स यथा सुखम् ।\\
कर्पासनिर्मितां बूलीं दत्वा ऋद्धि सुशोभनाम् \textbar{}\textbar{}\\
उपधानेन संयुक्तां लक्ष्मीवान् जायते नरः ।\\
हंसपिच्छमय रम्यमुपधानं ददाति यः \textbar{}\textbar{}\\
कीर्तिमान् जायते नित्य सुखानामपि भाजनम् ।\\
क्षौमं वा पट्टसूत्रं वा यो दद्यादुत्तरच्छदम् ॥\\
लावण्येन सदा युक्तो जायतेऽसौ जनप्रियः ।\\
प्रयच्छेदुत्तरपट सूक्ष्मकार्पाससुत्रजम् ॥\\
तस्याऽऽयुर्विपुलं लोके प्रयच्छति पितामहः ।

{ }{श्राद्धोपकरणनिरूपणम् । १७२}{\\
मृदुचर्ममयीं दद्याद्यो नरः पट्टगण्डकाम् \textbar{}\textbar{}\\
सोऽपि श्रिया समायुक्तो नीरोगो जायते भुवि ।\\
विचित्रैश्चर्मभिर्युक्तं रचितं मृदुभिस्तृणैः ॥\\
श्राद्धकाले तु योगिभ्यः स दुःखैर्नाभिभूयते ।\\
}{ अथ चामरव्यजनदर्पण केशप्रसाधनामि ।}{\\
सौरपुराणे ।\\
चामरं तालवृन्तं च श्वेतच्छत्रं च दर्पणम् ।\\
दत्वा पितृणामेतानि भूमिपालो भवेदिह ॥\\
बाराहपुराणे ।\\
चामरमधिकृत्य \textbar{}\\
पितृभ्यश्चामरं दत्वा स्वर्गे स्त्रीभिस्तु वीज्यते ।\\
तदेव कृष्णवर्णे तु दत्वा भूमिपतिर्भवेत् ॥\\
मयूरपिच्छनिर्माणं हेमदण्ड तु चामरम् ।\\
प्रतिपाद्य पितृभ्यस्तु राजराजो भवेदिह ॥\\
व्यामदी घैरतिश्वेतैरश्ववालधिसम्भवैः ।\\
निर्मित चामरं श्राद्धे दत्वा माण्डलिको भवेत् ॥\\
कृष्णाश्ववालरचित चामरं यस्तु यच्छति ।\\
सोऽपि पुण्येन तेनेह धनी भवति धर्मवान् \textbar{}\textbar{}\\
हेमाद्रौ त्रमस्कारखण्डे । व्यजनं ये प्रयच्छन्तीत्युपक्रम्य\\
रंचितं वालकेनाथ यदुशीरेण निर्मितम् ।\\
प्रदाय व्यजनं श्राद्धे मनस्तापं न विन्दति ॥\\
पट्टसूत्रेण रचितं वस्त्रैरन्यैरथापि वा ।\\
प्रयच्छेत्तालवृन्तं यः स भूपालो न संशयः ।\\
तालैर्द लैर्विरचितं कृतं भूर्जत्वचापि च ॥\\
-प्रदाय व्यजनं श्राद्धे महदारोग्यमाप्नुयात् ।\\
'कृतं च विदलच्छेदैः सुसूक्ष्मैश्चैव गुम्फितम् \textbar{}\textbar{}\\
दत्वा पितृभ्यो व्यजनमनन्तं सुखमश्नुते ।\\
भविष्यपुराणे-\\
दर्पणं कलधौतेन निर्मलेन सुनिर्मितम् \textbar{}\\
प्रतिपाद्य पितृभ्यो वै लोकं चान्द्रमसं व्रजेत् ॥\\
विमलेनाथ कांस्येन पञ्चाशत्पलिकेन तु ।

{१७१ वीरमित्रोदयस्य श्राद्धप्रकाशे-\\
~\\
कल्पित दर्पण दत्वा तेजस्वी जायते ध्रुवम् ॥\\
त्रिंशत्पलेन कांस्येन कृतमादर्शमण्डलम् ।\\
पञ्चविंशति पञ्चापि दत्वा वै कान्तिमान् भवेत् ॥\\
यो दर्पण विरचितं कांस्यस्य दशभिः पलैः ।\\
प्रतिपादयते सोऽपि लभते चक्षुरुत्तमम् ॥\\
स्कन्दपुराणे-\\
केशप्रसाधनं दत्वा करिदन्ताविनिर्मितम् ।\\
पितृकर्मणि धर्मात्मा सोऽश्विनोर्लोकमश्नुते ॥\\
तालस्य मालिकेरस्य वेणोर्वेत्रस्य वा पुनः ।\\
शलाकाभिर्विरचितं दत्वा केशप्रसाधनम् \textbar{}\textbar{}\\
सुभगश्च सुवेषश्च निश्चितं जायते नरः ।\\
दारुणा निर्मितं दत्वा केशसंस्कारसाधनम् ॥\\
प्राप्नोति सुन्दरान् केशान् दीर्घमायुश्च विन्दति ।\\
वराहरोमरचितां सुवैषद्यस्य कारिणीम् \textbar{}\textbar{}\\
पितृभ्यः कुञ्चिकां दत्वा पुरुषः सुभगो भवेत् ।\\
मारदीयपुराणे ।\\
यस्तु भोजनपात्राणि पितृभ्यः प्रतिपादयेत् ।\\
सौवर्णरजतादीनि तथा कांस्यमयान्यपि ॥\\
स पुमान् पात्रतामेति सर्वासामपि सम्पदाम् ।\\
घात्वादिनिर्मिता भुक्तिपात्राधारास्त्रिपादिकाः ॥\\
उत्सृजन् स्वपितृप्रत्यै प्रापयेत् ब्राह्मणालये ।\\
स ब्रातीनामशेषाणामाधारत्वं प्रपद्यते \textbar{}\textbar{}\\
पतत्प्रहं धातुमयं ष्ठीवनाऽऽचमनादिषु ।\\
उपयुक्तं श्राद्धकाले पितृभ्यः परिकल्पयेत् \textbar{}\textbar{}\\
द्विजस्य भवनस्थाने श्रियमाप्नोति पुष्कलाम \textbar{}\\
पात्राधाराणां दानमात्रमेव न तु तेषु पात्राणि स्थाप्यानि । तेषां\\
साझाद् भूमौ स्थापनविधानात् इति हेमाद्रिः ।\\
बहिपुराणे ।\\
ताम्बूलं यो ददातीह मुनिकर्पूरसंयुतम् ।\\
ऐश्वर्ये सोऽति विपुलं परत्रेह च विन्दति ॥\\
यचूर्णपर्ण पूगादिस्थापनार्थानि कृत्स्नशः ।\\
दद्यात् ताम्बूलपात्राणि पूगाद्यैः पूरितानि च ॥

{ श्राद्धोपकरणनिरूपणम् । १७२}{\\
तथा कर्पूरभाण्डञ्च कर्पूरेणाभिपूरितम् ।\\
दिव्यं वर्षसहस्रं हि भुङ्गे भोगान् दिवि स्थितः ।\\
वायुपुराणे । पात्रदानानि प्रक्रम्य -\/-\\
लवणेन तु पूर्णानि श्राद्धे पात्राणि दापयेत् ।\\
रसाः समुपतिष्ठन्ति भक्ष्यं सौभाग्यमेव च ॥\\
तिलपूर्णानि यो दद्यात् पात्राणीह द्विजन्मनाम् ।\\
तिले तिले निष्कशतं स ददाति न सशयः ॥

{ब्रह्मवैवर्ते ।}{\\
सुरभिद्रव्य तैलैस्तु गन्धवद्भिस्तथैव च ।\\
पूरयित्वा सुपात्राणि श्राद्धे सत्कृत्य दापयेत् \textbar{}\textbar{}\\
गन्धवहा महानद्यः सुखानि विविधानि च ।\\
दातारमुपतिष्ठन्ति युवत्यश्च पतिव्रताः ॥\\
ब्रह्माण्डपुराणे ।\\
वृक्षैर्मनेाहरफलैलताभिश्च समाकुलान् ।\\
आरामान् मे प्रयच्छन्ति पितृभ्यो जलपूर्वकम् ॥\\
ते चक्रवर्तिनो भूत्वा प्रशासति वसुन्धराम् ।\\
ये पुष्पवाटिकां रम्यां वृक्षैः कतिपयैर्युताम् ॥\\
प्रवच्छन्ति पितृभ्यस्ते भूमिपाला न संशयः ।\\
येप्येकं फलितं वृक्षं लतामण्डपमेव वा ॥\\
प्रयच्छति पितृप्रीत्यै ब्राह्मणानां महात्मनाम् ।\\
बहुपुत्रा बहुधनास्ते दृश्यन्ते महीतले ॥\\
ये तु क्रीत्वा तु लब्ध्वा वा फलान्यादाय भक्तितः ।\\
पितॄणां सम्प्रयच्छन्ति घनिनस्तेऽपि निश्चितम् । इति ।\\
अथ हिरण्यालङ्कारादि ।\\
तथा च नन्दिपुराणे । हिरण्यदाममुपक्रम्य -\\
अतः सम्पूज्य गन्धाद्यैर्वस्त्राद्यैरभिभूष्य च ।\\
हिरण्यं सम्प्रदातव्यमिदमस्मै स्वधेति हि ॥\\
दक्षिणादौ हि रजतं पित्र्ये कर्मणि शस्यते ।\\
अलङ्काराः प्रदातव्या यथाशक्ति हिरण्मयाः ॥\\
केयूरहारकटकमुद्रिकाकुण्डलाक्ष्यः ।\\
स्त्रीश्राद्धेषु प्रदेयाः स्युरलङ्कारास्तु योषिताम् ॥

{१७४ वीरमित्रोदयस्य श्राद्धप्रकाशे-\\
मञ्जीरमेखलादामकर्णिकाकङ्कणादयः ।\\
हारमाणिक्यवैडूर्यमुक्तागारुत्मतादिभिः \textbar{}\textbar{}\\
रखैर्विरचिताः स्वच्चैरलङ्कारा मनोहराः ।\\
पितृभ्यः सम्प्रदातव्या निजवित्तानुसारतः ॥\\
यानानि शिविकागन्त्री तुरङ्गादीनि यत्नतः ।\\
श्राद्धे देयानि विदुषा स्वसामर्थ्यानुसारतः ॥\\
अन्नानि च विचित्राणि स्वादूनि सतिलानि च ।\\
दातव्यानि यथाकाम पितृभ्यो ददता सदा ॥\\
एवं यः कुरुते श्राद्धं श्रद्धया धार्मिकोत्तमः ।\\
प्रक्षीणाशेषपापस्य तस्य संशुद्धचेतसः \textbar{}\textbar{}\\
विच्छिन्नक्लेश जालस्य मुक्तिरेवामलं फलम् ।\\
आदित्यपुराणे ।\\
पितॄन् सम्पूज्य वासाद्य हिरण्य प्रददाति यः ।-\\
तुलादानसम पुण्यं लभते नात्र संशयः ॥\\
रजतस्य प्रदानेन गोसहस्रफलं लभेत् ।\\
दक्षिणार्थं पृथक् देयं स्वर्ण रूप्यमथापि वा ।\\
तेनास्य वर्द्धते लक्ष्मीरायुर्दीर्घ च विन्दति ॥

{\\
अलङ्कारविशेषदाने फलविशेषः ।\\
स्कन्दपुराणे ।

{ मुर्वालङ्करणं दत्वा श्राद्धे बहुधनोचितम् ।\\
मूर्द्धाभिषितां भवति पृथिव्यां नात्र संशयः ॥\\
कर्णभूषणदानेन निश्चितं स्याद् बहुश्रुतः ।\\
कटकालङ्कृतिदानात्तु वाग्मी स्यान्मधुरस्वरः ॥\\
मेधावी जायते विद्वान् दन्तैर्हृदयभूषणैः ।\\
जायते बाहुभूषाभिः प्रदत्ताभिर्महाबलः ॥\\
हस्तालङ्करणं दत्वा दाता भवति विश्रुतः ।\\
स विश्ववन्द्यो भवति यो दद्यात् पादभूषणम् ॥\\
स्वर्गच्युतानि ह्येतानि फलान्युकानि सुरिभिः ।\\
पितृभूषणदानस्य स्वस्वमुख्यतमं फलम् ॥\\
रजजैर्भूषणैर्दन्तैर्न मुक्तिरपि दुर्लभा ।

{ }{ }{श्राद्धोपकरणनिरूपणम्}{ । }{ २७६}{\\
ब्रह्मपुराणे ।\\
यद्यद् देयं विशिष्टं च तत्तद् देयं पितॄन् प्रति ।\\
तत्राप्यनं जलं वस्त्रं भूषणानि विशेषतः \textbar{}\textbar{}\\
यानान्यपि प्रदेयानि पितॄणां परितृप्तये \textbar{}\\
यानदाने विष्णुधर्मोत्तरे विशेषः ।\\
शिविकां यः प्रयच्छेत्तु सर्वोपकरणैर्युताम् ।\\
दोलावाहनकर्मिभ्यो वृत्ति संवत्सरोचिताम् ॥\\
वर्षपर्याप्तमशनं कुटुम्बार्थे द्विजस्य तु ।\\
छत्रप्रदानमध्येव कर्तव्य पितृकर्माणि \textbar{}\textbar{}\\
यस्तु चित्रगतिं दद्यात् तुरङ्गं लक्षणान्वितम् ।\\
श्राद्धेषु तस्य देहान्ते सूर्यलोकेऽक्षयस्थितिः ॥\\
हेमाद्रौ चमत्कारखण्डे ।\\
दद्यान्मतङ्गजं यस्तु युवानं चारुलक्षणम् ।\\
स लोके लोकपालानामेकैकमयुतं वसेत् ॥\\
तेजस्विनं चारुगतिं लक्षण्यं यस्तुरङ्गमम् ।\\
दद्यात् पितृभ्यो विजयस्तस्याप्रतिहतो भवेत् ॥\\
रथं ददाति यो रम्यं युग्मैर्युक्तं तुरङ्गमैः ।\\
उक्षभिर्वा महाकायैस्तरुणैः सर्वलक्षणैः ॥\\
महाहवेषु कुत्रापि न तस्य स्यात् पराजयः ।\\
गन्त्रीं वा शकटं वापि लाहेचक्राक्षकूवराम् ॥\\
दत्वा पितॄणामाप्नोति घनर्द्धिमतिभूयसीम् ।

{\\
अथ गोमहिष्यादिदानम् ।\\
मत्स्यपुराणे ।\\
श्राद्धे गावो महिष्यश्च बलीबर्दास्तथैव च ।\\
प्रदातव्या महोष्ट्राच यश्चान्यद् वस्तु शोभनम् ॥\\
हेमाद्रो बृहद्विष्णुपुराणे ।\\
तरुणीं सुखसन्दोह्यां जीवद्वत्सां पयस्विनीम् ।\\
ददाति धेनु विप्रेभ्यस्तृप्तिमुद्दिश्य पैतृकीम् ॥\\
यस्तस्य सी दिविस्थस्य संर्वकामदुधा भवेत् ।\\
सुशीलां लक्षणवर्ती सवत्सां बहुदोहनाम् \textbar{}\textbar{}\\


{१७६ वीरमित्रोदयस्य श्राद्धप्रकाशे-}{\\
दत्वा पितृभ्यः कपिलां घण्टाचामरभूषणाम् ।\\
मुञ्चवल्काद्यथैर्षाका पृथक् भवति निर्मला \textbar{}\textbar{}\\
एवं स सर्वपापेभ्यः पृथक् भवति निर्मलः ।\\
अयुतानां शतं साम ब्रह्मलोके महीयते ॥\\
ददाति यस्तु महिषीमव्यङ्गाङ्गीमकोपनाम् ।\\
भूरिक्षीरां गुणवती सापत्यां बहुसर्पिषम् ॥\\
क्षीरस्य सर्पिषो दध्नः परिपूर्णा दिविहृदाः \textbar{}\\
पितॄनस्योपतिष्ठन्ति यावदाभूतसंपुवम् ॥\\
दातापि स्वर्गमाप्नोति वर्षाणामयुतानि षट् ।\\
यस्तु धुर्यान् बलीबर्दान् पृष्ठे भारवहानपि ॥\\
अविद्धनासिकान् दद्यात् अक्षुण्णवृषणांस्तथा ।\\
वृषरूपः स्वयं धर्मः तस्य साक्षात् प्रसीदति ॥\\
क्रमेलकान् भारवहान बहुयोजनगामिनः ।\\
येऽलङ्कृत्य प्रयच्छति राजानस्ते न संशयः ॥\\
अजाश्चैवावयश्चैव महिषा भारवाहनाः ।\\
पितृभ्यः सम्प्रदातव्याः सर्वपापक्षयार्थिना ॥\\
}{गवां वर्णविशेषात् फलविशेषो नारदीये ।}{\\
दत्वा पितृभ्यः श्वेतां गां श्वेतद्वीपे महीयते ।\\
प्रदाय धेनुं कृष्णाङ्गीं यमलोकं न पश्यति ॥\\
पीतवर्णा तु गां दत्वा न शोचति कृताकृते ।\\
प्रदाय रोहिणीं धेनुं सर्वे तरति दुष्कृतम् ॥\\
रोहिणीं = रक्ताम् ।\\
नीलां च सुरभिं दवा वंशच्छेदं न विन्दति ।\\
अन्येन येन येनापि धेनुं वर्णेन लक्षिताम् ।\\
दत्त्वा पितृभ्यो जयति लोके सुखमनुत्तमम् ॥\\
स्कन्दपुराणे ।\\
उष्ट्रीं वेगवहां यस्तु दद्यादुष्ट्रानथापि वा ।\\
तस्ये स्वर्गे प्रयातस्य गतिर्नैव विहन्यते ।\\
पशूनजाविकांश्चैव यस्तु श्राद्धे प्रयच्छति ।\\
प्रजया पंशुभिश्चैष गृहं सुपरिपूर्यते ॥

 श्राद्धोपकरणनिरूपणम् । १७७

{\\
अथ भूगृहपुस्तकाभयादिदानम् ।\\
ब्रह्मपुराणे ।\\
यथाशक्त्या प्रदातव्या भूमिः श्राद्धे विपश्चिता ।\\
पितॄणां सम्पदे सा हि सर्वकामप्रसूर्यतः \textbar{}\textbar{}\\
गारुडे विशेषः ।\\
गृहाणि च विचित्राणि पितृभ्यो यः प्रयच्छति ।\\
जाम्बूनदमयं दिव्यं यथाकामगमं शुभम् ॥\\
सर्वसम्पत्समोपेतं विमानं सोऽधिरोहति ।\\
ब्रह्माण्डपुराणे ।\\
ग्रामं वा खर्वटं चापि पितृभ्यः प्रददाति यः ।\\
शक्रस्य भवनं गत्वा यावदिन्द्रं स वर्तते ॥\\
श्राद्धे ददाति यः क्षेत्रं दशलाङ्गलसम्मितम् ।\\
पञ्चलाङ्गलिकं वापि यद्वा गोचर्ममात्रकम् ॥\\
अलाभे द्विहलं वापि हलमात्रमथापि वा ।\\
लाङ्गलैः स बलीबदैर्योक्त्रतोत्रादिसंयुतैः ॥\\
अभ्यैश्चैवोपकरणैरज्जुफालादिभिर्युतम् ।\\
वाजपेयस्य यज्ञस्य स फलं प्राप्नुयान्नरः ॥\\
काले स्वर्गात् परिभ्रष्टो भूपतिर्धार्मिको भवेत् ।\\
खर्वटो=नृहीनो ग्रामः \textbar{} दशलाङ्गलसम्मितम्}{
}{=दशभिर्लाङ्गलैर्यावत्}{\\
ऋष्टुं शक्यते तावत् क्षेत्रमिति । एवमप्रेपि ।\\
गोचर्मलक्षणं तु ।\\
विंशद्धस्तेन दण्डेन त्रिंशद्दण्डानि वर्तनम् ।\\
दश तान्येव गोचर्ममानमाह प्रजापतिः \textbar{}\textbar{}\\
इति स्मृत्यन्तरोकम् ।\\
ब्रह्मपुराण एव ।\\
शालीनामथवेक्षणां यवगोधूमयोरपि ।\\
माषमुद्गतिलानां च क्षेत्रमुत्पत्तिहेतुमत् \textbar{}\textbar{}\\
पितॄणां यक्षतो दत्वा विष्णोः सालोक्यमाप्नुयात् ।\\
पुनर्मानुषमायातो धनधान्यसमन्वितः ॥\\
तेजसा यशसा युक्तो विद्वान् वाग्मी च जायते ।\\
वी० मि २३

१७८ वीरमित्रोदयस्य श्राद्धप्रकाशे-

{\\
गृहं पक्केष्टकचितं सुधाभिर्धवलीकृतम् ।\\
मत्तवारणशोभादयं गवाक्षद्वाराभित्तिमत् ॥\\
अनेक भूमिसंयुक्तमेकभूमिकमेव च ।\\
पितृभ्यो यो ददातीह स याति ब्रघ्नविष्टपम् ॥\\
दत्वा गृहं पितृभ्यस्तु तृणच्छन्नमथापि वा ।\\
लभतेऽग्याणि वेश्मानि स्त्रीमान्ति धनवन्ति च ॥\\
पुस्तकानि सुवाच्यानि सच्छास्त्राणां ददाति यः ।\\
ब्राह्मणानां कुले यज्वा जायतेऽसौ बहुश्रुतः ॥\\
हेमाद्रौ चमत्कारखण्डे विशेषः ।\\
पुस्तकानि पितृभ्यस्तु वेदाङ्गानां ददाति यः ।\\
स श्रोत्रियान्वये भूत्वा जायते वेदवित्तमः ॥\\
दत्त्वा व्याकरण तु स्यात् शश्वत् शब्दविदां वरः ।\\
मीमांसायाः प्रदानेन सोमयाजी भवेन्नरः ॥\\
प्रदाय न्यायशास्त्राणि भवेद् विद्वत्तमः पुमान् ।\\
पुराणदाता भक्तः स्यात् पुराणपुरुषे हरौ ॥\\
मन्वादिधर्मशास्त्राणां दानाद् भवति धार्मिकः ।\\
कलाशास्त्रप्रदानेन कलासु कुशलो भवेत् ॥\\
यः श्राद्धदिवसे विद्वान् प्राणिनामभयं दिशेत् ।\\
भयं न तस्य किञ्चित् स्यात् इह लोके परत्र च ॥ इति ।\\
सौरपुराणे ।\\
यद्यस्य भयमुत्पन्नं स्वतो वा परतोऽपि वा ।\\
श्राद्धकर्मणि सम्प्राप्ते तत् तस्यापनयेत् सुधीः ॥\\
राजतश्चोरतो वापि व्यालाच्च श्वापदादपि ।\\
सञ्जातमुद्धरेद्भीत पितृकर्माणि शक्तितः ॥\\
एकतः क्रतवः सर्वे इत्यादि अभयदानप्रशंसामभिधायाह स एव ।\\
यथा ह्यभयदानेन तुष्यन्ति प्रपितामहाः ।\\
न तथा वस्त्रपानान्नरक्षालङ्कारभोजनैः ॥\\
एतस्माद्भयं देयं श्राद्धकाले विजानता ॥ इति ।\\
वामनपुराणे ।\\
बन्दकितास्तु ये केचित् स्वयं वा यदि वा परैः ।\\
येन केनाप्युपायेन यस्तान् मोचयते नरः \textbar{}\textbar{}}

  श्रद्धोपकरणनिरूपणम् । १७९

{\\
पितरस्तस्य गच्छन्ति शाश्वतं पदमव्ययम् ।\\
अञ्जनाभ्यञ्जनादयो ब्रह्मवैवर्ते ।\\
अञ्जनाभ्यञ्जनं चैव पितृभ्यां प्रतिपादयेत् ।\\
सुक्ष्मं चाभिनवं सूक्ष्मं पिण्डानामुपरि भ्यसेत् ॥\\
तत्र द्रव्यनियमो ब्रह्मवैवर्ते प्रभासखण्ड वायुपुराणेषु ।\\
श्रेष्ठमाहुस्त्रैककुदमञ्जनं नित्यमेव च ।\\
दीपात्}{ }{कृष्णतिलोद्भूततैलजादू यन्न धारितम् ।}{\\
त्रिककुदि पर्वतविशेषे भवं त्रैककुद-श्रोताञ्जनमिति यावत्
\textbar{}\textbar{}\\
ब्रह्माण्डपुराणे ।\\
पेषयित्वाञ्जनं सम्यग् वेद्या उत्तरतो बुधः ।\\
गृहीतदर्भपिज्जूलैस्त्रिभिः कुर्याद्यथाविधि ॥\\
एकं पवित्रं हस्ते स्यात् पितॄणां तु पृथक् पृथक् ।\\
तैलं पात्रेण दातव्यं पिण्डेभ्योऽभ्यञ्जन हि तत् ॥\\
ब्रह्मवैवर्ते ।\\
तिलतैलेन दातव्यं तथैवाभ्यञ्जनं बुधैः ।\\
असावङ्क्ष्व तथाभ्यङ्क्ष्वेत्यञ्जनादीनि दापयेत् \textbar{}\textbar{}\\
असावेतत्त इत्येवं सूत्रं चापि नियोजयेत् ।\\
इत्येवमञ्जनं दत्वा चक्षुष्मान् जायतेनरः ॥\\
अभ्यञ्जनप्रदानेन लभते रूपमुत्तमम् ।\\
लभेद् वस्त्राण्यनन्तानि पिण्डे सुत्रप्रदानतः \textbar{}\textbar{}\\
वायुपुराणे ।\\
अञ्जनाभ्यञ्जनं चैव पिण्डनिर्वपणं तथा ।\\
अश्वमेधफलेनैव संमितं मन्त्रपूर्वकम् \textbar{}\textbar{}\\
क्रियाः सर्वाः यथोद्दिष्टाः प्रयतेन समाचरेत्\\
ब्रह्मपुराणे ।\\
क्षौमं सूत्रं नवं दद्याच्छुणकर्पासजं तथा ।\\
पत्रोर्णं पट्टसूत्रं च कौशेय च विवर्जयेत् ॥\\
पत्रर्णम् = श्रौतकौशेयम् ।\\
वर्जयेत्तु दशाः प्राशो यद्यप्यहतवस्त्रजाः ।\\
न प्रीणन्ति तथैवाभिर्दातुश्चाप्यफलं भवेत् \textbar{}\textbar{}\\
आपस्तम्बेन तु दशा देयेत्युक्तम् ।\\


{१८० वीरमित्रोदयस्य श्राद्धप्रकाशे-\\
वाससो दशां छित्वापि निदधान, ऊर्णास्तुकां वा पूर्वे वयसि,\\
उत्तरे स्वं लोमेति । ऊर्णास्तुका=मेषलोमानि ।\\
कात्यायनः ।\\
एतद्व इत्यपास्यति सुत्राणि प्रतिपिण्डं ऊर्णा दशा वा, वयस्यु\\
तरे यजमानलोमानि वेति । उत्तरं वयः, पञ्चाशदुत्तरम् \textbar{} "पञ्चा\\
शत ऊर्द्ध उरोलोम यजमानस्य" इति शाट्यायनिवचनात् ।\\
ब्रह्मपुराणे ।\\
दद्यात् क्रमेण वासांसि दशां वां श्वेतवस्त्रजाम् ॥ इति ।\\
प्रकीर्णकदार्न प्रभासखण्डे ।\\
लोके श्रेष्ठतमं सर्वमात्मनश्चापि यत् प्रियम् ।\\
तत्तत् पितॄणां दातव्यं तदेवाक्षयमिच्छता ॥ इति ।\\
स्कन्दपुराणे ।\\
अलङ्कारान् बहुविधान् काञ्चनेन विनिर्मितान् ।\\
इत्यादिपूर्वोक्त सर्वदानान्यनुक्रस्य,\\
एतान् दद्यात्तु यः श्राद्धे पदार्थान् भोगसाधनान् ।\\
न तस्य दुर्लभं किञ्चिदिह लोके परत्र च ॥ इति ।\\
विष्णुधर्मोत्तरे पितृगाथाः ।\\
अपि स्यात् स कुलेऽस्माकं कश्चित् पुरुषसत्तमः ।\\
दद्यात् कृष्णाजिनं यो वः स्वर्णशृङ्गविधानतः ॥\\
अपि वा स्यात् कुलेऽस्माकं कश्चित् पुरुषसत्तमः ।\\
प्रसूयमानां यो दद्यात् धेनुं ब्राह्मणपुङ्गवे \textbar{}\textbar{}\\
बौधायनः ।\\
वापीकूपतडागानि वृक्षानाराममेव च ।\\
शालाक्षुक्षेत्र केदाराः समृद्धाः पुष्पवाटिकाः ॥\\
भाद्धेषु दत्त्वा प्रयतः पितॄन् आत्मानमेवच ।\\
समुद्धरत्यक्षौ दुःखात् यावदाभूतसंयुतात् \textbar{}\textbar{}\\
वायुपुराणे-\\
धेनुं भाद्धेषु वो दद्यात् गृष्टं कुम्भीपदानाम् ।\\
गावस्तमुपतिष्ठन्ति गवां पुष्टिश्च जायते ॥\\
सृष्टि = प्रथमप्रसूता गौः । अत्रोपकरणेष्वेतेषु यानि श्राद्धाङ्गार्चना-\\
ङ्गभूतानि तानि नामगोत्रा पुञ्चारणपूर्वकं पितृभ्यो देयानि गन्धादिष.\\
.

{ }{ श्रद्धापकरणनिरूपणम् । १८१}{\\
त् । यनि तु गोभूहिरण्यादीनि अर्चनानुपयोगीनि तानि श्राद्धकाल एव\\
दानप्रोक्तप्रकारेण देयानि तेषां उपरागादिकालान्तरवच्छ्राद्ध का.\\
लस्य तत्कालत्वेन विधानाच्छ्राद्धाङ्गत्वाभावात् । अत एव फलश्रव-\\
णमप्युपपद्यते ।\\
तथा वदिपुराणे ।\\
शक्त्याथ दक्षिणा देवा श्राद्धकर्मणि शक्तितः ।\\
ग्रामक्षेत्राण्यथारामा विचित्रा: पुष्पवाटिकाः ॥\\
बहुभौमानि रम्याणि गृहाणि शयनानि च ।\\
सुवर्णरलवासांसि रजतं भूषणानि च ॥\\
अनडुद्दो महिष्यश्च विविधान्यासनानि च ।\\
पादुका दासदासीच छत्रव्यजनचामरम् \textbar{}\textbar{}\\
लाङ्गलान् शकटान् गन्धान् गृहोपकरणानि च ।\\
येन येनोपयोगोऽस्ति विप्राणामात्मनस्तथा ॥\\
तत्तत् प्रदेयं श्राद्धेषु दक्षिणार्थं हितैषिणा ।\\
यथा हि गुणद् द्रव्यं तव भूरि यथा वथा ॥\\
जायते फलभूयस्त्वं श्राद्धकर्तुस्तथा तथा ॥ इति ।\\
ब्रह्मपुराणे -\\
यद्यदिष्टतमं लोके यच्चास्य दयितं गृहे ।\\
दक्षिणार्थे तु तद्देयं तस्य तस्याक्षयार्थिना \textbar{}\textbar{}\\
मत्स्यपुराणेऽपि ।\\
सतिलं नामगोत्रेण दद्याव शक्त्याथ दक्षिणाम् ।\\
गोभृहिरण्यवासांसि यानानि शयनानि च ॥\\
दद्याद्यदिष्टं विप्राणामात्मनः पितुरेव च ।\\
वित्तशाठवेन रहितः पितृभ्यः प्रीतिमाचरन् ॥ इति ।\\
ब्रह्माण्डपुराणे ।\\
सौवर्णरूप्यपात्राणि मनोज्ञानि शुभानि च ।\\
हस्त्यश्वरथयावानि समृद्धानि गृहाणि च ॥\\
उपानत्पादुकाच्छत्रचामराण्यार्जनानि च ।\\
यज्ञेषु दक्षिणा पुण्या सेति संचिन्तयेत् तदा ॥\\
या यज्ञेषु दक्षिणा सेयं दीयमानदक्षिणेति बुद्धिः कार्या ।

 १८२ वीरमित्रोदयस्य श्राद्धप्रकाशे-

{सौरपुराणे ।\\
वहीभिर्दक्षिणाभिस्तु यःश्राद्धे प्रीणयेद् द्विजान् ।\\
स पितॄणां प्रसादेन याति स्वर्गमनन्तकम् \textbar{}\textbar{}\\
अशक्तस्तु यथाशक्त्या श्राद्धे दद्यात्तु दक्षिणाम् ।\\
अदक्षिणं तु यत् श्राद्धं ह्रियते तद्धि राक्षसैः ॥\\
यज्ञोपवीतमथवा ह्यतिदारिद्र्यपीडितः ।\\
प्रदद्याद्दक्षिणार्थं वै तेन स्यात् कर्म सहुणम् \textbar{}\textbar{}\\
इति श्रीमत्सकलसामन्तकचक्रचूडामणिमरीचिमञ्जरीनीराजित.\\
चरणकमलश्रीमन्महाराजाधिराजप्रतापरुद्रतनूजश्रीमन्महाराज.\\
मधुकरसाहसनुचतुरुदधिवलयवसुन्धराहृदयपुरण्डरीक.\\
विकासदिनकर श्रीमन्महाराजाधिराजश्रीवीरसिंहदे-\\
वोर्जितश्रीहस पण्डितात्मज परशुराममिअसूनुसकल\\
विद्यापारावारपारीणधुरीण जगद्दारिद्र्यमहागज\\
पारीन्द्रविद्वज्जनजीवातुश्री मन्मित्रमिश्र.\\
कृते वीरमित्रोदयाभिधनिबन्धे श्राद्ध\\
प्रकाशे उपकरणानि ॥

{-\/-\/-\/-\/-\/-*-\/-\/-\/-\/-\/-\/-\\
अथ श्राइ दिनकृत्यम् ।\\
तत्राहिकोक्तविधिना दन्तधावनरहितं प्रातःसन्ध्यान्तं कर्म कृत्वा\\
स्वशास्त्रोक्तविधानेन श्रौतस्मार्तहोमं च कृत्वा पूर्वोकं श्राद्धदेशं
सं.\\
स्कुर्यात् । तथाच,\\
ब्रह्माण्डपुराणे ।\\
श्राद्धे भूमि: पञ्चगव्यैर्लिप्ता शोध्या तथोल्मुकैः ।\\
गौरमृत्तिकथा छन्ना प्रकीर्णा तिलसर्षपैः ॥\\
उल्मुकैः शोध्या= परितः उल्मुकादि निदध्यादित्यर्थः ।\\
तत्र " ये रूपाणी''ति पिण्डपितृयज्ञोपदिष्टो मन्त्रः प्रयोज्य इति\\
हेमाद्रिः । एवं संस्कृतायां भुवि यथोक्तानि पाकपात्राणि यथाई\\
प्रक्षालनादिभिः संशोध्य पाकोपक्रमं कुर्यात् ।\\
तथा च देवलः ।\\
तथैव यन्त्रितो दाता प्रातः स्नात्वा सहाम्बरः ।\\
आरभेत नवैः पात्रैरन्यारम्भं च बान्धवैः ॥ इति ॥

{ श्राद्धोपकरणनिरूपणम् । १८३\\
अशक्तः = स्वयं पाकारम्भं कृत्वा बान्धवैरन्वारम्भं समाप्ति कार\\
येदित्युत्तरार्द्धार्थः ।\\
पत्न्यां पाककर्तृत्वे लिङ्गं चमत्कारखण्डे ।\\
ततश्च श्रपयामास तदर्थ जनकोद्भवा ।\\
रामादेशात् स्वयं साध्वी विनयेन समन्विता 1\\
अत्र रामादेशादित्यनेनास्यानुकल्पत्वं सूचितम् ॥\\
अत्र विशेषो महाभारते ।\\
रजस्वला च या नारी व्यङ्गिता कर्णयोस्तथा ।\\
निर्वापे नोपतिष्ठेत संग्राह्मा नान्यवंशजा \textbar{}\textbar{}\\
निर्वापे= पाकारम्भप्रभृतिश्राद्धकर्माणि । अन्यवंशजा=मातृपितृवंश.\\
व्यतिरिक्तवंशसम्बन्धा । अत्र चौदनपाकोऽग्निमता, पितृभ्यो निर्व\\
पामीति मन्त्रेणायुजो मुष्टोस्तण्डुलान्निरूप्य कर्तव्यः ।\\
तथा च पाद्मे ।\\
अभिमानिर्वपेत् पैत्रं (१) चरु चासप्तमुष्टिभिः ।\\
पितृभ्यो निर्वपामीति सर्वं दक्षिणतो न्यसेत् ॥ इति ।\\
अत्र चरुग्रहणादोदन एवायं विधिः, न सूपादौ । अत एवात्रौदनो\\
ऽपि अनवस्त्रावितान्तरोष्मपक्क एव कार्यः, चरुशब्दस्य तत्रैव प्रसिद्धेः।\\
श्राद्धीयद्रव्यस्य च, सकृत् प्रक्षालनं कार्यम्, "सकृत् प्रक्षालितं पि.\\
तृभ्यः" इति गोभिलेोक्तेः । अत्र च साग्निकन श्राद्धार्थ वैश्वदेवार्थे च\\
पृथक् पृथक् पाकः कर्तव्यः ।\\
पित्रर्थं निर्वपेत् पाकं वैश्वदेवार्थमेव च\\
वैश्वदेवं न पित्रर्थे न दार्श वैश्वदेविकम् \textbar{}\textbar{}\\
इति साग्निकं प्रक्रम्य लौगाक्षिवचनात् । इदं च श्राद्धात् पूर्व मध्ये\\
वा वैश्वदेवकरणपक्षे सर्वेषां श्राद्धोत्तरकाले तत्करणे तु ( श्राद्धशे-\\
बातू । ) एवं निरशेरपि श्राद्धशेषेणैव ।\\
श्राद्धं निर्वर्त्य विधिवद् वैश्वदेवादिकं ततः ।\\
इति पैठीनसिवचनान् । ततः = श्राद्धीयान्नादित्येवं सकलनिबन्धकाराः।\\
कर्कस्तु ।\\
वैश्वदेवात्रादेव सर्वदा श्राद्धं कार्यमित्याह ।\\
श्राद्धदिने च परिजनेनापि स्नात्वा शुचितया स्थेयम् ॥

% \begin{center}\rule{0.5\linewidth}{0.5pt}\end{center}

 १ ) चरुं वासममुष्टिभिरिति कमलाकरोद्धृतः पाठः ।

१८४ वीरमित्रोदयस्य श्राद्धप्रकाशे-

{\\
तथा च भविष्ये ।\\
... ... ... ... ... ।\\
... ... ... ... ... l\\
हारीतोऽपि ।\\
कृतकर्माणः सत्रीबालवृद्धाः सुरभिस्नाताः शुचयः शुचिवास.\\
सः स्युरिति ।\\
सुरभिस्नाताः सुगन्धितैलादिद्रव्यस्नाताः । एतच्चाभ्युदयिकविष-\\
यमिति हेमाद्रिः । ततो निमन्त्रितानां ब्राह्मणानां पूर्वाहे एव
श्मश्रुकर्तनं\\
कारयेत् । स्नानाभ्यञ्जनं च दद्यात् ।\\
बतु प्रचेतसोकम् ।\\
तदुक्तं भविष्ये ।\\
तैलेनोद्वर्तने स्नानं दद्यात् पूर्वाह्न एव तु ।\\
श्राद्धभुग्भ्यो नखश्मश्रुच्छेदनं चापि कारयेत् ॥ इति ।\\
तत्र विशेषो देवलस्मृतौ ।\\
ततोऽनिवृत्ते मध्याहे फ्ऌसरोमनखान् द्विजान् ।\\
अभिगम्य यथामार्गे प्रयच्छेद दन्तधावनम् ॥\\
तैलमुद्वर्तनं स्नानं स्नानीयं च पृथग्विधम् ।\\
पात्रैरौदुम्बरैर्दद्यात् वैश्वदेविकपूर्वकम् ॥ इति ।\\
क्लृप्तरोमनखान् = वप्तरोमनखान् \textbar{} अनिवृत्त इति छेदः ।\\
अत एव मार्कण्डेयः ।\\
अहः षट्सु मुहूर्तेषु गतेषु प्रयतान् द्विजान् ।\\
प्रत्येकं प्रेषयेत् प्रेष्यान् ज्ञानायामलकोदकम् ॥\\
प्रेष्यप्रेषणं च स्वयं गमनाशकौ । अभिगम्येत्यनेन स्वस्यैव\\
गमनप्रतीतेः । तैलदानं चानिषिद्धासु तिथिषु वेदितव्यम् । निषि\\
द्वतैलासु तु आमलककल्कदानमिति हेमाद्रिः । तदपि च नामावा.\\
स्यायां "धात्रीफलैरमावास्यायां न स्वायात्" इति निषेधादिति स्मृ.\\
तिचन्द्रिकाकारः ।\\
अन्ये तु निषेधस्य पुरुषार्थत्वात् श्राद्धार्थत्वे तैलदानादिकं भ.\\
वत्येवेत्याहुः ।

{ तत्तु प्रचेतसोक्तम् ।\\
तैलमुद्वर्तनं ज्ञानं दद्यात् पूर्वाह्न एव तु ।\\
श्राद्धसुग्भ्यो नखश्मश्रुच्छेदनं न तु कारयेत् \textbar{}\textbar{}}

{ }{ श्राद्धदिने पूर्वाहकस्यम् \textbar{} १८५}

{\\
इति, तत् इमश्रुच्छेदनं निषिद्धतिथिविषयम् । वस्तुतस्तु कर्तृविष\\
यमिदं व्याख्येयम् ।\\
अत्र च श्मश्रुकरणादि युगपदेव तावतो नापितानुपादाब\\
कार्य, न तु प्रतिन्न ह्मणमावर्तनीयम् । प्रयोगविधिना तथावग.\\
मात् । अत्र च तैलोद्वर्तनादि स्नानीयदानान्त एक {[} एव {]}\\
पदार्थः तथैव प्रसिद्धेः ।\\
ततः श्राद्धकर्ता कुशजलव्यतिरिक्तं सर्व द्रव्यजातमुपकल्प्य\\
नित्यस्त्रानद्रव्याण्यादाय यथालाभं तीर्थे कर्माङ्गस्नान कुर्यात् । तथा\\
च श्राद्धमधिकृत्य भविष्ये \textbar{}\\
कर्तुः स्नानं भवेत्तीर्थे प्रातर्मध्याह्न एव तु ।\\
पताभ्यामेव स्नानाभ्यां प्रातर्मध्याह्निकस्नानयोस्तन्त्रेण सिद्धिर्ज्ञेया
\textbar{}\\
बाससि विशेषमाह प्रचेताः । " श्राद्धकृच्छुक्लवासाः स्यात्" इति ।\\
तिलके विशेषो वक्ष्यते ।\\
ततो मध्याह्नसन्ध्यान्त कृत्वा श्राद्धार्थमुदकं कुशांश्चाऽऽ\\
हरेदिति हेमाद्रिः \textbar{}\\
अन्ये तु कुशाऽऽहरण पाकात् पूर्व कार्यम्, पाकोत्तरकरणे माना\\
भावात्, ऊष्णमन्नं द्विजातिभ्यः श्रद्धया प्रतिपादयेत्' इति वचनो\\
कोष्णत्वासम्पत्तेः, दक्षेण सामान्यतो द्वितीयभागे तस्योक्तेश्च ।\\
उदकाहरणे विशेषो भारते ।\\
उदकाऽऽहरणे चैव स्तोतव्यो वरुण. प्रभुः ॥ इति ।\\
स्तोत्रे च वरुणदेवत्यो मन्त्रः स्वशाखानुसारेण ज्ञेयः । तीर्थोद.\\
काभावे च शुद्धोदक मणिकादेर्ग्राह्यम् । " कुम्भाद्वा मणिकाद्वे"नि\\
तस्य सर्वार्थत्वेन गोभिलेोक्ते । तेनोदकेन कि कार्यमित्यपेक्षायामाह ।\\
योगियाज्ञवल्क्यः ।\\
तेनोदकेन द्रव्याणि प्रोक्ष्याऽऽचम्य पुनर्गृहे \textbar{}\\
ततः कर्माणि कुर्वीत विहितानि च कानि चित् ॥\\
ततो नृवराहपूजां कुर्यात् ।\\
तथा च विष्णुधर्मोत्तरे ।\\
श्राद्धानि प्रयतः स्नातः स्वाचान्तः सुसमाहितः ।\\
शुक्लवासाः समभ्यर्च्य नृवराहं जनार्दनम् ॥\\
आरभेतेति शेषः ।\\
बी० मि २४

{१८६ वीरमित्रोदयस्य श्राद्धप्रकाशे-\\
शिवपुराणे ।\\
पूजयित्वा शिव भक्त्या पितृश्राद्धं प्रकल्पयेत् ।\\
अनयोश्चादृष्टार्थत्वात् समुच्चयः, तदतिक्रमनियमकारणाभा-\\
वात् । उपासकभेदेन व्यवस्थेत्यपरे । उभयोरभिन्नत्वात् ईश्व\\
संपूजनाभिप्रायमिति तु तत्वम् । इति पूर्वाहकृत्यम् ।\\
अथापराहकृत्यम् ।\\
प्रभासखण्डे ।\\
ततोऽपराहसमये प्राप्य कर्ता समाहितः ।\\
स्वयं समाहयेत् विप्रान् सवर्णैर्वा समाप्नुयात् ।\\
विष्णुपुराणे ।\\
पादशौचादिना गेहमागतानर्चयेद् द्विजान् ।\\
देवलः ।\\
ततः स्नानानिवृत्तेभ्यः प्रत्युत्थाय कृताञ्जलिः ।\\
पाद्यमाचमनीयं च सम्प्रयच्छेद्यथाक्रमम् ॥\\
इदं च पाद्यादि रथ्यारजाऽपनयनार्थम् ।\\
मार्कण्डेयः ।\\
स्नातः त्रातान् समाहूय स्वागतेनार्चयेत् पृथक् ।\\
अत्र पृथगित्यभिधानात् युगपदागतेष्वपि पृथगेव स्वागतप्रश्न
\textbar{}\textbar{}\\
अस्मिन्नवसरे पूर्वेद्युः द्वितीयं तृतीयं च सर्वेषु वृत्तेषु ।\\
वृत्तेषु = उपविष्टेषु । अत्र ब्राह्मणानामलङ्करणं कुर्यादित्याह ।\\
यमः ।\\
समाहूतानलङ्कुर्वीतेति ।\\
अलङ्करणप्रकारश्च कूर्मपुराणे ।\\
यथोपविष्टान् सर्वोस्तानलङ्कुर्वीत भूषणैः: 1\\
स्रग्दामभिः शिरोदेशे धूपदीपानुलेपनैः ॥ इति ।\\
इदं च मनुष्यसत्कारत्वात् यज्ञोपवीतिना कार्यमिति हेमाद्रिः ।\\
तथा प्रागुकं पाद्यादिकमपि । निमन्त्रणन्त्वपसव्येनेत्युक्तम् । तदन.\\
न्तरं गृहाङ्गणे मण्डलद्वयं गोमयेन {[} गोमूत्रेण {]} कार्यम् ।\\
तथात्र मात्स्ये ।\\
{[} संमार्जितायां कुर्वीत {]} भवनस्याग्रतो भुवि ।\\
गोमयेनोपलिप्सायां गोमूत्रेण तु मण्डले ।

 श्राद्धादिनेऽपराइ कृत्यम् । १८७

{शम्भुरपि ।\\
सम्मार्जितोपलिप्ते तु द्वारि कुर्वीत मण्डले ।\\
उदक्प्लव उदीच्यं स्यात् दक्षिणं दक्षिणा प्लवम् ॥\\
गोमये विशेषमाह जावालि: ।\\
अमेध्याशनशुन्यानां नीरुजां च तथा गवाम् ॥\\
अव्यङ्गानां च साद्यस्कं शुचि गोमयमाहरेत् ।\\
गोमयविशेषे निषेधमाह ।

{भृगुः ।\\
अत्यन्तजीर्णदेहाया बन्ध्यायाश्च विशेषतः ।\\
आर्ताया नवस्ताया न गोर्गोमयमाहरेत् ॥\\
मण्डलपरिमाणमाह ।\\
लौगाक्षिः ।\\
हस्तद्वयामितं कार्ये वैश्वदेविकमण्डलम् ।\\
तद्दक्षिणे चतुर्हस्तं पितॄणामविशोधने ॥\\
संग्रहे तु परिमाणान्तरमुक्तम् ।\\
प्रादेशमात्रं देवाना मण्डलं चतुरस्रकम् ।\\
वितस्तिमात्रं पित्र्ये तु मण्डलं चर्तुलं भवेत् ॥\\
मण्डलसमीपे च गर्तः कार्य इत्याह स एव ।\\
गर्तः पञ्चाङ्गुलो विप्रे जानुमात्रो महीभुजि ॥\\
प्रादेशमात्रो वैश्ये स्यात् साधिकः स्यात्तु शूद्रके ॥ इति ।\\
तिर्यगूर्द्ध प्रमाणेन खातव्यो दैवपित्र्ययोः \textbar{}\\
बौधायनः ।\\
चतुरस्रं त्रिकोणं च वर्तुलं चार्द्धचन्द्रकम् ।\\
कर्तव्य मानुपूर्वेण ब्राह्मणादिषु मण्डलम् \textbar{}\textbar{}\\
अयं च प्रकारः सङ्ग्रहोकेन वर्तुलत्वादिना विकल्पते ।\\
मण्डलयेोश्छादनमाह ।\\
व्याघ्रपादः ।\\
उत्तरेऽक्षत सयुक्तान् पूर्वाग्रान् विन्यसेत् कुशान् ।\\
दक्षिणे दक्षिणाग्रांश्च सतिलान् विन्यसेद् द्विजः ॥\\
अक्षतग्रहणं गन्धपुष्पाद्युपलक्षणार्थम् । "अक्षताभिः सपुष्पा-\\
भिस्तदभ्यर्च्य " इनि मत्स्यपुराणात् । अत्र च मण्डलकरणं तत्पूज.

{१८८ वीररामित्रोदयस्य श्राद्धप्रकाशे-\\
नञ्च विधिभेदेन पृथक् पदार्थ: । पृथक् पदार्थत्वात् क्रमेणानुष्ठेयम् ।\\
एव मण्डल पूजोत्तरं ब्राह्मणपादप्रक्षालन कार्यम् ।\\
तथा च मात्स्ये ।\\
अक्षताभिः सपुष्पाभिस्तदभ्यर्याऽपसव्यतः ।\\
विप्राणां क्षालयेत् पादावभिवन्द्य पुनः पुनः ॥\\
तश्च दैव पूर्वम् \textbar{} "पाद्यं चैव तथार्ध्ये च दैवे आदौ
प्रदापयेत्"\\
इति स्मृत्यन्तरात् ।\\
अत्र विशेषो ब्रह्माण्डपुराणे ।\\
इदं वः पाद्यमर्धे च चतुर्थ्यन्तं निवेदयेत् ।\\
अत्र मन्त्रो ब्रह्मनिरुको ।\\
शत्रोदेवीति मन्त्रेण पाद्यं चैव प्रदापयेत् \textbar{}\textbar{}\\
भविष्ये ।\\
अक्रोधनैः पठित्वा तु दद्यात् पाद्यं ततः परम् ।\\
एतत्ते पाद्यमित्युक्त्वा दद्यात् तोयं सपुष्पकम् ॥\\
एतदाचमनीय चत्याभाष्याचमनीयकम् ॥ इति ।\\
इंदं पाद्याचमनीयदानं पाद्यासनान्तरभावि भिन्नमेव, अत्रैवाग्रे\\
पुनरभिधानात् । अत्र च पाद्यार्घाचमनीयदानानां पृथक् पदार्थत्वात्\\
पदार्थानुसमयः कार्यः । आचमनीयं च मण्डलादुत्तरत उपवि-\\
टेभ्यो ब्राह्मणेभ्यो दद्यात् ।\\
मण्डलादुत्तरे देशेदद्यादाचमनीकम् ।\\
इति लौगाक्षिवचनात् ।\\
ब्राह्मणैश्च तथाऽऽचान्तव्य यथाऽऽचमनोदकपाद्योदकयोर्मिथः\\
संसर्गो न भवेत् ।\\
बत्राचमनवारीणि पादप्रक्षालनोदकैः ।\\
सङ्गच्छन्ते बुधाः श्राद्धमासुरं तत् प्रचक्षते ॥\\
इति नारदीयोक्तेः । अत्रावशिष्टपादप्रक्षालनोदकं तन्मण्डलोपरि\\
आचारान्निक्षिपेत् । ततो द्विराचम्य द्विजैः सह श्राद्धभूमिमागत्य\\
श्राद्धसिद्धिरस्तु, इति पृष्ट्वा तैश्चास्तु इति उक्त आसनान्युपकल्पयेत्
।\\
तथा च ऋतुः ।\\
दर्भपाणिद्विराचम्य लघुवासा जितेन्द्रियः ।\\
परिश्रिते शुचौ देशे गोमयेनोपलेपिते ॥

{ श्राद्धादिनेऽपराहृकृत्यम् । १८६\\
दक्षिणाप्रवणे सम्यगाचान्तान् प्रणतान् शुचीन् ।\\
आसनेषु सदर्भेषु विविक्तेषूपवेशयेत् ॥\\
यमो ऽपि ।\\
ततः सिद्धिमिति प्रोच्य कल्पितेष्वासनेष्वपि ।\\
आसध्वमिति तान् ब्रूयादासनं संस्पृशन्नपि ॥\\
प्रोच्य= वाचयित्वा ।\\
अत्रिः ।

{ विप्रासनानि देयानि तिलांश्चैव कुशैः सह ।\\
पृथक् पृथक् त्वासनानि तिलतैलेन दीपिकाः ॥\\
तैलग्रहणं घृताद्युपलक्षणार्थम् \textbar{} आसनानि=कुतुपाख्य कम्बला-\\
दीनि पूर्वोक्तानि ।\\
देवलः ।\\
ये चात्र विश्वदेवानां विप्राः पूर्वे निमन्त्रिताः ।\\
प्राङ्मुखान्यासनान्येषां द्विदर्मोपहतानि च ।\\
दक्षिणामुखयुक्तानि पितृणामासनानि च \textbar{}\textbar{}\\
दक्षिणाप्रकदर्भाणि प्रोक्षितानि तिलोदकैः ।\\
मुखशब्देन पीठादौ कल्पितं मुखं कम्बलादौ च दशा उच्यते ।\\
आसनानि च परस्परासंसृष्टानि स्थापनीयानि "आसनेषु विविक्तेषु\\
सदर्भेषूपेवशयेत्" इतिक्रतुवचनात् ।\\
अत्र हेतुमाह । गार्ग्यः ।\\
स्पर्शे स्पर्शे भवेत् पापमेकपङ्किनियोगतः ।\\
हीनवृत्तादिपङ्कौ च युक्त तस्माद्विवेचनम् \textbar{}\textbar{}\\
विवेचन=पङ्किभेदः \textbar{}\\
तत् साधनमाह । बृहस्पतिः ।\\
एकपङ्क्युपविष्टा ये न स्पृशन्ति परम्परम् ।\\
भस्मना कृतमर्यादा न तेषां सङ्करो भवेत् ॥\\
इव चासनदानादि देवपूर्वक कार्यम् "दैवपूर्व श्राद्ध"मिति\\
कार्तीयोक्तेः ।\\
{[} दैवपित्र्यब्राद्धणानां दिग्विशेषाभिमुख्यप्रकारोऽपि ।\\
द्वौ दैवेऽथर्वणौ विप्रौ प्राङ्मुखावुपवेशयेत् ।\\
पित्र्ये तदङ्मुखांस्त्रींश्च {]} बद्द्द्द्चाध्वर्युसामगान् ॥\\
इति शातातपोक्तेः ।

{१९० }{ वीरमित्रोदयस्य श्राद्धमकाशे-}{\\
अत्र विशेषः पैठीनसिनोक्तः ।\\
प्राङ्मुखान् विश्वेदेवानुपवेशयेत्, हविष्मत्सु आसनेषु पितॄन्\\
दक्षिणपूर्वेणेति ।\\
दक्षिणपूर्वेण=दक्षिण पूर्वाभिमुखानित्यर्थः । अथवा देवब्राह्मणापे-\\
क्षया दक्षिणस्यां दिशीत्यर्थ' ।\\
हारीतेन तु । पित्र्यब्राह्मणानां पूर्वाभिमुखतया दैविकानां\\
चोत्तराभिमुखतयोपवेशनमुक्तम् ।\\
दक्षिणाग्रेषु दर्भेमु प्राङ्मुखान् भोजयेत् । उदङ्मुखान् वैश्व\\
देवे इति ।\\
बौधायनेनाप्युक्तम् ।\\
सदर्भों पक्लृप्तेष्वासनेषु द्वौ द्वौ देवे, श्रीन् पित्र्ये,
एकैकमुभयत्र\\
वा प्राङ्मुखानुपवेशयेदुदङ्मुखान् वेति ।\\
तदेवमत्र दिग्विशेषाभिमुख्ये पञ्च पक्षा भवन्ति । तत्र वैश्व-\\
देविका: प्राङ्मुखाः, पिझ्यास्तूदङमुखा इत्येकः \textbar{} {[}
वैश्वदेविका.\\
प्राङ्मुखाः, पित्र्यास्तु दक्षिणपूर्वाभिमुखा इति द्वितीयः । वैश्वदे\\
विका उत्तराभिमुखाः, पित्र्यास्तु प्राङ्मुखा इति तृतीयः । वैश्व\\
देविकाः प्राङ्मुखाः, पित्र्यास्तदङ्मुखा इति चतुर्थ: ।{]} वैश्वदेविका\\
उदङ्मुखाः पित्र्याः प्राङ्मुखा इति पञ्चमः । अत्र स्वस्वगृह्यानुसा\\
रेण व्यवस्था \textbar{} स्वगृहये विशेषानाम्नाते तु वैश्वदेविकानां
प्राङ्मु•\\
खत्वं पितृणामुदङ्मुखत्वमिति । अयमेव पक्षोऽङ्गीकर्तव्यो बहुस्मृति\\
संमतः । अत्र दैवं प्रदक्षिणोपचारेण पितॄणामप्रदक्षिणोपचारेण का,\\
र्यम् \textbar{} "प्रदक्षिण तु देवाना पितॄणामप्रदक्षिण" मिति बौधायनोकेः
।\\
तथा च कात्यायने ।\\
पिण्डपितृयज्ञवदुपचारः पित्र्ये ।\\
पित्र्ये= पितृब्राह्मणे \textbar{}
पिण्डपितृयज्ञवदुपचारः=पिण्डपितृयञ्चवत् क्रिया ।\\
अपसव्यम् दक्षिणाभिमुखेन कर्तव्यम्, दक्षिणसंस्थ कर्तव्यमिति वा ।\\
तथा च यमः ।\\
दक्षिणसंस्था आसरिन् न स्पृशेयुः परस्परम् ॥ इति ।\\
दक्षिणबाहुभागे संस्था येषां ते दक्षिणसंस्था इति विग्रहः । एतच\\
ब्राह्मणपङ्केः पश्चिमोपक्रमप्रागपवर्गत्वे एवोपपद्यते । अत एवाह ।

{ श्राद्धदिने पराह्णकृत्यम् }{ }{।}{ १९१\\
छागलेयः ।\\
प्रतीच्यां समुपक्रम्य प्राच्यां निष्ठा यदा भवेत् ।\\
दक्षिणासस्थता ह्येषां पितॄणां श्राद्धकर्मणि ॥ इति ।\\
ततश्च प्रादक्षिणविधानात् दक्षिणादिगुपक्रममुद्गपवर्गं विश्वे-\\
देवानामुपवेशनम् }{ }{।}{ पश्चिमोपक्रम पूर्वदिगपवर्गं पि}{तॄ}{णाम् ।\\
अत्र विशेषः शङ्खलिखिताभ्यामुक्तः ।\\
ब्राह्मणानुपसंगृह्योपवेशयेदासनमन्वालभ्येति ।\\
यमः ।\\
आसनं संस्पृशन् सव्येन पाणिना दक्षिणेन ब्राह्मणानुपसंगृह्य\\
समाध्वमिति चोक्त्वोपवेशयेत् । इति ।\\
व्यासस्तु ।\\
आसतामिति तान् ब्रूयादासन संस्पृशन्नपि ।\\
उपस्तीर्णेषु चासीरन् न स्पृशेयुः परस्परम् ॥\\
उपवेशने मन्त्रो धर्मेणोक्तः ।\\
अन्वालभ्य ततो देवानुपवेश्य ततः पितॄन् ।\\
समस्ताभिर्व्याहृतिभिरासनेषूपवेशयेत् ॥\\
पैतृकाणामेवोपवेशने व्याहतय इत्यपरार्कस्मृतिचन्द्रिकाकारौ ।\\
अन्येतु -\\
उभयेषामपि व्याहृतिभिरुप्रवेशनमिच्छन्ति ॥\\
श्राद्धकौमुद्यां तु भविष्यपुराणे ।\\
यवोदकेन संप्रोदय स्पृष्ट्वा च पाणिनाऽऽसनम् ।\\
सव्याहृतिकां गायत्री जप्त्वा तानुपवेशयेत् ॥\\
एवं पित्रादिविप्राणामासनान्युपकल्प्य च ।\\
तिलोदकैश्च संप्रोक्ष्य जप्त्वा तानुपवेशयेत् ॥\\
इत्युभयेषामासनदानात् पूर्वं सव्याहृतिजप एवोक्तः । अत्र हेमा-\\
द्रौ समन्त्रकमुपवेशनप्रयोगवाक्यं ॐ भूर्भुवः स्वः, समाध्वमिति\\
वा, अत्रास्यतामिति वा, यजमानेनोच्चारणीयम् । तदनन्तरं द्विजे-\\
रपि ॐ सुसमास्मह इति प्रतिवद्भिरुपवेष्टव्यम् । एवं पित्रादिश्रा\\
द्वार्थान् ब्राह्मणानुपवेश्य मातामहश्राद्धार्थानप्युपवेशयेदित्युक्तम्
।\\
तत्र महाव्याहृतिसमाध्वमित्येतयोः करणमन्त्रत्वेनाऽऽम्नातयोरेका.\\
र्थयोर्विकल्प एव युक्तो न समुच्चयो वचनाभावत् । व्याहृतिगायत्र्या

\end{document}