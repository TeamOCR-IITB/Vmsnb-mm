\documentclass[11pt, openany]{book}
\usepackage[text={4.65in,7.45in}, centering, includefoot]{geometry}
\usepackage[table, x11names]{xcolor}
\usepackage{fontspec,realscripts}
\usepackage{polyglossia}
\setdefaultlanguage{sanskrit}
\setotherlanguage{english}
\setmainfont[Scale=1]{Shobhika}
\newfontfamily\s[Script=Devanagari, Scale=0.9]{Shobhika}
\newfontfamily\regular{Linux Libertine O}
\newfontfamily\en[Language=English, Script=Latin]{Linux Libertine O}
\newfontfamily\ab[Script=Devanagari, Color=purple]{Shobhika-Bold}
\newfontfamily\qt[Script=Devanagari, Scale=1, Color=violet]{Shobhika-Regular}
\newcommand{\devanagarinumeral}[1]{
\devanagaridigits{\number \csname c@#1\endcsname}} % for devanagari page numbers
\XeTeXgenerateactualtext=1 % for searchable pdf
\usepackage{enumerate}
\pagestyle{plain}
\usepackage{fancyhdr}
\pagestyle{fancy}
\renewcommand{\headrulewidth}{0pt}
\usepackage{afterpage}
\usepackage{multirow}
\usepackage{multicol}
\usepackage{wrapfig}
\usepackage{vwcol}
\usepackage{microtype}
\usepackage{amsmath,amsthm, amsfonts,amssymb}
\usepackage{mathtools}% <-- new package for rcases
\usepackage{graphicx}
\usepackage{longtable}
\usepackage{setspace}
\usepackage{footnote}
\usepackage{perpage}
\MakePerPage{footnote}
\usepackage{xspace}
\usepackage{array}
\usepackage{emptypage}
\usepackage{hyperref}% Package for hyperlinks
\hypersetup{colorlinks, citecolor=black, filecolor=black, linkcolor=blue, urlcolor=black}
\begin{document}
\begin{longtable}[]{@{}l@{}}
\toprule
\endhead
\begin{minipage}[t]{0.97\columnwidth}\raggedright
\begin{longtable}[]{@{}l@{}}
\toprule
\endhead
\begin{minipage}[t]{0.97\columnwidth}\raggedright
{ THE\\
}{ CHOWKHAMBA SANSKRIT SERIES}{\\
A\\
}{COLLECTION OF RARE \& EXTRAORDINARY SANSKRIT WORKS}{\\
NO.435.\\
}{वीरमित्रोदय -}{\\
श्राद्धप्रकाशः ।\\
}{ महामहोपाध्याय श्रीमित्र मिश्रविरचितः ।}{\\
भट्टराई - इत्युपपदेन\\
}{ न्यायाचार्य्यपण्डितपद्मप्रसादोपाध्यायेन}{\\
'संशोधितः ।

{~\\
}{VĪRAMITRODAY A}{\\
}{ SRĀDDHA PRAKĀS'A}{\\
BY\\
MAHAMAHOPADHYĀYA PANDIT MITRA MIS'RA\\
Edited By\\
Nyāyācharya\\
PANDIT PADMAPRASĀDA UPADHY ĀYA\\
BHATTARAI\\
Vol IX FASCICULUS IV-\\
-\/-\/-\/-\/-\/-\/-\/-\/-\/-\/-\/-\/-\/-\/-\/-\/-\/-\/-\/-\/-\/-\/-\/-\/-\/-\/-\/-\/-\/-\/-\/-\/-\/-\/-\/-\/-\/-\/-\/-\/-\/-\/-\/-\/-\/-\/-\/-\/-\/-\/-\/-\/-\/-\/-\/-\/-\/-\/-\/-\/-\/-\/-\/-\/-\/-\/-\/-\/-\/-\/-\/-\/-\/-\/-\/-\/-\/-\/-\/-\/-\/-\/-\/-\/-\/-\/-\/-\/-\/-\\
}{PUBLISHED \& SOLD BY THE SECRETARY,}{\\
CHOWKHAMBA SANSKRIT SERIES OFFICE\\
Benares City.\\
1935}\strut
\end{minipage}\tabularnewline
\bottomrule
\end{longtable}\strut
\end{minipage}\tabularnewline
\bottomrule
\end{longtable}

{\\


\begin{longtable}[]{@{}l@{}}
\toprule
\endhead
\begin{minipage}[t]{0.97\columnwidth}\raggedright
{PDF Creation and Uploading by:\\
Hari Pārṣada Dāsa (HPD)\\
on 14 December 2014.}\strut
\end{minipage}\tabularnewline
\bottomrule
\end{longtable}

{\\


\begin{longtable}[]{@{}l@{}}
\toprule
\endhead
\begin{minipage}[t]{0.97\columnwidth}\raggedright
\begin{longtable}[]{@{}l@{}}
\toprule
\endhead
\begin{minipage}[t]{0.97\columnwidth}\raggedright
{ }{ * श्रीः *\\
आनन्दवन विद्योतिसुमनोभिः सुसंस्कृता ।\\
सुवर्णाऽङ्कितभव्याभशतपत्रपरिष्कृता ॥ १ ॥\\
चौखम्बा - संस्कृतग्रन्थमाला मञ्जुलदर्शना ।\\
रसिकालिकुलं कुर्यादमन्दाऽऽमोदमोहितम् ॥ २ ॥\\
स्तबका: ४३५.}\strut
\end{minipage}\tabularnewline
\bottomrule
\end{longtable}\strut
\end{minipage}\tabularnewline
\bottomrule
\end{longtable}

{\\
~\\
}{Printed by Jai Krishna Das Gupta\\
at the Vidya Vilus Press, Benares

{ THE\\
}{ CHOWKHAMBA SANSKRIT SERIES}{\\
}{A\\
COLLECTION OF RARE \& EXTRAORDINARY SANSKRIT WORKS.\\
No's. 413, 431, 432 \& 435.\\
THE}{\\
}{V ĪRAMITRODAYA}{\\
SR ĀDDHAPRAK Ā S'A\\
}{ }{By}{\\
MAHANAKOPADHYAYA PANDIT MITRA MIS'RA

{ -\/-\/-\/-\/-\/-\/-\/-\/-\/-\/-\/-\/-\/-\/-\\
Edited with Introduction Index etc.,\\
By\\
NY}{ Ā}{ Y}{ Ā}{ CH }{ Ā}{ RYA\\
PANDIT PADMA PRASĀDA }{UP}{Ā}{DHY ĀYA}{\\
Bhattarai.\\
Vol. IX.}{\\
FASCICULAS I-IV. 1-8

{ -\/-\/-\/-\/-\/-\/-\/-\/-\/-\/-\/-\/-\/-\/-\/-\/-\/-\/-\/-\/-\/-\/-\\
PUBLISHED BY\\
JAI KRISHNA DAS HARIDAS GUPTA\\
}{The Chowkhamba Sanskrit Series Office.}{\\
BENARES.

{ -\/-\/-\/-\/-\/-\/-\/-\/-\/-\/-\/-\/-\/-\/-\/-\/-\/-\\
1935.

{ {[} Registered According to Aot XXV of 1867,\\
All Rights Reserved by the Publisher, {]}\\


{ PRINTED BY\\
JAI KRISHNA DAS GUPTA\\
}{ Vidya Vilas Press}{\\
Benares City.

 

{ -\/-\/-\/-\/-\/-\/-\/-\/-\/-\/-\/-\/-\/-\/-\/-\/-\/-\/-\/-\/-}

  चौखम्बा संस्कृत ग्रन्थमाला


-\/-\/-\/-\/-\/-\/-\/-\/-\/-\/-\/-\/-\/-\/-\/-\/-\/-\/-\/-\/-\/-\/-\/-\/-\/-\\
ग्रन्थ- संख्या ३०\\
ग्रन्थाङ्क: ४१३, ४३१, ४३२, ४३५.


-\/-\/-\/-\/-\/-\/-\/-\/-\/-\/-\/-\/-\/-\/-\/-\/-\/-\/-\/-\/-\/-\/-\/-\/-\/-\\
॥ श्रीः ॥\\
}{ वीरमित्रोदयस्य}{\\
}{ श्राद्धप्रकाशः ।\\
श्रीमहामहोपाध्यायश्रीमित्रमिश्रविरचितः ।

{ -\/-\/-\/-\/-\/-\/-\/-\/-\/-\/-\/-\/-\/-\/-\/-\/-\/-\\
भट्टराई - इत्युपपदेन\\
}{न्यायाचार्यपण्डितपद्मप्रसादोपाध्यायेन}{\\
संशोधितः ।


-\/-\/-\/-\/-\/-\/-\/-\/-\/-\/-\/-\/-\/-\/-\/-\/-\/-\/-\/-\/-\/-\/-\/-\/-\/-\\
प्रकाशक:-\\
जयकृष्णदास हरिदासगुप्तः -\\
}{चौखम्बा संस्कृत सीरिज़ आफिस,}{\\
बनारस सिटी \textbar{}}


-\/-\/-\/-\/-\/-\/-\/-\/-\/-\/-\/-\/-\/-\/-\/-\/-\/-\/-\/-\/-\/-\/-\/-\/-\/-\\
१६६१\\
राजकीयनियमानुसारेणास्य सर्वेऽभिकाराः प्रकाशकन स्वायत्तोकृताः

{ }{ प्रकाशक:-\\
}{जयकृष्णदास हरिदास गुप्तः-}{\\
}{ चौखम्बा संस्कृत सीरिज़ आफिस,}{\\
बनारस सिटी ।

{ }{श्रीगुरुः शरणम् ।\\
विज्ञापनम् ।\\
हो तत्तत्पदार्थसार्थनिर्णयनिविष्टमतयो विशिष्टशिष्टोपदिष्टनवनवप्र-\\
बन्धावलोकनकुतूहलिनो लोकानुग्रहैककृतिनः कृतिनः ! विदितमस्तु तत्र भ-\\
वतां भवतां यत्किल श्राद्धापेक्षितसमस्तवस्तुनिरूपणपरः
श्रीमन्महामहोपाध्याय\\
मित्रमिश्रसुधीमहोदयनिर्मितो वीरमित्रोदयाभिधमहाप्रबन्धान्तर्गतः
श्राद्ध-\\
प्रकाशाख्यो निबन्धः सम्प्रति सर्वात्मना मुद्रितभावेन प्रकाश्यत्वं
समगच्छत ।\\
यद्यपि शतशः सन्ति श्राद्धसम्बन्धिपदार्थसार्थनिरूपणपराणि निबन्धान्त-\\
राणि, तथापि नूनं तानि द्युमणेः पुरतः खद्योतखेलनमेवानुकुर्वत इति नास्ति\\
तत्र मात्रयापि विचिकित्सालवलेशः । अभिहित चात्मनैव-\\
मा कुर्वन्तु मुधा बुधाः परिचयं प्रन्थेषु नानाविधे-\\
ष्वत्यन्तं न हि तेषु सर्वविषयः कश्चित् क्वचिद् वर्तते ।\\
पश्यन्तु प्रणयादनन्यमनसो ग्रन्थं मदीयं त्विमं\\
धर्माधर्म समस्तनिर्णयविधिर्यस्मिन् दरीदृश्यते ॥ इति ।\\
तत्र कोऽयं मित्रमिश्रसुधीः, कदा कुत्र कतमो वा भूभागोऽनेन स्वजनुषा\\
सनाथीकृतः कतमञ्च विप्रकुलमनेन विशेषतोऽलङ्कृतमित्यादिका प्रेक्षावतां\\
प्रबन्धकृत्परिचयप्रतिपित्सा स्वयमेव प्रकाशान्तरे परिहृतेति परीक्षाकैस्तत
एव-\\
शक्यमखिलमप्युन्नेतुमिति कृतमात्मनो वाचाढतामात्र प्रकटनकौशलेन । समा-\\
लोचनायां च वस्तुतस्त एवाधिक्रियन्ते ये ज्ञानविज्ञानसम्पदा भूयसा तपसा च\\
प्रबन्धप्रणेतारमतिशयीरन्, अनेवंविधाश्च समालोचमानाः केवलं चप-\\
लतामेव चिरं चिन्वीरन् न तु वस्तुतथ्यातथ्यावधारणदक्षतामिति तत्रान-\\
घिकृता एव वयमिति किं सहसा सहासेन साहसेन । एतन्मुद्रणं च काशीस्थ-\\
गवर्त्मेण्टपुस्तकालयस्थहस्तलिखितपुस्तकमुपजीव्यैव संम्पादितम् । तत्र च

{६ }{भूमिका ।}{\\
तत्र तत्र सन्दिग्धासल्लग्नपाठविशेषो प्रन्थान्तरेभ्यो निर्णीय
समावेशितः,\\
तत्रोपलब्धानि पाठान्तराणि च तत्तद्ग्रन्थनामस्थलनिर्देशपुरःसरं
टिप्पण्या\\
मुट्टङ्कितानि । त्रुटितस्थलेषु स्वमनीषाकल्पित एव पाठ: ( ) {[} {]}\\
एतच्चिह्नद्वयान्तर्गततयोन्यस्तः, अधिकतया प्रतिभातो ग्रन्थस्थपाठश्च
क्वचित्\\
( ) एतच्चिह्नमध्य एव निवेशितः, तयोर्युक्तायुक्तत्वे चावहितान्तःकरणैः\\
कुशाप्रधिषणैरेवावधार्ये इति । संशोधने च स्वमतिमान्द्यान्नैशिककरणापाट\\
वात्, व्यासङ्गात्, आयसाक्षरनियोक्तृकृतवर्णवैपरीत्यात्, मुद्रणा-\\
वस्थायामक्षरस्खलनप्रभृतिदोषाद्वा बहुत्र 'व्य' स्थाने 'ब्य' इति, 'व'
स्थाने\\
'ब' इति 'ब' स्थाने 'व' इति, 'त्त' स्थाने 'त' इति, 'पु' स्थाने च\\
'षु' इति पतितं तत्र च पाठकमहोदयैरवधातव्यम् । उक्तदोषायत्तानामन्यासां\\
च कतिपयानामशुद्धीनां ज्ञापकं शुद्धिपत्रं, संक्षिप्तविषयानुक्रमणं
चोपन्य\\
स्तमिति कृतेऽपि भूयसि परिश्रमे तत्र तत्र बहुपराद्धं स्यादिति तत्र\\
कृतबुद्धयो विद्वांस एव शरणमिति कृतमनल्पजल्पनेनेति-\\


 संशोधकः ।

\begin{longtable}[]{@{}lll@{}}
\toprule
\endhead
\begin{minipage}[t]{0.30\columnwidth}\raggedright
{विषयः\\
मङ्गलाचरणम्\\
ग्रन्थनिर्मितिप्रवृत्तिहेतुकधनम् 

{सद्धे पतो विषयानुक्रमणिकाकथनम्\\
श्राद्धप्रशंसा\\
श्राद्धस्य कर्तव्यता\\
श्रद्धस्य स्वरूपकथनम्\\
श्राद्धप्रकाशस्थविषयानुक्रमणिका ।\\
आन्द्रलक्षर्णानरूपणम्\\
मुख्यगौणभेदेन श्राद्धशब्दप्रयोगविचारः\\
श्राद्धस्य यागदानरूपताविचारः\\
पित्रादीनां प्रत्येकं देवतात्वनिर्णयः\\
वस्वादीनां देवतात्वविचारः\\
पार्वणश्राद्धे देवतानिर्णयः\\
जीवत्पितृकश्राद्धे देवतानिर्णयः\\
द्विपितृकश्राद्धे देवतानिर्णयः\\
पुत्रिकापुत्रकर्तृकश्राद्धे देवतानिर्णयः\\
वैश्वदेविकश्राद्धे देवतानिरूपणम्\\
विकिरभु कोच्छिष्टयोर्देवता\\
निर्णयः\\
सन्न्यासाङ्गश्राद्धदेवतानिणेयः\\
विधवाकर्तृकश्राद्धदेवता निर्णयः\\
विभक्तिनिर्णयः\\
सम्बन्धगोत्रनामोच्चारणक्रमः\\
गन्धादिदाने सम्प्रदाननिर्णयः\\
श्राद्धोचितद्रव्यनिर्णयः\\
अधर्मोपार्जितद्रव्यनिषेधः\\
श्राद्धोचितद्रव्योत्पत्तिकथनम्\\
ग्राह्यधान्यम्\\
वज्र्ज्यधान्यम्\\
ग्राझाणि मूलफलानि\\
वयनि\\
ब्राह्मक्षीराणि\\
चज्यक्षोराणि\\
मांसविचारः\\
कालविशेषावच्छेदेन तृष्ठिकर-\\
पदार्थभेदकथनम्}\strut
\end{minipage} & \begin{minipage}[t]{0.30\columnwidth}\raggedright
{पृष्ठसंख्या \textbar{} विषयः\\
१ मायान्ननिरूपणम्\\
" वर्ज्यान्ननिरूपणम् }{ " }{श्राद्धीयब्राह्मणनिरूपणम्\\
२ ब्राह्मणप्रशंसा\\
३ श्राद्धीयप्रशस्तत्राह्मणनिरूपणम्\\
" पतिपावननिरूपणम्\\
४ पड़िपावनपावननिरूपणम्

{" योगिनां श्रद्धे नियोगकथनम्\\
८ गृहस्थाद्यपेक्षया योगिनां वैशिष्ट्यकथनम्\\
" ब्राह्मणानां श्राद्धभोजननियमाः\\
२८ भोजयितृनियमाः\\
२९ प्राचीनावीतयज्ञोपवीतविचारः\\
३० श्राद्धीयपदार्थाः\\
३१ यजमानजप्यानि\\
३४ सप्ताचिर्मन्त्रः\\
" प्रशस्त ब्राह्मणानुकल्पकथनम् सन्निहितब्राह्मणानामनतिक्रमणीयत्वम् ७६\\
१५ श्राद्धे वर्ज्यब्राह्मणकथनम्\\
१५ निमन्त्रणम्\\
७७\\
१९ निमन्त्रणीयब्राह्मणसङ्ख्या\\
२१ निमन्त्रणपूर्वकालकृत्यम्\\
२२ \textbar{} निमन्त्रितनियमाः\\
२६ कर्तृनियमाः\\
२७ कर्तृभोक्तृनियमाः\\
३९ सप्ताचिस्तोत्रम्\\
३७ पितृस्तवः\\
३८ \textbar{} श्राद्धदेशाः\\
४१ निषिद्धदेशाः\\
४९ श्राद्धोपकरणानि\\
" अर्धपात्राणि\\
१० पाकपात्राणि\\
भोजनपात्राणि\\
५१ \textbar{} परिवेषणपात्राणि\\
श्राद्धदेशादपास्यानि द्रव्याणि}\strut
\end{minipage} & \begin{minipage}[t]{0.30\columnwidth}\raggedright
 पृष्ठसंख्या

 ५९

{ ५६\\
७१\\
७३\\
१०५\\
१०६\\
१०८\\
१०९\\
११२\\
११४\\
११६\\
१२०\\
१२३\\
१२५\\
59\\
१३०\\
१३१\\
१३२\\
१३९\\
१४७\\
१४८\\
१५२\\
१५३\\
१५५\\
१५६}\strut
\end{minipage}\tabularnewline
\bottomrule
\end{longtable}

 विषयसूची ।

\begin{longtable}[]{@{}lll@{}}
\toprule
\endhead
\begin{minipage}[t]{0.30\columnwidth}\raggedright
{विषयः\\
गन्धाः\\
वज्र्यगन्धाः १५८\\
पुष्पाणि\\
चज्यंपुष्पाणि\\
धूपाः\\
निषिद्धधूपाः\\
दीपाः\\
आच्छादनम्\\
निषिद्धवस्त्राणि\\
यज्ञोपवीतम्\\
दण्डयोगपट्ट\\
कमण्डल्वादि\\
छत्रम्\\
उपानत्पादुके\\
आसनानि\\
शय्यादि\\
चामरव्यजनदर्पणकेशप्रसाधनानि\\
हिरण्यालङ्कारादि\\
अलङ्कारविशेषदाने फलविशेषः\\
गोमहिष्यादिदानम्\\
गवां वर्णविशेषात्फलविशेषः\\
भूगृह पुस्तकाभयादिदानम्\\
प्रकीर्णकदा मम्\\
श्राद्धदिने पूर्वाहकृत्यम्\\
श्राद्धदिनेऽपराहकृत्यम्\\
पुण्डरीकाक्ष स्मरणादिकृत्यम्\\
ब्राह्मणानामासनदानादिकृत्यम्\\
आवाहनम्\\
अर्ध्याद्युपचारविधिः\\
अर्धदानविधिः\\
संस्रव ग्रहणम्\\
गन्धादिदानम्\\
मण्डलकरणादयः पदार्थाः\\
अग्नौकरणम्\\
अझौकरणे देवतामन्त्रादयः\\
हुतावशिष्टप्रतिपत्तिः\\
परिवेषणम्}\strut
\end{minipage} & \begin{minipage}[t]{0.30\columnwidth}\raggedright
{पृष्ठसंख्या । विषयः\\
१५७ पात्रालम्भजपाङ्गुष्ठनिवेशनानि\\
१५८ \textbar{} अन्नसङ्कल्पः\\
१५८ सावित्रीजपादि\\
१६० विकिरदानादि\\
पिण्डदानकालः\\
१६१ पिण्डदानदेशः\\
पिण्डदानेतिकर्तव्यता\\
१६२ पिण्डदानेऽन्न विशेषः\\
१६४ अक्षय्योदकदानम्\\
१६५ \textbar{}वरयाचना\\
" दक्षिणादानम्\\
१६६ विसर्जनम्\\
१६७ पिण्डप्रतिपत्तिः\\
१६८ उच्छिष्टोद्वासनम्\\
" श्राद्धोत्तरकालीनकम\\
१६९ श्राद्धानुकल्पः\\
१७१आमहेमश्राद्धम्\\
१७३ ब्राह्मणानुकल्पः\\
१७४सांकल्पिकश्राद्धम्\\
१७५ विकृतिश्राद्धनिर्णयः\\
१७६वृद्धिश्राद्धम्\\
१७७ सामान्यकृष्णपक्षश्रारम्\\
१८० महालय श्राद्धम्\\
१८२ भरणीश्राद्धम्\\
१९२ अपरपक्ष त्रयोदशी श्राद्धम्\\
१९७ मघाश्राद्धम्\\
२०१ शस्त्रादिहतचतुर्दशीश्राद्धम्\\
२०८ दौहित्रकर्तृकश्राद्धम्\\
२१२ \textbar{} नित्यश्राद्धम्\\
२१७ \textbar{} सांवत्सरिकश्राद्धम्\\
२२१ श्रद्धभेदाः\\
२२६ \textbar{} श्राद्धविकृतिषूहः\\
२२७ श्रद्धाधिकारिनिरूपणम्\\
२३१ जीवच्छ्राद्धनिर्णयः\\
२३४ संन्यासाङ्गश्राद्धनिर्णयः\\
२३५ \textbar{} ग्रन्थसमाप्तिः}\strut
\end{minipage} & \begin{minipage}[t]{0.30\columnwidth}\raggedright
{पृष्ठसंख्या\\
२३८\\
२४०\\
२४१\\
२४२\\
२४४\\
७४७\\
२४९\\
२६६\\
२७३\\
२७५\\
२७७\\
२७९\\
99\\
२८०\\
२८१\\
२८३\\
33\\
२८६\\
२८६\\
२८७\\
२८९\\
३१५\\
३१६\\
३१८\\
33\\
३१९\\
३२१\\
३२२\\
३२३\\
३३१\\
३३४\\
३३९\\
३६१\\
३७१\\
३७२}\strut
\end{minipage}\tabularnewline
\bottomrule
\end{longtable}

{ इति संक्षितश्राद्धप्रकाशस्थविषयानुक्रमणिका ।\\
~\\
~\\
~\\


{ }{अथ\\
}{वीरमित्रोदयस्य श्राद्धप्रकाशः ।}{\\
\textbar{}\textbar{} श्रीगणेशाय नमः ॥\\
गवेषितघनाटवीशिखरिकन्दरो जानकी-\\
वियोगभरमन्दरोन्मथितचित्तवारांनिधिः ।\\
नवीन धनसुन्दरोल्लसदमन्दरोचिः क्रिया-\\
द्धनुर्धरधुरन्धरो रघुपुरन्दरो वः शिवम् ॥ १ ॥\\
द्राग्दारिद्र्यदवानलव्यतिकरव्यालीढहृद्भिक्षुक.\\
क्षेमाय क्षितिमण्डले समुदितः श्रीवीरसिंहाज्ञया ।\\
विद्वद्वृन्दशिरोमणिर्गुणानिधिः श्री मित्रमिश्रः कृती\\
तर्कव्यूहविभावुकः प्रकुरुते श्राद्धप्रकाशं परम् ॥ २ ॥\\
श्राद्धप्रशंसा तत्रादौ ततः श्राद्धस्य लक्षणम् ।\\
श्राद्धस्य यागदानत्वकथनं तदनन्तरम् ॥ ३ ॥\\
देवतानिर्णयः श्राद्धे विस्तरेण ततः परम् ।\\
विकिरोच्छिष्टसन्त्यागे देवता कीर्त्तिता ततः ॥ ४ ॥\\
सन्न्यासस्याङ्गभूते च श्रद्धे युद्देश्यकीर्त्तनम् ।\\
त्यागवाक्यप्रयोगश्च यथावच्च विवेचितम् ॥ ५ ॥\\
गन्धादिदाने तदनु सम्प्रदानविनिर्णयः ।\\
श्राद्धद्रव्यार्जनोपायौ कथितौ विहितेतरौ ॥ ६ ॥\\
श्राद्धग्राह्याणि धान्यानि श्राद्धवर्ज्यानि चाप्यथ \textbar{}\\
ततो मूलफलक्षीरमांसान्युक्तानि च क्रमात् ॥ ७ ॥\\
अथ कालविशेषान्तु तृप्तिकाराणि विस्तरात् ।\\
अथान्नानि ततस्तोयं श्राद्धे सम्यग्विवेचितम् ॥ ८ ॥\\
श्राद्धे विप्रास्ततस्तत्रानुकल्पाः परिकीर्त्तिताः ।\\
सन्निकृष्टद्विजत्यागे दूषणं तदनन्तरम् ॥ ९ ॥\\
श्राद्धे वर्ज्यास्ततो विप्रा विस्तरेण निरूपिताः ।\\
ततो निमन्त्रणं श्राद्धिविप्रसङ्ख्या ततः परम् ॥ १० ॥\\
-

{२ वीरमित्रोदयस्य श्राद्धप्रकाशे-}{\\
निमन्त्रणात्पूर्वकृत्यं नियमाः कर्तृविप्रयोः ।\\
प्राचीनावीतकथनं जानुपातादिकीर्त्तनम् ॥ ११ ॥\\
श्राद्धकाले च जप्यानि कर्त्तुरुक्तान्यतः परम् ।\\
विहिताश्च निषिद्धाश्च श्राद्धे देशास्तत. स्मृताः ॥ १२ ॥\\
भूमिस्वाम्यन्नदानं च कीर्त्तितं तदनन्तरम् \textbar{}\\
श्राद्धदेशादपास्यानि सर्वाण्युक्तान्यतः परम् ॥ १३ ॥\\
दर्भार्धपात्रशय्योपवीतच्छत्रासनादिकम् ।\\
श्राद्धोपकरणं सर्वे प्रत्येकमिह कीर्त्तितम् ॥ १४ ॥\\
श्राद्धपूर्वविधातव्यं तदनन्तरमीरितम् ।\\
ततः श्राद्धप्रयोगश्च सप्रमाण उदाहृतः ।। १५ ।।\\
उच्छिष्टोद्वासनं श्राद्धोत्तरकर्म ततः परम् ।\\
श्राद्धानुकल्पा विप्राणामनुकल्पाश्च कीर्त्तिताः ।। १६ ।।\\
अथ साङ्कल्पिकश्राद्धं विकृतिश्राद्धनिर्णयः ।\\
अष्टकान्वष्टकावृद्धिश्राद्धानामिह कीर्त्तनम् ॥ १७ ॥\\
कृष्णपक्षेषु सर्वेषु विशिष्य च महालये ।\\
श्राद्धमुक्तं त्रयोदश्यां तत्पक्षे श्राद्धमीरितम् ॥ १८ ॥\\
श्राद्धं शस्त्रहतानां च तदनन्तरमीरितम् ।\\
दौहित्रप्रतिपच्छ्राद्धं शुक्लपक्षे च कीर्त्तितम् ।। १९ ।।\\
नित्यश्राद्धं वार्षिकं च श्राद्धं समनुकीर्त्तितम् ।\\
श्राद्धप्रभेदा विकृताचूहाश्चाथ निरूपिताः \textbar{} २० ॥\\
श्राद्धाधिकारिणां जीवच्छ्राद्धस्य च विनिर्णयः ।\\
सन्न्यासाङ्गस्य तदनु श्राद्धस्येह विवेचनम् ॥ २१ ॥\\
एवमेते पदार्थास्तु मित्रमिश्रेण सूरिणा ।\\
श्राद्धप्रकाशे कथिता विचार्याचार्यसंहिताः ॥ २२ ॥\\
तत्र तावच्छ्राद्धप्रशंसामाह-\\
सुमन्तु,\\
श्राद्धात्परतरं नान्यच्छ्रेयस्करमुदाहृतम् ।\\
तस्मात्सर्वप्रयत्नेन श्राद्धं कुर्याद्विचक्षणः ॥\\
ब्रह्मवैवर्तेऽपि,\\
देवकार्यादपि सदा पितृकार्ये विशिष्यते ।\\
देवताभ्यः पितॄणां हि पूर्वमाप्यायनं शुभम् \textbar{}\textbar{}}

{ श्राद्धलक्षणम् । ३\\
पितॄणां पूर्वमाप्यानं च सर्वेषु दैवतकर्मसु कर्माङ्गनान्दीश्राद्धस्य\\
पूर्वमनुष्ठानात् बोध्यम् ।\\
यमोऽपि -\\
ये यजन्ति पितॄन्देवान् ब्राह्मणांश्च हुताशनान् ।\\
सर्वभूतान्तरात्मानं विष्णुमेव यजन्ति ते ॥ इति ।\\
ब्रह्मपुराणेऽपि -\\
यो वा विधानतः श्राद्धं कुर्यात्स्वविभवोचितम् ।\\
आब्रह्मस्तम्बपर्यन्तं जगत्प्रीणाति मानवः ॥ इति ।\\
नागरखण्डेऽपि -\\
श्राद्धे तु क्रियमाणे वै न किञ्चिद्यर्थतां व्रजेत् ।\\
उच्छिष्टमपि राजेन्द्र ! तस्माच्छ्राद्धं समाचरेत् ॥ इति ।\\
श्राद्धकर्तव्यतोक्ता ब्रह्मपुराणे -\\
तस्माच्छ्राद्धं नरो भक्त्या शाकैरपि यथाविधि ।\\
कुर्वीत श्रद्धया तस्य कुले कश्चिन सीदति ॥ इति ।\\
यथाविधि = यथाप्रकारम् । श्राद्धस्वरूपं चाह -\\
आपस्तम्बः,\\
अथैतन्मनुः श्राद्धशब्दं कर्म प्रोवाच, प्रजानिश्रेयसार्थं तत्र पि.\\
तरो देवताः, ब्राह्मणस्त्वाहवनीयार्थे, मासि मासि कार्यम् अपरपक्ष.\\
स्यापराह्नः श्रेयानिति ।\\
श्राद्धशब्दं = श्राद्धमितिशब्दो वाचको यस्य तत्तथा । त्यक्तद्रव्यप्र-\\
तिपत्त्यधिकरणत्वेनाहवनीयकार्यार्थत्वं ब्राह्मणस्य \textbar{} अपरपक्षस्य
=कृष्ण-\\
पक्षस्य \textbar{}\\
बृहस्पतिरपि -\/-\\
संस्कृतं व्यञ्जनाद्यं च पयोमधुघृतान्वितम् ।\\
श्रद्धया दीयते यस्माच्छ्राद्धं तेन निगद्यते ॥ इति ।\\
ब्रह्मपुराणे -\\
देशे काले च पात्रे च श्रद्धया विधिना च यत् ।\\
पितृनुद्दिश्य विप्रेभ्यो दत्तं श्राद्धमुदाहृतम् ॥ इति ।\\
दत्तं = प्रतिपादितम् ।\\
मरीचिरपि -\\
प्रेताम्पितॄनप्युद्दिश्य भोज्यं यत्प्रियमात्मनः ।\\
श्रद्धया दीयते यत्तु तच्छ्राद्धं परिकीर्त्तितम् ॥ इति ।

{ }{४ }{ वीरमित्रोदयस्य श्राद्धप्रकाशे-}{\\
प्रेतान्=अकृतसपिण्डीकरणान् । पितॄन् = कृतसपिण्डनान् । }{दीयते}{\\
यत्त्विति । अत्र यदिति क्रियाविशेषणम्। तथा च तादृशं यद्दानं तच्छ्रा.\\
द्धमित्यर्थः । अत्रापस्तम्बादिसकलवचनपर्यालोचनया प्रमीतमात्रोद्दे\\
इयकान्नत्यागविशेषस्य ब्राह्मणाद्यधिकरणकप्रतिपत्यङ्गकस्य श्राद्ध.\\
पदार्थत्वं प्रतीयते । एव च गन्धादिदानाग्नौकरणविकिरदानानां न\\
श्राद्धत्वं किन्तु तदङ्गत्वमेवेति बोध्यम् । ब्राह्मणप्रतिपत्तेरङ्गत्वं च
क्का\\
चित्कं विवक्षितम् । तेनाग्न्यादिप्रक्षेपाङ्गके श्राद्धविशेषे नाव्याप्तिः
।\\
तदङ्गत्वं च वाचनिकानिदेशव्यतिरिक्तप्रमाणविहितं विवक्षितम् ।\\
तेन " पिण्डवच्च पश्चिमा प्रतिपत्तिः" इतिछन्दोगपरिशिष्टवचनातिदि\\
ष्टतदङ्गके पित्र्यबलिदाने नातिव्याप्तिः । न च तच्छ्राद्धत्वेन कुतो\\
न सङ्गृह्यत इति वाच्यम् ।\\
श्राद्धं वा पितृयज्ञः स्यात्पित्र्यो बलिरथापि वा ।\\
इति छन्दोगपरिशिष्टस्वरसात्तस्य श्राद्धभिन्नत्वप्रतीतेः । तथाशिष्ट\\
व्यवहाराभावाच्च । पिण्डपितृयज्ञस्तु श्राद्धमेव । "तच्छ्राद्धमितरदमा.\\
वास्यायाम्" इति गोभिलवचनेन तस्यापि श्राद्धत्वोक्तेः । तच्छब्देन\\
पूर्वोक्तपिण्डपितृयज्ञपरामर्शात् । उदाहृतवाक्यैरपि तस्य श्राद्धत्व-\\
प्रतीतिः ।\\
पिण्डांस्तु गोजविप्रेभ्यो दद्यादन्नौ जलेऽपि वा ।\\
इत्यनेन याज्ञवल्क्यवचनेन तस्यापि ब्राह्मणप्रतिपत्त्यङ्गकत्वसिद्धेः ।\\
प्रमीतानामुद्देश्यत्वं च देवतात्वरूपम् । तेन फलभागितया तदुद्देश्य-\\
कब्राह्मणसम्प्रदानकान्नत्यागे नातिव्याप्तिः \textbar{} अन्नपदस्य च
भोज्य\\
स्थानीय द्रव्योपलक्षकत्वान्न हिरण्यश्राद्धादावव्याप्तिः । यस्तु नृसिं\\
हपुराणे -\\
दिव्यपितृभ्यो देवेभ्यः स्वपितृभ्यस्तथैव च ।\\
दत्त्वा श्राद्धमृषिभ्यश्च मनुष्येभ्यस्तथात्मनः ॥\\
इति देवादिश्राद्धे श्राद्धशब्दः स मासाग्निहोत्रवद्द्रौणस्तद्धर्मप्रा.\\
प्त्यर्थः । एवं च "दैविकं दशमं स्मृतम्'' इतिवक्ष्यमाणश्राद्धविभागो\\
ऽपि गौणमुख्यसाधारण एव । न च तत्रापि मुख्यता कि न स्यादितिवा.\\
व्यम् । श्राद्धपदव्युत्पादकेषु उदाहृतवाक्येषु प्रेतपद पितृपदयोरेव श्र.\\
वणात्प्रमीत मात्रकोद्देश्यकश्राद्धस्यैव मुख्यत्वावगतेः । एवं
चात्रोत्सर्ग.\\
पिण्डदानयोर्द्वयोरपि प्रत्येकं श्राद्धत्वं सिद्ध्यति । अत एव
ब्रह्मपुराणेऽ.\\


{ }{ }{ श्राद्धलक्षणम् । }{ ५\\
नोत्सर्गमुक्त्वा ``श्राद्धं कृत्वा प्रयत्नेन '' इति अन्नोत्सर्गमात्रे
श्राद्धपद-\\
प्रयोगो दृश्यते । गयादौ पिण्डदानमात्रेऽपि च शिष्टानां श्राद्धपद\\
प्रयोगोऽप्युपपद्यते । अत एवाहिताग्नेः "पित्रर्चनं पिण्डैरेव " इति नि\\
गमोऽप्यशक्ती केवलं पिण्डदानमाह । मघाश्राद्धादौ पिण्डदानं विना\\
श्राद्धसिद्धावपि न तस्य प्राधान्यहानिः । प्रधानस्यैव सतो वचनेन\\
तत्र पर्युदासात् । एतेन -\\
नित्यश्राद्धमदैव स्यादर्ध्यपिण्डविवर्जितम् ।\\
इतिहारीतवचनान्नित्यश्राद्धे,\\
भौजङ्गी तिथिमासाद्य यावच्चन्द्रार्कसङ्गमम् ।\\
तत्रापि महती पूजा कर्तव्या पितृदैवते \textbar{}\textbar{}\\
ऋक्षे पिण्डप्रदानं तु ज्येष्ठपुत्री विवर्जयेत् ।\\
इति देवीपुराणान्मघाश्राद्धे च पिण्डदाननिषेधोऽवगम्यते । न चा.\\
प्राप्तस्य निषेधो घटते प्राप्तिश्चात्रातिदेशेन स चाङ्गानामेवातः पिण्ड.\\
दानमङ्गम् । या तु-\\
अग्नौ हुतेन देवस्थाः पितृस्था द्विजतर्पणैः \textbar{}\\
नरकस्थाच तृप्यन्ति पिण्डैर्दचैस्त्रिभिर्भुवि ॥\\
इतिपिण्डदाने फलश्रुतिः सोऽर्थवादः "अङ्गेषु स्तुतिः परार्थ-\\
त्वात्'' ( अ० ४ पा० ३ अधि० ७ सू० १९ ) इति न्यायात् । गयादौ\\
पिण्डदानमात्रविधिस्तु अङ्गभूतपिण्डदानात्कर्मान्तरं प्रकरणान्तरस्था•\\
त्वादिति शूलपाण्याद्युक्तं परास्तम् । पूर्वोदाहृतवाक्यार्थपर्यालोचनया\\
उभयोरपि प्राधान्ये सिद्धे नित्यश्राद्धादौ श्राद्ध
विधिनैवाभयप्राप्तावेक\\
तरपर्युदासोपपत्तेः । तस्मात्सुष्ट्र्कमन्नोत्सर्गपिण्डदानयोः प्राधा-\\
न्यमिति ।\\
केचित्तु "अग्नौ हुतेन' इत्यादिवचनेऽप्रौकरणस्यापि फलश्रव\\
णात्तदपि प्रधानम् । अग्नौकरणोद्देश्यानां कव्यवाहनादीनां पितृपदेन\\
सङ्ग्रहात्पाणिहोमपक्षे ब्राह्मणाधिकरणकप्रतिपत्तिसम्भवाध्वेत्याहुः
\textbar{}\\
शूलपाणिस्तु सम्बोधनपदोपनीतान्पित्रादींश्चतुर्थ्यन्तपदेनोद्दिश्य ह-\\
विस्त्यागः श्राद्धमित्याह । अत्र फलभागित्वरूपोद्देश्यतानिवृत्त्यर्थं स\\
म्बोधनपदोपनीतानिति विशेषणम् । पित्रादीनित्यादिपदेन देवादयो-\\
ऽपि विवक्षिताः । तेन देवश्राद्धादावपि श्राद्धशब्दो मुख्य एवेति ।\\
मैथिलास्तु वेदबोधितपात्रालम्भपूर्वक हविस्त्यागः श्राद्धम् । अत्रा-

{६ }{ वीर}{ }{मित्रोदयस्य श्राद्धप्रकाशे-}{\\
लम्भो न श्राद्धविशेषणम् । अन्यथा तस्य वेदबोध्यत्वाभा\\
वेन तद्घटितश्राद्धस्य वेदाबोध्यत्वापत्तेः किन्तूपलक्षणम् । उपल\\
क्ष्यश्च स्वतो विलक्षणस्त्यागविशेष एव । एतेन यन्मत एको\\
द्दिष्टे पात्रालम्भो नास्ति तन्मत एकोद्दिष्टश्राद्धे नाव्याप्ति. ।\\
नापि सन्न्यासिकर्तृकात्मादिश्राद्धेऽपि सा । ``दैविक दश.\\
मं स्मृतम्'' इति विभागोऽपि च समञ्जसो भवति । पिण्डदानं त्व-\\
इमेव ।\\
श्राद्धं कृत्वा प्रयत्नेन त्वराक्रोधविवर्जित' ।\\
उष्णमनं द्विजातिभ्यः श्रद्धया प्रतिपादयेत् ॥\\
इत्यादिवाक्येषु आनन्तर्यवाचिक्त्वाप्रत्ययेनानोत्सर्गे एव श्राद्ध-\\
पदप्रयोगात् । "त्रिषु पिण्डः प्रवर्त्तत" इति चाङ्गमुखेन प्रधाननिर्दे.\\
शः पिण्डपितृयज्ञविषयं वा इत्याहुः ।\\
ननु त्यागो न श्राद्धं किन्तु -\\
पितॄनुद्दिश्य विप्रेभ्यो दत्त श्राद्धमुदाहृतम् ।\\
इति ब्रह्मपुराणात्,\\
श्रद्धया }{दीयते}{ यस्मात्तच्छ्राद्धं परिकीर्त्तितम् ।\\
इति मरीचिचचनात्,\\
प्रमीतस्य पितुः पुत्रै. श्राद्धं देयं प्रयत्नतः ।\\
इति स्मृत्यन्तरवचनाच्च दानकर्मणो द्रव्यस्यैव श्राद्धत्वावगमात्\\
त्यज्यमानं द्रव्यमेव श्राद्धम् । एवं च-\\
श्राद्धमामं तु कर्त्तव्यमिति वेदविदां स्थितिः ।\\
इति द्रव्यसामानाधिकरण्यमप्युपपद्यते । न चैवमादिषु श्राद्ध.\\
शब्दस्य लक्षणया द्रव्यपरत्वं प्रमाणाभावादिति चेत्, न । " श्राद्ध कु·\\
र्यात्" इत्यादी द्रव्यस्य सिद्धत्वेन साक्षाद्भावमान्वयासम्भवात् ।\\
क्रियापरत्वेन साक्षादन्वये सम्भवति अनुपस्थितत्यागादिक्रियाद्वारा\\
परम्परान्वयस्यानौचित्यात् । "सोमेन यजेत" इत्यादौ तु बलवत्या\\
प्रसिद्ध्या द्रव्यपरत्वे सोमपदस्यावधारिते सोमस्य श्रौतधात्वर्थद्वारा\\
भावनान्वयो युक्तो भवति । प्रकृते तु श्राद्धशब्दस्य द्रव्ये
प्रसिद्ध्यभा\\
वादश्रुतधात्वर्थद्वारकपरम्परयान्वयोऽनुचित एव । किञ्च । कुशयव\\
तिलगोधूममांसादिद्रव्यं पित्रादिभ्यः श्रद्धया देयमित्यादिना श्राद्धे\\
विधित्सितद्रव्यस्य
प्राप्तत्वात्तत्प्रख्यन्यायेनाग्निहोत्रादिपदवत्कर्मना.\\


{श्राद्धशब्दस्य कर्मनामत्वव्यवस्थापनम् । ७\\
मधेयतैवोचिता । किञ्च नित्यनैमित्तिककाम्यभेदा अग्रे वश्यन्ते ते\\
च प्रायशः कर्मण्येव प्रसिद्धा इत्यतोऽपि कर्मनामता । अत एवाप\\
स्तम्बेन "श्राद्धशब्दं कर्म" इत्युक्तम् ।\\
यत्तु -\\
प्रमीतस्य पितुः पुत्रैर्देयं श्राद्धं प्रयत्नतः ।\\
इति श्राद्धस्य देयत्वमुक्तम्, यच्च\\
श्राद्धमामं तु कर्त्तव्यमिति वेदविदां स्थिति ।\\
इति द्रव्यसामानाधिकरण्यम्, तत् श्राद्धशब्दस्य द्रव्ये लक्षणा-\\
मभिप्रेत्य । एवं च कर्मनामत्वे सिद्धे -\\
श्रद्धया }{दीयते}{ यस्मात्तेन श्राद्धं निगद्यते ।\\
इति बृहस्पतिवाक्यं श्राद्धशब्दस्य योगप्रदर्शनार्थम् । तेनान्यत्र\\
प्रयोगाभावाद्योगवशाच्च योगरूढोऽयं श्राद्धशब्द इति । यद्यपि च-\\
तस्माच्छ्रद्धां समासाद्य धर्मे धर्मात्समाचरेत् ।\\
इत्यादिविष्णुधर्मोत्तरादिवाक्यैः श्रद्धायाः सर्वकामार्थता तथा\\
"श्रद्धान्वित• श्राद्ध कुर्वीत " इति कात्यायनेन विशिष्य श्राद्धे
श्रद्धायाः\\
पुनरङ्गत्वोक्तेः श्रद्धाविशेषोऽत्राङ्गमिति बोध्यम् । अत एव
नन्दिपुराणे-\\
श्रद्धा माता तु भूतानां श्रद्धा श्राद्धेषु शस्यते ।\\
इति तस्यास्तत्र प्राशस्त्यमुक्तम् । एवं च सकलस्मृत्याद्येक.\\
वाक्यतया श्राद्धस्य त्यागरूपत्वे सिद्धे यत्कैश्चिद्ब्राह्मणभोजनस्य\\
श्राद्धपदार्थत्वमुक्तम् तद्विचारणीयम् । न च-\\
पितॄन्पितामहान्यक्षे भोजनेन यथाक्रमम् ।\\
प्रपितामहान्सर्वोश्च तत्पितृश्चानुपूर्वशः \textbar{}\textbar{}\\
इति ब्रह्माण्डपुराणे भोजनश्रवणात्तस्यैव श्राद्धत्वमिति वाच्यम् ।\\
भोजनपदस्य कर्मव्युत्पत्त्या -\/-\\
प्रेतान्पितृनप्युद्दिश्य मोज्यं यत्प्रियमात्मनः ।\\
श्रद्धया }{दीयते}{ यत्तु तच्छ्राद्धं परिकीर्तितम् ॥\\
इति मरीच्यादिवचनैकवाक्यतया भोज्यपरत्वात् । अन्यथा\\
प्रपितामहपितॄनुद्दिश्य भोजनाभावेन तत्पितॄंश्चानुपूर्वश इत्यस्या-\\
सङ्गतिः स्यादिति । अस्मिन्मते तेषां लपभागित्वेन त्यागोद्देश्य\\
त्वाच्च विरोधः ।

{८ }{वीरमित्रोदयस्य श्राद्धप्रकाशे-}{\\
अथ श्राद्धस्य यागदानरूपताविचारः ।\\
तत्र "य एवं विद्वान्पितृन्यजते" इति यजिप्रयोगदर्शनाद्यागल.\\
क्षणसत्वाच्च यागरूपता । न च "पितृभ्यो दद्यात्" इतिप्रयोगदर्श.\\
नाहानरूपताप्यस्येति शूलपाण्युक्तं युक्तम् । श्राद्धस्य
परस्वत्वजनकत्वा-\\
भावेन तल्लक्षणानाक्रान्तत्वात् । तथाहि । न तावत्तस्य पित्रादिस्वत्वो.\\
त्पादकत्वम् । तेषां ममेदमितिस्वीकारासम्भवात् । पित्रादीनां\\
भोक्तृत्वोक्तिस्तु शूलपाणेर्देवताधिकरणविरुद्धैव । वेदान्तमतरीत्या\\
पित्रादीनां भोक्तृत्वस्वीकारेऽपि "नह वै देवा अइनन्ति न पिवन्ति\\
एतदेवामृतं दृष्ट्वा तृप्यन्ति" इत्यादिपर्यालोचनया स्वत्वस्वीकार\\
स्त्वप्रमाणक एव । कथमन्यथा वेदान्तमते यागदानयोर्भेदः । नापि\\
ब्राह्मणस्वत्वोत्पादकत्वं श्राद्धस्य । तेषामनुद्देश्यत्वात् ।
ब्राह्मणस्वत्वं तु\\
त्यक्तद्रव्यप्रतिपत्या परमुत्पद्यतां यथा देवोद्देशेन त्यक्तहिरण्यादेः\\
ब्राह्मणे प्रतिपादनादेव ब्राह्मणस्वत्वं न तु देवोद्देश्यकत्यागात् ।
ददा-\\
तिप्रयोगस्तु लाक्षणिको व्याख्येयः । अन्ये तु "पितृवदुपवेश्य " इ.\\
त्याश्वलायनसूत्रेण ब्राह्मणे पित्रभेदबुद्धेरपि श्राद्धाङ्गत्व
प्रतिपादनात्पि\\
तुरुद्देश्यकत्यागेनापि ब्राह्मणगतस्वत्वोत्पत्तेः सुवचत्वात् ददातिर.\\
पि मुख्य एव । अतश्च यागदानरूपतेत्याहुः ।\\
अथ पित्रादीना प्रत्येकं देवतात्वनिर्णयः ।\\
तत्र तावत् "असावेतत्त इति यजमानस्य पित्रे, असावेतत्त इति\\
पितामहायासावेतत्त इति प्रपितामहाय " इति श्रुत्या, "नमो विश्वेभ्यो\\
देवेभ्य इत्यन्नमादौ प्राङ्मुखयोर्निवेद्य पित्रे पितामहाय प्रपितामहाय\\
नामगोत्राभ्यामुदङ्मुखेषु" इत्यादिस्मृत्या च प्रत्येकं देवतात्वावगतेः\\
"अश श्राद्धममावास्यायां पितृभ्यो दद्यात् " इति गौतमोक्तेः
"पितृभ्यो\\
दद्याद्राह्मणान् सन्निपात्य" इति वसिष्ठोकेश्च सहितानां तदवगतेस्तु\\
ल्यबलत्वाद्विकल्प इति केचित \textbar{} वस्तुतस्तु प्रत्येकमेव
देवतात्व.\\
मुक्तवचनात् न व्यासक्तम् । "पितृभ्यो दद्यात्'' इत्यस्य पित्रादीनां
प्रत्ये\\
कं क्रियान्वयेऽपि 'गर्गा भोज्यन्तां' 'ऋत्विग्भ्यो दक्षिणां ददाति इ\\
त्यादिवदुपपत्तेः । न च गर्गादीनामुद्देश्यत्वात्तद्गत साहित्यविवक्षा\\
भावेन युक्तः प्रत्येकं क्रियान्वयः । प्रकृते तु देवताया
उपादेयत्वात्सा-\\
हित्यं विवक्षितमेवेतिवाच्यम् । 'पितृभ्यो दद्यात्' इत्यत्र पितृपदे दे.\\
व्रताविधायित्वस्य तवाप्यसम्मतत्वेन तस्यानुवादत्वात् । अनुवाद्यवि.\\
•

{ }{पित्रादीनां प्रत्येकं देवतात्वनिर्णयः । ९}{\\
शेषणस्य च साहित्यस्याविवक्षितत्वात् इति बहवः । विश्वेषां देवानां\\
तु मिलितानामेव देवतात्वमुक्तवचनात् ।\\
एतद्वो अन्नमित्युक्त्वा विश्वान्देवांश्च संयजेत् ।\\
इति ब्रह्मपुराणाञ्च तथैव प्रतीतेरिति । तदपि देवतात्वं न सपक्षीका•\\
नाम् । पित्रे पितामहाय प्रपितामहायेत्यादिदेवताविधिषु तदश्रवणात्\\
निरपेक्षश्रुतिबाधप्रसङ्गाच्च । न च-\\
सपिण्डीकरणादूर्ध्व यत्पितृभ्यः प्रदीयते ।\\
सर्वत्रांशहरा माता इति धर्मेषु निश्चयः ॥\\
इति शातातपवचने मातुरंशहरत्वोक्तेस्तस्या अपि देवतात्वमिति\\
वाच्यम् । अत्र हि यजमानेन पिनुद्देशेन त्यक्तं यद्द्रव्यं तदंशहरत्वं
मा.\\
तुः प्रतिपाद्यते नतु यजमानद्रव्यांशहत्वम् । तेन यजमानेन पित्रु.\\
हेशेन त्यक्तस्य द्रव्यांशस्यांशं कामं माता पितुः सकाशाद् गृह्णातु न\\
त्वेतावता याजमाने द्रव्यत्यागे तस्या देवतात्वासिद्धिः । अत एव\\
त्रयस्त्रिंशतामग्निमुद्दिश्य त्यक्तेन द्रव्येण माद्यतामपि न देवतात्वमि\\
त्युक्तम्- "त्रिंशच्च परार्थत्वात् " (अ० ३ पा० २ अधि० १५ सु० ३६)\\
इत्यत्र \textbar{} नच-\\
अन्वष्टकासु वृद्धौ च गयायां च क्षयेऽहनि ।\\
मातुः श्राद्धं पृथक्कुर्यादन्यत्र पतिना सह ॥\\
इति शातातपेन पतिना सहेत्यभिधानात्सपत्नीकाना देवतात्वमिति\\
वाच्यम् । तथा सति "सहयुकेऽप्रधाने" (२।३।१९) इति अप्रधान.\\
विहिततृतीयया प्रत्युत्तरप्राधान्यापत्तेः सभर्तृके मातरित्यादिप्रयो.\\
गप्रसङ्गात् । नचैतदिष्टम् । वचनं तु -\\
स्वेन भर्त्रा सह श्राद्धं माता भुके स्वधामयम् ।\\
पितामही च स्वेनैव तथैव प्रपितामही \textbar{}\textbar{}\\
इति भोग एव मातुः पितृसहभावबोधकबृहस्पतिवचनैकवाक्य\\
तथा भोगे साहित्यमात्रपरं व्याख्येयम् । मातुरितिसम्बन्धसामान्या-\\
भिधायिन्याः षष्ठया भोगत्वरूपसम्बन्धविशेषपरतयोपपत्तेरिति ।\\
न योषिदृभ्य. पृथक् दद्यादवसानदिनाहते।\\
स्वभर्तृपिण्डमात्राभ्यस्तृप्तिरासां यतः स्मृता ॥\\
इति छन्दोगपरिशिष्टेन केवलभर्तृसम्बन्धिपिण्डभागस्यैव पत्नीतृ\\
प्तिहेतुत्वेनाभिधानाच्च न सपत्नीकानां देवतात्वमिति कल्पतरुप्रभृतयः
\textbar{}\\
२ वी० मी०

{ }{१० वीरमित्रोदयस्य श्राद्धप्रकाशे-}{\\
हेमाद्रिंप्रभृतयो दाक्षिणाश्यास्तु "न योषिदूभ्यः पृथक् दद्यात्"
इत्यत्र\\
पृथग्दानस्य लोकतोऽप्राप्तेर्निषेधे च विकल्पापत्तर्न पृथग्दद्यादिति\\
निषेधायोगादपृथग्दद्यादेकस्यामेव श्राद्धव्यक्तौ पितॄन् तद्योषितश्च\\
देवतात्वेनोद्दिशेदिति विधीयते । अतश्च सपत्नीकानां देवतात्व.\\
सिद्धिः । यद्यपि च पतिना सहेति तृतीयाबलात्पत्युरुपसर्जनत्वं प्रती\\
यते }{तथापि}{ बहुभिः प्रमाणैः पत्न्या एवोपसर्जनत्वप्रतीतस्तन्नात्र\\
विवक्षितमित्याहुः । एवं चास्मत्पितर्यज्ञदत्तशर्मन् अमुकगोत्र स\\
पत्नीकेत्यादि प्रयोगवाक्यमनुसन्धेयम् । वस्तुतस्तु -\\
सम्बन्धनामगोत्राणि यथावत्परिकर्तियेत् । इति,\\
नामगोत्रं पितॄणां हि प्रापकं इव्यकव्ययोः ।\\
इत्यादिवचनैस्तत्पत्नीनामपि गोत्रादि कीर्त्तनीयमेव । तथ\\
चास्मत्पितयज्ञदत्त शर्मन् अमुकगोत्रया सावित्रीनामिकया सहितै\\
तत्तुभ्यमित्यादि प्रयोगोऽनुसन्धेयः । एवं च बहुपत्नीकस्याप्यस्मि\\
न्पक्षे अमुकामुकनामिकाभिः सहितैतत्तुभ्यमित्यादिप्रयोगोऽपि ज्ञेयः ।\\
"तृप्तिरासाम्" इत्यनेन सर्वासामेव भोगश्रवणात् । न च "माता\\
भुङ्क्ते" इत्यनेन जनन्या एव भोगोऽभिधीयते न सर्वासामिति\\
वाच्यम् ।\\
(१) बह्नीनामेकपत्नीनामेका चेत्पुत्रिणी भवेत् ।\\
सर्वास्तास्तेन पुत्रेण प्राह पुत्रवतीर्मनुः ॥ अ० श्लो०१८३)\\
इति मनुवचनेन सपत्नीपुत्रेण सपत्न्यन्तरस्य पुत्रवतीत्वातिदेशे\\
सिद्धे मातृ कार्यातिदेशफलकस्य मातृत्वातिदेशस्याप्यर्थात्सिद्धौ पुत्र-\\
दत्ते पितृपिण्डे जनन्या इव विमातुर्भोगसिद्धेः । न चैतस्य वचनस्य -\\
पितृव्यभ्रातृमातृणामपुत्राणां तथैव च ।\\
मातामहस्यापुत्रस्य श्राद्धादि पितृवद्भवेत् ॥\\
इति हेमाद्रिधृतजातूकर्ण्यवचनैकवाक्यतयाऽपुत्रविमातर्येव पुत्र\\
वतीत्वातिदेशविध्यर्थकत्वमिति वाच्यम् । "सर्वाः पितृपत्म्यो मातर"\\
इति सुमन्तुवाक्ये सर्व पदेन सपुत्राया अपि मातृत्वातिदेशाज्जननी\\
वच्छ्राद्धसिद्धेरिति बहवः । अन्वष्टकायां सपत्नमातुरपि श्राद्धमभि-\\
दधानो नारायणवृत्तिकारोऽप्येवम् । हेमाद्रिस्तु सुमन्तुवाक्यस्य विवा\\
हप्रकरणस्थत्वेन तत्रैव सापिण्ड्यविधानार्थत्वात् "माता भुके" इत्यत्र

% \begin{center}\rule{0.5\linewidth}{0.5pt}\end{center}

{१ ) अत्र बहोनामित्यस्य स्थाने सर्वाद्यामित्यपि पाठ" ।\\


{ }{वस्वादीनां देवतात्वविचारः । ११}{\\


{च मातृपदस्य जनन्यामेव शक्तेः, ``एका चेत्पुत्रिणी" इत्यस्य च जातु\\
कर्ण्यवचनैकवाक्यतयाऽपुत्रविमातृविषयत्वात्सपुत्रायास्तस्याः प्राप\\
कवाक्याभावेन न कुत्रापि श्राद्धप्रसक्तिरस्तीत्याह । वस्तुतस्तु\\
"एका चेत्" इतिवचनस्यापि -\\
अपुत्रा ये मृताः केचित्त्रियो वा पुरुषाश्च ये ।\\
तेषामपि च देयं स्यादेकोद्दिष्टं न पार्वणम् ॥\\
इत्यापस्तम्बवचनेनापुत्राणामपि तासां पार्वणनिषेधात्र पार्वणवि\\
षयत्वं किन्तु आवश्यकसांवत्सरिकश्राद्धादिविषयत्वम् । पुन्नामक•\\
नरकत्राणरूपफलप्रतिपादनार्थत्वं चेति सर्वमनवद्यम् ।\\
अथ वस्वादीनां देवतात्वविचारः ।\\
ननु न जनकादीनां श्राद्धे देवतात्वं किन्तु वसुरुद्रादित्यादीनां\\
पित्रादिपदोपनीतानाम् । यथाह -\\
मनुः,\\
वसुन्वदन्ति हि पितॄन् रुद्रांश्चैव पितामहान् ।\\
प्रपितामहांस्तथादित्यान् श्रुतिरेषा सनातनी ॥ इति ।\\
याज्ञवल्क्यः -\/- ( अ० १ श्रा०प्र०३लो०२६९ )\\
वसुरुद्रादितिसुताः पितरः श्राद्धदेवताः ।\\
प्रीणयन्ति मनुष्याणां पितॄन् श्राद्धेन तर्पिताः ॥\\
हेमाद्रौ नन्दिपुराणे -\\
अग्निष्वात्ता ब्राह्मणानां पितरः परिकीर्त्तिताः ।\\
रात्रां बर्हिषदो नाम विशां काव्याः प्रकीर्तिताः ॥\\
सुकालिनस्तु शूद्राणां व्यामा म्लेच्छान्त्यजादिषु ।\\
तथा 

{ विष्णु: पितास्य जगतो दिव्यो यज्ञः स एव च ।\\
ब्रह्मा पितामहो ज्ञेयो ह्यहं च प्रपितामहः ॥\\
अहं = रुद्रः । महाभारते विष्णुवाक्यम्-\\
पिता पितामहश्चैव तथैव प्रपितामहः ।\\
अहमेवात्र विशेयस्त्रिषु पिण्डेषु संस्थितः ॥\\
विष्णुधर्मोत्तरे -\/-\\
पितुः पैतामहः पिण्डो वासुदेवः प्रकीर्त्तितः ।\\
पैतामहश्च निर्दिष्टस्तथा सङ्कर्षणः प्रभुः ॥

{१२ वीरमित्रोदयस्य श्राद्धप्रकाशे-}{\\
पितृपिण्डश्च विज्ञेयः प्रद्युम्नश्चापराजितः ।\\
आत्माऽनिरुद्धो विशेयः पिण्डनिर्वपणे बुधैः ॥\\
आत्मा = श्राद्धकर्त्ता ।\\
हेमाद्रौ स्मृतिः -\\
प्रथमो वरुणो ज्ञेयः प्राजापत्यस्तथापरः ।\\
तृतीयोऽग्निः स्मृतः पिण्ड एष पिण्डविधिः स्मृतः ॥\\
अक्रोधनाः शौचपराः सततं ब्रह्मचारिणः ।\\
न्यस्तशस्त्रा महाभागाः पितरः पूर्वदेवताः ॥\\
यस्मादुत्पत्तिरेतेषां सर्वेषामप्यशेषतः ।\\
ये }{च}{ यैरुपचर्या: स्युर्नियमैस्तान्निबोधत ॥\\
मनोर्हिरण्यगर्भस्य ये मरीच्यादयः सुताः ।\\
तेषामृषीणां सर्वेषां पुत्राः पितृगणाः स्मृताः ॥\\
विराट्सुताः सोमषदः साध्यानां पितरः स्मृताः ।\\
अग्निष्वात्ताश्च देवानां मरीच्या लोकविश्रुताः ॥\\
दैत्यदानवयक्षाणां गन्धर्वोरगरक्षसाम् ।\\
सुपर्णकिन्नराणां }{च}{ स्मृता बर्हिषदोऽत्रिजाः ।\\
सोमपा नाम विप्राणां क्षत्रियाणां हविर्भुजः ॥\\
वैश्यानामाज्यपा नाम शुद्राणां तु सुकालिनः ।\\
सोमपास्तु कवेः पुत्रा हविष्मन्तोऽङ्गिरः सुताः ।\\
पुलस्त्यस्याज्यपाः पुत्रा वसिष्ठस्य सुकालिनः ॥ इति ।\\
पूर्वदेवताः अनादिदेवताः । हिरण्यगर्भस्य तदपत्यस्य ।\\
मरीचिमत्र्यङ्गिरसौ पुलस्त्यं पुलहं क्रतुम् ।\\
प्रचेतसं वसिष्ठं }{च}{ भृगुं नारदमेवच ॥ ( अ० १ श्लो० ३५ )\\
इत्यनेन मनुना पूर्वमुक्ताः । विराट्सुताः हिरण्यगर्भपुत्रस्य मनुपि\\
तुर्विराजः सुताः । सोमषद सोमषन्नामानः । एते साध्यानां देवविशे-\\
षाणां पितरो भवन्ति । अग्निष्वात्ता एतन्नामान: मरीच्या मरीचि\\
पुत्रा देवानां पितर इत्यनुषङ्गः । एवमग्रेऽपि । श्राद्धप्रकरण एतेषा-\\
मुत्पस्यादिकथनमेतज्ज्ञानस्यापि श्राद्धाङ्गत्वज्ञापनार्थे ज्ञेयम् ।
तस्मा-\\
एकथं जनकादीनां देवतात्वमितिचेत्, अत्रोच्यते । "असावेतत्त इति\\
यजमानस्य पित्रेऽसावेतत्त इति पितामहायासावेतत्त इति प्रपिता-\\
महाय" इत्यादिश्रुतेः "एतत्तेऽसौ ये }{च}{ त्वामत्रान्विति तस्मै तस्मै य\\


{ }{पार्वणश्राद्धे देवतानिर्णयः । १३}{\\
एषां प्रेताः स्युः" इत्यादिकल्पसूत्रादिभ्यश्च जनकादीनां देवतात्वावग\\
मात् स्मृतिपुराणादौ वस्वादिपदप्रयोगस्तद्रूपतया ध्यानार्थः । तथा
}{च}{\\
पैठीनसिः,\\
य एवं विद्वान्पितृन्यजत इति ।\\
एवं = वस्वादिरूपतया ।\\
गोत्रनामभिरामन्त्र्य पितृनर्ध्य प्रदापयेत् । इति,\\
त्रियमाणे तु पितरि पूर्वेषामेव निर्वपेत् । इति,\\
यस्य पिता प्रेतः स्यात्स पित्रे पिण्डं निधाय पितामहात्परं\\
द्वाभ्यां दद्यात् ।\\
इति छन्दोगपरिशिष्ट - मनु - विष्णुस्मृतिषु पितॄणां }{नामगोत्रा.}{\\
दिभिरामन्त्रणादिविधानाद्वसुरुद्रादीनां}{ च}{ तदभावाज्जनकादीना\\
मेव देवतात्वम् \textbar{} केचित्तु यस्य पित्रादयस्त्रयोऽपि जीवन्ति
पातित्या.\\
दिना श्राद्धानां वा तत्कर्त्तृकतीर्थश्राद्धादौ देवता याज्ञवल्क्यादिव.\\
चनैर्विधीयन्त इत्याहुः ।\\
अथ पार्वणश्राद्धे देवतानिर्णयः ।\\
तत्र छन्दोगपरिशिष्टम् -\\
कर्षसमन्वितं मुक्त्वा तथाद्यं श्राद्धषोडशम् \textbar{}\\
प्रत्याब्दिकं }{च}{ शेषेषु पिण्डाः स्युः षडिति स्थितिः ॥ इति ।\\
कर्षूसमन्वितम् = अन्वष्टकाश्राद्धम् ।\\
कर्पूः खनेदिति गोभिलेन तत्र कर्षूविधानात् । कर्पूर्गर्ताः ।
कर्षूसमन्वित\\
सपिण्डीकरणम् । तत्रापि तत्सत्वादिति हेमाद्रि । तदयुक्तम् । श्राद्ध-\\
षोडशमित्यनेनैव तत्प्राप्तेः । यद्यपि }{च}{ स्मृत्यन्तरे सपिण्डकिरणस्य\\
षोडशश्राद्धेभ्यो भेदः स्मर्यते तथापि छन्दोगपरिशिष्टवाक्ये न तद्भेद
\textbar{}\\
द्वादश प्रतिमास्यानि आद्यषाण्मासिके }{तथापि}{ ।\\
सपिण्डीकरणं चेति प्रेतश्राद्धानि षोडश ॥\\
इतिछन्दोगपरिशिष्ट एव सपिण्डीकरणस्य षोडशश्राद्धान्तर्भावाभि.\\
धानात् । आद्यषाण्मासिके इत्यत्रैकादशाहे क्रियमाण श्राद्धमाद्यश-\\
ब्दार्थः । आद्यं श्राद्धषोडशमित्यत्र षोडशश्राद्धविशेषणमाद्यमिति ।\\
तेषां चाद्यत्वं सकलपार्वणेभ्य आदावनुष्ठेयत्वादिति परिशिष्टप्रका\\
शः । तथा च श्राद्धषोडशमिति हेमाद्रौ पाठः ।
सर्वथान्वष्टकादिव्यति.

{१४ वीरमित्रोदयस्य श्राद्धप्रकाशे-\\
रिक्तपार्वणेषु षट्दैवत्यता सिध्यति । अत एव मात्स्ये पिण्डान्वहार्यकं\\
प्रक्रम्य -\\
षट् }{च}{ तस्माद्धविःशेषात्पिण्डान्दत्त्वा तथोदकम् । इत्युक्तम् ।\\
विष्णुपुराणेऽपि -\/-\/-\\
पितृमातामहानां }{च}{ भोजयेच्चाप्युदङ्मुखान् । इति ।\\
बहुवचनं प्रभृत्यर्थम् । पितॄणां मातामहानां चेत्यर्थः ।\\
}{तथा}{ देवलः-\/-\\
एकेनापि हि विप्रेण ष‌ट्पण्डं श्राद्धमाचरेत् ।\\
षडयन्दापयेत्तत्र षड्भ्यो दद्यात्तथाशनम् ॥ इति ।\\
येषां तु गृह्येऽपि पित्रादित्रिदैवत्यमेव श्राद्धं विहित तैरपि माता-\\
महश्राद्धं कर्त्तव्यमेव ।\\
पितरो यत्र पूज्यन्ते तत्र मातामहा ध्रुवम् ।\\
अविशेषेण कर्त्तव्यं विशेषान्नरकं व्रजेत् ॥\\
इत्यकरणे धौम्यादिस्मृतौ दोषाभिधानात् ।\\
केचित्तु -\/-\\
बाहुल्यं वा स्वगृह्योकं यस्य यावत्प्रचोदितम् ।\\
तस्य तावति शास्त्रार्थे कृते सर्वः कृतो भवेत् ॥\\
इति छन्दोगपरिशिष्टवचनात्तैरपि पित्रादित्रयस्यैव श्राद्धं कार्यम् ।\\
धौम्यवाक्यं तु यच्छाखीयगृह्ये मातामहश्राद्धमुक्तं तद्गृयैकवाक्यतया\\
तच्छाखीयपरमित्याहु: । तन्न । }{तथा}{ सति धौम्यवाक्यस्यानुवादमात्र-\\
त्वेनानर्थक्यापत्तेः । वस्तुतस्तु तस्य तावतीत्युत्तरार्द्ध
तावन्मात्रकर\\
णस्यानुकल्पत्वं स्वरसतः प्रतीयते सर्वकरणस्य तु मुख्यत्वं तदपि\\
च न सर्वशाखोक्तपदार्थानुष्ठानेन तेषां तच्छाखीयं प्रत्येव प्रवृत्तेः
।\\
किन्तु शाखाविशेषमनधिकृत्य
प्रवृत्तपुराणस्मृत्याद्युक्तपदार्थानुष्ठानेन,\\
तेन मातामहश्राद्धमपीतरसाधारणपदार्थवत्साधारणमित्युपसंहर्त्त.\\
व्यम् ।\\
अन्ये तु मातामहश्राद्धं पुत्रिकापुत्रस्य घनग्राहिणो वा दौहित्र-\\
स्वावश्यकं नान्येषाम् । }{तथा}{ च-\\
बौधायनः,\\
पुत्रिकायां न यो जातो यो वा रिक्थं न विन्दति ।\\
न कुर्यादथवा कुर्यान्मातामहपुरः सरम् ॥ इति,\\
यमः-\/-\\
कुर्यान्मातामहश्राद्धं नियमात्पुत्रिकासुतः । इति,

 जीवत्पितृकश्राद्धे देवतानिर्णय १५

{ मनुः - - ( अ० ९श्लो० १३२)\\
दौहित्रो ह्यखिलं रिक्थमपुत्रस्य पितुर्हरेत् ।\\
स एव दद्याद् द्वौ पिण्डौ पित्रे मातामहाय }{च}{ ॥\\
लौगाक्षिरपि -\/-\\
श्राद्ध मातामहानां }{च}{ अवश्यं धनहारिणा ।\\
दौहित्रेण विधिज्ञेन कर्तव्य विधिवत्सदा \textbar{}\textbar{}\\
इत्याहुः ।\\
पुत्रिकापुत्रश्चाधिकारिनिर्णये वक्ष्यते । इति प्रकृतिपार्षणे देवता ।\\
एवं विकृतावपि विशेषवचनाभावे बोध्यम् ।\\
अथ जीवस्पितृकश्राद्धे देवता निर्णयः ।\\
विष्णु:-\/-\\
पितरि जीवति यः श्राद्धं कुर्याद्यस्य पिता प्रेतः स्यात्स पित्रे\\
पिण्डं निधाय पितामहात्पराभ्यां द्वाभ्यां दद्यात् । यस्य पिता पिता.\\
महश्च प्रेतौ स्यातां स ताभ्यां पिण्डं दत्त्वा पितामहपितामहाय\\
नद्यात् \textbar{} यस्य पितामहः प्रेतः स्यात्स तस्मै पिण्ड निधाय
प्रपिताम\\
हात्परं द्वाभ्यां दद्यात् । यस्य पिता प्रपितामहञ्च प्रेतौ स्यातां स\\
ताभ्यां पिण्डौ दत्त्वा पितामहप्रपितामहाय दद्यादिति ।\\
पितरि जीवति पितर्येव जीवतीत्यर्थः । एवमुत्तरत्र 'यः श्राद्धं कुर्यात्
'\\
इति यच्छब्दादकरणपक्षोऽपि सूचितः ।\\
यथा च- 

{ कात्यायनः,\\
सपितुः पितृकृत्येषु अधिकारो न विद्यते ।\\
पितुः पितृभ्यो वा दद्यात्सपितेत्यपरा श्रुतिः ॥ इति ।\\
हारीतोऽपि -\/-\\
जीवे पितरि वै पुत्रः श्राद्धकालं विवर्जयेत् ।\\
येषां वापि पिता दद्यात्तेषामेके प्रचक्षते ॥ इति ।\\
अन्तर्हतेभ्योऽनन्तर्हितेभ्यो वा प्रेतेभ्यस्त्रिभ्यः क्रमेण श्राद्धं
का-\\
र्यमिति विष्णुवचनसमुदायार्थः । मत्तान्तरमाह-\\
मनु:, - - ( अ० ३ श्लो० २२० \textbar{} २२१ \textbar{} २१२ )\\
ध्रियमाणे तु पितरि पूर्वेषामेव निर्वपेत् ।\\
विप्रवद्वापि तं श्राद्धे स्वकं पितरमाशयेत् ॥}{\\
-

{ }{१६ }{वीरमित्रोदयस्य}{ श्राद्धप्रकाशे-}{\\
पिता यस्य तु वृत्तः स्याज्जीवेच्चापि पितामहः ।\\
पितुः स नाम सङ्कीर्त्य कीर्त्तयेत्प्रपितामहम् ॥\\
पितामहो वा तच्छ्राद्धं भुञ्जीतेत्यब्रवीन्मनुः ।\\
कामं वा तदनुज्ञातः स्वयमेव समाचरेत् ॥ इति ।\\
ध्रियमाणे = जीवति । पूर्वेषां = पितुः पित्रादीनाम् । अथ वा
स्वपितरमा-\\
शयेत् = भोजयेत् । विप्रवत् = गन्धाद्यर्चन पूर्वकम् ।
विप्रविहितनिमन्त्रण\\
ह्मचर्यादिनियमयुक्तमिति मेधातिथिः । पितामहप्रपितामहस्थाने तु ब्रा.\\
ह्मणान्भोजयेत् । पितुर्भोजनपक्षे न पिण्डदानम् -\\
पितरं भोजयित्वा तु पिण्डौ निवृणुयात्परौ ।\\
इति यज्ञपार्श्वपरिशिष्टात् । अनेन स्वकं पितरमादितो भोजयि\\
स्वा तदूर्ध्वे पुरुषत्रयस्य श्राद्धं कर्त्तव्यमिति हलायुधव्याख्यानं
प्रश.\\
स्तम् । नाम सङ्कीर्त्यैति श्राद्धोपलक्षणार्थम् । प्रपितामहमिति
वृद्धप्र\\
पितामहस्याप्युपलक्षणम् ।\\
त्रयाणामुदकं कार्यं त्रिषु पिण्डः प्रवर्त्तते ।\\
चतुर्थः सम्प्रदातैषां पञ्चमो नोपपद्यते । (अ० श्लो०१८६ )\\
इति मनूक्ते । "यस्य पिता प्रेतः स्यात् इत्युदाहृत विष्णुवचनाच्च ।\\
काम वेति । तदनुज्ञात तेन जीवता पित्रा पितामहेन वा । उपलक्षणमेत\\
ज्जीवमात्रस्य \textbar{} कामम् = अभिलषितं समाचरेत् । तदयमर्थः । यो
जीवति\\
तदाज्ञाग्रहणमेव तच्छ्राद्धस्थानीयम् ।}{ तथा}{ }{च}{ पितरि जीवति पितु\\
राक्षां गृहीत्वा पितामहप्रपितामहयोरेव श्राद्धं कुर्यात् । पितामहे\\
जीवति पित्रे श्राद्धं दत्त्वा पितामहाज्ञां गृहीत्वा प्रपितामहायैव
श्राद्धं\\
दद्यादिति । एवमन्यत्र । एवं }{च}{ जीवन्तमतिक्रम्य दद्याज्जीवन्तं वा\\
भोजयेदाशां वा गृह्णीयादितिपक्षत्रयं सिद्धम् । तत्र भोजनपक्षः कलौ\\
निषिद्धः ।\\
प्रत्यक्षमर्चनं श्राद्धे निषिद्ध मनुरब्रवीत् ।\\
इति पृथ्वीचन्द्रोदये भविष्योक्तेः । चन्द्रिकायामप्येवम् । एवं जीव.\\
न्मातामहस्य \textbar{}\\
मातामहानामप्येवं श्राद्धं कुर्याद्विचक्षणः ।\\
इति जीवत्पितृकश्राद्धोक्तक्रमातिदेश कविष्णुवचनात् । पितरि जी-\\
वति स्वमातरि मातामहे }{च}{ मृते पितुः पितृभ्य इव पितुरेव मातृमा\\
तामह्योर्दातव्यम् ।

{ }{जीवत्पितृकश्राद्धे देवतानिर्णयः । १७}{\\
येभ्य एव पिता दद्यात्तेभ्यो दद्यात्स्वयं सुतः ।\\
इत्यविशेषविधानादिति दाक्षिणात्या । मदनरत्नादौ तु स्वमातृमाता\\
महेभ्य एव दद्यादित्युक्तम् \textbar{} तस्यायमाशयः । यत्र हि "न
जीवन्तमब्रि.\\
क्रस्य किञ्चिद्दद्यात्" इतिवचनेन जीवदतिक्रमदोषात् श्राद्धलोपप्रसक्ति\\
स्तत्रैव "येभ्य एव पिता दद्यात्" इत्यादिवचनैर्देवतान्तरं विधीयते ।\\
न }{च}{ जीवत्पितृकेण स्वमातृमातामहादिश्राद्धकरणे जीवदतिक्रमो\\
ऽस्ति, येन पितुर्मात्रादीनां देवतात्वं विधीयेत । "पितरो यत्र पूज्यन्ते\\
तत्र मातामहा ध्रुवम् । " इति मातामहश्राद्धविधायकवृद्धयाज्ञवल्क्य.\\
वचनेन श्राद्धकर्तृमातामहानामेव श्राद्धविधानाच न पितुर्मातामहानां\\
देवतात्वम् । एवं पितामहादिषु जीवत्स्वपि द्रष्टव्यम् । मृते पितरि\\
जीवन्मातृकः पितामह्यादिभ्यो दद्यादिति स्मृतितत्वादयः प्राच्याः ।\\
दाक्षिणात्यास्तु -\\
पितृवर्गे मातृवर्गे }{तथा}{ मातामहस्य }{च}{ \textbar{}\\
जीवेत्तु यदि वर्गाद्यस्तं वर्गे तु परित्यजेत् ॥\\
इति वचनात्तद्वर्गमेव त्यजेदित्याहुः ।\\
अत्र "जीवे पितरि " इत्यादिवचनाज्जीवत्पितृकस्य श्राद्धविकल्पे\\
प्राप्ते क्वचित्करणपक्षनियममाह -\\
कात्यायनः,\\
ब्राह्मणादिहते ताते पतिते सङ्गवर्जिते ।\\
व्युत्क्रमाच्च मृते देयं येभ्य एव ददात्यसौ ॥ इति ॥\\
अत्रादिपदेन चण्डालादिग्रहणम् । पतिते जीवत्यपि । स-\\
ङ्गवर्जिते = प्रव्रजिते जीवति । अत्र पतितो महापातकी ब्राह्मणादिह\\
तानां पृथगुपादानात् । व्युत्क्रम पितामहादौ जीवति पित्रादिमरणे ।\\
अयमर्थ:-\\
व्युत्क्रमाच्च प्रमीतानां नैव कार्या सपिण्डता ।\\
इति वचनेन व्युत्क्रममृतस्य सपिण्डीकरणनिषेधात्तस्य पितृ.\\
स्वाप्राप्तेः स जीवत्पितृकतुल्यत्वाद्येभ्यः पितामहादिभ्यो ददाति ते.\\
भ्यो दद्यादिति । वस्तुतस्तु "यस्य पिता प्रेतः स्यात्स पित्रे पिण्डं\\
निधाय" इत्यादिविष्णुवचने पितुरपि श्राद्धविधानानेडशी व्याख्या.\\
युक्ता, किन्तु ब्राह्मणहतादिसाहचर्यात् भ्युत्क्रमाच्छास्त्रमतिक्रम्यो.\\
करभेदादिनेति व्याख्येयम् ।\\
३ वी० मी०

{१८ }{ वीरमित्रोदयस्य श्राद्धप्रकाशे-}{\\
यदप्युक्त जीवति पितामहादौ पित्रादेः सपिण्डीकरणं ना\\
स्तीति, तदप्यसत् ।\\
मृते पितरि यस्याथ विद्यते }{च}{ पितामहः ।\\
तेन देयास्त्रयः पिण्डाः प्रपितामहपूर्वकाः \textbar{}\textbar{}\\
तेभ्यस्तु पैतृकः पिण्डो नियोक्तव्यश्च पूर्ववत् -\\
इति सपिण्डीकरण प्रक्रम्य ब्रह्मपुराणे तस्यापि सपिण्डीकरणोक्तेः ।\\
मैत्रायणीय परिशिष्टे त्वन्यत्राप्युक्तो नियमः \textbar{}\\
उद्वाहे पुत्रजनने पित्र्येष्ट्यां सौमिके मखे ।\\
तीर्थे ब्राह्मण आयाते षडेते जीवतः पितुः ॥\\
जीवतः पितुः पुत्रस्येति शेषः । अनादरेण वा षष्ठ्यर्थः । जीव\\
न्तमनादृत्य तदूर्ध्वपितृदेवत्यकर्मकाला इत्यर्थः । उद्वितीया.\\
दिविवाहे ।\\
नान्दीश्राद्धं पिता कुर्यादाद्ये पाणिग्रहे बुधः ।\\
अत ऊर्ध्व प्रकुर्वीत स्वयमेव तु नान्दिकम् ॥\\
इतिस्मृतेः । पुत्रजननं नामकरणादीनामुपलक्षणमिति निबन्धका-\\
राः । पित्र्येष्टिश्चातुर्मास्येषु साकमेधपर्वणि प्रसिद्धा । सौमिके मखे -
तृती.\\
यसवने सत्रेषु नाराशंसेषु पुरोडाशशकलपिण्डदाने \textbar{} तीर्थे =
तत्प्रा.\\
तौ । ब्राह्मण आयाते तत्सम्पत्तौ । "द्रव्य ब्राह्मणसम्पत्तिः" इति
याज्ञव.\\
ल्क्येन तत्सम्पत्तेरपि श्राद्धकालतोकेः ।\\
स्मृत्यन्तरमपि,\\
वृद्धौ तीर्थे}{ च }{संन्यस्ते ताते }{च}{ पतिते सति ।\\
येभ्य एव पिता दद्यात्तेभ्यो दद्यात्स्वयं सुतः ॥ इति ।\\
यत्तु सुमन्तुनोक्तम्-\/-\\
न जीवत्पितृकः कुर्याच्छ्राद्धमझिमृते द्विजः ।\\
येभ्य एव पिता दद्यात्तेभ्यः कुर्वीत साग्निकः ॥\\
इति जीवत्पितृकस्य श्राद्धं नास्तीति तद्वृद्धिश्राद्धेतरविषयम् ।\\
अनग्निकोऽपि कुर्वीत जन्मादौ वृद्धिकर्मणि ।\\
येभ्य एव पिता दद्यात्तानेवोद्दिश्य पार्वणम् ।\\
इतिहारीते तस्यापि वृद्धिश्राद्धोक्तेः । पार्वणं तद्धर्मकम् । जीवत्पि\\
तृक इति श्राद्धे देवतात्वेनाधिकारी जीवत्पिता यस्य स इत्यर्थः । तेन\\
संन्यस्तपतितपितृकस्य निरग्रेरपि श्राद्धाधिकारसिद्धिः । कचित्कर

 द्विपितृकश्राद्धनिर्णयः । १९

{णपक्षे नियममाह -\\
लौगाक्षिः,\\
दर्शश्राद्धं गयाश्राद्ध श्राद्धं चापरपक्षिकम् ।\\
न जीवत्पितृकः कुर्यात्तिलैः कृष्णैश्च तर्पणम् ॥ इति ।\\
ऋतुरपि,\\
अष्टकादिषु सङ्क्रान्तौ मन्वादिषु युगादिषु ।\\
चन्द्रसूर्यग्रहे पाते स्वेच्छया राजयोगतः ॥\\
जीवत्पिता नैव कुर्याच्छाद्धं काम्यं तथाखिलम् । इति ।\\
अथ द्विपितृकश्राद्धनिर्णयः ।\\
तत्र द्विपितृकमाह - बौधायनः,\\
मृतस्य }{च}{ प्रसुतो यः क्लीबस्य व्याधितस्य}{ च}{ ।\\
अन्येनानुमते वा स्यात्स्वक्षेत्रे क्षेत्रजः सुतः ॥\\
स एव द्विपिता द्विगोत्रश्च, मृतादीनां क्षेत्रेषु अन्येन यः प्रसूतः\\
स क्षेत्रज इत्यर्थः । अत्रविशेषमाह -\\
हारीत:.\\
जीवति क्षेत्रजमाहुरस्वातन्त्र्यान्मृते यामुष्यायणमगुप्तबीजत्वा.\\
दिति । जीवत्यजीवत्यपि क्रियाभ्युपगमात् द्विपितृको भवतीत्याह -\\
मनु,\\
(१) क्रियाभ्युपगमाश्वेवं बीजार्थं यत्प्रदीयते ।\\
तस्येह भागिनौ दृष्टो बांजी क्षेत्रिक एव}{ च}{ ॥ (अ०९. लो५३ )\\
अपुत्राभ्यां बीजिक्षेत्रिभ्यां मम क्षेत्र तव बीजम् इत्येवंनियोगेनो-\\
त्पादित उभयो. पुत्रो भवतीत्यर्थः । तत्र -\\
साङ्ख्यायनः -\/- उभावेकस्मिन्पिण्डे पितृभेद इति ।\\
उभौ-बीजिक्षेत्रिणौ । एकस्मिन्पिण्डे सङ्कीर्त्तयेदित्यर्थः-\\
हारीतः,\\
नाबीजं क्षेत्रं फलति नाक्षेत्र बीजं रोहति उभयदर्शनात् उभयो-\\
रपत्यमित्यपरे तेषामुत्पादयितुः प्रथमः प्रवरो भवति द्वौ द्वौ पिण्डौ\\
निर्वापे दद्युरेकपिण्डे वा द्वावनुकीर्तयेत् द्वितीये पुत्रस्तृतीये पौत्र
इति ।\\
उभयोरपत्यमिति क्रियानियमे बोध्यम् । तेषा=बीजिक्षेत्रिणां पितॄणां\\
मध्ये उत्पादयितुः बीजिनः प्रथमः प्रवरो भवति ततः क्षेत्रिणः । तदेत.

% \begin{center}\rule{0.5\linewidth}{0.5pt}\end{center}

( १ ) क्रियाभ्युपगमात्वेतदिति पाठान्तरम् ।

{ }{२० }{वीरमित्रोदयस्य}{ श्राद्धप्रकाशे -}{\\
स्क्रत्वन्तर्गतार्षेयप्रवरवरणे बोध्यम् । निर्वापे= पितृयज्ञे । एकापण्डे
वेत्यत्र\\
एकैकस्मिन्निति वीप्सा द्रष्टव्या । "यदि द्विपिता स्यादेकैकस्मिन्द्वौ
द्वावु\\
पलक्षयेत्" इत्यापस्तम्बवचनानुसारात् । तस्मादेकैकस्मिन्नेव पिण्डे\\
पितरौ पितामहौ प्रपितामहौ वानुकीर्त्तनीयावित्यर्थः । द्वितीये = पिता.\\
महपिण्डे । द्वितीय इति }{च}{ प्रपितामहस्याप्युपलक्षणम् । पुत्रो द्यामु\\
व्यायणस्य । द्वौ द्वौ निर्वापे दद्यात् द्वौ द्वावनुकीर्त्तयेद्वेत्यर्थ
\textbar{} तृतीये=\\
प्रपितामहपिण्डे । पौत्रो ह्यामुष्यायणस्य \textbar{}\\
अथ द्यामुष्यायणस्य पित्रादयोऽपि यदि यामुष्यायणास्तत्र\\
विशेषः कथ्यते । तत्र यदि ह्यामुष्यायणस्य पितामह {[}एक{]} एव\\
ह्यामुष्यायणस्तदा पितृभ्यां पिण्डद्वयं दत्वा पितामहाभ्यामपि\\
पिण्डद्वयं दत्त्वा प्रपितामहेभ्यः पिण्डत्रयं दद्यादित्येषं सप्त पि.\\
ण्डान्दद्यात् । यदा तु ह्यामुष्यायणस्यान्यतरः पिता द्यामुष्या.\\
यणो भवेत्तदा पितृभ्यां पिण्डद्वयं प्रदाय पितामहेभ्य. पिण्ड.\\
त्रयं दत्त्वा प्रपितामहेभ्यस्त्रयमित्यष्टौ पिण्डान्दद्यात् । यदा तु
तस्या.\\
न्यतरः पिता ह्यामुष्यायणः अन्यतमः पितामहोऽपि ह्यामुष्यायणः\\
तदा पितृभ्यां पिण्डद्वयं निरूप्य पितामहेभ्यस्त्रीन्पिण्डान्दत्त्वा
प्रपि-\\
तामहेभ्यश्चतुरो दद्यादित्येवं नव । यदा तु यामुष्यायणस्य द्वावपि\\
पितरौ ह्यामुष्यायणौ स्यातां तदा पितृभ्यां द्वौ पिण्डौ दत्त्वा पितामहे\\
भ्यश्चतुरो दत्वा प्रपितामहेभ्योऽपि चतुरो दद्यादित्येवं दश । यदा तु\\
द्यामुष्यायणस्य द्वावपि पितरौ यामुष्यायणौ पितामहस्त्वक एव या.\\
मुष्यायणः तदा पितृभ्या द्वौ पिण्डौ प्रदाय पितामहेभ्यश्चतुरो दत्वा\\
प्रपितामहेभ्यः पञ्च दद्यादित्येवमेकादश । यदा तु ह्यामुष्यायणस्य\\
द्वावपि पितरौ ह्यामुष्यायणौ द्वावपि पितामही ह्यामुष्यायणौ भवतः\\
तदा पितृभ्यां पिण्डद्वयं पितामहेभ्यश्चतुरः पिण्डान्दत्वा षट् प्रति.\\
महेभ्यः षड्भ्यो दद्यादित्येवं द्वादश । यदा तु ह्यामुष्यायणस्य द्वावपि\\
पितरौ ह्यामुष्यायणौ पितामहाश्च त्रयो ह्यामुष्यायणाः तदा पितृभ्यां\\
पिण्डद्वयं पितामहेभ्यश्चतुरः पिण्डान्दत्त्वा प्रपितामहेभ्यः सप्तभ्यः
सप्त\\
पिण्डान्दद्यादित्येवं त्रयोदश । यदा च ह्यामुष्यायणस्य द्वावपि
पितरौ\\
यामुष्यायणौ पितामहाञ्चत्वारोऽपि ध्यामुष्यायणास्तदा पितृभ्यां\\
पिण्डद्वयं पितामहेभ्यश्चतुर्भ्यश्चतुरः पिण्डान्दत्वा
प्रतितामहेभ्योऽष्ट.

% \begin{center}\rule{0.5\linewidth}{0.5pt}\end{center}

[ ] एतत्कोष्ठान्तर्गत• पाठोमूलपुस्तके नास्ति अस्माभिरेव पर्यालोच्य
सन्निवेशि : \textbar{}}

{ }{ पुत्रिकापुत्रकर्तृकश्राद्धे देवतानिर्णयः । २१}{\\
भ्योऽष्ट पिण्डान्दद्यादित्येवं चतुर्दशेम्याह्यम् । एकैकस्मिन्पिण्डे
द्वौ\\
द्वावनुकीर्त्तयेदित्येतस्मिन्पक्षे तु सर्वेषां त्रय एव पिण्डा भवन्ति
नाम.\\
कीर्त्तनन्तु यथोक्तक्रमेणेति ।\\
याज्ञवल्क्यः ।\\
अपुत्रेण परक्षेत्रे नियोगोत्पादितः सुतः ।\\
उभयोरप्यसौ ऋक्थी पिण्डदाता }{च}{ धर्मतः ॥\\
( याज्ञ० अ० २ दायविभागप्रकरणे लो० १२७ )

{नारदः,\\
}{ह्यामुष्यायणका दद्युर्द्वाभ्यां पिण्डादके पृथक्\\
ऋक्थादर्द्धाशमादधुर्बीजिक्षेत्रिकयोस्तथा ॥\\
अर्द्धांशमिति पूर्णाशयोग्यपुत्रान्तरसद्भावे ।}{\\
मरीचिः,\\
सगोत्रादन्यगोत्राद्वा योभवेद्विधवासुतः ।\\
पिण्डं श्राद्धविधानं }{च}{ क्षेत्रिणे प्राक् प्रदापयेत् ॥\\
बीजिने तु तत् पश्चात् क्षेत्री जीवति चेत्क्कचित् ।\\
बीजिने दधुरादौ तु मृते पश्चात्प्रदीयते \textbar{}\textbar{}\\
उभौ यदि मृतौ स्यातां बीजिन्यादौ तदा ददेत् ।\\
क्षेत्रिण्यादौ न दत्तं स्याद्वीजिने नोपतिष्ठति ॥\\
सगोत्रादन्यगोत्राद्वेति सवर्णमात्रादित्यर्थ इति कल्पतरु । श्राद्धविधा\\
नमिति । श्राद्धे देयत्वेन विधानं यस्य \textbar{} मृते= क्षेत्रिणि मृते ।
पश्चात्=\\
क्षेत्रिणे दानानन्तरम् \textbar{} प्रदीयत इत्यस्य बीजिन इति शेषः,
उभयोर्मरणे\\
पूर्व बीजिने पश्चात्क्षेत्रिणे दाने निन्दामाह - उभाविति । अनियोगजाते\\
व्यवस्थामाह-\\
नारदः,\\
जाता ये त्वनियुक्तायामेकेन बहुभिस्तथा ।\\
अॠमाजस्ते सर्वे बीजिनामेव ते सुताः ॥\\
दस्ते बीजिने पिण्डं माता चेच्छुल्कतो हृता ।\\
अशुल्कोपहृतायां तु पिण्डदा वोदुरेव ते ।\\
अऋक्थभाजः = क्षेत्रिण ऋक्थं न भजन्ते इति ।\\
अथ पुत्रिकापुत्रकर्तृकश्राद्धे देवतानिर्णयः ॥\\
सत्र पुत्रिकापुत्रञ्चतुर्विधः । पुत्रिकै पुत्रत्वेन गृहीता एकः पुत्रि-\\
कापुत्रः, द्वितीयस्तस्याः पुत्रः तृतीयस्तु दुहितर्येवोत्पन्नः संविदा

{२२ }{वीरमित्रोदयस्य श्राद्धप्रकाशे-}{\\
पुत्रत्वेन परिगृहीतो मातामहमात्रसम्बद्धः, चतुर्थस्तु तादृश एव\\
जनकमातामहोभयसम्बद्धः । एतद्विस्तरेणाधिकारिनिर्णये प्रपञ्चयि\\
व्यते । तत्राद्ये त्वसन्देह एव । द्वितीये
बौधायनयज्ञपार्श्वाभ्यामुक्तम्-\\
आदिशेत्प्रथमं पिण्ड मातरं पुत्रिकासुत \textbar{}\\
द्वितीयं पितरं तस्यास्तृतीये तु पितामहम् ॥\\
यद्यप्येतत्सकलपुत्रिकापुत्रविषयं प्रतिभाति । }{तथापि}{ द्वितीयस्यैव\\
भवति तस्य मातुः पितृस्थानीयत्वादिति हेमाद्रिः ।
मदनरत्नादयस्त्वाद्यव्य.\\
तिरिक्तानां मातृपूर्वकत्वमित्याहुः । उभयसम्बद्धेन मातामहश्राद्ध.\\
मादौ कार्यमित्याह-\/-\\
ऋष्यशृङ्गः,\\
तस्मादुभयसम्बद्धः पुत्रिकायाः सुतो ह्यसौ ।\\
पूर्व मातामहश्राद्धं पश्चात्पैतृकमाचरेत् ॥ इति ॥\\
अथ वैश्वदेविकश्राद्धदेवतानिरूपणम् ।\\
तत्र विश्वेषां देवानामुत्पत्तिर्ब्रह्मवैवर्त ब्रह्माण्डपुराणयोः-\/-\\
दक्षस्य दुहिता साक्षादिश्वा नामेति विश्रुता ।\\
विधिना सा तु धर्मश ? दत्ता धर्माय धीमते ।\\
तस्या. पुत्रा महात्मानो विश्वे देवा इति श्रुताः ।\\
विख्यातास्त्रिषु लोकेषु सर्वलोकनमस्कृताः ॥\\
ग्रमसहितायाम्,\\
विश्वेऽपि विश्वे देवास्तु दक्षिणे वाणपाणयः ।\\
द्विहस्ता वामभागे तु शरासनपरायणाः ।\\
एतेषां श्राद्धदेवतात्वे इतिहासस्तु ब्रह्मवैवर्त्तब्रह्माण्डयोः ।\\
समा नव महात्मानश्चेरुरुमं महत्तपः ।\\
हिमवच्छिखरे रम्ये देवर्षिगणसेविते \textbar{}\textbar{}\\
शुद्धेन मनसा प्रीताः पितरस्तानथाब्रुवन् ।\\
वरं वृणीध्वं प्रीताः स्मः कं कामं करवामहे ॥\\
ब्रह्मा चाह महातेजास्तपसा तैस्तु तर्पितः ।\\
प्रीतोऽस्मि तपसानेन कं कामं वितरामि वः ॥\\
एवमुक्तास्तदा विश्वे ब्रह्मणा विश्वकर्मणा ।\\
ऊचुस्ते सहिता सर्वे ब्रह्माणं लोकपावनम् \textbar{}\textbar{}\\


{ }{वैश्वदेविक श्राद्धदेवतानिरूपणम् । २३}{\\
श्राद्धेऽस्माकं भवेद्देयं स ह्येषः काङ्क्षितो वरः ।\\
प्रत्युवाच ततो ब्रह्मा तान्वै त्रिदशपूजितः \textbar{}\textbar{}\\
भविष्यत्येवमेवेति काङ्क्षितो वो वरोऽस्तु यः ।\\
पितृभिश्च तथेत्युक्तमेवमेतन्न संशयः ॥\\
सहास्माभिस्तु भोक्तव्यं यत्किञ्चित्यज्यते त्विह ।\\
अस्माकं कल्पिते श्राद्धे भवन्तोऽग्राशिनो हि वै ॥ 

{ भविष्यथ मनुष्येषु सत्यमेतदुदाहृतम् ।\\
माल्यैर्गन्धैस्तथान्नेन युष्मानप्यर्चयन्तु वै ॥\\
दत्ते दत्तेऽथ युष्मभ्यमस्मभ्यं दास्यते ततः ।\\
विसर्जनमथास्माकं पूर्व पश्चात्तु दैवतम् ॥ इति ॥\\
तेषां सविनियोगानि नामान्याह -\\
बृहस्पतिः,\\
क्रतुर्दक्षो वसुः सत्यः कामः कालस्तथैव }{च}{ ।\\
धुरिश्च रोचनश्चैव }{तथा}{ चैव पुरूरवाः ॥\\
आर्द्रवश्च दशैते तु विश्वे देवाः प्रकीर्त्तिताः ।\\
इष्टिश्राद्धे क्रतुर्दक्षः सत्यो नान्दीमुखे वसुः \textbar{}\textbar{}\\
नैमित्तिके कामकालौ काम्ये }{च}{ धुरिलोचनौ ।\\
पुरुरवा आर्द्रवश्च पार्वणे समुदाहृतौ ॥\\
उत्पत्ति नाम चैतेषां न विदुर्ये द्विजातयः ।\\
अयमुष्वारणीयस्तैः श्लोकः श्रद्धासमन्वितैः ॥\\
आगच्छन्तु महाभागा विश्वेदेवा महाबलाः ।\\
ये यत्र विहिता श्राद्धे सावधाना भवन्तु ते ॥ इति ॥\\
अत्र पुरुरवः शब्दः सान्तः । आर्द्रवशब्दस्त्वाकारादिरकारान्तः,\\
पुरूरवसमावमिति शङ्खलिखितोक्ततर्पणविधौ }{तथा}{ दर्शनात् बहुषु\\
निबन्धेषु }{तथा}{ पाठदर्शनाच्चेति श्रीदत्तादयः । कामधेनौ तु माकारादिः\\
सकारान्तो माद्रवः शब्दो लिखितः । गौडनिबन्धेषु तु पुरुरवः शब्द\\
उकारादियुक्ताद्यरेफवान् सकारान्तो दृश्यते क्वचित्तु अकारान्त एव\\
पुरुरवशब्दोडश्यते । इष्टिश्राद्धं "कर्माङ्गं नवमे प्रोक्त" मित्यनेन
विश्वामि-\\
त्रेणोक्तम् \textbar{} तच्च -\\
निषेककाले सोमे }{च}{ सीमन्तोन्नयने }{तथा}{ ।\\
श्चेयं पुंसवने चैव श्राद्धं कर्माङ्गमेवच ॥

{२४ }{वीरमित्रोदयस्य श्राद्धप्रकाशे-}{\\
इत्यादिनाभविष्यपुराणादिवाक्येन विवृतम् । नान्दीमुखन्तु वृद्धिश्रा

{द्धम् । तच्च-\\
वृद्धौ यत्क्रियते श्राद्धं वृद्धिश्राद्धं तदुच्यते ।\\
इति भविष्यपुराणेनोक्तम् \textbar{} वृद्धि: : = पुत्रजन्मादि \textbar{}
कल्पतरुकारारादयस्तु\\
इष्टिश्राद्धमिच्छाश्राद्धम्, तच्च "श्राद्ध प्रति रुचिश्चैव" इत्यनेन
याज्ञ.\\
वल्क्येन विहितमिति व्याख्यातवन्तः । तन्मते च कर्माऽङ्गश्राद्धस्य
ना.\\
न्दीमुखश्राद्धेऽन्तर्भावाद्वसुसत्यावेव तत्र देवौ । नैमित्तिके कामकाला\\
विति । नैमित्तिकपदं चात्र नवान्ननिमित्तश्राद्धपरम् ।\\
विश्वे देवाः क्रतुरक्षौ सर्वास्विष्टिषु कीर्त्तितौ ।\\
नित्यं नान्दीमुखे श्राद्धे वसुसत्यौ }{च}{ पैतृके \textbar{}\textbar{}\\
नवान्नलाभे देवौ हि कालकामौ सदैव हि ।\\
अपि कन्यागते सूर्ये श्राद्धे }{च}{ धुरिलोचनौ ।\\
पुरुरवार्द्रवौ चैव विश्वेदेवाश्च पार्वणे \textbar{}\\
हेमाद्रयादिधृतादित्यपुराणवचनैकवाक्यत्वात् । इष्टिषु = कर्माङ्गश्रा.\\
द्धेषु । नवान्नलाभ इति=नवानानां लाभो यस्मादिति व्युत्पत्त्या व्रीहि.\\
यवपाको धान्यपाकश्वोक्तः । कन्यागते सूर्ये इति काम्यश्राद्धस्याप्यु.\\
पलक्षणम् । "काम्ये }{च}{ धुरिलोचनौ" इति बृहस्पतिवचनैकवाक्य-\\
त्वात् । क्वचित्तु श्राद्धं चेत्यत्र काम्ये चेति पाठ एव ।
नवान्नश्राद्धस्य\\
नैमित्तिकत्वं शातातपवचने व्यक्तम् ।\\
यथा\\
नवोदके नवान्ने }{च}{ गृहप्रच्छादने }{तथा}{ ।\\
पितरः स्पृहयन्त्यन्नमष्टकासु मघासु }{च}{ ॥\\
तस्माद्दद्यात्सदोद्युक्तो विद्वत्सु ब्राह्मणेषु }{च}{ ।\\
नवोदके = वर्षोपक्रमे । गृहप्रच्छादने - नवगृहसम्पत्तौ । विद्वत्सु ब्रा.\\
ह्मणेषु च प्राप्तेषु । हेमादिस्तु " नैमित्तिके कालकामौ " इत्यत्र
यौगिकनै\\
मित्तिकपदश्रवणात्, आदित्यपुराणे नवानलाभपदं ग्रहोपरागादिनि-\\
मित्तकैनैमित्तिकश्राद्धमात्रोपलक्षणार्थमित्युक्तवान् । सर्वथा\\
एकोदिष्टन्तु यच्छ्राद्धं तन्नैमित्तिकमुच्यते ।\\
तदध्यदैवं कर्त्तव्यमयुग्मानाशयेद् द्विजान् \textbar{}\textbar{}\\
इति भविष्यपुराणवाक्येन परिभाषितैकोदिष्टरूपनैमित्तिके तेनैव\\
दैवनिषेधात् । ''नैमित्तिके कालकामौ' इत्यनेन तत्र देवताविशेषविधा

{वैश्वदेविकश्राद्धदेवतानिरूपणम् । }{ २५\\
नमनुपपन्नमिति नैमित्तिकशब्देनात्र योगपुरस्कारेण श्राद्धान्तरस्यैवा\\
भिधानमिति द्रष्टव्यम् । श्रीदत्तादयस्तु एकोद्दिष्टरूपे नैमित्तिके
देवनिषे\\
धात् ``नैमित्तिके कामकालौ '' इत्यत्र नैमित्तिकपदमेकोद्दिष्टरूपनैमि
-\\
न्तिकयोगात्सपिण्डीकरणपरम् ।\\
सपिण्डीकरणश्राद्धं देवपूर्व निवेदयेत् ।\\
इत्यनेन तस्य सदेवत्वाभिधानादित्याहु \textbar{} कल्पतरुकारस्तु नैमित्ति\\
कपदमत्र साग्निकक्रियमाणसांवत्सरिकैकोद्दिष्टपरम् । तस्य पार्वणवत्\\
क्रियमाणतया तत्र विश्वेषां देवानां सम्भवादित्युक्तवान् । काम्ये चेति ।\\
काम्यस्य निरुक्तिर्भविष्यपुराणे-\/-\\
कामाय तु हितं काम्यमभिप्रेतार्थसिद्धये ॥ इति ।\\
अभिप्रेतं = धनपुत्रादि । तत्सिद्धये = तत्प्राप्तये ।
तच्चापस्तम्बाद्युक्तं\\
तत्तत्तिथ्यादौ श्राद्धादि । तच्च वक्ष्यते । अत्र विश्वे देवाः
क्रतुदक्षावि\\
त्यादित्यपुराणादिवाक्येषु सर्वत्र द्विवचनश्रवणान्मिलितयोस्तयोः
श्राद्धे\\
देवतात्वम् । उत्पत्तिमित्यादि । ये विश्वेषां देवानामुत्पत्ति नाम च
न\\
जानन्ति तैर्मन्त्रलिङ्गादावाहनकाले आगच्छन्त्वित्यादिमन्त्रः पठनीयः,\\
अत्र तत्तच्छ्राद्धे तत्तद्देवताविशेषप्रतिपादनात्तत्तन्नाम
सङ्कीर्त्तनपूर्वक\\
मेव निमन्त्रणादिवाक्यं प्रयोज्यं, न तु केवलविश्वदेवपदेन । संशावि\\
घेरुद्देशप्रयोजनत्वात् । अत एव -\\
"इष्टिश्राद्धे क्रतुदक्षौ सङ्कीरयौं वैश्वदेविके ।\\
इतिहेमाद्रिधृतवाक्ये सङ्कीर्त्यावित्युक्तम् ।\\
श्रीदत्तस्तु नेदं वाक्यं ऋतुदक्षादिनामनिर्देशपरं }{तथा}{ श्रवणा-\\
भावात्, किन्तु योऽसौ गणो विश्वदेवपदनिर्देश्यस्तत्र क्रतुदक्षा-\\
दियुगस्य इष्टिश्राद्धादौ मुख्यत्वं चिन्तनीयमित्येवम्परम् \textbar{}
मुख्यत्वेन\\
चिन्तनीयत्वमप्यश्रुतमितिचेत्, 'प्राधान्येन व्यपदेशा भवन्ति' इति\\
न्यायात्प्राधान्य ऋतुदक्षादिव्यपदेशादेव लभ्यते, चिन्तनीयत्वं तु\\
एतेषामुत्पत्तिनामाज्ञाने आगच्छन्त्वित्यादिमन्त्रपाठविधानात्कल्प्य\\
ते । अत एव सर्वत्र मुनिवाक्ये बहुवचनश्रवणाद्गणस्यैवोद्देश्यता\\
प्रतीयते न तूभयोरेव । यथा "नमोविश्वेभ्यो देवेभ्य इत्येवमादौ\\
प्राङ्मुखयोर्निवेदयेत्'' इति विष्णुः,\\
विश्वान्देवान् यवैः पुण्यैरभ्यर्थ्यासनपूर्वकम् \textbar{}\\
इति मत्स्यपुराणम्, "विश्वान्देवानहमाषादयिष्य" इति कात्यायनः,\\
४ वी० मि०

{२६ }{ }{वीरमित्रोदयस्य श्राद्धप्रकाशे-}{\\
"विश्वे देवास आगत" इत्यावाहनमन्त्रश्च । न वा विश्वदेवपदेन क्र.\\
तुदक्षादिपदेन }{च}{ समुच्चयान्निर्देशः समुचितः, निरपेक्षश्रवणात् ।\\
किश्च समुचये कीदृशं वाक्यं, विश्वाभ्यां देवाभ्यां क्रतुदक्षाभ्यामिति\\
वा, विश्वेभ्यो देवेभ्यः क्रतुदक्षाभ्यामिति वा, विश्वेभ्यो देवेभ्यः
क्रतुः\\
दक्षेभ्य इति वा, विश्वेभ्यो देवेभ्यः क्रतुदक्षसंज्ञकेभ्य इति वा । नाद्यः
।\\
विश्वदेवपदस्य बहुवचनान्तस्यैव सर्वत्र दर्शनात्तस्य बहुवचनेनैव नि\\
र्देश्यत्वावगतेः । न द्वितीयः \textbar{} अनन्वयात् ।
तुल्याधिकरणयोर्विशेष\\
णयोः समानविभक्तिवचननियमात् । उर्वश्यप्सरस इतिवत्प्रयोगसा.\\
धुत्वेनान्वयो न विरुद्ध इति चेत्, प्रमाणसिद्धायां प्रयोगावश्यकत\\
त्कल्पनायामुचितत्वात् । न चेह}{ तथा}{ । न तृतीय: । ऋतुदक्षयोर्बहु\\
त्वाभावेन बहुवचनार्थानन्वयात् । अत एव न चतुर्थः । संज्ञाशब्दान्त.\\
र्भावेन देवतात्वाभावाश्चेति । तस्माद्यथोक्तैव व्याख्या ज्यायसीत्याह ।\\
वस्तुतस्तु "सङ्कीत्य वैश्यदेविके" इत्युक्तवाक्यैकवाक्यतया "इ.\\
टिश्राद्धे क्रतुर्दक्ष" इत्यादि बृहस्पतिवाक्यस्य कीर्त्तनपरत्वे सिद्धे
वि\\
श्वदेवपदस्य सर्वत्र बहुवचनान्तप्रयोगदर्शनेन बहुवचनान्तस्यैव त\\
स्य साधुत्वे निर्णीते उर्वश्यप्सरस इतिवत् क्रतुदक्षौ विश्वे देवा इत्या\\
दिप्रयोगो नानुपपन्नः । विशिष्य नामाज्ञाने विश्वेषां देवानाम् इत्या-\\
द्येव प्रयोक्तव्यम् ।\\
हेमाद्रिस्तु पूजार्थ बहुवचनमिति मन्वानः क्रतुदक्षसंशका विश्वे\\
देवा इति प्रयोगमङ्गीचकार ।\\
अत्र विश्वे देवा इत्यसमस्तमेव प्रयोज्यम् सर्वत्र }{तथा}{ प्रयो\\
गदर्शनात् । अत एव गणविशेषवाचकस्यापि विश्वशब्दस्य सर्व\\
नामकार्य न विरुद्धम् । अभिधायां सर्वनामतानिषेधस्मृतिस्तु आधु\\
निकसङ्केतविषयस्यैव विश्वशब्दस्य सर्वनामकार्यनिषेधादुपपन्ना ।\\
अथ विकिरभुक्तोच्छिष्टयोर्देवतानिर्णयः ।\\
तत्र मनुविष्णू-\\
असंस्कृतप्रमीतानां त्यागिनां कुलयोषिताम् ।\\
उच्छिष्टं भागधेयं स्याद्दर्भेषु विकिरस्तु यः ॥

{ }{( म० अ० ४ श्लोक० २४५ )}{\\
कुलयोषितां त्यागिनामित्यन्वयः ।\\
हारीतः,\\
अरुढदन्ता ये च मृता गर्भाद्य च विनिःसृताः ।

{विधवाकर्तृकश्राद्धदेवतानिर्णयः । ३७\\
मृता ये चाप्यसंस्कारास्तेषां भूमौ प्रदीयते ॥\\
मार्कण्डेयपुराणे,\\
अन्नप्रकिरणं यत्तु मनुष्यैः क्रियते भुवि ।\\
तेन तृप्तिमथायान्ति ये पिशाचत्वमागताः ॥\\
ये चादन्ताः कुले बाला. क्रियायोग्या ह्यसंस्कृताः ।\\
विपश्नास्तेऽत्र विकिरसंमार्जनजलाशिनः ॥

{मनुः,\\
उच्छेषणं भूमिगतमजिह्मस्याशठस्य च ।\\
दासवर्गस्य तत्पित्र्ये भागधेयं प्रचक्षते ॥ (अ०३ श्लो०२४६)\\
पित्र्ये= पितृकर्मणि ॥\\
वसिष्ठ,\\
प्राक् संस्कारात्प्रमीतानां सप्रेध्याणामिति श्रुतिः ।\\
भागधेयं मनुः प्राह उच्छिष्टोच्छेषणे उभे ॥\\
आश्वलायनगृह्यपरिशिष्टम्,\\
उच्छेषणं भूमिगतं विकिरं लेपनोदकम् ।\\
अनुप्रेते च विसृजेदप्रजानामनायुषाम् ॥ इति ।\\
अथ संन्यासाश्राद्धदेवतानिर्णयः ।\\
शौनकः, देवऋषिदिव्यमनुष्यभूतपितृमातृआत्मादीनां पृथक् पि.\\
ण्डदानैर्युग्मैर्ब्राह्मणैरष्टौ श्राद्धानि कुर्यादिति । तत्राद्ये
श्राद्धे ब्रह्मविष्णु\\
महेश्वराः । द्वितीये देवर्षिब्रह्मर्षिक्षत्रियर्षयः । क्वचितु
देवर्षिक्षत्रार्षे.\\
मनुष्यर्षय इति पाठः । तृतीये वसुरुद्रादित्याः । चतुर्थे सनकसन\\
न्दनसनातनाः । पञ्चमे पृथिव्यादिभूतानि चक्षुरादीनि करणानि\\
चतुर्विधो भूतग्रामः । षष्ठे पितृपितामहप्रपितामहाः, मातामहप्रमाता\\
महवृद्धप्रमातामहाश्च । सप्तमे मातृपितामहीप्रपितामह्यः । अष्टम\\
आत्मपितृपितामहाः । एते सर्वे नान्दीमुखविशेषणवन्तो देवताः ।\\
जीवच्छ्राद्धादिदेवतास्तत्तत्प्रकरणे वक्ष्यामः ।\\
अथ विधवाकर्तृकश्राद्धदेवता ।

{सङ्ग्रहे,\\
चत्वार: पार्वणाः प्रोक्ता विधवायाः सदैव हि ।\\
स्वभर्तृश्वशुरादीनां मातापित्रोस्तथैव च \textbar{}\textbar{}\\
ततो मातामहानां च श्राद्धदानमुपाक्रमेत् ।

{२८ }{ वीरमित्रोदयस्य श्राद्धप्रकाशे-}{\\
अशक्तौ तु -\\
स्वभर्तृप्रभृतित्रिभ्यः स्वपितृभ्यस्तथैव च ।\\
विधवा कारयेच्छ्राद्धं यथाकालमतन्द्रिता \textbar{}\textbar{}\\
अथ विभक्तिनिर्णयः ।\\
तत्र विभक्तिज्ञानस्यावश्यकत्वमुक्तं नागरखण्डे,\\
विभक्तिरहितं श्राद्धं क्रियते यद्विपर्ययात् ।\\
अकृतं तद्विजानीयात्पितॄणां नोपतिष्ठते ॥\\
तस्मात्सर्वप्रयत्नेन ब्राह्मणेन विजानता ।\\
विभक्तिभिर्यथोकाभिः श्राद्धं कार्य त्रिभिः सदा ॥\\
विभक्तिरहितम् =विद्दितविभक्तिरहितम् । तिसृभिरित्यर्थे त्रिभिरिति\\
छान्दसम् । एतचैकोद्दिष्टाभिप्रायेण । तत्रावाहनाभावेन द्वितीयाया\\
अप्रयोगात । अन्यत्र चतस्त्रो वक्ष्यमाणवचनात् । विभकीराह-\\
व्यास,\\
चतुर्थी स्वासने नित्यं सङ्कल्पे च विधीयते ।\\
प्रथमा तर्पणे प्रोक्का सम्बुद्धिमपरे जगुः ॥\\
सङ्कल्पे = अन्नात्सर्गे\\
धर्मोऽपि,\\
पृच्छाक्षयासने षष्ठी चतुर्थी चासने मता ।\\
अर्ध्यावनेजनं पिण्डं }{तथा}{ प्रत्यवनेजनम् ॥\\
सम्बुद्ध्यैतानि कुर्वन्ति शब्दशास्त्रविशारदाः ।\\
पृच्छा श्राद्धारम्भे प्रश्नवाक्यम् \textbar{} अक्षय्यं तदुदकदानम् ।
अवनेजनं-\\
पिण्डदानात्प्राक् पिण्डार्थरेखायां, पिण्डदानोत्तरं च पिण्डेषु: जल*\\
निनयनम् ।\\
भृगुः -\\
आवाहने द्वितीया स्यादेष शास्त्रविनिर्णयः ।\\
सङ्ग्रहकारस्तु क्वचिद्विशेषमाह -\\
अक्षय्यासनयोः षष्ठी द्वितीयाचाहने }{तथा}{ ।\\
अन्नदाने चतुर्थी स्याच्छेषाः सम्बुद्धयः स्मृताः ॥\\
भविष्योत्तरे तु,\\
चतुर्थी सर्वकार्येषु प्रथमा तर्पणे भवेत् ।\\
षष्ठीविभक्तिरक्षय्ये पितृकार्ये यथाविधि ॥

सम्बन्धगोत्रनामोच्चारणक्रमविचारः । ३९ 

{भृगुस्तु,\\
गन्धमाल्यं च धूपं च दीपमन्नं सदक्षिणम् ।\\
अपृथक्त्वेन दातव्यं चतुर्थ्या विधिमिच्छता ।\\
अपृथक्त्वेन=अवैषम्येण । अत्र दर्शितवचनेषु यत्र विरोधस्तत्र वि.\\
कल्पो द्रष्टव्यः, एकार्थत्वात् । स च यथाशाखं व्यवस्थितः । उक्तानु•\\
सारेण प्रयोगो गृह्यपरिशिष्टे ।\\
गोत्रं स्वरान्तं सर्वत्र गोत्रस्याक्षय्यकर्मणि ।\\
गोत्रस्तु तर्पणे कुर्यादेवं दाता न मुह्यति \textbar{}\textbar{}\\
सर्वत्रैव पितः प्रोक्तः पिता तर्पण कर्माणि ।\\
अक्षय्ये तु पितुः कार्यं पितॄणां तृप्तिमिच्छता ।\\
शर्मन्नर्थ्यादिके कार्ये शर्मणोऽक्षय्यकर्मणि ।\\
शर्मा तु तर्पणे कुर्यादेवं कुर्वन्न मुह्यति \textbar{}\textbar{}\\
गोत्रं = तच्छन्दः । स्वरान्तम् = अजन्तम् \textbar{} सर्वत्र =
अपवादादन्यत्र । यद्यपि\\
सप्तम्यन्तमप्यजन्तं भवति, }{तथापि}{ तस्याविहितत्वात्तद्व्युदासः । एवं\\
गोत्रस्येत्यादि तत्तद्विभक्तिप्रदर्शनार्थं द्रष्टव्यम् ।
एकवचनपुंल्लिङ्गपि\\
त्रादिशब्दाः प्रदर्शनार्थाः । मातृमातामहादिशब्दानामपि तत्र तत्र यो-\\
ज्यस्वात् । उक्तं च नागरखण्डे -\\
मातर्मात्रे }{तथा}{ मातुरालने कल्पनेऽक्षये ।\\
गोत्रे गोत्रायै गोत्रायाः प्रथमाद्या विभक्तयः ॥\\
देवि दैवये }{तथा}{ देव्या एवं मातुश्च कीर्त्तयेत् ।\\
इति विभक्तिनिर्णय ।\\
अथ सम्बन्धगोत्रनामोच्चारणक्रमः ।\\
तत्र प्रचेताः,\\
गोत्रसम्बन्धनामानि यथावत्प्रतिपादयेत् ।

{व्यासः,\\
आलप्य नामगोत्रेण कर्त्तव्यं सर्वदैव हि ।\\
अत्र सम्बन्धनामगोत्रशब्दानां नानाक्रमदर्शनाद्विकल्पः । स ऐ.\\
च्छिकः शाखाभेदेन वा व्यवस्थितः । सम्बन्धस्तु पित्रादिशब्दैरेव\\
निर्देष्टव्यो नास्मच्छब्देनेति केचित्प्राच्याः \textbar{} }{तन्न}{ ।
``नामगोत्रे उच्चार्य}{\\
मम पितरेतत्वेऽर्षे मम पितामह; एतत्तेऽर्धे मम प्रपितामहैतत्तेऽर्धम् "

{३० }{ वीरमित्रोदयस्य श्राद्धप्रकाशे-}{\\
इति मानवसूत्रे स्पष्टप्रयोगविधानात् । अत्र च गोत्रसगोत्रपदयोर्यद्यपि\\
पर्यायत्वं -\\
"पराशरसगोत्रस्य वृद्धस्य सुमहात्मनः " ।\\
इत्यादिप्रयोगात्प्रतीयते, तथाप्यत्र गोत्रशब्द एव प्रयोक्तव्यः ।\\
"अमुकामुकगोत्रैतत्तुभ्यमन्नं स्वधा नमः'' ।\\
इति ब्रह्मपुराणादिवचनादिति ।\\
अथ गन्धादिदाने सम्प्रदानं निर्णीयते ।\\
तत्र "गन्धान्ब्राह्मणसात्कृत्वा " इति मरीचिवचने तदधीनवचन-\\
तायां विहितेन सातिप्रत्ययेन दीयमानगन्धादेर्ब्राह्मणस्वामिकत्वावग\\
तेर्गन्धादिभोक्तृत्वाच्च "एतस्मिन्काले गन्धमाल्यधूपदीपाच्छाद-\\
नानां प्रदानम्" इत्याश्वलायनेन गन्धादेर्दानविधानात्तस्य च प्रतिग्रही.\\
तृव्यापारसाध्यत्वाद्देवतायाश्चाप्रतिग्रहीतृत्वाद्राह्मणानामेव सम्प्रदा\\
नश्वम् ।\\
अयं वो धूप इत्युक्त्वा तदग्रे }{च}{ दहेत्ततः ।\\
इति ब्रह्मपुराणाच्च तथा । अत्र तदप्रेशब्देन
पूर्ववाक्यनिर्दिष्टब्राह्म\\
णानामेवाग्रदेश उच्यते, न तु देवतायाः, अमूर्त्तत्वाद् देशान्वयायो\\
गाव \textbar{} धूपग्रहणं गन्धाद्युपलक्षणम् आश्वलायनसूत्रे प्रायपाठात
\textbar{}\\
उपवेश्यासने शुभ्रं छत्र तत्र प्रकल्पयेत् \textbar{}\\
आवरणार्थं च तद्वत्रं ब्राह्मणाय प्रदापयेत् ॥\\
इति वराहपुराणे चतुर्थीश्रुतेश्च ब्राह्मणसम्प्रदानकत्वावगमः ।\\
विभवे सति विप्रेभ्यो योऽस्मानुद्दिश्य दास्यति ।\\
इति विष्णुपुराणीयपितृवाक्ये स्पष्टं विप्राणां देवतानां च सम्प्रदान.\\
त्वोद्देश्यत्वभेदावगतेश्च । उद्देश्यता तु पितॄणामेव सम्प्रदान ( १ ) ( न
तु\\
सम्प्रदानत्व) मिति हरिहरः ।\\
श्रीदत्तादयस्तु " योऽस्मानुद्दिश्य दास्यति" इति विष्णुपुराणात्,\\
एतानि श्रद्धयोपेतः पितृभ्यो यो निवेदयेत् ।\\
इति वाराहाच स्पष्टं देवताया (२)\\
उद्देश्यत्वमवगम्यमानं नापहोतुं श\\
क्यम् । किञ्च "ते तृप्तास्तर्पयन्त्येनम्" इत्यादिवचनैः पितृतर्पकता\\
श्राद्धस्यावसीयते सा च तदुद्देशेन द्रव्यत्याग एवोपपद्यते, लोके\\
तथा दर्शनात् ।

% \begin{center}\rule{0.5\linewidth}{0.5pt}\end{center}

(१) अयं कोष्टान्तर्गतः पाठ आदर्शपुस्तके नास्ति । ( २ )
सम्प्रदानत्वमित्यर्थः ।

श्रा{द्धोचितद्रव्यनिर्णयः । ३१\\
श्वेतचन्दनकर्पूरकुङ्कुमानि शुभानि च ।\\
विलेपनार्थ दद्यात्तु यदन्यत्पितृवल्लभम् \textbar{}\textbar{}\\
इति ब्रह्मपुराणे पितृवल्लभत्वाभिधानात्-\\
निवेदितं च यत्तेन पुष्पगन्धानुलेपनम् ।\\
तद्भूषितानथ स तान्ददर्श पुरतः स्थितान् ॥\\
इति मार्कण्डेयात्पित्रुदेशेन निवेदनावगते:-\\
लोके श्रेष्ठतमं यच्च आत्मनश्चापि यत्प्रियम् ।\\
सर्वे पितॄणां दातव्यं तदेवाक्षयमिच्छता ॥\\
इति स्पष्टं पित्रुद्देश्यत्वावगतेश्च पितॄणामेवोद्देश्यत्वम् । सातिप्र\\
त्ययस्तु पित्रुद्देशेन त्यक्तमपि द्रव्यं ब्राह्मणेषु प्रतिपद्यमानं
भोज्या.\\
नमिव तत्स्वामिकं भवतीत्युपपद्यते । ब्राह्मणेभ्यो ददातीति वाक्यम-\\
पि हविःशेषमृत्विग्भ्यो ददातीतिवत्समर्पणाभिप्रायं व्याख्येयम् ।\\
पितृभ्यो ददातीति प्राधानान्तरङ्गभूतदेवतार्थत्वबोधक श्रुतिविरोधात्,\\
ब्राह्मणानां च बहिरङ्गत्वेन ब्राह्मणेभ्यो ददातीत्यत्रैव गौणताया न्या-\\
व्यत्वाच्च । अत एव पितृभ्यो दद्यादित्यत्र मुख्यदानासम्भवात्\\
दानैकदेशत्यागलक्षणाप्यदोष \textbar{} उपक्रमस्थपितृशब्दानुरोधेनोपसंहा.\\
रस्थददातिशब्द एव लक्षणाया न्याय्यत्वात् । तदप्रेणेति देशसम्ब\\
न्धस्तु पितृब्राह्मणयोरभेदेन चिन्तनादुपपादनीयः । तस्मात्सुष्ट्रकं\\
पितॄणां देवतास्वं गन्धादिद्रव्यत्याग इति । एवं विश्वेषां देवानामपि ।\\
इदं वः पाद्यमर्थे च पुष्पधूपविलेपनम् ।\\
अयं दीपप्रकाशश्च विश्वान्देवान्समर्पयेत् \textbar{}\textbar{}\\
इति ब्रह्मपुराणात् । अन्ये तु ब्राह्मणानां पित्रभेदेन चिन्तनविधेः\\
सत्वात्पित्रभेदेन तेषामेव सम्प्रदानत्वं घटते, एवं च न ददातिरपि\\
लाक्षणिको भविष्यति अत एव -\\
यमः,\\
ब्राह्मणैश्च सहाश्नन्ति पितरो ह्यन्तरिक्षगाः ।\\
वायुभूता न दृश्यन्ते भुक्ता यान्ति परां गतिम् ॥ इति ।\\
आचमनीयं दक्षिणादानं च ब्राह्मणार्थमेव शुद्धर्थत्वादा-\\
नन्त्यार्थत्वाच्चेत्याहुः । इति सम्प्रदाननिर्णय ।\\
अथ श्राद्धोचितद्रव्याणि ।\\
तत्रापस्तम्बः, एतैस्तीव्रतरा पितॄणां तृप्तिः, द्राघीयांसं कालं }{तथा}{\\
धर्मानुहृतेन द्रव्येण \textbar{}}

{३२ }{ वीरमित्रोदयस्य श्राद्धप्रकाशे-}{\\
एते = पूर्वोक्तैर्द्रव्यैः । द्राघीयांसम् दीर्घम् \textbar{}
धर्मानुहृतेन= धर्मोपार्जितेन -\\
शङ्खलिखितौ, धर्मेण वित्तमादाय पितृभ्यो दद्यात् ।\\
अतश्च धर्यैरेवोपायैरर्जितानि द्रव्याणि श्राद्धे देयानि । तांश्चाह-\\
मनुः,\\
सप्तवित्तागमा धर्म्या दायोलाभः क्रयो जयः\\
प्रयोग: कर्मयोगश्च सत्प्रतिग्रह एव च ॥\\
( अ० १० श्लो० ११५ )\\
वित्तागमा:= वित्तार्जनोपायाः। धर्म्या. = पुरुषस्य प्रत्यवायानुत्पादकाः
।\\
स्वस्वामिसम्बन्धेनैव निमित्तेन यदन्यदीयं द्रव्यमन्यस्य स्वं भवति\\
स दायः । अनन्यपरिगृहीतस्य जलतृणकाष्टादेर्वान्यादेर्वाधिगमो ला.\\
भः । द्रव्यविनिमयः क्रयः । वैरिपराभवो जयः । वृज्वर्थे परस्य द्रव्यस\\
मर्पणं प्रयोग । कर्मणा आविज्याध्यापनपौरोहित्यादिना शिल्पेन शु\\
श्रूषया वा योगः कर्मयोग \textbar{} दीयमानद्रव्यस्वीकारः प्रतिग्रहः । स
तु अ.\\
निषिद्ध: । अनिषिद्धत्वं च दायादिष्वपि द्रष्टव्यम् । पतेषु च दायला-\\
भक्रयास्साधारणाः, जयः क्षत्रियस्यैव, स च दण्डाऽऽकरकरादीनामु\\
पलक्षणम् । प्रयोगो वैश्यस्यैव स च कृषिवाणिज्यादेरुपलक्षणम् ।\\
आविज्यादिकर्मयोगः प्रतिग्रहश्च ब्राह्मणस्यैव । शिल्पादिकर्मयोगः\\
कारूणामेव । शुश्रूषालक्षणः शूद्रादेरेवेति विवेकः ।}{ तथा}{ च-\/-\\
गौतम \textbar{} स्वामी रिक्थसंविभागपरिग्रहाधिगमेषु ब्राह्मणस्याधिकं\\
लब्धं क्षत्रियस्य विजितं निर्विष्टं वैश्यशूद्रयोरिति ।\\
रिक्थ = पितृपैतामहं धनम् । सविभागव्यत्र भ्रात्रादेर्विभक्तभ्रात्रादि\\
द्रव्यम् \textbar{} अनन्यपरिगृहीतस्य जलकाष्ठादेः स्वीकारः परिग्रह अधि\\
गमो = निध्यादेः प्राप्तिः । एतेषु निमित्तेषु स्वामी भवति । अधिकम्=अ\\
साधारणम् । लब्धं=प्रतिग्रहादिभिः । विजितं = विजयादिभिर्लब्धम् ।\\
निर्विष्टं वैश्य कृप्यादिलम्धम्, शूद्रशुश्रूषादिलब्धम् ।\\
नारदोऽपि,\\
तत्पुनस्त्रिविधं ज्ञेयं शुक्लं शबलमेव च ।\\
कृष्णं च तस्य विज्ञेयो विभागः सप्तधा पुनः ॥\\
श्रुतशौर्यतपः कन्यागतं शिष्यान्वयागतम् ।\\
धनं सप्तविधं शुक्लमुदयोऽस्य तु तद्विधः ॥\\
कुसीदकृषिवाणिज्यशिल्पशुक्लानुवृत्तितः ।

{ }{ }{श्राद्धोचितद्रव्यनिर्णयः}{ \textbar{} }{ ३३\\
कृतोपकाराद्यत्प्राप्तं शबलं समुदाहृतम् ॥\\
पराप्रितिरूपकसाहस्रैः ।\\
व्याजेनोपार्जितं यत्तत्सर्वेषां कृष्णमुच्यते ॥\\
यथाविधेन द्रव्येण यत्किञ्चित्कुरुते नरः ।\\
तथाविधमवाप्नोति स फलं प्रेत्य चेह च ॥\\
आगतशब्दः प्रत्येकं श्रुतादिभिः सम्बध्यते । श्रुतं = विद्या । शौर्य=\\
शौर्बनिमित्तेन यद्विजितम्, अथ च भृतिरूपेण लब्धमिति द्वयम् ।\\
कन्यागत = कन्याप्रतिग्रहकाले श्वशुरादिभ्यो यल्लब्धम् । शिष्याद.\\
ध्याप्यात् पणव्यतिरेकेण यत्प्राप्तं गुरुदक्षिणादि \textbar{} अन्वयागतं =
दाय-\\
प्राप्तम् । अत्र यथाधिकारं शुक्कुत्वं ध्येयम् । न्याय्यवृद्ध्या
द्रव्यप्रयोगः\\
कुसीदम् \textbar{} कृषि =लाङ्गलकर्म । क्रयविक्रयव्यवहारेण धनवर्द्धनं
वाणिज्यम् \textbar{}\\
शिल्प=कारुकर्म । अनुवृत्तिर्द्विजातिशुश्रूषा । राजादिपार्श्वे वर्त्तमानो
यः\\
कञ्चित्किञ्चित ब्रूते त्वयेदं मम देयं मया च त्वदीयं कार्य कर्त्तव्यमि\\
ति पणं कृत्वा यदर्जितं तत्पार्श्विकम् । द्यूतम्=अक्षदेवनादि । चौर्यम् -
शौर्य-\\
निमित्तेन यद्विजितम् अथ च प्रच्छन्नापहारः । आर्त्तिः=परपीडा । प्रतिरूपकं
=\\
परवञ्चनार्थ मणिसुवर्णादेः प्रतिकृतिकरणम् । साहसम् = मारणव्या\\
पाराङ्गीकारेण पश्यतोहरत्वम् । व्याजं=दम्भः ।\\
ब्रह्मपुराणे-\\
शुक्लषित्तेन यो धर्म प्रकुर्यात् श्रद्धयान्वितः ।\\
तीर्थ पात्रं समासाद्य देवत्वेन समश्नुते ॥\\
राजसेन च भावेन वित्तेन शबलेन च ।\\
प्रदद्यादानमर्थिभ्यो मनुष्यत्वे तदश्नुते ॥\\
तमोवृतस्तु यो दद्यात्कृष्णं वित्तं तु मानवः ।\\
तिर्यक्त्वे तत्फलं प्रेत्य समश्नाति नराधमः ॥\\
एते च यथा सम्भवं स्त्रीपुंसयोरविशेषेण धर्म्या द्रव्यार्जनोपा.\\
याः । स्त्रीणामेव तु विशेषत आह-\\
याज्ञवल्क्यः, ( अ० २ दाय० श्लो० १४३ )\\
पितृमातृपतिभ्रातृदत्त मध्यग्न्युपागतम् \textbar{}\\
(१) आधिवेदनिकं चैव स्त्रीधनं परिकीर्तितम् ॥

% \begin{center}\rule{0.5\linewidth}{0.5pt}\end{center}

( १ ) आधिवेदनिकाय श्चेति मुद्रितपुस्तके पाठः, आद्यशब्देन
रिक्थक्रयसंवि.\\
भागपरिप्रहाधिगमप्राप्तमिति विज्ञानेश्वर: \textbar{}\\
५ बी० मि०

{३४ }{वीरमित्रोदयस्य}{ श्राद्धप्रकाशे-}{\\
अध्यग्न्युपागतम् = विवाहकाले अशिसमक्षं मातुलादिभिर्यहत्तम् ।\\
आधिवेदनिकम् = भर्त्रा यत् अधिवेदननिमित्तं दत्तम् \textbar{}\\
मनुः, ( अ० ९ श्लो० १९४ )\\
अध्यग्न्यध्यावाहनिकं दत्तं च प्रीतिपूर्वकम् ।\\
भ्रातृमातृपितृप्राप्तं षड्विधं स्त्रीधनम् स्मृतम् ॥\\
अथाधर्मोपार्जितप्रतिषेधः ।\\
शातातपः,\\
द्रव्येणान्यायलब्धेन यः करोत्यम् ।\\
न तत्फलमवाप्नोति तस्यार्थस्य दुरागमात् ॥\\
और्द्धदेहिकम् = श्राद्धादि । विशिष्यं निषेधमाहतु:-\\
शातातपव्यासौ,\\
वेदविक्रयनिर्दिष्टं स्त्रीषु यच्चार्जितं धनम् ।\\
अदेयं पितृदेवेभ्योयच क्लीबादुपागतम् \textbar{}\textbar{}\\
वेदविक्रयश्च षड्विध उक्तो भविष्यपुराणे ।\\
प्रख्यापनं च दानं च प्रश्नपूर्वः प्रतिग्रहः ।\\
याजनाध्यापने वादः षड्विधो वेदविक्रयः ॥\\
धनग्रहणार्थमहं वेदविदिति प्रख्यापनम् । दानं = पारायणादीनाम् ।\\
श्रोत्रियपराजय पर्यवसानो वादः । स्त्रीषु आर्जितं = स्त्री
व्यापारोपजीवने\\
नार्जितम्, स्त्रियाऽर्जितं, स्त्रीविक्रयार्जितं वा ।\\
मार्कण्डेयपुराणे ।\\
यथोत्कोचादिना प्राप्तं पतिताद्यदुपार्जितम् ।\\
(१) अन्याय्य कन्याशुल्काञ्च द्रव्यं चात्र विगर्हितम् ॥\\
पित्रर्थे मे प्रयच्छस्वेत्युक्त्वा यश्चाप्युपार्जितम् ।\\
वर्जनीयं सदा सद्भिस्तत्तद्वै श्राद्धकर्मणि ।\\
कार्यप्रतिबन्धनिवृत्यर्थमापत्प्रतीकारार्थ वा राजाधिकृतेभ्यो य.\\
हीयते स उत्कोचः । गोमिथुनाधिककन्या शुल्कम्; अन्याय्यकम्माशुल्कम्
\textbar{}\\
रेवाखण्डे ।\\
देवद्रव्यं गुरुद्रव्यं द्रव्यं चण्डेश्वरस्य च ।\\
त्रिविधं पतनं दृष्टं दानलङ्घनभक्षणात् \textbar{}\textbar{}\\
मत्स्यपुराणे,\\
हस्त्यश्वौ गामनड्वाहं मणिमुक्तादिकाञ्चनम् ।

% \begin{center}\rule{0.5\linewidth}{0.5pt}\end{center}

{( १ ) अन्यायकन्या शुल्कार्थमिति निर्णयसिन्धौ पाठः ।\\


{श्राद्धोचितद्रव्योत्पत्तिनिर्णयः । }{३५\\
प्रत्यक्षं हरते यस्तु पश्चादानं प्रयच्छति \textbar{}\textbar{}\\
स दाता नरकं याति यस्यार्थस्तस्य तत्फलम् ।

{तथा}{,\\
परिभुक्तमविज्ञातमपर्याप्तमसंस्कृतम् ।\\
यः प्रयच्छति विप्रेभ्यस्तद्भस्मन्यवतिष्ठते ॥\\
परिभुक्तं = कृतोपभोगम् । अपर्याप्तं = कार्याक्षमं जरङ्गवादि ।\\
अत्र हेमाद्रिः - धयैरेवोपायैरर्जितेन श्राद्धादीनि कर्त्तव्यानि\\
न तु यत्किञ्चिदुपायार्जितेनेत्यत्रैषां वचनानां तात्पर्याद्धर्मोपार्जि\\
तद्रव्यस्य श्राद्धाङ्गत्वम् । न च (१) लिप्सासुत्रोक पुरुषार्थताविरोधः,\\
तेन न्यायेन पुरुषार्थत्वे सिद्धे एभिर्वचनैः क्रत्वर्थता बोध्यत इत्याह
।\\
तस्माद्धर्मोपार्जितैरेव श्राद्धादीनि कर्त्तव्यानि ।\\
अथ श्राद्धोचितद्रव्योत्पत्तिः ।\\
मार्कण्डेयपुराणे,\\
व्रीहयश्च यवाश्चैव गोधूमाश्चाणवस्तिलाः ।\\
प्रियङ्गवो ह्युदाराश्च कोरदूषाः सचीनकाः ॥\\
भाषा मुद्रा मसूराश्च निष्पावा: सकुलत्थका ।\\
आढक्यश्चणकाश्चैव शणः सप्तदशः स्मृतः ॥\\
इत्येता अभवन् प्राम्यास्तथारण्याश्च जज्ञिरे ।\\
ओषध्यो यज्ञियास्तासां ग्राम्यारण्याश्चतुर्द्दश ॥\\
व्रीहयश्च यवाश्चैव गोधूमाञ्श्चाणवस्तिलाः ।\\
प्रियङ्गुसहिता होताः सप्तमास्तु कुलत्थकाः ॥\\
श्यामाकास्त्वथ नीवारा जर्तिलास्सगवेधुकाः ।\\
कुरुविन्दा मर्कटकास्तथा वेणुयवाश्च ये \textbar{}\textbar{}\\
ग्राम्यारण्याः स्मृता होते ओषध्यश्च चतुर्दश ।\\
व्रीहयः = षष्ठिका महाब्राह्मादयः । अणवः =वरयिकाः । प्रियङ्गुः = कङ्गुः
।\\
उदारः = चीन = कुरुविन्दो = व्रीहि विशेषः । कोरदूषा. = कोद्रवाः ।
नीवारा=आ.\\
रण्यवहियः । जर्तिला - कृष्णतिलाः । गवेधुकाः=कुसुम्भबीजतुल्याः ।\\
मर्कटा स्तृणधान्यविशेषाः । वेणुयवान् वंश बीजानि ।\\
ब्रह्मवैवतें,\\
त्वष्टा वै यजमानेन वार्यमाणे महात्मना ।\\
पपौ शचीपतिः सोमं पृथिव्यां विप्लुषोऽपतन् ॥

% \begin{center}\rule{0.5\linewidth}{0.5pt}\end{center}

(१) मी० अ० ४ १ ० १ सू० २ ।

{५६ }{वीरमित्रोदयस्य}{ श्राद्धप्रकाशे-}{\\
श्यामाकास्तु ततो जाताः पित्रर्थमपि पूजिताः ।\\
गोधूमाश्च यवाश्चैव समुन्द्रा रक्तशालयः ॥\\
एते सोमात्समुद्भूताः पितृणाममृतं ततः ।\\
तस्मात्प्रयक्षतो देया एते श्राद्धेषु वंशजैः ॥\\
मत्स्यपुराणे,\\
अमृत पिषतो वक्त्रात्सूर्यस्यामृतबिन्दवः \textbar{}\\
निपेतुर्ये तदुत्था हि शालिमुद्रेक्षवः स्मृताः \textbar{}\textbar{}\\
शर्करा परतस्तस्मादिक्षुसारोऽमृतात्मवान् ।\\
इष्टा रवेरतः पुण्या शर्करा हव्यकव्ययोः ॥\\
शर्करा = इक्षुविकारः ।\\
नागरखण्डे,\\
सृष्टि प्रकुर्वतो पूर्व पशवो लोककारिणा ।\\
खङ्गवार्ध्रीणशार्दूलाः पूर्वे शिष्टास्ततोऽपरे ॥\\
यो यथा प्रथमं सृष्टः स }{तथा}{ मेध्यतां गतः ।\\
गोधूमाश्च मसूराश्च भाषा मुद्रास्तथाणवः ॥\\
नीवाराश्चापि श्यामाका एवं सृष्टा यथाक्रमम् ।\\
शाकानि सृजता पूर्व कालशाकः स्वयम्भुवा ॥\\
असृज्यत तत्तः श्राद्धे स स्यादक्षय्यकारकः ।\\
रसांश्च सृजता पूर्व मधु सृष्टं स्वयम्भुवा ॥\\
तेन प्रशस्यते श्राद्धे पितॄणां तृप्तिदायकम् \textbar{}\\
यच्छ्राद्धं मधुना हीनं तद्रसैः सकलैरपि ॥\\
मिष्टान्नैरपि संयुक्तं पितॄणां नैव तृप्तये \textbar{}\\
अणुमात्रमपि श्राद्धं यदि न स्याश्च माक्षिकम् ॥\\
नामापि कीर्त्तनीयं स्यात्पितॄणां प्रीतये ततः ।\\
पशून्वै सृजता तेन पूर्वे गावो विनिर्मिताः ।\\
* तेन तासां पयः शस्त श्राद्धे सर्पिस्तथैव च ॥\\
गण्डकः । वार्ध्रणो= निगम उक्तः-\\
त्रिपिबं त्विन्द्रियक्षीणं यूथस्याप्रेचरं }{तथा}{ ।\\
रक्तवर्णे तु राजेन्द्र छागं वाणसं विदुः ॥\\
मुखेन सलिलं पिबनु लम्बतया कर्णाभ्यां यो जल स्पृशति स\\
त्रिपिव इत्युच्यते । इति ब्रीह्याद्युत्पतिः ।

{ }{ग्राह्यधान्यादिविचारः । }{ ३७\\
अथ ग्राह्याणि धान्यानि \textbar{}}

{मनुः,\\
हविर्यच्चिररात्राय यच्चानन्त्याय कल्पते ।\\
पितृभ्यो विधिवद्दत्तं तत्प्रवक्ष्याम्यशेषतः ॥ (अ० ३ श्लो० २६६ )\\
तिलैव्रीहियवैर्माषैरद्भिर्मूलफलेन च ।\\
दत्तेन मासं प्रयिन्ते विधिवत्पितरो नृणाम् ॥ (अ०३ श्लो० २६७)\\
अत्र चिररात्रशब्दो दीर्घकालवचनः । तिलादिगणनं नेतरपरि\\
संख्यार्थम् फलसम्बन्धार्थत्वात् । वचनान्तरवैयर्थ्यापत्तेश्च ।\\
}{तथा}{,\\
आनन्त्यायैव कल्पन्ते मुन्यन्नानि च सर्वशः ।\\
मुन्यन्नानि = नीवारादीनि ।\\
प्रचेताः,\\
कृष्णमाषास्तिलाश्चैव श्रेष्ठाः स्युर्यवशालयः ।\\
अत्रि,\\
अगोधूमं च यच्छ्राद्धं कृतमप्यकृतं भवेत् ।\\
बाह्ये,\\
यवैर्वीहितिलैर्माषैर्गोधूमैश्चणकैस्तथा ।\\
सन्तर्पयेत्पितृन्मुग्दै श्यामाकैः सर्षपद्रवैः \textbar{}\textbar{}\\
नीवारैर्हरिश्यामाकैः प्रियङ्गुभिरथार्चयेत् ।\\
सर्षपो = गौरसर्षपः ।\\
मार्कण्डेय पुराणे,\\
राजश्यामाकको श्राद्धे तद्वच्चैव प्रशान्तिका ।\\
नीवाराः पौष्कराचैव वन्यानां पितृतृप्तये ॥\\
यवव्रीही सगोधूमौ तिलाः मुद्राः ससर्षपाः ।\\
प्रियङ्गवः कोविदारा निष्पावाश्चात्र शोभनाः ॥\\
प्रशान्तिका = मध्यदेशप्रसिद्धोधान्यविशेषः । पौष्कराः = पद्मबीजानि ।\\
निष्पावाश्चात्र श्वेतवल्लाः ।\\
कूर्मपुराणे,\\
व्रीहिभिश्च यवैर्माषैरद्भिर्मूलफलेन वा ।\\
श्यामाकैश्चणकैः शाकैर्नीवारैश्च प्रियङ्गुभिः ॥\\
गोधूमैश्च तिलैर्मुद्गैमासं प्रीणयते पितॄन् \textbar{}.}

{३८ }{वीरमित्रोदयस्य}{ श्राद्धप्रकाशे}{-\\
व्यासः,\\
हवींषि श्राद्धकाले तु यानि श्राद्धविदो विदुः ।\\
तानि मे शृणु काम्यानि फल चैषां युधिष्ठिर ॥\\
वर्द्धमानतिलैः श्राद्धमक्षय्यं मनुरब्रवीत्\\
मार्कण्डेयः-\\
गोधूमैरिक्षुभिर्मुङ्गैः सतीनैश्चणकैरपि ।\\
श्राद्धेषु दत्तैः प्रीयन्ते मासमेक पितामहाः ।\\
सतीनाः = कलायाः \textbar{} ''कलायस्तु सतीनक" इत्याभेधानात् । मध्य.\\
देशे वटुरीति प्रसिद्धः ।\\
अथ वर्ज्याणि धान्यानि \textbar{}\\
वायुपुराणे,\\
अकृताप्रयणं धान्यजातं वै परिपाटलाः ।\\
राजमाषानणूंश्चैव मसूरांश्च विवर्जयेत् ॥\\
अन्ननिष्पत्तावाहिताग्नेरिष्टिविशेष आग्रयणम् । अनाहिता नेस्तु\\
नवाशेन स्थालीपाकः । सर्वेषां वा श्राद्धमेवाप्रयणम् ।\\
श्यामाकैरिक्षुभिश्चैव पितॄणा सार्वकामिकम् ।\\
कुर्यादाग्रयणं यस्तु स शीघ्रं सिद्धिमाप्नुयात् ॥\\
इति तत्रैवोक्तेः । धान्यजातं = विहितधान्यमात्रम् । इदं चाकृता\\
ग्रयणमित्यनेन सम्बध्यते । मसूरो= रोमाङ्गल्यकाख्याश्चिपिटका कृतिः\\
शिम्बीधान्यविशेषः ।\\
षट्त्रिंशन्मते,\\
कृष्णधान्यानि सर्वाणि वर्जयेच्छ्राद्धकर्माणि ।\\
न वर्जयेशिलांश्चैव मुद्रान्माषास्तथैव च \textbar{}\textbar{}\\
अत्र तिलमुद्गमाषाणामवर्जनं न तज्जातीयमात्रेऽन्वेति । अपि\\
तु पूर्षे कृष्णगुणनिषेधात्तस्यैव प्रतिप्रसवार्थ लाघवात् । इतरेषां व.\\
र्जनप्रसक्त्यभावाच्च । अतश्चेतरेषां न विधिर्न निषेधः । राजमाषाणां\\
तु नित्यं निषेध एव वक्ष्यमाणवचनात् । स्मृतिचन्द्रिकाकारस्तु
"मु}{द्गा}{ढकी\\
भाषवर्ज द्विदलानि न दद्यात्" इतिभारद्वाजवचने नज्युक्तस्य पाठस्य\\
दर्शनादित्याह । आढकी = शिम्बिधान्यविशेषः । मुद्गभाषौ प्रसिद्धौ । पाषा.\\
णयन्त्रभ्रमणेन प्रायशो द्विधा भिद्यते तद् द्विदलम् । अत्र च य आढकी\\
चणकनिप्पाबादीनां विधिः स कृष्णेतरविषयः । यश्च निषेधः स\\
कृष्णविषयः पूर्वोदाहृतवचनात् । यद्वा विहितप्रतिषिद्धत्वाद्विकल्पः ।

{ }{ }{ग्राह्यधान्यादिविचारः । }{३९\\
हारीतः -\/-\\
भाषमसूरकृतैलवणानि च श्राद्धे न दद्यात् ।\\
कृतलवणं = क्षारलवणम् । भाषाः = कृष्णभाषेतरा राजभाषशब्दवा\\
च्या: , तेषां हि "वर्ज्या मर्कटकाः श्राद्धे राजमाषास्तथैव च" इत्या-\\
दिवचनैः प्रतिषिद्धत्वात् । मर्कटा स्तृणधान्यविशेषा मकरा इति\\
प्रसिद्धाः ।\\
चतुर्विंशतिमते,\\
कोद्रवा राजमाषाश्च कुलस्था वरकास्तथा ।\\
निष्पावाश्च विशेषेण पञ्चैतांस्तु विवर्जयेत् \textbar{}\textbar{}\\
यावनालानपि}{ तथा}{ वर्जयन्ति विपश्चितः ।\\
कुलत्थाः = श्वेताः कृष्णाश्च वर्ज्याः । कृष्णमुङ्गादिवत्प्रतिप्रसवाभा\\
वात् । वरका=वनमुद्गाः । मधूलिकाशब्दस्य शालिभेदवाचित्वे सर्वे\\
यावनाला वर्ज्याः । यावनालविशेषवाचित्वे तद्वर्जम् । मधूलिकायाः\\
श्राद्धकर्मणि प्रचेतसा विहितत्वात् । निष्पणचाः = श्वेत शिम्बिधान्यतया\\
प्रसिद्धाः । ते च कृष्णाः । ``कृष्णधान्यानि सर्वाणि वर्जयेत्" इत्यनुरो\\
घात् । एवं च "निष्पावाश्चात्र शोभना" इतिपूर्वोक्त मार्कण्डेयवचने\\
निष्पात्राभ्यनुज्ञानं कृष्णेतरविषयम् ।\\
निगमः,\\
यावनालानपि }{तथा}{ वर्जयन्ति विपश्चितः ।\\
तैलमप्यापदि प्राज्ञाः सम्प्रयच्छन्ति याज्ञिकाः ॥\\
तैलमपि वर्जयन्ति इत्यन्वयः \textbar{}\\
आपदि तु उभयोरभ्यनुज्ञानं "न प्राप्तस्य विलोपोऽस्ति पैतृकस्य\\
विशेषत" इति वचनादिति हेमाद्रिः ।\\
षट्त्रिंशन्मते,\\
विप्रुषका मसूराश्च श्राद्धकर्माणि गर्हिता ।\\
विप्रुषीका शालिविशेषाः ।\\
मरीचिः,\\
कुलत्थाश्चणका श्राद्धे न देयाश्चैव कोद्रवाः ।\\
कटुकानि च सर्वाणि विरसानि तथैव च ॥\\
कटुकानि = पिप्पलादीनि ।\\
विष्णुरपि वज्यान्याह - राजमाषमसूरपर्युषितकृतलवणानि च ।

{४० }{वीरमित्रोदयस्य श्राद्धप्रकाशे-}{\\
पर्युषितम्, अनिषिद्धमपि ।\\
शङ्खः,\\
राजमाषान्मसुरांच कोद्रवान्कोरदूषकान् ।\\
वर्जयेदिति वक्ष्यमाणेन सम्बन्धः । कोद्रवो = वनकोद्रवः \textbar{}\\
व्यासः,\\
अश्राद्धेयानि धान्यानि कोद्रवाः पुलकास्तथा ।\\
पुलका:-पुलाकाः, तुच्छधान्यानीति यावत् । छान्दसोऽत्र इस्वः \textbar{}\\
पद्मपुराणे-\\
कोद्रवोद्दालवरककुसुम्भमधुकातसीः।\\
एतानि नैव देयानि पितॄणां प्रियमिच्छता \textbar{}\textbar{}\\
उद्दालः = श्लेष्मातकः । कुसुम्भं=प्रसिद्धम् । मधुक ज्येष्ठीमधु । त\\
च्छाकसंस्कारकत्वेन प्रसज्यमानं प्रतिषिध्यते । अतसी = प्रसिद्धा\\
तस्या अपि शाकत्वेन तैलप्रकृतित्वेन वा निषेधः ।\\
मात्स्ये, -

{ }{कोद्रवोहालवरककपित्थमधुकातसीः ।}{\\
कपित्थं = दधित्थफलम् ।\\
}{ब्रह्मपुराणे,}{\\
सर्वश्राद्धेऽञ्जनं पुष्पं कुसुम्भं राजसर्षपाः ।\\
तोवरी राजमाषं च कोद्रवं कोरदूषकम् ॥\\
वर्ज्य चापक्रियं सर्वे निशि यत्वाहृतं जलम् ।\\
अजनपुष्पम्=अञ्जनद्रुमकुसुमम् । तोवरी :- तूवरीति प्रसिद्धा । अपकि\\
यम् = उचितक्रियारहितमनं न्यूनाधिकलवणदानादिना रसादिद्दीन.\\
मिति यावत् ।\\
मार्कण्डेयः,\\
वर्ज्याश्चाभिषवा नित्यं शतपुष्पं गवेधुकम् ।\\
अभिषवाः = सन्धानानि । शतपुष्पं = मिशि: \textbar{}\textbar{}\\
ब्राह्मे,\\
विप्रुषान्मर्कटांश्चारान्कोद्रवांश्चापि वर्जयेत् ।\\
कूर्मपुराणे,

{ }{आढकीकोविदारांश्च (१) पालक्यं मरिचं}{ तथा}{ ।}{\\
चारः=चरकः ।\\


% \begin{center}\rule{0.5\linewidth}{0.5pt}\end{center}

( १ ) पालङ्कुथामिति मयूखे पाठ ।

{ }{ग्राह्यमूलफलनिर्णयः । }{४१\\
वर्जयेत्सर्वयत्नेन श्राद्धकाले द्विजोत्तमः ॥\\
मरिचान्यार्द्राणि न शुष्काणीति हेमाद्रि ।\\
ब्रह्मवैवर्ते,\\
मसूराः शतपुष्पाश्च कुसुम्भं श्रीनिकेतनम् ।\\
वर्ज्याश्चातियवा नित्यं यथा वृषयवासकौ ॥\\
श्रीनिकेतनम् = रक्तबिल्वम् । अतियवाः = शालिभेदाः । बुषो वासा ।\\
यवासको दुरालभा । इति वर्ज्यानि धान्यानि ।\\
अथ ग्राह्माणि मूलफलानि ।\\
तत्र शङ्खः-\\
आम्रान् पालेवतानिक्षून्मृद्वीकाभव्यदाडिमान् ।\\
विदार्यांश्च भरुण्डांश्च श्राद्धकाले प्रदापयेत् ॥\\
दद्याच्छ्राद्धे प्रयत्नेन शृङ्गाटबिसकेवुकान् ।\\
आम्रान् = चूतफलानि । पालेवतं - जम्बीराकारं फलं काश्मीरेषु प्रसि-\\
द्धम् \textbar{} मृद्वीका=द्राक्षा \textbar{} भव्यं = कर्मरङ्गम्
\textbar{} विदार्थान् = भूकुष्माण्डीकन्दान् ।\\
श्राद्धचिन्तामणौ तु विदारीम् इति पाठः । विडालिकन्द इति च व्याख्या.\\
नम् । भरुण्डः = काश्मीरदेशे प्रसिद्धो जलप्रभवः कन्दविशेषः । शृङ्गाटकः=\\
सिङ्गाडा इति प्रसिद्धः । मखाणा इति श्राद्धमञ्जर्याम् । बिसं पद्मि\\
नीमूलम् \textbar{} केबुक = कवकनामा आर्द्रकसदृशः कन्दविशेषः ।\\
}{ब्रह्मपुराणे,}{\\
आम्रमाम्रातकं बिल्वं दाडिमं बीजपूरकम् ।\\
चीणाकं लकुचं जम्बु भव्यं भूतं तथारुकम् ॥\\
प्राचीनामलक क्षीरं नालिकेर परूषकम् ।\\
नारङ्गं च सखर्जूरं द्राक्षा नीलकपित्थकम् \textbar{}\textbar{}\\
पटोलं च प्रियालैलाकर्कन्धुबदराणि च ।\\
वैकङ्कत वत्सकं च एवरुर्वारुकानि च ॥\\
एतानि फलजातीनि श्राद्धे देयानि यत्नतः ।\\
आम्रातक= कपीतनस्य फलम् अवाडा इति प्रसिद्धम् । बिल्वं=बाल.\\
श्रीफलम् । चीणाकम् = अतिदीर्घाकारमेर्वारुक सदृशम् । लकुच= लिकु·\\
चफलम् वडहरइति प्रसिद्धम् । जम्बु = राजजम्बुफलम् । भूत=बहिःके.\\
सरावृतं फलं कर्णाटके देशे प्रसिद्धम् । आरुकं = वम्यमेषरुकम् ।\\
आरुकम्=आरु इति प्रसिद्धमित्यन्ये । प्राचीनामलकं = पानीयामलकम् ।\\
क्षीरं = राजादनफलम् । नालिकेलं=प्रसिद्धम् \textbar{} परूषक=
कोङ्कणदेशेप्रसिद्धम् ।\\
६ वी०मि०

{४२ }{वीरमित्रोदयस्य श्राद्धप्रकाशे-}{\\
नीलकपित्थकं = हरितवर्णकपित्थफलम् । पटोलं=स्वादुपटोलफलम् ।\\
प्रियालं = चारवृक्षस्य फलम् चिरोजीति प्रसिद्धम् । कर्कन्धु=चनबद\\
रीफलम् । वैकङ्कत = स्रुवद्रुमस्य फलम् । वत्सकं = कुटजस्य फलम्
\textbar{}\\
एर्वारु: = स्वादुकर्कटी । वारुकानि वा लुकीफलानि ।\\
}{तथा}{-\\
कालशाकं तन्दुलीयं वास्तुकं मूलकं }{तथा}{ ।\\
शाकमारण्यक चैव दद्यात्पुष्पाण्यमून्यपि \textbar{}\textbar{}\\
मागधी दाडिमं चैव नागरार्द्रकतिन्तिडी ।\\
आम्रातकं जीरकं }{च}{ कुम्बरं चैव योजयेत् \textbar{}\\
तन्दुलयो=अल्पमारिषः \textbar{} वास्तुक= कण्टकवास्तुकमिति हेमाद्रिः ।\\
मूलकशब्देन दीर्घमूलं ब्राह्यम् । पिण्डमूलकस्य स्मृत्यन्तरे निषेधा.\\
स्, कौर्मे ``दीर्घमूलकमेव च" इत्युक्तेश्च \textbar{} आरण्यकानि
फलीचुञ्चप्रभृ.\\
तीनि । अमूनि वक्ष्यमाणानि पुष्पाणि वेत्यर्थः । मागधी= पिप्पली
\textbar{}\\
नागरं = शुण्ठी । तिन्तिडी = तिन्तिणी । जारक= गौरजीरकम् । कुम्बरं=\\
कुस्तुम्बुरुः ।\\
कूर्मपुराणे-\\
बिल्वामकलकमृद्वीकं पनसाम्रातदाडिमम् ।\\
भव्यं पारावताक्षोटखर्जूराम्रफलानि च ॥\\
कसेरुं कोविदारं च तालकन्दं }{तथा}{ बिसम् ।\\
तमालं शतकन्दं च (१) मध्वालुं शीतकन्दली ।\\
कालेयं कालशाकं च सुनिषण्णः सुवर्चला ।\\
मांसं शाकं दधि क्षीरं चुञ्चुर्वेत्राऽङ्करस्तथा \textbar{}\textbar{}\\
कट्फलं कौङ्कणी द्राक्षा लकुचं मोचकं }{तथा }{।\\
उष्ट्रग्रीवं कचोरश्च तिन्दुकं मधुसाह्वयम् \textbar{}\textbar{}\\
वैकङ्कृत नारिकेलं शृङ्गाटकपरूषकम् ।\\
पिप्पली मरिचं चैव पटोलं बृहतीफलम् \textbar{}\textbar{}\\
सुगन्धगन्धिः सिञ्चन्ती कलायाः सर्व एव च।\\
एवमाद्यानि चान्यानि स्वादूनि मधुराणि च ॥\\
(२) नागरं चात्र वै देयं दीर्घमूलकमेव च ।

% \begin{center}\rule{0.5\linewidth}{0.5pt}\end{center}

( १ ) गन्धालूरिति श्राद्धकाशिकायां पाठः। गन्धालू =कर्चूरशाकमिति
व्याख्यातम्।\\
( २ ) नागरं चार्द्रकं देयमिति निर्णयसिन्धुद्धृत पाठ \textbar{}}

{ }{ }{वर्ज्यमूलफळनिर्णयः ।}{ ४३\\
आमलकं धात्रीफलम् \textbar{} पारावतं = पालेवतम् \textbar{} आक्षोटो -
द्वीपान्तरीय•\\
पीलुफलमिति हेमाद्रिः \textbar{} कसेरुः = रोमशः कन्दः कसेरुनाम्ना
प्रसिद्धः ।\\
कोविदारं = काञ्चनारस्य तत्सदृशस्य वा वृक्षस्य फलं पुष्पं च । तालकन्दः
=\\
तालमूलकन्दः । तमालं = तापिच्छम् \textbar{} शतकन्दं = शतावरी \textbar{}
मध्वालुः = मधु-\\
रैकरसः कन्दः, मोहालुरिति प्रसिद्धः । शीतकन्दली - शीतकन्द: शालू.\\
कमितियावत् । कालेय= करालाख्यः शाकः । कालशाकं पूर्वोक्तम् ।\\
सुनिषष्ण: = चाङ्गेरीसदृशो जलप्रभवः शाकविशेषः, सुणसुणेति प्रसिद्धः ।\\
सुवर्चला=आदित्यभक्ता । चुञ्चु= चुञ्चुरिति प्रसिद्धः । वेत्राङ्कुरः
प्रसिद्धः ।\\
कट्फल=कट्फलाख्यस्य वृक्षस्य फलम् । कौङ्कणी द्राक्षा = कोङ्कणदेशप्रभवा
।\\
मोचकं = कदली । उष्ट्रग्रीव\\
= उष्ट्रग्रीवाकृति फलमुत्तरापथे प्रसिद्धम् । विन्दुक=\\
शितिशारकस्य फलम् \textbar{} डिण्डिसमिति क्वचित् । मधुसाह्वयं =
ज्येष्ठीमधु,\\
मधूकफलं वा । वैकङ्कतादीनि व्याख्यातानि । बृहतीफलं = कण्टकारिका-\\
फलम् । सुगन्धगन्धिः = कर्पूरकचोरकादिः । कन्दविशेषः इत्यन्ये । सिञ्च\\
न्ती = रुदन्ती । नागर = शुण्ठी । दीर्घमूलक=पिण्डमूलकादन्यन्मूलकम् ।\\
वायुपुराणे,\\
अगस्त्यस्य शिखास्ताम्राः कषायाः सर्व एव च ।\\
सुगन्धिः मत्स्यमांसं च कलायाः सर्व एव च ॥\\
देया इत्यनुषङ्गः । अगस्त्यस्य = मुनिद्रुमस्य । शिखाः = किशलयानि
\textbar{}\\
ताम्राः = लोहितवर्णाः । कषायाः = कषायरसाः ।\\
प्रभासखण्डे,\\
कट्फलं कतकं द्राक्षा लकुचं मोचमेव च ।\\
प्रियालं काकमाची च तिन्दुकं मधुकाह्वयम् ॥\\
आरामस्य तु सीमान्ताः कलायाः सर्व एव तु ।\\
एवमादीनि चान्यानि शस्तानि श्राद्धकर्मणि ॥\\
कतकम् = जलप्रसादनम् । सीमान्ता = नवपल्लवाः । इति ग्राह्यमूलफलानि ॥\\
अथ वर्ज्यानि ।\\
वायुपुराणे,\\
वर्जनीयानि वक्ष्यामि श्राद्धकर्मणि नित्यशः ।\\
लशुनं गृञ्जनं चैव पलाण्डुं पिण्डमूलकम् \textbar{}\textbar{}\\
कलम्बा यानि चान्यानि हीनानि रसगन्धतः ।\\
पिप्पली मरिचं चैव पटोलं बृहतीफलम् ॥

{૪૪ }{वीरमित्रोदयस्य श्राद्धप्रकाशे-}{\\
वांशं करीरं सुरसमर्कजं भूस्तृणानि च ।\\
अवेदोक्ताश्च निर्यासा लवणान्यूषणानि च \textbar{}\textbar{}\\
श्राद्धकर्माणि वर्ज्यानि याश्च भार्या रजस्वलाः ॥\\
लशुनं गृञ्जन च पलाण्डोरेव भेदौ । उपलक्षण चैतदन्येषां भे.\\
दानाम् । लशुनादीनां पुरुषार्थे निषेधे स्थितेऽपि श्राद्धार्थत्वेन पुन.\\
निषेधः । गृञ्जनं = हरिद्रक्तवर्णः कन्दः । पिण्डमूलकं = पिण्डाकृतिमूलम्
।\\
कलम्बा-वेणुपत्राकृतिपत्रो जलजः शाकः, वर्त्तीलफला तुम्बी वा । क.\\
टुपिप्पलीमरीचयोरार्द्रयोः स्वतन्त्रशाकत्वेन प्रतिषेधः । यत्तु तयोः\\
"पिप्पली मरिचं चे'' त्यादिपुराणेऽभ्यनुज्ञानं तच्छाकसंस्कारकत्वेन\\
शुष्कविषयमिति हेमाद्रिवाचस्पतिमिश्रादयः । पटोलं लताफल न आरण्यं,\\
लतायां नपुंसकप्रयोगानुपपत्तेः । "पटोलस्तिक्तकः पटुः" इत्यनुशा\\
सने पुंल्लिङ्गप्रयोगात् । बृहतीफलं = क्षुद्रवार्त्ताकफिलम् । वाशः
करीरो=वं\\
शाङ्करः । सुरस = निर्गुण्डीफलं पत्रं वा । अर्जकं = श्वेतकुठेरकपत्रम्
\textbar{}\\
भूस्तृण=भूतीकसंज्ञ. शाकभेद: कश्मीरेषु प्रसिद्धः ।
स्मृतिचन्द्रिकाकारस्तु\\
यस्य तले ग्रन्थिस्थानेषु च परिमण्डला अवयवा भवन्ति स भूस्तृणः ।\\
अवेदोक्त इति= वेदाभ्यनुज्ञातनिर्यासव्यतिरिक्तनिर्यासाः । }{तथा }{च-\/-\\
तैत्तिरीयश्रुतिः, अथो खलु य एव लोहितोयोवा वृश्चनान्निर्येषति\\
तस्य नाश्यं काममन्यस्येति ।\\
व्रश्चन = छेदः । लोहितो अवश्चनजोऽपि ब्रश्चनजश्चालोहिनोऽपि ।\\
निर्येषति = निर्याति । एव च लोहित निर्यासाश्र श्राद्धे वर्ज्या इत्यर्थः
।\\
लवणानि = कृत्रिमणि अनिषिद्धानि च प्रत्यक्षाणि ।}{ तथा}{ च -\\
विष्णुः, -न प्रत्यक्षं लवणं दद्यात् ।\\
सैन्धवमानससम्भवयोस्तु प्रत्यक्षयोरव्यनिषेध उक्तो ब्रह्माण्डे,\\
सैन्धवं लवणं चैव }{तथा}{ मानससम्भवम् ।\\
पवित्रे परमे होते प्रत्यक्षे अपि नित्यशः ॥\\
ऊषणानि = शाकसस्कारकाणि मरीचाद्यभ्यनुज्ञातवर्जे निषिद्धानि \textbar{}\\
स्मृतिचन्द्रिकायां तु लवणान्यूषराणि चेति पाठः । लवणानि ऊषराणि\\
तदा ऊषरमृत्तिकाकृतानि लवणानीत्यर्थः ।\\
विष्णुः । भूतृणशिशुसर्षपसुरसार्जक कूष्माण्डालाबुवार्त्ताकपाल\\
ड्क्योपोदकीतण्डुलीयककुसुम्भमहिषक्षीराणि वर्जयेत् । "मालातृ-\\
णकभूस्तुणे" इत्यमरसिंद्दे सुडागमसहितस्यैव पाठात्तद्रहितस्यात्र

{वर्ज्यमूलफलनिर्णयः । }{४५\\
प्रयोगश्छान्दसः । शिशुः = शोभाञ्जनः स च श्वेतपुष्पः, रक्तपुष्पस्य\\
सामान्यतः प्रतिषेधेन प्राप्त्यभावादिति हेमाद्रिः । वस्तुतस्तु उभयो\\
रपि ग्रहणं रक्तस्य पुरुषार्थत्वेन पृथक् निषेधौचित्यम् । सर्षपोऽत्र रा.\\
जसर्षपः कृष्णसर्षपापरनामधेयः । 'कुसुम्भं राजसर्षपा' इति स्मृत्य\\
न्तरे वर्ज्यत्वेन तस्यैवाभिधानात्, अतश्च गौरसर्षपोऽत्र ग्राह्य एव ।\\
कूष्माण्डम् = कर्कारु \textbar{} अलाबु-तुम्बीफलम् । तच्च वर्तुलम्,
'अलाम्बु }{च}{\\
तुलं चे'ति वचनात् । वार्ताकं = क्षुद्रवार्त्ता कीफलम् ।
पालङ्कथा=गन्धद्रव्य.\\
विशेषः । कुसुम्भशब्देन कण्टकिकुसुम्भस्य ग्रहणम्, अकण्टकि\\
कुसुम्भस्य सर्वदा प्रतिषेधात् । क्वचितु पिप्पली ससुकभूस्तृणासु\\
रीसर्षपसुरसाकूष्माण्डलाबुवार्ताकपालकपालड्क्यातण्डुलयिककुसु•\\
म्भपिण्डमूलमहिषीक्षीराणि वर्जयेदिति विष्णुपुराणवचने पाठः \textbar{}
ससुकं=\\
खदिरशाकम् । अत्र वाचस्पति मिश्रैस्तम्ब कमिति पाठोलिखितः, बीज-\\
पूरकमिति च व्याख्यातम् । आसुरीसर्षणे= राजसर्षपः \textbar{} सुरसा=तुलसी
।\\
भक्ष्यत्वेनास्याः प्रतिषेध इति वाचस्पतिमिश्र ।\\
पालाशखदिराश्वत्थतुलसीघातकावटाः ।\\
एतान्याहुः पवित्राणि ब्रह्मज्ञा हव्यकव्ययोः ॥\\
इति पुष्पप्रकरणे देवलोक्तेः पुष्पत्वेन तु आदेयैव । सुरसा = निर्गु\\
ण्डीत्यन्ये ।\\
हारीतः,\\
पालड्यानालिकापोतीकाशिग्रुवार्ताकभूस्तृणकाफल्लमाषमसुरक.\\
तलवणानि च श्राद्धे न दद्यादिति ।\\
नालिका=जलोद्भवा शाकत्वेन प्रसिद्धा तमालदला वल्लिः \textbar{} पोती\\
का= उपोदकी परपर्याया निद्रातिशयकारिणी शाकत्वेन प्रसिद्धा ।\\
काफल्ल : =आरण्यशाकः काश्मीरेषु प्रसिद्धः ।\\
उशना:\\
नालिकासणछत्राककुसुम्भालाबुविड्भवान् ।\\
कुम्भीकञ्चुकवृन्ताककोविदाराश्च वर्जयेत् ॥\\
वर्जयेद् गृञ्जनं श्राद्धे काञ्जिकं पिण्डमुलकम् ।\\
करञ्ज येऽपि चान्ये वै रसगन्धोत्कटास्तथा ।\\
छत्राकं = शतपुष्पा, वर्षाकालोद्भवं छत्रं वा । विग्रहणं चापवि•\\
त्रभूप्रदेशस्योपलक्षणम् । तेन तदुद्भवं न देयम् \textbar{} कुम्मी =
श्रीपर्णिका ।\\
कञ्चुकं = वृन्ताकाकरमलाबुफलम् । यत्वस्मिन् वाक्ये पृथक् अलाबु.

{४६ }{ वीरमित्रोदयस्य श्राद्धप्रकाशे-}{\\
ग्रहणं तदलाबुपत्रादेः शाकत्वेन प्रतिषेधार्थम् । वृन्ताकं = श्वेतवृत्ता\\
कम् । गुञ्जनं= गाजरमिति प्रसिद्धमिति हेमाद्रिः । काञ्जिकम् = आरनालम् ।\\
कर}{ञ्जं}{ = चिरबिल्वफलं । " चिरबिल्वोरक्तमालः करजश्च करञ्जके"\\
इत्यमरे ऽभिधानात् ।\\
भारद्वाजः,\\
नक्तोद्धृतन्तु यत्तोयं पल्वलाम्बु तथैव च ।\\
स्वल्पन्तु कूष्माण्ड फलं वज्रकन्दश्च पिप्पली ॥\\
तण्डुलीयकशाकं च माहिषं च पयोदधि \textbar{}\\
शिम्बिकानि करीराणि कोविदारगवेधुकम् ॥\\
कुलत्थसणजीराणि करम्भाणि तथैव च ।\\
अब्जादन्यद्रक्तपुष्पं शिग्रुक्षारस्तथैव च ॥\\
नीरसाण्यपि सर्वाणि भक्ष्यभोज्यानि चैव हि ।\\
एतानि नैव देयानि सर्वस्मिन् श्राद्धकर्माणि ॥\\
नक्तोद्धृतं = रात्रौ जलाशयादुद्धृतम् । स्वल्पमिति पल्वलविशेषणम् ।\\
एकस्या गो: तृप्तेर्यावत् पर्याप्त जलं तन्न ग्राह्यमित्यर्थः । वज्रकन्दः
=\\
आरण्यसुरणः । शिम्बिकानि= शिम्बिसंवन्धिधान्यानि प्रतिषिद्धानि ।\\
करीराणि= बदराकृतिफलानि । जीरकं = कृष्णजीरकम् ।\\
}{तथा}{ च व्यासः,\\
कृष्णाजाजी विडं चैव शीतपाकी तथैव च ।\\
कृष्णाजाजी = कृष्णजीरकम् । विड=लवणम् । शीतपाकी\\
काकजङ्का । कर\\
म्भाणि = दधिमिश्राः सक्तव । क्षार=यवक्षारादि । सर्वस्मिन्=
नित्यश्राद्धाद-\\
न्यत्र । 'यदन्नं पुरुषोभुङ्गे तदन्नास्तस्य देवता' इति नित्यश्राद्धं
प्रक्र.\\
स्य रामायणोक्तेः ।\\
सुमन्तुः । बीजपूरकमाषांश्च श्राद्धे न दद्यात् । बीजपूरो = मातु-\\
लुङ्गकः । 'फल पूरी बीजपूरो रुचको मातुलुङ्गक' इति अमरसिंहे\\
ऽभिधानात् ।\\
}{ब्रह्मपुराणे,}{\\
फले करुणकाकाले बहुपुत्रार्जुनफिलम् ।\\
जम्बीरं रक्तबिल्वञ्च शालस्यापि फलं त्यजेत् ॥\\
अलाबुः तिक्तपर्णी च कूष्माण्डी कटुपत्रिका ।\\
वार्ताकं शिम्बिजातं च लोमशाश्चिर्भटानि च ॥

{वर्ज्यमूलफलनिर्णयः ।}{ ४७\\
कालिङ्गं रक्तचारं च चीणाकं धृतचारकम् ।\\
श्राद्धकर्मणि वर्ज्यानि गन्धचिर्भटिकं}{ तथा}{ ॥\\
कपालं काचमांची च करक्ष पिण्डमूलकम् ।\\
गृञ्जन चुक्रिकां चुक्र गाजरं पातिकां }{तथा}{ ॥\\
जीवकं शतपुष्पा च नालिकां ग्राम्यशूकरम् \textbar{}\\
हलं नृत्यं सर्षपं च पलाण्डुं लशुनं त्यजेत् ॥\\
माणकन्दं विषकन्दं तथैव च गतास्थिगम् ।\\
पुरुषालं सपिण्डालु श्राद्धकर्मणि वर्जयेत् ॥\\
नोग्रगन्धं च दातव्यं कोविदारकशिप्रकौ ।\\
अत्यम्ल पिच्छिलं रूक्ष अन्यच मुनिसत्तम ॥\\
नैवदेयं गतरसं मद्यगन्धं च यद् भवेत् ।\\
हिङ्गुग्रगन्धा पनस भूनिम्बं निम्बराजिके ॥\\
कुस्तुम्बरं कलिङ्गोत्थ वर्जयेदम्लवेतसम् ।\\
पिण्याकं शिग्रुकं चैत्र मसूरं गृञ्जनं शणम् ॥\\
कोद्रवं कोकिलाक्षं च चुक्रं कम्बुच पद्मकम् ।\\
चकोरश्येनमांसं च वर्तुलाला बुनालिनी \textbar{}\textbar{}\\
फलं तालतरूणां च भुक्त्वा नरकमृच्छति ।\\
दत्वा पितृभ्यः तैः सार्द्धं व्रजेत् पृयवहं नरः \textbar{}\textbar{}\\
तस्मात् सर्वप्रयत्नेन नाहरेत विचक्षणः ।\\
निषिद्धानि वराहेण स्वयं पित्रर्थमादरात् ॥\\
करुणं=करुणजातीयं वृक्षफलं गुर्जरदेशे प्रसिद्धम् । काकोल= काकोली\\
फलम् उत्तरापथे प्रसिद्धम्। बहुपुत्रा = शतावरी । अर्जुनीफलं -
ककुभताफलम्\\
शाल :- सर्जः। अन्ते वर्तुलाला बुनिषेधात् पुनरत्राला बुग्रहणं उभयालाबु\\
निषेधार्थमिति पृथ्वचिन्द्रः ॥ तिक्तपर्णी कद्रुकरोहिणी \textbar{}
कटुपत्रिका = राजिका ।\\
रामशानि= कपिजम्बूपलानि । निर्भर्ट= गौरक्षाख्यः कर्कटीभेदः । रक्तचारं =\\
लोहितचारफलम् । धृतचारकं = चिरसंगृहीतचारफलम् । गन्धचिर्भटिकम् =\\
गन्धयुक्तचिर्भटम् । कपाल= नारिकेलम् । गृजन = हरिद्रक्तवर्णः पलाण्डु\\
भेदः । चुक्रिका चाङ्गेरी । 'चाङ्गेरी चुक्रिका
दन्तशठाऽम्बष्ठाम्ललोणिका'\\
इति अमरेऽभिधानात् । गुडमधुकासिकमस्त्वादिद्रवद्रव्यं धान्याद्यू\\
ष्मणि त्र्यहं सप्ताहादि कालं वा अवस्थाप्यातिशुक्ततां नीतं चुक्रमु·\\
च्यते । गाजरं=गाजरमिति प्रसिद्धं । जीवकः - जीवसंज्ञयैवोत्तरापथे प्र.

{४८ }{वीरमित्रोदयस्य श्राद्धप्रकाशे-}{\\
सिद्धः । हलं = जलपिप्पली \textbar{} नृत्यं = नटी \textbar{}
माणकन्दो=महच्छद इति मदन-\\
निघण्टुः । विषकन्दः = महिषीकन्दाख्यः, जुचालुरिति राजनिघण्टुः। गता•\\
स्थिगं=अस्थिकारहितं यत् किञ्चित् फलम् । पुरुषाल=\\
लक्ष्मणा कन्दः पुस्क\\
न्द इति राजनिघण्टुः । पिण्डालुः प्रसिद्धः । उप्रगन्धा=वचा \textbar{}
भूनिम्बः = किरा\\
ततिक्तकः । कुस्तुम्बर=कुवरम् \textbar{} पिण्याण= तिलकल्क: । कोकिलाक्षं =
इक्षुरः ।\\
पद्मक. = पद्मकनाम्ना वैद्यानां प्रसिद्धः । चकोरश्यनौ-पक्षिविशेषौ ।\\
मार्कण्डेयपुराणे-\\
गन्धारिकामलाबूनि लवणान्यौषणानि च ।\\
वर्जयेत्तानि वै श्राद्धे यत्र वाचा न शस्यते ॥\\
दात्रा प्रदानकाले सत्कारार्थ वाचा यन्न स्तूयते तदपि श्राद्धसा\\
द्गुण्याय न भवतीत्यर्थः ।\\
}{तथा}{-\\
पूति पर्युषितं चैव वार्ताकाभिषवा स्तथा ।\\
वर्जनीयानि वै श्राद्धे }{तथा}{ वस्त्र च लोहितम् \textbar{}\textbar{}\\
पूति = दुर्गन्धम् । पर्युषितं = गोधूमविकाराद्यपि । क्वचित्त तथा वसु
च\\
लोहितमिति पाठः । तदा लोहितं वसुकाख्यं शाकम् । यद्वा लोहित-\\
वर्ण वसु द्रव्यं करवीरपुष्पादि ।\\
कूर्मपुराणे -\\
पिप्पलीं रुचकं चैव }{तथा}{ चैव मसूरकम् ।\\
कुसुम्भपिण्डमूलं च तण्डुलीयकमेव च ॥\\
वर्जयेदिति वक्ष्यमाणेन सम्बन्धः \textbar{} रुचकं = सौवर्चलम् ।\\
पद्मपुराणे -\\
पद्मबिल्वकघत्तरपारिभद्राढरूषकाः ।\\
न देयाः पितृकार्येषु पयश्चाजर्विकं }{तथा}{ \textbar{}\textbar{}\\
पारिभद्रो = निम्बः \textbar{} आढरूषकः = सिंहास्यः, वासा इति प्रसिद्धः
।\\
व्यास:-\\
अवक्षुतावरुदितं }{तथा}{ श्राद्धेषु वर्जयेत् ।\\
ब्रह्माण्डपुराणे -\\
सुगन्धं पञ्चशिम्बञ्च कलायाः सर्व एव च ।\\
सुगन्धं = गन्धनाकुल्याः पत्रादि । निष्पावमसुरराजमाषमठकुलु.\\
त्थाण्यानि पञ्चशिम्बधान्यानि ।

{ग्राह्माग्राह्यक्षीरनिर्णयः । }{ ४९\\
अत्रेयं व्यवस्था श्राद्धप्रकरणे यद् द्रव्यं फलविशेषार्थमुपात्तं तत्\\
फलविशेषार्थिना देयं, यस्य तु फलसंयोगोनास्ति केवलं श्राद्धप्रक\\
रणे विधानं तस्य श्राद्धस्वरूपसम्पादकत्वेन श्राद्धाङ्गत्वम् । तदलाभे\\
तु अविहिताप्रतिषिद्धं विहितसदृशं देयम् । यस्य तु प्रकरणे निषेधः\\
तत् प्रतिनिधित्वेनापि नादेयम् । यस्य तु श्राद्धप्रकरणे विधिप्रतिषेधौ,\\
यथा बहुपुत्रापनसचीणाकतण्डुलीयचुक्रिकाज्येष्ठी\\
मधुहिङ्गुकोविदार\\
कुस्तुम्बरनारिकेलबीजपुरकादौ । तत्र प्रहणाग्रहणवद् विकल्पः । इति\\
वर्ज्यानि मूलफलानि ।\\
अथ }{ग्राह्याणि}{ क्षीराणि ।\\
पद्मपुराणे-

{ अनं च सदधिक्षीरं गोघृतं शर्करान्वितम् ।\\
मास प्रीणाति वै सर्वान् पितॄनित्यब्रवीदजः \textbar{}\textbar{}\\
मनुः ।\\
संवत्सरं तु गव्येन पयसा पायसेन च ।\\
पायसेन=परमानेन \textbar{}\\
सुमन्तुः ।\\
पयोदधि घृतं चैव गवां श्राद्धेषु पावनम् \textbar{}\\
महिषाणां घृतं प्राहुः श्रेष्ठं न तु पयः क्वचित् ॥\\
देवलः -\\
अजाविमहिषीणान्तु पयः श्राद्धेषु वर्जयेत् ।\\
विकारान् पयसश्चैव माहिषं तु घृतं हितम् ॥ इति क्षीराणि प्रायाणि ।\\
अथ वर्ज्यानि ।\\
भविष्यत्पुराण-\\
श्राद्धे तु महिषीक्षीरं अजाक्षीरं च वर्जयेत् ।\\
गवां चानिर्दशाहानां सन्धिनीनां पयस्त्यजेत् ॥\\
या कालद्वये प्राप्तदोहा एकदैव दुह्यते सा सन्धिनी । यद्वा वृषेण\\
सन्धिं प्राप्ता सा सन्धिनी । मृतवत्सा वा वत्सान्तरेण सन्धीयमाना ।\\
मार्कण्डेये-\\
मार्गमाविकमौष्ट्रं च सर्वमैकशफं च यत् ।\\
माहिषं चापरं चैव धेन्वा गोश्चाप्यनिर्दशम् ॥\\
वर्जनीयं सदा सद्भिः तत् पयः श्राद्धकर्मणि ।\\
७ बी० मि०

५०  वीरमित्रोदयस्य श्राद्धप्रकाशे-

{गौतम: -\\
स्यन्दिनीयमसूसन्धिनीनां च याश्च वत्सव्यपेताः । क्षीरं अपेय.\\
मित्यनुषङ्गः । स्यन्दिनी = स्वयमेव क्षरत्क्षीरा, स्रवयोनिर्वा ।
वत्सव्यपेता =\\
अवत्सा, वत्सं विना वा दुग्धा । "न हतवत्सायाः शोकाभिभूतत्वात्,\\
न दुग्धाया विना वत्सात्", इति हारीतवचनात् । अत्र शोकापगमे न\\
निषेधः हेतुमन्निगदात् । इति वज्र्ज्योनि ।\\
अथ मांसानि ।\\
मनुः,

{ मुन्यन्नानि पयः सोमो मांसं यञ्चानुपस्कृतम् ।\\
अक्षारलवणञ्चैव प्रकृत्या हविरुध्यते ॥ ( अ० ३ श्लो० २५७ )

{तथा}{,\\
पितॄणां मासिक श्राद्धं अन्वाहार्थ विदुर्बुधाः ।\\
तदामिषेण कर्तव्यं प्रशस्तेन प्रयत्नतः ॥ (अ० ३ श्लो० १२३)\\
मासिकं = प्रतिमासभवममावास्याश्राद्धं तन्मुनयोऽन्वाहार्यमिति वि.\\
दुः । तदामिषेण मांसेन प्रशस्तेनाप्रतिषिद्धेन विशेषविहितेन वा\\
कर्तव्यमित्यर्थः । अयं च मुख्यः कल्पः । }{तथा}{ च स्मृत्यन्तरम् सर्पि\\
मांसं च प्रथमः कल्पः । अभावे तैलं शाकमिति । मांसं च व्यञ्जन-\\
त्वेन देयं न तु स्वातन्त्र्येण, गुणांश्च सूपशाकाद्यान् इति वचनात् ।\\
स्मृत्यन्तरे,\\
विना मांसेन यच्छ्राद्धं कृतमप्यकृतं भवेत् ।\\
क्रव्यादाः पितरोयस्मादलाभे पायसादयः ॥

{तथा}{,\\
मांसं शाकं दधि क्षीरं मधु मुन्यन्नमेव च ।

{शङ्ख: ,\\
तित्तिरि च मयूरं च लावकं च कपिञ्जलम् ।\\
वाव्रणसं वर्तकं च भक्ष्यान्याह यमः सदा ॥\\
तित्तिरिः = चित्रपक्षः । मयूरः शिखण्डी \textbar{} लावकः = प्रसिद्धः ।
कपिञ्जल =\\
वभ्रुवर्ण: पक्षी । गौरतित्तिर इति केचित् । वाघ्रींणसः पूर्वोदाहृतनिग\\
मोक्तलक्षणः । वर्तकः = वृत्ताकारः पक्षिविशेषः ।\\
मनुः,\\
पाठीनरोहितावाद्यौ नियुक्तौ हव्यकव्ययोः ।\\
राजीवसिंहतुण्डांश्च सशल्कांश्चैव सर्वशः ॥ (अ० ५ श्लो०१६ )

{ }{कालविशेषावच्छेदेनतृप्तिकरपदार्थ निर्णयः ।}{ ५१\\
हव्यकव्ययोर्विनियुक्तो आद्यौ अदनीयौ । पाठीनः = चन्द्रकाख्यः ।\\
रोहितो = लोहितवर्णः । राजीवः पद्मवर्णः । सिंहतुण्डा = सिंहमुखा: । सश-\\
ल्काः = पृष्ठप्रतिष्ठितशुक्त्याकारशकलाः ।\\
अथ कालविशेषावच्छेदेन तृप्तिकराणि ।\\
मनु', (अ० ३ श्लो० २ आरभ्य २७ पर्यन्ताः)\\
तिलव्रीहियवैर्मासैरद्भिर्मूलफलेन वा ।\\
दत्तेन मासं प्रीयन्ते विधिवत् पितरानृणाम् ॥\\
द्वौ मासौ मत्स्यमांसेन त्रीन् मासान् हारिणेन तु ।\\
औरभ्रेणाथ चतुरः शाकुनेनेह पञ्च तु ॥\\
षण्मासान् छागमांसेन पार्षतेनाथ सप्त तु ।\\
अष्टावेणस्य मांसेन रौरवेण नवैव तु ॥\\
दशमासांस्तु तृप्यन्ति वराहमहिषामिषैः ।\\
शशकुर्मयोमसेन मासानेकादशैव तु \textbar{}\textbar{}\\
संवत्सरन्तु गव्येन पयसा पायसेन तु ।\\
व्याघ्रणसस्य मांसेन तृप्तिर्द्वादशवार्षिकी \textbar{}\textbar{}\\
कालशाकं महाशल्काः खड्गलोहामिषं मधु ॥\\
आनन्त्यायैव कल्पन्ते मुन्यन्नानि च सर्वशः \textbar{}\\
उरम्रो = मेषः । महिषो = लुलायः \textbar{} आमिषं = मांसम् । एतस्य च
विकल्पः,\\
"माहिषाणि च मांसानि श्राद्धेषु परिवर्जयेत्" इति विष्णुपुराणे प्रतिषे-\\
धात् । आवश्यकमांसाभावेऽनुकल्पविधानमिति केचित् । तत्राव.\\
श्यकहारिणादिभ्योऽधिकप्रशंसानुपपत्तेः ।\\
मार्कण्डेयपुराणे,\\
विदायैस्तु परुषैश्च बिसैः शृङ्गाटकैस्तथा ।\\
कंचुकैश्च }{तथा}{ कन्दैः कर्कन्धुक्दरैरापे \textbar{}\textbar{}\\
पालेवतैरारुकैश्चाप्यक्षोटैः पनसैस्तथा ।\\
काकोल्या क्षीरकाकोल्या }{तथा}{ पिण्डालुकैः शुभैः ।\\
लाजाभिश्च सधानामिस्त्रपुसैर्वारुचिर्भटैः ।\\
सर्षपाराजशाकाभ्यां इङ्गुदैराजजम्बुभिः ॥\\
प्रियालामलकैर्मुख्यैः फल्गुभिश्च विलम्बकैः ।\\
वंशाङ्कुरैस्तालकन्दैश्चुक्रिकाक्षीरिकावचैः ॥

{५२ }{वीरमित्रोदयस्य श्राद्धप्रकाशे-}{\\
वोचैः समोचैर्लकुचैस्तथा वै बीजपूरकैः ।\\
मुञ्जातकैः पद्मफलैर्भक्ष्यभोज्यैस्तु संस्कृतैः ।\\
रागखाण्डवचोष्यैश्च त्रिजातकसमन्वितैः ।\\
दत्तैस्तु मासं प्रीयन्ते श्रद्धेषु पितरोनृणां }{।}{\\
कन्दः =सूरणः । काकोलीक्षीरकाकोल्यौ गौडदेशे प्रसिद्धे । सर्ष\\
पेति स्त्रीलिङ्गतया निर्देश: छान्दसः । राजशाकं = राजयक्षवाक्यं\\
शाकम् । इxxxदी= तापसतरुः \textbar{} राजजम्बु - जम्बूविशेषः \textbar{}
मुख्यान्यामलका-\\
नि= स्थूलाम्यामलकानि । फल्गु = क्षुद्रामलकम् \textbar{}
विलम्बकानि=पटोलानि \textbar{}\\
मुञ्जातक गौडदेशे प्रसिद्धम् । शर्करामध्वादिद्रव्ययोगेन मधुरीकृता.\\
रसा रागाः । अम्लसंयोगेन खाण्डवाः ।\\
कात्यायनः-\/-\/-\/-\\
अथ तृप्तिः, ग्राम्याभिरोषधीभिर्मासं तृप्तिरारण्याभिर्वा, तदलाभे\\
मूलफलैरद्भिर्वा सहान्नेनोत्तरास्तर्पयन्ति छागोस्त्रमेषा आलब्धव्याः
\textbar{}\\
शेषाणि क्रीत्वा लध्वा स्वयं मृतानां वा हृत्य, यवेन मासम्, मासद्वयं,\\
मत्स्येन, त्रीन् मासान् हारिणेन मृगमांसेन, चतुः शाकुनेन, पञ्च
रौरवे\\
ण, षट् छागेन, सप्त कौर्मेणाष्टौ वाराहेण, नव मेषमांसेन, दश मा\\
द्विषेणैकादश पार्षतेन, सम्वत्सरं तु गव्येन पयसा पायसेन वा, वा\\
र्ध्रीणसस्य मासेन द्वादशवर्षाणि अक्षया तृप्तिः । खङ्गः कालशाकं\\
लोहछागोमधुमहाशल्को वर्षासु मघासु च श्राद्धं हस्तिछायायाम् ।\\
अथ तृप्तिरित्यनन्तरं उच्यते इत्यध्याहारः । प्राम्याभिरोषधीभिः =यवगो-\\
धूमादिभिः सकृद्दत्ताभिर्मास तृप्तिर्भवतीत्यर्थः । सम्पूर्ण
तृप्तिपर्याप्त\\
ग्राम्यारण्यालाभे मूलफलैरद्भिर्वा मासं तृप्तिः । अत्रादिभिर्मासं
तृप्ति\\
मुक्ता मांसैर्द्विमासादितृतिं वक्तुं तदुपादानविधिमाह छागइत्या-\\
दिना । उत्तरा मूलफलादयोऽन्नेन स्वल्पेन सहैव दत्तास्तर्पयन्ति न\\
केवलाः । उस्त्रोऽनड्वान् । अन्यौ प्रसिद्धौ । इदं च ग्राम्योपलक्षणम्
\textbar{}\\
ते च प्रोक्षणादिसंस्कारपूर्वकं आलब्धव्या: संज्ञपनीया इत्यर्थः ।\\
मारण्यानां पशूनां तु प्रोक्षणादि संस्कारमन्तरेणापि क्षत्रियादिना\\
स्वयं परेण वा हतानां क्रयाद्युपायसम्पादितं मांसं श्राद्धादौ देयम् ।\\
}{तथा}{ च । पुलस्त्यः,\\
वर्जयेद्दुरतः श्राद्धे यदप्रोक्षितमामिषम् ।

{ }{कालविशेषावच्छेदेनतृप्तिरपदार्थनिर्णयः । }{ ५३\\
राजानुत्यादितं यच्च व्याधिनाभिहताच्च यत् ॥\\
अप्रोक्षितं = प्रोक्षणादिरहितम् । राजानुत्पादितं मृगव्येन स्वयमनुत्पा.\\
दितं वर्जयेत् । व्याधिनाभिहतात् पशोर्यत् गृहीतं तदपि वर्जयेत्\\
इत्यर्थः । अयं च अप्रोक्षितनिषेधोग्राम्यपशुविषयः आरण्यास्तु प्रो.\\
क्षणमन्तरेणापि प्रशस्ता एवेति ब्रह्मपुराणे ।\\
आरण्यानां तु सर्वेषां प्रोक्षणं ब्रह्मणा कृतम् ।\\
अत एव तु ते भक्ष्या ब्राह्मणक्षत्रियादिभिः ॥

{हारीतः,\\
क्षत्रियैस्तु मृगव्येन विधिना समुपार्जितम् ।\\
श्राद्धकाले प्रशंसन्ति सिंहव्याघ्रहतं च यत् ॥\\
मनुः, ( अ० ५ श्लो० ३२ )\\
क्रीत्वा स्वयं वाप्युत्पाद्य परोपहृतमेव च ।\\
देवान् पितृश्चार्चयित्वा खादन् मांसं न दुष्यति ॥\\
अतश्च ग्राम्याणा प्रोक्षणादि कृत्वैवालम्भनम् । आरण्यानां तु\\
प्रोक्षणादिसंस्कारमन्तरेणैव क्षत्रियादिना स्वयं परेण वा हतां क्रि\\
यादिभिः सम्पाद्य मांसं देयम् ।\\
उशना:\\
तत्र व्रीहियवैर्माषैरर्चिता मासं पितरस्तृप्ता भवन्ति मासार्द्ध -\\
मात्स्येन, त्रयं हरिणमृगमांसेन, चतुरोमासान् कृष्णसारङ्गेण, }{पञ्च}{\\
शाकुनेन,}{ षट्}{ छागेन, सप्त पार्षतेन, अष्टौ वराहेण, नव रुरुणा, मेषेण\\
दश, एकादश कूर्मेण, पायसेन पयसा गव्येन सम्वत्सरं, वार्ध्रीण\\
सस्य मांसेन तृप्तिर्द्वादश वार्षिकी । खङ्गमांसेन आनन्त्यं अपि चो.\\
दाहरन्ति ।\\
खड्गश्च कालशाकं च लोहछागं तथैव च ।\\
महाशल्कश्च मध्वनं पित्र्येऽनन्त्याय कल्पते ॥

{यमः,}{\\
अद्भिर्मूलफलैः शाकैः पुण्यव्रीहियवैस्तथा ।\\
प्रीणाति मासं दत्तेन श्रद्धेनेह पितामहान् ।\\
मत्स्यैः प्रीणाति द्वौ मासौ त्रीन् मासान् हारिणेन तु ।\\
शल्यकश्चतुरोमासान् रुरुः प्रीणाति }{पञ्च}{ च ॥\\
शशः प्रीणाति षण्मासान् कूर्मः प्रीणाति सप्त तु ।\\
अष्टौ मासान् वराहस्तु मेषः प्रीणयते नव ॥

{५४ }{ वीरमित्रोदयस्य श्राद्धप्रकाशे-}{\\
माहिषं दश मासांस्तु गवय रुद्रसम्मितान् ।\\
गव्यं द्वादशमासांस्तु पयः पायसमेव च ॥\\
वार्ध्रीणसस्य मांसेन तृप्तिर्द्वादशवार्षिकी ।\\
आनन्त्याय भवेदन्तं खड्गमांलं पितृक्षये ॥\\
पितृक्षयोगया ज्ञेया तत्र दत्तं महाफलम् ।\\
कालशाकश्च खड्गश्च लोहछागस्तथैव च ॥\\
महाशकलमत्स्याश्च पित्र्येऽनन्त्याय कल्पिताः ।\\
यत् किश्चिन्मधुसंयुक्तं तदानन्त्याय कल्पते ॥\\
उपाकृतन्तु विधिना मन्त्रेणान्नं }{तथा}{ कृतम् ।\\
गवयो गो सदशः पशुः तस्य मांसं गावयम् \textbar{}\\
नागरखण्डे,\\
अप्राप्तौ खड्गमांसस्य }{तथा}{ वा}{र्ध्री}{णसस्य वा ।\\
मधुना सह दातव्यं पायसं पितृ तृप्तये ॥\\
तेनापि वार्षिकी तृप्तिः पितृणाञ्चोपजायते ।\\
अभावे चापि तस्यापि शिशुमारसमुद्भवम् ॥\\
मांसं तु तृप्तये प्रोक्तं वत्सरं मांसवर्जितम् ।\\
तदभावे वराहोत्थं दशमासप्रतुष्टिदम् \textbar{}\textbar{}\\
आरण्यमाहिषोत्थेन तृप्तिः स्यान्नवमासिकी ।\\
रुरुश्चैवाट मासान् वै एणे स्यात् सप्तमासिकी ॥\\
छागस्य मासषट्कन्तु शशकस्य }{च}{ पश्ञ्च वै ।\\
चतुरः शल्यकस्योक्तास्त्रयोविष्किरिकस्य च ॥\\
मासद्वयस्य मत्स्यस्य मांसं कापिञ्जलस्य च ।\\
नान्येषां योजयेन्मांसं पितृकार्ये कथञ्चन ॥\\
एतेषामपि चाभावे पायसेन नराधिप ।\\
अथान्नानि,\\
प्रचेताः । पायसतिलकृशरब्रह्मसुवर्चलाहरितमुद्रकृष्णमाषश्या.\\
माकप्रियङ्गुष्यवगोधूमेक्षुविकारान् दद्यात् । पायसं = पयसिसिद्ध ओ.\\
दनः । तिलतण्डुसिद्धओदनः कृशरम् \textbar{} ब्रह्मसुवर्चला=आदित्यभक्ता ।\\
विकारशब्दः प्रत्येकं सम्बध्यते । इदञ्चोपलक्षणं तेन यावद्विहितं द्रव्यं\\
तस्य सर्वस्य विकारान् दद्यादित्यर्थः । अत्र विकारपदश्रवणान्न\\
गोधूमव्रीहियवादीनां साक्षात् प्रदेयत्वं किन्तु व्रीहीणां पुरोडाश.

{काळविशेषावच्छेदेनतृप्ति}{कर}{पदार्थनिर्णयः । ५५}{\\
प्रकृतित्ववत् द्विजातिकर्तृकश्राद्धे ब्राह्मणभोजन साघनीभूतान्नप्रकृति-\\
त्वमेव \textbar{} आमश्राद्धे तु वचनात् साक्षात् प्रदेयता । अत्र च
कृशरा.\\
दीनां सम्भवत् समुच्चयः, भोजनसाधनत्वेन लोके तथैवावगतत्वात्।\\
देवलः,\\
ततोऽन्नं बहु संस्कारनैकव्यञ्जनभक्ष्यवत् ।\\
चोष्यपेयसमृद्धं च यथाशक्त्युपकल्पयेत् \textbar{}\textbar{}\\
सस्कारा=मरिचादयः । व्यञ्जनं पादि । दन्तैरवखण्डय यद्भश्यते\\
तद् भक्ष्यम् । चोष्यम् = इक्षुखण्डादिक । पेयं=पानकादि ।\\
ब्रह्मपुराणे-\\
गुडशर्करमत्स्यण्डी देयं फाणितमुर्मुरम् \textbar{}\\
गव्यं पयोदधि घृतं तैलं च तिलसम्भवम् ॥\\
शर्कराभेदो= मत्स्यण्डी । ईषत् कथितस्येक्षुरसस्य द्रव एव विकार:\\
फाणितम् । गुडमरिचैलामिश्रोगोधूमस्थूलचूर्णबिकाशे मुर्मुरुः । तैलस्य\\
विधिर्दीपादौ शाकपाकार्थे अभ्यङ्गादौ च ज्ञेयः ॥\\
}{तथा}{,\\
पायसं शालिमुद्राद्यं मोदकादींश्च भक्तितः ।\\
पूरिकां च रसालां च गोक्षीरं च नियोजयेत् ।।\\
यानि चाभ्यवहार्याणि स्वादुस्निग्धानि भोद्विजाः ।\\
ईषदुष्णकटून्येव देयानि श्राद्धकर्मणि ॥\\
मोदको=लडुकः ।

{ कूर्मपुराणे,\\
लाजान् मधुयुतान दद्यात् सक्तून् शर्करया सह ।\\
दद्याच्छ्राद्धे प्रयत्नेन शृङ्गाटकबिसे तथा ॥

{वायुपुराणे,\\
भक्ष्यान् वक्ष्ये करम्भञ्च इष्टका घृतपूरिका \textbar{}\\
कृशरं मधुसर्पिश्च पयः पायसमेव च ॥\\
स्निग्ध उष्णञ्च योदद्यात् अनिष्टोमफलं लभेत् ।\\
इष्टका = इष्टकाकृतिः खण्डेष्टकाख्याभक्ष्यविशेषः ॥

{सौरपुराणे,\\
विविधं पायसं दद्यात् भक्ष्याणि विविधानि च ।\\
लेह्यं चोष्यं यथाकाममुष्णमेव फलं विना ॥\\
विविधान्यपि मांसानि पितॄणां पितृपूर्वकम् \textbar{}}

{५६ }{वीरमित्रोदयस्य श्राद्धप्रकाशे-}{\\
अत्राग्निपाकादनीयात्फलादन्यत् फलम् तस्य तु कटूष्णस्यैव\\
स्वादुत्वात्तादृशस्यैव दानमिति हेमाद्रिः । फलमित्युपलक्षणं अपक्व\\
कन्दमूलफलादीनाम् । अत एव । कौर्मे,\\
उष्णमन्नं द्विजातिभ्योदातव्यं श्रेय इच्छता ।\\
अन्यत्र फलमूलेभ्यः पानकेभ्यस्तथैव च ।\\
वायुपुराणे,\\
घृतेन भोजयेत् विप्रान् घृतं भूमौ समुत्सृजेत् ।\\
शर्कराः क्षीरसंयुक्ताः पृथुका नित्यमक्षयाः ॥\\
स्युश्च संवत्सरं प्रीता वर्करैर्मेषकैणकैः ।\\
सक्तून् लाजांस्तथा पूपान् कुल्माषान् व्यञ्जनैः सह ॥\\
सर्पिः स्निग्धानि सर्वाणि दग्धा संस्कृतभोजयेत् ।\\
श्राद्धे श्वेतानि योदद्यात् पितरः प्रीणयन्ति तम् ।\\
घृतं भूमौ समुत्सृजेदित्यस्यायमर्थः }{तथा}{ घृतं परिवेष्यं यथा\\
पात्रमापूर्य भूमावुपसर्पति । वर्करैः =रुतरुणैः ।\\
देवेल:,\\
भोजनैः सतिलैः स्नेहैर्भस्यैः पूपविमिश्रितैः ।\\
मैत्रायणीये, तिलवन्मधुमश्चान्नं सामिषं दद्यात् । पिप्पल्यादयः । प्र.\\
भूतमन्नमिष्टं दद्यात् ।\\
कार्ष्णाजिनिः,\\
यदिष्टं जीवतोस्यासीत्तद्दद्यात् तस्य यत्नतः ।\\
अथवर्ज्यान्यन्नानि ।\\
साद्यायनिः,\\
वर्ज्यमत्रं त्रिधा प्रोक्त आद्यमाश्रयगर्हितम् ।\\
जातितोगर्हितं यच्च यच्च भावादिदूषितम् \textbar{}\textbar{}\\
अभोज्यान्नं विजानीयात् अन्नमाश्रयगर्हितम् ।\\
लशुनादिकमनं यत् ज्ञेयं जाति विगर्हितम् \textbar{}\textbar{}\\
दुष्टं भावक्रियावस्थासंसर्गैस्तु तृतीयकम् ।\\
अभोज्यान्नं = अभोज्याज्ञानं त्रिविधमपि चाहिक उक्तम् ।\\
गोभिलः,\\
अतिशुक्तोग्रलवणं विरसं भावदूषितम् ।\\
राजसं तामसञ्चैष हव्ये कव्ये च वर्जयेत् ॥

{ }{भक्ष्याभक्ष्यनिरूपणम् । ५७}{\\
अतिशुकं = द्रव्यान्तरसंसर्गेण कालवशाद्वातिकुत्सितरलम् । उग्रलवणं=\\
अधिकलवणम् । विरसं = अपगतरसम् । भावदूषितम् = कार्पण्यादि.\\
भिरन्तःकरणाधिकारदूषितम् । राजसतामसे भगवद्गीतासु -\\
कट्वम्ललवणात्युष्णतीक्ष्णरूक्षविदाहिनः ।\\
आहारा राजसस्येष्टा दुःखशोकामयप्रदाः ।\\
यातयामं गतरसं पूति पर्युषितं च यत् ॥\\
उच्छिष्टमपि चामेध्यं भोजनं तामसप्रियम् ।\\
हव्ये कव्ये चेति तत्साध्येषु कर्मस्वित्यर्थः ।\\
ब्रह्मपुराणे,\\
आसनारूढमनाद्यं पादोपहतमेव च ।\\
अमेध्यादागतैः स्पृष्टं शुक्तं पर्युषितं च यत् ॥\\
द्विःस्विनं परिदग्धं च तथैवाग्रावलेहितम् ।\\
शर्कराकेशपाषाणैः कीटैर्यच्चाप्युपद्रुतम् ॥\\
पिण्याक मथितं चैव तथातिलवणं च यत् ।\\
सिद्धाः कृताश्च ये भक्ष्याः प्रत्यक्षलवणीकृताः ॥\\
(१) वाग्भावदुष्टशञ्च }{तथा}{ दुष्टैश्चोपहृता अपि ।\\
वायसैश्चोपभुक्तानि वर्ज्यानि श्राद्धकर्मणि ॥\\
आसनारूढं = भूव्यतिरिक्ताधिकरणस्थापितपात्रस्थितम् \textbar{} पादोपहतं =\\
साक्षात् भोजनद्वारा वा पादस्पर्शदूषितम् । द्विःस्विन= वारद्वयं पक्कम्,\\
तच्च पक्कस्यौदनादेः शुष्कस्य पुनर्मार्दवायोदकं निनीय पाचितम्, यस्य\\
तु सुरणादेरधिश्रयणद्वयेनैव पाकः प्रसिद्धस्तस्य तु तावदवयव एक\\
एवासौ पाक इति तत्रापि न द्विःपक्कता इति हेमाद्रि । अग्रावलेहित = अग्रे\\
उपरि भागे मार्जारादिभिरवलीढम् । मथित= आलोडितं दधि ।\\
सिद्धाः कृताः = चिरसंस्थिताः पर्यटादयः ।\\
वर्ज्यमुदकमुक्तम् ब्रह्माण्डपुराणे -\\
दुर्गन्धि फोनलं वर्ज्ये }{तथा}{ वै पल्वलोदकम् ।\\
न लभेद्यत्र गौस्तृप्तिं नक्तं यच्चैव गृह्यते ॥\\
यन्न सर्वार्थमुत्सृष्ट यच्च भोज्यनिपानजम् ।\\
तद्वज्यै सलिलं तात सदैव पितृकर्मणि ॥\\
निपानजं कूपसमुद्धृतपश्वादिपेयोदकधारणार्थजलाशयजम् ।\\
इति श्राद्धीयभक्ष्याभक्ष्यनिर्णयः ।

% \begin{center}\rule{0.5\linewidth}{0.5pt}\end{center}

{( १ ) वाससा चावधूतानीति निर्णयसिन्धौ पाठः ।\\
८ वी० मि०

{५८ }{वीरमित्रोदयस्य श्राद्धप्रकाशे-}{\\
अथ श्राद्धीयब्राह्मणनिरूपणम् ।\\
तत्र ब्राह्मणत्वजातिमान् ब्राह्मणः । ब्राह्मणत्व च विशुद्धमाता.\\
पितृजन्यत्वज्ञानसहकृतप्रत्यक्षगम्य जातिरूपं रतत्वावान्तरजाति-\\
वत् । तदेव ब्राह्मणपदप्रवृत्तिनिमित्तम् । न च याजनाध्यापनप्रतिग्र\\
हादि प्रवृत्तिनिमित्तमस्त्विति वाच्यम् । याजनादिविधौ ब्राह्मणस्योहे.\\
श्यत्वेन पूर्वप्रसिद्धिसापेक्षत्वात्, याजनादेर्विधेयत्वेन
पश्चाद्भावित्वा\\
दवेष्ट्यधिकरणोक्तन्यायेन ( १ ) पूर्वप्रवृत्तिनिमित्तत्वायोगात् ( २ )
\textbar{} आ\\
हवनीयादौ त्वनुपपत्या तदाश्रयणम् (३) । प्रतिग्रहादिनिवृत्तेष्वव्या\\
तेरापदिक्षत्रियादिषु प्रतिग्रहादिप्रवृत्तेष्वतिव्याप्तेश्च ।\\
यन्तु यमशातातपाभ्यामुक्तम् ।\\
तपो धर्मो दया दानं सत्यं ज्ञान श्रुतिर्घृणा ।\\
विद्याविनयमस्तेयमेतद्ब्राह्मणलक्षणम् ॥\\
इति न तच्छब्दप्रवृत्तिनिमित्तपर, किन्तु हव्यकव्यसम्प्रदानब्राह्म\\
णप्राशस्त्यपरम् । अत एव -\\
बौधायने,\\
विद्या तपश्च योनिश्च एतद् ब्राह्मणलक्षणम् ।\\
विद्यातपोभ्यां यो हीनो जातिब्राह्मण एव सः ॥\\
इति विद्यारहितेऽपि जातिमात्रेण ब्राह्मणत्वं दर्शयति । इति ब्राह्मण\\
लक्षणम् ।\\
अथ ब्राह्मणप्रशंसा \textbar{}}

{तैत्तिरीये,}{\\
यावतीर्वै देवतास्ताः सर्वा वेदविदि ब्राह्मणे वसन्ति । इति ।\\
याज्ञवल्क्यः, ( अ० १ दानप्र० लो० १९८)\\
तपस्तप्त्वासृजद्ब्रह्मा ब्राह्मणान् वेदगुप्तये ।\\
तृप्त्यर्थ पितृदेवानां धर्मसंरक्षणाय च ॥ इति ।\\
भविष्येऽपि,

% \begin{center}\rule{0.5\linewidth}{0.5pt}\end{center}

{(१) पूर्वममि० अ० २ पा० ३ अधि० २ ।\\
( २ ) यथा ``राजानमभिषेचयेत्" इत्यभिषेकविधौ राज्ययोगात् पूर्वमेव प्रयु.\\
क्तस्य राजशब्दस्य क्षत्रियत्वमेव प्रवृत्तिनिमित्त न तु राज्ययोगः,
}{तथा}{ प्रकृतेऽपि\\
ब्राह्मणशब्दस्यादृष्टविशेषप्रयोज्य ब्राह्मणत्वमेव प्रवृत्तिनिमित्तं न तु
याजनादियोग\\
इति ध्येयम् ।\\
(३) आधानसाध्यस्यैवा ऽऽहवनीयपदार्थत्वादिति भाव• ।

{ }{प्रशस्तब्राह्मणनिरूपणम् }{। ५९\\
ब्राह्मणा दैवतं भूमौ ब्राह्मणा दिवि दैवतम् ।\\
ब्राह्मणेभ्यः परं नास्ति भूतं किञ्चिज्जगत्त्रये ॥\\
अदैवं दैवतं कुर्युः कुर्युर्देवमदैवतम् ।\\
ब्राह्मणा हि महाभागाः पूज्यन्ते सततं द्विजाः ॥\\
ब्राह्मणेभ्यः समुत्पन्ना देवाः पूर्वमिति श्रुतिः ।\\
ब्राह्मणेभ्यो जगत्सर्वे तस्मात् पूज्यतमाः स्मृताः ।\\
येषामश्नन्ति चक्रेण देवताः पितरस्तथा ।\\
ऋषयश्च}{ तथा}{ नागाः किम्भूतमधिकं ततः ॥ इति ब्राह्मणप्रशंसा \textbar{}\\
तत्र श्राद्धीयप्रशस्ता ब्राह्मणाः ।\\
तत्र श्रोत्रियस्य प्रशस्तत्वमाह-\\
वशिष्ठः,\\
श्रोत्रियायैव देयानि हव्यकव्यानि नित्यशः ।\\
अश्रोत्रियाय दत्तं हि नार्हन्ति पितृदेवताः ।\\
श्रोत्रियलक्षणमाह -\\
देवल,\\
एकां शालां सकल्पां वा षड्भिरङ्गैरधीत्य वा ।\\
षट्कर्मनिरतो विप्रः श्रोत्रियो नाम धर्मवित् ॥\\
}{षट्}{ कर्माणि यजनयाजनाध्ययनाध्यापनप्रतिग्रहदानानि ।\\
ब्रह्मवैवर्तेऽपि,\\
जन्मना ब्राह्मणो ज्ञेयः संस्कारैर्द्विज उच्यते ।\\
विद्यया याति विप्रत्वं त्रिभिः श्रोत्रिय उच्यते ।\\
अत्रानुस्यूतत्वात् सर्वत्र स्वशाखाध्ययनमेव श्रोत्रियपदप्रवृत्ति\\
निमित्तम्, इतरन्तु प्राशस्त्यार्थमिति बोध्यम् ।
``श्रोत्रियश्छन्दोऽधीते"\\
( पा० ५१२१८४) इति स्मृत्या कर्मानुष्ठानाद्यनपेक्ष्यैव छन्दोऽध्येतर्येव\\
श्रोत्रियेति निपातनाश्च \textbar{} हव्यानि देवोद्देशेन दत्तानि हवींषि ।
कव्यानि=\\
पितॄनुद्दिश्य दत्तानि । श्रोत्रियेष्वपि विशेषमाह -\\
मनुः,\\
ज्ञानोत्कृष्टेषु देयानि कव्यानि च हवींषि च ।\\
न हि हस्तावसुग्दिग्धौ रुधिरेणैव शुद्धतः ॥ इति ।\\
( अ० ३ श्लो० १३२ )\\
ज्ञानोत्कृष्टा = विद्योत्कृष्टाः । दिग्धौ = लिप्तौ । तथा च स एव
श्रो.

{६० }{ वीरमित्रोदयस्य श्राद्धप्रकाशे -\/-\/-}

{त्रियायैव देयानीत्युक्त्वा -\/-\\
सहस्रं हि सहस्राणामनृचां यत्र भुञ्जते ।\\
एकस्तान् मन्त्रवित् प्रीतः सर्वानर्हति धर्मतः ॥ इति\\
( अ० ३ श्लो० १३१ )\\
अनुचाम् = अधर्मज्ञानाम् । अध्ययनस्य "श्रोत्रियायैव " देयानि इत्यत\\
एव प्राप्तेः, अनधीयानानामप्राप्तेः । प्रीतस्तर्पितो भोजित इति यावत् ।\\
सर्वान्स्ताननुचानर्हति स्वीकरोति यैरभेदमापद्यते । एव च तेषु सह\\
स्रादिसङ्चेष्वपि भोजितेषु यत् फलं तदेकेनापि लभ्यत इत्यर्थ\\
इति हेमाद्रयादयः । वस्तुतस्तु एतस्य वाक्यस्य पूर्वोक्तश्रोत्रियसम्प्र.\\
दानकदानविधेः प्रशसार्थत्वमेव युक्तम्। मेघातिथिना तु अनुचा इति\\
प्रथमाबहुवचनान्तपाठ उद्धृतः, तस्मिन् पक्षे अनृचाः सहस्र यत्र\\
भुञ्जते इति सम्बन्धः ।\\
गौतम:,\\
श्रोत्रियान् वाक्रूपवयःशीलसम्पन्नान् युवभ्यो दानं प्रथममेके\\
पितृवदिति । वाकुसम्पन्नाः = वेदशास्त्रविदः । रूपसम्पन्नाः= सुन्दराः\\
वयःसम्पन्नाः = नातिस्थविरा अपरिणतवयसश्च शलिसम्पन्नाः=मनो-\\
वाक्कायैः सकलप्राणिहितकारिणः । युवभ्यो दानं प्रधानमेके म\\
क्यन्ते ते हि श्राद्धीयनियमसम्पादनक्षमा भवन्ति । पितृवदेके पित-\\
रमुद्दिश्य स्थविरं प्रपितामहमुद्दिश्य स्थविरतरमिति ।\\
वशिष्ठः । पितृभ्यो दद्यात् पूर्वेद्युर्ब्राह्मणान् सन्निपात्य यतीन्
गृ.\\
हस्थान् साधूनपरिणतवयसोऽविकर्मस्थान श्रोत्रियान् शिष्यानन-\\
न्तेवासिन इति । पूर्वेद्युः श्राद्धादिनात्, सन्निपात्य = आमन्त्र्य ।
यतयः = प्रत्र.\\
जिता: । ते च-\/-\\
मुण्डान् जटिलकाषायान् श्राद्धकाले विवर्जयेत् ।\\
शिखिभ्यो धातुरक्तेभ्यस्त्रिदण्डेभ्यः प्रदापयेत् ॥\\
इति वचनात् । साधुरुक्क आदित्यपुराणे-\/-\\
अक्रोधना धर्मपराः शान्ता शमदमे रताः ।

{ }{निस्पृहाश्च महाराज ते विप्राः साधवः स्मृताः ॥ इति ।}{\\
सौरपुराणेऽपि,\\
गङ्गायमुनयोर्मध्ये मध्यदेशः प्रकीर्तितः ।\\
तत्रोत्पन्ना द्विजा ये वै साघवस्ते प्रकीर्तिताः ॥ इति ।

{ }{प्रशस्त ब्राह्म}{ण}{निरूपणम् ।}{ ६१\\
अपरिणतवयसः = नातिस्थविरान, अविकर्मस्थान् = प्रतिषिद्धक्रियावर्ज\\
कानू । अनन्तेवासिनः = सेवकभिन्नान् ।\\
कात्यायनः । स्नातकानिति, एके यतीन, गृहस्थान् साधून् वा,\\
श्रोत्रियान् वृद्धाननवद्यान्, स्वकर्मस्था ँ श्च । कव्यं च
प्रशान्तेभ्यः\\
प्रदीयत इति । स्नातकास्त्रिविधाः ।\\
}{तथा }{च यमः-\\
वेदविद्याव्रतस्त्राताः श्रोत्रिया वेदपारगाः ।\\
तेभ्यो हव्य च कव्यं च प्रशान्तेभ्यः प्रदीयते ॥ इति ।\\
वेदमात्रमधीत्य स्रातो वेदस्नातः \textbar{} वेदार्थविचारं समाप्य स्रातो
विद्या.\\
स्नातः । व्रतानि समाप्य स्नातो व्रतस्नात इति ।\\
कात्यायनोऽपि ।\\
त्रयः स्नातका भवन्ति विद्यास्नातको व्रतस्नातको विद्याव्रतस्नात\\
क इति । समाप्य वेदं असमाप्य व्रतं यः समावर्तते स विद्यास्नातकः ।\\
समाप्य व्रतमसमाप्य वेद यः समावर्तते स व्रतस्नातकः । उभयं स.\\
माष्य यः समावर्तते स विद्याव्रतस्नातक इति । वृद्धो= विद्यातपोभ्यां
श्रेष्ठः\\
न तु वयसा, युवभ्यो दानमित्याद्युक्तेः । अयं दोषः स स्वतः पितृ\\
भ्यो मातृतश्च येषां नास्ति ते }{तथा}{ । ये मातृतः पितृतश्चेत्याश्वलाय.\\
नोक्तेः । स्ववर्णाश्रमविहितकर्मनिरताः स्वकर्मस्थाः ।\\
आपस्तम्बः,\\
शुचीन् मन्त्रवतो विप्रान् सर्वकृत्येषु भोजयेत् ।\\
मनु,

{ यत्नेन भोजयेत् श्राद्धे बहुवृर्च वेदपारगम् \textbar{}\\
शाखान्तगमथाध्वर्युं छन्दोगं वा (१) समाप्तिगम् ॥ इति ।

{ }{( अ० ३ श्लो० १४५ )}{\\
बहुवचम् =ॠऋग्वेदिनम् \textbar{} अध्वर्यु = यजुर्वेदिनम् ।
वेदपारगशाखान्त\\
गसमाप्तिगशब्दैः सम्पूर्णशास्त्राध्यायिन उच्यन्ते ।\\
बृहस्पतिः,\\
यद्येकं भोजयेत् श्राद्धे छन्दोगं तत्र भोजयेत् ।\\
ऋचो यजूंषि सामानि त्रितयं तत्र विद्यते ॥

% \begin{center}\rule{0.5\linewidth}{0.5pt}\end{center}

(१) समाप्तिकमिति मुद्रितपुस्तके पाठः ।

{६३ }{वीरमित्रोदयस्य श्राद्धप्रकाशे-}{\\
ऋचा तु तृप्यति पिता यजुषा च पितामहः ।\\
पितुः पितामहः साम्ना छन्दोगोऽभ्यधिकस्ततः ॥

{शातातपः,\\
भोजयेद्यद्यथर्वाणं दैवे पित्र्ये च कर्मणि ।\\
अनन्तमक्षयं चैव फलं तस्येति वै श्रुतिः ॥\\
अनन्तं = चिरस्थापि \textbar{} अक्षयं =अन्यूनम् ।\\
यस्त्वन्यं भोजयेच्छ्राद्धे विद्यमानेष्वथर्वसु ।\\
निराशास्तस्य गच्छन्ति देवताः पितृभिः सह ॥\\
तस्मात् सर्वप्रयत्नेन श्राद्धकाले त्वथर्वणम् ।\\
भोजयेद्धव्यकव्येषु पितॄणां च तदक्षयम् ॥ इति ।\\
एतेन यथा कन्या}{ तथा}{ हविरिति वचनात् भ्राम्यन्तो यजमानवेद\\
समानशाखा एव श्राद्धे ब्राह्मणा इति वदन्तो निरस्ताः । यथा कन्ये\\
त्यादिवाक्यस्य हविःषु निर्ज्ञातकुलशीलत्वादिसम्प्रदानककन्यासा•\\
दृश्यपरत्वेनाप्युपपत्तेः न कश्चिद्विरोधः । अत एव कुलवंशादिज्ञान.\\
मावश्यकमुक्तम् मात्स्ये, "ज्ञातवंशकुलान्वित" इति । येऽपि " यत्नेन\\
भोजयेच्छ्राद्धे" इत्यादिवचनेन बहूवृचादीनां त्रयाणामेव प्राप्तत्वात्,\\
"एषामन्यतमो यस्य भुञ्जीत श्रद्धमर्चितः । पितॄणां त}{स्य}{ तृप्तिः\\
स्यात् शाश्वती साप्तपौरुषी" इति वाक्यस्याथर्वणनिषेधार्थत्वमाहु-\\
स्तेऽप्येतेनैव निराकृता: । उक्तवाक्यस्य तु 'अथर्वाणं भोजयेत्' इति\\
प्रत्यक्षाविरोधात बद्द्वृचादिप्रशसार्थत्वेन छन्दोगं वा समाप्तिगमिति\\
चकारार्थसमुचयनिरासार्थत्वेन चोपपत्तेर्नाथर्वणनिषेधार्थत्वमिति ।\\
याज्ञवल्क्यः, ( अ० ९ श्राद्धप्र० ० २१९ )\\
अग्न्याः सर्वेषु वेदेषु श्रोत्रियो ब्रह्मविद् युवा ।\\
वेदार्थविद् ज्येष्ठसामा त्रिमधुः त्रिसुपर्णक. ॥\\
कर्मनिष्ठास्तपोनिष्ठाः पञ्चग्निर्ब्रह्मचारिणः ।\\
पितृमातृपराश्चैव ब्राह्मणाः श्राद्धसम्पदः ॥ इति ।\\
( अ० १ श्राद्धप्र० श्लो० २२१ )\\
चतुर्ष्वषि वेदेषु अध्येतॄणां मध्ये मुख्योऽध्येता अग्रयः । ब्रह्म =औ .\\
पनिषदात्मा \textbar{} त्रिमधुः =
त्रिमध्वाख्यवतानुष्ठानपूर्वकतदाख्यऋग्वेदैकदे.\\
शाभ्येता \textbar{} त्रिसुपर्णः = मधु मेतु मामित्यादि
तैत्तिरीयशाखान्तर्गतयजु\\
स्त्रितयाध्येता । य इदं त्रिसुपर्णमितिश्रुतौ तेष्वेव निर्देशात् ।
कल्पतरुस्तु

{ }{ }{प्रशस्तब्राह्मणनिरूपणम् । }{ ६३\\
'चतुष्कपदी युवतिः सुपेशा' इति ऋग्वेदान्तर्गतं सुपर्णपदोपलक्षित
\textbar{}\\
[ ऋक्त्रयमिन्याह । ज्येष्ठसामा= सामवेदभागाध्येता । पञ्चाग्निः =
गार्हपत्य-\\
दक्षिणाग्म्याहवनीयसभ्यावसथ्याः पञ्चाग्नयो यस्य सन्ति स }{तथा}{ ।\\
अथवा छान्दोग्योपनिषत्पठितपञ्चाग्निविद्याध्येता पञ्चाग्निः । एते ब्रा-\\
ह्मणा श्रद्धस्य सम्पदे समृद्धये भवन्तीत्यर्थः ।\\
ब्रह्माण्डपुराणे,\\
ये च भाष्यविदः केचिद् ये च व्याकरणे रता ।\\
अधीयानाः पुराण वै धर्मशास्त्रमथापि वा ॥\\
ये च पुण्येषु तीर्थेषु कृतस्नानाः कृतश्रमाः ।\\
मखेषु चैव सर्वेषु भवन्त्यवभृथप्लुताः ॥\\
अक्रोधनाः शान्तिपरास्तान् श्राद्धेषु नियोजयेत् ।\\
भाष्याणि= पाणिनि सुत्रादिभाष्याणि \textbar{}\\
मात्स्ये,\\
आथर्वणो वेदविञ्च ज्ञातवंशः कुलान्वितः ।\\
पुराणवेत्ता ब्रह्मण्यः स्वाध्यायजपतत्परः ॥\\
शिवभक्तः पितृपर: सूर्यभक्तोऽथ वैष्णवः ।\\
ब्रह्मण्यो योगविच्छान्तो विजितात्मा सुशीलवान् \textbar{}\textbar{}\\
एतांस्तु भोजयेत्रित्यं दैवे पित्र्ये च कर्मणि ।\\
मण्डलब्राह्मणशा ये ये सूकं पौरुष विदुः \textbar{}\textbar{}\\
तांस्तु दृष्ट्वा नरः क्षिप्रं सर्वपापैः प्रमुच्यते ।\\
शतरुद्रियजाध्येषु निरता ये द्विजोत्तमाः \textbar{}\textbar{}\\
पितॄन् सन्तारयन्त्येते श्राद्धे यत्नेन भोजिताः ।

{तथा,}{\\
गायत्रीजाप्यनिरतं हव्यकव्येषु योजयेत् । इति ।\\
ब्रह्मपुराणे ।\\
षडङ्गविद्याज्ञानयोगी यज्ञतत्वज्ञ एव च ॥\\
अयाचिताशी विप्रो यः श्राद्धकल्पज्ञ एव च ।\\
अष्टादशानां विद्यानामेकस्या अपि पारगः ॥\\
अष्टादशविद्याश्च याज्ञवल्क्योक्ताः चतुर्दश-\\
पुराणन्यायमीमांसाधर्मशास्त्राङ्गमिश्रिताः ।\\
वेदा: स्थानानि विद्यानां धर्मस्य च चतुर्दश ॥ इति ।\\
( अ० १ उपाद्यातप्र० लो० ३ )\\


{ }{६४ }{वीरमित्रोदयस्य श्राद्धप्रकाशे-}{\\
आयुर्वेदो धनुर्वेदो गान्धर्वञ्चार्थशास्त्रकम् \textbar{}\textbar{}\\
इति स्मृत्यन्तरोक्ताश्चत्वारः । शिक्षा कल्पो व्याकरणमित्यादि ष.\\
डङ्गानि ।\\
मन्वादिधर्मशास्त्रेतिहासवेत्तापि प्रशस्त इत्याह स एव -\\
बहुनात्र किमुक्तेन इतिहासपुराणवित् ।\\
अथर्वशिरसोऽध्येता तावुभौ पितृभिः पुरा ॥\\
तपः कृत्वा नियोगार्थ प्रार्थितौ पितृकर्मणि ।\\
अथर्वशिरो= वेदभागविशेषः ॥\\
ब्रह्यववर्ते,\\
अक्लृप्तान्नं घृणिकान्तं कृशवृतिमयाचकम् ।\\
एकान्तशीलं ह्रीमन्तं सदा श्राद्धेषु भोजयेत् \textbar{}\textbar{}\\
न विद्यते क्लृप्तमत्रं यस्यासौ अक्ऌप्तान्नः । घृणी दयावान् । क्लान्तो=\\
व्रतश्रान्तः । घृणी चासौ क्लान्तश्च घृणिक्लान्तः । अथवा घृणिभि: सौर\\
किरणैः क्लान्तः तीर्थयात्रादौ आतपेन पीडित इत्यर्थः । कृशा=प्रतिश्र\\
हादिसङ्कोचेन वृत्तिर्वर्तनं यस्य स}{ तथा}{ । ह्रीमान् लज्जावान् वागादि\\
चापलनिवृत्यर्थमेतत् ।\\
}{तथा}{,\\
पत्नीपुत्रसमायुक्तान् श्राद्धकर्मणि योजयेत् ।\\
देहस्यार्द्धं स्मृता पत्नी न समग्रो विना तया ॥\\
नचापुत्रस्य लोकोऽस्ति श्रुतिरेषा सनातनी । इति ।\\
लोकः=अदृष्टजन्यः अभ्युदयः ॥\\
नन्दिपुराणे,

{ }{यतीन् वा वालखिल्यान् वा भोजयेच्छ्राद्धकर्मणि ।\\
वानप्रस्थोपकुर्वाणौ पूजयेत् परितोषयेत् ॥}{\\
}{बार्हस्पत्यसंहितायां,\\
अङ्गिरोनारदभृगुवृहस्पत्युदिताः श्रुती: ।\\
पठन्ति ये श्रद्दधानास्तेऽभियोज्याः प्रयत्नतः ।\\
वेदान्तनिष्ठाः श्राद्धेषु व्याख्यातारो विशेषत: ॥इति॥

{संवर्त:,\\
प्रयत्नाद्धव्यकव्यानि पात्रीभूते द्विजन्मनि ।\\
प्रतिष्ठाप्यानि विद्वद्भिः फलानन्त्यमभीप्सुभिः ॥

{प्रशस्तब्राह्मणनिरूपणम्}{ । ६५\\
पात्रलक्षणमाह वृद्धशातातपः\\
स्वाध्यायवान्नियमवांस्तपस्वी ज्ञानविश्चयः ।\\
क्षान्तो दान्तो सत्यवादी विप्रः पात्रमिहोच्यते ॥ इति ।\\
स्वाध्यायव=त्त्वं चाध्ययनाध्यापनार्थज्ञानप्रवचनादिभिः ।\\
नियमा,\\
स्नान मौनोपवासेज्यास्वाध्यायोपस्थनिग्रहाः ।\\
नियमा गुरुशुश्रूषाशौचाक्रोधाप्रमादता \textbar{}\textbar{}\\
( अ० ३ प्रायश्चित्तप्र० श्लो० ३१३ )\\
इति याज्ञवल्क्योक्ताः । स्नान= नित्यं नैमित्तिकञ्च । मौन= निषिद्धवाक्\\
परिवर्जनम् । उपवासो विहितः । उपस्थनिग्रहः = निषिद्धरेतो विमोका-\\
निवृतिः । अप्रमादता = विधिप्रतिषेधयोः सावधानता ।\\
याज्ञवल्क्य, ( अ० १ दानप्र० श्लो० २०० )\\
न विद्यया केवलया तपसा वापि पात्रता ।\\
यत्र वृत्तमिमे चोभे तद्धि पात्र प्रचक्षते ॥\\
अस्यार्थः, केवलया विद्यया श्रुताध्ययनमात्रसम्पत्या न पात्रत्वं\\
नापि केवलेन तपसा शमदमादिना, अपि शब्दान्न केवलानुष्ठानेन\\
केवलया जात्या वा दाने पूर्णपात्रत्वं किन्तु यत्र ब्राह्मणे ( १ )
वृत्तम्, इमे\\
विद्यातपसी चोभे तदेव पूर्ण पात्रं मन्वादयो ब्रुवते । यद्यपि केवलवि\\
द्यातपसोरपि पात्रताप्रयोजकतास्त्येव, "किञ्चिद् वेदमयं पात्र\\
किश्चित् पात्रं तपोमयम्" इति वचनात् । तथापि नात्र
पूर्णपात्रतोत्का,\\
किञ्चित्पात्रमित्युक्तेरिति ।\\
महाभारते\\
साङ्ग} ँ{स्तु चतुरो वेदान् योऽधीते वै द्विजर्षभः ।\\
षड्भ्योऽनिवृत्तः कर्मभ्यस्तं पात्रमृषयो विदुः ॥\\
षड्भ्यो यजनादिभ्यः, अनिवृत्तिः सत्यधिकारे प्रवृत्तिः ।\\
देवल,\\
त्रिशुलः कृशवृत्तिश्च घृणालुः सकलेन्द्रियः ।\\
विमुक्तो योनिदोषेभ्यो ब्राह्मणः पात्रमुच्यते ॥\\
त्रीणि कुलविद्यावृत्तानि शुक्लानि विशुद्धानि पातित्यापादकदो-\\
परहितानि यस्य स त्रिशुक्ल । घृणालुः = दयालुः । योनिदोषा.
-योनिसाङ्कर्य-\\
कारणभूताः ।

% \begin{center}\rule{0.5\linewidth}{0.5pt}\end{center}

{(१) अनुष्ठानमिति विज्ञानेश्वरः ।\\
९ वी०मि०

६६ वीरमित्रोदयस्य श्राद्धप्रकाशे-

{वशिष्ट\\
स्वाध्यायाढ्य योनिमन्तं प्रशान्त\\
(१) वैतानज्ञं पापभीरुं बहुज्ञम् ।\\
स्त्रीषु क्षान्तं धार्मिकं गोशरण्यं\\
व्रतैः कान्तं ब्राह्मणं पात्रमाहुः \textbar{}\textbar{}\\
स्त्रीषु क्षान्तः = सहिष्णुः, स्त्र्यलाभजनितदुःखसहनशीलः । व्रतक्लान्तः
=\\
व्रतानुष्ठानेन शोषितेन्द्रियशरीरः । परमपात्रमाह व्यासः -\\
किञ्चिद्वेदमयं पात्रं किञ्चित् पात्रं तपोमयम् ।\\
असङ्कीर्णन्तु यत् पात्रं तत् पात्र परमं स्मृतम् ॥\\
असङ्कीर्ण = व्योनिसाङ्कर्यपतितसंसर्गरहितम् ।

{बृहस्पत्तिः,\\
ब्रह्मचारी भवेत् पात्रं पात्रं वेदस्य पारगः ।

{तथा}{,\\
पात्राणामुत्तमं पात्रं शूद्रानं यस्य नोदरे । इति ।\\
शूद्रात्पक्कमनतिक्रान्तत्र्यहमामं वा प्रतिगृह्य भुज्यते तत् शूद्रान्नम्
।\\
ऋष्यशृङ्गः,\\
अतिथिश्च }{तथा}{ पात्रं तत् पात्रं परमं विदुः ।\\
अतिथिरुक्तो मनुना -\\
एकरात्रन्तु निवसन्नतिथिर्ब्राह्मणः स्मृतः ।\\
अनित्यं हि स्थितो यस्मात् तस्मादतिथिरुच्यते ॥\\
( अ० ३ \textbar{} श्लो० १०२ )\\
भविष्यपुराणे,\\
क्षान्त्यस्पृहा दमः सत्यं दानं शीलं तपः श्रुतम् ।\\
एतदष्टाङ्गमुद्दिष्ट परमं पात्रलक्षणम् ॥ इति ।\\
अत्र सर्वत्र किञ्चिद्वेदमयं पात्रमित्यादि वचनात् विद्यादीनां प्र.\\
त्येकं पात्रताप्रयोजकत्वात्तत्र तत्र गुणाधिक्य पात्रतारतम्यायेति\\
ज्ञेयम् ।\\
यमः ।\\
ऋषिव्रती ऋषीकश्च }{तथा}{ द्वादशवार्षिक. ।\\
ब्रह्मदेयसुतश्चैव गर्भशुद्ध सहस्रदः \textbar{}\textbar{}\\
ऋषीकः = ऋषेः किञ्चिन्न्यूनगुणः ।

% \begin{center}\rule{0.5\linewidth}{0.5pt}\end{center}

{(१) वैतानस्थमिति हेमाद्रौ पाठः ।\\
~\\


{प्रशस्तब्राह्मणनिरूपणम्}{ । ६७\\
ऋषेः किंचिद् गुणैर्न्यूनो ऋषीक इति कीर्तितः ॥\\
इति सुमन्तुस्मरणात् । द्वादशवार्षिकवदेव्रतचारी द्वादशवा-\\
र्षिक । सहस्रदो=गोसहस्रद् इति मेघातिथिः । सहस्रदक्षिणक्रतुया.\\
जीत्यपरे ।\\
}{तथा}{ ।\\
चान्द्रायणव्रतचरः सत्यवादी पुराणवित् ।\\
विमुक्तः सर्वतो धीरो ब्रह्मभूतो द्विजोत्तमः ॥\\
सर्वतः - जायमानो वै ब्राह्मण इत्यादिश्रुतिबोधितर्णत्रयादिति\\
शेषः । सर्वतो - ब्रह्मातिरिक्तविषयादिति तु युक्तम्, 'धीरो ब्रह्मभूत\\
इति उत्तरग्रन्थसमभिव्याहारात् ।\\
}{तथा}{ ।\\
अनमित्रोऽनचामित्रो मैत्रश्चात्मविदेव च ।\\
न विद्यते अमित्रं शत्रु र्यस्य । न च स्वयं कस्यापि
शत्रुमैत्रीरतश्च\\
भवति ।\\
}{तथा}{ ।\\
स्नातको जप्यनिरतः सदा पुष्पबलिप्रियः ।\\
स्नातकः = स्नानोत्तरं प्राग् विवाहात् । परमेश्वरपूजाद्यर्थ पुष्पं\\
बलिरुपहारश्च साधनत्वेन प्रियं यस्य स }{तथा}{ ।\\
शङ्खलिखितावपि ।\\
अथ पाङ्केयाः । वेदवेदाङ्गवित् पञ्चाभिरनुचानः साङ्ख्ययोगो-\\
पनिषद्धर्मशास्त्रविच्छ्रोत्रिय, त्रिणाचिकेतः त्रिमधुः, त्रिसुपर्णको
ज्ये\\
ष्ठसामगः । साङ्ख्ययोगोपनिषद्धर्मशास्त्राध्यायी वेदपरः सदाग्निको}{\\
}{मातृपितृशुश्रूषुर्धर्मशास्त्ररतिरिति ।\\
साङ्ख्य= कापिलम् \textbar{} योगः पातञ्जलम् । उपनिषद - वेदान्ताः
\textbar{}\\
पूवार्द्ध सङ्ख्यादिविदुक्त उत्तरार्द्धे तु तदध्येनेति भेदः । सदाग्निको=
नि.\\
त्यानिमान् \textbar{}\\
हारीतस्तु \textbar{}\\
तस्मात् कुलीनाः श्रुतवन्तः शीलवन्तो वृत्तस्था: सत्यवादिनो.\\
ऽव्यङ्गाः पाङ्क्तेया इत्युपक्रम्य कुलशीलवृत्तरूपान् पाङ्क्तेयत्वप्रयो.\\
जकान् गुणानाह । स्थितिरविच्छिन्नवेदवेदिता योनिसंकरित्वमार्षे.\\
यत्वं चेति कुलगुणाः । स्थितिः=सन्तत्यविच्छेद:, आपद्यपि वर्णाश्रमा

{५८ }{ वीरमित्रोदयस्य श्राद्धप्रकाशे-}{\\
दिधर्माविच्छेदो वा । पित्रादिपारम्पर्येण वेदवेदित्वाविच्छेदवान.\\
विच्छिन्नवेदवेदिता, अथवा अविच्छिनो वेदो वेदी च यस्य स }{तथा}{\\
}{तस्य}{ भावस्तत्ता \textbar{}
अनतिक्रान्तविहितप्रतिषिद्धयोनिसंसर्गत्वम्-\\
अयोनि संकरित्वम् । स्वीयप्रवरणीयऋषिज्ञातृत्वमार्षेयत्वम् । एते कुलगुणाः
।\\
}{तथा}{ वेदो वेदाङ्गानि धर्मोऽध्यात्मं विज्ञानं स्थितिरिति षड्विधं\\
श्रुतम् । पाङ्क्तेयत्वनिमित्तमिति शेषः । वेदाश्चत्वारः । अङ्गानि=
शिक्षा-\\
दीनि षट् । धर्मो= धर्मशास्रम् \textbar{} अध्यात्मम्=आत्मानात्मविवेकशा-\\
स्वम् । विज्ञान=न्यायमीमांसादि । स्थितिः = अधीताविस्मरणम्, वेदा-\\
च्छुतस्य प्रमाणप्रमेयाद्यसंभावनापीरहारेण स्थैर्य तात्पर्यावधा\\
रणं येन जायते तत्, पुराणेतिहासं वा ।\\
आह च व्यास:-\\
इतिहासपुराणाभ्या वेदार्थमुपबृंहयोदति ।\\
एतेषां वेदादीनां षण्णां श्रवणं श्रुतमित्यर्थः ।\\
}{तथा}{ ब्रह्मण्यता देवपितृभक्तता समता सौम्यता, अपरो.\\
पतापिताऽनसूयता मृदुता = अपारुष्यं मैत्रता प्रियवादित्वं कृतश\\
ता शरण्यता प्रशान्तिश्चेति त्रयोदशविधं शीलम् ।\\
ब्रह्मणि साधुत्व ब्रह्मण्यता \textbar{} दम्भरहितं
दैवपितृकर्मानुष्ठातृत्वं देवपि-\\
तृभक्तता । रागद्वेषराहित्यं समता । सर्वजनहद्यचरित्रत्वं सौम्यता
\textbar{} परपी-\\
डाकारित्वाभावोऽ परोपतापिता । दोषानाविष्करणमनसूयता । अकठिन\\
हृदयत्वं मृदुता । कलह\\
वैमुख्यम्=अपारुष्यम् । मनःप्रसादो = मैत्रता \textbar{}\\
शरण्यता शरणार्हता \textbar{} विद्याद्यनुत्सेकः प्रशान्तिः ।\\
क्षमा दमो दया दानमहिंसाऽगुरुपीडनं शौचं स्नानं जपो होमः\\
तपः स्वाध्यायः सत्यवचनं संतोषो दृढव्रतत्वमुपवतित्वं चेति\\
षोडशगुणं वृत्तम् \textbar{} स्वत्वनिवृत्तिपूर्वकपरस्वत्वापादनं दानम् ।
देव-\\
तोद्देशेन द्रव्यत्यागः प्रक्षेपश्च मिलित होमः । दृढब्रतत्वं नाम स्वी\\
कृतव्रताऽपरित्यागः । उपव्रतम् = एकभक्त्याद्याहारलाघवादि,\\
अक्रोधो गुरुशुश्रूषा शौचमाहारलाघवम् ।\\
अप्रमादश्च नियमाः पञ्चैवोपव्रतानि च ॥\\
इति व्यासोक्ते: । इति प्रशस्तब्राह्मणाः ।\\
: अथ पङ्किावनादीननुक्रम्य " यत् किञ्चित् पितृदेवत्यमेभ्यो\\
'दत्तं तदक्षयम्" इति नियमेन पङ्क्तिपावनानां विनियोगात् पङ्कि

{ }{पङ्क्तिपावनानिरूपणम् । }{६९\\
पावनादय उच्यन्ते । तत्र पङ्गिपावनानां तावल्लक्षणमाहापस्तम्बः-\\
अपा}{ङ्क्त्यो}{पहता प}{ङ्क्ति:}{ }{ पाव्यते यैर्द्विजोत्तमैः ।\\
तान् निबोधत कार्त्स्न्येन द्विजाग्रचान् पङ्क्तिपावनान् ॥\\
अपाङ्ङ्क्त्या=अपाङ्क्तेया वक्ष्यमाणा देवलकादयः, तैरुपहतां\\
दूषितां पाङ्कं ये पावयन्ति पवित्रां कुर्वन्ति तान् निबोधत । तानाह-\\
अग्रधाः सर्वेषु वेदेषु सर्वप्रवचनेषु च ।\\
श्रोत्रियान्वयजाश्चैव विज्ञेयाः पङ्किपावनाः ॥\\
चतुर्षु वेदेषु, प्रोच्यते निरुच्यते वेदार्थो यैस्तेष्वङ्गेषु चाप्रथाः\\
श्रेष्ठाः संशयविपर्ययव्युदासेनानादिसम्प्रदायतोऽधीत षडङ्गविदः ।\\
(१)त्रिणाचिकेतः पञ्चाग्निस्त्रिसुपर्णः षडङ्गवित् ।\\
ब्रह्मदेयानुसन्तानो ज्येष्ठसामग एव च ॥\\
त्रिणाचिकेतः = त्रिणाचिकेताख्यवेदभागः, सच (पूर्ण) "पीतोदका ज.\\
ग्धतृणा" इत्यादिः, तदध्ययन सम्बन्धात् पुरुषोऽपि त्रिणाचिकेत उ\\
च्यते आचीर्णत्रिणाचिकेताख्यवेदभागाङ्गभूतत्रतश्च । ब्रह्मदेयानुसन्तान:=\\
ब्राह्मविवाहपरिणीतापुत्रः ।\\
तथा,

{ }{वेदार्थवित् प्रवक्ता च ब्रह्मचारी सहस्रदः ।\\
शतायुश्चैव विशेया ब्राह्मणाः पङ्किपावनाः \textbar{}\textbar{}}{\\
षडङ्गविदित्यनेन वेदार्थविदोप्युपादानेऽपि वेदार्थमात्रवेतृत्वेन पा.\\
}{ङ्क्त्ये}{यत्वमत्रोक्कमिति न पौनरुक्त्यम् \textbar{} केचित्तु
षडङ्गाध्ययनव्यतिरेके.\\
णापि प्राज्ञतया वेदार्थ यो वेत्ति सोऽत्र वेदार्थविदिति न पौनरुक्त्य\\
मित्याहुः । शतायु =ज्योतिः शास्त्रादवगतं शतवर्षमायुर्यस्य, स एव न\\
तु शतवार्षिकः । युवभ्यो दानमित्यनेन यूनामेव विधानात् इति के\\
चित् । वृद्धतम इति हेमाद्रि ।\\
देवल:,

{ }{अग्निचित् सोमपाश्चैते ब्राह्मणाः पङ्किपावनाः ।\\
ऋग्यजुःसामधर्मज्ञाः स्नातकाश्चाग्निहोत्रिणः \textbar{}\textbar{}\\
ब्रह्मदेयानुसन्तानाः छन्दोगो ज्येष्ठसामगः ॥

% \begin{center}\rule{0.5\linewidth}{0.5pt}\end{center}

{( १ ) कुत्रचित् तृणाचिकेत इति पाउ उपलभ्यते । तृणवत् सर्वमाचिकेति\\
जानातीति तदर्थः । }{तथा}{ च ब्रह्मपुराणम् \textbar{}\\
आचिकेतीति यो विश्वं तृणवत् सर्वनिस्पृहः \textbar{}\\
तृणाचिकेतः स गृही रागद्वेषविमत्सरः ॥ इति ।

{७० }{वीरमित्रोदयस्य श्राद्धप्रकाशे-}{\\
मन्त्रब्राह्मणविद् चतुर्मेधो वाजसनेयी यः स्वधर्मानधीते यस्य च\\
दशपुरुषं मातृपितृवंशः श्रोत्रियो विज्ञायते विद्वांसः, स्नातकाश्चै•\\
ते }{पङ्क्तिपावना}{ भवन्ति । चतुर्मेधः = सर्वतोमुखयाजी, अथवा चातु.\\
र्मास्ययाजी । तयोर्हि चत्वारो यज्ञाः सन्ति । वाजसनेयी = वाजसनेयि.\\
शाखाध्येता । धर्मः = प्रवर्ग्यभागः ।\\
यम,\\
वेदविद्याव्रतस्त्राताः ब्राह्मणाः }{पङ्क्तिपावनाः}{ ।\\
व्रतचर्यासु निरताः ये कृशाः कृशवृत्तयः ॥\\
अनुक्रान्ताः स्वधर्मेभ्यस्ते द्विजाः पडिपावनाः ।\\
सत्रिणो नियमस्थाश्च ये विप्राः श्रुतिसम्मताः ।\\
व्रतानि = कृच्छ्रचान्द्रायणादीनि । कृशाः = तपोभिः शोषितशरीराः ।\\
कृशवृत्तयोऽश्वस्तनादिवृत्तयः । सत्रिण= इति स्थाने मन्त्रिण इति क्वचित्\\
पाठः, तदा मन्त्रो गायत्र्यादिस्तनिष्ठा इत्यर्थः ।\\
}{तथा,}{\\
प्राणिहिंसानिवृत्ताश्च ते द्विजाः }{पङ्क्तिपावनाः}{ ।\\
अग्निहोत्ररताः क्षान्ताः क्षमावन्तोऽनसूयकाः ॥\\
ये प्रतिप्रहनिस्नेहास्ते द्विजाः पङ्गिपावनाः ।\\
हारीतः,\\
दशोभयतः श्रोत्रियास्त्रिणाचिकेतस्त्रिमधुस्त्रिसौपर्णस्त्रिशीर्षा\\
ज्येष्ठसामगः पञ्चाग्निः षडङ्गवित् रुद्रजाप्यूर्ध्वरेता ऋतुकालगामी\\
तत्वविश्चेति }{पङ्क्तिपावनाः}{ भवन्ति । अथाप्यत्रोदाहरन्ति ।\\
पचनः पाचनत्रेता यस्य पञ्चान्मयो गृहे ।\\
सायं प्रातः प्रदीप्यन्ते स विप्रः पङ्क्षिपावनः ॥\\
सहस्रसम्मितं प्राहुः स्नातकं पूर्ववत् गुणैः ।\\
पञ्चाग्न्यादिगुणैर्युक्तः शतसाहस्र उच्यते \textbar{}\textbar{}\\
अनुचां यदि वा कृत्स्नां पड़ि योजनमायताम् ।\\
पुनाति वेदविद् विप्रो नियुक्तः प}{ङ्क्ति}{मूर्धनि ॥\\
दशोभयत इति । मातृतः पितृतश्चेत्यर्थः । त्रिशीर्षा = अथर्व रुद्रवैश्वा\\
नरशिरसामध्येता \textbar{} ऊर्ध्वरेता = नैष्ठिक ब्रह्मचारी \textbar{}
तत्ववित्=आत्मज्ञान\\
वान् गृहस्थोऽपि \textbar{} पचनः वैश्वदेवार्थपाकोपयोग्यावसथ्योऽम्यो वा,\\
केषां चिच्छाखायामावसथ्यातिरिक्तस्य तस्योक्तत्वात् \textbar{} पाचनः =
लभ्यः ।\\
एतस्य च पाकप्रयोजककर्तृत्वमावसथ्यानियतत्वात् गौण्या वृत्यो.

{ }{प}{ङ्क्ति}{पावनपावननिरूपणम् । }{ ७१\\
पचर्यते । स्नातकः = पूर्वोक्तस्त्रिविधः । सहस्रसम्मितमिति सहपङ्\\
क्त्युपविष्टब्राह्मणपङ्क्तिपावकत्वात् । केचित्तु
सहस्रब्राह्मणभोजनजन्य\\
फलप्रापकत्वात् सहस्रमिति । एवमग्निमान् शतसहस्र लक्षं तन्मितां\\
प}{ङ्क्ति}{ पावयतीत्यर्थः ।\\
ब्रह्माण्डपुराणे -\/-\/-\/-\\
षडङ्गवित् ध्यानयोगी सर्वतन्त्रस्तथैव च ।\\
यायावरश्च पञ्चैते विज्ञेयाः }{पङ्क्तिपावनाः}{ \textbar{}\textbar{}\\
ये भाष्ये वैदिके केचित् ये च व्याकरणे रताः ।\\
अधीयानाः पुराणं वा धर्मशास्त्राण्यथापि च ॥\\
चतुर्दशानां विद्यानामेकस्यापि च पारगः ।\\
श्राद्धकल्पं पठेद्यस्तु सर्वे ते }{पङ्क्तिपावनाः}{ ॥\\
सर्वतन्त्रः = शालीनाख्यो गृहस्थः \textbar{} यायावरः =वरणवृत्या यातीति
(?)\\
यायावरो गृहस्थविशेषः ।\\
विष्णुरपि । अथ }{पङ्क्तिपावनाः}{ । स्त्रिणाचिकेतः पञ्चाग्निः ज्येष्ठ.\\
सामगो वेदाङ्गस्याप्येकस्य पारगस्तीर्थपूतो यज्ञपूतः सत्यप्तो दान\\
पूतो मन्त्रपूतो गायत्रीजपानरतो ब्रह्मदेयानुसन्तानस्त्रिसौपर्णो\\
जामाता दौहित्रश्चेति ।\\
कूर्मपुराणे,\\
महादेवार्चनरतो महादेवपरायणः ।\\
वैष्णवो वाथ यो नित्यं स विप्रः पङ्गिपावनः । इति पङ्क्तिपावनाः
\textbar{}\textbar{}\\
अथ प}{ङ्क्ति}{पावनपावना उच्यन्ते ।\\
मत्स्यपुराणे ।,\\
यश्च व्याकुरुते वाचं यश्च मीमांसतेऽध्वरम् ।\\
सामस्वरविधिज्ञश्च प}{ङ्क्ति}{पावनपावनः \textbar{}\textbar{}}

{वृद्धमनुः,\\
पूवार्द्ध तुल्यमेव ।\\
यश्चावैत्यात्मकैवल्यं पङ्गिपावनपावन. \textbar{}\\
यश्च व्याकुरुते वाचं प्रकृतिप्रत्ययविभागेन वाचः संस्कुरुते वैया-\\
करण इति यावत् । अध्वरं यज्ञं मीमांसते विचारयति मीमांसक इति\\
यावत् । आत्मकैवल्यम्=आत्मैकत्वम् ।\\
ब्रह्मवैवर्ते ।\\
वेदवेदाङ्गवेिद् यज्वा शान्तो दान्तः क्षमान्वितः ।

{७२ वीरमित्रोदयस्य श्राद्धप्रकाशे-\\
वीतस्पृहस्तपस्वी च }{पङ्क्तिपावनपावनः}{ ।\\
न रज्यते न च द्वेष्टि }{पङ्क्तिपावनपावनः}{ \textbar{}\textbar{}\\
सौरपुराणे ।\\
यो न निन्दति न द्वेष्टि न शोचति न काङ्क्षति ।\\
आत्मारामः पूर्णकामः }{पङ्क्तिपावनपावनः}{ \textbar{}\textbar{}\\
यस्तु सर्वाणि भूतानि आत्मन्येवानुपश्यति ।\\
सर्वभूतेषु चात्मानं स वै पावनपावनः \textbar{}\textbar{}\\
अशनायापिपासाभ्यां शीतोष्णादिभिरेव च ।\\
अस्पृष्टः शोकमोहाभ्यां }{पङ्क्तिपावनपावनः}{ \textbar{}\textbar{}\\
ओमित्येकाक्षरं ब्रह्म व्याहरत्यनिशं शुचिः ।\\
आत्मैकतावचीर्थोऽसौ }{पङ्क्तिपावनपावनः}{ ॥\\
गरुडपुराणे ।\\
साङ्ख्यशास्त्रार्थनिपुणो योगशास्त्रार्थतत्ववित् ।\\
वेदान्तनिष्ठबुद्धिश्च प्रत्याहारपरायणः \textbar{}\textbar{}\\
प्रियाऽप्रियाभ्यामस्पृष्टः}{ पङ्क्तिपावनपावनः}{ ।\\
प्राणायामपरो धीरो मैत्रः करुण एव च ॥\\
अध्यात्मज्ञानपूतात्मा }{पङ्क्तिपावनपावनः}{ \textbar{}\\
साङ्ख्यं वेदान्ताविरोधि । इति पङ्क्तिपावनपावनाः ॥\\
अथ योगिनां श्राद्धे नियोगः ।\\
ब्रह्मवैवर्ते ।\\
क्रियया गुरुपूजाभिर्योगं कुर्वन्ति योगिनः ।\\
तेन चाप्याययन्ते ते पितरो योगवर्द्धनात् ॥\\
आप्यायिताः पुनः सोमपितरो योगभूषिताः ।\\
आप्याययन्ते योगेन त्रैलोक्यं तेन जीवति ॥\\
पितॄणां हि बलं योगो योगात्सामः प्रवर्तते ।\\
तस्माच्छ्राद्धानि देयानि योगिनां यक्षतः सदा ॥\\
बहुछिद्रः पुरा प्रोक्तो मैत्रो यज्ञो महर्षिभिः ।\\
निष्प्रत्यूहश्च निश्छिद्रो जायते योगरक्षया ॥\\
जीवपरयोरेकीभावलक्षण: सम्बन्धो योगस्तद्वान् योगी \textbar{} नि-\\
प्रत्यूहः-यातुधानादिकृतोपघात राहतः । निश्छिद्रः सकलः ।\\
मार्कण्डेयपुराणेऽपि -\/-\\
योगिनश्च }{तथा}{ श्राद्धे भोजनीया विपश्चिता ।\\
योगाधारा हि पितरस्तस्मात्तान् पूजयेत् सदा ॥

 प्रशस्तब्राह्मणानुकल्पनिरूपणम् । ७३ 

{सदेति, प्रकृतामावास्याष्टकादिकाल इत्यर्थः ।\\
}{तथा}{,\\
ब्राह्मणानां सहस्त्रेभ्यो योगी त्वप्राशनो यदि ।\\
यजमान च भोक्त} ॄ{श्च नौरिवाम्भसि तारयेत् \textbar{}\textbar{}}

{तथा}{,\\
पितृगाथास्तथैवात्र गीयन्ते पितृवादिभिः ।\\
या गीताः पितृभिः पूर्वमैलस्यासन्महीपतेः ॥\\
कदा नः सन्ततावग्रथः कस्य चिद्भविता सुतः ।\\
यो योगिभुक्तशेषान्नाद् भुवि पिण्डान् प्रदास्यति ॥\\
ऐलस्यैलनाम्नः ।\\
इति योगिनां श्राद्धे नियोगकथनम् ।\\
अथ तेषामेव गृहस्थादिभ्योऽधिकत्वमुक्त वायुपुराणे,\\
गृहस्थानां सहस्रेण वानप्रस्थशतेन च ।\\
ब्रह्मचारिसहस्रेण योंगी त्वेको विशिष्यते ॥\\
कूर्मपुराणेऽपि,\\
तस्माद्यत्नेन योगीन्द्रमीश्वरज्ञानतत्परम् ।\\
भोजयेद्धव्यकव्येषु अलाभादितरान् द्विजान् ॥\\
योग्यतिक्रमणे दोषश्चोक्तः, छागलेयेन,\\
योगिनं समतिक्रम्य गृहस्थं यदि पूजयेत् ।\\
न तत् फलमवाप्नोति सर्व गोत्र प्रतापयेत् \textbar{}\textbar{}\\
गृहस्थम्=अयोगिनम् । तत्फलम् = }{तस्य}{ श्राद्धादिकर्मणः फलम् । गो\\
त्रम्= गोत्र जान्, योग्यतिक्रमजनितप्रत्यवायाग्निना, प्रतापयेत्, दाहयेत्
।\\
वृद्धशातातपोऽपि -\\
योगिनं समतिक्रम्य गृहस्थं यदि भोजयेत् ।\\
न तत् फलमवाप्नोति स्वर्गस्थमपि पातयेत् ॥\\
तस्माद् गृहस्थाद्यतिक्रमेण योगिभ्यो ऽत्रे देयमिति । इति योगिनां\\
श्राद्धे नियोगः ।\\
अथ प्रस्तब्राह्मणानुकल्प उच्यते ।

{तत्र मनुः-\\
एष वै प्रथमः कल्पः प्रदाने हव्यकव्ययोः ।\\
अनुकल्पस्त्वयं ज्ञेयः सदा सद्भिरनुष्ठितः ॥ इति ।

{ }{( अ० ३ श्लो० १४७ )}{\\
१० वी० मि०\\


{७४ }{ वीरमित्रोदयस्य श्राद्धप्रकाशे-}{\\
एष = श्रोत्रियादिरूपो मुख्यः कल्प उक्तः, गौणस्तूच्यते इत्यर्थः ।\\
सदा सद्भिरनुष्ठित इति तु प्रशंसामात्रम् ।\\
ब्रह्माण्डपुराणे,\\
अलाभे योगिभिक्षूणां भोजयेद्ध्यानिनः शुभान्\\
असम्भवेऽपि तेषां वै नैष्ठिकान् ब्रह्मचारिणः ॥\\
तदलाभेऽप्युदासीनं गृहस्थमपि भोजयेत् ।\\
उदासीनः= दातुरसम्बन्धी मित्रारिभावशून्यो वा । भिक्षवः= त्रि\\
दण्डाः । ध्यानिनो=वानप्रस्थाः । गृहस्थाश्वोक्तश्रोत्रियत्वादिगुणयुक्ता
।\\
गृहस्थेष्वपि योगिनो मुख्यास्तदभावे वेदार्थज्ञा, तदभावे कृतवे-\\
दाध्ययनमात्रा इत्युकं ब्रह्मवैवर्ते -\\
योगिनः प्रथमं पूज्याः श्राद्धेषु प्रयतात्मभिः ।\\
तद्भावे वेदविदः पाठमात्रास्ततः परम् ॥\\
विनियोज्या महानेष पात्रसाध्यो विधिर्मतः ।\\
श्रोत्रियश्रोत्रियपुत्रस्यालाभे केवलयोः श्रोत्रियतत्पुत्रयोः प्रसक्तौ\\
कस्य मुख्यत्वमित्यपेक्षायामाह मनुः -\/- ( अ० ३ श्लो० १३६ )\\
अश्रोत्रीयः पिता यस्य पुत्र. स्याद् वेदपारङ्ग: ।\\
अश्रोत्रियो वा पुत्रः स्यात् पिता स्याद् वेदपार}{ङ्ग: }{ ॥\\
ज्यायांसमनयेोर्विद्याद्यस्य स्याच्छ्रोत्रियः पिता ।\\
मन्त्रसम्पूजनार्थन्तु सत्कारमितरोऽर्हति ॥\\
( अ० ३ श्लो० १३७ )\\
अनयोरिति, [स्वयंश्रोत्रियश्रोत्रियपुत्रयोः प्रसक्तौ कस्य
मुख्यत्वमित्य\\
पेक्षायामाह मनुः ।] अश्रोत्रियपितृकस्य स्वयं श्रोत्रियस्य,
स्वयमश्रोत्रि\\
यस्य श्रोत्रियपितृकस्य च मध्य इत्यर्थः । इतरः = पित्रश्रोत्रियः स्वयं
श्रो\\
त्रियः । मन्त्रसंपूजनार्थ वेदपारायणाद्यर्थं सत्कारमर्हति न श्राद्धे
ब्राह्मणार्थ.\\
मित्यर्थः । एतच्च श्रोत्रियपुत्रस्य वृत्तस्थत्वे ज्ञेयम् । तथा
चाग्निपुराणे -\\
किं कुलेन विशालेन वृत्तहीनस्य देहिनः ।\\
क्रिमयः किं न जायन्ते कुसुमेषु सुगन्धिषु ॥\\
जातूकर्ण्योऽपि,\\
किंकुलेन विशालेन वृत्तहीनस्य देहिनः ।\\
अपि विद्याकुलैर्युक्तान् वृत्तहीनान् द्विजाधमान् ॥\\
अनर्हान् हव्यकव्येषु वाङ्मात्रेणापि नार्चयेत् । इति ।

{ }{प्रशस्तब्राह्मणानुकल्पनिरूपणम् \textbar{} }{ ७५\\
वृत्तविद्याकुलानां मध्ये विद्यावृत्ते गरीयसी इत्याह-\\
मनुः ।\\
}{किं}{ ब्राह्मणस्य पितर}{ किं}{ वा पृच्छसि मातरम् ।\\
वृत्तं चेदस्ति विद्या वा तन्मातापितरौ स्मृतौ इति ॥\\
विद्यावृत्तयोर्वृत्तं ज्याय इत्याह स एव-\\
गायत्रीमात्रसारस्तु वरं विप्रः सुयन्त्रितः ।\\
नायन्त्रितश्चतुर्वेदी सर्वाशी सर्वविक्रयी ॥\\
असम्बन्धिनो मुख्या इत्युक्तं तदलाभे मनुराह -\/-\\
मातामहं मातुलं च स्वस्त्रीयं स्वसुर गुरुम् ।\\
दौहित्रं विपति बन्धुं ऋत्विग्याज्यांस्तु भोजयेत् ।\\
( अ० ३ श्लो० १४८ )\\
गुरु = उपाध्याय: \textbar{} विशो दुहितुः पतिर्विट्पतिः, जामातेति यावत्
।\\
अतिथिरिति }{मेघातिथिः}{ ।\\
बन्धुमातुलशालादिरसगोत्रस्तथा गुणी \textbar{}\\
कामं श्राद्धे ऽर्चयेोन्मित्रं नाभिरूपमपि त्वरिम् ॥ अभिरूपं.\\
णादि युक्तम् ।\\
विष्णुपुराणे ।\\
ऋत्विक्स्वस्त्रीयदौहित्रजामातृश्वसुररास्तथा ।\\
मातुलोऽथ तपोनिष्ठः पितृमातृस्वसुः पतिः ।।\\
शिष्याः संबन्धिनश्चैव मातापितृरतश्च यः ।\\
एतान् नियोजयेच्छ्राद्धे पूर्वोकान् प्रथमं नृप ॥\\
ब्राह्मणान् पितृतुष्टयर्थमनुकल्पेष्वनन्तरान् ।\\
पूर्वोक्तान् = श्रोत्रियादीन् \textbar{} अनन्तरान् = मातामहादीन्
अन्यगोत्रान् \textbar{}\\
एतेषामप्यभावे उक्ताः कूर्मपुराणे -\\
अभावे ह्यन्यगोत्राणां सगोत्रानपि भोजयेत् ॥ इति ।\\
गर्गोऽपि,\\
नैकगोत्रेहविर्दद्याद् यथा कन्या }{तथा}{ हविः ।\\
अभावे ह्यन्यगोत्राणामेकगोत्रांस्तु भोजयेत् ।\\
असमप्रवराभावे समानप्रवरांस्तथा ॥ इतेि ।\\
अत्र विशेष उक्तोऽ त्रिणा,\\
षड्भ्यस्तु परतो भोज्याः श्राद्धे स्युर्गोत्रजा अपि ।

{७६ }{वीरमित्रोदयस्य}{ श्राद्धप्रकाशे-\\
षड्भ्यः स्ववंशजेभ्यः । एतेन सगोत्रेषु सप्तमादूर्ध्वं भोजनया\\
इति सिद्धम् ।\\
एतदभावे आह- गौतम ।\\
भोजयेन्नित्य ऊर्ध्व गुणवन्तमिति ।\\
आपस्तम्बस्तु, भोजयेत् ब्राह्मणान् ब्रह्मविदो योनिगोत्रमन्त्रान्ते.\\
वास्यसंबद्धान् गुणधान्यां तु परेषां समुदेतः ।\\
मन्त्रसंबद्ध = शिष्यः । समुदेतः = सोदर्यः । अत्र च विशेष उक्तोऽत्रिणा
-\\
पिता पितामहो भ्राता पुत्रो वाथ सपिण्डकः ।\\
न परस्परमचर्या: स्युर्न श्राद्धे ऋत्विजस्तथा \textbar{}\textbar{}\\
ऋत्विकूपित्रादयोऽप्येते सकुल्या ब्राह्मणाः स्मृताः ।\\
वैश्वदेवे नियोक्तव्या यद्येते गुणवत्तरा इति ॥\\
अयं च पित्रादिनिषेधो न तदुद्देश्यकश्राद्धे, जीवत्पितृकस्य "वि.\\
प्रवच्चापि तं श्राद्धे " इत्यादिना पित्राद्युपवेशनविधानात् किन्तु
अन्यो\\
द्देश्यकश्राद्ध एवेति बोध्यम् । ऋत्विजोऽपि श्राद्धेन भवन्तीत्यर्थ ।\\
अतश्च मनुवचनेऽपि ऋत्विग्विधिर्वैश्वदेवस्थान एवेति बोध्यम् ।\\
निर्गुणस्याप्यनुकल्पत्वमाह वशिष्ठ -\\
आनृशंस्यं परो धर्मो याचते यत् प्रदीयते ।\\
अयाचतः सीदमानान् सर्वोपायैर्निमन्त्रयेत् \textbar{}\textbar{}\\
सीदमानेभ्योऽयाचकेभ्यो देयमिति मुख्योधर्मस्तदभावे याच\\
मानेभ्योऽपीति ।\\
ननु मुख्याभावेऽनुकल्पस्य न्यायलभ्यत्वात् वचनमनर्थक मिति\\
वेत् सत्यम्, अनियमेन प्राप्तस्य नियमार्थानि वचनानि इति नानर्थ-\\
क्यम्, निषिद्धानामपि वा संबन्धिनामनुकल्पत्वार्थ वचनानीति ।\\
इत्यनुकरूपनिरूपणम् ।\\
अथ सन्निहितब्राह्मणानामनतिक्रमणीयत्वमुच्यते ।\\
तत्र तदतिक्रमे दोष उक्तः शातातपपराशराभ्याम् -\\
संन्निकृष्टमधीयान ब्राह्मण यस्त्वतिक्रमेत् \textbar{}\\
भोजने चैव दाने च दद्दत्यासप्तमं कुलमिति । सन्निकृष्टं = निकटस्थम् ।\\
सन्निहितस्याप्यवैदिकस्यातिक्रमे दोषाभाव उक्तो व्यासवशिष्ठशातातपैः\\
ब्राह्मणातिक्रमो नास्ति विप्रे वेदविवर्जिते ।\\
ज्वलन्तमग्निमुत्सृज्य न हि भस्मनि हूयते \textbar{}\textbar{}}

{ }{ }{वर्ज्यब्राह्मणनिरूपणम् \textbar{} }{ ७७\\
यत्तु भविष्यपुराणे,\\
सन्निकृष्टं द्विज यस्तु शुक्लजातिं प्रियंवदम् ।\\
मूर्ख वा पण्डितं वापि वृत्तहीनमथापि वा ॥\\
नातिक्रामेन्नरो विद्वान् दारिद्रयाभिहतं}{ तथा}{ ।\\
अतिक्रम्य द्विजो घोरे नरके पातयेत् स्वकान् ॥\\
तस्मान्नातिक्रमेत् प्राज्ञो ब्राह्मणान् प्रातिवेश्यकान् ।\\
सम्बन्धिनस्तथा सर्वान् दौहित्रं विट्पतिं }{तथा }{।\\
भागिनेयं विशेषण }{तथा}{ सम्बन्धिनः खग \textbar{}\textbar{}\\
इति मूर्खस्यापि सन्निकृष्टस्याऽनतिक्रम उक्तः, स गुणयुक्तास\\
न्निहितब्राह्मणालाभे । अत एव बौधायनः -\\
यस्य त्वेकगृहे मूर्खो दूरे वापि बहुश्रुतः ।\\
बहुश्रुताय दातव्यं नास्ति मूर्खे व्यतिक्रमः ॥ इति ।\\
एवञ्च सन्निहितानतिक्रमश्रवणं गुणयुक्तासन्निहितालाभे इति\\
ध्येयम् । अथवा सन्निहितानतिक्रमश्रवणं श्राद्धादन्यत्र ज्ञेयम्, तत्र\\
श्राद्धपदाश्रवणात्, शेषभोजनपरं वेति । एष चातिक्रमविचारो स\\
त्रिहितस्य वृत्तस्थत्वे ।\\
}{तथा}{ च भारते,\\
गायत्रीमात्रसारोऽपि ब्राह्मणः पूजितः खग ।\\
गृहासन्नो विशेषेण न भवेत् पतितः स चेत् ॥ इति ।\\
अथ श्राद्धे वर्ज्या ब्राह्मणा ।\\
तत्र वर्ज्येषु हव्यकव्यदानं न फलदमित्याह -\\
यमः,\\
ऊषरे तु यथेहोप्तं बीजमाशु विनश्यति ।\\
}{तथा}{ दत्तमनर्हेभ्यो हव्यं कव्य न रोहति ॥\\
ऊषरः = क्षारमृत्तिकादेशः । न केवलमनहेभ्यो दानमफलं किन्तु\\
दोषापादकमपीत्युक्तं वाराहपुराणे,\\
न श्राद्धे भेोजनीयाः स्युरनी ब्राह्मणाधमाः ।\\
यैर्भुक्ते नश्यति श्राद्ध}{ पितॄन्}{ दातृश्व पातयेत् ॥\\
पतनप्रयोजकीभूताधर्मयुक्तान्नकुर्यादित्यर्थ. ।\\
भोक्तुरप्यनर्हस्य दोष इत्युक्त मनुशातातपाभ्याम् -\\
यावतो ग्रसते पिण्डान् हव्यकव्येष्वमन्त्रवित् ।

{७८ }{वीरमित्रोदयस्य}{ श्राद्धप्रकाशे-}{\\
तावतो ग्रसते प्रेत्य ( १ ) दाप्तान् ऋष्टीरयोगडान् ॥\\
( अ० ३ श्लो १३३ )\\
अमन्त्रवित् = वेदाध्ययनशून्यः । दप्तिान्=अग्नितुल्यान् । ऋष्टीः =
आयुध-\\
विशेषान् \textbar{} अयोगडाः=अयोगोलकाः \textbar{} तत्र निन्द्यानाह -\\
मनुः, ( अ० ३ श्लो० १३८ )\\
न श्राद्धे भोजयेन्मित्रं धनैः कार्योऽस्य सङ्ग्रहः ।\\
नारि}ं{ निमन्त्रयेद्विद्वान् न श्राद्धे भोजयेद् द्विजम् ॥\\
गुणवतोऽपि मैत्रत्व पुरस्कारण प्रतिषेधः । मनोवाक्कायैरानुकू\\
ल्यैकप्रवृत्तिमैत्रीसंग्रहोऽनुरोध. । अरिः=शत्रुः । मित्रनिषेधो
गुणवदुदा\\
तीनातिक्रमणेन मित्रनिमन्त्रणस्य निषेधात् । शत्रुस्तु गुणवदुदासी\\
नाभावे गुणवानप्यनुकल्पत्वेनापि न ग्राह्य इत्युक्तं मनुना -\/-\\
कामं श्राद्धेऽर्चयेन्मित्र नाभिरूपमपि त्वरिम् ।\\
द्विषता हि हविर्भुक्त भवति प्रेत्य निष्फलम् ॥\\
( अ० ३ श्लो० १४४ )\\
काममिति पुण्यचद्ब्राह्मणाभावे \textbar{} अभिरूपां=गुणयुक्तः ॥
एकगोत्रा-\\
दयो न भोजनीया इत्युक्तं वायुपुराणे,\\
न भोजयेदेकगोत्रान्समानप्रवरांस्तथा ।\\
एतेभ्यो हि हविर्दत्तं भुञ्जते न पितामहाः ॥\\
असगोत्रब्राह्मणासम्भवेऽपि सगोत्राणां भोज्यत्वं षड्भ्य ऊर्ध्व\\
मेव, अत एव षण्णामेव पुरुषाणां निषेधमाह - अत्रिः-\/-\\
षड्भ्यस्तु पुरुषेभ्योऽर्वाक् न श्राद्धेयास्तु गोत्रिणः ।\\
षड्भ्यस्तु परतो भोज्या श्राद्धे स्युर्गोत्रजा अपि }{॥}{\\
तस्याप्यसम्भवे गौतम आह-\\
भोजयेदूर्ध्वं त्रिभ्यो गुणवत्तममिति ।\\
अत्रि:,\\
पिता पितामहो भ्राता पुत्रो वाथ सपिण्डकः ।\\
न परस्परमर्थ्याः स्युर्न श्राद्धे ऋत्विजस्तथा ॥\\
योनिमन्त्रसम्बद्धा अपि न भोजनीया इत्युक्तं ब्रह्माण्डपुराणे-\/-\\
न भोज्या योनिसम्बद्धा गोत्रसम्बन्धिनस्तथा ।\\
मन्त्रान्तेवासिसम्बद्धाः श्राद्धे विप्राः कथञ्चन ॥

% \begin{center}\rule{0.5\linewidth}{0.5pt}\end{center}

{\\
(१) दीप्तशूलर्ध्ययोगुडानिति मुद्रितमनुस्मृतौ पाठः ।

{वर्ज्यब्राह्मणनिरूपणम् \textbar{} ७९}

{ब्रह्मवैवर्ते,\\
शिष्याश्च ऋत्विजो याज्याः सुहृदः शत्रवस्तथा ।\\
श्राद्धेषु श्वसुरः श्यालो न भोज्या मातुलादयः \textbar{}\textbar{}\\
मनुः - ( अ० ३ श्लो० १६८ )\\
ब्राह्मणो ह्यनधीयानस्तृणाग्निरिव शाम्यति ।\\
तस्मै हव्यं न दातव्यं न हि भस्मनि हूयते ॥\\
कूर्मपुराणे,\\
यस्य वेदश्च वेदी च विच्छिद्येते त्रिपौरुषम् ।\\
स वै दुर्ब्राह्मणोऽनर्हः श्राद्धादिषु कदाचन \textbar{}\textbar{}\\
वेदिः = दर्शपूर्णमासिकी, सौमिकी वा । वेदवेदीविच्छेदस्य प्रत्ये\\
क निमित्तत्वं निमित्तविशेषणस्य साहित्यस्य हविरुभयत्ववत् ।\\
विष्णुः,\\
न वार्यपि प्रयच्छेत बिडालव्रतिके द्विजे ।\\
न बकव्रतिके पापे नावेदविदि धर्मवित् \textbar{}\textbar{}\\
बैडालव्रतिकचोक्तो मनुना-\/-\\
धर्मध्वजी सदा लुब्धः साग्निको लोकदाम्भिकः ।\\
बैडालव्रतिको ज्ञेयो हिंस्रः सर्वाभिसन्धकः ॥\\
( अ० ४ श्लो० १९५ )\\
यश्च धर्मध्वजो नित्यं सुराध्वज इवोच्छ्रितः ।\\
प्रच्छन्नानि च पापानि बैडालं नाम तद्व्रतम् ।\\
यमस्तु प्रकारान्तरेणाह -\\
य. कारणं पुरस्कृत्य व्रतचर्या निषेवते ।\\
पापं व्रतेन प्रच्छाद्य बैडालं नाम तद् व्रतम् ॥\\
अर्थ च विपुलं गृह्य ( १ ) हित्वा लिङ्गं निवर्तयेत् ।\\
आश्रमान्तरितं रक्षेत् बैडालं नाम तद् व्रतम् ॥\\
यो ह्यन्यायेन विपुलमर्थ संगृह्य राजादिभिस्तदपहारमाशङ्कमान'\\
पूर्वाश्रमलिङ्गानि परिवर्ज्योतिमाननीयाश्रमस्वीकारेण तद्रव्यं रक्ष\\
ति }{तस्य}{ तद्रतं बैडालसंज्ञं भवति ।\\
}{तथा}{,\\
प्रतिगृह्याश्रमं यस्तु स्थित्वा तत्र न तिष्ठति ।\\
आश्रमस्य तु लोपेन बैडालं नाम तद् व्रतम् ॥

% \begin{center}\rule{0.5\linewidth}{0.5pt}\end{center}

(१) दत्वेति दानखण्डे पाठः ।

{८० }{वीरमित्रोदयस्य}{ श्राद्धप्रकाशे-}{\\
यतीनामाश्रमं गत्वा प्रत्यायास्यति यः पुनः }{\textbar{}}{\\
यतिधर्मविलोपेन बैडालं नाम तद् व्रतम् \textbar{}\textbar{}\\
बकवृत्तिश्श्रोक्तो यमेन,\\
(१) अधोष्टिकृतिकः स्वार्थसाधनतत्परः \textbar{}\\
(२) स्वयं मिथ्याविनीतश्च बकवृत्तिपरो द्विजः ॥\\
अवेदविन्मन्त्रार्थज्ञानशून्यो नैकृतिकः । कृतन इति मिश्राः ।\\
वायुपुराणे\\
प्राह वेदान् वेदभृतो वेदान् यश्चोपजीवति ।\\
उभौ तौ नार्हतः श्राद्धं पुत्रिकापतिरेव च ॥\\
वेदभृतो वेदान् यः प्राहेत्यन्वयः वेदभृतो = भृतकाध्यापकः\\
वेतनग्रहणेन यः पाठयति स इत्यर्थः, }{तथा}{ यः सपणं वेदपारायणा\\
दितो द्रव्यमुपजीवति तौ, पुत्रिकापतिर्जामाता }{च}{ श्रद्धानर्हा-\\
इत्यर्थः ।\\
व्यासशातातपौ-\/-\\
( ३ ) सन्ध्या हीने व्रतभ्रष्टे विप्रे वेदविवर्जिते ।\\
दीयमानं रुदत्यन्न }{किं}{ मया दुष्कृतं कृतम् ॥\\
व्रतानि = महानाम्म्यादीनि तद्भ्रष्टे । केचित्तु व्रतात् ब्रह्मचर्याद्\\
भ्रष्ट इति वदन्ति ।\\
यमः,\\
न प्रतिग्रहमईन्ति वृषलाध्यापका द्विजाः ।\\
शुद्रस्याध्यापनाद्विप्रः पतत्यत्र न संशयः ॥\\
वृषलः = शूद्रः । श्राद्धानिमन्त्रणीय\\
ब्राह्मणानुकत्वाह-\/-कास्यायनः -\\
द्विर्नमशुक्लावक्लिघश्यावदन्तविप्रजननव्याधितव्यङ्गश्वित्रिकुष्ठि\\
कुनाखिवर्जनामिति । द्विर्नग्नो = दुश्चर्मा, खलातेर्वेति हेमाद्रिः ।
शुक्लोऽति\\
गौरः । विक्लिधो=हीनौष्ठः ।\\
तथाह । सुमन्तु,\\
यस्य नैवाधरोष्ठाभ्यां छाद्यते दशनावली ।\\
विक्लिधः स तु विज्ञेयो ब्राह्मणः पङ्क्तिदूषणः ॥

% \begin{center}\rule{0.5\linewidth}{0.5pt}\end{center}

{(१) अधोष्टिर्नैष्कृतिक इति मनुस्मृतौ पाठ: \textbar{}\\
(२) शठ इति दानखण्डे मनुस्मृतौ च पाठः ।\\
(३) नष्टशोचे, इति दानखण्डे पाठः ।

{ वर्ज्यब्राह्मणनिरूपणम्}{ । ८१\\
विचर्चिकाबहुल इति कल्पतरुः । गलव्रण इति शङ्खधरः । श्याव.\\
दन्तः = स्वभावात् कृष्णदन्तः । विद्धप्रजननः = छिन्नशिश्नोपरितनचर्मा ।\\
व्याधितः = दुर्विचिकित्स्यव्याधियुक्तः । ताश्च व्याधयो देवलेनोक्ता ।\\
उन्मादस्त्वग्दोषो राजयक्ष्मा श्वासो मधुमेहो भगन्दरो महोदरोऽ.\\
श्मरीत्यष्टौ महापापरोगाः । श्वित्री - श्वेतकुष्ठी }{\textbar{}}{ कुनखी =
उपाध्यभावे कु·\\
त्सितनखवान् ।\\
शङ्खलिखितो,\\
न वै दुष्टान् भोजयेत् । दुश्चर्मकुनखकुष्ठिश्वित्रिभ्यावदन्ता\\
ये चान्ये हीनातिरिक्ताङ्गास्तानपि वर्जयेत् । दुश्चर्मा-गजचर्मरोगी ।\\
हीनातिरिक्ताङ्कः = सङ्ख्यया परिमाणेन वा हीनमतिरिकं वाङ्ग यस्य\\
स हीनातिरिक्ताङ्गः ।\\
ब्रह्मपुराणे,\\
भोक्तुं श्राद्धे न चार्हन्ति दैवोपहतचेतसः ।\\
षण्डो मूकश्च कुनखी खल्वाटो दन्तरोगधान् ॥\\
श्यावदन्तः पूतिनासः छिनाङ्गश्चाधिकाङ्गुलिः ।\\
गलरोगी च गडुमान् स्फुटिताच सज्वरः ॥\\
खञ्जतूवरमण्डाश्च ये चान्ये हीनरूपिणः ।\\
षण्डाश्चोक्ता देवलेन-\/-\\
षण्डको वातजः षण्डः षण्ड: क्लीबो नपुंसकः ।\\
कीलकश्चेति षट्कोऽयं क्लीबभेदो विभाषितः ॥\\
तेषां स्त्रीतुल्यवाक्चेष्टः स्त्रीधर्मा षण्डको भवेत् ।\\
पुमान् भूत्वा सलिङ्गानि पश्चाच्छिन्द्यात्तथैव च ॥\\
स्त्री च पुंभावमास्थाय पुरुषाचारवद्गुणा ।\\
वातजो नाम षण्डः स्यात् स्त्रीषण्डो वापि नामतः ॥\\
असल्लिङ्गोऽपि षण्डः स्यात् षण्डस्तु म्लान मेहनः ।\\
अमेध्याशी पुमान क्लीषो नष्टरेता नपुंसकः ॥\\
स कीलक इति ज्ञेयो यः क्लैन्यादात्मनः स्त्रियम् ।\\
अन्येन सह संयोज्य पश्चात्तामेव सेवते ॥ इति ।।\\
मूकः = वागिन्द्रियरहितः खल्वाटः = शिरसि समूलकेशशून्यः । पूति\\
नास = पूतिवन् नासा यस्य सः । गलरोगी = गण्डमालादियुक्तः । गड्डुमान् =\\
कुब्जः । खञ्जः = पादहीनः । तूवरो =\\
यौवनेऽजातश्मश्रुः । मण्डः=मण्ड़ा\\
११ वी० मि०

{८२ }{वीरमित्रोदयस्य}{ श्राद्धप्रकाशे-}{\\
रूयनेत्ररोगवान् । अथवा जङ्घायां जायमानो व्रणविशेषो मण्डस्त.\\
द्वान् मण्डः ।\\
स्कन्दपुराणे ।\\
काणाः कुण्डाश्च मण्डाश्च मूकान्धबधिरा जड़ाः ।\\
अतिदीर्घा अतिर्हस्वा अतिस्थूलाः भृशं कृशाः ॥\\
निर्लोमानोऽतिलोमानो गौराः कृष्णा अतीव ये ।\\
एतान् विधर्जयेत् प्राज्ञः श्राद्धेषु श्रोत्रियानपि ॥\\
काण:- पकनेत्ररहितः }{\textbar{}}{ अन्धः = नेत्रद्वयरहितः । जडः =
संकल्प}{वि}{•\\
कल्पात्मकमनोव्यापारशून्यः ।\\
शालङ्कायनः ।\\
अविद्धकणैर्यद् भुक्त लम्बकर्णैस्तथैव च ।\\
दग्धकर्णैश्च यद् भुक्तं तद्वै रक्षांसि गच्छति ॥\\
लम्बकर्णश्चोतो गोभिलेन,\\
हनुस्थलादधः कर्णो लम्बौ तु परिकीर्तितौ ।\\
व्यङ्गुलौ व्यङ्गुलौ शस्तौ तेन शातातपोऽब्रवीत् }{\textbar{}\textbar{}}{\\
तेन व्यङ्गुलञ्यङ्गुलत्वयोर्न लम्बकर्णत्वमित्यर्थः ।\\
भविष्यपुराणे-\/-\/-\\
माब्राह्मणाय दातव्यं न देयं ब्राह्मणक्रिये ।\\
म ब्राह्मणब्रुवे चैव न च दुर्ब्राह्मणे धनम् ॥\\
अब्राह्मणाश्चोक्ता व्यासादिभिः -\\
व्यास:-\\
ब्रह्मबीजसमुत्पन्नो मन्त्रसंस्कारवर्जितः ।\\
जातिमात्रोपजीवी च भवेदब्राह्मणस्तु सः ।

{मनुः,\\
नानृक् ब्राह्मणो भवति न वणिक् न कुशीलवः ।\\
न शूद्रप्रेषणं कुर्वन् न स्तेनो न चिकित्सकः ॥\\
अनृक् = ऋग्वर्जितः }{\textbar{}}{ कुशीलवो - नृत्यगीतवृत्तिः
}{\textbar{}}{\\
शातातपः प्रकारान्तरेणाह-\\
अब्राह्मणास्तु षट् प्रोक्ता ऋषिः शातातपोऽब्रवीत् ।\\
आद्यो राजभृतस्तेषां द्वितीयः क्रयविक्रयी ॥\\
तृतीयो बहुयाज्यः स्यात् चतुर्थो ग्रामयाजकः ।\\
पञ्चमस्तावकस्तेषां ग्रामस्य नगरस्य वा ॥

{ }{वर्ज्य}{ब्राह्मणनिरूपणम् । }{ ८३\\
अनागतां तु यः पूर्वा सादित्यां यश्च पश्चिमाम् ।\\
नोपासीत द्विजः सन्ध्यां स षष्ठोऽब्राह्मणः स्मृतः ।\\
देवलो}ऽ{न्यथाह-\\
कूपमात्रोदकग्रामे विप्रः संवत्सरं वसन् ।\\
शौचाचारपरिभ्रंशाद् ब्राह्मण्याद्विप्रमुच्यते ।\\
इत्यब्राह्मणाः । ब्राह्मणक्रियो = ब्राह्मणस्य क्रियेव क्रिया यस्य स
क्षत्रिया-\\
दिः । ब्राह्मणबुवाश्चोक्काः व्यासादिभिः ।\\
गर्भाधानादिभिर्युक्तः तथोपनयमाप्तवान् ।\\
न धर्मवान् न चाधीते स भवेद् ब्राह्मणब्रुवः ॥\\
दुर्ब्राह्मणाश्श्रोक्ताः हारीतादिभिः ।\\
पक्षिमीनवृषघ्ना ये सर्पकच्छपघातिनः ।\\
नानाजन्तुवधे सक्ताः प्रोक्ता दुर्ब्राह्मणा हि ते ॥\\
इति दुर्ब्राह्मणाः ।\\
अथ वृषलीपतयः ।\\
तत्रोशनाः ।\\
बन्ध्या च वृषली ज्ञेया कुमारी या रजस्वला ।\\
यस्त्वेतामुद्वहेत् कन्यां ब्राह्मणो ज्ञानदुर्बलः
}{\textbar{}\textbar{}}{\\
अश्राद्धेयमपाङ्केयं तं विद्यात् वृषलीपतिम् ।\\
चमत्कारखण्डे तु अन्यथोकम् ।\\
वृषो हि भगवान् धर्मस्तस्य यः कुरुते त्वलम् ।\\
वृषलं तं विदुर्देवाः सर्वधर्मबहिस्कृतम् }{\textbar{}\textbar{}}{\\
तथैव या स्त्री वृषली तत्पतिर्वृषलीपतिः ।\\
अलं=वारणम् । तथैव धर्मनिराकरणकर्त्री ।\\
प्रभासखण्डे ।\\
वृषलीत्युच्यते शूद्रा तस्या यश्च पतिर्भवेत् ।\\
लालोच्छिष्टस्य संयोगात् पतितो वृषलीपतिः ॥\\
स्ववृषं तु परित्यज्य परेण तु वृषायते ।\\
वृषली सा तु विज्ञेया म शूद्रा वृषलीयते }{\textbar{}\textbar{}}{\\
चाण्डाली बन्धकी वेश्वा रजस्था या च कन्यका }{\textbar{}}{\\
ऊढा या }{च}{ स्वगोत्रा स्यात् वृषल्यः संप्रकीर्तिताः ।\\
}{आपस्तम्बः \textbar{}}

{ नीलाकर्षणकर्ता तु नीलीवस्त्रानुधारकः ।\\


{८४ }{वीरमित्रोदयस्य}{ श्राद्धप्रकाशे-}{\\
किञ्चिन्न }{तस्य}{ दातव्यं चाण्डालसदृशो हि सः ॥\\
नालीक्षेत्रे कर्षणकर्तेत्यर्थः ।\\
मनु:-.\\
अवतैयैर्द्विजैर्भुक्तं परिवेत्रादिभिस्तथा ।\\
आपाङ्केयैर्यदन्यैश्च तद्वै रक्षासि भुञ्जते ॥\\
( अ० ३ लो० १७० )\\
अवताः = असंयताः शौचाचारविवर्जिता इति यावत् । परिवेत्रादय\\
इत्यादिशब्दात् परिवित्तिः । तेषां लक्षणमुक्तं वृद्धयाज्ञवल्क्यादिभिः
।\\
तत्र वृद्धयाज्ञवल्क्यः-\\
आवसथ्यमनाहृत्य त्रेतायां यः प्रवर्तते ।\\
सोऽनाहिताग्निर्भवति परिषेत्ता तथोच्यते ॥\\
आवसथ्यम्= औपासनाझिम्। अनादृत्य=असंगृह्य }{\textbar{}}{ त्रेता
=र्गाहपत्याहवनी\\
यदक्षिणाग्नयः । अनाहिताग्निः=आधानजन्यफलरहितो भवतीत्यर्थः ।\\
मनुः ।\\
दाराग्निहोत्रसंयोगं कुरुते योऽप्रजे स्थिते ।\\
परिवेत्ता स विज्ञेयः परिवित्तिस्तु पूर्वजः ।\\
( अ० ३ श्लो० १७१ )\\
स्थिते दाराग्निहोत्रसंयोगं विनेति शेषः । अग्निहोत्रशब्द: कर्मव\\
धनोऽपि आधाने लाक्षणिक इति मेधातिथिप्रभृतयः । वस्तुतस्तु यथा\\
श्रुतः कर्मवाचक एव । तेन यथा ज्येष्ठेऽकृतदारे दारसंयोगात् दोषः\\
तथा ज्येष्ठेन अग्निहोत्रानारम्भे कनिष्ठस्याग्निहोत्र इत्यर्थः । न च
गार्ग्ये\\
णाघाने परिवित्यादेरुक्तत्वादेकश्रुतिकल्पनालाघवाद झिहोत्र पदमा\\
धानपरमिति वाच्यम् । भिन्नश्रुतिकल्पनस्य फलमुखत्वेनादुष्टत्वान्त\\
दनुरोधेनाग्निहोत्रशब्दे लक्षणायोगात् ।\\
दर्शेष्टि}ं{ पौर्णमासेष्टि}ं{ सोमेज्यामग्निसंग्रहम् ।\\
अग्निहोत्रं विवाहं च प्रयोगे प्रथमे स्थितम् ।\\
न कुर्याज्जनके ज्येष्ठे सोदरे वाप्यकुर्वति ॥\\
इत्यग्निहोत्रस्य त्रिकाण्डमण्डनेन पृथगुपादानाच्च ।\\
गर्ग:-\\
सोदर्यथे सत्यपि ज्येष्ठे न कुर्याद्दारसंग्रहम् ।\\
भावसथ्यं तथाधानं, पतितस्त्वन्यथा भवेत् ।

{ वर्ज्यब्राह्मणनिरूपणम्}{ । ८५\\
आवसथ्यम्=आवसथ्याधानम् । एतच्च दायविभागकालेऽग्निपरि-\\
ग्रह इत्येतत् पक्षाभिप्रायम्, विवाहकालीनस्य तु दारसङ्ग्रहमित्य.\\
नेनैव निरस्तत्वात् । पतितः = तत्तत्कर्मभ्यो न तु महापातकी परिवे\\
दनादीनामुपपातकेषु पाठात् । अयमेव पर्याहित इत्युक्त लौगाक्षिणा-\\
}{सोदर्थे}{ तिष्ठति ज्येष्ठे योऽग्न्याधेयं करोति हि ।\\
तयोः पर्याहितो ज्येष्ठः पर्याधाता कनिष्ठक ॥ इति ।\\
केषु चिदपवादमाह शातातपः ।\\
पितृव्यपुत्रसापत्नपरनारीसुतेषु च ।\\
ज्येष्ठष्वपि हि तिष्ठत्सु भ्रातृणां तु कर्नायसाम् ॥\\
दाराग्निहोत्रसंयोगे न दोषः परिवेदने ।\\
परनारीसुताः = ह्यामुष्यायणादयः परक्षेत्रे स्वीयपित्रोत्पादिताः । दत्त\\
क्कादयोऽपीति त्रिकाण्डमण्डनः ।\\
}{तथा}{-\\
क्लीबे देशान्तरस्थे च पतिते भिक्षुके }{तथा}{ ।\\
योगशास्त्राभियुक्ते च न दोषः परिवेदने ।\\
योगशास्त्राभियुक्ताः अतिविरक्ताः ।\\
कात्यायनः-\/-\\
देशान्तरस्थान क्लीबैकवृषणानसहोदरान् ।\\
वेश्यातिसक्तपतितशूद्रतुल्यातिरोगिणः ॥\\
जडमूकान्धवधिरकुब्जवामनषोडकान् ।\\
अतिवृद्धानभार्याश्च कृषिसकान् नृपस्य च ॥\\
धनवृद्धिप्रसक्तांश्च कामतोऽकारिणस्तथा ।\\
कुहकांस्तस्करांश्चापि परिविन्दन्न दुष्यति }{\textbar{}\textbar{}}{\\
एकवृषणो= रोगेण प्रवृद्धैकवृषणः । अतिरोगी = अचिकित्स्यरोगवान् ।\\
षोडको = भद्मचरणद्वयः । अभार्यो= नैष्ठिकब्रह्मचारी । कामतोऽकारिणः =
स्वेच्छ.\\
या विवाहमकुर्वाणाः }{\textbar{}}{ कुहकाः = वञ्चकाः }{\textbar{}}{ तस्करः
= ब्राह्मणसुवर्णव्यक्ति\\
रिक्तपरस्वापहारिणः । इतरस्य पतितपदेनैव सङ्ग्रहात् ।\\
देशान्तरस्थे तु वशिष्ठ आह-\\
अष्टौ दश द्वादश वर्षाणि ज्येष्ठं भ्रातरमनिविष्टमप्रतीक्ष.\\
माणः प्रायश्चित्ती भवतीति ।\\
अनिविष्टम्=अकृतविवाहं देशान्तरावस्थितम् ।

{८६ }{वीरमित्रोदयस्य}{ श्राद्धप्रकाशे-}{\\
गौतमोऽपि -\\
नष्टे भर्तरीति प्रक्रम्य द्वादशवर्षाणि ब्राह्मणस्य विद्यासम्बन्धेनेति\\
भर्तृप्रतीक्षाकालमुक्त्वाऽभिहितं भ्रातरि चैवं ज्यायसि यवीयान् क.\\
म्याग्न्याधेयेष्विति ।\\
नष्टे भर्तरि = कुत्र गत इत्यनिर्णीतेऽत्यन्तदूरदेशस्थित इति यावत् ।\\
विद्यासम्बन्धेनेति, विद्याग्रहणार्थे देशान्तरङ्गते भर्तरि
ब्राह्मणभार्यया द्वाद-\\
शवर्षाणि प्रतीक्ष्य भत्रै ध्वदेहिक कार्यम् । एवं ब्राह्मणे ज्येष्ठे
विद्या\\
ग्रहणार्थे देशान्तरं भ्रातरि गते द्वादशवर्षे प्रतीक्ष्य यवीयसाऽऽ\\
धानविवाहे कार्ये, कार्यार्थे तु गतेऽष्टौ द्वादश वर्षाणि प्रतीक्षणीयानि\\
तनोऽनागमं निश्चित्य कुर्यादिति तात्पर्यार्थः ।\\
अत एव सुमन्तुरप्रतीक्षणीयानाह -\\
व्यसनासक्तचितो }{वा}{ नास्तिको वाऽथवाग्रजः ।\\
कनीयान् धर्मकामस्तु आधानमथ कारयेत् ॥\\
षण्डादयस्तु विवाहानईत्वादेवाप्रतीक्षणीयाः ।\\
}{तथा}{ च स्मृतिः-\\
उन्मत्तः किल्विषी कुष्ठी पतितः क्लीब एव }{वा}{ ।\\
राजयक्ष्म्यामयावी च न न्याय्यः स्यात् प्रतीक्षितुम् ॥\\
खञ्जवामनकुब्जेषु गद्गदेषु जडेषु च ।\\
जात्यन्धे बधिरे मूके न दोषः परिवेदने }{\textbar{}\textbar{}}{\\
एवं विधे ज्येष्ठे सति कनिष्ठेन कृते विवाहे न दोष इत्यर्थः ।\\
सत्यधिकारिणि ज्येष्ठे तदनुज्ञायां सत्यां कनिष्ठस्याधाने दोषो ना\\
स्तीत्याह वशिष्ठः-\/-\\
अप्रजब्ध यदानग्निराध्यादनुजः कथम् ।\\
अग्रजानुमतः कुर्यादग्निहोत्रं यथाविधि ॥\\
अत्राग्निहोत्रशब्देनाघानमुच्यते उपक्रमानुरोधात् । एतश्चानुहा\\
ग्रहणं विवाहातिरिक्त सोदराणाम् ।\\
}{तथा}{ च हारीत:-\\
सोदराणान्तु सर्वेषां परिवेत्ता कथं भवेत् ।\\
दारैस्तु परिविद्यन्ते नाग्निहोत्रेण नेज्यया }{\textbar{}\textbar{}}{\\
इति अग्निहोत्रादिष्वनुज्ञाता न परिविद्यन्ते, वारै: पुनरनुश्चाक्ता-\\
अपि परिविद्यन्त इत्वर्थ इति हेमाद्रिः ।

{ }{ }{वर्ज्यब्राह्मणनिरूपणम् । }{ ८७\\
त्रिकाण्डमण्डनोऽपि आधान एव विशेषमाह-\\
ज्येष्ठे श्रद्धाविहीने सत्याधेयं तदनुज्ञया ।\\
पितुः सत्यप्यनुज्ञाने नादधीत कदाचन }{\textbar{}\textbar{}}{\\
पितर्यनाहितामा वप्यादधीताथवा सुतः ।\\
अग्निहोत्रं }{च}{ जुहुयादिति लौगाक्षिकारिका }{\textbar{}\textbar{}}{\\
पिता यस्याग्रजो भ्राता न कुर्याद्वा पितामहः ।\\
तपोऽग्निहोत्रं यज्ञं}{ वा}{ स }{वा}{ कुर्यात् कटाशयात् ॥\\
देशान्तरस्थे स एव-\\
प्रोषितस्तु यदा ज्येष्ठो न ज्ञायेताऽऽहितानलः ।\\
षड् वत्सरान् प्रतीक्षेत आदधीतानुजस्तदा ॥\\
यद्वा पापं भवेज्ज्येष्ठात् पूर्व भार्यापरिग्रहे ॥ इति ।\\
न तु आधानादाविति शेष इति कृतं पल्लवितेन ।\\
प्रकृतमनुसरामः । अपाङ्क्तेयाश्चोका मनुना,\\
ये स्तेनाः पतिताः क्लीवा ये च नास्तिकवृतयः ।\\
तान् हव्यकव्ययोर्विप्राननर्हान्मनुरब्रवीत् ॥\\
जटिलं चानधीयानं (१) दुर्वालं कितवं }{तथा}{ ।\\
याजयन्ति च ये प्रगान् तांश्च श्राद्धे न भोजयेत् ॥\\
चिकित्सकान् देवलकान् मांसविक्रयिणस्तथा ।\\
विपणेन च जीवन्ति वर्मास्ते हव्यकव्ययोः ॥\\
प्रेष्यो ग्रामस्य राशश्च कुनखी श्यावदन्तकः ।\\
प्रतिरोद्धा गुरोश्चैव त्यक्ताग्निर्वार्धुषिस्तथा
}{\textbar{}\textbar{}}{\\
यक्ष्मी च पशुपालश्च परिवेत्ता निराकृतिः ।\\
ब्रह्मद्विट् परिवित्तित्र गणाभ्यन्तरगश्च यः ॥\\
कुशीलवोऽवकीर्णी च वृषलीपतिरेव च ।\\
पौनर्भवश्च काणश्च यस्य चोपपतिर्गृहे ॥\\
भृतकाध्यापको यश्च भृतकाध्यापितश्च यः ।\\
शूद्रशिष्यो गुरुश्चैव वाग्दुष्टाः कुण्डगोलकौ ॥\\
अकारणात् परित्यक्ता मातापित्रोर्गुरोस्तथा ।\\
ब्राह्मर्योश्च सम्बन्धैः संयोगं पतितैर्गतः ॥

% \begin{center}\rule{0.5\linewidth}{0.5pt}\end{center}

{( १ ) दुर्बलमिति पाठान्तरम् ।\\


{८८ }{वीरमित्रोदयस्य श्राद्धप्रकाशे-}{\\
आगारदाही गरदः कुण्डाशी सोमविक्रयी ।\\
समुद्रयायी बन्दी च तैलिकः कूटकारकः ॥\\
पित्रा विवदमानश्च किङ्करो मद्यपस्तथा ।\\
पापरोग्यभिशस्तश्च दाम्भिको रसविक्रयी ।\\
धनुः शराणां कर्ता च यश्चाग्रेदिधिषूपतिः ।\\
मित्रध्रुग् द्यूतवृत्तिश्च पुत्राचार्यस्तथैव च
}{\textbar{}\textbar{}}{\\
भ्रामरी गण्डमाली च श्वित्र्यथो पिशुनस्तथा ।\\
उन्मत्तोऽन्धश्च वर्ज्यास्युर्वेदनिन्दक एव च ॥\\
हस्तिगोऽश्वोष्ट्रदमको नक्षत्रैर्यश्च जीवति ।\\
पक्षिणां पोषको यश्च युद्धाचार्यस्तथैव च }{\textbar{}\textbar{}}{\\
स्रोतसां भेदकश्चैव तेषां वावरणे रतः ।\\
गृहसंवेशको दूतो वृक्षारोपक एव च ॥\\
श्वक्रीडचोपजीवी च कन्यादूषक एव च ।\\
हिंस्रो वृषलवृत्तिश्च गणानां चैव याजकः }{\textbar{}\textbar{}}{\\
आचारहीनः क्लीबश्च नित्ययाचनकस्तथा ।\\
कृषिजीवी च शिल्पी च सद्भिर्निन्दित एव च ॥\\
औरभ्रिको माहिषिकः परपूर्वापतिस्तथा ।\\
प्रेतनिर्यापक श्चैव वर्जनीयाः प्रयक्षतः ॥\\
एतान् विगर्हिताचारानपायान् द्विजाधमान् ।\\
द्विजातिप्रवरो नित्यमुभयत्र }{विधर्जयेत्}{ ॥\\
( अ० श्लो० १५० - १६७)\\
स्तना=ब्राह्मणसुवर्णव्यतिरिक्तद्रव्यापहारकाः । पतिताः = द्विजाति.\\
कर्मभ्यो हानिः पतनं, तत्प्रयोजककर्मकर्तारः । महापातककर्तेति\\
मेघातिथिः । क्लीबः=षण्डः । स च पूर्वमुक्तः । नास्तिकाश्च "ये
व्यपेताः स्व.\\
कर्मभ्यो नास्तिकास्ते प्रकीर्तिताः" इति यमोक्ताः । अन्ये तु नास्ति पर-\\
लोक इत्येवमास्थिता नास्तिकास्तेषा वृत्तिराचारोऽश्श्रद्दधानता तद्वद्\\
वृत्तिर्येषां ते नास्तिकवृत्तयः । अथवा नास्तिकेभ्यो वृत्तिर्जीवनं
येषान्ते\\
}{तथा}{ । जटिलं = ब्रह्मचारिणम् । तं चानधीयानम्, अप्रारब्धाध्ययनम् ।\\
प्रारब्धाध्ययनस्य }{तस्य}{ विहितत्वात् । ननु तस्याश्रोत्रियत्वेन कथ\\
प्रसक्तिरिति चेत् "व्रतस्थमपि दौहित्रम्" इत्यनेन कथाश्चत् प्राप्तत्वा.\\
तू । अमुमेवार्थ सङ्ग्रहकार आह ।\\
~\\


{ }{वर्ज्यब्राह्मणनिरूपणम् \textbar{}}{ ८९\\
निश्चिताध्ययनेनैव गुणेन स्वीकृताखिलः ।\\
मूर्खो यो ब्रह्मचारी तु जटिलस्तं न भोजयेत् ॥\\
दौहित्रवदयं चापि भ्रान्त्या प्राप्तो निषिध्यते ।\\
ईदृशस्य गृहस्थस्य प्राप्त्यर्थमपरे विदुः }{\textbar{}\textbar{}}{\\
तन्न युक्तमविद्वांश्च गृहस्थश्व विरुध्यते ।\\
वस्तुतो जटिलो जटावान गृहस्थ एव बैखानसो }{वा, तथा}{ च\\
जटिलस्य पृथङ् निषेधः । दुर्वालः = कुत्सितकेशः खलतिर्वा । दुर्बल\\
मितिपाठे तु दुर्बलो विकोशध्वजः । पूगयाजकः = बहूनां याजकः }{\textbar{}}{
अत्र\\
श्राद्ध इत्युपादानात् यद्यपि पितृमात्रविषयत्व प्राप्यते}{ तथा}{ तद.\\
ङ्गभूतवैश्वदेविकोऽपि निषिध्यत एव, मन्द्रं प्रातः सवन इति वत् ।\\
कित्सकः = वृत्यर्थ धर्मार्थ वा चिकित्साकर्ता । तैत्तिरीयश्रुतौ
निषेधस्या.\\
विशेषेण श्रूयमाणत्वात् तस्माद् ब्राह्मणेन भेषजं न कार्यम् । "महतो\\
ह्येषोऽमेध्यो भिषक्" इति स्मृतिचन्द्रिकाकारः । तन्न साम्प्रतम् ।
निरूप-\\
धिक्रियमाणे तस्मिन् शुश्रुतादौ पुण्यश्रवणात् ।\\
देवलक उक्तो देवलेन-\/-\/-\\
देवार्चनपरो नित्यं वित्तार्थ वत्सरत्रयात् ।\\
असौ देवलको नाम हव्यकव्येषु गर्हितः }{\textbar{}\textbar{}}{\\
देवकोशोपजीवी च नाम्ना देवलको भवेत् ।\\
अपाङ्केयः स विज्ञेयः सर्वकर्मसु सर्वदा ॥ इति ।\\
विपणोपजीवी=अनापदि वणिग्वृत्तिः । तञ्च वाणिज्यं स्वयं कृत\\
बोध्यम् । कृषिवाणिज्ये चास्वयंकृते इति गौतमेन ब्राह्मणस्य मुख्य\\
वृत्तौ वाणिज्यस्याऽस्वयंकृतस्योपादानात् । मांसविक्रयश्च पृथगुपा\\
दानादस्वयंकृतोऽपि निषिद्ध एव । मांसपद चाविक्रेयोपलक्षणार्थम् ।\\
मेघातिथिस्तु मांसस्य पृथगुपादान विनिमयनिषेधार्थमित्याह ।\\
प्रेष्यः = आशाकरः ग्रामस्य राज्ञश्च । प्रत्येकं निषेधः । त्यक्ताग्निः=\\
बुद्धिपूर्व विच्छिन्नाग्निः ।\\
वार्धुषिकश्चोक्तो वशिष्ठेन -\/-\\
समर्धे धनमुद्धृत्य महर्षे यः प्रयच्छति ।\\
स }{वै}{ वार्धुषिको नाम ब्रह्मवादिषु गर्हितः }{\textbar{}\textbar{}}{\\
देवलस्तु,\\
विप्रं वार्धुषिकं विद्यादन्नवृद्ध्युपजीविनम् ।\\
१२ वी० मि०

{९० }{वीरमित्रोदयस्य श्राद्धप्रकाशे-}{\\
इत्याह ।\\
यक्ष्मी=क्षयी }{\textbar{}}{ पशुपाल: = जीवनार्थ यः पशून् पालयति सः ।\\
निराकृतिबोकः कात्यायनेन,\\
य आधायाग्निमालस्याद्देवादीनेभिरिष्टवान् ।\\
निराकर्ताऽमरादीनां स }{वै}{ शेयो निराकृतिः ॥ इति ।\\
अधीत्य विस्मृते वेदे भवेद् विप्रो निराकृतिः ।\\
इति देवलोक्को }{वा}{ ।\\
ब्रह्मद्विट् = वेदस्य ब्राह्मणानां च द्वेष्टा । परिवित्तिः = पूर्वोक्त:
। गणाभ्यतरः =\\
गणानां सङ्घानामेकवृत्युपजीविनां मध्ये तिष्ठतीत्यर्थः । कुशीलवो=नटः\\
अवकीर्णी चोक्तो देवलेन,\\
गूढलिङ्गयचकीर्णी स्याद्यश्च भगवतस्तथा }{\textbar{}}{\\
गूढलिङ्गी = ब्रह्मचारिलिङ्गत्यागी । भगवतः = स्त्रीसङ्गवान् ।
वृषलीपतयश्च\\
पूर्वमेवोक्ताः । पौनर्भवः = पुनर्भू: द्विः परिणता तस्याः पुरुषः । उपपतिः
=\\
जायाजार: स यस्य गृहे स इत्यर्थः । वेतनग्रहणपूर्वकमध्यापको भृत.\\
काध्यापकः । एवं भतकाध्यापितोऽपि । शूद्रशिष्यो= व्याकरणादिषु । गुरु.\\
श्च शूद्रस्यैव ।\\
वाग्दुष्टश्च कास्यायनेनोक्तः\\
हुङ्कारं चासनं चैष लोके यश्च विगर्हितम् ।\\
अनुकुर्यादनुब्रूयाद् वाग्दुष्टं तं}{ विवर्जयेत् \textbar{}\textbar{}}{\\
अभिशस्त इत्यन्ये, तन्न, अभिशस्त इति पृथगुपादानात् । कुण्ड\\
गोलकौ "अमृते जारजः कुण्डो मृते भर्तरि गोलक" इति पराशरोक्तौ ।\\
न चैतयोरब्राह्मणत्वेन प्राप्त्यभावात् कथ निषेध इति वाच्यम् । ब्रा.\\
ह्मण्यां ब्राह्मणोत्पन्नो ब्राह्मण इत्यपि
ब्राह्मणलक्षणस्योक्तस्तत्प्रस.\\
कस्यैव निषेधः । अथवा भ्रमप्राप्तस्य निषेध इति । गुरुः = पितृभिन्नः ।\\
एतेषां महापातकव्यतिरेकेण त्यागकर्ता । ब्राह्मर्यजनयाजनाध्यापनैः ।\\
पतितसंसर्गोऽर्वाक् संवत्सरात्, ऊर्ध्वन्तु तस्यापि पतितत्वात् ।\\
आगारदाही चोक्तो देवलेन-\/-\\
(१) आगारदाही स ज्ञेयः प्रेतं दग्धा ह्यनेकशः ।\\
स चाप्यागारदाही स्याद् द्वेषाद्यो वेश्मदाहकः ॥ इति ।

% \begin{center}\rule{0.5\linewidth}{0.5pt}\end{center}

{\\
( १ ) अगारदाहीत्यन्यत्र पाठ; ।

\end{document}