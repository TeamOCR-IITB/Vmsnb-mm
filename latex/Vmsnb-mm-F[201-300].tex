\documentclass[11pt, openany]{book}
\usepackage[text={4.65in,7.45in}, centering, includefoot]{geometry}
\usepackage[table, x11names]{xcolor}
\usepackage{fontspec,realscripts}
\usepackage{polyglossia}
\setdefaultlanguage{sanskrit}
\setotherlanguage{english}
\setmainfont[Scale=1]{Shobhika}
\newfontfamily\s[Script=Devanagari, Scale=0.9]{Shobhika}
\newfontfamily\regular{Linux Libertine O}
\newfontfamily\en[Language=English, Script=Latin]{Linux Libertine O}
\newfontfamily\ab[Script=Devanagari, Color=purple]{Shobhika-Bold}
\newfontfamily\qt[Script=Devanagari, Scale=1, Color=violet]{Shobhika-Regular}
\newcommand{\devanagarinumeral}[1]{
\devanagaridigits{\number \csname c@#1\endcsname}} % for devanagari page numbers
\XeTeXgenerateactualtext=1 % for searchable pdf
\usepackage{enumerate}
\pagestyle{plain}
\usepackage{fancyhdr}
\pagestyle{fancy}
\renewcommand{\headrulewidth}{0pt}
\usepackage{afterpage}
\usepackage{multirow}
\usepackage{multicol}
\usepackage{wrapfig}
\usepackage{vwcol}
\usepackage{microtype}
\usepackage{amsmath,amsthm, amsfonts,amssymb}
\usepackage{mathtools}% <-- new package for rcases
\usepackage{graphicx}
\usepackage{longtable}
\usepackage{setspace}
\usepackage{footnote}
\usepackage{perpage}
\MakePerPage{footnote}
\usepackage{xspace}
\usepackage{array}
\usepackage{emptypage}
\usepackage{hyperref}% Package for hyperlinks
\hypersetup{colorlinks, citecolor=black, filecolor=black, linkcolor=blue, urlcolor=black}
\begin{document}
?\textless{}???\textbar{}?s?t?y?l?e???\textbar{}?\textgreater{}?
???\textbar{}?b?o?d?y???\textbar{}?\{?
???\textbar{}?w?i?d?t?h???\textbar{}?:?
???\textbar{}?2?1?c?m???\textbar{}?;?
???\textbar{}?h?e?i?g?h?t???\textbar{}?:?
???\textbar{}?2?9???\textbar{}?.???\textbar{}?7?c?m???\textbar{}?;?
???\textbar{}?m?a?r?g?i?n???\textbar{}?:?
???\textbar{}?3?0?m?m???\textbar{}? ???\textbar{}?4?5?m?m???\textbar{}?
???\textbar{}?3?0?m?m???\textbar{}?
???\textbar{}?4?5?m?m???\textbar{}?;? ?\}?
?\textless{}?/???\textbar{}?s?t?y?l?e???\textbar{}?\textgreater{}?\textless{}?!???\textbar{}?D?O?C?T?Y?P?E???\textbar{}?
???\textbar{}?H?T?M?L???\textbar{}?
???\textbar{}?P?U?B?L?I?C???\textbar{}?
?"?-?/?/???\textbar{}?W?3?C???\textbar{}?/?/???\textbar{}?D?T?D???\textbar{}?
???\textbar{}?H?T?M?L???\textbar{}?
???\textbar{}?4???\textbar{}?.???\textbar{}?0???\textbar{}?/?/???\textbar{}?E?N???\textbar{}?"?
?"???\textbar{}?h?t?t?p???\textbar{}?:?/?/???\textbar{}?w?w?w???\textbar{}?.???\textbar{}?w?3???\textbar{}?.???\textbar{}?o?r?g???\textbar{}?/???\textbar{}?T?R???\textbar{}?/???\textbar{}?R?E?C???\textbar{}?-???\textbar{}?h?t?m?l?4?0???\textbar{}?/???\textbar{}?s?t?r?i?c?t???\textbar{}?.???\textbar{}?d?t?d???\textbar{}?"?\textgreater{}?
?\textless{}???\textbar{}?h?t?m?l???\textbar{}?\textgreater{}?\textless{}???\textbar{}?h?e?a?d???\textbar{}?\textgreater{}?\textless{}???\textbar{}?m?e?t?a???\textbar{}?
???\textbar{}?n?a?m?e???\textbar{}?=?"???\textbar{}?q?r?i?c?h?t?e?x?t???\textbar{}?"?
???\textbar{}?c?o?n?t?e?n?t???\textbar{}?=?"???\textbar{}?????\textbar{}?"?
?/?\textgreater{}?\textless{}???\textbar{}?s?t?y?l?e???\textbar{}?
???\textbar{}?t?y?p?e???\textbar{}?=?"???\textbar{}?t?e?x?t???\textbar{}?/???\textbar{}?c?s?s???\textbar{}?"?\textgreater{}?
???\textbar{}?p???\textbar{}?,? ???\textbar{}?l?i???\textbar{}? ?\{?
???\textbar{}?w?h?i?t?e???\textbar{}?-???\textbar{}?s?p?a?c?e???\textbar{}?:?
???\textbar{}?p?r?e???\textbar{}?-???\textbar{}?w?r?a?p???\textbar{}?;?
?\}?
?\textless{}?/???\textbar{}?s?t?y?l?e???\textbar{}?\textgreater{}?\textless{}?/???\textbar{}?h?e?a?d???\textbar{}?\textgreater{}?\textless{}???\textbar{}?b?o?d?y???\textbar{}?
???\textbar{}?s?t?y?l?e???\textbar{}?=?"?
???\textbar{}?f?o?n?t???\textbar{}?-???\textbar{}?f?a?m?i?l?y???\textbar{}?:?'???\textbar{}?S?h?o?b?h?i?k?a???\textbar{}?
???\textbar{}?R?e?g?u?l?a?r???\textbar{}?'?;?
???\textbar{}?f?o?n?t???\textbar{}?-???\textbar{}?s?i?z?e???\textbar{}?:???\textbar{}?1?6?p?t???\textbar{}?;?
???\textbar{}?f?o?n?t???\textbar{}?-???\textbar{}?w?e?i?g?h?t???\textbar{}?:???\textbar{}?4?0?0???\textbar{}?;?
???\textbar{}?f?o?n?t???\textbar{}?-???\textbar{}?s?t?y?l?e???\textbar{}?:???\textbar{}?n?o?r?m?a?l???\textbar{}?;?"?\textgreater{}?
?\textless{}???\textbar{}?p???\textbar{}?
???\textbar{}?s?t?y?l?e???\textbar{}?=?"?
???\textbar{}?m?a?r?g?i?n???\textbar{}?-???\textbar{}?t?o?p???\textbar{}?:???\textbar{}?0?p?x???\textbar{}?;?
???\textbar{}?m?a?r?g?i?n???\textbar{}?-???\textbar{}?b?o?t?t?o?m???\textbar{}?:???\textbar{}?0?p?x???\textbar{}?;?
???\textbar{}?m?a?r?g?i?n???\textbar{}?-???\textbar{}?l?e?f?t???\textbar{}?:???\textbar{}?0?p?x???\textbar{}?;?
???\textbar{}?m?a?r?g?i?n???\textbar{}?-???\textbar{}?r?i?g?h?t???\textbar{}?:???\textbar{}?0?p?x???\textbar{}?;?
?-???\textbar{}?q?t???\textbar{}?-???\textbar{}?b?l?o?c?k???\textbar{}?-???\textbar{}?i?n?d?e?n?t???\textbar{}?:???\textbar{}?0???\textbar{}?;?
???\textbar{}?t?e?x?t???\textbar{}?-???\textbar{}?i?n?d?e?n?t???\textbar{}?:???\textbar{}?0?p?x???\textbar{}?;?"?\textgreater{}?\textless{}???\textbar{}?s?p?a?n???\textbar{}?
???\textbar{}?s?t?y?l?e???\textbar{}?=?"?
???\textbar{}?f?o?n?t???\textbar{}?-???\textbar{}?f?a?m?i?l?y???\textbar{}?:?'???\textbar{}?M?S???\textbar{}?
???\textbar{}?S?h?e?l?l???\textbar{}? ???\textbar{}?D?l?g???\textbar{}?
???\textbar{}?2???\textbar{}?'?;?
???\textbar{}?f?o?n?t???\textbar{}?-???\textbar{}?s?i?z?e???\textbar{}?:???\textbar{}?1?1?p?t???\textbar{}?;?
???\textbar{}?f?o?n?t???\textbar{}?-???\textbar{}?w?e?i?g?h?t???\textbar{}?:???\textbar{}?6?0?0???\textbar{}?;?
???\textbar{}?c?o?l?o?r???\textbar{}?:?\#???\textbar{}?0?0?0?0?0?0???\textbar{}?;?
???\textbar{}?b?a?c?k?g?r?o?u?n?d???\textbar{}?-???\textbar{}?c?o?l?o?r???\textbar{}?:?\#???\textbar{}?f?f?f?f?f?f???\textbar{}?;?"?\textgreater{}???\textbar{}?à???\textbar{}?¥?§???\textbar{}?à???\textbar{}?¥?¯???\textbar{}?à???\textbar{}?¥?¨?
? ? ? ? ? ? ? ? ? ? ? ? ?
???\textbar{}?à???\textbar{}?¤???\textbar{}?µ?à???\textbar{}?¥?€???\textbar{}?à???\textbar{}?¤?°???\textbar{}?à???\textbar{}?¤?®???\textbar{}?à???\textbar{}?¤?¿???\textbar{}?à???\textbar{}?¤?¤???\textbar{}?à???\textbar{}?¥?????\textbar{}?à???\textbar{}?¤?°???\textbar{}?à???\textbar{}?¥?‹???\textbar{}?à???\textbar{}?¤?¦???\textbar{}?à???\textbar{}?¤?¯???\textbar{}?à???\textbar{}?¤?¸???\textbar{}?à???\textbar{}?¥?????\textbar{}?à???\textbar{}?¤?¯?
???\textbar{}?à???\textbar{}?¤?¶???\textbar{}?à???\textbar{}?¥?????\textbar{}?à???\textbar{}?¤?°???\textbar{}?à???\textbar{}?¤???\textbar{}?¾?à???\textbar{}?¤?¦???\textbar{}?à???\textbar{}?¥?????\textbar{}?à???\textbar{}?¤?§???\textbar{}?à???\textbar{}?¤???\textbar{}?ª?à???\textbar{}?¥?????\textbar{}?à???\textbar{}?¤?°???\textbar{}?à???\textbar{}?¤?•???\textbar{}?à???\textbar{}?¤???\textbar{}?¾?à???\textbar{}?¤?¶???\textbar{}?à???\textbar{}?¥?‡?-?\textless{}?/???\textbar{}?s?p?a?n???\textbar{}?\textgreater{}?\textless{}???\textbar{}?s?p?a?n???\textbar{}?
???\textbar{}?s?t?y?l?e???\textbar{}?=?"?
???\textbar{}?f?o?n?t???\textbar{}?-???\textbar{}?f?a?m?i?l?y???\textbar{}?:?'???\textbar{}?M?S???\textbar{}?
???\textbar{}?S?h?e?l?l???\textbar{}? ???\textbar{}?D?l?g???\textbar{}?
???\textbar{}?2???\textbar{}?'?;?
???\textbar{}?f?o?n?t???\textbar{}?-???\textbar{}?s?i?z?e???\textbar{}?:???\textbar{}?1?1?p?t???\textbar{}?;?
???\textbar{}?c?o?l?o?r???\textbar{}?:?\#???\textbar{}?0?0?0?0?0?0???\textbar{}?;?
???\textbar{}?b?a?c?k?g?r?o?u?n?d???\textbar{}?-???\textbar{}?c?o?l?o?r???\textbar{}?:?\#???\textbar{}?f?f?f?f?f?f???\textbar{}?;?"?\textgreater{}?\textless{}???\textbar{}?b?r???\textbar{}?
?/?\textgreater{}???\textbar{}?à???\textbar{}?¤???\textbar{}?œ?à???\textbar{}?¤???\textbar{}?ª?à???\textbar{}?¤???\textbar{}?¾?à???\textbar{}?¤?°???\textbar{}?à???\textbar{}?¥?????\textbar{}?à???\textbar{}?¤?¥???\textbar{}?à???\textbar{}?¤?¤???\textbar{}?à???\textbar{}?¥?????\textbar{}?à???\textbar{}?¤???\textbar{}?µ?à???\textbar{}?¥?‡?
???\textbar{}?à???\textbar{}?¤?¤???\textbar{}?à???\textbar{}?¥???
???\textbar{}?à???\textbar{}?¤?­???\textbar{}?à???\textbar{}?¤?¿???\textbar{}?à???\textbar{}?¤?¨???\textbar{}?à???\textbar{}?¥?????\textbar{}?à???\textbar{}?¤?¨???\textbar{}?à???\textbar{}?¤???\textbar{}?¾?à???\textbar{}?¤?°???\textbar{}?à???\textbar{}?¥?????\textbar{}?à???\textbar{}?¤?¥???\textbar{}?à???\textbar{}?¤?¤???\textbar{}?à???\textbar{}?¥?????\textbar{}?à???\textbar{}?¤???\textbar{}?µ?à???\textbar{}?¤???\textbar{}?¾?à???\textbar{}?¤?¦???\textbar{}?à???\textbar{}?¥?????\textbar{}?à???\textbar{}?¤?¯???\textbar{}?à???\textbar{}?¥?????\textbar{}?à???\textbar{}?¤?•???\textbar{}?à???\textbar{}?¥?????\textbar{}?à???\textbar{}?¤?¤?
???\textbar{}?à???\textbar{}?¤?????\textbar{}?à???\textbar{}?¤???\textbar{}?µ???\textbar{}?
???\textbar{}?à???\textbar{}?¤?¸???\textbar{}?à???\textbar{}?¤?®???\textbar{}?à???\textbar{}?¥?????\textbar{}?à???\textbar{}?¤???\textbar{}?š?à???\textbar{}?¥?????\textbar{}?à???\textbar{}?¤???\textbar{}?š?à???\textbar{}?¤?¯???\textbar{}?à???\textbar{}?¤???\textbar{}?ƒ???\textbar{}?
???\textbar{}?à???\textbar{}?¥?¤?
???\textbar{}?à???\textbar{}?¤?‰???\textbar{}?à???\textbar{}?¤???\textbar{}?ª?à???\textbar{}?¤???\textbar{}?µ?à???\textbar{}?¤?¿???\textbar{}?à???\textbar{}?¤?·???\textbar{}?à???\textbar{}?¥?????\textbar{}?à???\textbar{}?¤???\textbar{}?Ÿ?à???\textbar{}?¤?¬???\textbar{}?à???\textbar{}?¥?????\textbar{}?à???\textbar{}?¤?°???\textbar{}?à???\textbar{}?¤???\textbar{}?¾?à???\textbar{}?¤???\textbar{}?¹?à???\textbar{}?¥?????\textbar{}?à???\textbar{}?¤?®???\textbar{}?à???\textbar{}?¤?£???\textbar{}?à???\textbar{}?¤?¨???\textbar{}?à???\textbar{}?¤?¿?\textless{}???\textbar{}?b?r???\textbar{}?
?/?\textgreater{}???\textbar{}?à???\textbar{}?¤?¯???\textbar{}?à???\textbar{}?¤?®???\textbar{}?à???\textbar{}?¤???\textbar{}?¾?à???\textbar{}?¤?¨???\textbar{}?à???\textbar{}?¤???\textbar{}?¾?à???\textbar{}?¤???\textbar{}?¹???\textbar{}?
???\textbar{}?à???\textbar{}?¥?¤?\textless{}???\textbar{}?b?r???\textbar{}?
?/?\textgreater{}???\textbar{}?à???\textbar{}?¤?¸???\textbar{}?à???\textbar{}?¥?????\textbar{}?à???\textbar{}?¤?®???\textbar{}?à???\textbar{}?¤?¨???\textbar{}?à???\textbar{}?¥?????\textbar{}?à???\textbar{}?¤?¤???\textbar{}?à???\textbar{}?¤???\textbar{}?ƒ???\textbar{}?
???\textbar{}?à???\textbar{}?¥?¤?\textless{}???\textbar{}?b?r???\textbar{}?
?/?\textgreater{}???\textbar{}?à???\textbar{}?¤???\textbar{}?ª?à???\textbar{}?¤???\textbar{}?µ?à???\textbar{}?¤?¿???\textbar{}?à???\textbar{}?¤?¤???\textbar{}?à???\textbar{}?¥?????\textbar{}?à???\textbar{}?¤?°???\textbar{}?à???\textbar{}?¤???\textbar{}?ª?à???\textbar{}?¤???\textbar{}?¾?à???\textbar{}?¤?£???\textbar{}?à???\textbar{}?¤?¯???\textbar{}?à???\textbar{}?¤???\textbar{}?ƒ???\textbar{}?
???\textbar{}?à???\textbar{}?¤?¸???\textbar{}?à???\textbar{}?¤?°???\textbar{}?à???\textbar{}?¥?????\textbar{}?à???\textbar{}?¤???\textbar{}?µ?à???\textbar{}?¥?‡?
???\textbar{}?à???\textbar{}?¤?¤???\textbar{}?à???\textbar{}?¥?‡?
???\textbar{}?à???\textbar{}?¤???\textbar{}?š???\textbar{}?
???\textbar{}?à???\textbar{}?¤?®???\textbar{}?à???\textbar{}?¥???\textbar{}?Œ?à???\textbar{}?¤?¨???\textbar{}?à???\textbar{}?¤???\textbar{}?µ?à???\textbar{}?¥?????\textbar{}?à???\textbar{}?¤?°???\textbar{}?à???\textbar{}?¤?¤???\textbar{}?à???\textbar{}?¤???\textbar{}?¾?à???\textbar{}?¤?¨???\textbar{}?à???\textbar{}?¥?????\textbar{}?à???\textbar{}?¤???\textbar{}?µ?à???\textbar{}?¤?¿???\textbar{}?à???\textbar{}?¤?¤???\textbar{}?à???\textbar{}?¤???\textbar{}?¾???\textbar{}?
???\textbar{}?à???\textbar{}?¥?¤?\textless{}???\textbar{}?b?r???\textbar{}?
?/?\textgreater{}???\textbar{}?à???\textbar{}?¤?‰???\textbar{}?à???\textbar{}?¤???\textbar{}?š?à???\textbar{}?¥?????\textbar{}?à???\textbar{}?¤?›???\textbar{}?à???\textbar{}?¤?¿???\textbar{}?à???\textbar{}?¤?·???\textbar{}?à???\textbar{}?¥?????\textbar{}?à???\textbar{}?¤???\textbar{}?Ÿ?à???\textbar{}?¥?‹???\textbar{}?à???\textbar{}?¤???\textbar{}?š?à???\textbar{}?¥?????\textbar{}?à???\textbar{}?¤?›???\textbar{}?à???\textbar{}?¤?¿???\textbar{}?à???\textbar{}?¤?·???\textbar{}?à???\textbar{}?¥?????\textbar{}?à???\textbar{}?¤???\textbar{}?Ÿ?à???\textbar{}?¤?¸???\textbar{}?à???\textbar{}?¤?‚???\textbar{}?à???\textbar{}?¤?¸???\textbar{}?à???\textbar{}?¥?????\textbar{}?à???\textbar{}?¤???\textbar{}?ª?à???\textbar{}?¤?°???\textbar{}?à???\textbar{}?¥?????\textbar{}?à???\textbar{}?¤?¶?
???\textbar{}?à???\textbar{}?¤???\textbar{}?µ?à???\textbar{}?¤?°???\textbar{}?à???\textbar{}?¥?????\textbar{}?à???\textbar{}?¤???\textbar{}?œ?à???\textbar{}?¤?¯???\textbar{}?à???\textbar{}?¤?¨???\textbar{}?à???\textbar{}?¥?????\textbar{}?à???\textbar{}?¤?¤???\textbar{}?à???\textbar{}?¤???\textbar{}?ƒ???\textbar{}?
???\textbar{}?à???\textbar{}?¤???\textbar{}?ª?à???\textbar{}?¤?°???\textbar{}?à???\textbar{}?¤?¸???\textbar{}?à???\textbar{}?¥?????\textbar{}?à???\textbar{}?¤???\textbar{}?ª?à???\textbar{}?¤?°???\textbar{}?à???\textbar{}?¤?®???\textbar{}?à???\textbar{}?¥???
???\textbar{}?à???\textbar{}?¥?¥?
???\textbar{}?à???\textbar{}?¤?‡???\textbar{}?à???\textbar{}?¤?¤???\textbar{}?à???\textbar{}?¤?¿?
???\textbar{}?à???\textbar{}?¥?¤?\textless{}???\textbar{}?b?r???\textbar{}?
?/?\textgreater{}???\textbar{}?à???\textbar{}?¤?\ldots{}???\textbar{}?à???\textbar{}?¤?¤???\textbar{}?à???\textbar{}?¥?????\textbar{}?à???\textbar{}?¤?°?
???\textbar{}?à???\textbar{}?¤?®???\textbar{}?à???\textbar{}?¥???\textbar{}?Œ?à???\textbar{}?¤?¨???\textbar{}?à???\textbar{}?¤?¿???\textbar{}?à???\textbar{}?¤?¤???\textbar{}?à???\textbar{}?¥?????\textbar{}?à???\textbar{}?¤???\textbar{}?µ???\textbar{}?
???\textbar{}?à???\textbar{}?¤???\textbar{}?š???\textbar{}?
???\textbar{}?à???\textbar{}?¤?¬???\textbar{}?à???\textbar{}?¥?????\textbar{}?à???\textbar{}?¤?°???\textbar{}?à???\textbar{}?¤???\textbar{}?¹?à???\textbar{}?¥?????\textbar{}?à???\textbar{}?¤?®???\textbar{}?à???\textbar{}?¥?‹???\textbar{}?à???\textbar{}?¤?¦???\textbar{}?à???\textbar{}?¥?????\textbar{}?à???\textbar{}?¤?¯???\textbar{}?à???\textbar{}?¤?•???\textbar{}?à???\textbar{}?¤?¥???\textbar{}?à???\textbar{}?¤???\textbar{}?¾?à???\textbar{}?¤???\textbar{}?µ?à???\textbar{}?¥?????\textbar{}?à???\textbar{}?¤?¯???\textbar{}?à???\textbar{}?¤?¤???\textbar{}?à???\textbar{}?¤?¿???\textbar{}?à???\textbar{}?¤?°???\textbar{}?à???\textbar{}?¤?¿???\textbar{}?à???\textbar{}?¤?•???\textbar{}?à???\textbar{}?¥?????\textbar{}?à???\textbar{}?¤?¤???\textbar{}?à???\textbar{}?¤???\textbar{}?µ?à???\textbar{}?¤?¿???\textbar{}?à???\textbar{}?¤?·???\textbar{}?à???\textbar{}?¤?¯???\textbar{}?à???\textbar{}?¤?®???\textbar{}?à???\textbar{}?¥???
???\textbar{}?à???\textbar{}?¥?¤?
???\textbar{}?à???\textbar{}?¤?¤???\textbar{}?à???\textbar{}?¤?¥???\textbar{}?à???\textbar{}?¤???\textbar{}?¾???\textbar{}?
???\textbar{}?à???\textbar{}?¤???\textbar{}?š???\textbar{}?\textless{}???\textbar{}?b?r???\textbar{}?
?/?\textgreater{}???\textbar{}?à???\textbar{}?¤?¯???\textbar{}?à???\textbar{}?¤?®?
???\textbar{}?à???\textbar{}?¥?¤?
???\textbar{}?à???\textbar{}?¤?¬???\textbar{}?à???\textbar{}?¥?????\textbar{}?à???\textbar{}?¤?°???\textbar{}?à???\textbar{}?¤???\textbar{}?¹?à???\textbar{}?¥?????\textbar{}?à???\textbar{}?¤?®???\textbar{}?à???\textbar{}?¥?‹???\textbar{}?à???\textbar{}?¤?¦???\textbar{}?à???\textbar{}?¥?????\textbar{}?à???\textbar{}?¤?¯???\textbar{}?à???\textbar{}?¤???\textbar{}?¾?à???\textbar{}?¤?¶???\textbar{}?à???\textbar{}?¥?????\textbar{}?à???\textbar{}?¤???\textbar{}?š???\textbar{}?
?
???\textbar{}?à???\textbar{}?¤?•???\textbar{}?à???\textbar{}?¤?¥???\textbar{}?à???\textbar{}?¤???\textbar{}?¾???\textbar{}?:?
???\textbar{}?à???\textbar{}?¤?•???\textbar{}?à???\textbar{}?¥?????\textbar{}?à???\textbar{}?¤?°???\textbar{}?à???\textbar{}?¥?????\textbar{}?à???\textbar{}?¤?¯???\textbar{}?à???\textbar{}?¥?????\textbar{}?à???\textbar{}?¤???\textbar{}?ƒ???\textbar{}?
???\textbar{}?à???\textbar{}?¤???\textbar{}?ª?à???\textbar{}?¤?¿???\textbar{}?à???\textbar{}?¤?¤???\textbar{}?à???\textbar{}?¥???\textbar{}?ƒ?à???\textbar{}?¤?£???\textbar{}?à???\textbar{}?¤???\textbar{}?¾?à???\textbar{}?¤?®???\textbar{}?à???\textbar{}?¥?‡???\textbar{}?à???\textbar{}?¤?¤???\textbar{}?à???\textbar{}?¤???\textbar{}?¾?à???\textbar{}?¤?¦???\textbar{}?à???\textbar{}?¥?€???\textbar{}?à???\textbar{}?¤???\textbar{}?ª?à???\textbar{}?¥?????\textbar{}?à???\textbar{}?¤?¸???\textbar{}?à???\textbar{}?¤?¿???\textbar{}?à???\textbar{}?¤?¤???\textbar{}?à???\textbar{}?¤?®???\textbar{}?à???\textbar{}?¥???
???\textbar{}?à???\textbar{}?¥?¤?
???\textbar{}?à???\textbar{}?¤?‡???\textbar{}?à???\textbar{}?¤?¤???\textbar{}?à???\textbar{}?¤?¿?
???\textbar{}?à???\textbar{}?¥?¤?
???\textbar{}?à???\textbar{}?¤?‡???\textbar{}?à???\textbar{}?¤?¤???\textbar{}?à???\textbar{}?¤?¿?\textless{}???\textbar{}?b?r???\textbar{}?
?/?\textgreater{}???\textbar{}?à???\textbar{}?¤?¬???\textbar{}?à???\textbar{}?¥?????\textbar{}?à???\textbar{}?¤?°???\textbar{}?à???\textbar{}?¤???\textbar{}?¾?à???\textbar{}?¤???\textbar{}?¹?à???\textbar{}?¥?????\textbar{}?à???\textbar{}?¤?®???\textbar{}?à???\textbar{}?¤?£???\textbar{}?à???\textbar{}?¥?‹???\textbar{}?à???\textbar{}?¤???\textbar{}?ª?à???\textbar{}?¤???\textbar{}?µ?à???\textbar{}?¥?‡???\textbar{}?à???\textbar{}?¤?¶???\textbar{}?à???\textbar{}?¤?¨???\textbar{}?à???\textbar{}?¤?®???\textbar{}?à???\textbar{}?¥???
???\textbar{}?à???\textbar{}?¥?¤?\textless{}???\textbar{}?b?r???\textbar{}?
?/?\textgreater{}???\textbar{}?à???\textbar{}?¤?\ldots{}???\textbar{}?à???\textbar{}?¤?¥?
???\textbar{}?à???\textbar{}?¤???\textbar{}?ª?à???\textbar{}?¥?????\textbar{}?à???\textbar{}?¤?£???\textbar{}?à???\textbar{}?¥?????\textbar{}?à???\textbar{}?¤?¡???\textbar{}?à???\textbar{}?¤?°???\textbar{}?à???\textbar{}?¥?€???\textbar{}?à???\textbar{}?¤?•???\textbar{}?à???\textbar{}?¤???\textbar{}?¾?à???\textbar{}?¤?•???\textbar{}?à???\textbar{}?¥?????\textbar{}?à???\textbar{}?¤?·???\textbar{}?à???\textbar{}?¤?¸???\textbar{}?à???\textbar{}?¥?????\textbar{}?à???\textbar{}?¤?®???\textbar{}?à???\textbar{}?¤?°???\textbar{}?à???\textbar{}?¤?£???\textbar{}?à???\textbar{}?¤???\textbar{}?¾?à???\textbar{}?¤?¦???\textbar{}?à???\textbar{}?¤?¿???\textbar{}?à???\textbar{}?¤?•???\textbar{}?à???\textbar{}?¥???\textbar{}?ƒ?à???\textbar{}?¤?¤???\textbar{}?à???\textbar{}?¥?????\textbar{}?à???\textbar{}?¤?¯???\textbar{}?à???\textbar{}?¤?®???\textbar{}?à???\textbar{}?¥???
???\textbar{}?à???\textbar{}?¥?¤?\textless{}???\textbar{}?b?r???\textbar{}?
?/?\textgreater{}???\textbar{}?à???\textbar{}?¤?¤???\textbar{}?à???\textbar{}?¤?¤???\textbar{}?à???\textbar{}?¥?????\textbar{}?à???\textbar{}?¤?°???\textbar{}?à???\textbar{}?¤???\textbar{}?¾?à???\textbar{}?¤?¸???\textbar{}?à???\textbar{}?¤?¨???\textbar{}?à???\textbar{}?¥?‹???\textbar{}?à???\textbar{}?¤???\textbar{}?ª?à???\textbar{}?¤???\textbar{}?µ?à???\textbar{}?¤?¿???\textbar{}?à???\textbar{}?¤?·???\textbar{}?à???\textbar{}?¥?????\textbar{}?à???\textbar{}?¤???\textbar{}?Ÿ?à???\textbar{}?¤?¬???\textbar{}?à???\textbar{}?¥?????\textbar{}?à???\textbar{}?¤?°???\textbar{}?à???\textbar{}?¤???\textbar{}?¾?à???\textbar{}?¤???\textbar{}?¹?à???\textbar{}?¥?????\textbar{}?à???\textbar{}?¤?®???\textbar{}?à???\textbar{}?¤?£???\textbar{}?à???\textbar{}?¤???\textbar{}?¾?à???\textbar{}?¤?¨???\textbar{}?à???\textbar{}?¤???\textbar{}?¾?à???\textbar{}?¤?®???\textbar{}?à???\textbar{}?¤?---???\textbar{}?à???\textbar{}?¥?????\textbar{}?à???\textbar{}?¤?°???\textbar{}?à???\textbar{}?¤?¤???\textbar{}?à???\textbar{}?¤???\textbar{}?ƒ???\textbar{}?
???\textbar{}?à???\textbar{}?¤?•???\textbar{}?à???\textbar{}?¥?????\textbar{}?à???\textbar{}?¤?¶???\textbar{}?à???\textbar{}?¥???\textbar{}?ˆ?à???\textbar{}?¤?°???\textbar{}?à???\textbar{}?¤???\textbar{}?¾?à???\textbar{}?¤?¸???\textbar{}?à???\textbar{}?¥?????\textbar{}?à???\textbar{}?¤?¤???\textbar{}?à???\textbar{}?¥???\textbar{}?ƒ?à???\textbar{}?¤?¤???\textbar{}?à???\textbar{}?¤???\textbar{}?¾?à???\textbar{}?¤?¯???\textbar{}?à???\textbar{}?¤???\textbar{}?¾?à???\textbar{}?¤?‚?
???\textbar{}?à???\textbar{}?¤?­???\textbar{}?à???\textbar{}?¥?????\textbar{}?à???\textbar{}?¤???\textbar{}?µ?à???\textbar{}?¤?¿?
???\textbar{}?à???\textbar{}?¤?¯???\textbar{}?à???\textbar{}?¤???\textbar{}?œ?à???\textbar{}?¥?????\textbar{}?à???\textbar{}?¤???\textbar{}?ž?à???\textbar{}?¥?‹???\textbar{}?à???\textbar{}?¤???\textbar{}?ª???\textbar{}?-?\textless{}???\textbar{}?b?r???\textbar{}?
?/?\textgreater{}???\textbar{}?à???\textbar{}?¤???\textbar{}?µ?à???\textbar{}?¥?€???\textbar{}?à???\textbar{}?¤?¤???\textbar{}?à???\textbar{}?¥?€?
???\textbar{}?à???\textbar{}?¤???\textbar{}?ª?à???\textbar{}?¥?????\textbar{}?à???\textbar{}?¤?°???\textbar{}?à???\textbar{}?¤???\textbar{}?¾?à???\textbar{}?¤?™???\textbar{}?à???\textbar{}?¥?????\textbar{}?à???\textbar{}?¤?®???\textbar{}?à???\textbar{}?¥?????\textbar{}?à???\textbar{}?¤?--?
???\textbar{}?à???\textbar{}?¤?‰???\textbar{}?à???\textbar{}?¤???\textbar{}?ª?à???\textbar{}?¤???\textbar{}?µ?à???\textbar{}?¤?¿???\textbar{}?à???\textbar{}?¤?¶???\textbar{}?à???\textbar{}?¥?????\textbar{}?à???\textbar{}?¤?¯???\textbar{}?à???\textbar{}?¤???\textbar{}?¾?à???\textbar{}?¤???\textbar{}?µ?à???\textbar{}?¤?¿???\textbar{}?à???\textbar{}?¤???\textbar{}?œ?à???\textbar{}?¥?????\textbar{}?à???\textbar{}?¤???\textbar{}?ž?à???\textbar{}?¤???\textbar{}?¾?à???\textbar{}?¤?¤???\textbar{}?à???\textbar{}?¤???\textbar{}?¾?à???\textbar{}?¤?¶???\textbar{}?à???\textbar{}?¥?????\textbar{}?à???\textbar{}?¤???\textbar{}?š?à???\textbar{}?¤?¿???\textbar{}?à???\textbar{}?¤?¤???\textbar{}?à???\textbar{}?¥?????\textbar{}?à???\textbar{}?¤???\textbar{}?µ?à???\textbar{}?¤?¨???\textbar{}?à???\textbar{}?¤?¿???\textbar{}?à???\textbar{}?¤???\textbar{}?µ?à???\textbar{}?¥???\textbar{}?ƒ?à???\textbar{}?¤?¤???\textbar{}?à???\textbar{}?¥?????\textbar{}?à???\textbar{}?¤?¯???\textbar{}?à???\textbar{}?¤?°???\textbar{}?à???\textbar{}?¥?????\textbar{}?à???\textbar{}?¤?¥???\textbar{}?à???\textbar{}?¤?‚?
???\textbar{}?à???\textbar{}?¤???\textbar{}?ª?à???\textbar{}?¥?????\textbar{}?à???\textbar{}?¤?£???\textbar{}?à???\textbar{}?¥?????\textbar{}?à???\textbar{}?¤?¡???\textbar{}?à???\textbar{}?¤?°???\textbar{}?à???\textbar{}?¥?€???\textbar{}?à???\textbar{}?¤?•???\textbar{}?à???\textbar{}?¤???\textbar{}?¾?à???\textbar{}?¤?•???\textbar{}?à???\textbar{}?¥?????\textbar{}?à???\textbar{}?¤?·???\textbar{}?à???\textbar{}?¤?¸???\textbar{}?à???\textbar{}?¥?????\textbar{}?à???\textbar{}?¤?®?-?\textless{}???\textbar{}?b?r???\textbar{}?
?/?\textgreater{}???\textbar{}?à???\textbar{}?¤?°???\textbar{}?à???\textbar{}?¤?£???\textbar{}?à???\textbar{}?¤?‚?
???\textbar{}?à???\textbar{}?¤?•???\textbar{}?à???\textbar{}?¥?????\textbar{}?à???\textbar{}?¤?°???\textbar{}?à???\textbar{}?¥?????\textbar{}?à???\textbar{}?¤?¯???\textbar{}?à???\textbar{}?¤???\textbar{}?¾?à???\textbar{}?¤?¤???\textbar{}?à???\textbar{}?¥???
???\textbar{}?à???\textbar{}?¥?¤?\textless{}???\textbar{}?b?r???\textbar{}?
?/?\textgreater{}???\textbar{}?à???\textbar{}?¤?¤???\textbar{}?à???\textbar{}?¤?¥???\textbar{}?à???\textbar{}?¤???\textbar{}?¾???\textbar{}?
???\textbar{}?à???\textbar{}?¤???\textbar{}?š?à???\textbar{}?¤???\textbar{}?¾?à???\textbar{}?¤???\textbar{}?¹???\textbar{}?
???\textbar{}?à???\textbar{}?¤?•???\textbar{}?à???\textbar{}?¥?????\textbar{}?à???\textbar{}?¤?°???\textbar{}?à???\textbar{}?¤?¤???\textbar{}?à???\textbar{}?¥?????\textbar{}?à???\textbar{}?¤???\textbar{}?ƒ???\textbar{}?
?-?-?\textless{}???\textbar{}?b?r???\textbar{}?
?/?\textgreater{}???\textbar{}?à???\textbar{}?¤?\ldots{}???\textbar{}?à???\textbar{}?¤???\textbar{}?ª?à???\textbar{}?¤???\textbar{}?µ?à???\textbar{}?¤?¿???\textbar{}?à???\textbar{}?¤?¤???\textbar{}?à???\textbar{}?¥?????\textbar{}?à???\textbar{}?¤?°???\textbar{}?à???\textbar{}?¤???\textbar{}?ƒ???\textbar{}?
???\textbar{}?à???\textbar{}?¤???\textbar{}?ª?à???\textbar{}?¤???\textbar{}?µ?à???\textbar{}?¤?¿???\textbar{}?à???\textbar{}?¤?¤???\textbar{}?à???\textbar{}?¥?????\textbar{}?à???\textbar{}?¤?°???\textbar{}?à???\textbar{}?¥?‹?
???\textbar{}?à???\textbar{}?¤???\textbar{}?µ?à???\textbar{}?¤???\textbar{}?¾???\textbar{}?
???\textbar{}?à???\textbar{}?¤?¸???\textbar{}?à???\textbar{}?¤?°???\textbar{}?à???\textbar{}?¥?????\textbar{}?à???\textbar{}?¤???\textbar{}?µ?à???\textbar{}?¤???\textbar{}?¾?à???\textbar{}?¤???\textbar{}?µ?à???\textbar{}?¤?¸???\textbar{}?à???\textbar{}?¥?????\textbar{}?à???\textbar{}?¤?¥???\textbar{}?à???\textbar{}?¤???\textbar{}?¾???\textbar{}?
???\textbar{}?à???\textbar{}?¤?---???\textbar{}?à???\textbar{}?¤?¤???\textbar{}?à???\textbar{}?¥?‹???\textbar{}?à???\textbar{}?¤???\textbar{}?½?à???\textbar{}?¤???\textbar{}?ª?à???\textbar{}?¤?¿?
???\textbar{}?à???\textbar{}?¤???\textbar{}?µ?à???\textbar{}?¤???\textbar{}?¾???\textbar{}?
???\textbar{}?à???\textbar{}?¥?¤?\textless{}???\textbar{}?b?r???\textbar{}?
?/?\textgreater{}???\textbar{}?à???\textbar{}?¤?¯???\textbar{}?à???\textbar{}?¤???\textbar{}?ƒ???\textbar{}?
???\textbar{}?à???\textbar{}?¤?¸???\textbar{}?à???\textbar{}?¥?????\textbar{}?à???\textbar{}?¤?®???\textbar{}?à???\textbar{}?¤?°???\textbar{}?à???\textbar{}?¥?‡???\textbar{}?à???\textbar{}?¤?¤???\textbar{}?à???\textbar{}?¥???
???\textbar{}?à???\textbar{}?¤???\textbar{}?ª?à???\textbar{}?¥?????\textbar{}?à???\textbar{}?¤?£???\textbar{}?à???\textbar{}?¥?????\textbar{}?à???\textbar{}?¤?¡???\textbar{}?à???\textbar{}?¤?°???\textbar{}?à???\textbar{}?¥?€???\textbar{}?à???\textbar{}?¤?•???\textbar{}?à???\textbar{}?¤???\textbar{}?¾?à???\textbar{}?¤?•???\textbar{}?à???\textbar{}?¥?????\textbar{}?à???\textbar{}?¤?·?
???\textbar{}?à???\textbar{}?¤?¸?
???\textbar{}?à???\textbar{}?¤?¬???\textbar{}?à???\textbar{}?¤???\textbar{}?¾?à???\textbar{}?¤???\textbar{}?¹?à???\textbar{}?¥?????\textbar{}?à???\textbar{}?¤?¯???\textbar{}?à???\textbar{}?¤???\textbar{}?¾?à???\textbar{}?¤?­???\textbar{}?à???\textbar{}?¥?????\textbar{}?à???\textbar{}?¤?¯???\textbar{}?à???\textbar{}?¤?¨???\textbar{}?à???\textbar{}?¥?????\textbar{}?à???\textbar{}?¤?¤???\textbar{}?à???\textbar{}?¤?°???\textbar{}?à???\textbar{}?¤???\textbar{}?ƒ???\textbar{}?
???\textbar{}?à???\textbar{}?¤?¶???\textbar{}?à???\textbar{}?¥?????\textbar{}?à???\textbar{}?¤???\textbar{}?š?à???\textbar{}?¤?¿???\textbar{}?à???\textbar{}?¤???\textbar{}?ƒ???\textbar{}?
???\textbar{}?à???\textbar{}?¥?¥?
???\textbar{}?à???\textbar{}?¤?‡???\textbar{}?à???\textbar{}?¤?¤???\textbar{}?à???\textbar{}?¤?¿?
???\textbar{}?à???\textbar{}?¥?¤?\textless{}???\textbar{}?b?r???\textbar{}?
?/?\textgreater{}???\textbar{}?à???\textbar{}?¤?????\textbar{}?à???\textbar{}?¤???\textbar{}?µ?à???\textbar{}?¤?‚?
???\textbar{}?à???\textbar{}?¤???\textbar{}?ª?à???\textbar{}?¥?????\textbar{}?à???\textbar{}?¤?£???\textbar{}?à???\textbar{}?¥?????\textbar{}?à???\textbar{}?¤?¡???\textbar{}?à???\textbar{}?¤?°???\textbar{}?à???\textbar{}?¥?€???\textbar{}?à???\textbar{}?¤?•???\textbar{}?à???\textbar{}?¤???\textbar{}?¾?à???\textbar{}?¤?•???\textbar{}?à???\textbar{}?¥?????\textbar{}?à???\textbar{}?¤?·???\textbar{}?à???\textbar{}?¤?¸???\textbar{}?à???\textbar{}?¥?????\textbar{}?à???\textbar{}?¤?®???\textbar{}?à???\textbar{}?¤?°???\textbar{}?à???\textbar{}?¤?£?
???\textbar{}?à???\textbar{}?¤?•???\textbar{}?à???\textbar{}?¥???\textbar{}?ƒ?à???\textbar{}?¤?¤???\textbar{}?à???\textbar{}?¥?????\textbar{}?à???\textbar{}?¤???\textbar{}?µ?à???\textbar{}?¤???\textbar{}?¾???\textbar{}?
???\textbar{}?à???\textbar{}?¤???\textbar{}?ª?à???\textbar{}?¥???\textbar{}?ƒ?à???\textbar{}?¤?¥???\textbar{}?à???\textbar{}?¤?¿???\textbar{}?à???\textbar{}?¤???\textbar{}?µ?à???\textbar{}?¥?€???\textbar{}?à???\textbar{}?¤?¸???\textbar{}?à???\textbar{}?¥?????\textbar{}?à???\textbar{}?¤?¤???\textbar{}?à???\textbar{}?¥?????\textbar{}?à???\textbar{}?¤?¤???\textbar{}?à???\textbar{}?¤?¿???\textbar{}?à???\textbar{}?¤?‚?
???\textbar{}?à???\textbar{}?¤?•???\textbar{}?à???\textbar{}?¥?????\textbar{}?à???\textbar{}?¤?°???\textbar{}?à???\textbar{}?¥?????\textbar{}?à???\textbar{}?¤?¯???\textbar{}?à???\textbar{}?¤???\textbar{}?¾?à???\textbar{}?¤?¤???\textbar{}?à???\textbar{}?¥???
???\textbar{}?à???\textbar{}?¤?¤???\textbar{}?à???\textbar{}?¤?¥???\textbar{}?à???\textbar{}?¤???\textbar{}?¾???\textbar{}?
???\textbar{}?à???\textbar{}?¤???\textbar{}?š???\textbar{}?\textless{}???\textbar{}?b?r???\textbar{}?
?/?\textgreater{}???\textbar{}?à???\textbar{}?¤?®???\textbar{}?à???\textbar{}?¤???\textbar{}?¹?à???\textbar{}?¤???\textbar{}?¾?à???\textbar{}?¤?­???\textbar{}?à???\textbar{}?¤???\textbar{}?¾?à???\textbar{}?¤?°???\textbar{}?à???\textbar{}?¤?¤???\textbar{}?à???\textbar{}?¤?‚?
???\textbar{}?à???\textbar{}?¤?¨???\textbar{}?à???\textbar{}?¤?¿???\textbar{}?à???\textbar{}?¤?°???\textbar{}?à???\textbar{}?¥?????\textbar{}?à???\textbar{}?¤?®???\textbar{}?à???\textbar{}?¤?¿?
???\textbar{}?à???\textbar{}?¤???\textbar{}?ª?à???\textbar{}?¥?????\textbar{}?à???\textbar{}?¤?°???\textbar{}?à???\textbar{}?¤?¤???\textbar{}?à???\textbar{}?¤?¿?
???\textbar{}?à???\textbar{}?¤?\ldots{}???\textbar{}?à???\textbar{}?¤?¤???\textbar{}?à???\textbar{}?¥?????\textbar{}?à???\textbar{}?¤?°???\textbar{}?à???\textbar{}?¤?¿???\textbar{}?à???\textbar{}?¤???\textbar{}?µ?à???\textbar{}?¤???\textbar{}?¾?à???\textbar{}?¤?•???\textbar{}?à???\textbar{}?¥?????\textbar{}?à???\textbar{}?¤?¯???\textbar{}?à???\textbar{}?¤?®???\textbar{}?à???\textbar{}?¥???
???\textbar{}?à???\textbar{}?¥?¤?\textless{}???\textbar{}?b?r???\textbar{}?
?/?\textgreater{}???\textbar{}?à???\textbar{}?¤?¸???\textbar{}?à???\textbar{}?¥?????\textbar{}?à???\textbar{}?¤?¤???\textbar{}?à???\textbar{}?¥?‹???\textbar{}?à???\textbar{}?¤?¤???\textbar{}?à???\textbar{}?¤???\textbar{}?µ?à???\textbar{}?¥?????\textbar{}?à???\textbar{}?¤?¯???\textbar{}?à???\textbar{}?¤???\textbar{}?¾???\textbar{}?
???\textbar{}?à???\textbar{}?¤???\textbar{}?š?à???\textbar{}?¥?‡???\textbar{}?à???\textbar{}?¤???\textbar{}?¹???\textbar{}?
???\textbar{}?à???\textbar{}?¤???\textbar{}?ª?à???\textbar{}?¥???\textbar{}?ƒ?à???\textbar{}?¤?¥???\textbar{}?à???\textbar{}?¤?¿???\textbar{}?à???\textbar{}?¤???\textbar{}?µ?à???\textbar{}?¥?€?
???\textbar{}?à???\textbar{}?¤?¨???\textbar{}?à???\textbar{}?¤?¿???\textbar{}?à???\textbar{}?¤???\textbar{}?µ?à???\textbar{}?¤???\textbar{}?¾?à???\textbar{}?¤???\textbar{}?ª?à???\textbar{}?¤???\textbar{}?¾?à???\textbar{}?¤?¶???\textbar{}?à???\textbar{}?¥?????\textbar{}?à???\textbar{}?¤???\textbar{}?š?à???\textbar{}?¥?????\textbar{}?à???\textbar{}?¤?¯???\textbar{}?à???\textbar{}?¥?‹???\textbar{}?à???\textbar{}?¤?¤???\textbar{}?à???\textbar{}?¤?§???\textbar{}?à???\textbar{}?¤???\textbar{}?¾?à???\textbar{}?¤?°???\textbar{}?à???\textbar{}?¤?¿???\textbar{}?à???\textbar{}?¤?£???\textbar{}?à???\textbar{}?¥?€?
???\textbar{}?à???\textbar{}?¥?¤?\textless{}???\textbar{}?b?r???\textbar{}?
?/?\textgreater{}???\textbar{}?à???\textbar{}?¤???\textbar{}?µ?à???\textbar{}?¥???\textbar{}?ˆ?à???\textbar{}?¤?·???\textbar{}?à???\textbar{}?¥?????\textbar{}?à???\textbar{}?¤?£???\textbar{}?à???\textbar{}?¤???\textbar{}?µ?à???\textbar{}?¥?€?
???\textbar{}?à???\textbar{}?¤?•???\textbar{}?à???\textbar{}?¤???\textbar{}?¾?à???\textbar{}?¤?¶???\textbar{}?à???\textbar{}?¥?????\textbar{}?à???\textbar{}?¤?¯???\textbar{}?à???\textbar{}?¤???\textbar{}?ª?à???\textbar{}?¥?€?
???\textbar{}?à???\textbar{}?¤???\textbar{}?š?à???\textbar{}?¥?‡???\textbar{}?à???\textbar{}?¤?¤???\textbar{}?à???\textbar{}?¤?¿?
???\textbar{}?à???\textbar{}?¤?¤???\textbar{}?à???\textbar{}?¤?¥???\textbar{}?à???\textbar{}?¥???\textbar{}?ˆ?à???\textbar{}?¤???\textbar{}?µ?à???\textbar{}?¥?‡???\textbar{}?à???\textbar{}?¤???\textbar{}?¹???\textbar{}?
???\textbar{}?à???\textbar{}?¤?\ldots{}???\textbar{}?à???\textbar{}?¤?¯???\textbar{}?à???\textbar{}?¥?‡???\textbar{}?à???\textbar{}?¤?¤???\textbar{}?à???\textbar{}?¤?¿?
???\textbar{}?à???\textbar{}?¤???\textbar{}?š???\textbar{}?
???\textbar{}?à???\textbar{}?¥?¥?\textless{}???\textbar{}?b?r???\textbar{}?
?/?\textgreater{}???\textbar{}?à???\textbar{}?¤?¨???\textbar{}?à???\textbar{}?¤?¿???\textbar{}?à???\textbar{}?¤???\textbar{}?µ?à???\textbar{}?¤???\textbar{}?¾?à???\textbar{}?¤???\textbar{}?ª?à???\textbar{}?¥?‡?
???\textbar{}?à???\textbar{}?¤???\textbar{}?ª?à???\textbar{}?¤?¿???\textbar{}?à???\textbar{}?¤?¤???\textbar{}?à???\textbar{}?¥???\textbar{}?ƒ?à???\textbar{}?¤?¦???\textbar{}?à???\textbar{}?¥?‡???\textbar{}?à???\textbar{}?¤???\textbar{}?µ?à???\textbar{}?¤?¤???\textbar{}?à???\textbar{}?¥?????\textbar{}?à???\textbar{}?¤?¯???\textbar{}?à???\textbar{}?¤?¦???\textbar{}?à???\textbar{}?¥?????\textbar{}?à???\textbar{}?¤?°???\textbar{}?à???\textbar{}?¤???\textbar{}?µ?à???\textbar{}?¥?????\textbar{}?à???\textbar{}?¤?¯???\textbar{}?à???\textbar{}?¤?¤???\textbar{}?à???\textbar{}?¥?????\textbar{}?à???\textbar{}?¤?¯???\textbar{}?à???\textbar{}?¤???\textbar{}?¾?à???\textbar{}?¤?---???\textbar{}?à???\textbar{}?¥?‡?
???\textbar{}?à???\textbar{}?¤?¯?
???\textbar{}?à???\textbar{}?¤?†???\textbar{}?à???\textbar{}?¤?¶???\textbar{}?à???\textbar{}?¥?????\textbar{}?à???\textbar{}?¤???\textbar{}?µ?à???\textbar{}?¥?????\textbar{}?à???\textbar{}?¤?¯???\textbar{}?à???\textbar{}?¥?‹???\textbar{}?à???\textbar{}?¤?¤???\textbar{}?à???\textbar{}?¤???\textbar{}?ƒ???\textbar{}?
???\textbar{}?à???\textbar{}?¤???\textbar{}?ª?à???\textbar{}?¥?????\textbar{}?à???\textbar{}?¤?°???\textbar{}?à???\textbar{}?¤?•???\textbar{}?à???\textbar{}?¥?????\textbar{}?à???\textbar{}?¤?·???\textbar{}?à???\textbar{}?¥?‡???\textbar{}?à???\textbar{}?¤?¤???\textbar{}?à???\textbar{}?¤???\textbar{}?µ?à???\textbar{}?¥?????\textbar{}?à???\textbar{}?¤?¯???\textbar{}?à???\textbar{}?¤?‚?
???\textbar{}?à???\textbar{}?¤?¦???\textbar{}?à???\textbar{}?¥?????\textbar{}?à???\textbar{}?¤?°???\textbar{}?à???\textbar{}?¤???\textbar{}?µ?à???\textbar{}?¥?????\textbar{}?à???\textbar{}?¤?¯???\textbar{}?à???\textbar{}?¤?‚?
???\textbar{}?à???\textbar{}?¤?¤?-?\textless{}???\textbar{}?b?r???\textbar{}?
?/?\textgreater{}???\textbar{}?à???\textbar{}?¤?¸???\textbar{}?à???\textbar{}?¥?????\textbar{}?à???\textbar{}?¤?¯???\textbar{}?à???\textbar{}?¤???\textbar{}?¾?à???\textbar{}?¤?§???\textbar{}?à???\textbar{}?¤???\textbar{}?¾?à???\textbar{}?¤?°???\textbar{}?à???\textbar{}?¤?­???\textbar{}?à???\textbar{}?¥?‚???\textbar{}?à???\textbar{}?¤?¤???\textbar{}?à???\textbar{}?¤???\textbar{}?¾???\textbar{}?
???\textbar{}?à???\textbar{}?¤?¯???\textbar{}?à???\textbar{}?¤?¸???\textbar{}?à???\textbar{}?¥?????\textbar{}?à???\textbar{}?¤?®???\textbar{}?à???\textbar{}?¤???\textbar{}?¾?à???\textbar{}?¤?¤???\textbar{}?à???\textbar{}?¥???
?,?
???\textbar{}?à???\textbar{}?¤?¤???\textbar{}?à???\textbar{}?¤?¸???\textbar{}?à???\textbar{}?¥?????\textbar{}?à???\textbar{}?¤?®???\textbar{}?à???\textbar{}?¤???\textbar{}?¾?à???\textbar{}?¤?¤???\textbar{}?à???\textbar{}?¥???
???\textbar{}?à???\textbar{}?¤???\textbar{}?ª?à???\textbar{}?¥?‚???\textbar{}?à???\textbar{}?¤?°???\textbar{}?à???\textbar{}?¥?????\textbar{}?à???\textbar{}?¤???\textbar{}?µ?à???\textbar{}?¤?‚?
???\textbar{}?à???\textbar{}?¤???\textbar{}?ª?à???\textbar{}?¥???\textbar{}?ƒ?à???\textbar{}?¤?¥???\textbar{}?à???\textbar{}?¤?¿???\textbar{}?à???\textbar{}?¤???\textbar{}?µ?à???\textbar{}?¥?€???\textbar{}?à???\textbar{}?¤?¸???\textbar{}?à???\textbar{}?¥?????\textbar{}?à???\textbar{}?¤?¤???\textbar{}?à???\textbar{}?¥?‹???\textbar{}?à???\textbar{}?¤?¤???\textbar{}?à???\textbar{}?¤???\textbar{}?µ?à???\textbar{}?¥?????\textbar{}?à???\textbar{}?¤?¯???\textbar{}?à???\textbar{}?¤???\textbar{}?¾???\textbar{}?
???\textbar{}?à???\textbar{}?¤???\textbar{}?µ?à???\textbar{}?¥???\textbar{}?ˆ?à???\textbar{}?¤?·???\textbar{}?à???\textbar{}?¥?????\textbar{}?à???\textbar{}?¤?£???\textbar{}?à???\textbar{}?¤???\textbar{}?µ?à???\textbar{}?¥?€???\textbar{}?à???\textbar{}?¤?¤???\textbar{}?à???\textbar{}?¥?????\textbar{}?à???\textbar{}?¤?¯???\textbar{}?à???\textbar{}?¤???\textbar{}?¾?à???\textbar{}?¤?¦???\textbar{}?à???\textbar{}?¤?¿?\textless{}???\textbar{}?b?r???\textbar{}?
?/?\textgreater{}???\textbar{}?à???\textbar{}?¤???\textbar{}?ª?à???\textbar{}?¥???\textbar{}?ƒ?à???\textbar{}?¤?¥???\textbar{}?à???\textbar{}?¤?¿???\textbar{}?à???\textbar{}?¤???\textbar{}?µ?à???\textbar{}?¥?€?
???\textbar{}?à???\textbar{}?¤?¸???\textbar{}?à???\textbar{}?¥?????\textbar{}?à???\textbar{}?¤?¤???\textbar{}?à???\textbar{}?¥?‹???\textbar{}?à???\textbar{}?¤?¤???\textbar{}?à???\textbar{}?¤???\textbar{}?µ?à???\textbar{}?¥?????\textbar{}?à???\textbar{}?¤?¯???\textbar{}?à???\textbar{}?¤???\textbar{}?¾???\textbar{}?
???\textbar{}?à???\textbar{}?¥?¤?
???\textbar{}?à???\textbar{}?¤???\textbar{}?µ?à???\textbar{}?¥???\textbar{}?ˆ?à???\textbar{}?¤?·???\textbar{}?à???\textbar{}?¥?????\textbar{}?à???\textbar{}?¤?£???\textbar{}?à???\textbar{}?¤???\textbar{}?µ?à???\textbar{}?¥?€???\textbar{}?à???\textbar{}?¤?¤???\textbar{}?à???\textbar{}?¥?????\textbar{}?à???\textbar{}?¤?¯???\textbar{}?à???\textbar{}?¤???\textbar{}?¾?à???\textbar{}?¤?¦???\textbar{}?à???\textbar{}?¤?¿?
???\textbar{}?à???\textbar{}?¤???\textbar{}?ª?à???\textbar{}?¥???\textbar{}?ƒ?à???\textbar{}?¤?¥???\textbar{}?à???\textbar{}?¤?•???\textbar{}?à???\textbar{}?¥???
???\textbar{}?à???\textbar{}?¤?¨???\textbar{}?à???\textbar{}?¤???\textbar{}?¾?à???\textbar{}?¤?®???\textbar{}?à???\textbar{}?¤???\textbar{}?ª?à???\textbar{}?¤?¦???\textbar{}?à???\textbar{}?¥???\textbar{}?ˆ?à???\textbar{}?¤???\textbar{}?ƒ???\textbar{}?
???\textbar{}?à???\textbar{}?¤?‡???\textbar{}?à???\textbar{}?¤?¤???\textbar{}?à???\textbar{}?¤?¿?
???\textbar{}?à???\textbar{}?¤?•???\textbar{}?à???\textbar{}?¤?°???\textbar{}?à???\textbar{}?¤?£???\textbar{}?à???\textbar{}?¤???\textbar{}?¾?à???\textbar{}?¤?®???\textbar{}?à???\textbar{}?¥?????\textbar{}?à???\textbar{}?¤?¨???\textbar{}?à???\textbar{}?¤???\textbar{}?¾?à???\textbar{}?¤?¨???\textbar{}?à???\textbar{}?¤???\textbar{}?¾?à???\textbar{}?¤?¤???\textbar{}?à???\textbar{}?¥???
???\textbar{}?à???\textbar{}?¥?¤?\textless{}???\textbar{}?b?r???\textbar{}?
?/?\textgreater{}???\textbar{}?à???\textbar{}?¤?¤???\textbar{}?à???\textbar{}?¤?¤???\textbar{}?à???\textbar{}?¤?¶???\textbar{}?à???\textbar{}?¥?????\textbar{}?à???\textbar{}?¤???\textbar{}?š???\textbar{}?
???\textbar{}?à???\textbar{}?¥???
???\textbar{}?à???\textbar{}?¤???\textbar{}?µ?à???\textbar{}?¥???\textbar{}?ˆ?à???\textbar{}?¤?·???\textbar{}?à???\textbar{}?¥?????\textbar{}?à???\textbar{}?¤?£???\textbar{}?à???\textbar{}?¤???\textbar{}?µ?à???\textbar{}?¥?????\textbar{}?à???\textbar{}?¤?¯???\textbar{}?à???\textbar{}?¥???\textbar{}?ˆ???\textbar{}?
???\textbar{}?à???\textbar{}?¤?¨???\textbar{}?à???\textbar{}?¤?®???\textbar{}?à???\textbar{}?¤???\textbar{}?ƒ???\textbar{}?
???\textbar{}?à???\textbar{}?¥?¤?
???\textbar{}?à???\textbar{}?¤?•???\textbar{}?à???\textbar{}?¤???\textbar{}?¾?à???\textbar{}?¤?¶???\textbar{}?à???\textbar{}?¥?????\textbar{}?à???\textbar{}?¤?¯???\textbar{}?à???\textbar{}?¤???\textbar{}?ª?à???\textbar{}?¥?????\textbar{}?à???\textbar{}?¤?¯???\textbar{}?à???\textbar{}?¥???\textbar{}?ˆ???\textbar{}?
???\textbar{}?à???\textbar{}?¤?¨???\textbar{}?à???\textbar{}?¤?®???\textbar{}?à???\textbar{}?¤???\textbar{}?ƒ???\textbar{}?
???\textbar{}?à???\textbar{}?¥?¤?
???\textbar{}?à???\textbar{}?¤?•???\textbar{}?à???\textbar{}?¥?????\textbar{}?à???\textbar{}?¤?·???\textbar{}?à???\textbar{}?¤?¯???\textbar{}?à???\textbar{}?¤???\textbar{}?¾?à???\textbar{}?¤?¯???\textbar{}?à???\textbar{}?¥???\textbar{}?ˆ???\textbar{}?
???\textbar{}?à???\textbar{}?¤?¨???\textbar{}?à???\textbar{}?¤?®???\textbar{}?à???\textbar{}?¤???\textbar{}?ƒ???\textbar{}?
???\textbar{}?à???\textbar{}?¥?¤?
???\textbar{}?à???\textbar{}?¤?•???\textbar{}?à???\textbar{}?¥?????\textbar{}?à???\textbar{}?¤?·???\textbar{}?à???\textbar{}?¤?¯???\textbar{}?à???\textbar{}?¤?¶???\textbar{}?à???\textbar{}?¤?¬???\textbar{}?à???\textbar{}?¥?????\textbar{}?à???\textbar{}?¤?¦???\textbar{}?à???\textbar{}?¥?‹?\textless{}???\textbar{}?b?r???\textbar{}?
?/?\textgreater{}???\textbar{}?à???\textbar{}?¤?¨???\textbar{}?à???\textbar{}?¤?¿???\textbar{}?à???\textbar{}?¤???\textbar{}?µ?à???\textbar{}?¤???\textbar{}?¾?à???\textbar{}?¤?¸???\textbar{}?à???\textbar{}?¤???\textbar{}?µ?à???\textbar{}?¤???\textbar{}?š?à???\textbar{}?¤?¨???\textbar{}?à???\textbar{}?¤???\textbar{}?ƒ???\textbar{}?
???\textbar{}?à???\textbar{}?¥?¤?
???\textbar{}?à???\textbar{}?¤?¤???\textbar{}?à???\textbar{}?¤?¥???\textbar{}?à???\textbar{}?¤???\textbar{}?¾???\textbar{}?
???\textbar{}?à???\textbar{}?¤???\textbar{}?š???\textbar{}?
???\textbar{}?à???\textbar{}?¤?®???\textbar{}?à???\textbar{}?¤?¨???\textbar{}?à???\textbar{}?¥?????\textbar{}?à???\textbar{}?¤?¤???\textbar{}?à???\textbar{}?¥?????\textbar{}?à???\textbar{}?¤?°???\textbar{}?à???\textbar{}?¥?‡?
???\textbar{}?à???\textbar{}?¥?¤?\textless{}???\textbar{}?b?r???\textbar{}?
?/?\textgreater{}???\textbar{}?à???\textbar{}?¤?°???\textbar{}?à???\textbar{}?¥?‡???\textbar{}?à???\textbar{}?¤???\textbar{}?µ?à???\textbar{}?¤?¤???\textbar{}?à???\textbar{}?¥?€???\textbar{}?à???\textbar{}?¤?°???\textbar{}?à???\textbar{}?¤?®???\textbar{}?à???\textbar{}?¤?§???\textbar{}?à???\textbar{}?¥?????\textbar{}?à???\textbar{}?¤???\textbar{}?µ?à???\textbar{}?¤?®???\textbar{}?à???\textbar{}?¤?¸???\textbar{}?à???\textbar{}?¥?????\textbar{}?à???\textbar{}?¤?®???\textbar{}?à???\textbar{}?¤?¿???\textbar{}?à???\textbar{}?¤?¨???\textbar{}?à???\textbar{}?¥???
???\textbar{}?à???\textbar{}?¤?¯???\textbar{}?à???\textbar{}?¥?‹???\textbar{}?à???\textbar{}?¤?¨???\textbar{}?à???\textbar{}?¤???\textbar{}?¾?à???\textbar{}?¤???\textbar{}?µ?à???\textbar{}?¤?¸???\textbar{}?à???\textbar{}?¥?????\textbar{}?à???\textbar{}?¤?®???\textbar{}?à???\textbar{}?¤?¿???\textbar{}?à???\textbar{}?¤?¨???\textbar{}?à???\textbar{}?¥?????\textbar{}?à???\textbar{}?¤?---???\textbar{}?à???\textbar{}?¥?‹???\textbar{}?à???\textbar{}?¤?·???\textbar{}?à???\textbar{}?¥?????\textbar{}?à???\textbar{}?¤?~???\textbar{}?à???\textbar{}?¥?‡???\textbar{}?à???\textbar{}?¤???\textbar{}?½?à???\textbar{}?¤?¸???\textbar{}?à???\textbar{}?¥?????\textbar{}?à???\textbar{}?¤?®???\textbar{}?à???\textbar{}?¤?¿???\textbar{}?à???\textbar{}?¤?¨???\textbar{}?à???\textbar{}?¥???
???\textbar{}?à???\textbar{}?¤?•???\textbar{}?à???\textbar{}?¥?????\textbar{}?à???\textbar{}?¤?·???\textbar{}?à???\textbar{}?¤?¯?
???\textbar{}?à???\textbar{}?¤?‡???\textbar{}?à???\textbar{}?¤?¤???\textbar{}?à???\textbar{}?¤?¿?
???\textbar{}?à???\textbar{}?¥?¤?\textless{}???\textbar{}?b?r???\textbar{}?
?/?\textgreater{}???\textbar{}?à???\textbar{}?¤?¬???\textbar{}?à???\textbar{}?¤?°???\textbar{}?à???\textbar{}?¤???\textbar{}?¾?à???\textbar{}?¤???\textbar{}?¹?à???\textbar{}?¤???\textbar{}?ª?à???\textbar{}?¥?????\textbar{}?à???\textbar{}?¤?°???\textbar{}?à???\textbar{}?¤???\textbar{}?¾?à???\textbar{}?¤?£???\textbar{}?à???\textbar{}?¥?‡?
???\textbar{}?à???\textbar{}?¤?¤???\textbar{}?à???\textbar{}?¥???
???\textbar{}?à???\textbar{}?¤?\ldots{}???\textbar{}?à???\textbar{}?¤?•???\textbar{}?à???\textbar{}?¥?????\textbar{}?à???\textbar{}?¤?·???\textbar{}?à???\textbar{}?¤?¯???\textbar{}?à???\textbar{}?¥?‡???\textbar{}?à???\textbar{}?¤?¤???\textbar{}?à???\textbar{}?¤?¿?
???\textbar{}?à???\textbar{}?¤???\textbar{}?ª?à???\textbar{}?¤?~???\textbar{}?à???\textbar{}?¥?????\textbar{}?à???\textbar{}?¤?¯???\textbar{}?à???\textbar{}?¤?¤???\textbar{}?à???\textbar{}?¥?‡?
???\textbar{}?à???\textbar{}?¥?¤?\textless{}???\textbar{}?b?r???\textbar{}?
?/?\textgreater{}???\textbar{}?à???\textbar{}?¤???\textbar{}?ª?à???\textbar{}?¥?????\textbar{}?à???\textbar{}?¤?°???\textbar{}?à???\textbar{}?¤?£???\textbar{}?à???\textbar{}?¤?®???\textbar{}?à???\textbar{}?¥?????\textbar{}?à???\textbar{}?¤?¯?
???\textbar{}?à???\textbar{}?¤?¶???\textbar{}?à???\textbar{}?¤?¿???\textbar{}?à???\textbar{}?¤?°???\textbar{}?à???\textbar{}?¤?¸???\textbar{}?à???\textbar{}?¤???\textbar{}?¾???\textbar{}?
???\textbar{}?à???\textbar{}?¤?­???\textbar{}?à???\textbar{}?¥?‚???\textbar{}?à???\textbar{}?¤?®???\textbar{}?à???\textbar{}?¤?¿???\textbar{}?à???\textbar{}?¤?‚?
???\textbar{}?à???\textbar{}?¤?¨???\textbar{}?à???\textbar{}?¤?¿???\textbar{}?à???\textbar{}?¤???\textbar{}?µ?à???\textbar{}?¤???\textbar{}?¾?à???\textbar{}?¤???\textbar{}?ª?à???\textbar{}?¤?¸???\textbar{}?à???\textbar{}?¥?????\textbar{}?à???\textbar{}?¤?¯?
???\textbar{}?à???\textbar{}?¤???\textbar{}?š???\textbar{}?
???\textbar{}?à???\textbar{}?¤?§???\textbar{}?à???\textbar{}?¤???\textbar{}?¾?à???\textbar{}?¤?°???\textbar{}?à???\textbar{}?¤?¿???\textbar{}?à???\textbar{}?¤?£???\textbar{}?à???\textbar{}?¥?€???\textbar{}?à???\textbar{}?¤?®???\textbar{}?à???\textbar{}?¥???
???\textbar{}?à???\textbar{}?¥?¤?\textless{}???\textbar{}?b?r???\textbar{}?
?/?\textgreater{}???\textbar{}?à???\textbar{}?¤???\textbar{}?µ?à???\textbar{}?¥???\textbar{}?ˆ?à???\textbar{}?¤?·???\textbar{}?à???\textbar{}?¥?????\textbar{}?à???\textbar{}?¤?£???\textbar{}?à???\textbar{}?¤???\textbar{}?µ?à???\textbar{}?¥?€?
???\textbar{}?à???\textbar{}?¤?•???\textbar{}?à???\textbar{}?¤???\textbar{}?¾?à???\textbar{}?¤?¶???\textbar{}?à???\textbar{}?¥?????\textbar{}?à???\textbar{}?¤?¯???\textbar{}?à???\textbar{}?¤???\textbar{}?ª?à???\textbar{}?¥?€?
???\textbar{}?à???\textbar{}?¤???\textbar{}?š?à???\textbar{}?¥?‡???\textbar{}?à???\textbar{}?¤?¤???\textbar{}?à???\textbar{}?¤?¿?
???\textbar{}?à???\textbar{}?¤?\ldots{}???\textbar{}?à???\textbar{}?¤?•???\textbar{}?à???\textbar{}?¥?????\textbar{}?à???\textbar{}?¤?·???\textbar{}?à???\textbar{}?¤?¯???\textbar{}?à???\textbar{}?¥?‡???\textbar{}?à???\textbar{}?¤?¤???\textbar{}?à???\textbar{}?¤?¿?
???\textbar{}?à???\textbar{}?¤???\textbar{}?š???\textbar{}?
???\textbar{}?à???\textbar{}?¤?¨???\textbar{}?à???\textbar{}?¤???\textbar{}?¾?à???\textbar{}?¤?®???\textbar{}?à???\textbar{}?¤?¤???\textbar{}?à???\textbar{}?¤???\textbar{}?ƒ???\textbar{}?
???\textbar{}?à???\textbar{}?¥?¥?\textless{}???\textbar{}?b?r???\textbar{}?
?/?\textgreater{}???\textbar{}?à???\textbar{}?¤?¤???\textbar{}?à???\textbar{}?¤?¤???\textbar{}?à???\textbar{}?¥?????\textbar{}?à???\textbar{}?¤?°???\textbar{}?à???\textbar{}?¥???\textbar{}?ˆ?à???\textbar{}?¤???\textbar{}?µ???\textbar{}?
???\textbar{}?à???\textbar{}?¤???\textbar{}?ª?à???\textbar{}?¥???\textbar{}?ƒ?à???\textbar{}?¤?¥???\textbar{}?à???\textbar{}?¤?¿???\textbar{}?à???\textbar{}?¤???\textbar{}?µ?à???\textbar{}?¥?€???\textbar{}?à???\textbar{}?¤?‚?
???\textbar{}?à???\textbar{}?¤???\textbar{}?ª?à???\textbar{}?¥?????\textbar{}?à???\textbar{}?¤?°???\textbar{}?à???\textbar{}?¤?¤???\textbar{}?à???\textbar{}?¤?¿?
???\textbar{}?à???\textbar{}?¤???\textbar{}?µ?à???\textbar{}?¤?°???\textbar{}?à???\textbar{}?¤???\textbar{}?¾?à???\textbar{}?¤???\textbar{}?¹?à???\textbar{}?¤???\textbar{}?µ?à???\textbar{}?¤???\textbar{}?¾?à???\textbar{}?¤?•???\textbar{}?à???\textbar{}?¥?????\textbar{}?à???\textbar{}?¤?¯???\textbar{}?à???\textbar{}?¤?®???\textbar{}?à???\textbar{}?¥???
???\textbar{}?à???\textbar{}?¥?¤?\textless{}???\textbar{}?b?r???\textbar{}?
?/?\textgreater{}???\textbar{}?à???\textbar{}?¤???\textbar{}?ª?à???\textbar{}?¥?????\textbar{}?à???\textbar{}?¤?°???\textbar{}?à???\textbar{}?¤?£???\textbar{}?à???\textbar{}?¤?®???\textbar{}?à???\textbar{}?¥?????\textbar{}?à???\textbar{}?¤?¯?
???\textbar{}?à???\textbar{}?¤?¶???\textbar{}?à???\textbar{}?¤?¿???\textbar{}?à???\textbar{}?¤?°???\textbar{}?à???\textbar{}?¤?¸???\textbar{}?à???\textbar{}?¤???\textbar{}?¾???\textbar{}?
???\textbar{}?à???\textbar{}?¤?­???\textbar{}?à???\textbar{}?¥?‚???\textbar{}?à???\textbar{}?¤?®???\textbar{}?à???\textbar{}?¤?¿???\textbar{}?à???\textbar{}?¤?‚?
???\textbar{}?à???\textbar{}?¤?¨???\textbar{}?à???\textbar{}?¤?¿???\textbar{}?à???\textbar{}?¤???\textbar{}?µ?à???\textbar{}?¤???\textbar{}?¾?à???\textbar{}?¤???\textbar{}?ª?à???\textbar{}?¤?¸???\textbar{}?à???\textbar{}?¥?????\textbar{}?à???\textbar{}?¤?¥???\textbar{}?à???\textbar{}?¤???\textbar{}?¾?à???\textbar{}?¤?¨???\textbar{}?à???\textbar{}?¤?®???\textbar{}?à???\textbar{}?¤???\textbar{}?¾?à???\textbar{}?¤?---???\textbar{}?à???\textbar{}?¤?¤???\textbar{}?à???\textbar{}?¤???\textbar{}?ƒ???\textbar{}?
???\textbar{}?à???\textbar{}?¥?¥?\textless{}???\textbar{}?b?r???\textbar{}?
?/?\textgreater{}???\textbar{}?à???\textbar{}?¤?¸???\textbar{}?à???\textbar{}?¥?????\textbar{}?à???\textbar{}?¤?¤???\textbar{}?à???\textbar{}?¥?????\textbar{}?à???\textbar{}?¤???\textbar{}?µ?à???\textbar{}?¥?€???\textbar{}?à???\textbar{}?¤?¤???\textbar{}?à???\textbar{}?¤???\textbar{}?¾?à???\textbar{}?¤?¨???\textbar{}?à???\textbar{}?¥?‡???\textbar{}?à???\textbar{}?¤?¨?
???\textbar{}?à???\textbar{}?¤?®???\textbar{}?à???\textbar{}?¤?¨???\textbar{}?à???\textbar{}?¥?????\textbar{}?à???\textbar{}?¤?¤???\textbar{}?à???\textbar{}?¥?????\textbar{}?à???\textbar{}?¤?°???\textbar{}?à???\textbar{}?¥?‡???\textbar{}?à???\textbar{}?¤?£?
???\textbar{}?à???\textbar{}?¤?¤???\textbar{}?à???\textbar{}?¥?????\textbar{}?à???\textbar{}?¤???\textbar{}?µ?à???\textbar{}?¤???\textbar{}?¾?à???\textbar{}?¤?‚?
???\textbar{}?à???\textbar{}?¤???\textbar{}?š???\textbar{}?
???\textbar{}?à???\textbar{}?¤?­???\textbar{}?à???\textbar{}?¤?•???\textbar{}?à???\textbar{}?¥?????\textbar{}?à???\textbar{}?¤?¤???\textbar{}?à???\textbar{}?¥?????\textbar{}?à???\textbar{}?¤?¯???\textbar{}?à???\textbar{}?¤???\textbar{}?¾???\textbar{}?
???\textbar{}?à???\textbar{}?¤???\textbar{}?µ?à???\textbar{}?¥?????\textbar{}?à???\textbar{}?¤?¯???\textbar{}?à???\textbar{}?¤???\textbar{}?µ?à???\textbar{}?¤?¸???\textbar{}?à???\textbar{}?¥?????\textbar{}?à???\textbar{}?¤?¥???\textbar{}?à???\textbar{}?¤?¿???\textbar{}?à???\textbar{}?¤?¤???\textbar{}?à???\textbar{}?¤???\textbar{}?ƒ???\textbar{}?
???\textbar{}?à???\textbar{}?¥?¤?\textless{}?/???\textbar{}?s?p?a?n???\textbar{}?\textgreater{}?\textless{}?/???\textbar{}?p???\textbar{}?\textgreater{}?\textless{}?/???\textbar{}?b?o?d?y???\textbar{}?\textgreater{}?\textless{}?/???\textbar{}?h?t?m?l???\textbar{}?\textgreater{}?

{ }{ पुण्डरीकाक्षस्मरणादिकृत्यम् । १९३}{\\
मेदिनी लोकमाता च क्षितिस्तूर्वी धरा मही ॥\\
भूमिः शैलशिला च त्वं स्थिरा तुभ्यं नमो नमः ।\\
धरणी काश्यपी क्षोणी रसा विश्वम्भरा च भूः ॥\\
जगत्प्रतिष्ठा वसुधा त्वं हि मातर्नमोऽस्तु ते ।\\
वैष्णवी भूतदेवी व पृथिवी त्वं नमोऽस्तु ते ॥ इति ।\\
एवं पृथिवीस्तुतिं कृत्वा श्राद्धभूमिं गयात्मकत्वेनाभिध्याय तत्र\\
स गदाधरं ध्यात्वा तयोश्च नमस्कार कृत्वा तदुत्तरं श्राद्धं कुर्यात् ।\\
तदुक्तं ब्रह्माण्डपुराणे ।\\
श्राद्धभूमिं गयां ध्यात्वा ध्यात्वा देवं गदाधरम् ।\\
ताभ्यां चैव नमस्कारं ततः श्राद्ध प्रवर्तयेत् ॥\\
ताभ्यामिति = षष्ठ्यर्थे चतुर्थी, तयोर्नमस्कारं कृत्वेत्यर्थः । ततश्च\\
ॐ गयायै नमः । ॐ गदाधराय नमः । एवं गयागदाधरनमस्कारं\\
कृत्वा जप्यान् मन्त्रान् जपेत् । तथा चाह -\\
प्रचेताः ।\\
अपसव्यं ततः कृत्वा जप्त्वा मन्त्रं तु वेष्णवम् ।\\
गायत्रीं प्रणवं चापि ततः श्राद्धमुपक्रमेत् }{॥}{\\
वैष्णवमन्त्रा `` इदं विष्णुः'' इत्यादयः ।\\
ब्रह्मपुराणे ।\\
उपविश्य जपेद् धीमान् गायत्रीं तदनुज्ञया ।\\
तथा ।\\
पापापहं पावनीयं अश्वमेधफलं तथा ।\\
मन्त्रं वक्ष्याम्यह तस्मादमृतं ब्रह्मनिर्मितम् ॥\\
देवताभ्यः पितृभ्यश्च महायोगिभ्य एव च ।\\
नमः स्वाहायै स्वधायै नित्यमेव नमो नमः ।

{आद्यावसाने श्राद्धस्य ( १ ) त्रिरावर्ता जपेत् सदा ॥\\
अश्वमेधफलं ह्येतद् द्विजः सत्कृत्य पूजितम् ।\\
पिण्डनिर्वपणे चापि जपेदेतत् समाहितः ॥\\
पठ्यमानमिदं श्रुत्वा श्राद्धकाल उपस्थिते ।\\
पितरः क्षिप्रमायान्ति राक्षसाः प्रद्रवन्ति च ॥ इति ।\\
तथा ।\\
अमूर्तीनां समूर्तीनां पितॄणां दीप्ततेजसाम् ।

% \begin{center}\rule{0.5\linewidth}{0.5pt}\end{center}

% \begin{center}\rule{0.5\linewidth}{0.5pt}\end{center}

{१९४ वीरमित्रोदयस्य श्राद्धप्रकाशे-}{\\
नमस्यामि सदा तेषां ध्यायिनां योगचक्षुषाम् ॥ इति ।\\
तथा ।\\
चतुर्भिश्च चतुर्भिश्च द्वाभ्यां पञ्चभिरेव च ।\\
हूयते च पुनर्द्वाभ्यां स मे विष्णुः प्रसीद तु ॥ इति ।\\
अत्र कर्माङ्गभूतदेशकालयोः शिष्टाचारप्रामाण्येन संकीर्तनं\\
कृत्वा ब्राह्मणाभ्यनुज्ञाग्रहणार्थं वक्ष्यमाणेतिकर्तव्यतया पृच्छां कु-\\
यत् । तदुक्तं-\\
ब्रह्माण्डपुराणे ।\\
उभौ हस्तौ समौ कृत्वा जानुभ्यामन्तरे स्थितौ ।\\
सप्रश्रयश्चोपविष्टान् सर्वान् पृच्छेद् द्विजोत्तमान् ॥\\
श्राद्धं करिष्य इत्येवं पृच्छेद्विप्रान् समाहितः ।\\
कुरुष्वेति स तैरुतो दद्यात् दर्भासनं ततः }{॥}{\\
सप्रश्रयो = विनयान्वितः । सर्वान् पृच्छेदिति = सर्व प्रश्नपक्षो
वैकल्पिकः ।\\
तथा च कात्यायनः ।\\
प्रश्नेषु पङ्क्तिमूर्धन्यं पृच्छति, सर्वान् वेति । प्रश्नेष्विति बहुवच\\
ननिर्देशात् सर्वप्रश्नविषयता । पङ्क्तिमूर्धन्यः =पङ्क्त्यादौ
उपविष्टः । अत्र\\
पृच्छाप्रकारे विशेषो-\\
ब्रह्मपुराणे ।\\
पितॄन् पितामहान् पक्षे भोजनेन यथाक्रमम् ।\\
प्रपितामहान् सर्वांश्च तत्पितॄंश्चानुपूर्वशः }{॥}{\\
भुज्यते तद्भोजनं तेनेत्यर्थः । सर्वान् = पितृपितामहप्रपितामहान् ।\\
तत्पितॄन् = प्रपितामहपितॄन् लेपभाजः । अनुपूर्वशः = अनुक्रमेणेत्यर्थः ।\\
अपृच्छायां दोष:-\\
कालिकाखण्डे ।\\
अपृच्छन् प्रवरेद्यस्तु नरो विप्रांश्च पार्वति ।\\
तस्य प्रियं मत्प्रमुखा नाचरन्ति दिवौकसः ॥ इति ।\\
अत्र प्रश्नानन्तरं कुरुष्वेति ब्राह्मणैरनुज्ञातो नीवीबन्धनं कु-\\
र्यात् । श्राद्धप्रकृतिभूते पिण्डपितृयज्ञे द्वितीयावनेजनानन्तरम् ` अथ\\
नीषीमुद्धृत्य नमस्करोति पितृदेवत्या वै नीविरिति' नीविविस्रंसनो-\\
पदेशात् । कात्यायनसूत्रे च ` नीवीं विस्रंस्य नमो व इति अञ्जलिं क\\
रोतीति' ततश्च विस्रंसनस्य बन्धनपूर्वकत्वात् तद्बन्धनस्याऽना-\\
म्नानेऽपि तद्बन्धनं कर्मादौ कर्तव्यमित्यर्थसिद्धं भवति । तथा च तत्र

{ }{पुण्डरीकाक्षस्मरणादिकृत्यम् । १९५}{\\
कर्काचार्या आहुः । अत्र नीविवि}{स्रं}{सनविधानात् कर्मारम्भे नीवि\\
बन्धः कर्तव्य इत्यर्थाद्गम्यत इति ।\\
तथा भविष्यपुराणे ।\\
बध्नीयात्तु तथा नीविं न व प्रेक्षेत दुर्जनम् ।\\
स्यात् कर्ता नियतस्त्वेव यावच्छ्राद्धं समाप्यते ॥\\
नीविर्विपरिवर्तिको व्यत ? इति सर्वयाज्ञिकाः । सा च वामभागकक्षा\\
यावर्तनेन सिद्ध्यति ।\\
हेमाद्रिकारास्तु ।\\
नीविर्नाम तिलकुशान्वितानां परिहितवस्त्रोत्तराञ्चलदशानां\\
वामकटिसलँलग्नवस्त्रबहिर्भागेन संवेष्ट्य गोपनम् ।\\
नीविश्रद्धालवस्तु `` निहम्नि सर्वे यदमेध्यवत्'' इति तिलविक-\\
रणे दर्शयिष्यमाणं मन्त्रं नीविबन्धने पठन्ति ।\\
अत्र केचिन्नीविबन्धनं वाजसनेयिनामेवेच्छन्ति । तेषामेव\\
श्रुतिसूत्रयोर्विधानात् ।\\
अपरे तु भविष्यत्पुराणात् सर्वशाखिनामित्याहुः।\\
नीविबन्धनानन्तर श्राद्धरक्षार्थं वैश्वदेविकप्रदेशे यवान् विकि•\\
रेतू । तदुकं -\\
ब्रह्मपुराणे ।\\
अक्षतैर्दैवतानां व रक्षां चक्रे गदाधरः ।\\
अक्षतास्तु यवौषध्यः सर्वदेवास्त्रसम्भवाः ॥\\
रक्षन्ति सर्वोस्त्रिदशान् रक्षार्थं निर्मिता हि ते ।\\
देवदानवदैत्येषु यक्षरक्षःसु चैव हि ॥\\
न कश्चिद् दुष्कृतं तेषां कर्तुं शक्तश्चराचरे ।\\
देवतानां हि रक्षार्थं नियुक्ता विष्णुना पुरा ।।\\
एवं वैश्वदेविकप्रदेशे यवप्रकिरणं कृत्वा पित्र्यब्राह्मणप्रदेशे प-\\
रितस्तिलान् गौरसर्षपांश्च प्रकिरेदिति । तदुक्तं-\\
निगमे ।\\
``अपहता असुरा रक्षांसि वेदिषदः'' इति तिलान् गौरसर्षपांश्च\\
श्राद्धभूमौ घनं तिलान् विकिरेदिति ।\\
घनम् = निबिडम् ।\\
ब्रह्माण्डपुराणे ।\\
रक्षार्थं पितृसत्रस्य त्रिः कृत्वः सर्वतो दिशम् ।

{१९६ वीरामित्रोदयस्य श्राद्धप्रकाशे-}{\\
तिलांस्तु प्रक्षिपेन्मन्त्रमुच्चार्याऽपहता इति ॥\\
अत्र प्रतीकेन अपहता असुरा रक्षांसि पिशाचा ये क्षयन्ति पृ-\\
थिवीमनु । अन्यत्रेतो गच्छन्तु यत्रैषां गतं मन इति मन्त्रो विवक्षित\\
इति हेमाद्रिः । मन्त्रान्तरमपि -\\
ब्रह्माण्डपुराणे ।\\
परितः पितृविप्राणामपेतो यन्त्वितीरयेत् ।\\
असुये ईयुरितिचापसव्यं विकिरेत् तिलान् ॥\\
अपेतो यन्तु पणयो सुम्ना देवपीयवः । द्युभिरहोभिरक्तुभिर्व्यक्तं\\
धमो ददात्वसानमस्मै । इति एको मन्त्रः ।``असुंय ईयु'' रिति च तत्प\\
दैरुपलक्षिन `` उदीरतामवर उत्परास उन्मध्यमाः पितरः सोम्यासः ।\\
असुय ईयुरवृका ऋतज्ञास्ते नोवन्तु पितरो हवेपु'' इति ज्ञेयः ।\\
मार्कण्डेयपुराणे ।\\
रक्षोघ्नांस्तु जपेन् मन्त्रांस्तिलैश्च विकिरेन्महीम् ।\\
सिद्धार्थकैश्च रक्षार्थं श्राद्धे हि प्रचुर छलम् ॥\\
सौरपुराणे ।\\
उपवेश्य ततो विप्रान् दत्वा चैव कुशासनम् ।\\
पश्चाच्छ्राद्धस्य रक्षार्थं तिलांश्च विकिरेत् ततः ॥\\
तथा भविष्यत्पुराणे ।\\
सिद्धार्थकैः कृष्णतिलैः कार्ये वाप्यवकीरणम् ।\\
रुरुसूर्याग्निवस्तानां दर्शनं चापि यत्नतः ॥\\
विष्णुधर्मोत्तरे ।\\
अपयन्त्यसुरा द्वाभ्यां यातुधाना विसर्जनम् ।\\
तिलैः कुर्यात् प्रयत्नेन अथवा गौरसर्षपैः ॥\\
अपयन्त्यसुराः पितृरूपा ये रूपाणि प्रतिमुच्याचरन्ति परापुरो\\
निपुरो ये भरन्त्यनिष्ठाल्लोकात् प्रणुदात्वस्मात् । अपयन्त्यसुरा\\
पितृषद उदीरत्तामवर उत्परास उन्मध्यमाः पितरः सोम्यासः ।\\
असुये ईयुरवृका ऋतज्ञा स्तेनोऽवन्तु पितरो हवेष्विति द्वौ\\
मन्त्रौ तथा तत्रैव ।\\
निहन्मि सर्वे यदमेध्यवद्भवेत् धृताश्च सर्वेऽसुरदानवा मया ।\\
ये राक्षसा ये च पिशाचगुह्यका हता मया यातुधानाश्च सर्वे ॥\\
एतेन मन्त्रेण सुसंयतात्मा तिलान् सुकृष्णान् विकिरेच्च दिक्षु ।

{ }{ ब्राह्मणानामासनदानादिकृत्यम् । १६७}{\\
द्वारदेशे कुशतिलप्रक्षेपमन्त्रः -\/-\\
स्कन्दपुराणे ।\\
तिला रक्षन्त्वसुरान् दर्भा रक्षन्तु राक्षसान् ।\\
पङ्क्तिं वै श्रोत्रियो रक्षेदतिथिः सर्वरक्षकः । इति ॥\\
प्राच्यां दिशि दिक्षुमध्ये च तिलविकिरणे मन्त्रो -\/-\\
भविष्यत्पुराणे ।\\
अग्निष्वात्ताः पितृगणाः प्राचीं रक्षन्तु मे दिशम् ।\\
तथा बर्हिषदः पान्तु याम्यां ये पितरः स्मृताः ।\\
प्रतीच्यामाज्यपास्तद्वदुदीचीमपि सोमपाः}{॥ }{\\
अधोर्ध्वमपि कोणेषु हविष्मन्तश्च सर्वशः ।\\
रक्षोभूतपिशाचेभ्यस्तथैवासुरदोषतः}{॥ }{\\
सर्वतश्चाधिपस्तेषां यमो रक्षां करोतु मे ।\\
वायुभूतपि}{तॄ}{णान्तु तृप्तिर्भवतु शाश्वती ॥ इति ।\\
एवं तिलविकरणं कृत्वा दुष्टदृष्टिनिपातादिदूषित पाकादिश्रा\\
द्धीयद्रव्यस्य पवित्रमन्त्रैरद्भिः प्रोक्षणं कुर्यात् ।\\
तथा च वशिष्ठ. ।\\
शुद्धवतीभिः कूष्माण्डीभिः पावमानीभिश्च पाकादि प्रोक्षेत् ।\\
शुद्धवतीभिः = छन्दोगैः पठ्यमानाभिः । कूष्माण्डैः = याजुषैः । पावमानीभिः
=\\
बहूवृचैः । ताश्च पूर्वमुक्ता एव ।\\
भविष्ये ।\\
ततः श्राद्धीयद्रव्याणि मृत्तिकातिलवारिभिः ।\\
पितॄणान्तु समभ्युक्ष्य भवेत् कर्तोत्तरामुखः ॥\\
अथ ब्राह्मणानामासनदानादिकृत्यम् ।\\
तत्र देवपूर्व श्राद्धमित्युक्तत्वात् पूर्वे देवे पश्चात् पित्रे ।\\
तत्र यद्यपि ।\\
पाणिप्रक्षालनं दत्वा विष्टरार्थं कुशानपि ।\\
आवाहयेदनुज्ञातो विश्वे देवास इत्यृचा }{॥ }{\\
यवैरन्ववकीर्याथ भाजने सपवित्रके ।\\
शन्नोदेव्या पयः क्षिप्त्वा यवोसीति यवांस्तथा ॥\\
या दिव्या इति मन्त्रेण हस्तेष्वर्धं विनिःक्षिपेत् ।\\
दत्वोदकं गन्धमाल्य धूपदानं सदपिकम् ॥ 

{१९८ वीरमित्रोदयस्य श्राद्धप्रकाशै -}{\\
तथाच्छादनदान च करशौचार्थमम्बु च ।\\
अपसव्यं ततः कृत्वा पितॄणामप्रदक्षिणम् }{॥ }{\\
द्विगुणांस्तु कुशान् दत्वा उशन्तस्त्वेतृचा पितॄन् ।\\
आवाह्य तदनुज्ञातो जपेदायन्तुनस्ततः }{॥ }{\\
यवार्थास्तु तिलैः कार्याः कुर्यादर्घ्यादिपूर्ववत् ।\\
दत्वार्धं संस्रवास्तेषां पात्रे कृत्वा विधानतः ॥\\
पितृभ्यस्थानमसीति न्युब्जं पात्रं करोत्यधः ।\\
इति याज्ञवल्क्येनासनाद्याच्छादनदानान्तान् काण्डानुसमयेन दै-\\
वान् पदार्थानभिधाय ` अपसव्यं ततः कृत्वे' त्यादिना काण्डानुसमये\\
नैव पित्र्याः पदार्थाः अभिहिताः,\\
तथापि प्रयोगविध्यनुमतप्रधानप्रत्यासत्यनुग्रह रूपन्यायानुगृ-\\
हीतकात्यायनसूत्राद्वैवपित्र्यक्रियासमवायिन }{आसनावाहनादयः}{\\
पदार्थानुसमयेनाप्यनुष्ठेयाः ।\\
तथा च कात्यायनः ।\\
आसनेषु दर्भानास्तीर्य विश्वान् देवानावाहयिष्ये इति पृच्छति\\
आवाहयेत्यनुज्ञातो `` विश्वदेवास आगत'' इत्यनयावाह्यावकीर्य `` वि-\\
श्वेदेवाः शृणुतेममिति जपित्वा'' पितॄन् आवाहयिष्यत इति पृच्छति ।\\
आवाहयेत्यनुज्ञात ``उशन्तस्त्वे''त्यनया वाह्यावकीर्य ``आयन्तु न'' इति\\
जपित्वा यज्ञियवृक्षचमसेषु पवित्रान्तर्हितेष्वेकैकस्मिन्नप आसिञ्चति\\
`` शन्नोदेवी '' रिति । एकैकस्मिन्नेव तिलानावपति `` तिलोसि सोमदेव-\\
त्यो गोसवो देवनिर्मितः । प्रत्तमद्भिः पृक्तः स्वधया पितॄल्लोकान्\\
श्रीणाहि नः स्वाहे ''ति सौवर्णराज तौदुम्बरखङ्गमणिमयानां पात्राणा\\
मन्यतमेषु वा यानि वा विद्यन्ते पत्रपुटेषु वैकैकस्यैकैकेन ददाति, सप.\\
वित्रेषु हस्तेषु "या दिव्या आपः पयसा सम्बभूवुर्या अन्तरिक्षा उत\\
पार्थिवीर्याः । हिरण्यवर्णा यज्ञियास्तान आपः शिवाः शं स्योनाः\\
सुहवा भवन्तु इत्यसावेषते अर्ध इति प्रथमे पात्रे सभ्स्रवान् सम-\\
चनीय "पितृभ्यः स्थानमसी" ति न्युब्ज पात्रं निद्धाति । अत्र गन्ध-\\
पुष्पधूपदीपवाससां च प्रदानमिति । अत्र पदार्थानुसमयपक्षे पूर्व\\
दैवब्राह्मणासने पश्चात् पित्र्यब्राह्मणासने दर्भास्तरणम् । एवमावा-\\
हनादिष्वपि द्रष्टव्यम् । अत्र यद्यपि उपवेशनोत्तरं कुशास्तरणं आ\\
स्नातम् । तथापि सामर्थ्याद्यवाणूपाकवत् पूर्वमनुष्ठेयम् ।

{ }{ब्राह्मणानामासनदानादिकृत्यम् । १९९}{\\
अत्र दर्भानिति बहुवचननिर्देशात् त्रयाणामास्तरणं कार्यम् ।\\
अत्रासनदानात् पूर्व विप्रहस्ते पाणिप्रक्षालनार्थमुदकं देयम् ।\\
तथा च-\\
याज्ञवल्क्यः ।\\
पाणिप्रक्षालनं दत्त्वा विष्टरार्थ कुशानपि ।\\
मासने इति शेषः । नतु हस्ते । तथा च-\/-\\
प्रचेताः ।\\
दर्भाश्वासने दद्यात्रतु पाणौ कदा चन ।\\
देवपितृमनुष्याणां स्यात् त्वष्टिः शाश्वती तथा ॥\\
तथाक्रियमाणे, हस्ते दर्भासने प्रदीयमाने, तत्र देवपितृम.\\
नुष्यप्राजापत्यतीर्थानां सद्भावात् देवादीनां ममेदममेदमिति परस्परं\\
स्वष्टि = कलहः स्यात् ।\\
तथा च नागरखण्डे।\\
हस्ते तोयं परिक्षेप्यं न दर्भास्तु कथं च न ।\\
यो हस्ते चासनं दद्यात् तं दर्भ बुद्धिवर्जितः ॥\\
पितरो नासने तत्र प्रकुर्वन्ति निवेशनम् ।\\
अत्र विशेषो ब्रह्माण्डपुराणे ।\\
आसनं चासने दद्यात् वामे वा दक्षिणेऽपि वा ॥ इति ।\\
वा शब्दो व्यवस्थितविकल्पार्थ' । तेन पितृब्राह्मणस्य वामे=चाम.\\
भागावस्थिते आसने । वैश्वदेविकब्राह्मणस्य दक्षिणे । तथा च तदुः\\
त्तरं तत्रैव पठ्यते ।\\
पितृकर्मणि वामे वै दैवे कर्मणि दक्षिणे ।\\
अत्र काठके विशेषः । प्रदद्यादासने दर्भानिति प्रकृत्य-\/-\\
देवानां सयवा दर्भाः पितॄणां च तिलैः सह ॥ इति ॥\\
विशेषान्तरमप्युक्तं बौधायनेन ।\\
प्रदक्षिणं तु देवानां पितृणाम प्रदक्षिणम् ।\\
देवानां सयवा दर्भाः पितॄणां द्विगुणास्तिलैः ॥\\
द्विगुणा = संश्लिष्टमूलाप्राः ।\\
तथा च ब्रह्मपुराणे ।\\
लिष्टमूलाप्रदर्भास्तु सतिलान वेद तत्र वित् ।\\
तानारोग्यासने तत्र ददौ सव्येन चाssसनम् ।\textbar{}}

?\textless{}?!?D?O?C?T?Y?P?E? ?H?T?M?L? ?P?U?B?L?I?C?
?"?-?/?/?W?3?C?/?/?D?T?D? ?H?T?M?L? ?4?.?0?/?/?E?N?"?
?"?h?t?t?p?:?/?/?w?w?w?.?w?3?.?o?r?g?/?T?R?/?R?E?C?-?h?t?m?l?4?0?/?s?t?r?i?c?t?.?d?t?d?"?\textgreater{}?
? ?

?

?

?

? ? ?p?,? ?l?i? ?\{? ?w?h?i?t?e?-?s?p?a?c?e?:? ?p?r?e?-?w?r?a?p?;? ?\}?
? ?b?o?d?y? ?\{? ?w?i?d?t?h?:? ?2?1?c?m?;? ?h?e?i?g?h?t?:?
?2?9?.?7?c?m?;? ?m?a?r?g?i?n?:? ?3?0?m?m? ?4?5?m?m? ?3?0?m?m?
?4?5?m?m?;? ?\}? ?
?\textless{}?/?s?t?y?l?e?\textgreater{}?\textless{}?/?h?e?a?d?\textgreater{}?

? ? ?

?

?à?¥?¨?\textless{}?/?s?p?a?n?\textgreater{}?

?à?¥?¦?à?¥?¦?
?à?¤?µ?à?¥?€?à?¤?°?à?¤?®?à?¤?¿?à?¤?¤?à?¥???à?¤?°?à?¥?‹?à?¤?¦?à?¤?¯?à?¤?¸?à?¥???à?¤?¯?
?à?¤?¶?à?¥???à?¤?°?à?¤?¾?à?¤?¦?à?¥???à?¤?§?à?¤?ª?à?¥???à?¤?°?à?¤?•?à?¤?¾?à?¤?¶?à?¥?‡?-?\textless{}?/?s?p?a?n?\textgreater{}?

?

?à?¤?¦?à?¥?‡?à?¤?µ?à?¤?¾?à?¤?¨?à?¤?¾?à?¤?‚? ?à?¤?¤?à?¥???
?à?¤?‹?à?¤?œ?à?¤?µ? ?à?¤???à?¤?µ? ?à?¤?¦?à?¤?°?à?¥???à?¤?­?à?¤?¾?à?¤?ƒ?
?à?¥?¤?

?à?¤?¤?à?¤?¥?à?¤?¾? ?à?¤?š?
?à?¤?¬?à?¥?ƒ?à?¤?¹?à?¤?¸?à?¥???à?¤?ª?à?¤?¤?à?¤?¿?à?¤?ƒ? ?à?¥?¤?

?à?¤?‹?à?¤?œ?à?¥?‚?à?¤?¨?à?¥??? ?à?¤?¸?à?¤?­?à?¥???à?¤?¯?à?¥?‡?à?¤?¨?
?à?¤?µ?à?¥?ˆ? ?à?¤?•?à?¥?ƒ?à?¤?¤?à?¥???à?¤?µ?à?¤?¾?
?à?¤?¦?à?¥?ˆ?à?¤?µ?à?¥?‡? ?à?¤?¦?à?¤?°?à?¥???à?¤?­?à?¤?¾?à?¤?ƒ?
?à?¤?ª?à?¥???à?¤?°?à?¤?¦?à?¤?•?à?¥???à?¤?·?à?¤?¿?à?¤?£?à?¤?®?à?¥???
?à?¥?¤?

?à?¤?¦?à?¥???à?¤?µ?à?¤?¿?à?¤?---?à?¥???à?¤?£?à?¤?¾?à?¤?¨?à?¤?ª?à?¤?¸?à?¤?µ?à?¥???à?¤?¯?à?¥?‡?à?¤?¨?
?à?¤?¦?à?¤?¦?à?¥???à?¤?¯?à?¤?¾?à?¤?¤?à?¥???
?à?¤?ª?à?¤?¿?à?¤?¤?à?¥???à?¤?°?à?¥???à?¤?¯?à?¥?‡?à?¤?½?à?¤?ª?à?¤?¸?à?¤?µ?à?¥???à?¤?¯?à?¤?µ?à?¤?¤?à?¥???
?à?¥?¥?

?à?¤?\ldots{}?à?¤?ª?à?¤?¸?à?¤?µ?à?¥???à?¤?¯?à?¤?µ?à?¤?¤?à?¥??? ?=?
?à?¤?\ldots{}?à?¤?ª?à?¥???à?¤?°?à?¤?¦?à?¤?•?à?¥???à?¤?·?à?¤?¿?à?¤?£?à?¤?®?à?¤?¿?à?¤?¤?à?¥???à?¤?µ?à?¤?°?à?¥???à?¤?¥?à?¤?ƒ?
?à?¥?¤?

?à?¤?¨?à?¤?¾?à?¤?---?à?¤?°?à?¤?--?à?¤?£?à?¥???à?¤?¡?à?¥?‡? ?à?¤?µ?
?à?¤?µ?à?¤?¿?à?¤?¶?à?¥?‡?à?¤?·?à?¤?ƒ? ?à?¥?¤?

?à?¤?‹?à?¤?œ?à?¥???à?¤?­?à?¤?¿?à?¤?ƒ?
?à?¤?¸?à?¤?¾?à?¤?•?à?¥???à?¤?·?à?¤?¤?à?¥?ˆ?à?¤?°?à?¥???à?¤?¦?à?¤?°?à?¥???à?¤?­?à?¥?ˆ?à?¤?ƒ?
?à?¤?¸?à?¥?‹?à?¤?¦?à?¤?•?à?¥?ˆ?à?¤?°?à?¥???à?¤?¦?à?¤?•?à?¥???à?¤?·?à?¤?¿?à?¤?£?à?¤?¾?à?¤?¦?à?¤?¿?à?¤?¶?à?¤?¿?
?à?¥?¤?

?à?¤?¦?à?¥?‡?à?¤?µ?à?¤?¾?à?¤?¨?à?¤?¾?à?¤?®?à?¤?¾?à?¤?¸?à?¤?¨?à?¤?‚?
?à?¤?¦?à?¤?¦?à?¥???à?¤?¯?à?¤?¾?à?¤?¤?à?¥???
?à?¤?ª?à?¤?¿?à?¤?¤?à?¥?„?à?¤?£?à?¤?¾?à?¤?‚?
?à?¤?¤?à?¥???à?¤?µ?à?¤?¨?à?¥???à?¤?ª?à?¥?‚?à?¤?°?à?¥???à?¤?µ?à?¤?¶?à?¤?ƒ?
?\textless{}?/?s?p?a?n?\textgreater{}?

?à?¥?¥?\textless{}?/?s?p?a?n?\textgreater{}?

?

?à?¤?µ?à?¤?¿?à?¤?·?à?¤?®?à?¥?ˆ?à?¤?°?à?¥???à?¤?¦?à?¥???à?¤?µ?à?¤?¿?à?¤?---?à?¥???à?¤?£?à?¥?ˆ?à?¤?°?à?¥???à?¤?¦?à?¤?°?à?¥???à?¤?­?à?¥?ˆ?à?¤?ƒ?
?à?¤?¸?à?¤?¤?à?¤?¿?à?¤?²?à?¥?‡?à?¤?°?à?¥???à?¤?µ?à?¤?¾?à?¤?®?à?¤?ª?à?¤?¾?à?¤?°?à?¥???à?¤?¶?à?¥???à?¤?µ?à?¤?---?à?¥?ˆ?à?¤?ƒ?
?à?¥?¤?

?à?¤?\ldots{}?à?¤?¤?à?¥???à?¤?°?
?à?¤?ª?à?¤?¿?à?¤?¤?à?¥?ƒ?à?¤?£?à?¤?¾?à?¤?‚?
?à?¤?µ?à?¤?¿?à?¤?·?à?¤?®?à?¤?¸?à?¤?--?à?¥???à?¤?¯?à?¤?¦?à?¤?°?à?¥???à?¤?­?à?¤?µ?à?¤?¿?à?¤?§?à?¤?¾?à?¤?¨?à?¤?¾?à?¤?¤?à?¥???
?à?¤?¦?à?¥?‡?à?¤?µ?à?¤?¾?à?¤?¨?à?¤?¾?à?¤?‚?
?à?¤?¸?à?¤?®?à?¤?¸?à?¤?‚?à?¤?--?à?¥???à?¤?¯?à?¤?¾?à?¤?•?à?¤?¾?

?à?¤?¦?à?¤?°?à?¥???à?¤?­?à?¤?¾?
?à?¤?­?à?¤?µ?à?¤?¨?à?¥???à?¤?¤?à?¥?€?à?¤?¤?à?¤?¿?
?à?¤?ª?à?¥???à?¤?°?à?¤?¤?à?¥?€?à?¤?¯?à?¤?¤?à?¥?‡? ?à?¥?¤?
?à?¤?†?à?¤?¸?à?¤?¨?à?¥?‡?
?à?¤?ª?à?¥???à?¤?°?à?¤?¦?à?¤?¤?à?¥???à?¤?¤?à?¤?•?à?¥???à?¤?¶?à?¤?¾?à?¤?¨?à?¤?¾?à?¤?‚?
?à?¤?¸?à?¥???à?¤?µ?à?¥?€?à?¤?•?à?¤?°?à?¤?£?à?¥?‡?à?¤?½?à?¤?ª?à?¤?¿?
?à?¤?®?

?à?¤?¤?à?¥???à?¤?µ?à?¤?µ?à?¤?¿?à?¤?¶?à?¥?‡?à?¤?·?à?¤?ƒ?
?à?¤?ª?à?¥???à?¤?°?à?¤?š?à?¥?‡?à?¤?¤?à?¤?¸?à?¥?‹?à?¤?•?à?¥???à?¤?¤?à?¤?ƒ?
?à?¥?¤?

?à?¤?¦?à?¥?‡?à?¤?µ?à?¥?‡? ?à?¤?¤?à?¥??? ?à?¤?‹?à?¤?œ?à?¤?µ?à?¥?‹?
?à?¤?¦?à?¤?°?à?¥???à?¤?­?à?¤?¾?
?à?¤?ª?à?¥???à?¤?°?à?¤?¦?à?¤?¾?à?¤?¤?à?¤?µ?à?¥???à?¤?¯?à?¤?¾?à?¤?ƒ?
?à?¤?ª?à?¥?ƒ?à?¤?¥?à?¤?•?à?¥??? ?à?¤?ª?à?¥?ƒ?à?¤?¥?à?¤?•?à?¥??? ?à?¥?¤?

?à?¤?§?à?¤?°?à?¥???à?¤?®?à?¥?‹?à?¤?½?à?¤?¸?à?¥???à?¤?®?à?¥?€?à?¤?¤?à?¥???à?¤?¯?à?¤?¥?
?à?¤?®?à?¤?¨?à?¥???à?¤?¤?à?¥???à?¤?°?à?¥?‡?à?¤?£?
?à?¤?---?à?¥?ƒ?à?¤?¹?à?¥???à?¤?£?à?¥?€?à?¤?¯?à?¥???à?¤?¸?à?¥???à?¤?¤?à?¥?‡?
?à?¤?¤?à?¥??? ?à?¤?¤?à?¤?¾?à?¤?¨?à?¥???
?à?¤?•?à?¥???à?¤?¶?à?¤?¾?à?¤?¨?à?¥??? ?à?¥?¥?

?â?€?œ?à?¤?§?à?¤?°?à?¥???à?¤?®?à?¥?‹?à?¤?½?à?¤?¸?à?¥???à?¤?®?à?¤?¿?
?à?¤?µ?à?¤?¿?à?¤?¶?à?¤?¿?à?¤?°?à?¤?¾?à?¤?œ?à?¤?¾?
?à?¤?ª?à?¥???à?¤?°?à?¤?¤?à?¤?¿?à?¤?·?à?¥???à?¤?~?à?¤?¿?à?¤?¤?â?€???
?à?¤?‡?à?¤?¤?à?¤?¿?
?à?¤?¸?à?¥?‹?à?¤?¤?à?¥???à?¤?°?à?¤?¾?à?¤?®?à?¤?£?à?¥?€?à?¤?ª?à?¥???à?¤?°?à?¤?•?à?¤?°?à?¤?£?à?¤?ª?à?¤?~?à?¤?¿?à?¤?¤?à?¥?‡?à?¤?¨?

?à?¤?®?à?¤?¨?à?¥???à?¤?¤?à?¥???à?¤?°?à?¥?‡?à?¤?£?
?à?¤?¦?à?¥?‡?à?¤?µ?à?¤?¿?à?¤?•?à?¤?¾?
?à?¤?¬?à?¥???à?¤?°?à?¤?¾?à?¤?¹?à?¥???à?¤?®?à?¤?£?à?¤?¾?
?à?¤?†?à?¤?¸?à?¤?¨?à?¥?‡?
?à?¤?ª?à?¥???à?¤?°?à?¤?¦?à?¤?¤?à?¥???à?¤?¤?à?¤?¾?à?¤?¨?à?¥???
?à?¤?•?à?¥???à?¤?¶?à?¤?¾?à?¤?¨?à?¥???
?à?¤?---?à?¥?ƒ?à?¤?¹?à?¤?¾?à?¤?¯?à?¥???à?¤?ƒ? ?=?
?à?¤?\ldots{}?à?¤?™?à?¥???à?¤?---?à?¥?€?à?¤?•?à?¥???à?¤?°?à?¥???à?¤?¯?à?¥???à?¤?ƒ?

?à?¤?¸?à?¥???à?¤?µ?à?¥?€?à?¤?•?à?¥???à?¤?°?à?¥???à?¤?¯?à?¥???à?¤?°?à?¤?¿?à?¤?¤?à?¤?¿?
?à?¤?¯?à?¤?¾?à?¤?µ?à?¤?¤?à?¥??? ?,? ?à?¤?¨? ?à?¤?¤?à?¥???
?à?¤?¹?à?¤?¸?à?¥???à?¤?¤?à?¥?‡?à?¤?¨?
?à?¤?---?à?¥?ƒ?à?¤?¹?à?¥???à?¤?£?à?¥?€?à?¤?¯?à?¥???à?¤?ƒ?
?à?¤?¤?à?¤?¸?à?¥???à?¤?¯?
?à?¤?ª?à?¥???à?¤?°?à?¤?¤?à?¤?¿?à?¤?·?à?¤?¿?à?¤?¦?à?¥???à?¤?§?à?¤?¤?à?¥???à?¤?µ?à?¤?¾?à?¤?¤?à?¥???
?à?¥?¤?

?à?¤?•?à?¥?‡?à?¤?š?à?¤?¿?à?¤?¤?à?¥???à?¤?¤?à?¥???
?à?¤?§?à?¤?°?à?¥???à?¤?®?à?¤?¾?à?¤?¸?à?¥?€?à?¤?¤?à?¤?¿?
?à?¤?ª?à?¤?¾?à?¤?~?à?¤?®?à?¤?¾?à?¤?¹?à?¥???à?¤?¸?à?¥???à?¤?¤?à?¤?¤?à?¥???à?¤?°?
?à?¤?§?à?¤?°?à?¥???à?¤?®?à?¤?¾?à?¤?¸?à?¤?¿?
?à?¤?¸?à?¥???à?¤?§?à?¤?°?à?¥???à?¤?®?à?¥?‡?à?¤?¤?à?¥???à?¤?¯?à?¤?¨?à?¥?‡?à?¤?¨?
?à?¤?ª?à?¥???à?¤?°?à?¤?¾?à?¤?µ?à?¤?°?à?¥???à?¤?£?à?¤?¿?â?€?¢?

?à?¤?•?à?¥?‡?à?¤?¨? ?à?¤?®?à?¤?¨?à?¥???à?¤?¤?à?¥???à?¤?°?à?¥?‡?à?¤?£?
?à?¤?---?à?¥?ƒ?à?¤?¹?à?¥???à?¤?£?à?¥?€?à?¤?¯?à?¥???à?¤?°?à?¤?¿?à?¤?¤?à?¤?¿?
?à?¥?¤?

?à?¤?•?à?¥?‡?à?¤?š?à?¤?¿?à?¤?¤?à?¥???à?¤?¤?à?¥??? ?â?€?œ?
?à?¤?§?à?¤?°?à?¥???à?¤?®?à?¤?¾?à?¤?¸?à?¤?¿?
?à?¤?¸?à?¥???à?¤?§?à?¤?°?à?¥???à?¤?®?à?¤?¾?
?à?¤?®?à?¥?‡?à?¤?¨?à?¥???à?¤?¯?à?¤?¸?à?¥???à?¤?®?à?¥?‡?
?à?¤?¨?à?¥?ƒ?à?¤?•?à?¥???à?¤?•?à?¤?¾?à?¤?¨?à?¤?¿?
?à?¤?§?à?¤?¾?à?¤?°?à?¤?¯? ?à?¤?¬?à?¥???à?¤?°?à?¤?¹?à?¥???à?¤?®?
?à?¤?§?à?¤?¾?à?¤?°?à?¤?¯?

?à?¤?•?à?¥???à?¤?·?à?¤?¤?à?¥???à?¤?°?à?¤?‚? ?à?¤?§?à?¤?¾?à?¤?°?à?¤?¯?
?à?¤?µ?à?¤?¿?à?¤?¶?à?¤?‚?
?à?¤?§?à?¤?¾?à?¤?°?à?¤?¯?à?¥?‡?à?¤?¤?à?¥???à?¤?¯?à?¤?¨?à?¥???à?¤?¤?à?¤?‚?
?à?¤?®?à?¤?¨?à?¥???à?¤?¤?à?¥???à?¤?°?à?¤?‚?
?à?¤?ª?à?¤?~?à?¤?¨?à?¥???à?¤?¤?à?¤?¿? ?à?¥?¤? ?à?¤?¤?à?¤?¨?à?¥???à?¤?¨?
?\textless{}?/?s?p?a?n?\textgreater{}?

?à?¥?¤?\textless{}?/?s?p?a?n?\textgreater{}?

? ?à?¤?§?à?¤?°?à?¥???à?¤?®?à?¤?¾?à?¤?¸?à?¤?¿?
?à?¤?¸?à?¥???à?¤?§?à?¤?°?à?¥???à?¤?®?à?¥?‡?

?à?¤?•?à?¤?¾?à?¤?¤?à?¥???à?¤?°?
?à?¤?µ?à?¤?¾?à?¤?•?à?¥???à?¤?¯?à?¤?¸?à?¥???à?¤?¯?
?à?¤?ª?à?¤?°?à?¤?¿?à?¤?¸?à?¤?®?à?¤?¾?à?¤?ª?à?¥???à?¤?¤?à?¤?¤?à?¥???à?¤?µ?à?¤?¾?à?¤?¤?à?¥???
?à?¥?¤? ?à?¤?¸?à?¤?®?à?¥???à?¤?š?à?¥???à?¤?š?à?¤?¯?à?¤?¸?à?¥???à?¤?¯?
?à?¤?š?à?¤?¾?à?¤?¨?à?¥???à?¤?¯?à?¤?¾?à?¤?¯?à?¥???à?¤?¯?à?¤?¤?à?¥???à?¤?µ?à?¤?¾?à?¤?¤?à?¥???
?à?¥?¤? ?à?¤?µ?à?¤?¿?.?

?à?¤?¨?à?¤?¿?à?¤?¯?à?¥?‹?à?¤?---?à?¤?¶?à?¥???à?¤?š?
?à?¤?§?à?¤?°?à?¥???à?¤?®?à?¤?¾?à?¤?¸?à?¥?€?à?¤?¤?à?¥???à?¤?¯?à?¥???à?¤?¤?à?¥???à?¤?•?à?¥???à?¤?°?à?¤?¾?à?¤?®?à?¤?¤?à?¥???à?¤?¯?à?¥???à?¤?¤?à?¥???à?¤?¤?à?¤?°?à?¤?ª?à?¥?‚?à?¤?°?à?¥???à?¤?µ?à?¤?¾?à?¤?°?à?¥???à?¤?¦?à?¥???à?¤?§?
?à?¤?®?à?¤?®?à?¥?‡?à?¤?¨?à?¥???à?¤?¯?à?¤?¸?à?¥???à?¤?®?à?¥?‡?à?¤?¤?à?¤?¿?
?à?¤?--?à?¤?°?à?¥?‡? ?à?¤?•?à?¤?°?à?¥?‹?à?¤?¤?à?¥?€?à?¤?¤?à?¤?¿?

?à?¤?ª?à?¥?ƒ?à?¤?¥?à?¤?•?à?¥??? ?à?¤?ª?à?¥?ƒ?à?¤?¥?à?¤?•?à?¥???
?à?¤?¶?à?¥???à?¤?°?à?¥???à?¤?¤?à?¥?Œ? ?à?¤?¦?à?¥?ƒ?à?¤?·?à?¥???à?¤?Ÿ?
?à?¤?‡?à?¤?¤?à?¥???à?¤?¯?à?¤?¤?à?¥?‹?à?¤?½?à?¤?ª?à?¤?¿?
?à?¤?®?à?¤?¨?à?¥???à?¤?¤?à?¥???à?¤?°?à?¤?­?à?¥?‡?à?¤?¦?à?¤?ƒ? ?à?¥?¤?

?à?¤?\ldots{}?à?¤?¤?à?¥???à?¤?°?à?¤?¾?à?¤?¸?à?¤?¨?à?¤?¾?à?¤?¦?à?¤?¿?
?à?¤?¦?à?¥?ˆ?à?¤?µ?à?¥?‹?à?¤?ª?à?¤?š?à?¤?°?à?¤?£?à?¤?®?à?¥???à?¤?¦?à?¤?™?à?¥???à?¤?®?à?¥???à?¤?--?à?¥?‡?à?¤?¨?,?
?à?¤?ª?à?¤?¿?à?¤?¤?à?¥???à?¤?°?à?¥???à?¤?¯?à?¤?‚?
?à?¤?¦?à?¤?•?à?¥???à?¤?·?à?¤?¿?à?¤?£?à?¤?¾?à?¤?®?à?¥???à?¤?--?à?¥?‡?à?¤?¨?à?¥?‡?à?¤?¤?à?¥???à?¤?¯?à?¤?¾?à?¤?¹?-?-?

?à?¤?‹?à?¤?¤?à?¥???à?¤?ƒ? ?à?¥?¤?

?à?¤?‰?à?¤?¦?à?¤?™?à?¥???à?¤?®?à?¥???à?¤?--?à?¤?¸?à?¥???à?¤?¤?à?¥???
?à?¤?¦?à?¥?‡?à?¤?µ?à?¤?¾?à?¤?¨?à?¤?¾?à?¤?‚?
?à?¤?ª?à?¤?¿?à?¤?¤?à?¥?„?à?¤?£?à?¤?¾?à?¤?‚?
?à?¤?¦?à?¤?•?à?¥???à?¤?·?à?¤?¿?à?¤?£?à?¤?¾?à?¤?®?à?¥???à?¤?--?à?¤?ƒ?
?à?¥?¤?

?à?¤?†?à?¤?¸?à?¤?¨?à?¤?¾?à?¤?°?à?¥???à?¤?˜?à?¥???à?¤?¯?à?¤?¾?à?¤?¦?à?¤?¿?à?¤?•?à?¤?‚?
?à?¤?¦?à?¤?¦?à?¥???à?¤?¯?à?¤?¾?à?¤?¤?à?¥???
?à?¤?¸?à?¤?¤?à?¥???à?¤?¯?à?¤?®?à?¥?‡?à?¤?µ?
?à?¤?¯?à?¤?¥?à?¤?¾?à?¤?µ?à?¤?¿?à?¤?§?à?¤?¿? ?à?¥?¥? ?à?¤?‡?à?¤?¤?à?¤?¿?
?à?¥?¤?

?à?¤?\ldots{}?à?¤?¤?à?¥???à?¤?°?à?¤?¾?à?¤?¸?à?¤?¨?à?¤?¦?à?¤?¾?à?¤?¨?à?¤?µ?à?¤?¾?à?¤?•?à?¥???à?¤?¯?à?¥?‡?
?à?¤?ª?à?¤?¿?à?¤?¤?à?¥???à?¤?°?à?¤?¾?à?¤?¦?à?¤?¿?à?¤?ª?à?¤?¦?à?¥?‡?
?â?€?œ?
?à?¤?\ldots{}?à?¤?•?à?¥???à?¤?·?à?¤?¯?à?¥???à?¤?¯?à?¤?¾?à?¤?¸?à?¤?¨?à?¤?¯?à?¥?‹?à?¤?ƒ?
?à?¤?·?à?¤?·?à?¥???à?¤?~?à?¥?€? ?â?€???
?à?¤?¤?à?¥???à?¤?¯?à?¤?¾?à?¤?¦?à?¤?¿?à?¤?µ?

?à?¤?µ?à?¤?¨?à?¥?‡?à?¤?¨? ?à?¤?·?à?¤?·?à?¥???à?¤?~?à?¥?€?,? ?â?€?œ?
?à?¤?š?à?¤?¤?à?¥???à?¤?°?à?¥???à?¤?¥?à?¥?€?
?à?¤?š?à?¤?¾?à?¤?²?à?¤?¨?à?¥?‡? ?à?¤?®?à?¤?¤?à?¤?¾?â?€???
?à?¤?‡?à?¤?¤?à?¥???à?¤?¯?à?¤?¨?à?¥?‡?à?¤?¨?
?à?¤?š?à?¤?¤?à?¥???à?¤?°?à?¥???à?¤?¥?à?¥?€? ?à?¤?µ?à?¤?¾? ?à?¥?¤?
?à?¤?¬?à?¥???à?¤?°?à?¤?¾?à?¤?¹?à?¥???à?¤?®?à?¤?£?à?¤?¾?
?à?¤?\ldots{}?à?¤?ª?à?¤?¿?

?à?¤?†?à?¤?¸?à?¤?¨?à?¤?¾?à?¤?¦?à?¤?¿? ?à?¤?²?à?¤?§?à?¥???à?¤?µ?à?¤?¾?
?à?¤?¸?à?¥???à?¤?µ?à?¤?¾?à?¤?¸?à?¤?¨?à?¤?‚?
?à?¤?¸?à?¥???à?¤?---?à?¤?¨?à?¥???à?¤?§?
?à?¤?‡?à?¤?¤?à?¥???à?¤?¯?à?¤?¾?à?¤?¦?à?¤?¿?
?à?¤?¬?à?¥???à?¤?°?à?¥?‚?à?¤?¯?à?¥???à?¤?ƒ? ?à?¥?¤? ?à?¤?¤?à?¤?¥?à?¤?¾?
?à?¤?š? ?à?¤?¨?à?¤?¾?à?¤?---?à?¤?°?à?¤?--?à?¤?£?à?¥???à?¤?¡?à?¥?‡?
?à?¥?¤?\textless{}?/?s?p?a?n?\textgreater{}?

?
?\textless{}?/?s?p?a?n?\textgreater{}?\textless{}?/?p?\textgreater{}?\textless{}?/?b?o?d?y?\textgreater{}?\textless{}?/?h?t?m?l?\textgreater{}?

{आवाहनमकारः । २०१}{\\
गन्धमाल्यासनादीनां प्रदानेषु द्विजोत्तमः ।\\
सुगन्धोऽस्तु सुपोऽस्तु चेत्यादि समुदाहरेत् ॥\\
एवमासनं दत्वा पुनर्ब्राह्मणहस्ते जल दद्यात् ।\\
तथा चावलायनगृह्यसूत्रे ।\\
अपः प्रदाय दर्भान् द्विगुणभुनानासनं प्रदायापः प्रदायेति,\\
आसनदानानन्तरं पादयोरधस्तादपि कुशान् दद्यादित्युक्तम् ।\\
भविष्यपुराणे ।\\
कुशान्निधाय तत्र द्वौ ब्राह्मणावुपवेशयेत् ।\\
पादयोश्च कुशान् दद्यादिति ।\\
यद्यपि चात्र ब्राह्मणपादयोः कुशदानं दैव एवोक्तम् । तथापि\\
दक्षिणाग्रास्त था दर्भा विष्टरेषु निवेशिताः ।\\
पादयोश्चैव विप्राणां येष्वन्त्रमुपभुञ्जते ॥\\
इति सामान्यतो महाभारतोकः पित्र्येऽपि ज्ञेयम् ।\\
एव दैवब्राह्मणेभ्य आसन दत्त्वा पितृब्राह्मणेभ्य आसनं दद्यात् ।\\
तत्र विशेषो बृहस्पतिनोक्तः ।\\
आसने चार्घदाने च पिण्डदानावनेजने ।\\
सम्बन्धनामगोत्राणि यथाईमनुकीर्तयेत् \textbar{}\textbar{}\\
तथा च मात्स्ये ।\\
आसनेषूपक्लनेषु दर्भवत्सु विधानतः ।\\
उपस्पृष्टोदकान् विप्रानुपवेश्य निमन्त्रयत् ॥ इति ।\\
{[}नि {]} मन्त्रणवत् कार्यं, तथा च स एव यथा प्रथममेवं द्वितीयं\\
तृतीयं चेति ।\\
अथावाहनम् ।\\
तत्र कात्यायनः ।\\
आसनेषु दर्भानास्तीर्य विश्वान् देवानावादयिष्य इति पृच्छति,\\
`` आवाहये '' त्यनुज्ञातो `` विश्वेदेवास आगते '' त्यनयाऽऽवाह्याव\\
कीर्य `` विश्वेदेवाः शृणुतेम'' मिति जपित्वा पितॄनावाइयिष्य इति\\
पृच्छति, आवाहयेत्यनुज्ञात `` उशन्तस्त्वे '' त्यनयावा ह्यावकीर्य `` आ\\
यन्तु न'' इति जपित्वेति ।\\
ब्राह्मणोपवेशानन्तरं यज्ञोपवीती उदङ्मुखो विश्वान् दे-\\
वानावाहयिष्य इति पृच्छति पङ्किमूर्धन्यं सर्वान् वा । `` आ\\
वी० मि २६ 

{२०२ वीरमित्रोदयस्य श्राद्धप्रकाशे-}{\\
वाहये''ति पङ्किमूर्धन्येन सर्वैर्वानुज्ञातो यजमानो `` विश्वेदेवास\\
आगते'' त्यनयर्चावाह्यावकीर्य भूमौ प्रदक्षिणं ब्राह्मणदक्षिणपाद-\\
प्रभृतिशिरः पर्यन्तं प्रकीर्य, स्मृत्यन्तराद् यवान् । `` विश्वेदेवाः
शृणु·\\
तेम ''मति जपित्वा प्राचीनावीनी दक्षिणाभिमुखो भूत्वोदङ्मुख\\
पितृब्राह्मणान् सर्वानेकं वा पङ्गिमूर्धन्य पितॄनावाहयिष्य इति पृ\\
च्छति, `` आवाहये''ति पामूर्धन्येन सर्वैर्वानुज्ञातो यजमान "उश\\
न्तस्त्व"त्यनयर्चाऽवाह्यावकीर्याऽप्रदक्षिणं तिला, "मायन्तु न"\\
इति जपेदिति सूत्रार्थः ।\\
अत्र यद्यपि सामान्यतोऽवकिरणमुक्तं तथापि दैवे यवैः, पित्र्ये\\
तिलैरिति ज्ञेयम् ।\\
तथा च याज्ञवल्क्य. ।\\
आवाहयेत्यनुज्ञातो विश्वेदेवास इत्यृचा । यवैरन्ववकीर्येत्यादि ।\\
भविष्यपुराणेऽपि ।\\
आवाहयेद्यवैर्देवानपसव्यं तिलैः पितॄन् ।\\
अतश्च यत् कर्केण दैवेऽपि तिलैरेवावकरणं विशेषानुपदेशा\\
दित्युक्त, तव परास्तम् । अपसव्यमित्यनेन पितृष्वप्रादक्षिण्यविधानात्\\
प्रादक्षिण्येन देवानामिति सिद्धं भवति । तथा च यमेन दैवे सम-\\
न्त्रकं प्रादक्षिण्येन यवविकरणमुक्तम् ।\\
यवहस्तस्ततो देवान् विज्ञाप्यावाहनं प्रति ।\\
आवाहयेदनुज्ञानो विश्वेदेवास इत्यृचा ॥\\
विश्वेदवाः शृणुतेति मन्त्रं ब्रह्मावतोऽक्षतान् ।\\
ओषधय इति मन्त्रेण विकिरेत्तु प्रदक्षिणम् ॥\\
प्रदक्षिणम् = पादादिमस्तकान्तमित्यर्थः ।\\
तथा च\\
भर्तृयज्ञः ।\\
विश्वेदेवास आगत मन्त्रेणानेन पार्थिव ।\\
तेषामावाहन कार्यमक्षतैश्च शिरोऽन्ततः इति ॥\\
दक्षिणपादप्रभृतिशिरःपर्यन्तमित्यर्थः । अयमुक्तो यवाक्षतारो-\\
पणप्रकार: प्रचेतसा यवस्थाने पुष्पारोपणमुदाहृत्य प्रदर्शितः, ``पा-\\
दप्रभृतिमूर्द्धान्तं देवानां पुष्पपूजन'' मिति । मत्स्यपुराणे तु
विकरणे-\\
मन्त्रान्तरमुक्तम् ।

{ }{ आवाहनप्रकारः । २०३}{\\
विश्वेदेवास इत्याभ्यामावाह्य विकिरेद्यवान् ।\\
( १ ) यवोऽसि धान्यराजस्त्वं वारुणो (२) मधुमिश्रितः ॥\\
निर्णोदः सर्वपापानां (३) पवित्रमृषिसयुतम् । इति ।\\
अत्र कात्यायनसूत्रे आसनेषु दर्भानास्तीर्य विश्वान् देवानावा-\\
हयिष्य इति दर्भास्तरणावाहनयोरानन्तर्यविधानाद् यदासनदानो.\\
त्तरकालीनं तृतीयं निमन्त्रणमाम्नातं तदापस्तम्बविषयम् । हेमाद्रिस्तु\\
दर्भानास्तीर्येत्यनेन दर्भासनदानस्य पूर्वकालतायामभिधीयमाना\\
यामपि स्मृत्यन्तरानुसारान्मध्ये निमन्त्रणानुष्ठानऽपि न पूर्वकाल-\\
ताश्रुतिविरोधभीः, इत्याह ।\\
एतदावाहनं च निरङ्गुष्ठं ब्राह्मणदक्षिणहस्तं गृहीत्वा कर्त्तव्यम् ।\\
तदुक्त ब्रह्मपुराणे ।\\
देवानावाहयिष्ये तत्प्राहुरावाहयेति च ।\\
निरङ्गुष्ठं गृहीत्वा तु विश्वान् देवान् समाह्वयेत् ॥\\
अपरे तु विप्राङ्गुष्ठं गृहीत्वेति पाठमाहुः ।\\
अत्र विश्वेषा देवानां पुरुरवार्द्रवादिनामाद्यज्ञाने विशेषमाह -\\
शङ्खः ।\\
नाम चैव तथोत्पत्तिं न विदुर्ये द्विजातयः ।\\
श्लोकमेतं पठेयुस्ते ब्राह्मणाना समीपतः ॥\\
आगच्छन्तु महाभागा इति श्लोकात्मकं मन्त्रं वैश्वदेविकं ब्रा-\\
ह्मणसमीपे `` विश्वेदेवाः शृणुवेम '' मिति जपानन्तरं जपेयुरित्यर्थः ।\\
तथाच पुराणं-\\
ततो मन्त्री जपेन्मौनी विश्वेदेवास आगत ।\\
विश्वेदेवाः शृणुतेति द्वितीयं तदनन्तरम् }{॥}{\\
तृतीय तु जपन्मन्त्रमागच्छान्त्वित्यतः परम् । इति ।\\
विश्वेदेवास आगतेत्यनया ऋचा प्रथममावाह्य तदनन्तरं `` विश्वेदे-\\
वाः शृणुते ''ति द्वितीयं मन्त्रं जपित्वा अतः परं ``आगच्छन्तु महाभा\\
गा'' इति तृतीयं मन्त्रं जपेदित्यन्वयः । एतच्च प्रागुक्तम् । अत्र वैश्व

% \begin{center}\rule{0.5\linewidth}{0.5pt}\end{center}

% \begin{center}\rule{0.5\linewidth}{0.5pt}\end{center}

?

? ?b?o?d?y?\{? ?w?i?d?t?h?:? ?2?1?c?m?;? ?h?e?i?g?h?t?:? ?2?9?.?7?c?m?;?
?m?a?r?g?i?n?:? ?3?0?m?m? ?4?5?m?m? ?3?0?m?m? ?4?5?m?m?;? ?\}?
?\textless{}?/?s?t?y?l?e?\textgreater{}?\textless{}?!?D?O?C?T?Y?P?E?
?H?T?M?L? ?P?U?B?L?I?C? ?"?-?/?/?W?3?C?/?/?D?T?D? ?H?T?M?L?
?4?.?0?/?/?E?N?"?
?"?h?t?t?p?:?/?/?w?w?w?.?w?3?.?o?r?g?/?T?R?/?R?E?C?-?h?t?m?l?4?0?/?s?t?r?i?c?t?.?d?t?d?"?\textgreater{}?
?

?

?

?

? ?p?,? ?l?i? ?\{? ?w?h?i?t?e?-?s?p?a?c?e?:? ?p?r?e?-?w?r?a?p?;? ?\}?
?\textless{}?/?s?t?y?l?e?\textgreater{}?\textless{}?/?h?e?a?d?\textgreater{}?

? ?

?

?à?¥?¨?à?¥?¦?à?¥?ª? ? ? ? ? ? ? ? ? ? ? ? ? ? ? ? ? ? ?
?à?¤?µ?à?¥?€?à?¤?°?à?¤?®?à?¤?¿?à?¤?¤?à?¥???à?¤?°?à?¥?‹?à?¤?¦?à?¤?¯?à?¤?¸?à?¥???à?¤?¯?
?à?¤?¶?à?¥???à?¤?°?à?¤?¾?à?¤?¦?à?¥???à?¤?§?à?¤?ª?à?¥???à?¤?°?à?¤?•?à?¤?¾?à?¤?¶?à?¥?‡?-?\textless{}?/?s?p?a?n?\textgreater{}?

?

?à?¤?¦?à?¥?‡?à?¤?µ?à?¤?¿?à?¤?•?à?¤?¬?à?¥???à?¤?°?à?¤?¾?à?¤?¹?à?¥???à?¤?®?à?¤?£?à?¤?¾?à?¤?¨?à?¥?‡?à?¤?•?à?¤?¤?à?¥???à?¤?µ?à?¥?‡?à?¤?½?à?¤?ª?à?¤?¿?
?à?¤?¨?
?à?¤?ª?à?¥???à?¤?°?à?¤?¤?à?¤?¿?à?¤?¬?à?¥???à?¤?°?à?¤?¾?à?¤?¹?à?¥???à?¤?®?à?¤?£?à?¤?®?à?¤?¾?à?¤?µ?à?¤?¾?à?¤?¹?à?¤?¨?à?¤?¾?à?¤?µ?à?¥?ƒ?à?¤?¤?à?¥???à?¤?¤?à?¤?¿?à?¤?ƒ?,?
?à?¤?¸?à?¤?•?à?¥?ƒ?à?¤?¦?à?¤?¾?à?¤?µ?à?¤?¾?à?¤?¹?-?

?à?¤?¨?à?¥?‡?à?¤?¨?à?¥?ˆ?à?¤?µ?à?¤?¾?à?¤?¨?à?¥?‡?à?¤?•?à?¤?¬?à?¥???à?¤?°?à?¤?¾?à?¤?¹?à?¥???à?¤?®?à?¤?£?à?¤?¾?à?¤?§?à?¤?¿?à?¤?·?à?¥???à?¤?~?à?¤?¾?à?¤?¨?à?¥?‡?
?à?¤?¦?à?¥?‡?à?¤?µ?à?¤?¤?à?¤?¾?à?¤?§?à?¥???à?¤?¯?à?¤?¾?à?¤?¸?à?¤?¸?à?¤?®?à?¥???à?¤?­?à?¤?µ?à?¤?¾?à?¤?¤?à?¥???
?\textbar{}?
?à?¤?†?à?¤?µ?à?¤?¾?à?¤?¹?à?¤?¨?à?¤?¾?à?¤?¦?à?¥?‚?à?¤?°?à?¥???à?¤?§?à?¥???à?¤?µ?à?¤?‚?
?à?¤?¤?à?¥???

?à?¤?ª?à?¥???à?¤?°?à?¤?¤?à?¤?¿?à?¤?¬?à?¥???à?¤?°?à?¤?¾?à?¤?¹?à?¥???à?¤?®?à?¤?£?à?¤?‚?
?à?¤?¦?à?¥?‡?à?¤?µ?à?¤?¤?à?¤?¾?à?¤?°?à?¤?¾?à?¤?§?à?¤?¨?à?¤?¾?à?¤?°?à?¥???à?¤?¥?à?¤?‚?
?à?¤?•?à?¥???à?¤?°?à?¤?¿?à?¤?¯?à?¤?¤?à?¥?‡?
?à?¤?¯?à?¤?µ?à?¤?¾?à?¤?°?à?¥?‹?à?¤?ª?à?¤?£?à?¤?¾?à?¤?¦?à?¤?¿?
?à?¤?¸?à?¤?¨?à?¥???à?¤?¨?à?¤?¿?à?¤?ª?à?¤?¤?à?¥???à?¤?¯?à?¥?‹?à?¤?ª?à?¤?•?à?¤?¾?à?¤?°?à?¤?•?à?¤?‚?

?à?¤?¤?à?¤?¤?à?¥???à?¤?¤?à?¤?¤?à?¥???à?¤?ª?à?¥???à?¤?°?à?¥?‹?à?¤?¡?à?¤?¾?à?¤?¶?à?¤?ª?à?¥???à?¤?°?à?¤?¥?à?¤?¨?à?¤?¾?à?¤?¦?à?¤?¿?à?¤?µ?à?¤?¤?à?¥???
?à?¤?ª?à?¥???à?¤?°?à?¤?¤?à?¤?¿?à?¤?¬?à?¥???à?¤?°?à?¤?¾?à?¤?¹?à?¥???à?¤?®?à?¤?£?à?¤?®?à?¤?¾?à?¤?µ?à?¤?°?à?¥???à?¤?¤?à?¤?¨?à?¥?€?à?¤?¯?à?¤?®?à?¥???,?
?à?¤?¸?à?¤?¨?à?¥???à?¤?¨?à?¤?¿?à?¤?ª?à?¤?¤?à?¥???à?¤?¯?à?¥?‹?à?¤?ª?à?¤?•?à?¤?¾?

?à?¤?°?à?¤?•?à?¥?‡?à?¤?·?à?¥???à?¤?µ?à?¤?¾?à?¤?µ?à?¥?ƒ?à?¤?¤?à?¥???à?¤?¤?à?¤?¿?à?¤?‚?
?à?¤?µ?à?¤?¿?à?¤?¨?à?¤?¾?
?à?¤?¬?à?¥???à?¤?°?à?¤?¾?à?¤?¹?à?¥???à?¤?®?à?¤?£?à?¤?¾?à?¤?¨?à?¥???à?¤?¤?à?¤?°?à?¥?‡?
?à?¤?•?à?¤?¾?à?¤?°?à?¥???à?¤?¯?à?¤?¾?à?¤?¸?à?¤?¿?à?¤?¦?à?¥???à?¤?§?à?¥?‡?à?¤?ƒ?
?à?¥?¤? ?â?€?œ?
?à?¤?µ?à?¤?¿?à?¤?¶?à?¥???à?¤?µ?à?¥?‡?à?¤?¦?à?¥?‡?à?¤?µ?à?¤?¾?à?¤?ƒ?
?à?¤?¶?à?¥?ƒ?à?¤?£?à?¥???

?à?¤?¤?à?¥?‡?â?€??? ?à?¤?¤?à?¤?¿?
?à?¤?®?à?¤?¨?à?¥???à?¤?¤?à?¥???à?¤?°?à?¤?œ?à?¤?ª?à?¤?¸?à?¥???à?¤?¤?à?¥???à?¤?µ?à?¤?¾?à?¤?µ?à?¤?¾?à?¤?¹?à?¤?¨?à?¤?¾?à?¤?‚?à?¤?¤?à?¤?°?à?¤?•?à?¤?¾?à?¤?²?à?¥?‡?
?à?¤?•?à?¥???à?¤?°?à?¤?¿?à?¤?¯?à?¤?®?à?¤?¾?à?¤?£?à?¥?‹?à?¤?½?à?¤?ª?à?¤?¿?
?à?¤?†?à?¤?°?à?¤?¾?à?¤?¦?à?¥???à?¤?ª?à?¤?•?à?¤?¾?à?¤?°?à?¤?•?

?à?¤?¤?à?¥???à?¤?µ?à?¤?¾?à?¤?¨?à?¥???à?¤?¨?à?¤?¾?à?¤?µ?à?¤?°?à?¥???à?¤?¤?à?¤?¤?à?¥?‡?
?à?¤?¸?à?¤?•?à?¥?ƒ?à?¤?œ?à?¥???à?¤?œ?à?¤?ª?à?¥?‡?à?¤?¨?à?¥?ˆ?à?¤?µ?à?¤?¾?à?¤?¦?à?¥?ƒ?à?¤?·?à?¥???à?¤?Ÿ?à?¤?¸?à?¤?¿?à?¤?¦?à?¥???à?¤?§?à?¥?‡?à?¤?ƒ?,?
?à?¤?‡?à?¤?¤?à?¤?¿?
?à?¤?¹?à?¥?‡?à?¤?®?à?¤?¾?à?¤?¦?à?¥???à?¤?°?à?¤?¿?à?¤?¸?à?¥???à?¤?®?à?¥?ƒ?à?¤?¤?à?¤?¿?à?¤?š?à?¤?¨?à?¥???à?¤?¦?à?¥???à?¤?°?à?¤?¿?à?¤?•?à?¤?¾?à?¤?•?à?¤?¾?à?¤?°?à?¥?Œ?
?à?¥?¤?

? ? ? ? ?à?¤?\ldots{}?à?¤?¨?à?¥???à?¤?¯?à?¥?‡? ?à?¤?¤?à?¥??? ?à?¥?¤?

?à?¤?¨?à?¤?¿?à?¤?°?à?¤?™?à?¥???à?¤?---?à?¥???à?¤?·?à?¥???à?¤?Ÿ?à?¤?‚?
?à?¤?---?à?¥?ƒ?à?¤?¹?à?¥?€?à?¤?¤?à?¥???à?¤?µ?à?¤?¾? ?à?¤?¤?à?¥???
?à?¤?µ?à?¤?¿?à?¤?¶?à?¥???à?¤?µ?à?¤?¾?à?¤?¨?à?¥???
?à?¤?¦?à?¥?‡?à?¤?µ?à?¤?¾?à?¤?¨?à?¥???
?à?¤?¸?à?¤?®?à?¤?¾?à?¤?¹?à?¥???à?¤?µ?à?¤?¯?à?¥?‡?à?¤?¦?à?¤?¿?à?¤?¤?à?¥???à?¤?¯?à?¤?¨?à?¥?‡?à?¤?¨?
?à?¤?¨?à?¤?¿?à?¤?°?à?¤?™?à?¥???à?¤?---?à?¥???à?¤?·?à?¥???à?¤?~?à?¤?---?à?¥???à?¤?°?

?à?¤?¹?à?¤?£?à?¤?µ?à?¤?¿?à?¤?¶?à?¤?¿?à?¤?·?à?¥???à?¤?Ÿ?à?¤?¾?
?à?¤?µ?à?¤?¾?à?¤?¹?à?¤?¨?à?¤?µ?à?¤?¿?à?¤?§?à?¤?¾?à?¤?¨?à?¤?¾?à?¤?¦?à?¥???
?à?¤?---?à?¥?ƒ?à?¤?¹?à?¥???à?¤?¯?à?¤?®?à?¤?¾?à?¤?£?à?¤?µ?à?¤?¿?à?¤?¶?à?¥?‡?à?¤?·?à?¤?¤?à?¥???à?¤?µ?à?¤?¾?à?¤?¦?à?¤?¾?à?¤?µ?à?¥?ƒ?à?¤?¤?à?¥???à?¤?¤?à?¤?¿?à?¤?°?à?¤?¿?à?¤?¤?à?¥???à?¤?¯?à?¤?¾?à?¤?¹?à?¥???à?¤?ƒ?
?à?¥?¤?

? ? ? ? ? ? ?à?¤?ª?à?¤?¿?à?¤?¤?à?¥?„?à?¤?£?à?¤?¾?à?¤?‚? ?à?¤?¤?à?¥???
?à?¤?†?à?¤?µ?à?¤?¾?à?¤?¹?à?¤?¨?à?¥?‡?
?à?¤?µ?à?¤?¿?à?¤?¶?à?¥?‡?à?¤?·?à?¥?‹?

?à?¤?®?à?¤?¾?à?¤?°?à?¥???à?¤?•?à?¤?£?à?¥???à?¤?¡?à?¥?‡?à?¤?¯?à?¤?ª?à?¥???à?¤?°?à?¤?¾?à?¤?£?à?¥?‡?
?à?¥?¤?

? ? ? ? ?à?¤?¦?à?¤?°?à?¥???à?¤?­?à?¤?¾?à?¤?¸?à?¥???à?¤?¤?à?¥???
?à?¤?¦?à?¥???à?¤?µ?à?¤?¿?à?¤?---?à?¥???à?¤?£?à?¤?¾?à?¤?¨?à?¥???
?à?¤?¦?à?¤?¤?à?¥???à?¤?µ?à?¤?¾?
?à?¤?¤?à?¥?‡?à?¤?­?à?¥???à?¤?¯?à?¥?‹?à?¤?½?à?¤?¨?à?¥???à?¤?œ?à?¥???à?¤?ž?à?¤?¾?à?¤?®?à?¤?µ?à?¤?¾?à?¤?ª?à?¥???à?¤?¯?
?à?¤?š? ?à?¥?¤?

? ? ? ?
?à?¤?®?à?¤?¨?à?¥???à?¤?¤?à?¥???à?¤?°?à?¤?ª?à?¥?‚?à?¤?°?à?¥???à?¤?µ?à?¤?‚?
?à?¤?ª?à?¤?¿?à?¤?¤?à?¥?„?à?¤?£?à?¤?¾?à?¤?‚? ?à?¤?¤?à?¥???
?à?¤?•?à?¥???à?¤?°?à?¥???à?¤?¯?à?¤?¾?à?¤?¦?à?¤?¾?à?¤?µ?à?¤?¾?à?¤?¹?à?¤?¨?à?¤?‚?
?à?¤?¬?à?¥???à?¤?§?à?¤?ƒ? ?à?¥?¥?

?à?¤?¬?à?¥???à?¤?°?à?¤?¹?à?¥???à?¤?®?à?¤?ª?à?¥???à?¤?°?à?¤?¾?à?¤?£?à?¥?‡?
?à?¥?¤?

? ? ? ?
?à?¤?ª?à?¤?¿?à?¤?¤?à?¥?„?à?¤?¨?à?¤?¾?à?¤?µ?à?¤?¾?à?¤?¹?à?¤?¯?à?¤?¾?à?¤?®?à?¥?€?à?¤?¤?à?¤?¿?
?à?¤?¸?à?¥???à?¤?µ?à?¤?¯?à?¤?®?à?¥???à?¤?•?à?¥???à?¤?¤?à?¥???à?¤?µ?à?¤?¾?
?à?¤?¸?à?¤?®?à?¤?¾?à?¤?¹?à?¤?¿?à?¤?¤?à?¤?ƒ? ?à?¥?¤?

? ? ? ?
?à?¤?†?à?¤?µ?à?¤?¾?à?¤?¹?à?¤?¯?à?¤?¸?à?¥???à?¤?µ?à?¥?‡?à?¤?¤?à?¤?¿?
?à?¤?ª?à?¤?°?à?¥?ˆ?à?¤?°?à?¥???à?¤?•?à?¥???à?¤?¤?à?¤?¸?à?¥???à?¤?¤?à?¥???à?¤?µ?à?¤?¾?à?¤?µ?à?¤?¾?à?¤?¹?à?¤?¯?à?¥?‡?à?¤?š?à?¥???à?¤?›?à?¥???à?¤?š?à?¤?¿?à?¤?ƒ?
?à?¥?¤?

?[?à?¤?ª?à?¤?¿?à?¤?¤?à?¤?°?à?¥?‹?]?
?à?¤?¦?à?¤?¿?à?¤?µ?à?¥???à?¤?¯?à?¤?¾? ?=? ?à?¤?µ?à?¤?¸?à?¥???
?à?¤?°?à?¥???à?¤?¦?à?¥???à?¤?°?à?¤?¾?à?¤?¾?à?¤?¦?à?¥?‡?à?¤?¤?à?¥???à?¤?¯?à?¤?¾?à?¤?ƒ?,?
?à?¤?®?à?¤?¾?à?¤?¨?à?¥???à?¤?·?à?¤?¾?à?¤?ƒ? ?=?
?à?¤?¯?à?¤?œ?à?¤?®?à?¤?¾?à?¤?¨?à?¤?¸?à?¥???à?¤?¯?
?à?¤?ª?à?¤?¿?à?¤?¤?à?¥???à?¤?°?à?¤?¾?à?¤?¦?à?¤?¯?à?¤?ƒ? ?\textbar{}?

?à?¤?®?à?¤?¾?à?¤?¨?à?¤?µ?à?¥?‡?
?à?¤?¶?à?¥???à?¤?°?à?¤?¾?à?¤?¦?à?¥???à?¤?§?à?¤?•?à?¤?²?à?¥???à?¤?ª?à?¥?‡?
?à?¤?¤?à?¥??? ?à?¤?ª?à?¤?¿?à?¤?¤?à?¥?„?à?¤?¨?à?¥???
?à?¤?ª?à?¤?¿?à?¤?¤?à?¤?¾?à?¤?®?à?¤?¹?à?¤?¾?à?¤?¨?à?¥???
?à?¤?ª?à?¥???à?¤?°?à?¤?ª?à?¤?¿?à?¤?¤?à?¤?¾?à?¤?®?à?¤?¹?à?¤?¾?à?¤?¨?à?¤?¾?à?¤?µ?à?¤?¾?à?¤?¹?à?¤?¯?à?¤?¿?à?¤?·?à?¥???à?¤?¯?à?¤?¾?à?¤?®?à?¥?€?à?¤?¤?à?¤?¿?

?à?¤?‰?à?¤?•?à?¥???à?¤?¤?à?¥???à?¤?µ?à?¤?¾?
?à?¤?¬?à?¥???à?¤?°?à?¤?¾?à?¤?¹?à?¥???à?¤?®?à?¤?£?à?¥?ˆ?à?¤?°?à?¤?¨?à?¥???à?¤?œ?à?¥???à?¤?ž?à?¤?¾?à?¤?¤?
?à?¤?‡?à?¤?¤?à?¤?¿? ?à?¤?‰?à?¤?•?à?¥???à?¤?¤?à?¤?®?à?¥??? ?à?¥?¤?

?à?¤?\ldots{}?à?¤?¤?à?¥???à?¤?°? ?à?¤?µ?à?¤?¿?à?¤?¶?à?¥?‡?à?¤?·?à?¤?ƒ?
?à?¤?¶?à?¥???à?¤?²?à?¥?‹?à?¤?•?à?¤?---?à?¥?‹?à?¤?­?à?¤?¿?à?¤?²?à?¥?‡?à?¤?¨?à?¥?‹?à?¤?•?à?¥???à?¤?¤?à?¤?ƒ?
?à?¥?¤?

? ? ? ? ?à?¤?†?à?¤?µ?à?¤?¾?à?¤?¹?à?¤?¨?à?¥?‡?
?à?¤?½?à?¤?®?à?¥???à?¤?•?à?¤?---?à?¥?‹?à?¤?¤?à?¥???à?¤?°?à?¤?¾?à?¤?¨?à?¤?¸?à?¥???à?¤?®?à?¤?¤?à?¥???à?¤?ª?à?¤?¿?\textless{}?/?s?p?a?n?\textgreater{}?

?à?¤?¤?à?¥?„?à?¤?¨?à?¥???à?¤?ª?à?¤?¿?\textless{}?/?s?p?a?n?\textgreater{}?

?à?¤?¤?à?¤?¾?à?¤?®?à?¤?¹?à?¤?¾?à?¤?¨?à?¥??? ?à?¥?¤?

? ? ? ?
?à?¤?ª?à?¥???à?¤?°?à?¤?ª?à?¤?¿?à?¤?¤?à?¤?¾?à?¤?®?à?¤?¹?à?¤?¾?à?¤?¨?à?¥???
?à?¤?µ?à?¤?¿?à?¤?ª?à?¥???à?¤?°?à?¥?‡?à?¤?¨?à?¥???à?¤?¦?à?¥???à?¤?°?à?¤?¶?à?¤?°?à?¥???à?¤?®?à?¤?£?à?¥?‹?à?¤?½?à?¤?¥?
?à?¤?­?à?¤?µ?à?¥?‡?à?¤?¤?à?¥???à?¤?¤?à?¤?¦?à?¤?¾?
?\textless{}?/?s?p?a?n?\textgreater{}?

?à?¥?¥?\textless{}?/?s?p?a?n?\textgreater{}?

?

?à?¤?ª?à?¤?¿?à?¤?¤?à?¥???à?¤?°?à?¤?¾?à?¤?µ?à?¤?¾?à?¤?¹?à?¤?¨?à?¤?°?à?¥?‚?à?¤?ª?à?¥?‡?
?à?¤?ª?à?¤?¦?à?¤?¾?à?¤?°?à?¥???à?¤?¥?à?¥?‡?à?¤?½?à?¤?¨?à?¥???à?¤?·?à?¥???à?¤?~?à?¥?€?à?¤?¯?à?¤?®?à?¤?¾?à?¤?¨?à?¥?‡?
?à?¤?¤?à?¤?¦?à?¤?™?à?¥???à?¤?---?à?¤?­?à?¥?‚?à?¤?¤?à?¤?ª?à?¥???à?¤?°?à?¤?¯?à?¥?‹?à?¤?---?à?¤?µ?à?¤?¾?à?¤?•?à?¥???à?¤?¯?à?¥?‡?à?¤?½?à?¤?®?à?¥???à?¤?•?à?¤?---?à?¥?‹?-?

?à?¤?¤?à?¥???à?¤?°?à?¤?¾?à?¤?¨?à?¤?¸?à?¥???à?¤?®?à?¤?¤?à?¥???à?¤?ª?à?¤?¿?à?¤?¤?à?¥?ƒ?à?¤?¨?à?¤?®?à?¥???à?¤?•?à?¤?¶?à?¤?°?à?¥???à?¤?®?à?¤?£?
?à?¤?‡?à?¤?¤?à?¥???à?¤?¯?à?¥?‡?à?¤?¤?à?¤?¤?à?¥???à?¤?ª?à?¤?¦?à?¤?œ?à?¤?¾?à?¤?¤?
?à?¤?­?à?¤?µ?à?¥?‡?à?¤?¤?à?¥??? ?-?
?à?¤?‰?à?¤?š?à?¥???à?¤?š?à?¤?¾?à?¤?°?à?¤?¯?à?¥?‡?à?¤?¦?à?¤?¿?à?¤?¤?à?¥???à?¤?¯?à?¤?°?à?¥???à?¤?¥?à?¤?ƒ?
?à?¥?¤?

?à?¤?¬?à?¥???à?¤?°?à?¤?¹?à?¥???à?¤?®?à?¤?ª?à?¥???à?¤?°?à?¤?¾?à?¤?£?à?¥?‡?
?à?¤?¤?à?¥??? ?à?¤?µ?à?¤?¿?à?¤?¶?à?¥?‡?à?¤?·?à?¤?ƒ? ?à?¥?¤?

? ? ? ? ? ? ?à?¤?ª?à?¤?¿?\textless{}?/?s?p?a?n?\textgreater{}?

?à?¤?¤?à?¥?„?\textless{}?/?s?p?a?n?\textgreater{}?

?à?¤?¨?à?¤?¾?à?¤?µ?à?¤?¾?à?¤?¹?à?¤?¯?à?¤?¿?à?¤?·?à?¥???à?¤?¯?à?¥?‡?à?¤?½?à?¤?¹?à?¤?‚?
?à?¤?¶?à?¥?‡?à?¤?·?à?¤?¾?à?¤?¨?à?¥???
?à?¤?µ?à?¤?¿?à?¤?ª?à?¥???à?¤?°?à?¤?¾?à?¤?¨?à?¥???
?à?¤?µ?à?¤?¦?à?¥?‡?à?¤?¤?à?¥???à?¤?¤?à?¤?¤?à?¤?ƒ? ?à?¥?¤?

? ? ? ? ?
?à?¤?†?à?¤?µ?à?¤?¾?à?¤?¹?à?¤?¯?à?¤?¸?à?¥???à?¤?µ?à?¥?‡?à?¤?¤?à?¥???à?¤?¯?à?¥???à?¤?•?à?¥???à?¤?¤?à?¤?¸?à?¥???à?¤?¤?à?¥?ˆ?à?¤?ƒ?
?à?¤?¸?à?¤?¾?à?¤?µ?à?¤?§?à?¤?¾?à?¤?¨?à?¤?¾?
?à?¤?­?à?¤?µ?à?¤?¨?à?¥???à?¤?¤?à?¥???à?¤?µ?à?¤?¿?à?¤?¤?à?¤?¿? ?à?¥?¥?

? ?à?¤?¯?à?¥?‡?à?¤?·?à?¥???
?à?¤?µ?à?¥?ˆ?à?¤?¶?à?¥???à?¤?µ?à?¤?¦?à?¥?‡?à?¤?µ?à?¤?¿?à?¤?•?à?¤?µ?à?¤?¿?à?¤?ª?à?¥???à?¤?°?à?¥?‡?à?¤?·?à?¥???à?¤?µ?à?¤?¾?à?¤?µ?à?¤?¾?à?¤?¹?à?¤?¨?à?¤?‚?
?à?¤?•?à?¥?ƒ?à?¤?¤?à?¤?‚? ?à?¤?¤?à?¤?¦?à?¤?¿?à?¤?¤?à?¤?°?à?¥?‡?
?à?¤?¶?à?¥?‡?à?¤?·?à?¤?¾?à?¤?ƒ?
?à?¤?ª?à?¤?¿?à?¤?¤?à?¥???à?¤?°?à?¥???à?¤?¯?à?¤?µ?à?¤?¿?à?¤?ª?à?¥???à?¤?°?à?¤?¾?à?¤?‚?

?à?¤?‡?à?¤?¤?à?¥???à?¤?¯?à?¤?°?à?¥???à?¤?¥?à?¤?ƒ? ?à?¥?¤?
?à?¤?‡?à?¤?¹?à?¤?¾?à?¤?µ?à?¤?¾?à?¤?¹?à?¤?¨?à?¤?ª?à?¥???à?¤?°?à?¤?¶?à?¥???à?¤?¨?à?¥?‹?à?¤?¤?à?¥???à?¤?¤?à?¤?°?à?¤?¾?à?¤?¨?à?¤?¨?à?¥???à?¤?¤?à?¤?°?à?¤?‚?
?à?¤?¤?à?¤?¾?à?¤?¨?à?¥???
?à?¤?µ?à?¤?¿?à?¤?ª?à?¥???à?¤?°?à?¤?¾?à?¤?¨?à?¥???
?à?¤?­?à?¤?µ?à?¤?¨?à?¥???à?¤?¤?à?¤?ƒ?
?à?¤?¸?à?¤?¾?à?¤?µ?à?¤?§?à?¤?¾?à?¤?¨?à?¤?¾?

?à?¤?­?à?¤?µ?à?¤?¨?à?¥???à?¤?¤?à?¥???à?¤?µ?à?¤?¿?à?¤?¤?à?¤?¿?
?à?¤?¯?à?¤?œ?à?¤?®?à?¤?¾?à?¤?¨?à?¥?‹?
?à?¤?¬?à?¥???à?¤?°?à?¥?‚?à?¤?¯?à?¤?¾?à?¤?¤?à?¥??? ?à?¥?¤?
?à?¤?\ldots{}?à?¤?¨?à?¥?‡?à?¤?¨? ?à?¤?µ? ?à?¤?­?à?¤?µ?à?¤?¾?à?¤?®?à?¤?ƒ?
?à?¤?¸?à?¤?¾?à?¤?µ?à?¤?§?à?¤?¾?à?¤?¨?à?¤?¾? ?à?¤?‡?à?¤?¤?à?¤?¿?

?à?¤?µ?à?¤?¿?à?¤?ª?à?¥???à?¤?°?à?¤?¾?à?¤?£?à?¤?¾?à?¤?‚?
?à?¤?ª?à?¥???à?¤?°?à?¤?¤?à?¥???à?¤?¯?à?¥???à?¤?¤?à?¥???à?¤?¤?à?¤?°?à?¤?‚?
?à?¤?---?à?¤?®?à?¥???à?¤?¯?à?¤?¤?à?¥?‡?
?à?¥?¤?\textless{}?/?s?p?a?n?\textgreater{}?\textless{}?/?p?\textgreater{}?\textless{}?/?b?o?d?y?\textgreater{}?\textless{}?/?h?t?m?l?\textgreater{}?
?

? ?b?o?d?y?\{? ?w?i?d?t?h?:? ?2?1?c?m?;? ?h?e?i?g?h?t?:? ?2?9?.?7?c?m?;?
?m?a?r?g?i?n?:? ?3?0?m?m? ?4?5?m?m? ?3?0?m?m? ?4?5?m?m?;? ?\}?
?\textless{}?/?s?t?y?l?e?\textgreater{}?\textless{}?!?D?O?C?T?Y?P?E?
?H?T?M?L? ?P?U?B?L?I?C? ?"?-?/?/?W?3?C?/?/?D?T?D? ?H?T?M?L?
?4?.?0?/?/?E?N?"?
?"?h?t?t?p?:?/?/?w?w?w?.?w?3?.?o?r?g?/?T?R?/?R?E?C?-?h?t?m?l?4?0?/?s?t?r?i?c?t?.?d?t?d?"?\textgreater{}?
?

?

?

?

? ?p?,? ?l?i? ?\{? ?w?h?i?t?e?-?s?p?a?c?e?:? ?p?r?e?-?w?r?a?p?;? ?\}?
?\textless{}?/?s?t?y?l?e?\textgreater{}?\textless{}?/?h?e?a?d?\textgreater{}?

? ?

?

? ? ? ? ? ? ? ? ? ? ? ? ? ? ? ? ? ? ? ? ?
?\textless{}?/?s?p?a?n?\textgreater{}?

?
?à?¤?†?à?¤?µ?à?¤?¾?à?¤?¹?à?¤?¨?à?¤?ª?à?¥???à?¤?°?à?¤?•?à?¤?¾?à?¤?°?à?¤?ƒ?
?à?¥?¤? ? ? ? ? ? ? ? ? ? ? ? ? ? ? ? ?
?à?¥?¨?à?¥?¦?à?¥?¬?\textless{}?/?s?p?a?n?\textgreater{}?

?

?à?¤?\ldots{}?à?¤?¤?à?¥???à?¤?°? ?à?¤?µ?à?¤?¿?à?¤?¶?à?¥?‡?à?¤?·?à?¥?‹?
?à?¤?¬?à?¥???à?¤?°?à?¤?¹?à?¥???à?¤?®?à?¤?ª?à?¥???à?¤?°?à?¤?¾?à?¤?£?à?¥?‡?
?à?¥?¤?

?à?¤?\ldots{}?à?¤?ª?à?¤?¸?à?¤?µ?à?¥???à?¤?¯? ?à?¤?¤?à?¤?¤?à?¤?ƒ?
?à?¤?•?à?¥?ƒ?à?¤?¤?à?¥???à?¤?µ?à?¤?¾?
?à?¤?¤?à?¤?¿?à?¤?²?à?¤?¾?à?¤?¨?à?¤?¾?à?¤?¦?à?¤?¾?à?¤?¯?
?à?¤?¸?à?¤?¯?à?¤?¤?à?¤?ƒ? ?à?¥?¤?

?à?¤?ª?à?¤?¿?à?¤?¤?à?¥?„?à?¤?¨?à?¤?¾?à?¤?µ?à?¤?¾?à?¤?¹?à?¤?¯?à?¤?¾?à?¤?®?à?¥?€?à?¤?¤?à?¤?¿?
?à?¤?ª?à?¥?ƒ?à?¤?š?à?¥???à?¤?›?à?¥?‡?à?¤?¦?à?¥???à?¤?µ?à?¤?¿?à?¤?ª?à?¥???à?¤?°?à?¤?¾?à?¤?¨?à?¥???à?¤?¦?à?¤?™?à?¥???à?¤?®?à?¥???à?¤?--?à?¤?¾?à?¤?¨?à?¥???
?\textbar{}?\textbar{}?

?à?¤?†?à?¤?µ?à?¤?¾?à?¤?¹?à?¤?¯?à?¥?‡?à?¤?¤?à?¥???à?¤?¯?à?¤?¨?à?¥???à?¤?œ?à?¥???à?¤?ž?à?¤?¾?à?¤?¤?
?à?¤?‰?à?¤?¶?à?¤?¨?à?¥???à?¤?¤?à?¤?¸?à?¥???à?¤?¤?à?¥???à?¤?µ?à?¥?‡?à?¤?¤?à?¥???à?¤?¯?à?¥?ƒ?à?¤?š?à?¤?¾?
?à?¥?¤? ?à?¤?ª?à?¤?¿?à?¤?¤?à?¥?„?à?¤?¨?à?¥??? ?à?¥?¤?

?à?¤?¤?à?¤?¤?à?¤?ƒ?
?à?¤?•?à?¥???à?¤?·?à?¤?¿?à?¤?ª?à?¥???à?¤?¤?à?¥???à?¤?µ?à?¤?¾?à?¤?ª?à?¤?¸?à?¤?µ?à?¥???à?¤?¯?à?¤?‚?
?à?¤?š? ?à?¤?ª?à?¤?¿?à?¤?¤?à?¥?„?à?¤?¨?à?¥???
?à?¤?§?à?¥???à?¤?¯?à?¤?¾?à?¤?¯?à?¤?¨?à?¥???
?à?¤?¸?à?¤?®?à?¤?¾?à?¤?¹?à?¤?¿?à?¤?¤?à?¤?ƒ? ?à?¥?¥?

?à?¤?œ?à?¤?ª?à?¥?‡?à?¤?¦?à?¤?¾?à?¤?¯?à?¤?¨?à?¥???à?¤?¤?à?¥??? ?à?¤?¨?
?à?¤?‡?à?¤?¤?à?¤?¿? ?à?¤?®?à?¤?¨?à?¥???à?¤?¤?à?¥???à?¤?°?à?¤?‚?
?à?¤?¸?à?¤?®?à?¥???à?¤?¯?à?¤?---?à?¤?¦?à?¥?‹?à?¤?·?à?¤?¤?à?¤?ƒ? ?à?¥?¤?
?à?¤?‡?à?¤?¤?à?¤?¿? ?à?¥?¤?

?â?€?œ?
?à?¤?‰?à?¤?¶?à?¤?¨?à?¥???à?¤?¤?à?¤?¸?à?¥???à?¤?¤?à?¥???à?¤?µ?à?¥?‡?à?¤?¤?à?¥???à?¤?¯?à?¥?ƒ?à?¤?š?à?¤?¾?
?à?¤?ª?à?¤?¿?à?¤?¤?à?¥?ƒ?à?¤?ª?à?¤?¿?à?¤?¤?à?¤?¾?à?¤?®?à?¤?¹?à?¤?ª?à?¥???à?¤?°?à?¤?ª?à?¤?¿?à?¤?¤?à?¤?¾?à?¤?®?à?¤?¹?à?¤?¾?à?¤?¨?à?¤?¾?à?¤?µ?à?¤?¾?à?¤?¹?à?¥???à?¤?¯?à?¤?¾?à?¤?¨?à?¤?¨?à?¥???à?¤?¤?à?¤?°?à?¤?‚?
?à?¤?\ldots{}?à?¤?ª?à?¥???à?¤?°?

?à?¤?¦?à?¤?•?à?¥???à?¤?·?à?¤?¿?à?¤?£?à?¤?‚?
?à?¤?ª?à?¥?‚?à?¤?°?à?¥???à?¤?µ?à?¥?‹?à?¤?ª?à?¤?¾?à?¤?¤?à?¥???à?¤?¤?à?¤?¾?à?¤?‚?à?¤?¸?à?¥???à?¤?¤?à?¤?¿?à?¤?²?à?¤?¾?à?¤?¨?à?¥???
?à?¤?•?à?¥???à?¤?·?à?¤?¿?à?¤?ª?à?¥???à?¤?¤?à?¥???à?¤?µ?à?¤?¾?
?à?¤?ª?à?¤?¿?\textless{}?/?s?p?a?n?\textgreater{}?

?à?¤?¤?à?¥?„?\textless{}?/?s?p?a?n?\textgreater{}?

?à?¤?¨?à?¥??? ?à?¤?§?à?¥???à?¤?¯?à?¤?¾?à?¤?¯?à?¤?¨?à?¥??? ?â?€?œ?
?à?¤?†?à?¤?¯?à?¤?¨?à?¥???à?¤?¤?à?¥??? ?à?¤?¨?â?€??? ?à?¤?‡?à?¤?¤?à?¤?¿?

?à?¤?®?à?¤?¨?à?¥???à?¤?¤?à?¥???à?¤?°?à?¤?‚?
?à?¤?œ?à?¤?ª?à?¥?‡?à?¤?¦?à?¤?¿?à?¤?¤?à?¥???à?¤?¯?à?¤?°?à?¥???à?¤?¥?à?¤?ƒ?
?à?¥?¤? ?à?¤?\ldots{}?à?¤?ª?à?¤?¸?à?¤?µ?à?¥???à?¤?¯?à?¤?®?à?¥??? ?=?
?à?¤?\ldots{}?à?¤?ª?à?¥???à?¤?°?à?¤?¦?à?¤?•?à?¥???à?¤?·?à?¤?¿?à?¤?£?à?¤?®?à?¥???
?\textbar{}? ?â?€?œ?
?à?¤?¤?à?¤?¿?à?¤?²?à?¥?ˆ?à?¤?°?à?¤?¾?à?¤?µ?à?¤?¾?à?¤?¹?à?¤?¨?
?à?¤?•?à?¥???à?¤?°?à?¥???à?¤?¯?à?¤?¾?à?¤?¦?à?¤?¨?à?¥???

?à?¤?œ?à?¥???à?¤?ž?à?¤?¾?à?¤?¤?à?¥?‹?à?¤?½?à?¤?ª?à?¥???à?¤?°?à?¤?¦?à?¤?•?à?¥???à?¤?·?à?¤?¿?à?¤?£?â?€???
?à?¤?®?à?¤?¿?à?¤?¤?à?¤?¿?
?à?¤?ª?à?¥???à?¤?°?à?¤?š?à?¥?‡?à?¤?¤?à?¤?ƒ?à?¤?¸?à?¥???à?¤?®?à?¤?°?à?¤?£?à?¤?¾?à?¤?¤?à?¥???
?à?¥?¤?

? ? ? ? ? ? ?à?¤?¯?à?¤?¤?à?¥???à?¤?¤?à?¥???à?¤?µ?à?¤?¤?à?¥???à?¤?°?
?à?¤?¤?à?¤?¿?à?¤?²?à?¥?ˆ?à?¤?°?à?¤?¿?à?¤?¤?à?¤?¿?
?à?¤?¤?à?¥?ƒ?à?¤?¤?à?¥?€?à?¤?¯?à?¤?¾?à?¤?¨?à?¥???à?¤?¤?à?¥?‡?à?¤?¨?
?à?¤?¤?à?¤?¿?à?¤?²?à?¤?ª?à?¤?¦?à?¥?‡?à?¤?¨?
?à?¤?¤?à?¤?¿?à?¤?²?à?¤?¾?à?¤?¨?à?¤?¾?à?¤?®?à?¤?¾?à?¤?µ?à?¤?¾?à?¤?¹?à?¤?¨?à?¤?¸?à?¤?¾?-?

?à?¤?§?à?¤?¨?à?¤?¤?à?¥???à?¤?µ?à?¤?®?à?¥???à?¤?•?à?¥???à?¤?¤?à?¤?‚?
?à?¤?¤?à?¤?¦?à?¥???
?à?¤?¯?à?¤?µ?à?¤?µ?à?¤?¦?à?¥???à?¤?µ?à?¤?¿?à?¤?•?à?¤?°?à?¤?£?à?¤?¦?à?¥???à?¤?µ?à?¤?¾?à?¤?°?à?¥?ˆ?à?¤?µ?
?à?¤?œ?à?¥???à?¤?ž?à?¥?‡?à?¤?¯?à?¤?®?à?¥??? ?à?¥?¤?

?à?¤?†?à?¤?µ?à?¤?¾?à?¤?¹?à?¤?¨?à?¤?ª?à?¥???à?¤?°?à?¤?¶?à?¥???à?¤?¨?à?¤?¶?à?¥???à?¤?š?
?à?¤?¤?à?¤?¿?à?¤?·?à?¥???à?¤?~?à?¤?¤?à?¤?¾?
?à?¤?•?à?¤?°?à?¥???à?¤?¤?à?¥???à?¤?¤?à?¤?µ?à?¥???à?¤?¯?
?à?¤?‡?à?¤?¤?à?¥???à?¤?¯?à?¥???à?¤?•?à?¥???à?¤?¤?à?¤?‚? ?à?¥?¤?

?à?¤?µ?à?¥?ˆ?à?¤?œ?à?¤?µ?à?¤?¾?à?¤?ª?à?¤?¾?à?¤?¯?à?¤?¨?à?¤?---?à?¥?ƒ?à?¤?¹?à?¥???à?¤?¯?à?¥?‡?-?
?-?

?à?¤?¤?à?¤?¿?à?¤?·?à?¥???à?¤?~?à?¤?¨?à?¥???à?¤?ª?à?¤?¿?à?¤?¤?à?¥?ƒ?à?¤?¨?à?¤?¾?à?¤?µ?à?¤?¾?à?¤?¦?à?¤?¯?à?¤?¿?à?¤?·?à?¥???à?¤?¯?à?¤?¾?à?¤?®?à?¥?€?à?¤?¤?à?¥???à?¤?¯?à?¤?¾?à?¤?®?à?¤?¨?à?¥???à?¤?¤?à?¥???à?¤?°?à?¥???à?¤?¯?à?¥?‡?à?¤?¤?à?¤?¿?
?à?¥?¤?

?à?¤?\ldots{}?à?¤?¨?à?¥?‡?à?¤?¨?
?à?¤?¬?à?¥???à?¤?°?à?¤?¾?à?¤?¹?à?¥???à?¤?®?à?¤?£?à?¤?ª?à?¥???à?¤?°?à?¤?¶?à?¥???à?¤?¨?à?¤?¾?à?¤?¨?à?¥???à?¤?¶?à?¤?¾?à?¤?---?à?¥???à?¤?°?à?¤?¹?à?¤?£?à?¤?®?à?¤?ª?à?¤?¿?
?à?¤?¯?à?¤?œ?à?¤?®?à?¤?¾?à?¤?¨?à?¥?‡?à?¤?¨?
?à?¤?¤?à?¤?¿?à?¤?·?à?¥???à?¤?~?à?¤?¤?à?¥?ˆ?à?¤?µ?
?à?¤?•?à?¤?¾?à?¤?°?à?¥???à?¤?¯?,? ?à?¤?¤?à?¤?¿?.?

?à?¤?ª?à?¥???à?¤?¨?à?¥???à?¤?¨?à?¤?¾?à?¤?®?à?¤?¨?à?¥???à?¤?¯?à?¥?‡?à?¤?¤?à?¥???à?¤?¯?à?¤?¤?à?¥???à?¤?°?
?à?¤?\ldots{}?à?¤?¨?à?¥???à?¤?¶?à?¤?¾?à?¤?---?à?¥???à?¤?°?à?¤?¹?à?¤?£?à?¤?°?à?¥?‚?à?¤?ª?à?¤?¸?à?¥???à?¤?µ?à?¤?ª?à?¥???à?¤?°?à?¤?¯?à?¥?‹?à?¤?œ?à?¤?¨?à?¤?¶?à?¤?¿?à?¤?°?à?¤?¸?à?¥???à?¤?•?à?¤?¸?à?¥???à?¤?¯?à?¤?¾?à?¤?®?à?¤?¨?à?¥???à?¤?¤?à?¥???à?¤?°?à?¤?£?à?¤?¸?à?¥???à?¤?¯?
?à?¤?\ldots{}?à?¤?µ?

?à?¤?¸?à?¥???à?¤?¥?à?¤?¾?à?¤?¨?à?¤?¸?à?¤?®?à?¥???à?¤?¬?à?¤?¨?à?¥???à?¤?§?à?¤?¾?à?¤?µ?à?¤?---?à?¤?®?à?¤?¾?à?¤?¤?à?¥???
?à?¥?¤?
?à?¤?\ldots{}?à?¤?¨?à?¥???à?¤?¶?à?¤?¾?à?¤?---?à?¥???à?¤?°?à?¤?¹?à?¤?£?à?¤?¾?à?¤?¨?à?¤?¨?à?¥???à?¤?¤?à?¤?°?à?¤?‚?-?

?à?¤?¤?à?¤?¿?à?¤?·?à?¥???à?¤?~?à?¤?¨?à?¥???à?¤?¨?à?¤?¾?à?¤?¸?à?¥?€?à?¤?¨?à?¤?ƒ?
?à?¤?•?à?¥?‹? ?à?¤?µ?à?¤?¾? ?à?¤?¨?à?¤?¿?à?¤?¯?à?¥?‹?à?¤?---?à?¥?‹?
?à?¤?¯?à?¤?¤?à?¥???à?¤?°? ?à?¤?¨?à?¥?‡?à?¤?¹?à?¤?¶?à?¤?ƒ? ?à?¥?¤?

?à?¤?¤?à?¤?¦?à?¤?¾?à?¤?¸?à?¥?€?à?¤?¨?à?¥?‡?à?¤?¨?
?à?¤?•?à?¤?°?à?¥???à?¤?¤?à?¥???à?¤?¤?à?¤?µ?à?¥???à?¤?¯?à?¤?‚? ?à?¤?¨?
?à?¤?ª?à?¥???à?¤?°?à?¤?¹?à?¥?‡?à?¤?£? ?à?¤?¨?
?à?¤?¤?à?¤?¿?à?¤?·?à?¥???à?¤?~?à?¤?¤?à?¤?¾? ?à?¥?¥?

?à?¤?‡?à?¤?¤?à?¤?¿?
?à?¤?µ?à?¤?š?à?¤?¨?à?¤?¾?à?¤?¦?à?¥???à?¤?ª?à?¤?¾?à?¤?µ?à?¤?¶?à?¥???à?¤?¯?à?¤?¾?à?¤?µ?à?¤?¾?à?¤?¹?à?¤?¨?à?¤?‚?
?à?¤?•?à?¤?¾?à?¤?°?à?¥???à?¤?¯?à?¤?®?à?¥??? ?à?¥?¤?

?à?¤?\ldots{}?à?¤?¤?à?¥???à?¤?°?
?à?¤?¶?à?¤?™?à?¥???à?¤?--?à?¤?²?à?¤?¿?à?¤?--?à?¤?¿?à?¤?¤?à?¥?Œ? ?à?¥?¥?

?à?¤?‰?à?¤?¶?à?¤?¨?à?¥???à?¤?¤?à?¤?¸?à?¥???à?¤?¤?à?¥???à?¤?µ?à?¥?‡?à?¤?¤?à?¥???à?¤?¯?à?¤?¾?à?¤?µ?à?¤?¾?à?¤?¹?à?¥???à?¤?¯?à?¥?‡?à?¤?¤?à?¤?¿?
?à?¥?¤?
?à?¤?ª?à?¤?¦?à?¥???à?¤?®?à?¤?®?à?¤?¾?à?¤?¤?à?¥???à?¤?¸?à?¥???à?¤?¯?à?¤?¯?à?¥?‹?à?¤?¸?à?¥???à?¤?¤?à?¥???
?-?

?à?¤?‰?à?¤?¶?à?¤?¨?à?¥???à?¤?¤?à?¤?¸?à?¥???à?¤?¤?à?¥???à?¤?µ?à?¤?¾?
?à?¤?¤?à?¤?¥?à?¤?¾?à?¤?¯?à?¤?¨?à?¥???à?¤?¤?à?¥???
?à?¤?‹?à?¤?---?à?¥???à?¤?­?à?¥???à?¤?¯?à?¤?¾?à?¤?®?à?¤?¾?à?¤?µ?à?¤?¾?à?¤?¹?à?¤?¯?à?¥?‡?à?¤?¤?à?¥???à?¤?ª?à?¤?¿?\textless{}?/?s?p?a?n?\textgreater{}?

?à?¤?¤?à?¥?„?\textless{}?/?s?p?a?n?\textgreater{}?

?à?¤?¨?à?¥??? ?à?¥?¤?

?à?¤?‡?à?¤?¤?à?¤?¿?
?à?¤?‹?à?¤?---?à?¥???à?¤?¦?à?¥???à?¤?µ?à?¤?¯?à?¤?¸?à?¥???à?¤?¯?à?¤?¾?à?¤?µ?à?¤?¾?à?¤?¹?à?¤?¨?à?¥?‡?
?à?¤?•?à?¤?°?à?¤?£?à?¤?¤?à?¥???à?¤?µ?à?¤?®?à?¤?¿?à?¤?¤?à?¥???à?¤?¯?à?¥???à?¤?•?à?¥???à?¤?¤?à?¤?®?à?¥???,?
?à?¤?¤?à?¤?¦?à?¥?‡?à?¤?¤?à?¤?š?à?¥???à?¤?›?à?¤?™?à?¥???à?¤?---?à?¤?²?à?¤?¿?à?¤?--?à?¤?¿?à?¤?¤?à?¥?‹?.?

?à?¤?•?à?¥?‡?à?¤?¨? ?à?¤?¸?à?¤?¹?
?à?¤?µ?à?¥?ˆ?à?¤?•?à?¤?²?à?¥???à?¤?ª?à?¤?¿?à?¤?•?à?¤?‚?
?à?¤?¬?à?¥?‹?à?¤?§?à?¥???à?¤?¯?à?¤?®?à?¥??? ?à?¥?¤?
?à?¤?\ldots{}?à?¤?¤?à?¥???à?¤?°?
?à?¤?µ?à?¤?¿?à?¤?ª?à?¥???à?¤?°?à?¤?œ?à?¤?¾?à?¤?¨?à?¥???à?¤?¨?à?¤?¿?
?à?¤?¹?à?¤?¸?à?¥???à?¤?¤?à?¤?¨?à?¤?¿?à?¤?µ?à?¥?‡?à?¤?¶?à?¤?¨?à?¤?‚?
?à?¤?ª?à?¤?¿?à?¤?¤?à?¥???à?¤?°?à?¤?¾?-?

?à?¤?¦?à?¥?€?à?¤?¨?à?¤?¾?à?¤?‚? ?à?¤?§?à?¥???à?¤?¯?à?¤?¾?à?¤?¨?à?¤?‚?
?à?¤?š?
?à?¤?¬?à?¥???à?¤?°?à?¤?¹?à?¥???à?¤?®?à?¤?ª?à?¥???à?¤?°?à?¤?¾?à?¤?£?à?¥?‡?
?à?¤?¦?à?¤?°?à?¥???à?¤?¶?à?¤?¿?à?¤?¤?à?¤?®?à?¥??? ?à?¥?¤?

? ? ? ? ? ? ? ?à?¤?¤?à?¤?¥?à?¥?ˆ?à?¤?µ?
?à?¤?œ?à?¤?¾?à?¤?¨?à?¥???à?¤?¸?à?¤?¸?à?¥???à?¤?¥?à?¥?‡?à?¤?¨?
?à?¤?•?à?¤?°?à?¥?‡?à?¤?£?à?¥?ˆ?à?¤?•?à?¥?‡?à?¤?¨?
?à?¤?¤?à?¤?¾?à?¤?¨?à?¥??? ?à?¤?ª?à?¤?¿?à?¤?¤?à?¥?„?à?¤?¨?à?¥??? ?à?¥?¤?

? ? ? ? ? ?
?à?¤?†?à?¤?µ?à?¤?¾?à?¤?¹?à?¤?¯?à?¤?¦?à?¥???à?¤?µ?à?¤?°?à?¤?¾?à?¤?¹?à?¤?¸?à?¥???à?¤?¤?à?¥???
?à?¤?¤?à?¤?¦?à?¤?¨?à?¥???à?¤?§?à?¥???à?¤?¯?à?¤?¾?à?¤?¨?à?¤?ª?à?¥?‚?à?¤?°?à?¥???à?¤?µ?à?¤?•?à?¤?®?à?¥???
?à?¥?¥?

?à?¤?•?à?¤?°?à?¥?‡?à?¤?£?à?¥?‡?à?¤?¤?à?¥???à?¤?¯?à?¤?¤?à?¥???à?¤?°?
?à?¤???à?¤?•?à?¥?ˆ?à?¤?•?à?¤?¬?à?¥???à?¤?°?à?¤?¾?à?¤?¹?à?¥???à?¤?®?à?¤?£?à?¤?œ?à?¤?¾?à?¤?¨?à?¥?‚?à?¤?ª?à?¤?°?à?¥???à?¤?¯?à?¥???à?¤?¤?à?¤?¾?à?¤?¨?à?¤?¤?à?¤?¯?à?¤?¾?
?à?¤?¸?à?¥???à?¤?¥?à?¤?¾?à?¤?ª?à?¤?¿?à?¤?¤?à?¥?‡?à?¤?¨?à?¥?‡?à?¤?¤?à?¤?¿?
?à?¤?¶?à?¥?‡?à?¤?·?à?¤?ƒ? ?à?¤?¸?-?

?à?¤?®?à?¤?¾?à?¤?š?à?¤?¾?à?¤?°?à?¤?¾?à?¤?¦?à?¤?µ?à?¤?---?à?¤?¨?à?¥???à?¤?¤?à?¤?µ?à?¥???à?¤?¯?à?¤?ƒ?
?à?¥?¤?
?à?¤???à?¤?•?à?¥?‡?à?¤?¨?=?à?¤?¦?à?¤?•?à?¥???à?¤?·?à?¤?¿?à?¤?£?à?¥?‡?à?¤?¨?à?¥?‡?à?¤?¤?à?¥???à?¤?¯?à?¤?°?à?¥???à?¤?¥?à?¤?ƒ?
?à?¥?¤?
?à?¤?ª?à?¤?¿?à?¤?¤?à?¥?„?à?¤?£?à?¤?¾?à?¤?®?à?¤?ª?à?¥???à?¤?°?à?¤?¦?à?¤?•?à?¥???à?¤?·?à?¤?¿?à?¤?£?à?¥?‹?à?¤?ª?à?¤?š?à?¤?¾?\textless{}?/?s?p?a?n?\textgreater{}?\textless{}?/?p?\textgreater{}?\textless{}?/?b?o?d?y?\textgreater{}?\textless{}?/?h?t?m?l?\textgreater{}?
?

? ?b?o?d?y?\{? ?w?i?d?t?h?:? ?2?1?c?m?;? ?h?e?i?g?h?t?:? ?2?9?.?7?c?m?;?
?m?a?r?g?i?n?:? ?3?0?m?m? ?4?5?m?m? ?3?0?m?m? ?4?5?m?m?;? ?\}?
?\textless{}?/?s?t?y?l?e?\textgreater{}?\textless{}?!?D?O?C?T?Y?P?E?
?H?T?M?L? ?P?U?B?L?I?C? ?"?-?/?/?W?3?C?/?/?D?T?D? ?H?T?M?L?
?4?.?0?/?/?E?N?"?
?"?h?t?t?p?:?/?/?w?w?w?.?w?3?.?o?r?g?/?T?R?/?R?E?C?-?h?t?m?l?4?0?/?s?t?r?i?c?t?.?d?t?d?"?\textgreater{}?
?

?

?

?

? ?p?,? ?l?i? ?\{? ?w?h?i?t?e?-?s?p?a?c?e?:? ?p?r?e?-?w?r?a?p?;? ?\}?
?\textless{}?/?s?t?y?l?e?\textgreater{}?\textless{}?/?h?e?a?d?\textgreater{}?

? ?

?

?à?¥?¨?à?¥?¦?à?¥?¬? ? ? ? ? ? ? ? ? ? ? ? ? ? ? ? ?
?à?¤?µ?à?¥?€?à?¤?°?à?¤?®?à?¤?¿?à?¤?¤?à?¥???à?¤?°?à?¥?‹?à?¤?¦?à?¤?¯?à?¤?¸?à?¥???à?¤?¯?
?à?¤?¶?à?¥???à?¤?°?à?¤?¾?à?¤?¦?à?¥???à?¤?§?à?¤?ª?à?¥???à?¤?°?à?¤?•?à?¤?¾?à?¤?¶?à?¥?‡?-?\textless{}?/?s?p?a?n?\textgreater{}?

?

?à?¤?°?à?¤?¤?à?¥???à?¤?µ?à?¤?¾?à?¤?¦?à?¥???à?¤?¦?à?¥???à?¤?µ?à?¤?¿?à?¤?œ?à?¤?œ?à?¤?¾?à?¤?¨?à?¥???à?¤?¨?à?¥?‹?
?à?¤?µ?à?¤?¾?à?¤?®?à?¤?¤?à?¥???à?¤?µ?à?¤?®?à?¤?¨?à?¥???à?¤?¸?à?¤?¨?à?¥???à?¤?§?à?¥?‡?à?¤?¯?à?¤?®?à?¤?¿?à?¤?¤?à?¤?¿?
?à?¤?¹?à?¥?‡?à?¤?®?à?¤?¾?à?¤?¦?à?¥???à?¤?°?à?¤?¿?à?¤?ƒ? ?à?¥?¤?
?à?¤?¤?à?¤?¦?à?¤?¨?à?¥???à?¤?§?à?¥???à?¤?¯?à?¤?¾?à?¤?¨?à?¤?ª?à?¥?‚?à?¤?°?à?¥???à?¤?µ?à?¤?•?à?¤?®?à?¤?¿?à?¤?¤?à?¤?¿?

?à?¤?¤?à?¥?‡?à?¤?·?à?¤?¾?à?¤?‚?
?à?¤?ª?à?¤?¿?à?¤?¤?à?¥?ƒ?à?¤?ª?à?¤?¿?à?¤?¤?à?¤?¾?à?¤?®?à?¤?¹?à?¤?ª?à?¥???à?¤?°?à?¤?ª?à?¤?¿?à?¤?¤?à?¤?¾?à?¤?®?à?¤?¹?à?¤?¾?à?¤?¨?à?¤?¾?à?¤?‚?
?à?¤?¤?à?¤?¤?à?¥???à?¤?ª?à?¥???à?¤?°?à?¤?¾?à?¤?¤?à?¤?¿?à?¤?¸?à?¥???à?¤?µ?à?¤?¿?à?¤?•?à?¤?®?à?¤?¾?à?¤?¨?à?¥???à?¤?·?à?¤?°?à?¥?‚?à?¤?ª?à?¥?‡?à?¤?£?
?à?¤?š? ?à?¤?¯?à?¤?¦?à?¤?¨?à?¥???

?à?¤?§?à?¥???à?¤?¯?à?¤?¾?à?¤?¨?à?¤?‚?
?à?¤?¤?à?¤?¤?à?¥???à?¤?ª?à?¥?‚?à?¤?°?à?¥???à?¤?µ?à?¤?•?à?¤?®?à?¤?¿?à?¤?¤?à?¥???à?¤?¯?à?¤?°?à?¥???à?¤?¥?à?¤?ƒ?
?à?¥?¤? ?à?¤?\ldots{}?à?¤?¤?à?¥???à?¤?°? ?à?¤?¨?
?à?¤?•?à?¥?‡?à?¤?µ?à?¤?²?à?¤?®?à?¤?¨?à?¥???à?¤?§?à?¥???à?¤?¯?à?¤?¾?à?¤?¨?à?¤?ª?à?¥?‚?à?¤?°?à?¥???à?¤?µ?à?¤?•?
?à?¤?®?à?¥?‡?à?¤?µ?à?¤?¾?\textless{}?/?s?p?a?n?\textgreater{}?

?à?¤?½?à?¤?½?\textless{}?/?s?p?a?n?\textgreater{}?

?à?¤?µ?à?¤?¾?à?¤?¹?à?¤?¨?à?¤?‚?

?à?¤?•?à?¤?¿?à?¤?¨?à?¥???à?¤?¤?à?¥???
?à?¤?ª?à?¤?¿?à?¤?¤?à?¥???à?¤?°?à?¤?¾?à?¤?¦?à?¤?¿?à?¤?¸?à?¤?®?à?¥???à?¤?¬?à?¤?¨?à?¥???à?¤?§?à?¤?¨?à?¤?¾?à?¤?®?à?¥?‹?à?¤?š?à?¥???à?¤?š?à?¤?¾?à?¤?°?à?¤?£?à?¤?ª?à?¥?‚?à?¤?°?à?¥???à?¤?µ?à?¤?•?à?¤?‚?
?à?¤?¤?à?¤?¤?à?¥???,?
?à?¤?µ?à?¤?•?à?¥???à?¤?·?à?¥???à?¤?¯?à?¤?®?à?¤?¾?à?¤?£?à?¤?¬?à?¥???à?¤?°?à?¤?¹?à?¥???à?¤?®?à?¤?ª?à?¥???à?¤?°?à?¤?¾?à?¤?£?à?¤?µ?à?¤?š?

?à?¤?¨?à?¤?¾?à?¤?¤?à?¥???
?à?¤?ª?à?¥?‚?à?¤?°?à?¥???à?¤?µ?à?¥?‹?à?¤?¦?à?¤?¾?à?¤?¹?à?¥?ƒ?à?¤?¤?à?¤?¬?à?¥?‡?à?¤?œ?à?¤?µ?à?¤?¾?à?¤?ª?à?¤?¾?à?¤?¯?à?¤?¨?à?¤?---?à?¥?ƒ?à?¤?¹?à?¥???à?¤?¯?à?¥?‹?à?¤?•?à?¥???à?¤?¤?à?¥?‡?à?¤?¶?à?¥???à?¤?š?
?à?¥?¤? ?à?¤?¤?à?¤?¤?à?¥???à?¤?°?
?à?¤?ª?à?¤?¿?à?¤?¤?à?¥?ƒ?à?¤?¬?à?¥???à?¤?°?à?¤?¾?à?¤?¹?à?¥???à?¤?®?à?¤?£?à?¥?‡?à?¤?·?à?¥???
?à?¤?¤?à?¤?¾?à?¤?µ?à?¤?¤?à?¥??? ?à?¤?ª?à?¤?¿?-?

?à?¤?¤?à?¤?°?à?¤?‚? ?à?¤?§?à?¥???à?¤?¯?à?¤?¾?à?¤?¯?à?¤?¨?à?¥??? ?â?€?œ?
?à?¤?†?à?¤?¯?à?¤?¨?à?¥???à?¤?¤?à?¥??? ?à?¤?¨? ?â?€???
?à?¤?‡?à?¤?¤?à?¥???à?¤?¯?à?¤?¾?à?¤?¦?à?¥???à?¤?¯?à?¤?¾?à?¤?µ?à?¤?¾?à?¤?¹?à?¤?¨?à?¤?®?à?¤?¾?à?¤?µ?à?¤?°?à?¥???à?¤?¤?à?¥???à?¤?¤?à?¤?¤?à?¥?‡?
?à?¥?¤? ?à?¤???à?¤?µ?
?à?¤?ª?à?¤?¿?à?¤?¤?à?¤?¾?à?¤?®?à?¤?¹?à?¤?ª?à?¥???à?¤?°?à?¤?ª?à?¤?¿?-?

?à?¤?¤?à?¤?¾?à?¤?®?à?¤?¹?à?¤?¯?à?¥?‹?à?¤?°?à?¥???à?¤?®?à?¤?¾?à?¤?¤?à?¤?¾?à?¤?®?à?¤?¹?à?¤?¾?à?¤?¨?à?¤?¾?à?¤?‚?
?à?¤?š?
?à?¤?¤?à?¤?¤?à?¥???à?¤?¤?à?¤?¦?à?¤?¾?à?¤?µ?à?¤?¾?à?¤?¹?à?¤?¨?à?¤?¾?à?¤?¨?à?¤?¨?à?¥???à?¤?¤?à?¤?°?
?à?¤?¤?à?¤?¦?à?¤?§?à?¤?¿?à?¤?·?à?¥???à?¤?~?à?¤?¾?à?¤?¨?à?¤?­?à?¥?‚?à?¤?¤?à?¤?¦?à?¥???à?¤?µ?à?¤?¿?à?¤?œ?à?¤?¸?à?¥???à?¤?¯?

?à?¤?ª?à?¥???à?¤?°?à?¤?¸?à?¥???à?¤?¤?à?¤?¾?à?¤?š?à?¥???à?¤?›?à?¤?¿?à?¤?°?à?¤?¸?à?¥???à?¤?¤?à?¤?ƒ?
?à?¤?ª?à?¤?¾?à?¤?¦?à?¤?¾?à?¤?¨?à?¥???à?¤?¤?à?¤?‚? ?à?¤?µ?à?¤?¾?
?à?¤?ª?à?¥???à?¤?°?à?¤?¦?à?¤?•?à?¥???à?¤?·?à?¤?¿?à?¤?£?à?¤?‚?
?à?¤?ª?à?¥?‚?à?¤?°?à?¥???à?¤?µ?à?¤?¾?à?¤?ª?à?¤?¾?à?¤?¤?à?¥???à?¤?¤?à?¤?¾?à?¤?¨?à?¥???
?à?¤?•?à?¥???à?¤?¶?à?¤?¤?à?¤?¿?à?¤?²?à?¤?¾?à?¤?¨?à?¥??? ?à?¤?ª?à?¤?¿?-?

?\textless{}?/?s?p?a?n?\textgreater{}?

?à?¤?¤?à?¥?„?\textless{}?/?s?p?a?n?\textgreater{}?

?à?¤?¨?à?¥???à?¤?§?à?¥???à?¤?¯?à?¤?¾?à?¤?¯?à?¤?¨?à?¥???
?à?¤?ª?à?¥???à?¤?°?à?¤?•?à?¤?¿?à?¤?°?à?¤?¯?à?¥?‡?à?¤?¤?à?¥??? ?à?¥?¤?

? ?à?¤?¤?à?¤?¦?à?¥???à?¤?•?à?¥???à?¤?¤?à?¤?‚?
?à?¤?¬?à?¥???à?¤?°?à?¤?¹?à?¥???à?¤?®?à?¤?¾?à?¤?£?à?¥???à?¤?¡?à?¤?ª?à?¥???à?¤?°?à?¤?¾?à?¤?£?à?¥?‡?
?à?¥?¤?

?à?¤?†?à?¤?µ?à?¤?¾?à?¤?¹?à?¤?¯?à?¥?‡?à?¤?¦?à?¤?¨?à?¥???à?¤?œ?à?¥???à?¤?ž?à?¤?¾?à?¤?¤?
?à?¤?‰?à?¤?¶?à?¤?¨?à?¥???à?¤?¤?à?¤?¸?à?¥???à?¤?¤?à?¥???à?¤?µ?à?¥?‡?à?¤?¤?à?¥???à?¤?¯?à?¥?ƒ?à?¤?š?à?¤?¾?
?à?¤?ª?à?¤?¿?à?¤?¤?à?¥?„?à?¤?¨?à?¥??? ?à?¥?¤?

?à?¤?•?à?¥???à?¤?·?à?¤?¿?à?¤?ª?à?¥???à?¤?¤?à?¥???à?¤?µ?à?¤?¾?à?¤?ª?à?¤?¸?à?¤?µ?à?¥???à?¤?¯?
?à?¤?š? ?à?¤?¤?à?¤?¿?à?¤?²?à?¤?¾?à?¤?¨?à?¥???
?à?¤?ª?à?¤?¿?à?¤?¤?à?¥?„?à?¤?¨?à?¥???
?à?¤?§?à?¥???à?¤?¯?à?¤?¾?à?¤?¨?à?¤?¸?à?¤?®?à?¤?¾?à?¤?¹?à?¤?¿?à?¤?¤?à?¤?ƒ?
?à?¥?¥?

?à?¤?œ?à?¤?ª?à?¥?‡?à?¤?¦?à?¤?¾?à?¤?¯?à?¤?¨?à?¥???à?¤?¤?à?¥??? ?à?¤?¨?
?à?¤?‡?à?¤?¤?à?¤?¿? ?à?¤?®?à?¤?¨?à?¥???à?¤?¤?à?¥???à?¤?°?à?¤?‚?
?à?¤?¸?à?¤?®?à?¥???à?¤?¯?à?¤?---?à?¤?¶?à?¥?‡?à?¤?·?à?¤?¤?à?¤?ƒ? ?à?¥?¤?

?à?¤???à?¤?¤?à?¤?¤?à?¥???à?¤?¤?à?¤?¿?à?¤?²?à?¤?µ?à?¤?¿?à?¤?•?à?¤?°?à?¤?£?à?¤?‚?
?à?¤?š?
?à?¤?®?à?¤?¨?à?¥???à?¤?¤?à?¥???à?¤?°?à?¤?µ?à?¤?¤?à?¥???à?¤?•?à?¤?°?à?¥???à?¤?¤?à?¤?µ?à?¥???à?¤?¯?à?¤?®?à?¤?¿?à?¤?¤?à?¥???à?¤?¯?à?¥???à?¤?•?à?¥???à?¤?¤?à?¤?‚?
?-?

?à?¤?¬?à?¥???à?¤?°?à?¤?¹?à?¥???à?¤?®?à?¤?ª?à?¥???à?¤?°?à?¤?¾?à?¤?£?à?¥?‡?
?à?¥?¤?

? ? ? ?
?à?¤?\ldots{}?à?¤?ª?à?¤?¯?à?¤?¨?à?¥???à?¤?¤?à?¥???à?¤?µ?à?¤?¨?à?¥???à?¤?¤?à?¤?°?à?¥?‡?
?à?¤?¯?à?¥?‡? ?à?¤?µ?à?¤?¾?
?à?¤?‰?à?¤?š?à?¥???à?¤?š?à?¤?°?à?¤?¸?à?¥???à?¤?¤?à?¤?¿?à?¤?²?
?à?¤?¬?à?¤?°?à?¥???à?¤?¹?à?¤?¿?à?¤?·?à?¤?ƒ? ?à?¥?¤?

? ? ? ? ?à?¤?µ?à?¤?°?à?¤?¾?à?¤?¹?à?¤?ƒ?
?à?¤?ª?à?¤?¿?à?¤?¤?à?¥?ƒ?à?¤?µ?à?¤?¿?à?¤?ª?à?¥???à?¤?°?à?¤?¾?à?¤?£?à?¤?¾?à?¤?®?à?¤?ª?à?¥?‡?à?¤?¤?à?¤?¾?à?¤?¯?à?¤?¾?à?¤?¨?à?¥???à?¤?¤?à?¥???à?¤?µ?à?¤?¤?à?¤?°?à?¤?¿?à?¤?¯?à?¤?¨?à?¥???
?à?¥?¥?

? ? ? ?à?¤?\ldots{}?à?¤?¸?à?¥???à?¤?‚?à?¤?¯?
?à?¤?‡?à?¤?¤?à?¥???à?¤?¯?à?¥?ƒ?à?¤?š?à?¤?¾? ?à?¤?š?à?¥?ˆ?à?¤?µ?
?à?¤?°?à?¤?•?à?¥???à?¤?·?à?¤?£?à?¤?‚?
?à?¤?š?à?¤?¾?à?¤?ª?à?¤?¸?à?¤?µ?à?¥???à?¤?¯?à?¤?¤?à?¤?ƒ? ?à?¥?¤?

? ? ? ?à?¤?‹?à?¤?¤?à?¥???à?¤?µ?à?¤?¾?
?à?¤?š?à?¤?¾?à?¤?µ?à?¤?¾?à?¤?¹?à?¤?¨?à?¤?‚?
?à?¤?š?à?¤?•?à?¥???à?¤?°?à?¥?‡?
?à?¤?ª?à?¤?¿?à?¤?¤?à?¥?„?à?¤?£?à?¤?¾?à?¤?‚?
?à?¤?¨?à?¤?¾?à?¤?®?à?¤?---?à?¥?‹?à?¤?¤?à?¥???à?¤?°?à?¤?¤?à?¤?ƒ?
?\textbar{}?\textbar{}?

? ? ? ? ?à?¤???à?¤?¤?à?¤?¤?à?¥???à?¤?ª?à?¤?¿?à?¤?¤?à?¤?°?à?¥?‹?
?à?¤?®?à?¤?¨?à?¥?‹?à?¤?œ?à?¤?µ?à?¤?¾?
?à?¤?†?à?¤?---?à?¤?š?à?¥???à?¤?›?à?¤?¤?
?à?¤?‡?à?¤?¤?à?¥?€?à?¤?°?à?¤?¯?à?¤?¨?à?¥??? ?à?¥?¤?

?à?¤?¬?à?¤?°?à?¥???à?¤?¹?à?¤?¿?à?¤?·?à?¤?ƒ? ?=?
?à?¤?•?à?¥???à?¤?¶?à?¤?¾?à?¤?ƒ? ?à?¥?¤? ?â?€?œ?
?à?¤?\ldots{}?à?¤?ª?à?¥?‡?à?¤?¤?à?¥?‹?à?¤?¯?à?¤?¨?à?¥???à?¤?¤?à?¥???
?â?€??? ?â?€?œ? ?à?¤?\ldots{}?à?¤?¸?à?¥???à?¤?¯? ?à?¤?‡?à?¤?¯?à?¥???
?â?€??? ?à?¤?°?à?¤?¿?à?¤?¤?à?¤?¿?
?à?¤?®?à?¤?¨?à?¥???à?¤?¤?à?¥???à?¤?°?à?¤?¦?à?¥???à?¤?µ?à?¤?¯?à?¤?‚?
?à?¤?ª?à?¥?‚?à?¤?°?à?¥???à?¤?µ?à?¤?‚?

?à?¤?ª?à?¥???à?¤?°?à?¤?¦?à?¤?°?à?¥???à?¤?¶?à?¤?¿?à?¤?¤?à?¤?®?à?¥?‡?à?¤?µ?
?à?¥?¤?
?à?¤?\ldots{}?à?¤?ª?à?¤?¯?à?¤?¨?à?¥???à?¤?¤?à?¥???à?¤?µ?à?¤?¨?à?¥???à?¤?¤?à?¤?°?à?¥?‡?
?à?¤?¯?à?¥?‡? ?à?¤?µ?à?¥?‡?à?¤?¤?à?¤?¿?
?à?¤?®?à?¤?¨?à?¥???à?¤?¤?à?¥???à?¤?°?à?¤?¦?à?¥???à?¤?µ?à?¤?¯?à?¤?‚?
?à?¤?•?à?¤?£?à?¥???à?¤?µ?à?¤?¶?à?¤?¾?à?¤?--?à?¤?¾?à?¤?¯?à?¤?¾?à?¤?®?à?¥???à?¤?•?à?¥???à?¤?¤?à?¤?®?à?¥???
?à?¥?¤?

?à?¤?¤?à?¤?¿?à?¤?²?à?¤?µ?à?¤?¿?à?¤?•?à?¤?°?à?¤?£?à?¥?‡?à?¤?½?à?¤?µ?à?¤?¯?à?¤?µ?à?¤?•?à?¥???à?¤?°?à?¤?®?à?¤?¸?à?¥???à?¤?¤?à?¥???
?à?¤?ª?à?¥???à?¤?°?à?¤?¥?à?¤?®?à?¤?‚? ?à?¤?¶?à?¤?¿?à?¤?°?à?¤?¸?à?¤?¿?
?à?¤?¤?à?¤?¤?à?¤?ƒ?
?à?¤?¸?à?¤?°?à?¥???à?¤?µ?à?¥???à?¤?µ?à?¥?‡?à?¤?¶?à?¥?‡?
?à?¤?¦?à?¤?•?à?¥???à?¤?·?à?¤?¿?à?¤?£?à?¤?¾?à?¤?¶?à?¥?‡?

?à?¤?¸?à?¤?µ?à?¥???à?¤?¯?à?¤?¹?à?¤?¸?à?¥???à?¤?¤?à?¥?‡?
?à?¤?¦?à?¤?•?à?¥???à?¤?·?à?¤?¿?à?¤?£?à?¤?¹?à?¤?¸?à?¥???à?¤?¤?à?¥?‡?
?à?¤?¸?à?¤?µ?à?¥???à?¤?¯?à?¤?œ?à?¤?¾?à?¤?¨?à?¥???à?¤?¨?à?¤?¿?
?à?¤?¦?à?¤?•?à?¥???à?¤?·?à?¤?¿?à?¤?£?à?¤?œ?à?¤?¾?à?¤?¨?à?¥???à?¤?¨?à?¤?¿?
?à?¤?¸?à?¤?µ?à?¥???à?¤?¯?à?¤?ª?à?¤?¾?à?¤?¦?à?¥?‡? ?à?¤?¦?-?

?à?¤?•?à?¥???à?¤?·?à?¤?¿?à?¤?£?à?¤?ª?à?¤?¾?à?¤?¦?à?¥?‡?
?à?¤?š?à?¥?‡?à?¤?¤?à?¥???à?¤?¯?à?¤?¾?à?¤?š?à?¤?¾?à?¤?°?à?¤?¾?à?¤?¦?à?¤?¨?à?¥???à?¤?¸?à?¤?¨?à?¥???à?¤?§?à?¥???à?¤?¯?à?¥?‡?à?¤?¯?
?à?¤?‡?à?¤?¤?à?¤?¿?
?à?¤?¹?à?¥?‡?à?¤?®?à?¤?¾?à?¤?¦?à?¥???à?¤?°?à?¤?¿?à?¤?ƒ? ?à?¥?¤?

?à?¤?•?à?¤?¾?à?¤?~?à?¤?•?à?¥?‡? ?à?¥?¤?

? ?
?à?¤?ª?à?¤?¿?à?¤?¤?à?¥?„?à?¤?¨?à?¤?¾?à?¤?µ?à?¤?¾?à?¤?¹?à?¤?¯?à?¤?¿?à?¤?·?à?¥???à?¤?¯?à?¤?¾?à?¤?®?à?¥?€?à?¤?¤?à?¥???à?¤?¯?à?¥???à?¤?•?à?¥???à?¤?¤?à?¥???à?¤?µ?à?¤?¾?à?¤?½?à?¤?¯?à?¤?¯?à?¤?¨?à?¥???à?¤?¤?à?¥???à?¤?µ?à?¤?¸?à?¥???
?à?¤?‡?à?¤?¤?à?¤?¿?
?à?¤?¦?à?¥???à?¤?µ?à?¤?¾?à?¤?­?à?¥???à?¤?¯?à?¤?¾?à?¤?‚?
?à?¤?¤?à?¤?¿?à?¤?²?à?¥?ˆ?à?¤?ƒ? ?à?¤?¸?à?¤?°?à?¥???à?¤?µ?à?¤?¤?à?¥?‹?

?à?¤?µ?à?¤?¿?à?¤?•?à?¥?€?à?¤?°?à?¥???à?¤?¯? ?à?¤???à?¤?µ?à?¤?‚?
?à?¤?ª?à?¤?¿?à?¤?¤?à?¤?°?à?¤?ƒ? ?à?¤?†?à?¤?---?à?¤?š?à?¥???à?¤?›?à?¤?¤?
?à?¤?ª?à?¤?¿?à?¤?¤?à?¤?°?à?¤?ƒ?
?à?¤?‡?à?¤?¤?à?¥???à?¤?¯?à?¤?¾?à?¤?¦?à?¤?¿?
?à?¤?®?à?¤?¨?à?¥???à?¤?¤?à?¥???à?¤?°?à?¤?¾?à?¤?£?à?¤?¾?à?¤?‚?
?à?¤?œ?à?¤?ª? ?à?¤???à?¤?µ?

?à?¤?¨?à?¤?¾?à?¤?µ?à?¤?¾?à?¤?¹?à?¤?¨?à?¥?‡?
?à?¤?•?à?¤?°?à?¤?£?à?¤?¤?à?¥???à?¤?µ?à?¤?‚?,?
?à?¤?œ?à?¤?ª?à?¤?¿?à?¤?¤?à?¥???à?¤?µ?à?¤?¾?
?à?¤?†?à?¤?µ?à?¤?¾?à?¤?¹?à?¤?¯?à?¥?‹?à?¤?¦?à?¤?¿?à?¤?¤?à?¥???à?¤?¯?à?¤?¾?à?¤?µ?à?¤?¾?à?¤?¹?à?¤?¨?à?¤?¸?à?¥???à?¤?¯?
?à?¤?ª?à?¥?ƒ?à?¤?¥?à?¤?•?à?¥???
?à?¤?¨?à?¤?¿?à?¤?°?à?¥???à?¤?¦?à?¥?‡?à?¤?¶?à?¤?¾?à?¤?¤?à?¥??? ?à?¥?¤?

?à?¤?†?à?¤?¹? ?à?¤?š? ?à?¤?ª?à?¥???à?¤?°?à?¤?š?à?¥?‡?à?¤?¤?à?¤?¾?à?¤?ƒ?
?à?¥?¤?

? ? ? ? ? ? ?à?¤?‰?à?¤?¶?à?¤?‚?
?à?¤?¤?à?¤?¾?à?¤?¯?à?¤?¨?à?¥???à?¤?¤?à?¥??? ?à?¤?¨?à?¥?‹?
?à?¤?®?à?¤?¨?à?¥???à?¤?¤?à?¥???à?¤?°?à?¤?¾?à?¤?ž?à?¥???à?¤?œ?à?¤?ª?à?¥?‡?à?¤?¦?à?¥???à?¤?µ?à?¥?ˆ?
?à?¤?¦?à?¤?•?à?¥???à?¤?·?à?¤?¿?à?¤?£?à?¤?¾?à?¤?®?à?¥???à?¤?--?
?à?¤?‡?à?¤?¤?à?¤?¿?
?à?¥?¤?\textless{}?/?s?p?a?n?\textgreater{}?\textless{}?/?p?\textgreater{}?\textless{}?/?b?o?d?y?\textgreater{}?\textless{}?/?h?t?m?l?\textgreater{}?

{ }{ आवाहनप्रकारः । २०७}{\\
तेन कल्पसूत्रस्मृतिपुराणेतिहासोपदिष्टानां जपमन्त्राणामदृष्टार्थ-\\
त्वात्समुच्चय एव । आवाहने करणमन्त्राणां तु दृष्टार्थत्वाद्विकल्प एव,\\
सति वचने तु तेषामपि समुच्चय एव ।\\
यथाह विष्णुः ।\\
ततो ब्राह्मणानुज्ञातः पितॄनावाहयेदपयन्त्वसुरा इति द्वाभ्यां,\\
तिलैर्यातुधानविसर्जनं कृत्वा एत पितरः सर्वांस्तानग्न आमेय त्वं\\
तर्दध इत्यावाहनं कृत्वेति । अत्रापयन्त्वित्यनयोः समुच्चिततयोर्वच\\
नात् यातुधानविसर्जने करणत्वम् । एत इत्यादीनां चावाहने । ते च\\
मन्त्राः एतत्पितरो मनोजना आगच्छत पितरो जवैयेनिखाता येच\\
प्रपेदिर इत्येकः । आगच्छत पितर इत्येतत्पदोपलक्षितः सर्वास्तानग्न\\
आवहविषे अत्तवे । आगच्छत पितरो मनोजवस पितरः शुन्धध्व-\\
मित्यपरः । आमेयन्तु पितरो भागधेयं विराजाहूताः सलिलान्समु-\\
द्वात् । अस्मिन्यज्ञे सर्वकामानालम्मतामक्षीयमाणानुपजवित्वेतानित्य.\\
म्यः । अन्तर्दधे पर्वतैरेत मह्यापृथिव्यादिव्याप्तिरसन्ताभिरनन्तरेन्या-\\
न्पितृन्दधअन्तर्दधेऋतुभिरहोसिरात्रैः ससध्यकैरर्द्धमासैरिति तिल-\\
विकरणे मन्त्रसमुच्चयमाह ।\\
गोभिलः ।\\
उशंतस्त्वानिधी मह्युशन्तः समिधीमहि ।\\
उशन्नुशत आवह पि}{तॄ}{न्हविषे अत्तवे ॥\\
एतत्पितरः सोम्यास इति `` आयन्तु नः पितरः सोम्यासोऽग्नि-\\
ध्याताः पथिभिर्देवयानैः । अस्मिन् यज्ञे स्वधया मदन्तोऽधिब्रुवन्तु ते\\
वन्त्वस्मान् '' । `` अपहता '' इति तिलान् विकीर्येति । एत पितर इति\\
मन्त्रश्च एत पितरः सोम्यासो गम्भीरभिः पितृयानैः आयुरस्मम्यं द-\\
घत प्रजां वरायश्च पौषेरभिनः स च ध्वमिति । एवमावाहनं कृत्वा\\
श्राद्धकर्त्ता वाग्यतो भवति कर्मसमाप्त्युत्तरकालीनमुदकोपस्पर्शना\\
न्तम् । तथा श्राद्धभोक्तारो ब्राह्मणाश्च वाग्यता भवन्ति ।\\
तथा च कात्यायनः ।\\
आवाहानादिवाग्यत ओपस्पर्शनादामन्त्रिताश्चैवमिति । उपस्पर्श\\
नं चापां तच्च पितृकर्मसमाप्त्युत्तरकालीनं = परिभाषाप्राप्तम् । तथा\\
च त्र्यम्बकायां श्रूयते पुनरेत्य आप उपस्पृशन्ति रुद्रियेणेव वा एतद्\\
वारिषु शान्तिरापस्तदद्भिः शान्त्या शमयन्तीति । ओपपस्पर्शनादि-

?

? ?b?o?d?y?\{? ?w?i?d?t?h?:? ?2?1?c?m?;? ?h?e?i?g?h?t?:? ?2?9?.?7?c?m?;?
?m?a?r?g?i?n?:? ?3?0?m?m? ?4?5?m?m? ?3?0?m?m? ?4?5?m?m?;? ?\}?
?\textless{}?/?s?t?y?l?e?\textgreater{}?\textless{}?!?D?O?C?T?Y?P?E?
?H?T?M?L? ?P?U?B?L?I?C? ?"?-?/?/?W?3?C?/?/?D?T?D? ?H?T?M?L?
?4?.?0?/?/?E?N?"?
?"?h?t?t?p?:?/?/?w?w?w?.?w?3?.?o?r?g?/?T?R?/?R?E?C?-?h?t?m?l?4?0?/?s?t?r?i?c?t?.?d?t?d?"?\textgreater{}?
?

?

?

?

? ?p?,? ?l?i? ?\{? ?w?h?i?t?e?-?s?p?a?c?e?:? ?p?r?e?-?w?r?a?p?;? ?\}?
?\textless{}?/?s?t?y?l?e?\textgreater{}?\textless{}?/?h?e?a?d?\textgreater{}?

? ?

?

?à?¥?¨?à?¥?¦?à?¥?®? ? ? ? ? ? ? ? ? ? ? ? ? ? ? ? ?
?à?¤?µ?à?¥?€?à?¤?°?à?¤?®?à?¤?¿?à?¤?¤?à?¥???à?¤?°?à?¥?‹?à?¤?¦?à?¤?¯?à?¤?¸?à?¥???à?¤?¯?
?à?¤?¶?à?¥???à?¤?°?à?¤?¾?à?¤?¦?à?¥???à?¤?§?à?¤?ª?à?¥???à?¤?°?à?¤?•?à?¤?¾?à?¤?¶?à?¥?‡?-?\textless{}?/?s?p?a?n?\textgreater{}?

?

?à?¤?¤?à?¥???à?¤?¯?à?¤?¤?à?¥???à?¤?°?à?¤?¾?à?¤?­?à?¤?¿?à?¤?µ?à?¤?¿?à?¤?§?à?¤?¾?à?¤?µ?à?¥?‡?à?¤?µ?à?¤?¾?à?¤?™?à?¥???
?à?¤?¦?à?¥???à?¤?°?à?¤?·?à?¥???à?¤?Ÿ?à?¤?µ?à?¥???à?¤?¯?à?¤?ƒ?
?à?¤?•?à?¤?°?à?¥???à?¤?®?à?¤?µ?à?¤?¿?à?¤?·?à?¤?¯?à?¤?¤?à?¥???à?¤?µ?à?¤?¾?à?¤?¤?à?¥???
?à?¥?¤?
?à?¤?†?à?¤?¯?à?¥???à?¤?·?à?¥???à?¤?®?à?¤?¾?à?¤?¦?à?¤?¿?à?¤?¤?à?¤?¿?
?à?¤?•?à?¤?°?à?¥???à?¤?®?à?¤?¸?à?¥???

?à?¤?•?à?¤?¾?à?¤?¤?à?¥???à?¤?¯?à?¤?¾?à?¤?¯?à?¤?¨?à?¤?µ?à?¤?¾?à?¤?°?à?¥???à?¤?¤?à?¥???à?¤?¤?à?¤?¿?à?¤?•?à?¤?¾?à?¤?¤?à?¥???
?à?¥?¤?

? ? ?à?¤?•?à?¥?‡?à?¤?š?à?¤?¿?à?¤?¤?à?¥???
?à?¤?¤?à?¥?ƒ?à?¤?ª?à?¥???à?¤?¤?à?¤?¾?à?¤?¨?à?¥???
?à?¤?œ?à?¥???à?¤?ž?à?¤?¾?à?¤?¤?à?¥???à?¤?µ?à?¤?¾?à?¤?½?à?¤?¨?à?¥???à?¤?¨?à?¤?‚?
?à?¤?ª?à?¥???à?¤?°?à?¤?•?à?¥?€?à?¤?°?à?¥???à?¤?¯?
?à?¤?¸?à?¤?•?à?¥?ƒ?à?¤?¤?à?¥???à?¤?¸?à?¤?•?à?¥?ƒ?à?¤?¦?à?¤?ª?à?¥?‹?
?à?¤?¦?à?¤?¤?à?¥???à?¤?µ?à?¥?‡?à?¤?¤?à?¥???à?¤?¯?à?¤?ª?à?¤?¾?à?¤?‚?
?à?¤?¦?à?¤?¾?à?¤?¨?à?¤?®?à?¥???-?

?à?¤?ª?à?¤?¸?à?¥???à?¤?ª?à?¤?°?à?¥???à?¤?¶?à?¤?¨?à?¤?®?à?¤?¿?à?¤?š?à?¥???à?¤?›?à?¤?¾?à?¤?¨?à?¥???à?¤?¤?
?à?¥?¤? ?à?¤?¤?à?¤?¤?à?¥???
?à?¤?ª?à?¥???à?¤?¨?à?¤?°?à?¥???à?¤?¨?à?¤?¾?à?¤?¤?à?¥?€?à?¤?µ?
?à?¤?¶?à?¥?‹?à?¤?­?à?¤?¤?à?¥?‡? ?à?¥?¤?
?à?¤?‰?à?¤?ª?à?¤?¸?à?¥???à?¤?ª?à?¤?°?à?¥???à?¤?¶?à?¤?¨?à?¤?¶?à?¤?¬?à?¥???à?¤?¦?à?¤?¾?à?¤?­?à?¤?¾?-?

?à?¤?µ?à?¤?¾?à?¤?¤?à?¥??? ?à?¥?¤? ?à?¤?\ldots{}?à?¤?¤?à?¥???à?¤?°?
?à?¤?¶?à?¥???à?¤?°?à?¤?¾?à?¤?¦?à?¥???à?¤?§?à?¤?•?à?¤?°?à?¥???à?¤?¤?à?¥???à?¤?¤?à?¤?¾?
?â?€?œ? ?à?¤?†?à?¤?¯?à?¤?¨?à?¥???à?¤?¤?à?¥??? ?à?¤?¨? ?â?€???
?à?¤?‡?à?¤?¤?à?¤?¿?
?à?¤?œ?à?¤?ª?à?¤?¾?à?¤?¨?à?¤?¨?à?¥???à?¤?¤?à?¤?°?à?¤?‚?
?à?¤?\ldots{}?à?¤?ª?à?¤?¹?à?¥?ƒ?à?¤?¤?à?¤?¾?
?à?¤?\ldots{}?à?¤?¸?à?¥???à?¤?°?à?¤?¾?

?à?¤?°?à?¤?•?à?¥???à?¤?·?à?¤?¾?à?¤?‚?à?¤?¸?à?¤?¿?
?à?¤?µ?à?¥?‡?à?¤?¦?à?¤?¿?à?¤?·?à?¤?¦?â?€??? ?à?¤?‡?à?¤?¤?à?¤?¿?
?à?¤?®?à?¤?¨?à?¥???à?¤?¤?à?¥???à?¤?°?à?¥?‡?à?¤?£?
?à?¤?---?à?¥?ƒ?à?¤?¹?à?¤?---?à?¤?°?à?¥???à?¤?­?à?¥?‡?
?à?¤?¸?à?¤?°?à?¥???à?¤?µ?à?¤?¾?à?¤?¸?à?¥???
?à?¤?¦?à?¤?¿?à?¤?•?à?¥???à?¤?·?à?¥???
?à?¤?¤?à?¥???à?¤?°?à?¤?¿?à?¤?°?à?¤?ª?à?¥???à?¤?°?à?¤?¦?à?¤?•?à?¥???à?¤?·?à?¤?¿?à?¤?£?

?à?¤?¤?à?¤?¿?à?¤?²?à?¤?¤?à?¥???à?¤?°?à?¤?¿?à?¤?•?à?¤?°?à?¤?£?
?à?¤?•?à?¥???à?¤?°?à?¥???à?¤?¯?à?¤?¾?à?¤?¤?à?¥??? ?à?¥?¤?
?à?¤?¤?à?¤?¦?à?¥???à?¤?•?à?¥???à?¤?¤?à?¤?‚? ?à?¥?¤?

?à?¤?¬?à?¥???à?¤?°?à?¤?¹?à?¥???à?¤?®?à?¤?ª?à?¥???à?¤?°?à?¤?¾?à?¤?£?à?¥?‡?
?à?¥?¤?

? ? ? ?à?¤?œ?à?¤?ª?à?¥?‡?à?¤?¦?à?¤?¾?à?¤?¯?à?¤?¨?à?¥???à?¤?¤?à?¥???
?à?¤?¨? ?à?¤?‡?à?¤?¤?à?¤?¿? ?à?¤?®?à?¤?¨?à?¥???à?¤?¤?à?¥???à?¤?°?à?¤?‚?
?à?¤?¸?à?¤?®?à?¥???à?¤?¯?à?¤?---?à?¤?¶?à?¥?‡?à?¤?·?à?¤?¤?à?¤?ƒ? ?à?¥?¤?

? ? ? ?à?¤?°?à?¤?•?à?¥???à?¤?·?à?¤?¾?à?¤?°?à?¥???à?¤?¥?à?¤?‚?
?à?¤?ª?à?¤?¿?à?¤?¤?à?¥?ƒ?à?¤?¸?à?¤?¤?à?¥???à?¤?°?à?¤?¸?à?¥???à?¤?¯?
?à?¤?¤?à?¥???à?¤?°?à?¤?¿? ?à?¤?•?à?¥?ƒ?à?¤?¤?à?¥???à?¤?µ?à?¤?¾?
?à?¤?¸?à?¤?°?à?¥???à?¤?µ?à?¤?¤?à?¥?‹? ?à?¤?¦?à?¤?¿?à?¤?¶?à?¤?®?à?¥???
?\textless{}?/?s?p?a?n?\textgreater{}?

?à?¥?¥?\textless{}?/?s?p?a?n?\textgreater{}?

?

? ? ? ?à?¤?¤?à?¤?¿?à?¤?²?à?¤?¾?à?¤?‚?à?¤?¸?à?¥???à?¤?¤?à?¥???
?à?¤?ª?à?¥???à?¤?°?à?¤?•?à?¥???à?¤?·?à?¤?¿?à?¤?ª?à?¤?¨?à?¥???à?¤?®?à?¤?¨?à?¥???à?¤?¤?à?¥???à?¤?°?à?¥?ˆ?à?¤?°?à?¥???à?¤?š?à?¥???à?¤?š?à?¤?¾?à?¤?°?à?¥???à?¤?¯?à?¤?¾?à?¤?ª?à?¤?¹?à?¤?¤?à?¤?¾?
?à?¤?‡?à?¤?¤?à?¤?¿?
?à?¥?¤?\textless{}?/?s?p?a?n?\textgreater{}?\textless{}?/?p?\textgreater{}?
?

?

?à?¤?¤?à?¤?¥?à?¤?¾? ?à?¥?¤?

? ? ? ?à?¤?¤?à?¤?¤?à?¤?¸?à?¥???à?¤?¤?à?¤?¿?à?¤?²?à?¤?¾?à?¤?¨?à?¥???
?à?¤?---?à?¥?ƒ?à?¤?¹?à?¥?‡? ?à?¤?¤?à?¤?¸?à?¥???à?¤?®?à?¤?¿?à?¤?¨?à?¥???
?à?¤?µ?à?¤?¿?à?¤?•?à?¤?¿?à?¤?°?à?¥?‡?à?¤?š?à?¥???à?¤?š?à?¤?¾?à?¤?ª?à?¥???à?¤?°?à?¤?¦?à?¤?•?à?¥???à?¤?·?à?¤?¿?à?¤?£?à?¤?®?à?¥???
?à?¥?¤?

? ? ? ?à?¤?¶?à?¥???à?¤?°?à?¤?¦?à?¥???à?¤?§?à?¤?¯?à?¤?¾?
?à?¤?ª?à?¤?°?à?¤?¯?à?¤?¾? ?à?¤?¯?à?¥???à?¤?•?à?¥???à?¤?¤?à?¥?‹?
?à?¤?œ?à?¤?ª?à?¤?¨?à?¥???à?¤?¨?à?¤?ª?à?¤?¹?à?¤?¤?à?¤?¾? ?à?¥?¤?
?à?¤?‡?à?¤?¤?à?¤?¿? ?à?¥?¥?

?à?¤?¤?à?¤?¤?à?¤?ƒ? ?-? ?à?¤?†?à?¤?¯?à?¤?¨?à?¥???à?¤?¤?à?¥??? ?à?¤?¨?
?à?¤?‡?à?¤?¤?à?¤?¿?
?à?¤?‰?à?¤?ª?à?¥?‹?à?¤?¤?à?¥???à?¤?¤?à?¤?°?à?¤?®?à?¥??? ?à?¥?¤?
?à?¤?‡?à?¤?¤?à?¤?¿?
?à?¤?†?à?¤?µ?à?¤?¾?à?¤?¹?à?¤?¨?à?¤?µ?à?¤?¿?à?¤?§?à?¤?¿? ?à?¥?¤?

? ? ? ? ? ? ? ? ? ? ? ? ? ? ?
?à?¤?\ldots{}?à?¤?¥?à?¤?¾?à?¤?§?à?¥???à?¤?¯?à?¤?¾?à?¤?¦?à?¥???à?¤?¯?à?¥???à?¤?ª?à?¤?š?à?¤?¾?à?¤?°?à?¤?µ?à?¤?¿?à?¤?§?à?¤?¿?à?¤?ƒ?
?à?¥?¤?

?à?¤?•?à?¤?¾?à?¤?¤?à?¥???à?¤?¯?à?¤?¾?à?¤?¯?à?¤?¨?à?¤?ƒ? ?à?¥?¤?

? ? ? ?
?à?¤?¯?à?¤?œ?à?¥???à?¤?ž?à?¤?¿?à?¤?¯?à?¤?µ?à?¥?ƒ?à?¤?•?à?¥???à?¤?·?à?¤?š?à?¤?®?à?¤?¸?à?¥?‡?à?¤?·?à?¥???
?à?¤?ª?à?¤?µ?à?¤?¿?à?¤?¤?à?¥???à?¤?°?à?¤?¾?à?¤?¨?à?¥???à?¤?¤?à?¤?°?à?¥???à?¤?¹?à?¤?¿?à?¤?¤?à?¥?‡?à?¤?·?à?¥???à?¤?µ?à?¥?‡?à?¤?•?à?¥?ˆ?à?¤?•?à?¤?¸?à?¥???à?¤?®?à?¤?¿?à?¤?¨?à?¥???à?¤?¨?à?¤?ª?
?à?¤?†?à?¤?¸?à?¤?¿?à?¤?ž?à?¥???à?¤?š?à?¤?¤?à?¤?¿? ?â?€?œ?à?¤?¶?-?

?à?¤?¨?à?¥???à?¤?¨?à?¥?‹?à?¤?¦?à?¥?‡?à?¤?µ?à?¥?€?à?¤?°?à?¤?¿?à?¤?¤?à?¤?¿?
?â?€???
?à?¤???à?¤?•?à?¥?ˆ?à?¤?•?à?¤?¸?à?¥???à?¤?®?à?¤?¿?à?¤?¨?à?¥???à?¤?¨?à?¥?‡?à?¤?µ?
?à?¤?¤?à?¤?¿?à?¤?²?à?¤?¾?à?¤?¨?à?¤?¾?à?¤?µ?à?¤?ª?à?¤?¤?à?¤?¿? ?à?¥?¤?

? ? ? ? ?à?¤?¤?à?¤?¿?à?¤?²?à?¥?‹?à?¤?½?à?¤?¸?à?¤?¿?
?à?¤?¸?à?¥?‹?à?¤?®?à?¤?¦?à?¥?‡?à?¤?µ?à?¤?¤?à?¥???à?¤?¯?à?¥?‹?
?à?¤?---?à?¥?‹?à?¤?¸?à?¤?µ?à?¥?‹?
?à?¤?¦?à?¥?‡?à?¤?µ?à?¤?¨?à?¤?¿?à?¤?°?à?¥???à?¤?®?à?¤?¿?à?¤?¤?à?¤?ƒ?
?à?¥?¤?

? ? ? ?
?à?¤?ª?à?¥???à?¤?°?à?¤?¤?à?¥???à?¤?¨?à?¤?®?à?¤?¦?à?¥???à?¤?­?à?¤?¿?à?¤?ƒ?
?à?¤?ª?à?¥?ƒ?à?¤?•?à?¥???à?¤?¤?à?¤?ƒ?
?à?¤?¸?à?¥???à?¤?µ?à?¤?§?à?¤?¯?à?¤?¾?
?à?¤?ª?à?¤?¿?à?¤?¤?à?¥?ƒ?à?¤?²?à?¥???à?¤???à?¤?²?à?¥?‹?à?¤?•?à?¤?¾?à?¤?¨?à?¥???
?à?¤?ª?à?¥???à?¤?°?à?¥?€?à?¤?£?à?¤?¾?à?¤?¹?à?¤?¿? ?à?¤?¨?à?¤?ƒ?
?à?¤?¸?à?¥???à?¤?µ?à?¤?¾?à?¤?¹?à?¤?¾? ?à?¥?¤?

?à?¤?‡?à?¤?¤?à?¤?¿? ?à?¥?¤?
?à?¤?¸?à?¥?Œ?à?¤?µ?à?¤?£?à?¤?°?à?¤?¾?à?¤?œ?à?¤?¤?à?¥?Œ?à?¤?¦?à?¥???à?¤?®?à?¥???à?¤?¬?à?¤?°?à?¤?--?à?¤?™?à?¥???à?¤?---?à?¤?®?à?¤?£?à?¤?¿?à?¤?®?à?¤?¯?à?¤?¾?à?¤?¨?à?¤?¾?à?¤?‚?
?à?¤?ª?à?¤?¾?à?¤?¤?à?¥???à?¤?°?à?¤?¾?à?¤?£?à?¤?¾?à?¤?®?à?¤?¨?à?¥???à?¤?¯?à?¤?¤?à?¤?®?à?¥?‡?à?¤?·?à?¥???

?à?¤?¯?à?¤?¾?à?¤?¨?à?¤?¿? ?à?¤?µ?à?¤?¾?
?à?¤?µ?à?¤?¿?à?¤?¦?à?¥???à?¤?¯?à?¤?¨?à?¥???à?¤?¤?à?¥?‡?
?à?¤?ª?à?¤?¤?à?¥???à?¤?°?à?¤?ª?à?¥???à?¤?Ÿ?à?¥?‡?à?¤?·?à?¥???
?à?¤?µ?à?¤?¾?
?à?¤???à?¤?•?à?¥?ˆ?à?¤?•?à?¤?¸?à?¥???à?¤?µ?à?¥?ˆ?à?¤?•?à?¥?‡?à?¤?•?à?¥?‡?à?¤?¨?
?à?¤?¦?à?¤?¦?à?¤?¾?à?¤?¤?à?¤?¿?
?à?¤?¸?à?¤?ª?à?¤?µ?à?¤?¿?à?¤?¤?à?¥???à?¤?°?à?¥?‡?à?¤?·?à?¥??? ?

?à?¤?¹?à?¤?¸?à?¥???à?¤?¤?à?¥?‡?à?¤?·?à?¥??? ?â?€?œ? ?à?¤?¯?à?¤?¾?
?à?¤?¦?à?¤?¿?à?¤?µ?à?¥???à?¤?¯?à?¤?¾? ?à?¤?†?à?¤?ª?à?¤?ƒ?
?à?¤?ª?à?¤?¯?à?¤?¸?à?¤?¾?
?à?¤?¸?à?¤?®?à?¥???à?¤?¬?à?¤?­?à?¥?‚?à?¤?µ?à?¥???à?¤?°?à?¥???à?¤?¯?à?¤?¾?
?à?¤?\ldots{}?à?¤?¨?à?¥???à?¤?¤?à?¤?°?à?¤?¿?à?¤?•?à?¥???à?¤?·?à?¤?¾?
?à?¤?‰?à?¤?¤? ?à?¤?ª?à?¤?¾?à?¤?°?à?¥???à?¤?¥?à?¤?¿?à?¤?µ?à?¥?€?

?à?¤?°?à?¥???à?¤?¯?à?¤?¾?à?¤?ƒ? ?à?¥?¤?
?à?¤?¹?à?¤?¿?à?¤?°?à?¤?£?à?¥???à?¤?¯?à?¤?µ?à?¤?°?à?¥???à?¤?£?à?¤?¾?
?à?¤?¯?à?¤?œ?à?¥???à?¤?ž?à?¤?¿?à?¤?¯?à?¤?¾?à?¤?¸?à?¥???à?¤?¤?à?¤?¾?à?¤?¨?
?à?¤?†?à?¤?ª?à?¤?ƒ? ?à?¤?¶?à?¤?¿?à?¤?µ?à?¤?¾?à?¤?ƒ? ?à?¤?¶?à?¤?‚?
?à?¤?¸?à?¥???à?¤?¯?à?¥?‹?à?¤?¨?à?¤?¾?à?¤?ƒ?
?à?¤?¸?à?¥???à?¤?¹?à?¤?µ?à?¤?¾? ?à?¤?­?à?¤?µ?-?

?à?¤?¨?à?¥???à?¤?¤?à?¥???â?€??? ?à?¥?¤? ?à?¤?‡?à?¤?¤?à?¤?¿? ?à?¥?¤?
?à?¤???à?¤?·?à?¤?¤?à?¥?‡?à?¤?½?à?¤?°?à?¥???à?¤?§? ?à?¤?‡?à?¤?¤?à?¤?¿?
?à?¥?¤?
?à?¤?¯?à?¤?œ?à?¥???à?¤?ž?à?¤?¿?à?¤?¯?à?¤?µ?à?¥?ƒ?à?¤?•?à?¥???à?¤?·?à?¤?¾?à?¤?ƒ?
?à?¤?š?à?¤?®?à?¤?¸?à?¤?¾?à?¤?¶?à?¥???à?¤?š?à?¥?‹?à?¤?ª?à?¤?•?à?¤?°?à?¤?£?à?¥?‡?
?à?¤?µ?à?¥???à?¤?¯?à?¤?¾?à?¤?--?à?¥???à?¤?¯?à?¤?¾?

?à?¤?¤?à?¤?¾?à?¤?ƒ? ?à?¥?¤? ?à?¤?¤?à?¥?‡?à?¤?·?à?¤?¾?à?¤?‚? ?à?¤?š?
?à?¤?¸?à?¥???à?¤?¥?à?¤?¾?à?¤?ª?à?¤?¨?
?à?¤?•?à?¥???à?¤?¶?à?¥?‹?à?¤?ª?à?¤?°?à?¤?¿?
?à?¤?•?à?¤?¾?à?¤?°?à?¥???à?¤?¯?à?¤?‚? ?,?
?à?¤?¦?à?¤?•?à?¥???à?¤?·?à?¤?¿?à?¤?£?à?¤?¾?à?¤?---?à?¥???à?¤?°?à?¥?‡?à?¤?·?à?¥???
?à?¤?•?à?¥???à?¤?¶?à?¥?‡?à?¤?·?à?¥???
?à?¤?¨?à?¤?¿?à?¤?§?à?¤?¾?à?¤?¯?à?¥?‡?à?¤?¤?à?¤?¿?

?à?¤?¬?à?¥?ˆ?à?¤?œ?à?¤?µ?à?¤?¾?à?¤?ª?à?¤?---?à?¥?ƒ?à?¤?¹?à?¥???à?¤?¯?à?¥?‹?à?¤?•?à?¥???à?¤?¤?à?¥?‡?à?¤?ƒ?
?à?¥?¤?
?à?¤?¦?à?¤?•?à?¥???à?¤?·?à?¤?¿?à?¤?£?à?¤?¾?à?¤?---?à?¥???à?¤?°?à?¤?¤?à?¥???à?¤?µ?à?¤?‚?
?à?¤?š? ?à?¤?ª?à?¤?¿?à?¤?¤?à?¥?„?à?¤?£?à?¤?¾?à?¤?‚?
?à?¤?¦?à?¥?ˆ?à?¤?µ?à?¥?‡?à?¤?·?à?¥???
?à?¤?ª?à?¥???à?¤?°?à?¤?¾?à?¤?---?à?¤?---?à?¥???à?¤?°?à?¥?‡?à?¤?·?à?¥???
?à?¤?¨?à?¤?¿?à?¤?§?à?¤?¾?à?¤?¨?à?¤?®?à?¥???

?à?¤?ª?à?¤?µ?à?¤?¿?à?¤?¤?à?¥???à?¤?°?à?¤?¾?à?¤?¨?à?¥???à?¤?¤?à?¤?°?à?¥???à?¤?¹?à?¤?¿?à?¤?¤?à?¥?‡?à?¤?·?à?¥???à?¤?µ?à?¤?¿?à?¤?¤?à?¤?¿?
?à?¥?¤? ?à?¤?ª?à?¤?µ?à?¤?¿?à?¤?¤?à?¥???à?¤?°? ?à?¤?š?
?à?¤?ª?à?¤?°?à?¤?¿?à?¤?­?à?¤?¾?à?¤?·?à?¤?¾?à?¤?¯?à?¤?¾?à?¤?‚?
?à?¤?µ?à?¥???à?¤?¯?à?¤?¾?à?¤?--?à?¥???à?¤?¯?à?¤?¾?à?¤?¤?à?¤?®?à?¥???
?à?¥?¤? ?à?¤?ª?à?¤?µ?à?¤?¿?à?¤?¤?à?¥???à?¤?°?à?¤?¾?à?¤?£?à?¤?¾?à?¤?‚?

?à?¤?š? ?à?¤?¸?à?¤?‚?à?¤?--?à?¥???à?¤?¯?à?¤?¾?
?à?¤?‰?à?¤?•?à?¥???à?¤?¤?à?¤?¾?à?¤?ƒ?
?à?¤?š?à?¤?¤?à?¥???à?¤?°?à?¥???à?¤?µ?à?¤?¿?à?¤?‚?à?¤?¶?à?¤?¤?à?¤?¿?à?¤?®?à?¤?¤?à?¥?‡?
?à?¥?¤?\textless{}?/?s?p?a?n?\textgreater{}?\textless{}?/?p?\textgreater{}?\textless{}?/?b?o?d?y?\textgreater{}?\textless{}?/?h?t?m?l?\textgreater{}?
?

? ?b?o?d?y?\{? ?w?i?d?t?h?:? ?2?1?c?m?;? ?h?e?i?g?h?t?:? ?2?9?.?7?c?m?;?
?m?a?r?g?i?n?:? ?3?0?m?m? ?4?5?m?m? ?3?0?m?m? ?4?5?m?m?;? ?\}?
?\textless{}?/?s?t?y?l?e?\textgreater{}?\textless{}?!?D?O?C?T?Y?P?E?
?H?T?M?L? ?P?U?B?L?I?C? ?"?-?/?/?W?3?C?/?/?D?T?D? ?H?T?M?L?
?4?.?0?/?/?E?N?"?
?"?h?t?t?p?:?/?/?w?w?w?.?w?3?.?o?r?g?/?T?R?/?R?E?C?-?h?t?m?l?4?0?/?s?t?r?i?c?t?.?d?t?d?"?\textgreater{}?
?

?

?

?

? ?p?,? ?l?i? ?\{? ?w?h?i?t?e?-?s?p?a?c?e?:? ?p?r?e?-?w?r?a?p?;? ?\}?
?\textless{}?/?s?t?y?l?e?\textgreater{}?\textless{}?/?h?e?a?d?\textgreater{}?

? ?

?

? ? ? ? ? ? ? ? ? ? ? ? ? ? ? ?\textless{}?/?s?p?a?n?\textgreater{}?

?à?¤?\ldots{}?à?¤?°?à?¥???à?¤?˜?à?¤?¸?à?¤?‚?à?¤?ª?à?¤?¾?à?¤?¦?à?¤?¨?à?¤?ª?à?¥???à?¤?°?à?¤?•?à?¤?¾?à?¤?°?à?¤?ƒ?
?à?¥?¤? ? ? ? ? ? ? ? ? ? ? ? ? ? ? ?
?à?¥?¨?à?¥?¦?à?¥?¯?\textless{}?/?s?p?a?n?\textgreater{}?

?

?à?¤?¦?à?¥???à?¤?µ?à?¥?‡? ?à?¤?¦?à?¥???à?¤?µ?à?¥?‡?
?à?¤?¶?à?¤?²?à?¤?¾?à?¤?•?à?¥?‡?
?à?¤?¦?à?¥?‡?à?¤?µ?à?¤?¾?à?¤?¨?à?¤?¾?à?¤?‚?
?à?¤?ª?à?¤?¾?à?¤?¤?à?¥???à?¤?°?à?¥?‡?
?à?¤?•?à?¥?ƒ?à?¤?¤?à?¥???à?¤?µ?à?¤?¾? ?à?¤?ª?à?¤?¯?à?¤?ƒ?
?à?¤?•?à?¥???à?¤?·?à?¤?¿?à?¤?ª?à?¥?‡?à?¤?¤?à?¥??? ?à?¥?¤?

?à?¤?¶?à?¤?¨?à?¥???à?¤?¨?à?¥?‹?à?¤?¦?à?¥?‡?à?¤?µ?à?¥???à?¤?¯?à?¤?¾?,?
?à?¤?¯?à?¤?µ?à?¥?‹?à?¤?½?à?¤?¸?à?¥?€?à?¤?¤?à?¤?¿?
?à?¤?¯?à?¤?µ?à?¤?¾?à?¤?¨?à?¤?ª?à?¤?¿? ?à?¤?¤?à?¤?¤?à?¤?ƒ?
?à?¤?•?à?¥???à?¤?·?à?¤?¿?à?¤?ª?à?¥?‡?à?¤?¤?à?¥??? ?à?¥?¥?

?à?¤?ª?à?¥???à?¤?·?à?¥???à?¤?ª?à?¤?§?à?¥?‚?à?¤?ª?à?¤?¾?à?¤?¦?à?¤?¿?à?¤?­?à?¤?¿?à?¤?ƒ?
?à?¤?ª?à?¥?‚?à?¤?œ?à?¤?¾?à?¤?‚? ?à?¤?•?à?¥?ƒ?à?¤?¤?à?¥???à?¤?µ?à?¤?¾?
?à?¤?ª?à?¤?¾?à?¤?¤?à?¥???à?¤?°?à?¥?‡?à?¤?·?à?¥???
?à?¤?®?à?¤?¾?à?¤?¨?à?¤?µ?à?¤?ƒ? ?à?¥?¤?

?à?¤?ª?à?¤?¿?à?¤?¤?à?¥?ƒ?à?¤?ª?à?¤?¾?à?¤?¤?à?¥???à?¤?°?à?¥?‡?
?à?¤?µ?à?¤?¿?à?¤?¶?à?¥?‡?à?¤?·?à?¥?‹?à?¤?½?à?¤?¯?à?¤?‚?
?à?¤?¤?à?¤?¿?à?¤?²?à?¥?‹?à?¤?½?à?¤?¸?à?¥?€?à?¤?¤?à?¤?¿?
?à?¤?¤?à?¤?¿?à?¤?²?à?¤?¾?à?¤?¨?à?¥???
?à?¤?•?à?¥???à?¤?·?à?¤?¿?à?¤?ª?à?¥?‡?à?¤?¤?à?¥??? ?à?¥?¥?

?à?¤?¤?à?¤?¿?à?¤?¸?à?¥???à?¤?°?à?¤?¸?à?¥???à?¤?¤?à?¤?¿?à?¤?¸?à?¥???à?¤?°?à?¤?ƒ?
?à?¤?¶?à?¤?²?à?¤?¾?à?¤?•?à?¤?¾?à?¤?¸?à?¥???à?¤?¤?à?¥???
?à?¤?ª?à?¤?¿?à?¤?¤?à?¥?ƒ?à?¤?ª?à?¤?¾?à?¤?¤?à?¥???à?¤?°?à?¥?‡?à?¤?·?à?¥???
?à?¤?ª?à?¤?¾?à?¤?°?à?¥???à?¤?µ?à?¤?£?à?¥?‡? ?à?¥?¥?

?à?¤?\ldots{}?à?¤?¤?à?¥???à?¤?°?
?à?¤?ª?à?¤?µ?à?¤?¿?à?¤?¤?à?¥???à?¤?°?à?¤?•?à?¤?°?à?¤?£?à?¥?‡?
?à?¤?®?à?¤?¨?à?¥???à?¤?¤?à?¥???à?¤?°?à?¤?®?à?¤?°?à?¥???à?¤?§?à?¤?ª?à?¤?¾?à?¤?¤?à?¥???à?¤?°?à?¥?‹?à?¤?ª?à?¤?°?à?¤?¿?
?à?¤?š?
?à?¤?ª?à?¤?µ?à?¤?¿?à?¤?¤?à?¥???à?¤?°?à?¤?¨?à?¤?¿?à?¤?§?à?¤?¾?à?¤?¨?à?¤?®?à?¤?¾?à?¤?¹?
?-?

?à?¤?ª?à?¥???à?¤?°?à?¤?š?à?¥?‡?à?¤?¤?à?¤?¾?à?¤?ƒ? ?à?¥?¤?

? ? ? ? ?à?¤?ª?à?¤?µ?à?¤?¿?à?¤?¤?à?¥???à?¤?°?à?¥?‡? ?à?¤?¸?à?¥???à?¤?¥?
?à?¤?‡?à?¤?¤?à?¤?¿? ?à?¤?®?à?¤?¨?à?¥???à?¤?¤?à?¥???à?¤?°?à?¥?‡?à?¤?£?
?à?¤?ª?à?¤?µ?à?¤?¿?à?¤?¤?à?¥???à?¤?°?à?¥?‡?
?à?¤?•?à?¤?¾?à?¤?°?à?¤?¯?à?¥?‡?à?¤?¦?à?¥??? ?à?¤?¬?à?¥???à?¤?§?à?¤?ƒ?
?à?¥?¤?

? ? ? ? ?à?¤?¤?à?¥?‡?
?à?¤?¨?à?¤?¿?à?¤?§?à?¤?¾?à?¤?¯?à?¤?¾?à?¤?°?à?¥???à?¤?§?à?¤?ª?à?¤?¾?à?¤?¤?à?¥???à?¤?°?à?¥?‡?à?¤?·?à?¥???
?à?¤?¶?à?¤?¨?à?¥???à?¤?¨?à?¥?‹?à?¤?¦?à?¥?‡?à?¤?µ?à?¤?¤?à?¥???à?¤?¯?à?¤?¿?à?¤?ª?à?¤?ƒ?
?à?¤?•?à?¥???à?¤?·?à?¤?¿?à?¤?ª?à?¥?‡?à?¤?¤?à?¥???
?\textless{}?/?s?p?a?n?\textgreater{}?

?à?¥?¥?\textless{}?/?s?p?a?n?\textgreater{}?

?

?à?¤?ª?à?¤?µ?à?¤?¿?à?¤?¤?à?¥???à?¤?°?à?¥?‡? ?à?¤?¸?à?¥???à?¤?¥?
?à?¤?‡?à?¤?¤?à?¤?¿? ?à?¤?¯?à?¤?œ?à?¥???à?¤?ƒ?
?à?¤?¶?à?¤?¾?à?¤?--?à?¤?¾?à?¤?­?à?¥?‡?à?¤?¦?à?¥?‡?à?¤?¨?
?à?¤?µ?à?¥???à?¤?¯?à?¤?µ?à?¤?¸?à?¥???à?¤?¥?à?¤?¿?à?¤?¤?à?¤?‚?,?
?à?¤?ª?à?¤?µ?à?¤?¿?à?¤?¤?à?¥???à?¤?°?à?¥?‡? ?à?¤?¸?à?¥???à?¤?¥?à?¥?‹?
?à?¤?µ?à?¥?ˆ?à?¤?·?à?¥???à?¤?£?à?¤?µ?à?¥?€?

?à?¤?µ?à?¤?¾?à?¤?¯?à?¥???à?¤?°?à?¥???à?¤?µ?à?¥?‹?
?à?¤?®?à?¤?¨?à?¤?¸?à?¤?¾?
?à?¤?ª?à?¥???à?¤?¨?à?¤?¾?à?¤?¤?à?¥???à?¤?µ?à?¤?¿?à?¤?¤?à?¤?¿?,?
?à?¤?¤?à?¤?¥?à?¤?¾? ?à?¤?ª?à?¤?µ?à?¤?¿?à?¤?¤?à?¥???à?¤?°?à?¥?‡?
?à?¤?¸?à?¥???à?¤?¥?à?¥?‹?
?à?¤?µ?à?¥?ˆ?à?¤?·?à?¥???à?¤?£?à?¤?µ?à?¥???à?¤?¯?à?¥?Œ?
?à?¤?‡?à?¤?¤?à?¤?¿? ?à?¥?¤? ?à?¤?\ldots{}?à?¤?¤?à?¥???à?¤?°?

?à?¤?š? ?à?¤?ª?à?¤?µ?à?¤?¿?à?¤?¤?à?¥???à?¤?°?à?¥?‡? ?à?¤?¸?à?¥???à?¤?¥?
?à?¤?‡?à?¤?¤?à?¤?¿?
?à?¤?¦?à?¥???à?¤?µ?à?¤?¿?à?¤?µ?à?¤?š?à?¤?¨?à?¤?¾?à?¤?¨?à?¥???à?¤?¤?à?¥?‹?à?¤?½?à?¤?ª?à?¤?¿?
?à?¤?®?à?¤?¨?à?¥???à?¤?¤?à?¥???à?¤?°?à?¤?ƒ? ?,?
?à?¤?¤?à?¥???à?¤?°?à?¤?¿?à?¤?¤?à?¥???à?¤?µ?à?¤?¯?à?¥???à?¤?•?à?¥???à?¤?¤?à?¥?‡?
?à?¤?ª?à?¤?¿?à?¤?¤?à?¥???à?¤?°?à?¥???à?¤?¯?à?¤?ª?à?¤?µ?à?¤?¿?à?¤?¤?à?¥???à?¤?°?à?¥?‡?-?

?à?¤?½?à?¤?ª?à?¤?¿? ?à?¤?ª?à?¥???à?¤?°?à?¤?¯?à?¥?‹?à?¤?œ?à?¥???à?¤?¯?
?à?¤???à?¤?µ?,?
?à?¤?ª?à?¤?¾?à?¤?¶?à?¤?¾?à?¤?§?à?¤?¿?à?¤?•?à?¤?°?à?¤?£?à?¤?¨?à?¥???à?¤?¯?à?¤?¾?à?¤?¯?à?¤?¾?à?¤?¦?à?¤?¿?à?¤?¤?à?¤?¿?
?à?¤?•?à?¤?°?à?¥???à?¤?•?à?¤?ƒ? ?à?¥?¤?

? ? ? ?à?¤?\ldots{}?à?¤?¨?à?¥???à?¤?¯?à?¥?‡? ?à?¤?¤?à?¥???
?à?¤?¬?à?¤?¹?à?¥???à?¤?·?à?¥???
?à?¤?¦?à?¥???à?¤?µ?à?¤?¿?à?¤?µ?à?¤?š?à?¤?¨?à?¤?¸?à?¥???à?¤?¯?à?¤?¾?à?¤?¸?à?¤?¾?à?¤?§?à?¥???à?¤?¤?à?¥???à?¤?µ?à?¤?¾?à?¤?²?à?¥???à?¤?²?
?à?¤?ª? ?à?¤???à?¤?µ?,?
?à?¤?®?à?¥?‡?à?¤?·?à?¥???à?¤?¯?à?¤?§?à?¤?¿?à?¤?•?à?¤?°?à?¤?£?à?¤?¨?à?¥???à?¤?¯?à?¤?¾?-?

?à?¤?¯?à?¥?‡?à?¤?¨?à?¥?‡?à?¤?¤?à?¥???à?¤?¯?à?¤?¾?à?¤?¹?à?¥???à?¤?ƒ?
?à?¥?¤?

? ? ? ?à?¤?\ldots{}?à?¤?¤?à?¥???à?¤?°?
?à?¤?•?à?¤?¾?à?¤?¤?à?¥???à?¤?¯?à?¤?¾?à?¤?¯?à?¤?¨?à?¤?¾?à?¤?¦?à?¤?¿?à?¤?­?à?¤?¿?à?¤?°?à?¥???à?¤?¦?à?¤?°?à?¥???à?¤?¶?à?¤?ª?à?¥?‚?à?¤?°?à?¥???à?¤?£?à?¤?®?à?¤?¾?à?¤?¸?à?¤?ª?à?¥???à?¤?°?à?¤?•?à?¤?°?à?¤?£?à?¥?‡?
?à?¤?¸?à?¤?®?à?¤?¨?à?¥???à?¤?¤?à?¥???à?¤?°?à?¤?•?à?¤?¸?à?¥???à?¤?¯?
?à?¤?ª?à?¤?µ?à?¤?¿?à?¤?¤?à?¥???à?¤?°?à?¤?•?à?¤?°?-?

?à?¤?£?à?¤?¸?à?¥???à?¤?¯?à?¤?¾?à?¤?®?à?¥???à?¤?¨?à?¤?¾?à?¤?¤?à?¤?¤?à?¥???à?¤?µ?à?¤?¾?à?¤?š?à?¥???à?¤?›?à?¥???à?¤?°?à?¤?¾?à?¤?¦?à?¥???à?¤?§?à?¥?‡?
?à?¤?š?à?¥?‹?à?¤?ª?à?¤?¦?à?¥?‡?à?¤?¶?à?¤?¾?à?¤?¤?à?¤?¿?à?¤?¦?à?¥?‡?à?¤?¶?à?¤?¯?à?¥?‹?à?¤?°?à?¤?­?à?¤?¾?à?¤?µ?à?¤?¾?à?¤?¦?à?¤?®?à?¤?¨?à?¥???à?¤?¤?à?¥???à?¤?°?à?¤?•?à?¤?®?à?¥?‡?à?¤?µ?,?
?à?¤?ª?à?¥???à?¤?°?à?¤?¯?à?¥?‹?-?

?à?¤?---?à?¤?¾?à?¤?¤?à?¥???à?¤?ª?à?¥?‚?à?¤?°?à?¥???à?¤?µ?à?¥?‡?
?à?¤?›?à?¥?‡?à?¤?¦?à?¤?¨?à?¤?®?à?¤?¿?à?¤?¤?à?¤?¿?
?à?¤?®?à?¥?ˆ?à?¤?¥?à?¤?¿?à?¤?²?à?¤?¾?à?¤?ƒ? ?à?¥?¤?
?à?¤?¤?à?¤?¨?à?¥???à?¤?¨? ?\textless{}?/?s?p?a?n?\textgreater{}?

?à?¥?¤?\textless{}?/?s?p?a?n?\textgreater{}?

?
?à?¤?‰?à?¤?¦?à?¤?¾?à?¤?¹?à?¥?ƒ?à?¤?¤?à?¤?ª?à?¥???à?¤?°?à?¤?š?à?¥?‡?à?¤?¤?à?¥?‹?à?¤?µ?à?¤?š?à?¤?¨?à?¤?µ?à?¤?¿?à?¤?°?à?¥?‹?à?¤?§?à?¤?¾?à?¤?¤?à?¥???
?à?¥?¤?

?à?¤?\ldots{}?à?¤?¤?à?¥???à?¤?°? ?à?¤?š? ?à?¤?¨?
?à?¤?¬?à?¥???à?¤?°?à?¤?¾?à?¤?¹?à?¥???à?¤?®?à?¤?£?à?¤?­?à?¥?‡?à?¤?¦?à?¥?‡?à?¤?¨?
?à?¤?ª?à?¤?¾?à?¤?¤?à?¥???à?¤?°?à?¤?­?à?¥?‡?à?¤?¦?à?¤?ƒ?,?
?à?¤?•?à?¤?¿?à?¤?¨?à?¥???à?¤?¤?à?¥???
?à?¤?¦?à?¥?‡?à?¤?µ?à?¤?¤?à?¤?¾?à?¤?­?à?¥?‡?à?¤?¦?à?¥?‡?à?¤?¨?à?¥?ˆ?à?¤?µ?
?à?¥?¤?
?à?¤?µ?à?¥?ˆ?à?¤?¶?à?¥???à?¤?µ?à?¤?¦?à?¥?‡?à?¤?µ?à?¤?¿?à?¤?•?à?¥?‡?

?à?¤?¤?à?¥???
?à?¤?¦?à?¥?‡?à?¤?µ?à?¤?¤?à?¥?ˆ?à?¤?•?à?¥???à?¤?¯?à?¥?‡?à?¤?½?à?¤?ª?à?¤?¿?
?à?¤?ª?à?¤?¾?à?¤?¤?à?¥???à?¤?°?à?¤?¦?à?¥???à?¤?µ?à?¤?¯?à?¤?‚?
?à?¤?¤?à?¤?¥?à?¤?¾? ?à?¤?š?-?

?à?¤?ª?à?¤?¾?à?¤?¦?à?¥???à?¤?®?à?¤?®?à?¤?¾?à?¤?¤?à?¥???à?¤?¸?à?¥???à?¤?¯?à?¤?¯?à?¥?‹?à?¤?°?à?¥???à?¤?•?à?¥???à?¤?¤?à?¤?‚?-?

? ? ? ?
?à?¤?µ?à?¤?¿?à?¤?¶?à?¥???à?¤?µ?à?¥?‡?à?¤?¦?à?¥?‡?à?¤?µ?à?¤?¾?à?¤?¨?à?¥???
?à?¤?¯?à?¤?µ?à?¥?ˆ?à?¤?ƒ?
?à?¤?ª?à?¥???à?¤?·?à?¥???à?¤?ª?à?¥?ˆ?à?¤?°?à?¤?­?à?¥???à?¤?¯?à?¤?°?à?¥???à?¤?¥?à?¥???à?¤?¯?à?¤?¾?à?¤?¸?à?¤?¨?à?¤?ª?à?¥?‚?à?¤?°?à?¥???à?¤?µ?à?¤?•?à?¤?®?à?¥???
?à?¥?¤?

? ? ? ? ?à?¤?ª?à?¥?‚?à?¤?°?à?¤?¯?à?¥?‡?à?¤?¤?à?¥???
?à?¤?ª?à?¤?¾?à?¤?¤?à?¥???à?¤?°?à?¤?¯?à?¥???à?¤?---?à?¥???à?¤?®?à?¤?‚?
?à?¤?¤?à?¥??? ?à?¤?¸?à?¥???à?¤?¥?à?¤?¾?à?¤?ª?à?¥???à?¤?¯?à?¤?‚?
?à?¤?¦?à?¤?°?à?¥???à?¤?­?à?¤?ª?à?¤?µ?à?¤?¿?à?¤?¤?à?¥???à?¤?°?à?¤?•?à?¥?‡?
?\textless{}?/?s?p?a?n?\textgreater{}?

?à?¥?¥?\textless{}?/?s?p?a?n?\textgreater{}?

?

? ?
?à?¤?¶?à?¤?¨?à?¥???à?¤?¨?à?¥?‹?à?¤?¦?à?¥?‡?à?¤?µ?à?¥?€?à?¤?¤?à?¥???à?¤?¯?à?¤?ª?à?¥?‹?
?à?¤?¦?à?¤?¦?à?¥???à?¤?¯?à?¤?¾?à?¤?¦?à?¥???à?¤?¯?à?¤?µ?à?¥?‹?à?¤?½?à?¤?¸?à?¥?€?à?¤?¤?à?¤?¿?
?à?¤?¯?à?¤?µ?à?¤?¾?à?¤?¨?à?¤?ª?à?¤?¿? ?à?¥?¥? ?à?¤?‡?à?¤?¤?à?¤?¿?
?à?¥?¤?

?à?¤?¤?à?¤?¥?à?¤?¾? ?à?¤?š? ?â?€?œ? ?à?¤?¦?à?¥???à?¤?µ?à?¥?‡?
?à?¤?¦?à?¥???à?¤?µ?à?¥?‡? ?à?¤?¶?à?¤?²?à?¤?¾?à?¤?•?à?¥?‡?
?à?¤?¦?à?¥?‡?à?¤?µ?à?¤?¾?à?¤?¨?à?¤?¾?à?¤?‚?
?à?¤?ª?à?¤?¾?à?¤?¤?à?¥???à?¤?°?à?¥?‡?
?à?¤?•?à?¥?ƒ?à?¤?¤?à?¥???à?¤?µ?à?¤?¾? ?à?¤?ª?à?¤?¯?à?¤?ƒ?
?à?¤?•?à?¥???à?¤?·?à?¤?¿?à?¤?ª?à?¥?‡?à?¤?¤?à?¥???â?€??? ?
?à?¤?‡?à?¤?¤?à?¤?¿? ?à?¤?¦?à?¥???à?¤?µ?à?¥?‡?

?à?¤?¦?à?¥???à?¤?µ?à?¥?‡? ?à?¤?‡?à?¤?¤?à?¤?¿?
?à?¤?µ?à?¥?€?à?¤?ª?à?¥???à?¤?¸?à?¤?¾?,?
?à?¤?\ldots{}?à?¤?°?à?¥???à?¤?§?à?¤?ª?à?¤?¾?à?¤?¤?à?¥???à?¤?°?à?¤?¦?à?¥???à?¤?µ?à?¤?¿?à?¤?¤?à?¥???à?¤?µ?à?¥?‡?
?à?¤???à?¤?µ? ?à?¤?¸?à?¤?¾?à?¤?°?à?¥???à?¤?¥?à?¤?¿?à?¤?•?à?¤?¾?
?à?¤?­?à?¤?µ?à?¤?¤?à?¤?¿? ?à?¥?¤? ?à?¤?¤?à?¥???à?¤?°?à?¥?€?à?¤?£?à?¤?¿?
?à?¤?ª?à?¤?¾?à?¤?¤?à?¥???à?¤?°?à?¤?¾?à?¤?ª?à?¥???à?¤?¯?à?¥???à?¤?ª?-?

?à?¤?•?à?¤?²?à?¥???à?¤?ª?à?¤?¯?à?¥?‡?à?¤?¤?,?
?à?¤?µ?à?¥?ˆ?à?¤?¶?à?¥???à?¤?µ?à?¤?¦?à?¥?‡?à?¤?µ?à?¥?‡?
?à?¤???à?¤?•?à?¤?®?à?¥???,?
?à?¤???à?¤?•?à?¥?ˆ?à?¤?•?à?¤?®?à?¥???à?¤?­?à?¤?¯?à?¤?¤?à?¥???à?¤?°?
?à?¤?µ?à?¥?‡?à?¤?¤?à?¤?¿?
?à?¤?®?à?¤?¾?à?¤?¨?à?¤?µ?à?¤?®?à?¥?ˆ?à?¤?¤?à?¥???à?¤?°?à?¤?¾?à?¤?¯?à?¤?£?à?¥?€?à?¤?µ?à?¤?µ?à?¤?š?à?¤?¨?à?¥?‡?

?à?¤???à?¤?•?à?¤?®?à?¤?ª?à?¥???à?¤?¯?à?¥???à?¤?•?à?¥???à?¤?¤?à?¤?®?à?¥???,?
?à?¤?\ldots{}?à?¤?¤?à?¥?‹? ?à?¤?µ?à?¤?¿?à?¤?•?à?¤?²?à?¥???à?¤?ª?à?¤?ƒ?
?à?¥?¤?
?à?¤???à?¤?•?à?¥?ˆ?à?¤?•?à?¤?®?à?¥???à?¤?­?à?¤?¯?à?¤?¤?à?¥???à?¤?°?
?à?¤?µ?à?¥?‡?à?¤?¤?à?¤?¿?
?à?¤?ª?à?¤?¾?à?¤?¤?à?¥???à?¤?°?à?¤?¾?à?¤?²?à?¤?¾?à?¤?­?à?¤?µ?à?¤?¿?à?¤?·?-?

?à?¤?¯?à?¤?®?à?¥??? ?à?¥?¤?
?à?¤???à?¤?•?à?¤?¬?à?¥???à?¤?°?à?¤?¾?à?¤?¹?à?¥???à?¤?®?à?¤?£?à?¤?ª?à?¤?•?à?¥???à?¤?·?à?¥?‡?
?à?¤?‡?à?¤?¦?à?¤?®?à?¤?¿?à?¤?¤?à?¤?¿? ?à?¤?¤?à?¥???
?à?¤?¹?à?¥?‡?à?¤?®?à?¤?¾?à?¤?¦?à?¥???à?¤?°?à?¤?¿?à?¤?ƒ? ?à?¥?¤?
?à?¤?\ldots{}?à?¤?°?à?¥???à?¤?§?à?¤?ª?à?¤?¾?à?¤?¤?à?¥???à?¤?°?à?¤?¸?à?¥???à?¤?¥?à?¤?¾?à?¤?ª?à?¤?¨?à?¤?‚?
?à?¤?š? ?à?¤?¬?à?¥???à?¤?°?à?¤?¾?à?¤?¹?à?¥???à?¤?®?-?

?à?¤?£?à?¤?¾?à?¤?ª?à?¥???à?¤?°?à?¥?‡? ?à?¤?¨? ?à?¤?‡?à?¤?¤?à?¤?¿?
?à?¤?®?à?¤?¦?à?¤?¨?à?¤?°?à?¤?¤?à?¥???à?¤?¨?à?¤?ƒ? ?à?¥?¤?
?à?¤???à?¤?µ?à?¤?‚? ?à?¤?ª?à?¤?¾?à?¤?¤?à?¥???à?¤?°?à?¤?¾?à?¤?£?à?¤?¿?
?à?¤?¨?à?¤?¿?à?¤?§?à?¤?¾?à?¤?¯? ?à?¤?¤?à?¥?‡?à?¤?·?à?¥???
?à?¤?ª?à?¤?µ?à?¤?¿?à?¤?¤?à?¥???à?¤?°?à?¤?¨?à?¤?¿?à?¤?§?à?¤?¾?à?¤?¨?à?¤?‚?

?à?¤?š? ?à?¤?•?à?¥?ƒ?à?¤?¤?à?¥???à?¤?µ?à?¤?¾?
?à?¤???à?¤?•?à?¥?ˆ?à?¤?•?à?¤?¸?à?¥???à?¤?®?à?¤?¿?à?¤?¨?à?¥???à?¤?ª?à?¤?¾?à?¤?¤?à?¥???à?¤?°?à?¥?‡?
?â?€?œ? ?à?¤?¶?à?¤?¨?à?¥???à?¤?¨?à?¥?‹?à?¤?¦?à?¥?‡?à?¤?µ?à?¥?€? ?â?€???
?à?¤?°?à?¤?¿?à?¤?¤?à?¥???à?¤?¯?à?¤?¨?à?¤?¯?à?¤?°?à?¥???à?¤?š?à?¤?¾?à?¤?½?à?¤?ª?
?à?¤?†?à?¤?¸?à?¤?¿?à?¤?ž?à?¥???à?¤?š?à?¤?¤?à?¤?¿? ?à?¥?¤?

?à?¤?¤?à?¤?¥?à?¤?¾? ?à?¤?š?
?à?¤?­?à?¤?µ?à?¤?¿?à?¤?·?à?¥???à?¤?¯?à?¤?ª?à?¥???à?¤?°?à?¤?¾?à?¤?£?à?¥?‡?
?à?¥?¤?

?à?¤?µ?à?¥?€?à?¥?¦? ?à?¤?®?à?¤?¿?
?à?¥?¨?à?¥?­?\textless{}?/?s?p?a?n?\textgreater{}?\textless{}?/?p?\textgreater{}?\textless{}?/?b?o?d?y?\textgreater{}?\textless{}?/?h?t?m?l?\textgreater{}?

{२१० वीरमित्रोदयस्य श्राद्धप्रकाशे-}{\\
शन्नोदेवीरभिष्टय इत्यृचा पूरयेज्जलैः ।\\
उदङ्मुखो यवोऽसीति देवपात्रे यवान् क्षिपेत् ॥\\
ॐतिलोसीति मन्त्रेण प्रत्येकं निर्वपेत्तिलान् । इति ।\\
अत्र प्रतिपात्रं मन्त्रावृत्तिर्जलासिञ्चने ज्ञेया करणत्वात् ।\\
आश्वलायनेन तु `` शन्नोदेवी '' रित्यनेनानुमन्त्रयेदित्युक्तं शन्नोदेवी-\\
रभिष्टय इत्यनुमन्त्रितासु तिलानावपतीत्यादिना । तदा तु सकृदेव-\\
मन्त्रः शक्यत्वात् ।\\
नागरखण्डे,\\
पितॄणामर्धपात्रेषु तथैव च जलं क्षिपेत् ।\\
तिलोसि सोमदेवत्यो गोसवो देवनिर्मितः ॥\\
}{प्रज्ञम}{द्भिः पृक्तः स्वधया पितॄनिमाँल्लोकान्प्रीणाहीति ।\\
पृथक् तिलांश्च तत्रैव पितृतीर्थेन यत्नतः ॥ इति ।\\
तथैव चेति = शन्नोदेवत्यनेन मन्त्रेणेत्यर्थः । मन्त्रे पाठविशेषश्च यथा\\
गृह्यं व्यवस्थितो ज्ञेयः ।\\
अत्र यवतिलावापानन्तरं गन्धपुष्पप्रक्षेपः कार्यं इत्युक्तं -\\
सौरपुराणे ।\\
शन्नो देव्या जलं क्षिप्त्वा सपवित्रे तु भाजने ।\\
यवान् यवोसीति तथा गन्धं पुष्पं च निक्षिपेत् ॥\\
गन्धपुष्पप्रक्षेपमन्त्र उक्तः-\\
चतुर्विंशतिमते ।\\
द्वे द्वे पवित्रे देवानां पात्रे कृत्वा पय: क्षिपेत् ।\\
शन्नोदेवीति वै तोयं यवोऽसीति च वै यवान् ॥

{तथा,\\
श्रीश्चतेति च वै पुष्पं गन्धद्वारेति चन्दनम् ।\\
पुष्पधूपादिभिः पूज्य देवपात्राणि मानवः ।\\
यत्रादिशब्देन दीपोऽपि संगृह्यते । गन्धाद्युपचारानन्तरं उत्प-\\
वनमुक्तं-\\
मानवे मैत्रायणीयसूत्रे ।\\
सुमनसश्चोत्पूय यवान् प्रक्षिप्येति ।\\
उदकप्रक्षेपानन्तरं `` यवोऽसी '' त्यादिना मन्त्रेण तत्र यवान् प्रक्षि\\
प्य `` श्रीश्चत'' इत्यनेन सुमनसः पुष्पाणि निधाय प्रोक्षणविदुत्पकां\\
कृत्वा वक्ष्यमाणप्रकारेणार्घ्यदानं कुर्यादित्यर्थ इति हेमाद्रिः ।

{ अर्घसंपादनप्रकारः । २११}{\\
अनन्तरं वैतान्यर्धपात्राणि सावित्र्याभिमन्त्रणीयानि । सावित्र्यो-\\
दकपात्रमभिमन्त्र्येति मैत्रसूत्रात् ।\\
पाद्यमात्स्ययोः,\\
शन्नो देवीत्यपो दद्याद्यवोऽसीति यवानपि ।\\
गन्धपुष्पैस्तु सम्पूज्य वैश्वदेव प्रतिन्यसेत् ॥\\
वैश्वदेवब्राह्मणं प्रति पुरतो न्यसेत्, स्थापयेदित्यर्थः ।\\
बौधायने विशेषः ।\\
दक्षिणेनाग्निं स्वधापात्रं स्थापयेत् । आमागन्तु पितरो देवयानाः ,\\
तिलोसि सोमदेवत्यो गोसवो देवनिर्मितः ।\\
प्रत्नमद्भिः पृक्तः स्वधया पितृल्ँलोकान्प्रीणाहि नः स्वधा नमः ॥\\
इति तिलानोप्य ``मधुष्वाता ऋतायते मधु क्षरन्ति सिन्धवः । मा-\\
ध्वीर्नः सम्त्वोषधीः'' मधुनक्तमुतोषसो मधुमत्पार्थिव }{४ }{रजः । म.\\
धुद्यौरस्तुनः पिता । `` मधुमान्नो वनस्पतिर्मधुमां अस्तु सूर्यः ।
माध्वी-\\
र्गावो भवन्तु नः'' । `` सोमस्य त्विषिरसि'' शन्नोदेवीरभिष्टय आपो\\
भवन्तु पीतये । शंयोरमिस्रवन्तु नः । एतैः स्वधां मिश्रीकृत्य
निरस्तं\\
नमुचेः शिर इति किञ्चिन्निरस्य गन्धपुष्पाक्षतैः पितृभ्यो नमः पिता -\\
महेस्यो नमः प्रपितामहेभ्यो नम इत्यर्चयित्वा दर्भैः प्रस्थापयेत् ।\\
अस्यावमर्थः । अग्निं दक्षिणेन अन्वाहार्यपचनस्यावसथ्यस्य वामेर्दक्षि\\
णतः । स्वधापात्रम् अर्घ्योदकधारणार्थं पात्रं स्थापयित्वा `` आमागन्तु\\
पितरो देवयाना'' इत्यादिकं मन्त्रमुच्चारयेत् । तदन्तरं तिलोसी\\
त्यादिना मन्त्रेण तिलान् प्रक्षिप्य
मधुव्वातेत्यादिभिर्मन्त्रैर्मन्त्रलिङ्गान्म-\\
धुघृतोदकरूपां स्वधां मिश्रीकुर्यात् । तत्र मधुप्रकाशकत्वात् मधु-\\
व्वाता'' इति तृचेन मधुप्रक्षेपः । त्विषिस्तेजः, तेजो वै घृतमिति घृ-\\
तप्रकाशकत्वात् `` सोमस्य त्विषि''रित्यनेन घृतप्रक्षेपः । अपःप्रका\\
शकत्वात् `` यशोदेवी''रित्यपाम् । ततो निरस्तं नमुचेः शिर'' इत्य\\
नेन तस्मादर्घौदकात्किञ्चिदेकदेश बहिर्निरस्य पितृभ्यो नम इत्या-\\
दिमन्त्रत्रयेण प्रत्येकं गन्धपुष्पाक्षतं निक्षिप्य तत्र कुशत्रयं
स्थापयेत् ।\\
निगमपरिशिष्टे कात्यायनेन तु आहिताग्निकर्तृके षट्देवत्ये श्राद्धे च\\
त्वार्यर्धपात्राणि तेषां दक्षिणाभिमुखोल्लिखितलेखायां स्थापनं स्र-\\
गादिषट्द्रव्योपेतानामद्भिः प्रपूरणं तत्र च मन्त्रान्तराण्युक्तानि ।\\
यथा ।\\
उल्लिल्य दक्षिणां लेखायां कृत्वा लौहांश्चमसांश्चतुरः स्रक्तिल

{२१२ चीरमित्रोदयस्य श्राद्धप्रकाशे-}{\\
पयोदधिमधुघृतमिश्रान् महाव्याहृत्यापोहिष्ठीयशन्नोदेवीरित्याद्भिः प्र.\\
पूर्वेति ।\\
स्रक् = पुष्पमाला ।लौहान्ताम्रमयान् । त्रपुसीसकृष्णायसव्यति.\\
रिकैस्तैजसधातुभिर्विरचितान् इति हेमाद्रिः । तांश्चतुःसंख्यकान् यथो.\\
कायां लेखायां समासाद्य अत्र चतुर्थस्य मातामहार्थता वक्ष्यते ` चतु-\\
र्थेन मातामहादीनवनेज्ये 'तिवचनात् । तांश्च स्रगादिष द्रव्योपेतान्\\
कृत्वा महाव्याहृत्यादिभिर्मन्त्रैरद्भिः प्रपूरयेत् ।
महाव्याहृतयश्चाध्ये-\\
त्प्रसिद्धाः । आपोहिष्ठीयं आपोहिष्ठेतितृचम् । हारीतेन तु अपां\\
निषेचने मन्त्रान्तरमपामुत्पवने चोक्तम् । उदपात्रेषु समन्यायन्तीत्यप\\
आसिञ्च्य सुमनश्चोत्पूयेति । समन्यायन्त्यन्याः समानपूर्वं नद्यः\\
पृणन्ति । तमुशुचिं शुचयोदीविचां समपान्नपानं परितस्थुराप इति ।\\
उत्पवनं च प्रोक्षणीवदित्युक्तं प्राक् । तच्च
हस्तद्वयाङ्गुष्ठोपकनिष्ठिका-\\
भ्यां कुशपवित्रं गृहीत्वा तूष्र्णामुत्पवनं कार्यम् ।\\
केचितु प्रोक्षण्युत्पवनमन्त्रेणोत्पवनं इच्छन्ति । स च सवितुर्व:\\
प्रसव उत्पुनाम्यच्छिद्रेण पवित्रेण सूर्यस्य रश्मिभिरित्यादिः प्रतिशा-\\
खं भिन्नः ।\\
ब्रह्मपुराणे तु अष्टावर्धद्रव्याण्युक्तानि ।\\
एष तेऽर्ध इति प्रोच्य तेभ्यो दद्यादथाष्टधा ।\\
जलं क्षीरं दधि घृतं तिलतण्डुलसर्षपान् }{॥}{\\
कुशाग्राणि च पुष्पाणि दत्वाचामेत्ततः स्वयम् ॥ इति ।\\
अष्टधेतिउपक्रमात् क्षीराद्यष्टद्रव्योपेतं जलं देयमिति प्रतीयते ।\\
इति अर्धसम्पादनविधि ।\\
अथ अर्धदानविधिः ।\\
तत्रार्धसम्पादनोत्तरमाह -\\
जातूकर्ण्यः ।\\
ततोऽर्धपात्रसम्पत्तिं वाचयित्वा द्विजोत्तमान् ।\\
तदग्रे चार्घ्यपात्रं तु स्वाहार्घ्या इति विन्यसेत् ॥\\
अयमर्थः दैवार्धपात्रसत्पत्तिरस्त्विति कृताञ्जलिर्देवब्राह्मणान् पृ-\\
च्छेत् । ॐ अस्तु दैवार्धपात्रसम्पत्तिरिति ब्राह्मणैः प्रतिवचने कृते स-\\
ति आस्तीर्णदर्भसहितमर्धपात्रमुद्धृत्यं `` स्वाहार्थ्यां '' इत्यनेन
वाक्येन\\
वैश्वदेविकब्राह्मणानां पुरतः स्थापयेत् इति । तदनन्तरं च वेश्वदै\\
विकब्राह्मणहस्ते तत्पात्रोदकप्रक्षेपः कर्त्तव्यः । यथाह-\/-}

?

? ?b?o?d?y?\{? ?w?i?d?t?h?:? ?2?1?c?m?;? ?h?e?i?g?h?t?:? ?2?9?.?7?c?m?;?
?m?a?r?g?i?n?:? ?3?0?m?m? ?4?5?m?m? ?3?0?m?m? ?4?5?m?m?;? ?\}?
?\textless{}?/?s?t?y?l?e?\textgreater{}?\textless{}?!?D?O?C?T?Y?P?E?
?H?T?M?L? ?P?U?B?L?I?C? ?"?-?/?/?W?3?C?/?/?D?T?D? ?H?T?M?L?
?4?.?0?/?/?E?N?"?
?"?h?t?t?p?:?/?/?w?w?w?.?w?3?.?o?r?g?/?T?R?/?R?E?C?-?h?t?m?l?4?0?/?s?t?r?i?c?t?.?d?t?d?"?\textgreater{}?
?

?

?

?

? ?p?,? ?l?i? ?\{? ?w?h?i?t?e?-?s?p?a?c?e?:? ?p?r?e?-?w?r?a?p?;? ?\}?
?\textless{}?/?s?t?y?l?e?\textgreater{}?\textless{}?/?h?e?a?d?\textgreater{}?

? ?

?

? ? ? ? ? ? ? ? ? ? ? ? ? ? ? ? ? ? ? ? ? ?
?\textless{}?/?s?p?a?n?\textgreater{}?

? ? ?
?à?¤?\ldots{}?à?¤?°?à?¥???à?¤?˜?à?¤?¦?à?¤?¾?à?¤?¨?à?¤?ª?à?¥???à?¤?°?à?¤?•?à?¤?¾?à?¤?°?à?¤?ƒ?
?à?¥?¤? ? ? ? ? ? ? ? ? ? ? ? ? ? ? ? ? ? ? ? ?
?à?¥?¨?à?¥?§?à?¥?©?\textless{}?/?s?p?a?n?\textgreater{}?

?

?à?¤?•?à?¤?¾?à?¤?¤?à?¥???à?¤?¯?à?¤?¾?à?¤?¯?à?¤?¨?à?¤?ƒ? ?à?¥?¤?

? ? ?
?à?¤???à?¤?•?à?¤?¸?à?¥???à?¤?¯?à?¥?ˆ?à?¤?•?à?¥?ˆ?à?¤?•?à?¥?‡?à?¤?¨?
?à?¤?¦?à?¤?¦?à?¤?¾?à?¤?¤?à?¤?¿?
?à?¤?¸?à?¤?ª?à?¤?µ?à?¤?¿?à?¤?¤?à?¥???à?¤?°?à?¥?‡?à?¤?·?à?¥???
?à?¤?¹?à?¤?¸?à?¥???à?¤?¤?à?¥?‡?à?¤?·?à?¥??? ?à?¤?¯?à?¤?¾?
?à?¤?¦?à?¤?¿?à?¤?µ?à?¥???à?¤?¯?à?¤?¾? ?à?¤?†?à?¤?ª?à?¤?ƒ?

?à?¤?ª?à?¤?¯?à?¤?¸?à?¤?¾?
?à?¤?¸?à?¤?®?à?¥???à?¤?¬?à?¤?­?à?¥?‚?à?¤?µ?à?¥???à?¤?°?à?¥???à?¤?¯?à?¤?¾?
?à?¤?\ldots{}?à?¤?¨?à?¥???à?¤?¤?à?¤?°?à?¤?¿?à?¤?•?à?¥???à?¤?·?à?¤?¾?
?à?¤?‰?à?¤?¤?
?à?¤?ª?à?¤?¾?à?¤?°?à?¥???à?¤?¥?à?¤?¿?à?¤?µ?à?¥?€?à?¤?°?à?¥???à?¤?¯?à?¤?¾?à?¤?ƒ?
?à?¥?¤?
?à?¤?¹?à?¤?¿?à?¤?°?à?¤?£?à?¥???à?¤?¯?à?¤?µ?à?¤?°?à?¥???à?¤?£?à?¤?¾?

?à?¤?¯?à?¤?œ?à?¥???à?¤?ž?à?¤?¿?à?¤?¯?à?¤?¾?à?¤?¸?à?¥???à?¤?¤?à?¤?¾?à?¤?¨?
?à?¤?†?à?¤?ª?à?¤?ƒ? ?à?¤?¶?à?¤?¿?à?¤?µ?à?¤?¾?à?¤?ƒ?
?\textless{}?/?s?p?a?n?\textgreater{}?

?à?¤?¸?à?¤?¸?à?¥???à?¤?¯?à?¥?‹?à?¤?¨?à?¤?¾?à?¤?ƒ?\textless{}?/?s?p?a?n?\textgreater{}?

? ?à?¤?¸?à?¥???à?¤?¹?à?¤?µ?à?¤?¾?
?à?¤?­?à?¤?µ?à?¤?¨?à?¥???à?¤?¤?à?¥???à?¤?µ?à?¤?¿?à?¤?¤?à?¥???à?¤?¯?à?¤?¸?à?¤?¾?à?¤?µ?à?¥?‡?à?¤?·?à?¤?¤?à?¥?‡?-?

?à?¤?½?à?¤?°?à?¥???à?¤?˜? ?à?¤?‡?à?¤?¤?à?¤?¿? ?à?¥?¤?
?à?¤?\ldots{}?à?¤?¸?à?¥???à?¤?¯?à?¤?¾?à?¤?¯?à?¤?®?à?¤?°?à?¥???à?¤?¥?à?¤?ƒ?
?à?¥?¤?
?à?¤?¬?à?¥???à?¤?°?à?¤?¾?à?¤?¹?à?¥???à?¤?®?à?¤?£?à?¤?¹?à?¤?¸?à?¥???à?¤?¤?à?¥?‡?à?¤?·?à?¥???
?à?¤?¸?à?¤?ª?à?¤?µ?à?¤?¿?à?¤?¤?à?¥???à?¤?°?à?¥?‡?à?¤?·?à?¥???
?à?¤?\ldots{}?à?¤?°?à?¥???à?¤?§?à?¤?‚?

?à?¤?¦?à?¤?¦?à?¤?¾?à?¤?¤?à?¤?¿? ?à?¤?¯?à?¤?¾?
?à?¤?¦?à?¤?¿?à?¤?µ?à?¥???à?¤?¯?à?¤?¾? ?à?¤?‡?à?¤?¤?à?¤?¿?
?à?¤?®?à?¤?¨?à?¥???à?¤?¤?à?¥???à?¤?°?à?¥?‡?à?¤?£?,?
?à?¤?ª?à?¤?µ?à?¤?¿?à?¤?¤?à?¥???à?¤?°?à?¤?¾?à?¤?£?à?¤?¿? ?à?¤?š?
?à?¤?ª?à?¥???à?¤?°?à?¤?•?à?¥?ƒ?à?¤?¤?à?¤?¤?à?¥???à?¤?µ?à?¤?¾?à?¤?¦?à?¥???
?à?¤?¯?à?¤?¾?à?¤?¨?à?¤?¿? ?à?¤?\ldots{}?à?¤?¨?à?¥???à?¤?¤?

?à?¤?°?à?¥???à?¤?˜?à?¤?¾?à?¤?¨?à?¤?¾?à?¤?°?à?¥???à?¤?¥?à?¤?¾?à?¤?¨?à?¤?¿?
?à?¤?¤?à?¤?¾?à?¤?¨?à?¥???à?¤?¯?à?¥?‡?à?¤?µ?
?à?¤?¬?à?¥???à?¤?°?à?¤?¾?à?¤?¹?à?¥???à?¤?®?à?¤?£?à?¤?¹?à?¤?¸?à?¥???à?¤?¤?à?¥?‡?
?à?¤?¸?à?¥???à?¤?¥?à?¤?¾?à?¤?ª?à?¥???à?¤?¯?à?¤?¾?à?¤?¨?à?¤?¿?
?à?¤?†?à?¤?---?à?¥???à?¤?¨?à?¥?‡?à?¤?¯?à?¥???à?¤?¯?à?¤?¾?à?¤?---?à?¥???à?¤?¨?à?¥?€?à?¤?§?à?¥???à?¤?°?à?¤?®?à?¥???à?¤?ª?à?¤?¤?à?¤?¿?à?¤?·?à?¥???à?¤?~?à?¤?¤?à?¥?‡?

?à?¤?‡?à?¤?¤?à?¤?¿?à?¤?µ?à?¤?¤?à?¥??? ?à?¥?¤? ?à?¤?¨? ?à?¤?š?
?à?¤?µ?à?¤?¿?à?¤?¨?à?¤?¿?à?¤?¯?à?¥???à?¤?•?à?¥???à?¤?¤?à?¤?µ?à?¤?¿?à?¤?¨?à?¤?¿?à?¤?¯?à?¥?‹?à?¤?---?à?¤?µ?à?¤?¿?à?¤?°?à?¥?‹?à?¤?§?à?¤?ƒ?
?à?¥?¤?

? ? ? ? ?à?¤?•?à?¥???à?¤?¶?à?¤?¾?
?à?¤?¦?à?¤?°?à?¥???à?¤?µ?à?¥???à?¤?¯?à?¤?¾?à?¤?¦?à?¤?¯?à?¥?‹?
?à?¤?®?à?¤?¨?à?¥???à?¤?¤?à?¥???à?¤?°?à?¤?¾?
?à?¤?¬?à?¥???à?¤?°?à?¤?¾?à?¤?¹?à?¥???à?¤?®?à?¤?£?à?¤?¾?à?¤?¶?à?¥???à?¤?š?
?à?¤?¬?à?¤?¹?à?¥???à?¤?¶?à?¥???à?¤?°?à?¥???à?¤?¤?à?¤?¾?à?¤?ƒ? ?à?¥?¤?

? ? ? ? ?à?¤?¨? ?à?¤?¤?à?¥?‡?
?à?¤?¨?à?¤?¿?à?¤?°?à?¥???à?¤?®?à?¤?¾?à?¤?²?à?¥???à?¤?¯?à?¤?¤?à?¤?¾?à?¤?‚?
?à?¤?¯?à?¤?¾?à?¤?¨?à?¥???à?¤?¤?à?¤?¿?
?à?¤?µ?à?¤?¿?à?¤?¨?à?¤?¿?à?¤?¯?à?¥?‹?à?¤?œ?à?¥???à?¤?¯?à?¤?¾?
?à?¤?ª?à?¥???à?¤?¨?à?¤?ƒ? ?à?¤?ª?à?¥???à?¤?¨?à?¤?ƒ? ?à?¥?¥?

?à?¤?‡?à?¤?¤?à?¤?¿?
?à?¤?µ?à?¤?š?à?¤?¨?à?¥?‡?à?¤?¨?à?¤?¾?à?¤?---?à?¥???à?¤?¨?à?¥?‡?à?¤?¯?à?¥?€?à?¤?®?à?¤?¨?à?¥???à?¤?¤?à?¥???à?¤?°?à?¤?µ?à?¤?¦?à?¥?‡?à?¤?µ?
?à?¤?µ?à?¤?¿?à?¤?¨?à?¤?¿?à?¤?¯?à?¥???à?¤?•?à?¥???à?¤?¤?à?¤?µ?à?¤?¿?à?¤?¨?à?¤?¿?à?¤?¯?à?¥?‹?à?¤?---?à?¤?¸?à?¥???à?¤?¯?à?¤?¾?à?¤?¦?à?¥?‹?à?¤?·?à?¤?¤?à?¥???à?¤?µ?à?¤?¾?à?¤?¤?à?¥???
?à?¥?¤? ?à?¤?\ldots{}?.?

?à?¤?¸?à?¤?¾?à?¤?µ?à?¤?¿?à?¤?¤?à?¥???à?¤?¯?à?¤?¤?à?¥???à?¤?°?
?à?¤?¯?à?¤?¥?à?¤?¾?à?¤?¦?à?¥?‡?à?¤?µ?à?¤?¤?à?¤?‚?
?à?¤?¨?à?¤?¾?à?¤?®?à?¤?¾?à?¤?¦?à?¥?‡?à?¤?¶?à?¤?ƒ?
?\textless{}?/?s?p?a?n?\textgreater{}?

?à?¥?¤?\textless{}?/?s?p?a?n?\textgreater{}?

?
?à?¤?\ldots{}?à?¤?¸?à?¤?¾?à?¤?µ?à?¤?¿?à?¤?¤?à?¥???à?¤?¯?à?¤?ª?à?¤?¨?à?¥?‹?à?¤?¦?
?à?¤?‡?à?¤?¤?à?¤?¿?
?à?¤?•?à?¤?¾?à?¤?¤?à?¥???à?¤?¯?à?¤?¾?à?¤?¯?à?¤?¨?à?¥?‡?à?¤?¨?

?à?¤?ª?à?¤?°?à?¤?¿?à?¤?­?à?¤?¾?à?¤?·?à?¤?¿?à?¤?¤?à?¤?¤?à?¥???à?¤?µ?à?¤?¾?à?¤?¤?à?¥???
?à?¥?¤? ?à?¤?\ldots{}?à?¤?¤?à?¥???à?¤?°?
?à?¤???à?¤?•?à?¤?¦?à?¥?‡?à?¤?µ?à?¤?¤?à?¤?¾?à?¤?¸?à?¤?®?à?¥???à?¤?¬?à?¤?¨?à?¥???à?¤?§?à?¥?‡?à?¤?¨?
?à?¤?¯?à?¤?¾?à?¤?µ?à?¤?¨?à?¥???à?¤?¤?à?¥?‹?
?à?¤?¬?à?¥???à?¤?°?à?¤?¾?à?¤?¹?à?¥???à?¤?®?à?¤?£?à?¤?¾?à?¤?ƒ?
?à?¤?¤?à?¤?¾?à?¤?µ?à?¤?¤?à?¤?¾?à?¤?‚?

?à?¤?¹?à?¤?¸?à?¥???à?¤?¤?à?¥?‡?à?¤?·?à?¥???
?à?¤?¤?à?¤?¤?à?¥???à?¤?ª?à?¤?¾?à?¤?¤?à?¥???à?¤?°?à?¤?---?à?¤?¤?à?¤?¾?à?¤?°?à?¥???à?¤?˜?à?¥?‹?à?¤?ª?à?¤?µ?à?¥?‡?à?¤?¶?à?¤?ƒ?
?à?¤?ª?à?¥???à?¤?°?à?¤?¤?à?¤?¿?à?¤?ª?à?¤?¾?à?¤?¦?à?¥???à?¤?¯?à?¤?ƒ?
?à?¤?¤?à?¤?¤?à?¥???à?¤?°? ?à?¤?š? ?â?€?œ? ?à?¤?¯?à?¤?¾?
?à?¤?¦?à?¤?¿?à?¤?µ?à?¥???à?¤?¯?à?¤?¾? ?â?€??? ?à?¤?‡?à?¤?¤?à?¤?¿? ?

?à?¤?®?à?¤?¨?à?¥???à?¤?¤?à?¥???à?¤?°?à?¤?¸?à?¥???à?¤?¯?à?¤?¾?à?¤?µ?à?¥?ƒ?à?¤?¤?à?¥???à?¤?¤?à?¤?¿?à?¤?ƒ?
?à?¥?¤?
?à?¤?®?à?¤?¨?à?¥???à?¤?¤?à?¥???à?¤?°?à?¤?ª?à?¥???à?¤?°?à?¤?•?à?¤?¾?à?¤?¶?à?¥???à?¤?¯?à?¤?œ?à?¤?²?à?¤?¾?à?¤?µ?à?¤?¯?à?¤?µ?à?¤?­?à?¥?‡?à?¤?¦?à?¤?¾?à?¤?¤?à?¥???
?à?¥?¤? ?à?¤?¨? ?à?¤?š?
?à?¤?µ?à?¥?‡?à?¤?¦?à?¤?¿?à?¤?ª?à?¥???à?¤?°?à?¥?‹?à?¤?•?à?¥???à?¤?·?à?¤?£?à?¤?®?à?¤?¨?à?¥???à?¤?¤?à?¥???à?¤?°?-?

?à?¤?µ?à?¤?¤?à?¥???
?à?¤?•?à?¥???à?¤?°?à?¤?¿?à?¤?¯?à?¤?¾?à?¤?­?à?¥???à?¤?¯?à?¤?¾?à?¤?µ?à?¥?ƒ?à?¤?¤?à?¥???à?¤?¤?à?¥?Œ?
?à?¤?µ?à?¤?¿?à?¤?¹?à?¤?¿?à?¤?¤?à?¤?¤?à?¥???à?¤?µ?à?¤?¾?à?¤?¤?à?¥???
?à?¤?¸?à?¤?•?à?¥?ƒ?à?¤?¦?à?¥?‡?à?¤?µ?à?¤?®?à?¤?¨?à?¥???à?¤?¤?à?¥???à?¤?°?
?à?¤?‡?à?¤?¤?à?¤?¿? ?à?¤?µ?à?¤?¾?à?¤?š?à?¥???à?¤?¯?à?¤?®?à?¥??? ?à?¥?¤?
?à?¤?¤?à?¤?¤?à?¥???à?¤?°? ?à?¤?ª?à?¥???à?¤?°?à?¤?•?à?¤?¾?-?

?à?¤?¶?à?¥???à?¤?¯?à?¤?¾?à?¤?¯?à?¤?¾?
?à?¤?µ?à?¥?‡?à?¤?¦?à?¥?‡?à?¤?°?à?¤?­?à?¤?¿?à?¤?¨?à?¥???à?¤?¨?à?¤?¤?à?¥???à?¤?µ?à?¥?‡?à?¤?¨?
?à?¤?¤?à?¤?¥?à?¤?¾?
?à?¤?¨?à?¤?¿?à?¤?°?à?¥???à?¤?£?à?¤?¯?à?¤?¾?à?¤?¤?à?¥??? ?à?¥?¤?
?à?¤?‡?à?¤?¹? ?à?¤?¤?à?¥???
?à?¤?ª?à?¥???à?¤?°?à?¤?•?à?¤?¾?à?¤?¶?à?¥???à?¤?¯?à?¤?ª?à?¥???à?¤?°?à?¤?•?à?¥?ƒ?à?¤?¤?à?¤?œ?à?¤?²?à?¤?¾?à?¤?µ?-?

?à?¤?¯?à?¤?µ?à?¤?­?à?¥?‡?à?¤?¦?à?¤?¾?à?¤?¨?à?¥???à?¤?®?à?¤?¨?à?¥???à?¤?¤?à?¥???à?¤?°?à?¤?¾?à?¤?µ?à?¥?ƒ?à?¤?¤?à?¥???à?¤?¤?à?¤?¿?à?¤?°?à?¥?‡?à?¤?µ?
?à?¤?¯?à?¥???à?¤?•?à?¥???à?¤?¤?à?¤?¾? ?à?¥?¤?

? ? ? ?à?¤?•?à?¥?‡?à?¤?š?à?¤?¿?à?¤?¤?à?¥???à?¤?¤?à?¥??? ?à?¤?¯?à?¤?¾?
?à?¤?¦?à?¤?¿?à?¤?µ?à?¥???à?¤?¯?à?¤?¾?
?à?¤?‡?à?¤?¤?à?¥???à?¤?¯?à?¤?¾?à?¤?°?à?¤?­?à?¥???à?¤?¯?
?à?¤?­?à?¤?µ?à?¤?¨?à?¥???à?¤?µ?à?¤?¿?à?¤?¤?à?¥?€?à?¤?•?à?¤?¾?à?¤?°?à?¤?¾?à?¤?¨?à?¥???à?¤?¤?à?¥?‹?
?à?¤?œ?à?¤?ª?à?¤?®?à?¤?¨?à?¥???à?¤?¤?à?¥???à?¤?°?à?¥?‹?à?¤?½?à?¤?¯?à?¤?®?à?¥???
?à?¥?¤? ?à?¤?\ldots{}?-?

?à?¤?¸?à?¤?¾?à?¤?µ?à?¥?‡?à?¤?·? ?à?¤?¤?à?¥?‡?à?¤?½?à?¤?°?à?¥???à?¤?˜?
?à?¤?‡?à?¤?¤?à?¤?¿?
?à?¤?\ldots{}?à?¤?°?à?¥???à?¤?˜?à?¤?¦?à?¤?¾?à?¤?¨?à?¥?‡?
?à?¤?•?à?¤?°?à?¤?£?à?¤?®?à?¤?¨?à?¥???à?¤?¤?à?¥???à?¤?°?à?¤?ƒ? ?,?
?à?¤?‡?à?¤?¤?à?¤?¿?à?¤?•?à?¤?¾?à?¤?°?à?¤?¦?à?¥???à?¤?µ?à?¤?¯?à?¤?•?à?¤?°?à?¤?£?à?¤?¾?à?¤?¤?à?¥???
?à?¥?¤? ?à?¤?¤?à?¤?¥?à?¤?¾?

?à?¤?š?
?à?¤?®?à?¤?¾?à?¤?¨?à?¤?µ?à?¤?®?à?¥?ˆ?à?¤?¤?à?¥???à?¤?°?à?¤?¾?à?¤?¯?à?¤?£?à?¥?€?à?¤?¯?à?¤?¸?à?¥?‚?à?¤?¤?à?¥???à?¤?°?à?¥?‡?
?à?¤?ª?à?¤?µ?à?¤?¿?à?¤?¤?à?¥???à?¤?°?à?¥?‡? ?à?¤?ª?à?¤?¾?à?¤?£?à?¥?Œ?
?à?¤?ª?à?¥???à?¤?°?à?¤?¦?à?¤?¾?à?¤?¯?
?à?¤?¹?à?¤?¿?à?¤?°?à?¤?£?à?¥???à?¤?¯?à?¤?µ?à?¤?°?à?¥???à?¤?£?à?¤?¾?
?à?¤?‡?à?¤?¤?à?¥???à?¤?¯?à?¥???à?¤?•?à?¥???à?¤?¤?à?¥???à?¤?µ?à?¤?¾?

?à?¤?µ?à?¤?¿?à?¤?¶?à?¥???à?¤?µ?à?¥?‡?à?¤?¦?à?¥?‡?à?¤?µ?à?¤?¾?
?à?¤???à?¤?·? ?à?¤?µ?à?¥?‹? ?à?¤?\ldots{}?à?¤?°?à?¥???à?¤?˜?
?à?¤?‡?à?¤?¤?à?¤?¿? ?à?¥?¤? ?à?¤?¤?à?¥?‡?à?¤?¨?
?à?¤?•?à?¤?°?à?¤?£?à?¤?®?à?¤?¨?à?¥???à?¤?¤?à?¥???à?¤?°?à?¤?¸?à?¥???à?¤?¯?à?¥?ˆ?à?¤?µ?à?¤?š?
?à?¤?\ldots{}?à?¤?°?à?¥???à?¤?˜?à?¤?¦?à?¤?¾?à?¤?¨?à?¥?‡?
?à?¤?•?à?¤?°?à?¤?£?à?¤?¤?à?¥???à?¤?µ?à?¤?‚?

?à?¤?¤?à?¤?¤?à?¥???à?¤?°?
?à?¤?ª?à?¥???à?¤?°?à?¤?¤?à?¤?¿?à?¤?¬?à?¥???à?¤?°?à?¤?¾?à?¤?¹?à?¥???à?¤?®?à?¤?£?à?¤?‚?
?à?¤?¯?à?¤?¾? ?à?¤?¦?à?¤?¿?à?¤?µ?à?¥???à?¤?¯?à?¤?¾? ?à?¤?‡?à?¤?¤?à?¤?¿?
?à?¤?\ldots{}?à?¤?°?à?¥???à?¤?˜?à?¥???à?¤?¯?à?¤?®?à?¤?¨?à?¥???à?¤?¤?à?¥???à?¤?°?à?¥?‹?à?¤?½?à?¤?ª?à?¥???à?¤?¯?à?¤?¾?à?¤?µ?à?¤?°?à?¥???à?¤?¤?à?¥???à?¤?¤?à?¤?¤?
?à?¤?‡?à?¤?¤?à?¥???à?¤?¯?à?¤?¾?à?¤?¹?à?¥???à?¤?ƒ? ?à?¥?¤?

? ? ? ?à?¤?\ldots{}?à?¤?ª?à?¤?°?à?¥?‡? ?à?¤?¤?à?¥??? ?à?¤?¯?à?¤?¾?
?à?¤?¦?à?¤?¿?à?¤?µ?à?¥???à?¤?¯?à?¤?¾? ?à?¤?‡?à?¤?¤?à?¤?¿?
?à?¤?®?à?¤?¨?à?¥???à?¤?¤?à?¥???à?¤?°?à?¤?¸?à?¥???à?¤?¯?
?à?¤?œ?à?¤?ª?à?¤?¾?à?¤?°?à?¥???à?¤?¥?à?¤?¤?à?¥???à?¤?µ?à?¥?‡?à?¤?½?à?¤?ª?à?¥?‡?à?¤?¿?
?à?¤?ª?à?¥???à?¤?°?à?¤?¤?à?¥???à?¤?¯?à?¤?°?à?¥???à?¤?˜?à?¤?ª?à?¥???à?¤?°?à?¤?¦?à?¤?¾?à?¤?¨?à?¤?®?à?¤?¾?à?¤?µ?à?¥?ƒ?-?

?à?¤?¤?à?¥???à?¤?¤?à?¤?¿?à?¤?®?à?¤?¾?à?¤?¹?à?¥???:? ?à?¥?¤?
?à?¤?µ?à?¤?•?à?¥???à?¤?·?à?¥???à?¤?¯?à?¤?®?à?¤?¾?à?¤?£?à?¤?¬?à?¥???à?¤?°?à?¤?¹?à?¥???à?¤?®?à?¤?ª?à?¥???à?¤?°?à?¤?¾?à?¤?£?à?¤?µ?à?¤?š?à?¤?¨?à?¤?¾?à?¤?¤?à?¥???
?à?¥?¤?

? ? ? ?à?¤?\ldots{}?à?¤?¤?à?¥???à?¤?°? ?â?€?œ?
?à?¤?---?à?¤?¨?à?¥???à?¤?§?à?¤?ª?à?¥???à?¤?·?à?¥???à?¤?ª?à?¥?ˆ?à?¤?¶?à?¥???à?¤?š?
?à?¤?¸?à?¤?®?à?¥???à?¤?ª?à?¥?‚?à?¤?œ?à?¥???à?¤?¯?à?¥?‡?â?€???
?à?¤?¤?à?¥???à?¤?¯?à?¤?°?à?¥???à?¤?§?à?¤?ª?à?¤?¾?à?¤?¤?à?¥???à?¤?°?
?à?¤?ª?à?¥?‚?à?¤?œ?à?¤?¨?à?¤?®?à?¤?­?à?¤?¿?à?¤?§?à?¤?¾?à?¤?¯?
?à?¤?ª?à?¥???à?¤?¨?à?¤?°?à?¥???à?¤?•?à?¥???à?¤?¤?à?¤?‚?

?à?¤?ª?à?¤?¾?à?¤?¦?à?¥???à?¤?®?à?¤?®?à?¤?¤?à?¥???à?¤?¸?à?¥???à?¤?¯?à?¤?¯?à?¥?‹?à?¤?ƒ?
?à?¥?¤?

? ?
?à?¤?---?à?¤?¨?à?¥???à?¤?§?à?¤?ª?à?¥???à?¤?·?à?¥???à?¤?ª?à?¥?ˆ?à?¤?°?à?¤?²?à?¤?™?à?¥???à?¤?•?à?¥?ƒ?à?¤?¤?à?¥???à?¤?¯?
?à?¤?¯?à?¤?¾?
?à?¤?¦?à?¤?¿?à?¤?µ?à?¥???à?¤?¯?à?¥?‡?à?¤?¤?à?¥???à?¤?¯?à?¤?°?à?¥???à?¤?˜?à?¤?®?à?¥???à?¤?¤?à?¥???à?¤?¸?à?¥?ƒ?à?¤?œ?à?¥?‡?à?¤?¤?à?¥???
?à?¥?¤? ?à?¤?‡?à?¤?¤?à?¤?¿? ?à?¥?¤? ?à?¤?‡?à?¤?¦?à?¤?‚? ?à?¤?š?
?à?¤?¬?à?¥???à?¤?°?à?¤?¾?à?¤?¹?à?¥???à?¤?®?-?

?à?¤?£?à?¤?ª?à?¥?‚?à?¤?œ?à?¤?¨?à?¤?®?à?¥??? ?à?¥?¤?
?à?¤?¤?à?¤?š?à?¥???à?¤?š?
?à?¤?\ldots{}?à?¤?°?à?¥???à?¤?˜?à?¤?ª?à?¤?¾?à?¤?¤?à?¥???à?¤?°?à?¤?¸?à?¥???à?¤?¥?à?¤?¿?à?¤?¤?à?¤?ª?à?¤?µ?à?¤?¿?à?¤?¤?à?¥???à?¤?°?à?¤?ª?à?¥???à?¤?°?à?¤?¦?à?¤?¾?à?¤?¨?à?¤?¾?à?¤?¨?à?¤?¨?à?¥???à?¤?¤?à?¤?°?à?¤?‚?
?à?¤?¸?à?¤?µ?à?¤?¿?à?¤?¶?à?¥?‡?à?¤?·?à?¤?§?à?¤?°?à?¥???à?¤?®?à?¤?•?à?¤?‚?

?à?¤?•?à?¤?°?à?¥???à?¤?¤?à?¥???à?¤?¤?à?¤?µ?à?¥???à?¤?¯?à?¤?®?à?¤?¿?à?¤?¤?à?¥???à?¤?¯?à?¤?¾?à?¤?¹?-?

?à?¤?---?à?¤?¾?à?¤?°?à?¥???à?¤?---?à?¥???à?¤?¯?à?¤?ƒ? ?à?¥?¤?

? ? ? ?à?¤?¦?à?¤?¤?à?¥???à?¤?µ?à?¤?¾? ?à?¤?¹?à?¤?¸?à?¥???à?¤?¤?à?¥?‡?
?à?¤?ª?à?¤?µ?à?¤?¿?à?¤?¤?à?¥???à?¤?°? ?à?¤?š?
?à?¤?ª?à?¥?‚?à?¤?œ?à?¤?¾?à?¤?™?à?¥???à?¤?•?à?¥?ƒ?à?¤?¤?à?¥???à?¤?µ?à?¤?¾?
?à?¤?š? ?à?¤?ª?à?¤?¾?à?¤?¦?à?¤?¤?à?¤?ƒ?
?à?¥?¤?\textless{}?/?s?p?a?n?\textgreater{}?\textless{}?/?p?\textgreater{}?\textless{}?/?b?o?d?y?\textgreater{}?\textless{}?/?h?t?m?l?\textgreater{}?

{२१४ वीरमित्रोदयस्य आकाशे-}{\\
या दिव्या इति मन्त्रेण हस्तेष्वर्ये विनिक्षिपेत् ॥ इति ।\\
पादतः = पादप्रभृति मूर्द्धान्तम् । तथा च-\\
प्रचेताः ।\\
पादप्रभृतिमूर्द्धान्तं देवानां पुष्पपूजनम् ।\\
तच्च पूर्वे दक्षिणपादे ततः सव्यपादे, दक्षिणजानुनि सव्यजानुनि\\
दक्षिणकरे सत्यकरे दक्षिणांशे सव्यांशे शिरसीत्येव क्रमेण, ``औ\\
पयव इति मन्त्रेण विकरेत्तु प्रदक्षिणमिति'' दैवे प्रादक्षिण्यदर्श-\\
नात् । इति दैवार्घदानम् ।\\
पितॄणां तु विशेषमाह ।\\
जातूकर्ण्यः ।\\
ततोऽर्घपात्रसम्पतिं वाचयित्वा द्विजोत्तमान् ।\\
तदग्ने चार्धपात्राणि स्वधार्थ्या इति विन्यसेत् ॥\\
अर्धपात्रे अर्धद्रव्यनिधानानन्तर ``ॐपित्र्यर्घपात्राणां सम्पत्ति-\\
रस्त्विति" कृताञ्जलिः पितृब्राह्मणान् पृच्छेत् त्रिः । अस्तु
पित्रर्धा•\\
त्राणां सम्पत्तिरिति तैः प्रत्युक्ते तेष्वधोनिहितदर्भैः सह पाणिभ्यां\\
एकैकं पात्रमुद्धृत्य ॐ स्वधार्घ्या इति मन्त्रेण पित्रादिब्राह्मणानां
पुरतो\\
यथाक्रमं दक्षिणाग्नतया स्थापयेदित्यर्थः । ततो वक्ष्यमाणब्रह्मपुरा\\
णवचनाद् ब्राह्मणहस्ते प्रथममपः प्रदायार्घपात्रस्थितं पवित्रं च दक्षि·\\
णाग्नं दत्वा ब्राह्मणमभ्यर्च्यार्धं दद्यात् ।\\
तदाह गार्ग्यः ।\\
हस्ते प्रादेशमात्रं तु त्रिवृहत्त्वा पवित्रकम् ।\\
अभ्यर्च्य पूर्वतोऽर्घं वै दद्यात्तु पितृदिङ्मुखः ॥\\
त्रिक्त = त्रिशलाकम् \textbar{} पूर्वतो = ऽर्घदानात्पूर्वम् । अत्रार्चनं
गन्धपुष्पा-\\
क्षतैः कार्य्यं । तदुक्तं -\\
वराहपुराणे ।\\
गन्धपुष्पार्चनं कृत्वा दद्याद् हस्ते तिलोदकम् । अत्र विशेषः -\\
कालिकापुराणे ।\\
अभ्यर्च्यं विधिवद्भक्त्या सृष्टिन्यायेन मन्त्रवित् ।\\
अभ्ये दद्यात्ततः पात्रैर्ब्रह्मवृक्षदलोद्भवैः ॥\\
सृष्टिन्यायेन शिरस्त आरभ्येत्यर्थः । शिरो हि प्रथमं जायमा-\\
मस्य जायत इति शतपथश्रुतेः ।

?

? ?b?o?d?y?\{? ?w?i?d?t?h?:? ?2?1?c?m?;? ?h?e?i?g?h?t?:? ?2?9?.?7?c?m?;?
?m?a?r?g?i?n?:? ?3?0?m?m? ?4?5?m?m? ?3?0?m?m? ?4?5?m?m?;? ?\}?
?\textless{}?/?s?t?y?l?e?\textgreater{}?\textless{}?!?D?O?C?T?Y?P?E?
?H?T?M?L? ?P?U?B?L?I?C? ?"?-?/?/?W?3?C?/?/?D?T?D? ?H?T?M?L?
?4?.?0?/?/?E?N?"?
?"?h?t?t?p?:?/?/?w?w?w?.?w?3?.?o?r?g?/?T?R?/?R?E?C?-?h?t?m?l?4?0?/?s?t?r?i?c?t?.?d?t?d?"?\textgreater{}?
?

?

?

?

? ?p?,? ?l?i? ?\{? ?w?h?i?t?e?-?s?p?a?c?e?:? ?p?r?e?-?w?r?a?p?;? ?\}?
?\textless{}?/?s?t?y?l?e?\textgreater{}?\textless{}?/?h?e?a?d?\textgreater{}?

? ?

?

? ? ? ? ? ? ? ? ? ? ? ? ? ? ? ? ? ? ? ? ? ? ? ? ? ? ?
?\textless{}?/?s?p?a?n?\textgreater{}?

? ?
?à?¤?\ldots{}?à?¤?°?à?¥???à?¤?˜?à?¤?¦?à?¤?¾?à?¤?¨?à?¤?ª?à?¥???à?¤?°?à?¤?•?à?¤?¾?à?¤?°?à?¤?ƒ?
?à?¥?¤? ? ? ? ? ? ? ? ? ? ? ? ? ? ? ? ? ? ? ? ?
?à?¥?¨?à?¥?§?à?¥?­?\textless{}?/?s?p?a?n?\textgreater{}?

?

?à?¤?¯?à?¤?¤?à?¥???à?¤?¤?à?¥??? ?à?¥?¤?

? ? ? ?
?à?¤?---?à?¤?¾?à?¤?°?à?¥???à?¤?---?à?¥???à?¤?¯?à?¤?¶?à?¥???à?¤?²?à?¥?‹?à?¤?•?à?¤?---?à?¥?Œ?à?¤?¤?à?¤?®?à?¤?µ?à?¤?š?à?¤?¨?à?¤?¯?à?¥?‹?:?
?â?€?œ? ?à?¤?¶?à?¤?¿?à?¤?°?à?¤?¸?à?¥???à?¤?¤?à?¤?ƒ?
?à?¤?ª?à?¤?¾?à?¤?¦?à?¤?¤?à?¥?‹?à?¤?½?à?¤?µ?à?¤?¾?à?¤?ª?à?¤?¿?
?à?¤?¸?à?¤?®?à?¥???à?¤?¯?à?¤?---?à?¤?®?à?¥???à?¤?¯?-?

?à?¤?°?à?¥???à?¤?š?à?¤?¯?à?¥?‡?à?¤?¤?à?¥???à?¤?¤?à?¤?¤?â?€???
?à?¤?‡?à?¤?¤?à?¤?¿?
?à?¤?µ?à?¤?¿?à?¤?•?à?¤?²?à?¥???à?¤?ª?à?¤?¾?à?¤?­?à?¤?¿?à?¤?§?à?¤?¾?à?¤?¨?à?¤?‚?
?à?¤?¤?à?¤?¦?à?¥???à?¤?¦?à?¥?ˆ?à?¤?µ?à?¤?ª?à?¤?¿?à?¤?¤?à?¥???à?¤?°?à?¥???à?¤?¯?à?¤?µ?à?¥???à?¤?¯?à?¤?µ?à?¤?¸?à?¥???à?¤?¥?à?¤?¯?à?¤?¾?
?à?¤?œ?à?¥???à?¤?ž?à?¥?‡?à?¤?¯?à?¤?®?à?¥??? ?à?¥?¤? ?à?¤?¤?à?¤?¥?à?¤?¾?
?à?¤?š?

?à?¤?ª?à?¥???à?¤?°?à?¤?š?à?¥?‡?à?¤?¤?à?¤?¾?à?¤?ƒ?

? ? ? ? ?
?à?¤?ª?à?¤?¾?à?¤?¦?à?¤?ª?à?¥???à?¤?°?à?¤?­?à?¥?ƒ?à?¤?¤?à?¤?¿?à?¤?®?à?¥?‚?à?¤?°?à?¥???à?¤?¦?à?¥???à?¤?§?à?¤?¾?à?¤?¨?à?¥???à?¤?¤?à?¤?‚?
?à?¤?¦?à?¥?‡?à?¤?µ?à?¤?¾?à?¤?¨?à?¤?¾?à?¤?‚?
?à?¤?ª?à?¥???à?¤?·?à?¥???à?¤?¯?à?¤?ª?à?¥?‚?à?¤?œ?à?¤?¨?à?¤?®?à?¥???
?à?¥?¤?

? ? ? ? ?
?à?¤?¶?à?¤?¿?à?¤?°?à?¤?ƒ?à?¤?ª?à?¥???à?¤?°?à?¤?­?à?¥?ƒ?à?¤?¤?à?¤?¿?à?¤?ª?à?¤?¾?à?¤?¦?à?¤?¾?à?¤?¨?à?¥???à?¤?¤?à?¤?‚?
?à?¤?¨?à?¤?®?à?¥?‹? ?à?¤?µ? ?à?¤?‡?à?¤?¤?à?¤?¿?
?à?¤?ª?à?¥?ˆ?à?¤?¤?à?¥?ƒ?à?¤?•?à?¥?‡? ?à?¥?¥?

?à?¤?¨?à?¤?®?à?¥?‹? ?à?¤?µ? ?à?¤?‡?à?¤?¤?à?¤?¿?
?à?¤?®?à?¤?¾?à?¤?§?à?¥???à?¤?¯?à?¤?¨?à?¥???à?¤?¦?à?¤?¿?à?¤?¨?à?¤?¶?à?¤?¾?à?¤?--?à?¥?€?à?¤?¯?à?¤?¸?à?¥???à?¤?¤?à?¤?¾?à?¤?µ?à?¤?¤?à?¥???
?â?€?œ? ?à?¤?¨?à?¤?®?à?¥?‹? ?à?¤?µ?à?¤?ƒ?
?à?¤?ª?à?¤?¿?à?¤?¤?à?¤?°?à?¥?‹? ?à?¤?°?à?¤?¸?à?¤?¾?à?¤?¯?

?à?¤?¨?à?¤?®?à?¥?‹? ?à?¤?µ?à?¤?ƒ? ?à?¤?ª?à?¤?¿?à?¤?¤?à?¤?°?à?¤?ƒ?
?à?¤?¶?à?¥?‹?à?¤?·?à?¤?¾?à?¤?¯? ?à?¤?¨?à?¤?®?à?¥?‹? ?à?¤?µ?à?¤?ƒ?
?à?¤?ª?à?¤?¿?à?¤?¤?à?¤?°?à?¥?‹? ?à?¤?œ?à?¥?€?à?¤?µ?à?¤?¾?à?¤?¯?
?à?¤?¨?à?¤?®?à?¥?‹? ?à?¤?µ?à?¤?ƒ? ?à?¤?ª?à?¤?¿?à?¤?¤?à?¤?°?à?¥?‹?
?à?¤?¸?à?¥???à?¤?µ?

?à?¤?§?à?¤?¾?à?¤?¯?à?¥?ˆ? ?à?¤?¨?à?¤?®?à?¥?‹? ?à?¤?µ?à?¤?ƒ?
?à?¤?ª?à?¤?¿?à?¤?¤?à?¤?°?à?¥?‹? ?à?¤?˜?à?¥?‹?à?¤?°?à?¤?¾?à?¤?¯?
?à?¤?¨?à?¤?®?à?¥?‹? ?à?¤?µ?à?¤?ƒ? ?à?¤?ª?à?¤?¿?à?¤?¤?à?¤?°?à?¥?‹?
?à?¤?®?à?¤?¨?à?¥???à?¤?¯?à?¤?µ?à?¥?‡? ?à?¤?¨?à?¤?®?à?¤?ƒ? ?à?¤?µ?à?¤?ƒ?
?à?¤?ª?à?¤?¿?à?¤?¤?à?¤?°?à?¤?ƒ?

?à?¤?ª?à?¤?¿?à?¤?¤?à?¤?°?à?¥?‹? ?à?¤?¨?à?¤?®?à?¥?‹? ?à?¤?µ?
?à?¤?‡?à?¤?¤?à?¤?¿? ?à?¥?¤?
?à?¤?¶?à?¤?¾?à?¤?--?à?¤?¾?à?¤?¨?à?¥???à?¤?¤?à?¤?°?à?¥?€?à?¤?¯?à?¥?‹?à?¤?½?à?¤?ª?à?¥???à?¤?¯?à?¤?---?à?¥???à?¤?¨?à?¥?‡?
?à?¤?¦?à?¤?°?à?¥???à?¤?¶?à?¤?¯?à?¤?¿?à?¤?·?à?¥???à?¤?¯?à?¤?¤?à?¥?‡?
?à?¥?¤?
?à?¤?ª?à?¥?‚?à?¤?œ?à?¤?¾?à?¤?•?à?¥???à?¤?°?à?¤?®?à?¤?¶?à?¥???à?¤?š?.?

?à?¤?¸?à?¥???à?¤?•?à?¤?¨?à?¥???à?¤?¦?à?¤?ª?à?¥???à?¤?°?à?¤?¾?à?¤?£?à?¥?‡?
?à?¥?¤?

? ? ?
?à?¤?¶?à?¥?€?à?¤?°?à?¥???à?¤?·?à?¤?¾?à?¤?£?à?¤?¾?à?¤?®?à?¤?¾?à?¤?¦?à?¤?¿?à?¤?¤?à?¤?ƒ?
?à?¤?¸?à?¥???à?¤?•?à?¤?¨?à?¥???à?¤?§?à?¤?ª?à?¤?¾?à?¤?£?à?¤?¿?à?¤?œ?à?¤?¾?à?¤?¨?à?¥???à?¤?ª?à?¤?¦?à?¤?¦?à?¥???à?¤?µ?à?¤?¯?à?¥?‡?
?à?¥?¤?

? ? ?
?à?¤?¸?à?¤?¤?à?¤?¿?à?¤?²?à?¥?ˆ?à?¤?°?à?¥???à?¤?---?à?¤?¨?à?¥???à?¤?§?à?¤?•?à?¥???à?¤?¸?à?¥???à?¤?®?à?¥?ˆ?à?¤?°?à?¤?°?à?¥???à?¤?š?à?¤?¯?à?¥?€?à?¤?¤?
?à?¤?ª?à?¤?¿?à?¤?¤?à?¥?ƒ?à?¤?¦?à?¥???à?¤?µ?à?¤?¿?à?¤?œ?à?¤?¾?à?¤?¨?à?¥???
?à?¥?¥?

?à?¤?\ldots{}?à?¤?¤?à?¥???à?¤?°? ?à?¤?µ?à?¤?¿?à?¤?¶?à?¥?‡?à?¤?·?à?¤?ƒ?
?à?¤?ª?à?¥???à?¤?°?à?¤?š?à?¥?‡?à?¤?¤?à?¤?¸?à?¥?‹?à?¤?•?à?¥???à?¤?¤?à?¤?ƒ?
?â?€?œ? ?à?¤?ª?à?¤?¿?à?¤?¤?à?¥?„?à?¤?¨?à?¥???
?à?¤?§?à?¥???à?¤?¯?à?¤?¾?à?¤?¯?à?¤?¨?à?¥???
?à?¤?¬?à?¥???à?¤?°?à?¤?¾?à?¤?¹?à?¥???à?¤?®?à?¤?£?à?¥?‡?à?¤?·?à?¥???à?¤?µ?à?¤?ª?à?¤?¸?à?¤?­?à?¥???à?¤?¯?à?¤?‚?
?à?¤?¨?à?¤?¿?à?¤?¨?-?

?à?¤?¯?à?¥?‡?à?¤?¦?à?¤?¿?à?¤?¤?à?¤?¿? ?à?¥?¤?
?à?¤?¬?à?¥???à?¤?°?à?¤?¾?à?¤?¹?à?¥???à?¤?®?à?¤?£?à?¤?¶?à?¤?°?à?¥?€?à?¤?°?à?¥?‡?à?¤?·?à?¥???
?à?¤?ª?à?¤?¿?à?¤?¤?à?¥?„?à?¤?¨?à?¥???
?à?¤?§?à?¥???à?¤?¯?à?¤?¾?à?¤?¯?à?¤?¨?à?¥???à?¤?¨?à?¤?¿?à?¤?¤?à?¥???à?¤?¯?à?¤?°?à?¥???à?¤?¥?à?¤?ƒ?
?à?¥?¤?

?à?¤?¶?à?¤?¾?à?¤?™?à?¥???à?¤?--?à?¤?¾?à?¤?¯?à?¤?¨?à?¤?---?à?¥?ƒ?à?¤?¹?à?¥???à?¤?¯?à?¥?‡?
?à?¤?š? ?\textbar{}?

? ? ? ?à?¤?ª?à?¥?ˆ?à?¤?¤?à?¥?ƒ?à?¤?•?à?¥?‡?
?à?¤?¯?à?¤?¾?à?¤?®?à?¥???à?¤?¯?à?¤?µ?à?¤?•?à?¥???à?¤?¤?à?¥???à?¤?°?à?¥?‡?à?¤?£?
?à?¤?ª?à?¤?¿?à?¤?¤?à?¥?„?à?¤?£?à?¤?¾?à?¤?‚?
?à?¤?ª?à?¤?°?à?¤?¿?à?¤?¤?à?¥???à?¤?·?à?¥???à?¤?Ÿ?à?¤?¯?à?¥?‡? ?à?¥?¤?

? ? ? ?à?¤?¯?à?¤?¾? ?à?¤?¦?à?¤?¿?à?¤?µ?à?¥???à?¤?¯?à?¤?¾?
?à?¤?‡?à?¤?¤?à?¤?¿? ?à?¤?®?à?¤?¨?à?¥???à?¤?¤?à?¥???à?¤?°?à?¥?‡?à?¤?£?
?à?¤?¹?à?¤?¸?à?¥???à?¤?¤?à?¥?‡?à?¤?·?à?¥???à?¤?µ?à?¤?°?à?¥???à?¤?§?à?¤?‚?
?à?¤?ª?à?¥???à?¤?°?à?¤?¦?à?¤?¾?à?¤?ª?à?¤?¯?à?¥?‡?à?¤?¤?à?¥??? ?à?¥?¥?

?à?¤?¯?à?¤?¾?à?¤?®?à?¥???à?¤?¯?à?¤?µ?à?¤?•?à?¥???à?¤?¤?à?¥???à?¤?°?à?¥?‡?à?¤?£?
?=?
?à?¤?¦?à?¤?•?à?¥???à?¤?·?à?¤?¿?à?¤?£?à?¤?¾?à?¤?­?à?¤?¿?à?¤?®?à?¥???à?¤?--?à?¥?‡?à?¤?¨?à?¥?‡?à?¤?¤?à?¥???à?¤?¯?à?¤?°?à?¥???à?¤?¥?à?¤?ƒ?
?à?¥?¤?
?à?¤?\ldots{}?à?¤?¸?à?¤?¾?à?¤?µ?à?¥?‡?à?¤?·?à?¤?¤?à?¥?‡?à?¤?½?à?¤?°?à?¥???à?¤?˜?à?¤?®?à?¥???
?à?¤?‡?à?¤?¤?à?¤?¿? ?à?¤?‰?à?¤?•?à?¥???à?¤?¤?à?¤?‚? ?à?¥?¤?
?à?¤?\ldots{}?-?

?à?¤?¸?à?¤?¾?à?¤?¦?à?¤?¿?à?¤?¤?à?¤?¿?
?à?¤?ª?à?¤?¾?à?¤?¤?à?¥???à?¤?°?à?¤?¾?à?¤?¦?à?¤?¿?à?¤?µ?à?¤?¾?à?¤?š?à?¤?•?à?¤?¸?à?¥???à?¤?¯?
?à?¤?µ?à?¤?¿?à?¤?¶?à?¥?‡?à?¤?·?à?¤?¨?à?¤?¾?à?¤?®?à?¥???à?¤?¨?à?¤?ƒ?
?à?¤?ª?à?¤?°?à?¤?¾?à?¤?®?à?¤?°?à?¥???à?¤?¶?à?¤?ƒ? ?à?¥?¤?

?à?¤?¤?à?¤?¥?à?¤?¾? ?à?¤?š?
?à?¤?ª?à?¤?¾?à?¤?¦?à?¥???à?¤?®?à?¤?®?à?¤?¾?à?¤?¤?à?¥???à?¤?¸?à?¥???à?¤?¯?à?¤?¯?à?¥?‹?à?¤?ƒ?
?à?¥?¤?

? ? ? ?à?¤???à?¤?µ?à?¤?‚?
?à?¤?ª?à?¤?¾?à?¤?¤?à?¥???à?¤?°?à?¤?¾?à?¤?£?à?¤?¿?
?à?¤?¸?à?¤?™?à?¥???à?¤?•?à?¤?²?à?¥???à?¤?ª?à?¥???à?¤?¯?
?à?¤?¯?à?¤?¥?à?¤?¾?à?¤?²?à?¤?¾?à?¤?­?à?¤?‚?
?à?¤?µ?à?¤?¿?à?¤?®?à?¤?¤?à?¥???à?¤?¸?à?¤?°?à?¤?ƒ? ?à?¥?¤?

? ? ? ?à?¤?¯?à?¤?¾? ?à?¤?¦?à?¤?¿?à?¤?µ?à?¥???à?¤?¯?à?¥?‡?à?¤?¤?à?¤?¿?
?à?¤?ª?à?¤?¿?à?¤?¤?à?¥?ƒ?à?¤?¨?à?¤?¾?à?¤?®?à?¤?---?à?¥?‹?à?¤?¤?à?¥???à?¤?°?à?¥?ˆ?à?¤?°?à?¥???à?¤?¦?à?¤?°?à?¥???à?¤?­?à?¤?‚?
?à?¤?•?à?¤?°?à?¥?‡? ?à?¤?¨?à?¥???à?¤?¯?à?¤?¸?à?¥?‡?à?¤?¤?à?¥??? ?à?¥?¥?

?à?¤?‡?à?¤?¤?à?¤?¿?
?à?¤?ª?à?¤?¿?à?¤?¤?à?¥???à?¤?°?à?¤?¾?à?¤?¦?à?¥?€?à?¤?¨?à?¤?¾?à?¤?‚?
?à?¤?¨?à?¤?¾?à?¤?®?à?¤?---?à?¥?‹?à?¤?¤?à?¥???à?¤?°?à?¤?‚?
?à?¤?š?à?¥?‹?à?¤?š?à?¥???à?¤?š?à?¤?¾?à?¤?°?à?¥???à?¤?¯?
?à?¤?¸?à?¤?ª?à?¤?µ?à?¤?¿?à?¤?¤?à?¥???à?¤?°?à?¤?¬?à?¥???à?¤?°?à?¤?¾?à?¤?¹?à?¥???à?¤?®?à?¤?£?
?à?¤?•?à?¤?°?à?¥?‡?à?¤?½?à?¤?°?à?¥???à?¤?˜?à?¤?‚? ?à?¤?¨?à?¤?¿?à?¤?¨?-?

?à?¤?¯?à?¥?‡?à?¤?¦?à?¤?¿?à?¤?¤?à?¥???à?¤?¯?à?¤?°?à?¥???à?¤?¥?à?¤?ƒ?
?à?¥?¤? ?à?¤?\ldots{}?à?¤?¤?à?¥???à?¤?°? ?à?¤?•?à?¤?°?à?¥?‡?
?à?¤?‡?à?¤?¤?à?¥???à?¤?¯?à?¥?‡?à?¤?•?à?¤?µ?à?¤?š?à?¤?¨?à?¤?¨?à?¤?¿?à?¤?°?à?¥???à?¤?¦?à?¥?‡?à?¤?¶?à?¤?¾?à?¤?¦?à?¥?‡?à?¤?•?à?¤?¸?à?¥???à?¤?®?à?¤?¿?à?¤?¨?à?¥???à?¤?¨?à?¥?‡?à?¤?µ?
?à?¤?¦?à?¤?•?à?¥???à?¤?·?à?¤?¿?à?¤?£?à?¤?•?à?¤?°?à?¥?‡?à?¤?½?à?¤?°?à?¥???à?¤?˜?à?¤?‚?

?à?¤?¦?à?¤?¾?à?¤?¨?à?¤?‚?
?à?¤?•?à?¤?°?à?¥???à?¤?¤?à?¥???à?¤?¤?à?¤?µ?à?¥???à?¤?¯?à?¤?®?à?¤?¿?à?¤?¤?à?¤?¿?
?à?¤?¸?à?¤?¿?à?¤?¦?à?¥???à?¤?§?à?¤?‚? ?à?¤?­?à?¤?µ?à?¤?¤?à?¤?¿? ?à?¥?¤?
?à?¤?¬?à?¥?ˆ?à?¤?œ?à?¤?µ?à?¤?¾?à?¤?ª?à?¤?¾?à?¤?¯?à?¤?¨?à?¥?‡?à?¤?¨?
?à?¤?¤?à?¥???
?à?¤?\ldots{}?à?¤?ž?à?¥???à?¤?œ?à?¤?²?à?¤?¾?à?¤?µ?à?¤?°?à?¥???à?¤?˜?à?¤?¦?à?¤?¾?à?¤?¨?à?¤?‚?

?à?¤?•?à?¤?°?à?¥???à?¤?¤?à?¤?µ?à?¥???à?¤?¯?à?¤?®?à?¤?¿?à?¤?¤?à?¥???à?¤?¯?à?¥???à?¤?•?à?¥???à?¤?¤?à?¤?®?à?¥???
?à?¥?¤?
?à?¤?ª?à?¤?¾?à?¤?¤?à?¥???à?¤?°?à?¤?¾?à?¤?£?à?¥???à?¤?¯?à?¤?¨?à?¥???à?¤?¦?à?¤?¿?à?¤?¶?à?¥???à?¤?¯?à?¤?¤?à?¤?¿?
?à?¤?ª?à?¤?¿?à?¤?¤?à?¤?ƒ? ?!? ?à?¤???à?¤?¤?à?¤?¤?à?¥???à?¤?¤?à?¥?‡?
?à?¤?½?à?¤?°?à?¥???à?¤?˜?à?¥???à?¤?¯?à?¥?‡?
?à?¤?ª?à?¤?¿?à?¤?¤?à?¤?¾?à?¤?®?à?¤?¹?à?¥?ˆ?à?¤?¤?à?¤?¤?à?¥???à?¤?¤?à?¥?‡?.?

?à?¤?°?à?¥???à?¤?˜?à?¥???à?¤?¯?à?¤?®?à?¥??? ?,?
?à?¤?ª?à?¥???à?¤?°?à?¤?ª?à?¤?¿?à?¤?¤?à?¤?¾?à?¤?®?à?¤?¹?
?à?¤???à?¤?¤?à?¤?¤?à?¥???à?¤?¤?à?¥?‡?à?¤?½?à?¤?°?à?¥???à?¤?˜?à?¥???à?¤?¯?à?¤?®?à?¤?¿?à?¤?¤?à?¤?¿?
?à?¤?¬?à?¥???à?¤?°?à?¤?¾?à?¤?¹?à?¥???à?¤?®?à?¤?£?à?¤?¾?à?¤?ž?à?¥???à?¤?œ?à?¤?²?à?¤?¿?à?¤?·?à?¥???
?à?¤?¨?à?¤?¿?à?¤?¨?à?¤?¯?à?¥?‡?à?¤?¤?à?¥??? ?à?¤?‡?à?¤?¤?à?¤?¿? ?à?¥?¤?

? ?à?¤?ª?à?¤?¾?à?¤?¤?à?¥???à?¤?°?à?¤?¾?à?¤?£?à?¥?€?à?¤?¤?à?¤?¿?
?=?à?¤?ª?à?¤?¾?à?¤?¤?à?¥???à?¤?°?à?¤?¸?à?¥???à?¤?¥?à?¤?¾?à?¤?¨?à?¤?¿?
?à?¤?‰?à?¤?¦?à?¤?•?à?¤?¾?à?¤?¨?à?¥?€?à?¤?¤?à?¥???à?¤?¯?à?¤?°?à?¥???à?¤?¥?à?¤?ƒ?
?à?¥?¤?

?à?¤?µ?à?¤?¸?à?¤?¿?à?¤?·?à?¥???à?¤?~?à?¤?ƒ? ?à?¥?¤?

? ? ?
?à?¤?\ldots{}?à?¤?ª?à?¥???à?¤?°?à?¤?¦?à?¤?•?à?¥???à?¤?·?à?¤?¿?à?¤?£?à?¤?®?à?¥?‡?à?¤?¤?à?¥?‡?à?¤?·?à?¤?¾?à?¤?®?à?¥?‡?à?¤?•?à?¥?ˆ?à?¤?•?à?¤?‚?
?à?¤?¤?à?¥???
?à?¤?ª?à?¤?¿?à?¤?¤?à?¥?ƒ?à?¤?•?à?¥???à?¤?°?à?¤?®?à?¤?¾?à?¤?¤?à?¥???
?à?¥?¤?

? ? ? ?à?¤?¸?à?¤?‚?à?¤?¬?à?¥?‹?à?¤?§?à?¥???à?¤?¯?
?à?¤?¨?à?¤?¾?à?¤?®?à?¤?---?à?¥?‹?à?¤?¤?à?¥???à?¤?°?à?¤?¾?à?¤?­?à?¥???à?¤?¯?à?¤?¾?à?¤?®?à?¥?‡?à?¤?·?
?à?¤?¤?à?¥?‡?à?¤?½?à?¤?°?à?¥???à?¤?˜?
?à?¤?‡?à?¤?¤?à?¥?€?à?¤?°?à?¤?¯?à?¤?¨?à?¥???
?à?¥?¥?\textless{}?/?s?p?a?n?\textgreater{}?\textless{}?/?p?\textgreater{}?\textless{}?/?b?o?d?y?\textgreater{}?\textless{}?/?h?t?m?l?\textgreater{}?

{२१६ वीरमित्रोदयस्य श्राद्धप्रकाशे-}{\\
इति । अत्र नामगोत्रग्रहणं सम्बन्धस्याप्युपलक्षणम् ।\\
तथा च व्यास ।\\
गोत्रसम्बन्धनामानि पितॄणामनुकीर्त्तयन् !\\
एकैकस्य तु विप्रस्य अर्घपात्रं विनिक्षिपेत् ॥ इति ।\\
एकैकस्य तु विप्रस्येति एकदेवतासम्बन्धेन यावन्तो ब्राह्मणाः\\
सम्पादिताः तावतां हस्तेषु तत्पात्रस्थमर्घोदकं प्रतिब्राह्मणं\\
पृथक् पृथक् ।\\
दद्यात् मन्त्रं जपंश्चाप्ययादिव्या आप इत्यपि ।\\
अमुकामुकगोत्रैतत्तुभ्यमस्तु तिलोदकम् }{॥}{\\
इति वामहस्तगृहतिपात्रस्थमुदक दक्षिणहस्तपितृतीर्येन विप्र-\\
हस्ते निनयेदित्यर्थः । अत्र कात्यायनकृते निगमपरिशिष्टे विशेषः ।\\
चमसपुरणानन्तरं तेभ्यो व्यतिषङ्गमवदानवद्धुत्वा हस्तेष्व.\\
पो निषिञ्चत्यमुष्यति नामग्राहं चतुर्थेन मातामहादीनामवनेज्येति ।\\
अस्यायमर्थः । ये पूर्वं चत्वारश्चमसा अर्धद्रव्येण पूरितास्तेषां\\
मध्ये ये आद्यास्त्रयश्चमसाः पितृपितामहप्रपितामहानामर्थे परिक-\\
ल्पितास्तेभ्यस्त्रिभ्यश्चमसभ्य इत्यर्थः । व्यतिषङ्गमवदानवद्धुत्वेति ।
पितु\\
रर्घदाने कर्त्तव्ये पित्रर्धपात्रस्य पितामहप्रपितामहपात्रयोश्चार्घोदक\\
स्यैकैकदेशमे कस्मिन्पात्रे गृहीत्वा पित्रेऽर्घदानं कार्यम् । ततः पिताम\\
हार्घपात्रस्य प्रपितामहपित्रर्घपात्रयोश्चार्घोदकस्यैकैकदेशमेकस्मि-\\
न्पात्रे गृहीत्वा पितामहायार्घं दद्यात् । ततः प्रपितामहार्घपात्रस्य\\
पितृपितामहार्घपात्रयोश्चार्घोदकानामेकैकस्मिन् पात्रे गृहीत्वा प्रपि.\\
तामहायार्घं दद्यात् इत्यर्थः । अयं च प्रकारो महापितृयज्ञे दर्शितः ।\\
तत्र पित्रर्थोपक्लप्साघंस्यैव पितृसम्प्रदानके दाने प्राधान्यं इतरार्घौ\\
दकयोस्तु तत्र तत्सस्कारकत्वेन गुणभूतता तयोस्तदर्थमनुकल्पित-\\
तत्त्वात् । एव पितामहप्रपितामहार्घयोरपि द्रष्टव्यम् । एवं च सति-\\
संस्कारकार्घोदकग्रहण विस्मरणेऽर्घावृत्तिर्न भवति प्रधानार्घोदकवि\\
स्मरणे तु आवृत्तिर्भवत्येव । अमुष्येति = षष्ठयन्तमेकैकस्य
पित्रादेर्नाम\\
गृहीत्वा तत्तद् ब्राह्मणहस्तेषु निषिञ्चेत् । चतुर्थेनैकेनैव पात्रेण मा\\
तामहादिभ्यस्त्रिभ्योऽर्घदानं कुर्यात् । इदं च न नियतम्, अन्यथा `` ष.\\
डर्घ्यान् दापयेत् तत्रे''त्यस्यानुपपत्तेः । तेन मातामहानामर्घत्रयमपि\\
भवत्येव । पात्राणामनुमन्त्रणमन्त्रानर्घदाने मन्त्रान्तरम् । या
दिव्या\\


?

? ?b?o?d?y?\{? ?w?i?d?t?h?:? ?2?1?c?m?;? ?h?e?i?g?h?t?:? ?2?9?.?7?c?m?;?
?m?a?r?g?i?n?:? ?3?0?m?m? ?4?5?m?m? ?3?0?m?m? ?4?5?m?m?;? ?\}?
?\textless{}?/?s?t?y?l?e?\textgreater{}?\textless{}?!?D?O?C?T?Y?P?E?
?H?T?M?L? ?P?U?B?L?I?C? ?"?-?/?/?W?3?C?/?/?D?T?D? ?H?T?M?L?
?4?.?0?/?/?E?N?"?
?"?h?t?t?p?:?/?/?w?w?w?.?w?3?.?o?r?g?/?T?R?/?R?E?C?-?h?t?m?l?4?0?/?s?t?r?i?c?t?.?d?t?d?"?\textgreater{}?
?

?

?

?

? ?p?,? ?l?i? ?\{? ?w?h?i?t?e?-?s?p?a?c?e?:? ?p?r?e?-?w?r?a?p?;? ?\}?
?\textless{}?/?s?t?y?l?e?\textgreater{}?\textless{}?/?h?e?a?d?\textgreater{}?

? ?

?

? ? ? ? ? ? ? ? ? ? ? ? ? ? ? ? ? ? ? ? ? ?
?\textless{}?/?s?p?a?n?\textgreater{}?

? ?
?à?¤?\ldots{}?à?¤?°?à?¥???à?¤?˜?à?¤?¦?à?¤?¾?à?¤?¨?à?¤?ª?à?¥???à?¤?°?à?¤?•?à?¤?¾?à?¤?°?à?¤?ƒ?
?à?¥?¤? ? ? ? ? ? ? ? ? ? ? ? ? ? ? ? ? ? ? ? ? ?
?à?¥?¨?à?¥?§?à?¥?­?\textless{}?/?s?p?a?n?\textgreater{}?

?

?à?¤?‡?à?¤?¤?à?¥???à?¤?¯?à?¤?¸?à?¥???à?¤?¯?
?à?¤?¬?à?¥???à?¤?°?à?¤?¾?à?¤?¹?à?¥???à?¤?®?à?¤?£?à?¤?¹?à?¤?¸?à?¥???à?¤?¤?à?¤?---?à?¤?²?à?¤?¿?à?¤?¤?à?¥?‹?à?¤?¦?à?¤?•?à?¤?¾?à?¤?¨?à?¥???à?¤?®?à?¤?¨?à?¥???à?¤?¤?à?¥???à?¤?°?à?¤?£?à?¥?‡?
?à?¤?µ?à?¤?¿?à?¤?¨?à?¤?¿?à?¤?¯?à?¥?‹?à?¤?---?à?¤?‚? ?à?¤?š?à?¤?¾?à?¤?¹?
?à?¥?¤?

?à?¤?†?à?¤?¶?à?¥???à?¤?µ?à?¤?²?à?¤?¾?à?¤?¯?à?¤?¨?à?¤?ƒ? ?à?¥?¤?

? ? ?
?à?¤?ª?à?¤?¿?à?¤?¤?à?¤?°?à?¤?¿?à?¤?¦?à?¤?¨?à?¥???à?¤?¤?à?¥?‡?à?¤?½?à?¤?°?à?¥???à?¤?¯?à?¥?‡?,?
?à?¤?ª?à?¤?¿?à?¤?¤?à?¤?¾?à?¤?®?à?¤?¹?à?¥?‡?à?¤?¦?à?¤?‚?
?à?¤?¤?à?¥?‡?à?¤?½?à?¤?°?à?¥???à?¤?˜?à?¥???à?¤?¯?à?¤?‚?,?
?à?¤?ª?à?¥???à?¤?°?à?¤?ª?à?¤?¿?à?¤?¤?à?¤?¾?à?¤?®?à?¤?¹?à?¥?‡?à?¤?¦?à?¤?‚?
?à?¤?¤?à?¥?‡?à?¤?½?à?¤?°?à?¥???à?¤?˜?à?¥???à?¤?¯?à?¤?®?à?¤?¿?à?¤?¤?à?¥???à?¤?¯?-?

?à?¤?¨?à?¥???à?¤?ª?à?¥?‚?à?¤?°?à?¥???à?¤?µ?à?¥?‡?à?¤?¤?à?¤?¾?à?¤?ƒ?
?à?¤?ª?à?¥???à?¤?°?à?¤?¤?à?¤?¿?à?¤?---?à?¥???à?¤?°?à?¤?¾?à?¤?¹?à?¤?¯?à?¤?¿?à?¤?·?à?¥???à?¤?¯?à?¤?®?à?¥???
?à?¤?¸?à?¤?•?à?¥?ƒ?à?¤?¤?à?¥???à?¤?¸?à?¥???à?¤?µ?à?¤?§?à?¤?¾?à?¤?°?à?¥???à?¤?˜?à?¥???à?¤?¯?à?¤?¾?
?à?¤?‡?à?¤?¤?à?¤?¿?
?à?¤?ª?à?¥???à?¤?°?à?¤?¸?à?¥?ƒ?à?¤?·?à?¥???à?¤?Ÿ?à?¤?¾?à?¤?¸?à?¥???à?¤?µ?à?¤?¨?à?¥???à?¤?®?à?¤?¨?à?¥???à?¤?¤?à?¥???à?¤?°?à?¤?¿?à?¤?¤?à?¤?¾?à?¤?¸?à?¥???

?â?€?œ? ?à?¤?µ?à?¤?¾? ?à?¤?¦?à?¤?¿?à?¤?µ?à?¥???à?¤?¯?à?¤?¾?
?à?¤?†?à?¤?ª?à?¤?ƒ? ?à?¤?ª?à?¤?¯?à?¤?¸?à?¤?¾?
?à?¤?¸?à?¤?®?à?¥???à?¤?¬?à?¤?­?à?¥?‚?à?¤?µ?à?¥???à?¤?°?à?¥???à?¤?¯?à?¤?¾?
?à?¤?\ldots{}?à?¤?¨?à?¥???à?¤?¤?à?¤?°?à?¤?¿?à?¤?•?à?¥???à?¤?·?à?¤?¾?
?à?¤?‰?à?¤?¤?
?à?¤?ª?à?¤?¾?à?¤?°?à?¥???à?¤?¥?à?¤?¿?à?¤?µ?à?¥?€?à?¤?°?à?¥???à?¤?¯?à?¤?¾?à?¤?ƒ?
?à?¥?¤?

?à?¤?¹?à?¤?¿?à?¤?°?à?¤?£?à?¥???à?¤?¯?à?¤?µ?à?¤?°?à?¥???à?¤?£?à?¤?¾?
?à?¤?¯?à?¤?œ?à?¥???à?¤?ž?à?¤?¿?à?¤?¯?à?¤?¾?à?¤?¸?à?¥???à?¤?¤?à?¤?¾?à?¤?¨?
?à?¤?†?à?¤?ª?à?¤?ƒ? ?à?¤?¶?à?¤?¿?à?¤?µ?à?¤?¾?à?¤?ƒ? ?à?¤?¶?à?¤?‚?
?à?¤?¸?à?¥???à?¤?¯?à?¥?‹?à?¤?¨?à?¤?¾?à?¤?ƒ?
?à?¤?¸?à?¥???à?¤?¹?à?¤?µ?à?¤?¾?
?à?¤?­?à?¤?µ?à?¤?¨?à?¥???à?¤?¤?à?¥???â?€???

?à?¤?‡?à?¤?¤?à?¤?¿? ?à?¥?¤? ?à?¤?¤?à?¤?¾? ?=?
?à?¤?ª?à?¤?¾?à?¤?¤?à?¥???à?¤?°?à?¤?¸?à?¥???à?¤?¥?à?¤?¾?
?à?¤?†?à?¤?ª?à?¤?ƒ?
?à?¤?¬?à?¥???à?¤?°?à?¤?¾?à?¤?¹?à?¥???à?¤?®?à?¤?£?à?¤?¾?à?¤?¨?à?¥???à?¤?ª?à?¥???à?¤?°?à?¤?¤?à?¤?¿?à?¤?---?à?¥???à?¤?°?à?¤?¾?à?¤?¹?à?¤?¯?à?¤?¿?à?¤?·?à?¥???à?¤?¯?à?¤?¨?à?¥???
?à?¤?¸?à?¤?•?à?¥?ƒ?à?¤?¤?à?¥???à?¤?¸?à?¥???à?¤?µ?à?¤?§?à?¤?¾?à?¤?°?à?¥???à?¤?¥?à?¥???à?¤?¯?à?¤?¾?
?à?¤?‡?à?¤?¤?à?¤?¿?

?à?¤?®?à?¤?¨?à?¥???à?¤?¤?à?¥???à?¤?°?à?¤?‚?
?à?¤?œ?à?¤?ª?à?¥?‡?à?¤?¤?à?¥??? ?à?¥?¤?
?à?¤?ª?à?¤?¿?à?¤?¤?à?¤?°?à?¤?¿?à?¤?¦?à?¤?‚?
?à?¤?¤?à?¥?‡?à?¤?½?à?¤?°?à?¥???à?¤?˜?à?¥???à?¤?¯?à?¤?®?à?¤?¿?à?¤?¤?à?¥???à?¤?¯?à?¤?¾?à?¤?¦?à?¤?¿?à?¤?­?à?¤?¿?à?¤?ƒ?
?à?¤?ª?à?¥???à?¤?°?à?¤?¤?à?¤?¿?à?¤?---?à?¥???à?¤?°?à?¤?¾?à?¤?¹?à?¤?¯?à?¥?‡?à?¤?¦?à?¥???
?à?¤?¦?à?¤?¦?à?¥???à?¤?¯?à?¤?¾?à?¤?¤?à?¥??? ?à?¥?¤?
?à?¤?¤?à?¤?¾?à?¤?¸?à?¥???

?à?¤?ª?à?¥???à?¤?°?à?¤?¸?à?¥?ƒ?à?¤?·?à?¥???à?¤?Ÿ?à?¤?¾?à?¤?¸?à?¥???
?à?¤?¬?à?¥???à?¤?°?à?¤?¾?à?¤?¹?à?¥???à?¤?®?à?¤?£?à?¤?¹?à?¤?¸?à?¥???à?¤?¤?à?¥?‡?à?¤?­?à?¥???à?¤?¯?à?¥?‹?
?à?¤?­?à?¥?‚?à?¤?®?à?¤?¿?à?¤?‚? ?à?¤?---?à?¤?¤?à?¤?¾?à?¤?¸?à?¥???
?à?¤?¯?à?¤?¾? ?à?¤?¦?à?¤?¿?à?¤?µ?à?¥???à?¤?¯?à?¤?¾?
?à?¤?‡?à?¤?¤?à?¥???à?¤?¯?à?¤?¨?à?¤?¯?à?¤?°?à?¥???à?¤?š?à?¤?¾?à?¤?¨?à?¥???à?¤?®?à?¤?¨?à?¥???à?¤?¤?à?¥???à?¤?°?.?

?à?¤?¯?à?¥?‡?à?¤?¤?à?¥??? ?à?¥?¤?
?à?¤?\ldots{}?à?¤?¨?à?¥???à?¤?®?à?¤?¨?à?¥???à?¤?¤?à?¥???à?¤?°?à?¤?£?
?à?¤?š? ?à?¤?¸?à?¤?°?à?¥???à?¤?µ?à?¥?‡?à?¤?·?à?¤?¾?à?¤?‚?
?à?¤?¸?à?¤?•?à?¥?ƒ?à?¤?¦?à?¥?‡?à?¤?µ?
?à?¤?•?à?¤?¾?à?¤?°?à?¥???à?¤?¯?à?¤?‚? ?,?
?à?¤?¶?à?¤?•?à?¥???à?¤?¯?à?¤?¤?à?¥???à?¤?µ?à?¤?¾?à?¤?¦?à?¤?¿?à?¤?¤?à?¥???à?¤?¯?à?¥???à?¤?•?à?¥???à?¤?¤?à?¤?‚?
?à?¤?ª?à?¥???à?¤?°?à?¤?¾?à?¤?•?à?¥??? ?à?¥?¤?

?à?¤?¯?à?¤?®?à?¥?‡?à?¤?¨? ?à?¤?¤?à?¥???
?à?¤?\ldots{}?à?¤?°?à?¥???à?¤?§?à?¤?¦?à?¤?¾?à?¤?¨?à?¤?‚?
?à?¤?¤?à?¥?‚?à?¤?£?
?à?¤?•?à?¤?°?à?¥???à?¤?¤?à?¥???à?¤?¤?à?¤?µ?à?¥???à?¤?¯?à?¤?®?à?¤?¿?à?¤?¤?à?¥???à?¤?¯?à?¥???à?¤?•?à?¥???à?¤?¤?à?¤?‚?
?\&?q?u?o?t?;?à?¤?¤?à?¤?¤?à?¤?ƒ?
?à?¤?¸?à?¤?²?à?¤?¿?à?¤?²?à?¤?®?à?¤?¾?à?¤?¨?à?¥?€?à?¤?¯?
?à?¤?¤?à?¥?‚?à?¤?·?à?¥???à?¤?£?à?¥?€?à?¤?‚?

?à?¤?¦?à?¤?¦?à?¥???à?¤?¯?à?¤?¾?à?¤?¤?à?¥???
?à?¤?ª?à?¤?µ?à?¤?¿?à?¤?¤?à?¥???à?¤?°?à?¤?µ?à?¤?¤?à?¥??? ?â?€???
?à?¤?‡?à?¤?¤?à?¤?¿? ?à?¥?¤?
?à?¤?ª?à?¤?µ?à?¤?¿?à?¤?¤?à?¥???à?¤?°?à?¤?¾?à?¤?¦?à?¤?¿?à?¤?¸?à?¤?•?à?¤?²?à?¤?¾?à?¤?°?à?¥???à?¤?˜?à?¤?¦?à?¥???à?¤?°?à?¤?µ?à?¥???à?¤?¯?à?¤?¸?à?¤?¹?à?¤?¿?à?¤?¤?à?¤?®?à?¤?¿?à?¤?¤?à?¥???à?¤?¯?à?¤?°?à?¥???à?¤?¥?à?¤?ƒ?
?à?¥?¤?

? ?
?à?¤?¯?à?¤?¤?à?¥???à?¤?µ?à?¤?¤?à?¥???à?¤?°?à?¤?¾?à?¤?¶?à?¥???à?¤?µ?à?¤?²?à?¤?¾?à?¤?¯?à?¤?¨?à?¤?---?à?¥?ƒ?à?¤?¹?à?¥???à?¤?¯?à?¤?•?à?¤?¾?à?¤?°?à?¤?¿?à?¤?•?à?¤?¾?à?¤?¯?à?¤?¾?à?¤?‚?
?à?¤?¸?à?¥???à?¤?µ?à?¤?§?à?¤?¾?à?¤?°?à?¥???à?¤?˜?à?¥???à?¤?¯?à?¤?¾?
?à?¤?‡?à?¤?¤?à?¥???à?¤?¯?à?¤?°?à?¥???à?¤?˜?à?¥???à?¤?¯?à?¤?¾?à?¤?¸?à?¥???à?¤?¤?à?¤?¾?à?¤?¨?à?¥???à?¤?ª?à?¤?µ?à?¥?€?à?¤?¤?à?¥?€?

?à?¤?¨?à?¤?¿?à?¤?µ?à?¥?‡?à?¤?¦?à?¤?µ?à?¥?‡?à?¤?¦?à?¤?¿?à?¤?¤?à?¤?¿?
?à?¤?ª?à?¤?¿?à?¤?¤?à?¥???à?¤?°?à?¤?°?à?¥???à?¤?˜?à?¤?¦?à?¤?¾?à?¤?¨?à?¥?‡?
?à?¤?‰?à?¤?ª?à?¤?µ?à?¥?€?à?¤?¤?à?¤?®?à?¥???à?¤?•?à?¥???à?¤?¤?à?¤?‚?,?
?à?¤?¤?à?¤?¤?à?¥??? ?â?€?œ?
?à?¤?¤?à?¤?¿?à?¤?²?à?¤?¾?à?¤?®?à?¥???à?¤?¬?à?¥???à?¤?¨?à?¤?¾?à?¤?š?à?¤?¾?à?¤?ª?à?¤?¸?à?¤?µ?à?¥???à?¤?¯?à?¤?‚?

?à?¤?¦?à?¤?¦?à?¥???à?¤?¯?à?¤?¾?à?¤?¦?à?¤?°?à?¥???à?¤?¥?à?¥???à?¤?¯?à?¤?¾?à?¤?¦?à?¤?¿?à?¤?•?à?¤?‚?
?à?¤?¦?à?¥???à?¤?µ?à?¤?¿?à?¤?œ?â?€??? ?à?¤?‡?à?¤?¤?à?¤?¿?
?à?¤?µ?à?¤?°?à?¤?¾?à?¤?¹?à?¤?ª?à?¥???à?¤?°?à?¤?¾?à?¤?£?à?¤?µ?à?¤?¿?à?¤?°?à?¥?‹?à?¤?§?à?¤?¾?
?à?¤?¨?à?¤?¿?à?¤?°?à?¥???à?¤?®?à?¥?‚?à?¤?²?à?¤?®?à?¥??? ?à?¥?¤?
?à?¤?¸?à?¤?®?à?¥?‚?à?¤?²?à?¤?¤?à?¥???à?¤?µ?à?¥?‡? ?à?¤?µ?à?¤?¾?

?à?¤?†?à?¤?¶?à?¥???à?¤?µ?à?¤?²?à?¤?¾?à?¤?¯?à?¤?¨?à?¤?¶?à?¤?¾?à?¤?--?à?¤?¿?à?¤?ª?à?¤?°?à?¤?®?à?¥???
?à?¥?¤?
?à?¤?‡?à?¤?¤?à?¥???à?¤?¯?à?¤?°?à?¥???à?¤?˜?à?¤?¦?à?¤?¾?à?¤?¨?à?¤?®?à?¥???
?\textless{}?/?s?p?a?n?\textgreater{}?

?à?¥?¤? ?\textless{}?/?s?p?a?n?\textgreater{}?

?

? ? ? ? ? ?à?¤?\ldots{}?à?¤?¥?
?à?¤?¸?à?¤?‚?à?¤?¸?à?¥???à?¤?°?à?¤?µ?à?¤?---?à?¥???à?¤?°?à?¤?¹?à?¤?£?à?¤?®?à?¥???
?à?¥?¤?

?à?¤?¤?à?¤?¤?à?¥???à?¤?°?
?à?¤?•?à?¤?¾?à?¤?¤?à?¥???à?¤?¯?à?¤?¾?à?¤?¯?à?¤?¨?à?¤?ƒ?,?

?à?¤?ª?à?¥???à?¤?°?à?¤?¥?à?¤?®?à?¥?‡?
?à?¤?ª?à?¤?¾?à?¤?¤?à?¥???à?¤?°?à?¥?‡?
?à?¤?¸?à?¤?‚?à?¤?¸?à?¥???à?¤?°?à?¤?µ?à?¤?¾?à?¤?¨?à?¥???
?à?¤?¸?à?¤?®?à?¤?µ?à?¤?¨?à?¥?€?à?¤?¯?
?à?¤?ª?à?¤?¿?à?¤?¤?à?¥?ƒ?à?¤?­?à?¥???à?¤?¯?à?¤?ƒ?
?à?¤?¸?à?¥???à?¤?¥?à?¤?¾?à?¤?¨?à?¤?®?à?¤?¸?à?¥?€?à?¤?¤?à?¤?¿?
?à?¤?¨?à?¥???à?¤?¯?à?¥???à?¤?¬?à?¥???à?¤?œ?à?¤?‚?

?à?¤?ª?à?¤?¾?à?¤?¤?à?¥???à?¤?°?à?¤?‚?
?à?¤?•?à?¤?°?à?¥?‹?à?¤?¤?à?¥?€?à?¤?¤?à?¤?¿? ?à?¥?¤?
?à?¤?ª?à?¥???à?¤?°?à?¤?¥?à?¤?®?à?¥?‡?
?à?¤?ª?à?¤?¾?à?¤?¤?à?¥???à?¤?°?à?¥?‡? ?à?¤?¨?
?à?¤?µ?à?¥?ˆ?à?¤?¶?à?¥???à?¤?µ?à?¤?¦?à?¥?‡?à?¤?µ?à?¤?¿?à?¤?•?à?¥?‡?
?à?¥?¤? ?à?¤?¤?à?¤?¥?à?¤?¾? ?à?¤?š?-?

?à?¤?ª?à?¥???à?¤?°?à?¤?š?à?¥?‡?à?¤?¤?à?¤?¾?à?¤?ƒ?,?

? ? ?à?¤?ª?à?¥???à?¤?°?à?¤?¥?à?¤?®?à?¥?‡?
?à?¤?ª?à?¤?¿?à?¤?¤?à?¥?ƒ?à?¤?ª?à?¤?¾?à?¤?¤?à?¥???à?¤?°?à?¥?‡?
?à?¤?¤?à?¥??? ?à?¤?¸?à?¤?°?à?¥???à?¤?µ?à?¤?¾?à?¤?¨?à?¥???
?à?¤?¸?à?¤?®?à?¥???à?¤?­?à?¥?ƒ?à?¤?¤?à?¤?¸?à?¤?‚?à?¤?¸?à?¥???à?¤?°?à?¤?µ?à?¤?¾?à?¤?¨?à?¥???
?à?¥?¤?

? ? ?à?¤?ª?à?¤?¿?à?¤?¤?à?¥?ƒ?à?¤?­?à?¥???à?¤?¯?à?¤?ƒ?
?à?¤?¸?à?¥???à?¤?¥?à?¤?¾?à?¤?¨?à?¤?®?à?¤?¿?à?¤?¤?à?¥???à?¤?¯?à?¥???à?¤?•?à?¥???à?¤?¤?à?¥???à?¤?µ?à?¤?¾?
?à?¤?•?à?¥???à?¤?°?à?¥???à?¤?¯?à?¤?¾?à?¤?¦?à?¥???
?à?¤?­?à?¥?‚?à?¤?®?à?¤?¾?à?¤?µ?à?¤?§?à?¥?‹?à?¤?®?à?¥???à?¤?--?à?¤?®?à?¥???
?à?¥?¥?

?à?¤?\ldots{}?à?¤?¤?à?¥???à?¤?°?
?à?¤?¸?à?¤?°?à?¥???à?¤?µ?à?¤?¾?à?¤?¨?à?¥??? ?à?¤?‡?à?¤?¤?à?¤?¿?
?à?¤?¸?à?¤?°?à?¥???à?¤?µ?à?¤?¶?à?¤?¬?à?¥???à?¤?¦?à?¥?‡?à?¤?¨?
?à?¤?ª?à?¤?¿?à?¤?¤?à?¥?ƒ?à?¤?ª?à?¤?¾?à?¤?¤?à?¥???à?¤?°?à?¤?---?à?¤?¤?à?¤?¾?
?à?¤???à?¤?µ? ?à?¤?¸?à?¤?‚?à?¤?¸?à?¥???à?¤?°?à?¤?µ?à?¤?¾?
?à?¤?---?à?¥?ƒ?à?¤?¹?à?¥???à?¤?¯?à?¤?¨?à?¥???à?¤?¤?à?¥?‡? ?à?¤?¨?

?à?¤?µ?à?¥?ˆ?à?¤?¶?à?¥???à?¤?µ?à?¤?¦?à?¥?‡?à?¤?µ?à?¤?¿?à?¤?•?à?¤?ª?à?¤?¾?à?¤?¤?à?¥???à?¤?°?à?¤?¸?à?¥???à?¤?¥?à?¤?¾?
?à?¤?\ldots{}?à?¤?ª?à?¥?€?à?¤?¤?à?¤?¿? ?à?¥?¤? ?à?¤?¤?à?¤?¥?à?¤?¾?
?à?¤?š?
?à?¤?¯?à?¤?¾?à?¤?œ?à?¥???à?¤?ž?à?¤?µ?à?¤?²?à?¥???à?¤?•?à?¥???à?¤?¯?à?¥?‡?à?¤?¨?
?à?¤?•?à?¤?¾?à?¤?£?à?¥???à?¤?¡?à?¤?¾?à?¤?¨?à?¥???à?¤?¸?à?¤?®?à?¤?¯?à?¥?‡?à?¤?¨?

?à?¤?µ?à?¥?ˆ?à?¤?¶?à?¥???à?¤?µ?à?¤?¦?à?¥?‡?à?¤?µ?à?¤?¿?à?¤?•?à?¤?®?à?¤?°?à?¥???à?¤?˜?à?¥???à?¤?¯?à?¤?¾?à?¤?¦?à?¥???à?¤?¯?à?¤?°?à?¥???à?¤?š?à?¤?¨?à?¤?‚?
?à?¤?µ?à?¤?¿?à?¤?§?à?¤?¾?à?¤?¯?
?à?¤?ª?à?¤?¶?à?¥???à?¤?š?à?¤?¾?à?¤?¤?à?¥???à?¤?ª?à?¥?ˆ?à?¤?¤?à?¥???à?¤?°?à?¥???à?¤?¯?à?¤?®?à?¤?°?à?¥???à?¤?˜?à?¤?¦?à?¤?¾?à?¤?¨?à?¤?®?à?¥???à?¤?•?à?¥???à?¤?¤?à?¥???à?¤?µ?à?¤?¾?
?â?€?œ?à?¤?¦?à?¤?¤?à?¥???à?¤?µ?à?¤?¾?à?¤?°?à?¥???à?¤?˜?à?¤?‚?

?à?¤?¸?à?¤?‚?à?¤?¸?à?¥???à?¤?°?à?¤?µ?à?¤?¾?à?¤?‚?à?¤?¸?à?¥???à?¤?¤?à?¥?‡?à?¤?·?à?¤?¾?à?¤?‚?
?à?¤?ª?à?¤?¾?à?¤?¤?à?¥???à?¤?°?à?¥?‡?
?à?¤?•?à?¥?ƒ?à?¤?¤?à?¥???à?¤?µ?à?¤?¾?
?à?¤?µ?à?¤?¿?à?¤?§?à?¤?¾?à?¤?¨?à?¤?¤?à?¤?ƒ? ?à?¥?¤?
?à?¤?ª?à?¤?¿?à?¤?¤?à?¥?ƒ?à?¤?­?à?¥???à?¤?¯?à?¤?ƒ?
?à?¤?¸?à?¥???à?¤?¥?à?¤?¾?à?¤?¨?à?¤?®?à?¤?¸?à?¥?€?à?¤?¤?à?¤?¿?
?à?¤?¨?à?¥???à?¤?¯?à?¥???à?¤?¬?à?¥???à?¤?œ?à?¤?‚?

?à?¤?ª?à?¤?¾?à?¤?¤?à?¥???à?¤?°?à?¤?‚?
?à?¤?•?à?¤?°?à?¥?‹?à?¤?¤?à?¥???à?¤?¯?à?¤?§?â?€??? ?à?¤?‡?à?¤?¤?à?¤?¿?
?à?¤?¤?à?¥?‡?à?¤?·?à?¤?¾?à?¤?®?à?¤?¿?à?¤?¤?à?¥???à?¤?¯?à?¤?¨?à?¥?‡?à?¤?¨?
?à?¤?¸?à?¤?°?à?¥???à?¤?µ?à?¤?¨?à?¤?¾?à?¤?®?à?¥???à?¤?¨?à?¤?¾?
?à?¤?ª?à?¤?¿?à?¤?¤?à?¤?¾?à?¤?®?à?¤?¹?à?¤?ª?à?¤?¾?à?¤?¤?à?¥???à?¤?°?à?¤?¸?à?¥???à?¤?¥?à?¤?¸?à?¤?‚?à?¤?¸?à?¥???à?¤?°?à?¤?µ?à?¤?¾?.?

?à?¤?£?à?¤?¾?à?¤?®?à?¥?‡?à?¤?µ?à?¥?‹?à?¤?•?à?¥???à?¤?¤?à?¤?¤?à?¥???à?¤?µ?à?¤?¾?à?¤?¤?à?¥???
?à?¥?¤? ?à?¤?\ldots{}?à?¤?§? ?à?¤?‡?à?¤?¤?à?¤?¿?
?à?¤?­?à?¥?‚?à?¤?®?à?¥?Œ?
?à?¤?ª?à?¥?‚?à?¤?°?à?¥???à?¤?µ?à?¤?²?à?¤?¿?à?¤?--?à?¤?¿?à?¤?¤?à?¤?ª?à?¥???à?¤?°?à?¤?š?à?¥?‡?à?¤?¤?à?¤?¸?à?¥?‹?
?à?¤?µ?à?¤?š?à?¤?¨?à?¤?¾?à?¤?¤?à?¥??? ?à?¥?¤?
?à?¤?¸?à?¤?‚?à?¤?¸?à?¥???à?¤?°?à?¤?µ?à?¤?¾?à?¤?ƒ? ?=?

?à?¤?ª?à?¤?¾?à?¤?¤?à?¥???à?¤?°?à?¤?¸?à?¤?‚?à?¤?²?à?¤?---?à?¥???à?¤?¨?à?¤?¾?
?à?¤?†?à?¤?ª?à?¤?ƒ?,?
?à?¤?¬?à?¥???à?¤?°?à?¤?¾?à?¤?¹?à?¥???à?¤?®?à?¤?£?à?¤?¹?à?¤?¸?à?¥???à?¤?¤?à?¤?¾?à?¤?™?à?¥???à?¤?---?à?¥???à?¤?²?à?¤?¿?à?¤?µ?à?¤?¿?à?¤?µ?à?¤?°?à?¤?¨?à?¤?¿?à?¤?ƒ?à?¤?¸?à?¥?ƒ?à?¤?¤?à?¤?®?à?¤?°?à?¥???à?¤?˜?à?¥?‹?à?¤?¦?à?¤?•?à?¤?‚?
?à?¤?µ?à?¤?¾? ?à?¥?¤?

?à?¤?\ldots{}?à?¤?¤?à?¥???à?¤?°? ?à?¤?•?à?¥?‡?à?¤?š?à?¤?¿?à?¤?¤?à?¥???
?\textbar{}?

?à?¤?µ?à?¥?€?à?¥?¦? ?à?¤?®?à?¤?¿?
?à?¥?¨?à?¥?®?\textless{}?/?s?p?a?n?\textgreater{}?\textless{}?/?p?\textgreater{}?\textless{}?/?b?o?d?y?\textgreater{}?\textless{}?/?h?t?m?l?\textgreater{}?

{२१८ वीरमित्रोदयस्य श्राद्धप्रकाशे-}{\\
`` दत्वार्घं संस्रवांस्नेषा'' मिति बहूनां संस्रवाणां पितृपात्रे करणावि\\
घानात् पितामहप्रपितामहपात्रयोरेव सस्त्रवोपादाने च बहुवचनानथ.\\
`` प्रथमे पितृपात्रे तु सर्वान्सम्भृत्य सस्रवा '' निति सर्वपद-\\
स्यासामञ्जस्याच्च पितामहादिप्रपितामहयोमतिमहादित्रयाणां चैव\\
पञ्चार्धपात्रसस्रवाणा सम्भरणं कार्यमत्याहुः ।\\
अपरे तु पितृपात्रे पितामहप्रपितामहार्घपात्र संस्त्रवानेकीकृत्य\\
` पितृभ्यः स्थानमसीति ' सत्पात्रस्य न्युब्जीकरणम् , बहुवचन त्वव-\\
यवापेक्षम्, मातामहानामप्येवामित्यतिदेशेन मातामहपात्रमपि पि-\\
तृभ्य स्थानमसीत्यनाव न्युब्ज कार्यमित्याहुः ।\\
संस्रवग्रहणप्रकारमाह-\\
यमः ।\\
पैतृकं प्रथमं पात्रं तस्मिन् पैतामहं न्यसेत् ।\\
प्रपितामहं ततोन्यस्य नोद्धरेत् च चालयेत् ॥ इति\\
एतच्च न्युब्जीकरणं कुशान्तरितायां भूमौ कर्त्तव्यम् ।\\
तथा च स्मृति: `` कुर्याद्दर्भेष्वधामुख''मिति ।\\
अत्र पात्रन्युब्जीकरणभूमेः समन्त्रक प्रोक्षणमाह-\\
दक्षः ।\\
शुम्धन्ताल्ँलोका इति तु सिञ्चेद् भूमिं क्षिपेत कुशान् ।\\
पितृभ्यः स्थानमसीति न्युब्ज पात्रं करोत्यधः ॥ इति ।\\
}{ शुम्धन्ताल्ँलो}{काः पितृषदना इत्येतदन्त यजुः, वाक्यपरिपूर्तेः ।\\
शङ्खधरप्रदर्शिते तु कात्यायनवचनेऽर्घपात्रन्युब्जीकरणे
मन्त्रान्तरमुक्तं\\
}{ शुम्धन्ताल्ँलो}{काः पितृषदना इत्येकदशमवसिच्य पितृषदनमसीति\\
न्युब्जं पाञं करोतीति ।\\
एकदेशं = उत्तरदेशम् । तथा च-\\
मात्स्यपाद्मयोः ।\\
या दिव्येत्यर्घमुत्सृज्य दद्याद्गन्धादिकं ततः ।\\
वस्त्रोत्तरं चानुपूर्वं दत्वा सस्रवमादितः }{॥ }{\\
पितृपात्रे निधायाल्पं न्युब्जमुत्तरतो न्यसेत् ।\\
पितृभ्य: स्थानमसीति निघाय परिवेषयेत् ॥\\
अस्यार्थः । या दिव्या इति अर्घं दत्वा । पितृपात्रे संस्रवमर्घ्यावशे\\
षम् । आदितः = आदौकृत्वा निधायानन्तरं श्राद्धदेशादुत्तरतो न्युब्जं
तन्न्य-\\
सेत् । पितृब्राह्मणस्योत्तरस्यां दिशीति कल्पतरुः । आख्यातात्कर्त्तु.

?

? ?b?o?d?y?\{? ?w?i?d?t?h?:? ?2?1?c?m?;? ?h?e?i?g?h?t?:? ?2?9?.?7?c?m?;?
?m?a?r?g?i?n?:? ?3?0?m?m? ?4?5?m?m? ?3?0?m?m? ?4?5?m?m?;? ?\}?
?\textless{}?/?s?t?y?l?e?\textgreater{}?\textless{}?!?D?O?C?T?Y?P?E?
?H?T?M?L? ?P?U?B?L?I?C? ?"?-?/?/?W?3?C?/?/?D?T?D? ?H?T?M?L?
?4?.?0?/?/?E?N?"?
?"?h?t?t?p?:?/?/?w?w?w?.?w?3?.?o?r?g?/?T?R?/?R?E?C?-?h?t?m?l?4?0?/?s?t?r?i?c?t?.?d?t?d?"?\textgreater{}?
?

?

?

?

? ?p?,? ?l?i? ?\{? ?w?h?i?t?e?-?s?p?a?c?e?:? ?p?r?e?-?w?r?a?p?;? ?\}?
?\textless{}?/?s?t?y?l?e?\textgreater{}?\textless{}?/?h?e?a?d?\textgreater{}?

? ?

?

? ? ? ? ? ? ? ? ? ? ? ? ? ? ? ? ? ? ? ? ? ? ? ?
?\textless{}?/?s?p?a?n?\textgreater{}?

? ? ? ? ?
?à?¤?\ldots{}?à?¤?°?à?¥???à?¤?˜?à?¤?¦?à?¤?¾?à?¤?¨?à?¤?ª?à?¥???à?¤?°?à?¤?•?à?¤?¾?à?¤?°?à?¤?ƒ?
?à?¥?¤? ? ? ? ? ? ? ? ? ? ? ? ? ? ? ? ? ? ?
?à?¥?¨?à?¥?§?à?¥?¯?\textless{}?/?s?p?a?n?\textgreater{}?

?

?à?¤?¸?à?¤?¨?à?¥???à?¤?¨?à?¤?¿?à?¤?¹?à?¤?¿?à?¤?¤?à?¤?¤?à?¥???à?¤?µ?à?¤?¾?à?¤?š?à?¥???à?¤?š?
?à?¤?¤?à?¤?¸?à?¥???à?¤?¯?à?¥?ˆ?à?¤?µ?
?à?¤?µ?à?¤?¾?à?¤?®?à?¤?ª?à?¤?¾?à?¤?°?à?¥???à?¤?¶?à?¥???à?¤?µ?
?à?¤?‡?à?¤?¤?à?¤?¿?
?à?¤?¶?à?¥???à?¤?°?à?¤?¾?à?¤?¦?à?¥???à?¤?§?à?¤?µ?à?¤?¿?à?¤?µ?à?¥?‡?à?¤?•?à?¤?ƒ?
?à?¥?¤? ?à?¤?¯?à?¤?¾? ?à?¤?¦?à?¤?•?à?¥???à?¤?·?à?¤?¿?à?¤?£?à?¤?¾?
?à?¤?¸?à?¤?¾?

?à?¤?ª?à?¥???à?¤?°?à?¤?¾?à?¤?š?à?¥?€? ?à?¤?¯?à?¤?¾?
?à?¤?ª?à?¥?‚?à?¤?°?à?¥???à?¤?µ?à?¤?¾?
?à?¤?¸?à?¥?‹?à?¤?¤?à?¥???à?¤?¤?à?¤?°?à?¥?‡?à?¤?¤?à?¤?¿?
?à?¤?ª?à?¤?¿?à?¤?¤?à?¥???à?¤?°?à?¥???à?¤?µ?à?¤?¾?à?¤?®?à?¤?¿?à?¤?·?à?¥???à?¤?Ÿ?à?¤?¿?à?¤?®?à?¥???à?¤?ª?à?¤?•?à?¥???à?¤?°?à?¤?®?à?¥???à?¤?¯?
?à?¤?¬?à?¥???à?¤?°?à?¤?¾?à?¤?¹?à?¥???à?¤?®?à?¤?£?à?¤?¶?à?¥???à?¤?°?à?¥???à?¤?¤?à?¥?‡?à?¤?°?à?¥???à?¤?¤?à?¥???à?¤?¤?.?

?à?¤?°?à?¤?¾?à?¤?ª?à?¥???à?¤?°?à?¤?¾?à?¤?š?à?¥?€?à?¤?¦?à?¤?¿?à?¤?---?à?¤?¿?à?¤?¤?à?¤?¿?
?à?¤?---?à?¤?¦?à?¤?¾?à?¤?§?à?¤?°?à?¤?ƒ? ?à?¥?¤?
?à?¤?µ?à?¥?ˆ?à?¤?¶?à?¥???à?¤?µ?à?¤?¦?à?¥?‡?à?¤?µ?à?¤?¿?à?¤?•?à?¤?¦?à?¥???à?¤?µ?à?¤?¿?à?¤?œ?à?¤?ª?à?¤?™?à?¥???à?¤?•?à?¥?‡?à?¤?°?à?¥???à?¤?¤?à?¥???à?¤?¤?à?¤?°?
?à?¤???à?¤?µ?à?¥?‡?à?¤?¤?à?¤?¿?
?à?¤?¹?à?¥?‡?à?¤?®?à?¤?¾?à?¤?¦?à?¥???à?¤?°?à?¤?¿?à?¤?ƒ? ?à?¥?¤?

? ? ? ?à?¤?¨?à?¤?¾?à?¤?---?à?¤?°?à?¤?--?à?¤?£?à?¥???à?¤?¡?à?¥?‡?
?à?¤?š?-?

? ? ? ? ? ? ?à?¤?\ldots{}?à?¤?§?à?¥?‹?à?¤?®?à?¥???à?¤?--?à?¤?‚?
?à?¤?¤?à?¥??? ?à?¤?¤?à?¤?¤?à?¥???à?¤?ª?à?¤?¾?à?¤?¤?à?¥???à?¤?°?à?¤?‚?
?à?¤?µ?à?¤?¿?à?¤?œ?à?¤?¨?à?¥?‡?
?à?¤?¸?à?¥???à?¤?¥?à?¤?¾?à?¤?ª?à?¤?¯?à?¥?‡?à?¤?¤?à?¥???à?¤?¤?à?¤?¤?à?¤?ƒ?
?à?¥?¤?

?à?¤?¤?à?¤?¦?à?¤?¿?à?¤?¤?à?¤?¿? ?=?
?à?¤?¯?à?¤?¸?à?¥???à?¤?®?à?¤?¿?à?¤?¨?à?¥???à?¤?ª?à?¤?¿?à?¤?¤?à?¥?ƒ?à?¤?ª?à?¤?¾?à?¤?¤?à?¥???à?¤?°?à?¥?‡?
?à?¤?¸?à?¤?‚?à?¤?¸?à?¥???à?¤?°?à?¤?µ?à?¤?¾?à?¤?ƒ?
?à?¤?¸?à?¤?‚?à?¤?---?à?¥?ƒ?à?¤?¹?à?¥?€?à?¤?¤?à?¤?¾?à?¤?ƒ? ?à?¥?¤?
?à?¤?µ?à?¤?¿?à?¤?œ?à?¤?¨?à?¥?‡? ?=?
?à?¤?¶?à?¥???à?¤?°?à?¤?¾?à?¤?¦?à?¥???à?¤?§?à?¥?€?à?¤?¯?à?¤?¦?à?¥???à?¤?°?-?

?à?¤?µ?à?¥???à?¤?¯?à?¤?¾?à?¤?¨?à?¤?¯?à?¤?¨?à?¤?¾?à?¤?ª?à?¤?¨?à?¤?¯?à?¤?¨?à?¤?ª?à?¤?°?à?¤?¿?à?¤?µ?à?¥?‡?à?¤?·?à?¤?£?à?¤?¾?à?¤?¦?à?¤?¿?à?¤?•?à?¤?°?à?¥???à?¤?¤?à?¥?ƒ?à?¤?œ?à?¤?¨?
?à?¤?¸?à?¤?‚?à?¤?µ?à?¤?°?à?¤?£?à?¤?¦?à?¥?‹?à?¤?·?à?¤?°?à?¤?¹?à?¤?¿?à?¤?¤?à?¥?‡?
?à?¥?¤? ?à?¤?\ldots{}?à?¤?¤?à?¥???à?¤?°?
?à?¤?ª?à?¥???à?¤?°?à?¤?š?à?¥?‡?-?

?à?¤?¤?à?¤?¸?à?¤?¾? ?à?¤?µ?à?¤?¿?à?¤?¶?à?¥?‡?à?¤?·?
?à?¤?‰?à?¤?•?à?¥???à?¤?¤?à?¤?ƒ? ?à?¥?¤?

? ? ? ?
?à?¤?\ldots{}?à?¤?°?à?¥???à?¤?˜?à?¤?¸?à?¤?‚?à?¤?¸?à?¥???à?¤?°?à?¤?µ?à?¤?ª?à?¤?¾?à?¤?¤?à?¥???à?¤?°?à?¤?‚?
?à?¤?¤?à?¥???
?à?¤?¶?à?¥???à?¤?¨?à?¥???à?¤?§?à?¤?¨?à?¥???à?¤?¤?à?¤?¾?à?¤?®?à?¤?¿?à?¤?¤?à?¤?¿?
?à?¤?¸?à?¤?‚?à?¤?¸?à?¥???à?¤?ª?à?¥?ƒ?à?¤?¶?à?¥?‡?à?¤?¤?à?¥??? ?à?¥?¤?

?à?¤?ª?à?¤?¿?à?¤?¤?à?¥?ƒ?à?¤?­?à?¥???à?¤?¯?à?¤?ƒ?
?à?¤?¸?à?¥???à?¤?¥?à?¤?¾?à?¤?¨?à?¤?®?à?¤?¸?à?¥?€?à?¤?¤?à?¤?¿?
?à?¤?¨?à?¥???à?¤?¯?à?¥???à?¤?¬?à?¥???à?¤?œ?
?à?¤?•?à?¥???à?¤?°?à?¥???à?¤?¯?à?¤?¾?à?¤?¤?à?¥???
?à?¤?ª?à?¤?µ?à?¤?¿?à?¤?¤?à?¥???à?¤?°?à?¤?¾?à?¤?£?à?¤?¿?
?à?¤?†?à?¤?¦?à?¤?¾?à?¤?¯? ?à?¤?¤?à?¥?ˆ?à?¤?ƒ? ?à?¤?ª?-?

?à?¤?µ?à?¤?¿?à?¤?¤?à?¥???à?¤?°?à?¥?ˆ?à?¤?ƒ?
?à?¤?¸?à?¤?¹?à?¤?¾?à?¤?°?à?¥???à?¤?˜?à?¤?ª?à?¤?¾?à?¤?¤?à?¥???à?¤?°?à?¤?‚?
?à?¤?¨?à?¥???à?¤?¯?à?¥???à?¤?¬?à?¥???à?¤?œ?
?à?¤?•?à?¥???à?¤?°?à?¥???à?¤?¯?à?¤?¾?à?¤?¤?à?¥??? ?à?¥?¤?

? ? ? ? ?
?à?¤?¬?à?¥?ˆ?à?¤?œ?à?¤?µ?à?¤?¾?à?¤?ª?à?¤?---?à?¥?ƒ?à?¤?¹?à?¥???à?¤?¯?à?¥?‡?
?à?¤?¤?à?¥??? ?à?¤?\ldots{}?à?¤?°?à?¥???à?¤?˜?à?¤?‚?
?à?¤?¦?à?¤?¤?à?¥???à?¤?µ?à?¤?¾?
?à?¤?\ldots{}?à?¤?¨?à?¤?¨?à?¥???à?¤?¤?à?¤?°?à?¤?®?à?¥?‡?à?¤?µ?
?à?¤?¶?à?¥?‡?à?¤?·?à?¤?¸?à?¥???à?¤?¯?
?à?¤?•?à?¥???à?¤?¶?à?¥?‡?à?¤?·?à?¥???
?à?¤?¨?à?¤?¿?à?¤?¨?à?¤?¯?à?¤?¨?à?¤?‚?

?à?¤?‰?à?¤?•?à?¥???à?¤?¤?à?¤?®?à?¥??? ?à?¥?¤?
?à?¤?ª?à?¤?¿?à?¤?¤?à?¥?ƒ?à?¤?­?à?¥???à?¤?¯?à?¥?‹?à?¤?½?à?¤?•?à?¥???à?¤?·?à?¤?¯?à?¥???à?¤?¯?à?¤?®?à?¤?¸?à?¥???à?¤?¤?à?¤?¿?à?¤?¯?à?¤?¤?à?¤?¿?
?à?¤?¶?à?¥?‡?à?¤?·?à?¤?‚?
?à?¤?¦?à?¤?°?à?¥???à?¤?­?à?¥?‡?à?¤?·?à?¥???à?¤?µ?à?¤?µ?à?¤?¨?à?¥?‡?à?¤?œ?à?¤?¯?à?¤?¤?à?¥?€?à?¤?¤?à?¥???à?¤?¯?à?¥???à?¤?•?à?¥???à?¤?¤?à?¥???à?¤?µ?à?¤?¾?

?à?¤?ª?à?¤?¿?à?¤?¤?à?¤?¾?à?¤?®?à?¤?¹?à?¤?¾?à?¤?¦?à?¤?¿?à?¤?·?à?¥???
?à?¤?\ldots{}?à?¤?°?à?¥???à?¤?˜?à?¤?¦?à?¤?¾?à?¤?¨?à?¤?¸?à?¥???à?¤?¯?à?¥?‹?à?¤?•?à?¥???à?¤?¤?à?¤?¤?à?¥???à?¤?µ?à?¤?¾?à?¤?¤?à?¥???
?à?¥?¤? ?à?¤?¦?à?¤?°?à?¥???à?¤?­?à?¥?‡?à?¤?·?à?¥??? ?à?¤?‡?à?¤?¤?à?¤?¿?
?=? ?à?¤?¯?à?¥?‡?à?¤?·?à?¥???
?à?¤?¦?à?¤?°?à?¥???à?¤?­?à?¥?‡?à?¤?·?à?¥???
?à?¤?\ldots{}?à?¤?°?à?¥???à?¤?˜?

?à?¤?ª?à?¤?¾?à?¤?¤?à?¥???à?¤?°?à?¤?¾?à?¤?£?à?¤?¿?
?à?¤?¸?à?¥???à?¤?¥?à?¤?¾?à?¤?ª?à?¤?¿?à?¤?¤?à?¤?¾?à?¤?¨?à?¤?¿?
?à?¤?¤?à?¥?‡?à?¤?·?à?¥???à?¤?µ?à?¤?¿?à?¤?¤?à?¥???à?¤?¯?à?¤?°?à?¥???à?¤?¥?:?
?à?¥?¤? ?à?¤?¤?à?¤?¦?à?¤?¨?à?¤?¨?à?¥???à?¤?¤?à?¤?°?à?¤?‚? ?à?¤?š?
?à?¤?¤?à?¤?¾?à?¤?¨?à?¤?¿?
?à?¤?ª?à?¤?¾?à?¤?¤?à?¥???à?¤?°?à?¤?¾?à?¤?£?à?¤?¿?

?à?¤?¤?à?¤?¥?à?¥?ˆ?à?¤?µ?
?à?¤?¸?à?¥???à?¤?¥?à?¤?¾?à?¤?ª?à?¤?¯?à?¥?‡?à?¤?¤?à?¥???
?\textless{}?/?s?p?a?n?\textgreater{}?

?à?¥?¤?\textless{}?/?s?p?a?n?\textgreater{}?

?

? ? ? ? ?à?¤?¤?à?¤?¥?à?¤?¾? ?à?¤?š?
?à?¤?®?à?¥?ˆ?à?¤?¤?à?¥???à?¤?°?à?¤?¾?à?¤?¯?à?¤?£?à?¥?€?à?¤?¯?à?¥?‡?
?à?¤?¶?à?¤?¿?à?¤?·?à?¥???à?¤?Ÿ?à?¤?®?à?¥??? ?à?¥?¤?

? ? ? ? ?à?¤?µ?à?¤?°?à?¥???à?¤?¹?à?¤?¿?à?¤?·?à?¤?¿?
?à?¤?¨?à?¤?¿?à?¤?¨?à?¤?¯?à?¥?‡?à?¤?¦?à?¤?­?à?¥???à?¤?¯?à?¥???à?¤?•?à?¥???à?¤?·?à?¥???à?¤?¯?
?à?¤?ª?à?¥???à?¤?°?à?¤?¤?à?¥???à?¤?¯?à?¤?¾?à?¤?¸?à?¤?¾?à?¤?¦?à?¤?¯?à?¥?‡?à?¤?¦?à?¤?¿?à?¤?¤?à?¤?¿?
?à?¥?¤?

?à?¤?¯?à?¤?¤?à?¥???à?¤?°?
?à?¤?ª?à?¥?‚?à?¤?°?à?¥???à?¤?µ?à?¤?®?à?¤?°?à?¥???à?¤?˜?à?¤?¦?à?¤?¾?à?¤?¨?à?¤?¾?à?¤?°?à?¥???à?¤?¥?à?¤?®?à?¤?¾?à?¤?¸?à?¤?¾?à?¤?¦?à?¤?¿?à?¤?¤?à?¤?¾?à?¤?¨?à?¤?¿?
?à?¤?\ldots{}?à?¤?°?à?¥???à?¤?˜?à?¤?‚? ?à?¤?¦?à?¤?¤?à?¥???à?¤?µ?à?¤?¾?
?à?¤?¤?à?¤?¤?à?¥???à?¤?°?à?¥?ˆ?à?¤?µ?
?à?¤?ª?à?¥???à?¤?¨?à?¤?°?à?¤?¾?à?¤?¸?à?¤?¾?à?¤?¦?-?

?à?¤?¯?à?¥?‡?à?¤?¤?à?¥???,?
?à?¤?ª?à?¥???à?¤?°?à?¤?¤?à?¥???à?¤?¯?à?¤?¾?à?¤?¸?à?¤?¾?à?¤?¦?à?¤?¨?à?¤?¶?à?¤?¬?à?¥???à?¤?¦?à?¤?¾?à?¤?¤?à?¥???
?à?¥?¤?

?à?¤?\ldots{}?à?¤?¤?à?¥???à?¤?°?à?¤?¿?à?¤?£?à?¤?¾? ?à?¤?¤?à?¥???
?à?¤?¸?à?¤?‚?à?¤?¸?à?¥???à?¤?°?à?¤?µ?à?¤?¾?à?¤?£?à?¤?¾?à?¤?‚?
?à?¤?ª?à?¥???à?¤?°?à?¤?¥?à?¤?®?à?¥?‡?
?à?¤?ª?à?¤?¾?à?¤?¤?à?¥???à?¤?°?à?¥?‡?
?à?¤?¸?à?¤?‚?à?¤?---?à?¥???à?¤?°?à?¤?¹?à?¤?‚? ?à?¤?¸?à?¥???à?¤?¤? ?[?
?à?¤?•?à?¥?ƒ?]?à?¤?¤?à?¥???à?¤?µ?à?¤?¾?
?à?¤?¸?à?¥???à?¤?µ?à?¤?§?à?¤?¾?à?¤?µ?à?¤?¾?à?¤?š?.?

?à?¤?¨?à?¤?¾?à?¤?¤?à?¥???à?¤?ª?à?¥?‚?à?¤?°?à?¥???à?¤?µ?à?¤?•?à?¤?¾?à?¤?²?à?¤?‚?
?à?¤?¤?à?¥?‡?à?¤?·?à?¤?¾?à?¤?‚?
?à?¤?ª?à?¥???à?¤?°?à?¤?¤?à?¤?¿?à?¤?ª?à?¤?¤?à?¥???à?¤?¤?à?¤?¿?à?¤?°?à?¥???à?¤?•?à?¥???à?¤?¤?à?¤?¾?
?à?¥?¤?

? ? ? ? ? ? ?à?¤?\ldots{}?à?¤?ª?à?¤?¸?à?¤?µ?à?¥???à?¤?¯?à?¤?‚?
?à?¤?¤?à?¤?¤?à?¤?ƒ? ?à?¤?•?à?¥?ƒ?à?¤?¤?à?¥???à?¤?µ?à?¤?¾?
?à?¤?ª?à?¤?¿?à?¤?£?à?¥???à?¤?¡?à?¤?ª?à?¤?¾?à?¤?°?à?¥???à?¤?¶?à?¥???à?¤?µ?à?¥?‡?
?à?¤?¸?à?¤?®?à?¤?¾?à?¤?¹?à?¤?¿?à?¤?¤?à?¤?ƒ? ?à?¥?¤?

? ? ? ? ? ?
?à?¤?•?à?¥???à?¤?·?à?¤?¿?à?¤?ª?à?¥???à?¤?¤?à?¥???à?¤?µ?à?¤?¾?
?à?¤?¦?à?¤?°?à?¥???à?¤?­?à?¤?ª?à?¤?µ?à?¤?¿?à?¤?¤?à?¥???à?¤?°?à?¤?¾?à?¤?£?à?¤?¿?
?à?¤?®?à?¥?‹?à?¤?š?à?¤?¯?à?¥?‡?à?¤?¤?à?¥???à?¤?¸?à?¤?‚?à?¤?¸?à?¥???à?¤?°?à?¤?µ?à?¤?¾?à?¤?‚?à?¤?¸?à?¥???à?¤?¤?à?¤?¤?à?¤?ƒ?
?à?¥?¥?

? ?à?¤?‡?à?¤?¦?à?¤?‚? ?à?¤?š?
?à?¤?¸?à?¤?‚?à?¤?¸?à?¥???à?¤?°?à?¤?µ?à?¤?§?à?¤?¾?à?¤?°?à?¤?£?à?¤?‚?
?à?¤?ª?à?¤?¾?à?¤?¤?à?¥???à?¤?°?à?¤?¸?à?¥???à?¤?¯?à?¥?‹?à?¤?¨?à?¥???à?¤?¤?à?¤?¾?à?¤?¨?à?¤?¸?à?¥???à?¤?¥?à?¤?¾?à?¤?ª?à?¤?¨?à?¥?‡?
?à?¤???à?¤?µ?à?¥?‹?à?¤?ª?à?¤?ª?à?¤?¦?à?¥???à?¤?¯?à?¤?¤?à?¥?‡? ?à?¤?¨?
?à?¤?¨?à?¥???à?¤?¯?à?¥???à?¤?¬?à?¥???à?¤?œ?à?¥?€?-?

?à?¤?•?à?¤?°?à?¤?£?à?¥?‡? ?à?¥?¤?
?à?¤?\ldots{}?à?¤?¤?à?¥?‹?à?¤?½?à?¤?¨?à?¥?‡?à?¤?¨?à?¤?¾?à?¤?°?à?¥???à?¤?¥?à?¤?¾?à?¤?¦?à?¥???à?¤?¤?à?¥???à?¤?¤?à?¤?¾?à?¤?¨?à?¤?¸?à?¥???à?¤?¯?à?¥?ˆ?à?¤?µ?
?à?¤?¸?à?¤?‚?à?¤?¸?à?¥???à?¤?°?à?¤?µ?à?¤?ª?à?¤?¾?à?¤?¤?à?¥???à?¤?°?à?¤?¸?à?¥???à?¤?¯?
?à?¤?¸?à?¥???à?¤?¥?à?¤?¾?à?¤?ª?à?¤?¨?à?¤?®?à?¥???à?¤?•?à?¥???à?¤?¤?à?¤?‚?
?à?¤?­?à?¤?µ?à?¤?¤?à?¤?¿? ?à?¥?¤?

?à?¤?\ldots{}?à?¤?¸?à?¥???à?¤?®?à?¤?¿?à?¤?¨?à?¥???à?¤?ª?à?¤?•?à?¥???à?¤?·?à?¥?‡?
?à?¤?¸?à?¤?‚?à?¤?¸?à?¥???à?¤?°?à?¤?µ?à?¤?¾?à?¤?§?à?¤?¾?à?¤?°?à?¥?€?à?¤?•?à?¥?ƒ?à?¤?¤?à?¤?¸?à?¥???à?¤?¯?
?à?¤?ª?à?¤?¿?à?¤?¤?à?¥?ƒ?à?¤?ª?à?¤?¾?à?¤?¤?à?¥???à?¤?°?à?¤?¸?à?¥???à?¤?¯?
?à?¤?ª?à?¤?¿?à?¤?¤?à?¤?¾?à?¤?®?à?¤?¹?à?¤?ª?à?¤?¾?à?¤?¤?à?¥???à?¤?°?à?¥?‡?à?¤?£?
?à?¤?¨?à?¥???à?¤?¯?à?¥???à?¤?¬?à?¥???à?¤?œ?à?¥?‡?-?

?à?¤?¨?à?¥?‹?à?¤?¨?à?¥???à?¤?¤?à?¤?¾?à?¤?¨?à?¥?‡?à?¤?¨? ?à?¤?µ?à?¤?¾?
?à?¤?ª?à?¤?¿?à?¤?§?à?¤?¾?à?¤?¨?à?¤?‚?
?à?¤?•?à?¤?°?à?¥???à?¤?¤?à?¥???à?¤?¤?à?¤?µ?à?¥???à?¤?¯?à?¤?‚?
?à?¤?¤?à?¤?¦?à?¥???à?¤?ª?à?¤?°?à?¤?¿?
?à?¤?ª?à?¥???à?¤?°?à?¤?ª?à?¤?¿?à?¤?¤?à?¤?¾?à?¤?®?à?¤?¹?à?¤?ª?à?¤?¾?à?¤?¤?à?¥???à?¤?°?à?¤?®?à?¥???
?à?¥?¤? ?à?¤?¤?à?¤?¥?à?¤?¾? ?à?¤?š?

?à?¤?•?à?¤?¾?à?¤?¤?à?¥???à?¤?¯?à?¤?¾?à?¤?¯?à?¤?¨?à?¤?ƒ? ?à?¥?¤?

? ? ? ? ?à?¤?ª?à?¥?ˆ?à?¤?¤?à?¥?ƒ?à?¤?•?à?¤?‚?
?à?¤?ª?à?¥???à?¤?°?à?¤?¥?à?¤?®?à?¤?‚?
?à?¤?ª?à?¤?¾?à?¤?¤?à?¥???à?¤?°?à?¤?‚?
?à?¤?¤?à?¤?¸?à?¥???à?¤?®?à?¤?¿?à?¤?¨?à?¥???
?à?¤?ª?à?¥?ˆ?à?¤?¤?à?¤?¾?à?¤?®?à?¤?¹?à?¤?‚?
?à?¤?¨?à?¥???à?¤?¯?à?¤?¸?à?¥?‡?à?¤?¤?à?¥??? ?à?¥?¤?

? ? ? ? ? ?à?¤?ª?à?¥???à?¤?°?à?¤?ª?à?¥?ˆ?à?¤?¤?à?¤?¾?à?¤?®?à?¤?¹?à?¤?‚?
?à?¤?¤?à?¤?¤?à?¥?‹?à?¤?¨?à?¥???à?¤?¯?à?¤?¸?à?¥???à?¤?¯?
?à?¤?¨?à?¥?‹?à?¤?¦?à?¥???à?¤?§?à?¤?°?à?¥?‡?à?¤?¨?à?¥???à?¤?¨? ?à?¤?µ?
?à?¤?š?à?¤?¾?à?¤?²?à?¤?¯?à?¥?‡?à?¤?¤?à?¥??? ?à?¥?¥? ?à?¤?‡?à?¤?¤?à?¤?¿?
?à?¥?¤?

? ? ? ?à?¤?\ldots{}?à?¤?¤?à?¥???à?¤?°?
?à?¤?ª?à?¥???à?¤?¤?à?¥???à?¤?°?à?¤?•?à?¤?¾?à?¤?®?à?¤?¸?à?¥???à?¤?¯?
?à?¤?µ?à?¤?¿?à?¤?¶?à?¥?‡?à?¤?·? ?à?¤?‰?à?¤?•?à?¥???à?¤?¤?à?¥?‹?
?à?¤?µ?à?¥?ƒ?à?¤?¦?à?¥???à?¤?§?à?¤?¶?à?¤?¾?à?¤?¤?à?¤?¾?à?¤?¤?à?¤?ª?à?¥?‡?à?¤?¨?
?\textless{}?/?s?p?a?n?\textgreater{}?

?à?¥?¤?\textless{}?/?s?p?a?n?\textgreater{}?\textless{}?/?p?\textgreater{}?\textless{}?/?b?o?d?y?\textgreater{}?\textless{}?/?h?t?m?l?\textgreater{}?
?

? ?b?o?d?y?\{? ?w?i?d?t?h?:? ?2?1?c?m?;? ?h?e?i?g?h?t?:? ?2?9?.?7?c?m?;?
?m?a?r?g?i?n?:? ?3?0?m?m? ?4?5?m?m? ?3?0?m?m? ?4?5?m?m?;? ?\}?
?\textless{}?/?s?t?y?l?e?\textgreater{}?\textless{}?!?D?O?C?T?Y?P?E?
?H?T?M?L? ?P?U?B?L?I?C? ?"?-?/?/?W?3?C?/?/?D?T?D? ?H?T?M?L?
?4?.?0?/?/?E?N?"?
?"?h?t?t?p?:?/?/?w?w?w?.?w?3?.?o?r?g?/?T?R?/?R?E?C?-?h?t?m?l?4?0?/?s?t?r?i?c?t?.?d?t?d?"?\textgreater{}?
?

?

?

?

? ?p?,? ?l?i? ?\{? ?w?h?i?t?e?-?s?p?a?c?e?:? ?p?r?e?-?w?r?a?p?;? ?\}?
?\textless{}?/?s?t?y?l?e?\textgreater{}?\textless{}?/?h?e?a?d?\textgreater{}?

? ?

?

?à?¥?¨?à?¥?¨?à?¥?¦? ? ? ? ? ? ? ? ? ? ? ? ? ?
?à?¤?µ?à?¥?€?à?¤?°?à?¤?®?à?¤?¿?à?¤?¤?à?¥???à?¤?°?à?¥?‹?à?¤?¦?à?¤?¯?à?¤?¸?à?¥???à?¤?¯?
?à?¤?¶?à?¥???à?¤?°?à?¤?¾?à?¤?¦?à?¥???à?¤?§?à?¤?ª?à?¥???à?¤?°?à?¤?•?à?¤?¾?à?¤?¶?à?¥?‡?-?\textless{}?/?s?p?a?n?\textgreater{}?

?

? ? ? ? ? ?à?¤?ª?à?¥???à?¤?°?à?¤?¥?à?¤?®?à?¥?‡?
?à?¤?ª?à?¤?¿?à?¤?¤?à?¥?ƒ?à?¤?ª?à?¤?¾?à?¤?¤?à?¥???à?¤?°?à?¥?‡?
?à?¤?¤?à?¥???
?à?¤?¸?à?¤?°?à?¥???à?¤?µ?à?¤?¾?à?¤?¨?à?¥???à?¤?¸?à?¤?®?à?¥???à?¤?­?à?¥?ƒ?à?¤?¤?à?¥???à?¤?¯?
?à?¤?¸?à?¤?‚?à?¤?¸?à?¥???à?¤?°?à?¤?µ?à?¤?¾?à?¤?¨?à?¥??? ?à?¥?¤?

? ? ? ? ?à?¤?\ldots{}?à?¤?¨?à?¤?•?à?¥???à?¤?¤?à?¤?¿?
?à?¤?µ?à?¤?¦?à?¤?¨?à?¤?‚?
?à?¤?ª?à?¤?¶?à?¥???à?¤?š?à?¤?¾?à?¤?¤?à?¥???à?¤?ª?à?¥???à?¤?¤?à?¥???à?¤?°?à?¤?•?à?¤?¾?à?¤?®?à?¥?‹?
?à?¤?­?à?¤?µ?à?¥?‡?à?¤?¦?à?¥???à?¤?¯?à?¤?¦?à?¤?¿?
?\textless{}?/?s?p?a?n?\textgreater{}?

?à?¥?¥?\textless{}?/?s?p?a?n?\textgreater{}?

?

?à?¤?\ldots{}?à?¤?¤?à?¥???à?¤?°?
?à?¤?¸?à?¤?‚?à?¤?¸?à?¥???à?¤?°?à?¤?µ?à?¤?¾?à?¤?­?à?¤?¿?à?¤?®?à?¤?¨?à?¥???à?¤?¤?à?¥???à?¤?°?à?¤?£?à?¥?‡?
?à?¤?®?à?¥???à?¤?--?à?¤?¾?à?¤?ž?à?¥???à?¤?œ?à?¤?¨?à?¥?‡? ?à?¤?š?
?à?¤?®?à?¤?¨?à?¥???à?¤?¤?à?¥???à?¤?°?à?¤?ƒ?
?à?¤?•?à?¤?¾?à?¤?¤?à?¥???à?¤?¯?à?¤?¾?à?¤?¯?à?¤?¨?à?¥?‡?à?¤?¨?à?¥?‹?à?¤?•?à?¥???à?¤?¤?à?¤?ƒ?
?à?¥?¤?

? ? ? ? ?à?¤?\ldots{}?à?¤?°?à?¥???à?¤?˜?à?¥???à?¤?¯?à?¤?‚?
?à?¤?¸?à?¥???à?¤?µ?à?¤?§?à?¤?¯?à?¤?¾?
?à?¤?‡?à?¤?¤?à?¥???à?¤?¯?à?¥???à?¤?•?à?¥???à?¤?¤?à?¥???à?¤?µ?à?¤?¾?
?à?¤?¸?à?¤?‚?à?¤?¸?à?¥???à?¤?°?à?¤?µ?à?¤?¾?à?¤?¨?à?¤?­?à?¤?¿?à?¤?®?à?¤?¨?à?¥???à?¤?¤?à?¥???à?¤?°?à?¤?¯?à?¥?‡?à?¤?¤?à?¥???
?à?¥?¤?

? ? ? ? ?à?¤?¯?à?¥?‡? ?à?¤?¦?à?¥?‡?à?¤?µ?à?¤?¾? ?à?¤?‡?à?¤?¤?à?¤?¿?
?à?¤?®?à?¤?¨?à?¥???à?¤?¤?à?¥???à?¤?°?à?¥?‡?à?¤?£?
?à?¤?ª?à?¥???à?¤?¤?à?¥???à?¤?°?à?¤?•?à?¤?¾?à?¤?®?à?¥?‹?
?à?¤?®?à?¥???à?¤?--?à?¤?‚?
?à?¤?¸?à?¥???à?¤?ª?à?¥?ƒ?à?¤?¶?à?¥?‡?à?¤?¤?à?¥??? ?à?¥?¤?

?à?¤?¬?à?¥???à?¤?°?à?¤?¹?à?¥???à?¤?®?à?¤?ª?à?¥???à?¤?°?à?¤?¾?à?¤?£?à?¥?‡?
?à?¤?¤?à?¥???

? ? ? ?
?à?¤?¤?à?¥?‡?à?¤?·?à?¥???à?¤?µ?à?¤?°?à?¥???à?¤?˜?à?¤?ª?à?¤?¾?à?¤?¤?à?¥???à?¤?°?à?¤?¸?à?¥???à?¤?¥?à?¥?‡?à?¤?·?à?¥???
?à?¤?¸?à?¤?‚?à?¤?¸?à?¥???à?¤?°?à?¤?µ?à?¥?‡?à?¤?·?à?¥???
?à?¤?\ldots{}?à?¤?¨?à?¥???à?¤?¯?à?¥?‹?à?¤?¦?à?¤?•?à?¤?¾?à?¤?¸?à?¥?‡?à?¤?•?à?¥?‡?à?¤?¨?
?à?¤?¤?à?¤?¾?à?¤?¨?à?¥???à?¤?¯?à?¤?°?à?¥???à?¤?˜?à?¤?ª?à?¤?¾?à?¤?¤?à?¥???à?¤?°?à?¤?¾?à?¤?£?à?¤?¿?

?à?¤?¸?à?¤?®?à?¥???à?¤?ª?à?¥?‚?à?¤?°?à?¥???à?¤?£?
?à?¤?¤?à?¥?‡?à?¤?·?à?¥?‚?à?¤?¦?à?¤?•?à?¥?‡?à?¤?·?à?¥???
?à?¤?ª?à?¥???à?¤?¤?à?¥???à?¤?°?à?¤?•?à?¤?¾?à?¤?®?à?¥?‡?à?¤?¨?
?à?¤?¯?à?¤?œ?à?¤?®?à?¤?¾?à?¤?¨?à?¥?‡?à?¤?¨?
?à?¤?¸?à?¥???à?¤?µ?à?¤?®?à?¥???à?¤?--?à?¤?ª?à?¥???à?¤?°?à?¤?¤?à?¤?¿?à?¤?¬?à?¤?¿?à?¤?®?à?¥???à?¤?¬?à?¤?¾?à?¤?µ?à?¤?²?à?¥?‹?à?¤?•?à?¤?¨?à?¤?‚?

?à?¤?•?à?¤?¾?à?¤?°?à?¥???à?¤?¯?à?¤?®?à?¤?¿?à?¤?¤?à?¥???à?¤?¯?à?¥???à?¤?•?à?¥???à?¤?¤?à?¤?®?à?¥???
?à?¥?¤?

? ?à?¤?¤?à?¥?‡?à?¤?·?à?¥???
?à?¤?¸?à?¤?‚?à?¤?¸?à?¥???à?¤?°?à?¤?µ?à?¤?ª?à?¤?¾?à?¤?¤?à?¥???à?¤?°?à?¥?‡?à?¤?·?à?¥???
?à?¤?œ?à?¤?²?à?¤?ª?à?¥?‚?à?¤?°?à?¥???à?¤?£?à?¥?‡?à?¤?·?à?¥???
?à?¤?¤?à?¤?¦?à?¥???à?¤?°?à?¤?¸?à?¤?ƒ? ?à?¥?¤?

? ? ?à?¤?ª?à?¥???à?¤?¤?à?¥???à?¤?°?à?¤?•?à?¤?¾?à?¤?®?à?¥?‹?
?à?¤?®?à?¥???à?¤?--?
?à?¤?ª?à?¤?¶?à?¥???à?¤?¯?à?¥?‡?à?¤?¨?à?¥???à?¤?®?à?¤?¨?à?¥???à?¤?¤?à?¥???à?¤?°?à?¤?‚?
?à?¤?ª?à?¥?‚?à?¤?°?à?¥???à?¤?µ?à?¤?¾?à?¤?®?à?¥???à?¤?--?à?¥?‹?
?à?¤?œ?à?¤?ª?à?¤?¨?à?¥??? ?\textless{}?/?s?p?a?n?\textgreater{}?

?à?¥?¥?\textless{}?/?s?p?a?n?\textgreater{}?

?

?à?¤?¶?à?¥???à?¤?¨?à?¥???à?¤?§?à?¤?¨?à?¥???à?¤?¤?à?¤?¾?à?¤?²?à?¥???à?¤???à?¤?²?à?¥?‹?à?¤?•?à?¤?¾?à?¤?ƒ?
?à?¤?ª?à?¤?¿?à?¤?¤?à?¥?ƒ?à?¤?·?à?¤?¦?à?¤?¨?à?¤?¾?à?¤?ƒ?
?à?¤?ª?à?¤?¿?à?¤?¤?à?¥?ƒ?à?¤?·?à?¤?¦?à?¤?¨?à?¤?®?à?¤?¸?à?¥?€?à?¤?¤?à?¤?¿?
?à?¤?š? ?à?¥?¤?

?à?¤?œ?à?¤?²?à?¤?ª?à?¥?‚?à?¤?°?à?¥???à?¤?£?à?¤?¾?à?¤?ƒ? ?=?
?à?¤?œ?à?¤?²?à?¤?¾?à?¤?¨?à?¥???à?¤?¤?à?¤?°?à?¤?ª?à?¥?‚?à?¤?°?à?¥???à?¤?£?à?¤?¾?
?à?¤?‡?à?¤?¤?à?¤?¿?
?à?¤?¹?à?¥?‡?à?¤?®?à?¤?¾?à?¤?¦?à?¥???à?¤?°?à?¤?¿?à?¤?ƒ? ?à?¥?¤?

?à?¤?¨?à?¤?¾?à?¤?---?à?¤?°?à?¤?--?à?¤?£?à?¥???à?¤?¡?à?¥?‡?à?¤?¤?à?¥???

? ? ?
?à?¤?†?à?¤?¯?à?¥???à?¤?·?à?¥???à?¤?•?à?¤?¾?à?¤?®?à?¤?¸?à?¥???à?¤?¯?
?à?¤?¯?à?¤?œ?à?¤?®?à?¤?¾?à?¤?¨?à?¤?¸?à?¥???à?¤?¯?
?à?¤?¸?à?¤?‚?à?¤?¸?à?¥???à?¤?°?à?¤?µ?à?¥?‹?à?¤?¦?à?¤?•?à?¥?‡?à?¤?¨?
?à?¤?¨?à?¥?‡?à?¤?¤?à?¥???à?¤?°?à?¤?¾?à?¤?­?à?¥???à?¤?¯?à?¥???à?¤?•?à?¥???à?¤?·?à?¤?£?à?¤?®?à?¥???à?¤?•?à?¥???à?¤?¤?à?¤?®?à?¥???
?à?¥?¤?

? ? ? ? ? ?
?à?¤?ª?à?¤?¿?à?¤?¤?à?¥?ƒ?à?¤?ª?à?¤?¾?à?¤?¤?à?¥???à?¤?°?à?¥?‡?
?à?¤?¸?à?¤?®?à?¤?¾?à?¤?§?à?¤?¾?à?¤?¯?
?à?¤?\ldots{}?à?¤?°?à?¥???à?¤?˜?à?¤?ª?à?¤?¾?à?¤?¤?à?¥???à?¤?°?à?¤?¾?à?¤?£?à?¤?¿?
?à?¤?•?à?¥?ƒ?à?¤?¤?à?¥???à?¤?¸?à?¥???à?¤?¨?à?¤?¶?à?¤?ƒ? ?à?¥?¤?

? ? ? ? ? ?
?à?¤?†?à?¤?¯?à?¥???à?¤?·?à?¥???à?¤?•?à?¤?¾?à?¤?®?à?¤?¸?à?¥???à?¤?¤?à?¥???à?¤?¤?
?à?¤?¤?à?¥?‹?à?¤?¯?à?¤?‚?
?à?¤?²?à?¥?‹?à?¤?š?à?¤?¨?à?¤?¾?à?¤?­?à?¥???à?¤?¯?à?¤?¾?à?¤?‚?
?à?¤?ª?à?¤?°?à?¤?¿?à?¤?•?à?¥???à?¤?·?à?¤?¿?à?¤?ª?à?¥?‡?à?¤?¤?à?¥???
?à?¥?¥?

?à?¤?ª?à?¥???à?¤?°?à?¤?¥?à?¤?®?à?¥?‡?
?à?¤?ª?à?¤?¾?à?¤?¤?à?¥???à?¤?°?à?¥?‡?
?à?¤???à?¤?•?à?¥?€?à?¤?•?à?¥?ƒ?à?¤?¤?à?¤?‚?
?à?¤?¸?à?¤?‚?à?¤?¸?à?¥???à?¤?°?à?¤?µ?à?¥?‹?à?¤?¦?à?¤?•?à?¤?‚?
?à?¤?¹?à?¤?¸?à?¥???à?¤?¤?à?¥?‡?
?à?¤?---?à?¥?ƒ?à?¤?¹?à?¥?€?à?¤?¤?à?¥???à?¤?µ?à?¤?¾? ?â?€?œ?
?à?¤?†?à?¤?ª?à?¤?ƒ? ?à?¤?¶?à?¤?¿?à?¤?µ?à?¤?¾?à?¤?ƒ?

?à?¤?¶?à?¤?¿?à?¤?µ?à?¤?¤?à?¤?®?à?¤?¾?à?¤?ƒ?
?à?¤?¶?à?¤?¾?à?¤?¨?à?¥???à?¤?¤?à?¤?¾?à?¤?ƒ?
?à?¤?¶?à?¤?¾?à?¤?¨?à?¥???à?¤?¤?à?¤?¤?à?¤?®?à?¤?¾?à?¤?¸?à?¥???à?¤?¤?à?¤?¾?à?¤?¸?à?¥???à?¤?¤?à?¥?‡?
?à?¤?•?à?¥?ƒ?à?¤?£?à?¥???à?¤?µ?à?¤?¨?à?¥???à?¤?¤?à?¥???
?à?¤?­?à?¥?‡?à?¤?·?à?¤?œ?â?€??? ?à?¤?®?à?¤?¿?à?¤?¤?à?¤?¿?
?à?¤?®?à?¤?¨?à?¥???à?¤?¤?à?¥???à?¤?°?à?¥?‡?à?¤?£?

?à?¤?ª?à?¥???à?¤?°?à?¤?¾?à?¤?™?à?¥???à?¤?®?à?¥???à?¤?--?à?¥?‹?à?¤?ª?à?¤?µ?à?¤?¿?à?¤?·?à?¥???à?¤?Ÿ?à?¤?¸?à?¥???à?¤?¯?
?à?¤?¯?à?¤?œ?à?¤?®?à?¤?¾?à?¤?¨?à?¤?¸?à?¥???à?¤?¯?à?¤?¾?à?¤?¨?à?¥???à?¤?¯?à?¥?‡?à?¤?¨?à?¤?¾?à?¤?­?à?¤?¿?à?¤?·?à?¥?‡?à?¤?•?à?¤?ƒ?
?à?¤?•?à?¥???à?¤?°?à?¤?¿?à?¤?¯?à?¤?¤? ?à?¤?‡?à?¤?¤?à?¤?¿?
?à?¤?µ?à?¤?¾?à?¤?œ?à?¤?¸?à?¤?¨?à?¥?‡?-?

?à?¤?¯?à?¤?¿?à?¤?¨?à?¤?¾?à?¤?®?à?¤?¾?à?¤?š?à?¤?¾?à?¤?°?
?à?¤?‡?à?¤?¤?à?¤?¿?
?à?¤?¹?à?¥?‡?à?¤?®?à?¤?¾?à?¤?¦?à?¥???à?¤?°?à?¤?¿?à?¤?ƒ? ?à?¥?¤?
?à?¤???à?¤?¤?à?¤?¾?à?¤?¨?à?¤?¿? ?à?¤?š?
?à?¤?®?à?¥???à?¤?--?à?¤?¾?à?¤?ž?à?¥???à?¤?œ?à?¤?¨?à?¤?¾?à?¤?¦?à?¥?€?à?¤?¨?à?¤?¿?
?à?¤?•?à?¤?¾?à?¤?®?à?¥???à?¤?¯?à?¤?¾?à?¤?¨?à?¤?¿? ?à?¤?•?.?

?à?¤?°?à?¥???à?¤?®?à?¤?¾?à?¤?£?à?¤?¿?
?à?¤?ª?à?¥???à?¤?°?à?¥???à?¤?·?à?¤?¾?à?¤?°?à?¥???à?¤?¥?à?¤?¾?à?¤?¨?à?¤?¿?
?à?¤?¶?à?¥???à?¤?°?à?¤?¾?à?¤?¦?à?¥???à?¤?§?à?¤?¾?à?¤?¨?à?¤?™?à?¥???à?¤?---?à?¤?­?à?¥?‚?à?¤?¤?à?¤?¾?à?¤?¨?à?¤?¿?
?à?¤?ª?à?¥???à?¤?°?à?¤?¾?à?¤?™?à?¥???à?¤?®?à?¥???à?¤?--?à?¥?‡?à?¤?¨?
?à?¤?¯?à?¤?œ?à?¥???à?¤?ž?à?¥?‹?à?¤?ª?à?¤?µ?à?¥?€?à?¤?¤?à?¤?¿?à?¤?¨?à?¤?¾?
?à?¤?¯?à?¤?œ?à?¤?®?à?¤?¾?

?à?¤?¨?à?¥?‡?à?¤?¨? ?à?¤?•?à?¤?¾?à?¤?°?à?¥???à?¤?¯?à?¤?¾?à?¤?£?à?¤?¿?
?à?¤???à?¤?¤?à?¤?¨?à?¥???à?¤?®?à?¥???à?¤?--?à?¤?®?à?¤?¾?à?¤?°?à?¥???à?¤?œ?à?¤?¨?à?¤?¾?à?¤?¦?à?¥?‡?à?¤?ƒ?
?à?¤?•?à?¤?¾?à?¤?®?à?¥???à?¤?¯?à?¤?¤?à?¥???à?¤?µ?à?¤?¾?à?¤?¨?à?¥???à?¤?¨?à?¤?¿?à?¤?¤?à?¥???à?¤?¯?à?¤?µ?à?¤?¤?à?¥???à?¤?•?à?¥?ƒ?à?¤?¤?à?¤?¾?à?¤?°?à?¥???à?¤?˜?à?¤?ª?à?¤?¾?à?¤?¤?à?¥???à?¤?°?à?¤?¸?à?¤?‚?à?¤?¸?à?¥???à?¤?¤?à?¥???à?¤?°?à?¤?µ?.?

?à?¤?ª?à?¥???à?¤?°?à?¤?¤?à?¤?¿?à?¤?ª?à?¤?¤?à?¥???à?¤?¤?à?¥?‡?à?¤?°?à?¥???à?¤?¬?à?¤?¾?à?¤?§?à?¤?ƒ?
?à?¥?¤?

? ? ?à?¤?\ldots{}?à?¤?¤?à?¥???à?¤?°?à?¤?¿?à?¤?£?à?¤?¾? ?à?¤?¤?à?¥???
?à?¤?---?à?¤?¨?à?¥???à?¤?§?à?¤?ª?à?¥???à?¤?·?à?¥???à?¤?ª?à?¤?¾?à?¤?¦?à?¤?¿?à?¤?­?à?¤?¿?à?¤?°?à?¤?­?à?¥???à?¤?¯?à?¤?°?à?¥???à?¤?š?à?¤?¿?à?¤?¤?à?¤?ª?à?¥???à?¤?°?à?¤?¦?à?¥?‡?à?¤?¶?à?¥?‡?
?à?¤?ª?à?¤?¾?à?¤?¤?à?¥???à?¤?°?à?¤?¾?à?¤?£?à?¥???à?¤?¯?à?¤?ª?à?¤?¾?à?¤?¸?à?¥???à?¤?¯?
?à?¤?¨?à?¥???à?¤?¯?à?¥???à?¤?¬?à?¥???à?¤?œ?à?¥?€?-?

?à?¤?•?à?¤?°?à?¤?£?à?¤?‚?,?
?à?¤?¨?à?¥???à?¤?¯?à?¥???à?¤?¬?à?¥???à?¤?œ?à?¥?€?à?¤?•?à?¥?ƒ?à?¤?¤?à?¤?¸?à?¥???à?¤?¯?à?¤?¾?à?¤?°?à?¥???à?¤?š?à?¤?¿?à?¤?¤?à?¤?¸?à?¥???à?¤?¯?
?à?¤?®?à?¤?¨?à?¥???à?¤?¤?à?¥???à?¤?°?à?¥?‡?à?¤?£?à?¥?ˆ?à?¤?µ?à?¤?¾?à?¤?ª?à?¤?¿?à?¤?§?à?¤?¾?à?¤?¨?à?¤?®?à?¥???
?à?¤?\ldots{}?à?¤?ª?à?¤?¿?à?¤?§?à?¤?¾?à?¤?¯?à?¤?•?à?¤?ª?à?¤?¾?à?¤?¤?à?¥???à?¤?°?-?

?à?¤?¸?à?¥???à?¤?µ?à?¤?¾?à?¤?ª?à?¥???à?¤?¯?à?¤?°?à?¥???à?¤?š?à?¤?¨?à?¤?®?à?¥???à?¤?•?à?¥???à?¤?¤?à?¤?®?à?¥???
?à?¥?¤?

? ? ? ?
?à?¤?---?à?¤?¨?à?¥???à?¤?§?à?¤?¾?à?¤?¦?à?¤?¿?à?¤?­?à?¤?¿?à?¤?¸?à?¥???à?¤?¤?à?¤?¦?à?¤?­?à?¥???à?¤?¯?à?¤?°?à?¥???à?¤?š?à?¥???à?¤?¯?
?à?¤?¤?à?¥?ƒ?à?¤?¤?à?¥?€?à?¤?¯?à?¥?‡?à?¤?¨?à?¤?¾?à?¤?ª?à?¤?¿?à?¤?§?à?¤?¾?à?¤?ª?à?¤?¯?à?¥?‡?à?¤?¤?à?¥???
?à?¥?¤? ?à?¤?‡?à?¤?¤?à?¥???à?¤?¯?à?¥?‡?à?¤?µ?
?à?¤?œ?à?¥???à?¤?ž?à?¥?‡?à?¤?¯?à?¤?®?à?¥??? ?à?¥?¤?

?à?¤?¬?à?¥???à?¤?°?à?¤?¹?à?¥???à?¤?®?à?¤?ª?à?¥???à?¤?°?à?¤?¾?à?¤?£?à?¥?‡?
?à?¤?¤?à?¥??? ?à?¥?¤?

? ? ? ? ? ?
?à?¤?¤?à?¤?¤?à?¤?¸?à?¥???à?¤?¤?à?¥?‡?à?¤?·?à?¥???à?¤?µ?à?¤?°?à?¥???à?¤?§?à?¤?ª?à?¤?¾?à?¤?¤?à?¥???à?¤?°?à?¥?‡?à?¤?·?à?¥???
?à?¤?¸?à?¤?¾?à?¤?ª?à?¤?¿?à?¤?§?à?¤?¾?à?¤?¨?à?¥?‡?à?¤?·?à?¥???
?à?¤?µ?à?¥?ˆ? ?à?¤?ª?à?¤?¿?à?¤?¤?à?¥?„?à?¤?¨?à?¥??? ?à?¥?¤?

? ? ? ? ? ? ?à?¤?ª?à?¥?‚?à?¤?œ?à?¤?¯?à?¥?‡?
?à?¤?¤?à?¥???à?¤?ª?à?¤?¿?à?¤?¤?à?¥?ƒ?à?¤?ª?à?¥?‚?à?¤?°?à?¥???à?¤?¯?à?¥?‡?
?à?¤?¤?à?¥???
?à?¤?ª?à?¤?¾?à?¤?§?à?¤?¾?à?¤?°?à?¥???à?¤?---?à?¥???à?¤?¯?à?¤?•?à?¥???à?¤?¸?à?¥???à?¤?®?à?¤?¾?à?¤?¦?à?¤?¿?à?¤?­?à?¤?¿?à?¤?ƒ?
?à?¥?¥? ?à?¤?‡?à?¤?¤?à?¤?¿? ?à?¥?¤?

? ? ? ? ? ?
?à?¤???à?¤?¤?à?¤?¨?à?¥???à?¤?¨?à?¥???à?¤?¯?à?¥???à?¤?¬?à?¥???à?¤?œ?à?¥?€?à?¤?•?à?¥?ƒ?à?¤?¤?à?¤?‚?
?à?¤?ª?à?¥???à?¤?°?à?¤?¾?à?¤?---?à?¥???à?¤?¤?à?¥???à?¤?¤?à?¤?¾?à?¤?¨?à?¤?•?à?¤?°?à?¤?£?à?¤?¾?à?¤?¨?à?¥???à?¤?¨?
?à?¤?š?à?¤?¾?à?¤?²?à?¤?¨?à?¥?€?à?¤?¯?à?¤?®?à?¥??? ?à?¥?¤?

?\textless{}?/?s?p?a?n?\textgreater{}?\textless{}?/?p?\textgreater{}?\textless{}?/?b?o?d?y?\textgreater{}?\textless{}?/?h?t?m?l?\textgreater{}?

{ }{ गन्धादिदानप्रकारः । २२१}{\\
अत्राहापस्तम्बः।\\
नोद्धरेत्प्रथमं पात्रं पितॄणामर्धपातितम् ।\\
आवृतास्तत्र तिष्ठन्ति यावद्विप्रविसर्जनम् ॥ इति ।\\
अत्र यावद्विप्रविसर्जनमिति स्वगृह्यस्त्रोक्तपात्रोत्तान करणकालो-\\
पलक्षणार्थम् । उद्धरणे दोष आश्वलायनगृहये ।\\
उद्धरेद्यदि तत्पात्रं ब्राह्मणो ज्ञानदुर्बलः ।\\
अभोज्यं तद्भवेच्छ्राद्धं क्रुद्धे पितृगणे गते ।\\
उशनसापि उद्घाटने दोष उक्त. ।\\
उत्तानं विवृतं वापि पितृपात्रं यदा भवेत् ॥\\
अभोज्यं तद्भवेदनं क्रुद्धैः पितृगणैर्गतैः ।\\
विवृत्तम् = उद्घाटितम् ।\\
तथा ।\\
पात्रं दृष्ट्वा व्रजन्त्याशु पितरः प्रशयन्ति च ।\\
उद्धृतमिति सम्बन्धः ।\\
अथ गन्धादिदानम् ।\\
वैजवापगृह्ये,\\
तस्योपरि कुशान्दत्वा प्रदद्याद्देवपूर्वकम् ।\\
गन्धपुष्पाणि धूपं च दीपं वस्त्रोपवीतके ॥\\
तस्य = न्युब्जीकृतस्य पितृपात्रस्योपरि कुशान्कृत्वा गन्धपुष्पादी-\\
नि दत्वा अनन्तरं वैश्वदेविकब्राह्मणकरे गन्धादि दद्यात् ।\\
विष्णुनात्वलङ्करणमप्युक्तम् }{।}{\\
निवेद्य चानुलेपनवस्त्रपुष्पालङ्कारधूपैः शक्त्या विप्रान् सम•\\
भ्यर्च्येति ।\\
विष्णुधर्मोत्तरे -\\
निवेद्य विप्रेषु ततः पाद्यार्च्यौ प्रयतः क्रमात् ।\\
गन्धैः पुष्पैश्च धूपैश्च वस्त्रैश्वाऽऽप्यविभूषणैः ॥\\
अर्चयेद्ब्राह्मणान् शक्त्या श्रद्दधानः समाहितः ।\\
आदौ समर्चयेद्विप्रान्वैश्वदेवनिवेशितान् ॥\\
निवेशिताञ्च पित्रर्थे ततः पश्चात् समर्चयेत् ॥ इति ।\\
कौशिकसूत्रे तु अञ्जनादर्शयोरपि प्रदानमुक्तम् ।\\
गन्धमाल्यधूपाजनादर्शप्रदीपस्याहरणमिति ।

?

? ?b?o?d?y?\{? ?w?i?d?t?h?:? ?2?1?c?m?;? ?h?e?i?g?h?t?:? ?2?9?.?7?c?m?;?
?m?a?r?g?i?n?:? ?3?0?m?m? ?4?5?m?m? ?3?0?m?m? ?4?5?m?m?;? ?\}?
?\textless{}?/?s?t?y?l?e?\textgreater{}?\textless{}?!?D?O?C?T?Y?P?E?
?H?T?M?L? ?P?U?B?L?I?C? ?"?-?/?/?W?3?C?/?/?D?T?D? ?H?T?M?L?
?4?.?0?/?/?E?N?"?
?"?h?t?t?p?:?/?/?w?w?w?.?w?3?.?o?r?g?/?T?R?/?R?E?C?-?h?t?m?l?4?0?/?s?t?r?i?c?t?.?d?t?d?"?\textgreater{}?
?

?

?

?

? ?p?,? ?l?i? ?\{? ?w?h?i?t?e?-?s?p?a?c?e?:? ?p?r?e?-?w?r?a?p?;? ?\}?
?\textless{}?/?s?t?y?l?e?\textgreater{}?\textless{}?/?h?e?a?d?\textgreater{}?

? ?

?

?à?¥?¨?à?¥?¨?à?¥?¨? ? ? ? ? ? ? ?
?à?¤?µ?à?¥?€?à?¤?°?à?¤?®?à?¤?¿?à?¤?¤?à?¥???à?¤?°?à?¥?‹?à?¤?¦?à?¤?¯?à?¤?¸?à?¥???à?¤?¯?
?à?¤?¶?à?¥???à?¤?°?à?¤?¾?à?¤?¦?à?¥???à?¤?§?à?¤?ª?à?¥???à?¤?°?à?¤?•?à?¤?¾?à?¤?¶?à?¥?‡?-?\textless{}?/?s?p?a?n?\textgreater{}?

?

?à?¤?\ldots{}?à?¤?¤?à?¥???à?¤?°?à?¤?¾?à?¤?¹?à?¤?°?à?¤?£?à?¤?¸?à?¥???à?¤?¯?
?à?¤?¦?à?¥?ƒ?à?¤?·?à?¥???à?¤?Ÿ?à?¤?¾?à?¤?°?à?¥???à?¤?¥?à?¤?¤?à?¥???à?¤?µ?à?¤?¾?à?¤?¦?à?¥???
?à?¤?¬?à?¥???à?¤?°?à?¤?¾?à?¤?¹?à?¥???à?¤?®?à?¤?£?à?¥?‡?à?¤?­?à?¥???à?¤?¯?à?¤?¸?à?¥???à?¤?¤?à?¤?¦?à?¥???à?¤?µ?à?¥?‡?à?¤?¯?à?¤?®?à?¤?¿?à?¤?¤?à?¥???à?¤?¯?à?¥???à?¤?•?à?¥???à?¤?¤?à?¤?‚?
?à?¤?­?à?¤?µ?à?¤?¤?à?¤?¿? ?à?¥?¤?

?à?¤???à?¤?¤?à?¤?¦?à?¥???à?¤?---?à?¤?¨?à?¥???à?¤?§?à?¤?¾?à?¤?¦?à?¤?¿?à?¤?¦?à?¤?¾?à?¤?¨?à?¤?‚?
?à?¤?š?
?à?¤?¦?à?¥?‡?à?¤?µ?à?¤?¤?à?¥?‹?à?¤?¦?à?¥???à?¤?¦?à?¥?‡?à?¤?¶?à?¥?‡?à?¤?¨?à?¥?ˆ?à?¤?µ?
?à?¤?¤?à?¤?¦?à?¤?§?à?¤?¿?à?¤?·?à?¥???à?¤?~?à?¤?¾?à?¤?¨?à?¤?­?à?¥?‚?à?¤?¤?à?¤?¦?à?¥???à?¤?µ?à?¤?¿?à?¤?œ?à?¤?•?à?¤?°?à?¥?‡?
?à?¤?•?à?¤?°?à?¥???à?¤?¤?à?¥???à?¤?¤?à?¤?µ?à?¥???à?¤?¯?à?¤?®?à?¥???
?à?¥?¤?

?à?¤?¨? ?à?¤?¤?à?¥???
?à?¤?¦?à?¥???à?¤?µ?à?¤?¿?à?¤?œ?à?¤?¸?à?¤?®?à?¥???à?¤?ª?à?¥???à?¤?°?à?¤?¦?à?¤?¾?à?¤?¨?à?¤?•?à?¤?®?à?¤?¿?à?¤?¤?à?¥???à?¤?¯?à?¥???à?¤?•?à?¥???à?¤?¤?à?¤?‚?
?à?¤?¸?à?¤?®?à?¥???à?¤?ª?à?¥???à?¤?°?à?¤?¦?à?¤?¾?à?¤?¨?à?¤?¨?à?¤?¿?à?¤?°?à?¥???à?¤?£?à?¤?¯?à?¥?‡?
?à?¥?¤?

? ? ?
?à?¤?¯?à?¤?¾?à?¤?œ?à?¥???à?¤?ž?à?¤?µ?à?¤?²?à?¥???à?¤?•?à?¥???à?¤?¯?à?¥?‡?à?¤?¨?
?à?¤?¤?à?¥???
?à?¤?---?à?¤?¨?à?¥???à?¤?§?à?¤?¾?à?¤?¦?à?¤?¿?à?¤?¦?à?¤?¾?à?¤?¨?à?¤?¸?à?¥???à?¤?¯?à?¥?‹?à?¤?¦?à?¤?•?à?¤?ª?à?¥?‚?à?¤?°?à?¥???à?¤?µ?à?¤?•?à?¤?¤?à?¥???à?¤?µ?à?¤?®?à?¥???à?¤?•?à?¥???à?¤?¤?à?¤?®?à?¥???
?à?¥?¤?

? ? ? ? ? ? ?à?¤?¦?à?¤?¤?à?¥???à?¤?µ?à?¥?‹?à?¤?¦?à?¤?•?à?¤?‚?
?à?¤?---?à?¤?¨?à?¥???à?¤?§?à?¤?®?à?¤?¾?à?¤?²?à?¥???à?¤?¯?à?¤?‚?
?à?¤?§?à?¥?‚?à?¤?ª?à?¤?¦?à?¤?¾?à?¤?¨?à?¤?‚?
?à?¤?¸?à?¤?¦?à?¥?€?à?¤?ª?à?¤?•?à?¤?®?à?¥??? ?à?¥?¤?

?à?¤?---?à?¥?Œ?à?¤?¤?à?¤?®?à?¥?‡?à?¤?¨? ?à?¤?¤?à?¥???
?à?¤?µ?à?¤?¿?à?¤?§?à?¥???à?¤?¯?à?¥???à?¤?ª?à?¤?¦?à?¤?¿?à?¤?·?à?¥???à?¤?Ÿ?à?¤?¾?à?¤?¨?à?¤?¾?à?¤?‚?
?à?¤?¸?à?¤?°?à?¥???à?¤?µ?à?¤?¦?à?¤?¾?à?¤?¨?à?¤?¾?à?¤?¨?à?¤?¾?à?¤?®?à?¥???à?¤?¦?à?¤?•?à?¤?ª?à?¥?‚?à?¤?°?à?¥???à?¤?µ?à?¤?•?à?¤?¤?à?¥???à?¤?µ?à?¤?®?à?¥???à?¤?•?à?¥???à?¤?¤?à?¤?®?à?¥???

? ? ?
?à?¤?\ldots{}?à?¤?¨?à?¥???à?¤?¨?à?¤?­?à?¤?¿?à?¤?•?à?¥???à?¤?·?à?¤?¾?à?¤?¦?à?¤?¾?à?¤?¨?à?¤?®?à?¤?ª?à?¥???à?¤?ª?à?¥?‚?à?¤?°?à?¥???à?¤?µ?à?¤?®?à?¥???
?à?¥?¤? ?à?¤?¦?à?¤?¦?à?¤?¾?à?¤?¤?à?¤?¿?à?¤?·?à?¥???
?à?¤?š?à?¥?ˆ?à?¤?µ?à?¤?‚?à?¤?§?à?¤?°?à?¥???à?¤?®?à?¥???à?¤?¯?à?¥?‡?à?¤?·?à?¥???à?¤?µ?à?¤?¿?à?¤?¤?à?¤?¿?
?à?¥?¤? ?à?¤?¤?à?¤?¥?à?¤?¾? ?à?¤?¸?à?¤?°?à?¥???à?¤?µ?à?¤?¾?

?à?¤?£?à?¥???à?¤?¯?à?¥???à?¤?¦?à?¤?•?à?¤?ª?à?¥?‚?à?¤?°?à?¥???à?¤?µ?à?¤?¾?à?¤?£?à?¤?¿?
?à?¤?¦?à?¤?¾?à?¤?¨?à?¤?¾?à?¤?¨?à?¤?¿?
?à?¤?µ?à?¤?¿?à?¤?¹?à?¤?¾?à?¤?°?à?¤?µ?à?¤?°?à?¥???à?¤?œ?à?¤?®?à?¤?¿?à?¤?¤?à?¤?¿?
?à?¥?¤? ?à?¤?\ldots{}?à?¤?¤?à?¥???à?¤?°?
?à?¤?µ?à?¤?¿?à?¤?¶?à?¥?‡?à?¤?·?à?¥?‹? ?-?

?à?¤?¬?à?¥???à?¤?°?à?¤?¹?à?¥???à?¤?®?à?¤?ª?à?¥???à?¤?°?à?¤?¾?à?¤?£?à?¥?‡?-?-?-?

? ? ? ? ?
?à?¤?¶?à?¥???à?¤?µ?à?¥?‡?à?¤?¤?à?¤?š?à?¤?¨?à?¥???à?¤?¦?à?¤?¨?à?¤?•?à?¤?°?à?¥???à?¤?ª?à?¥???à?¤?ª?à?¥?‚?à?¤?°?à?¤?•?à?¥???à?¤?™?à?¥???à?¤?•?à?¥???à?¤?®?à?¤?¾?à?¤?¨?à?¤?¿?
?à?¤?¶?à?¥???à?¤?­?à?¤?¾?à?¤?¨?à?¤?¿? ?à?¤?¤?à?¥??? ?à?¥?¤?

? ? ? ? ?
?à?¤?µ?à?¤?¿?à?¤?²?à?¥?‡?à?¤?ª?à?¤?¨?à?¤?¾?à?¤?°?à?¥???à?¤?¥?à?¤?‚?
?à?¤?¦?à?¤?¦?à?¥???à?¤?¯?à?¤?¾?à?¤?¤?à?¥???à?¤?¤?à?¥???
?à?¤?¯?à?¤?š?à?¥???à?¤?š?à?¤?¾?à?¤?¨?à?¥???à?¤?¯?à?¤?¤?à?¥???à?¤?ª?à?¤?¿?à?¤?¤?à?¥?ƒ?à?¤?¬?à?¤?²?à?¥???à?¤?²?à?¤?­?à?¤?®?à?¥???
?à?¥?¤? ?à?¤?‡?à?¤?¤?à?¤?¿? ?à?¥?¤?

?à?¤?\ldots{}?à?¤?¤?à?¥???à?¤?°?
?à?¤?®?à?¤?¨?à?¥???à?¤?¤?à?¥???à?¤?°?à?¤?‚?
?à?¤?µ?à?¤?¿?à?¤?§?à?¤?¿?à?¤?µ?à?¤?¿?à?¤?¶?à?¥?‡?à?¤?·?à?¤?‚?
?à?¤?š?à?¤?¾?à?¤?¹? ?-?

?à?¤?µ?à?¥???à?¤?¯?à?¤?¾?à?¤?¸?à?¤?ƒ?,?

? ? ? ? ?
?à?¤?µ?à?¤?¿?à?¤?ª?à?¤?µ?à?¤?¿?à?¤?¤?à?¥???à?¤?°?à?¤?•?à?¤?°?à?¥?‹?
?à?¤?---?à?¤?¨?à?¥???à?¤?§?à?¥?ˆ?à?¤?°?à?¥???à?¤?---?à?¤?¨?à?¥???à?¤?§?à?¤?¦?à?¥???à?¤?µ?à?¤?¾?à?¤?°?à?¥?‡?à?¤?¤?à?¤?¿?
?à?¤?ª?à?¥?‚?à?¤?œ?à?¤?¯?à?¥?‡?à?¤?¤?à?¥??? ?à?¥?¤?

? ? ? ?
?à?¤?---?à?¤?¨?à?¥???à?¤?§?à?¤?¦?à?¥???à?¤?µ?à?¤?¾?à?¤?°?à?¥?‡?à?¤?¤?à?¤?¿?
?à?¤?®?à?¤?¨?à?¥???à?¤?¤?à?¥???à?¤?°?à?¥?‡?à?¤?£?
?à?¤?µ?à?¤?¿?à?¤?ª?à?¥???à?¤?°?à?¥?‡?
?à?¤?---?à?¤?¨?à?¥???à?¤?§?à?¤?¾?à?¤?¨?à?¥???à?¤?ª?à?¥???à?¤?°?à?¤?¦?à?¤?¾?à?¤?ª?à?¤?¯?à?¥?‡?à?¤?¤?à?¥???
?à?¥?¤?

?à?¤?°?à?¥???à?¤?---?à?¤?¨?à?¥???à?¤?§?à?¥?Œ?à?¤?°?à?¥???à?¤?µ?à?¤?¿?à?¤?---?à?¤?¤?à?¤?ª?à?¤?µ?à?¤?¿?à?¤?¤?à?¥???à?¤?°?à?¤?•?à?¤?°?à?¤?ƒ?
?à?¤?¶?à?¥???à?¤?°?à?¤?¾?à?¤?¦?à?¥???à?¤?§?à?¤?•?à?¤?°?à?¥???à?¤?¤?à?¤?¾?
?à?¤?µ?à?¤?¿?à?¤?ª?à?¥???à?¤?°?à?¤?¾?à?¤?¨?à?¥???
?à?¤?ª?à?¥?‚?à?¤?œ?à?¤?¯?à?¥?‡?à?¤?¤?à?¥???,?
?à?¤?²?à?¤?²?à?¤?¾?à?¤?Ÿ?à?¤?---?à?¤?²?-?

?à?¤?µ?à?¤?•?à?¥???à?¤?·?à?¤?ƒ?
?à?¤?•?à?¥???à?¤?•?à?¥???à?¤?·?à?¤?¿?à?¤?•?à?¤?•?à?¥???à?¤?·?à?¤?¾?à?¤?¦?à?¥???à?¤?¯?à?¤?™?à?¥???à?¤?---?à?¥?‡?à?¤?·?à?¥???
?à?¤?µ?à?¤?¿?à?¤?²?à?¤?¿?à?¤?®?à?¥???à?¤?ª?à?¥?‡?à?¤?¦?à?¤?¿?à?¤?¤?à?¥???à?¤?¯?à?¤?°?à?¥???à?¤?¥?à?¤?ƒ?
?à?¥?¤? ?à?¤?\ldots{}?à?¤?¤?à?¥???à?¤?°?
?à?¤?µ?à?¤?¿?à?¤?ª?à?¤?µ?à?¤?¿?à?¤?¤?à?¥???à?¤?°?à?¤?•?à?¤?°?à?¤?µ?à?¤?¿?à?¤?§?à?¤?¾?à?¤?¨?à?¤?‚?

?à?¤?²?à?¥?‡?à?¤?ª?à?¤?¨?à?¤?‚?
?à?¤?•?à?¥???à?¤?°?à?¥???à?¤?µ?à?¤?¤?à?¤?ƒ?
?à?¤?¶?à?¥???à?¤?°?à?¤?¾?à?¤?¦?à?¥???à?¤?§?à?¤?•?à?¤?°?à?¥???à?¤?¤?à?¥???à?¤?¤?à?¥???à?¤?°?à?¥?‡?à?¤?µ?
?à?¤?¨? ?à?¤?ª?à?¥???à?¤?¨?à?¤?ƒ?
?à?¤?•?à?¤?°?à?¤?---?à?¥?ƒ?à?¤?¹?à?¥?€?à?¤?¤?à?¤?---?à?¤?¨?à?¥???à?¤?§?à?¥?ˆ?à?¤?ƒ?
?à?¤?¸?à?¥???à?¤?µ?à?¤?¾?à?¤?™?à?¥???à?¤?---?à?¤?µ?à?¤?¿?à?¤?²?à?¥?‡?à?¤?ª?à?¤?¨?à?¤?‚?

?à?¤?•?à?¥???à?¤?°?à?¥???à?¤?µ?à?¤?¤?à?¤?¾?à?¤?‚?
?à?¤?¬?à?¥???à?¤?°?à?¤?¾?à?¤?¹?à?¥???à?¤?®?à?¤?£?à?¤?¾?à?¤?¨?à?¤?¾?à?¤?®?à?¤?ª?à?¤?¿?
?à?¥?¤? ?à?¤?\ldots{}?à?¤?¤? ?à?¤???à?¤?µ?
?à?¤?µ?à?¥?ƒ?à?¤?¦?à?¥???à?¤?§?à?¤?¶?à?¤?¾?à?¤?¤?à?¤?¾?à?¤?¤?à?¤?ª?à?¥?‡?à?¤?¨?
?à?¥?¤?

? ? ? ? ? ?à?¤?ª?à?¤?µ?à?¤?¿?à?¤?¤?à?¥???à?¤?°?à?¤?‚? ?à?¤?¤?à?¥???
?à?¤?•?à?¤?°?à?¥?‡? ?à?¤?•?à?¥?ƒ?à?¤?¤?à?¥???à?¤?µ?à?¤?¾? ?à?¤?¯?à?¤?ƒ?
?à?¤?¸?à?¤?®?à?¤?¾?à?¤?²?à?¤?­?à?¤?¤?à?¥?‡?
?à?¤?¦?à?¥???à?¤?µ?à?¤?¿?à?¤?œ?à?¤?¾?à?¤?¨?à?¥??? ?à?¥?¤?

? ? ? ? ?à?¤?°?à?¤?¾?à?¤?•?à?¥???à?¤?·?à?¤?¸?à?¤?¾?à?¤?¨?à?¤?¾?à?¤?‚?
?à?¤?­?à?¤?µ?à?¥?‡?à?¤?š?à?¥???à?¤?›?à?¥???à?¤?°?à?¤?¾?à?¤?¦?à?¥???à?¤?§?à?¤?‚?
?à?¤?¨?à?¤?¿?à?¤?°?à?¤?¾?à?¤?¶?à?¤?¾?à?¤?ƒ?
?à?¤?ª?à?¤?¿?à?¤?¤?à?¤?°?à?¥?‹? ?à?¤?---?à?¤?¤?à?¤?¾?à?¤?ƒ? ?à?¥?¥?
?à?¤?‡?à?¤?¤?à?¤?¿? ?à?¥?¤?

?à?¤?¶?à?¥???à?¤?°?à?¤?¾?à?¤?¦?à?¥???à?¤?§?à?¤?•?à?¤?°?à?¥???à?¤?¤?à?¥???à?¤?¤?à?¤?°?à?¤?¿?
?à?¤?š?
?à?¤?µ?à?¤?¿?à?¤?²?à?¥?‡?à?¤?ª?à?¤?¨?à?¤?•?à?¤?¾?à?¤?°?à?¤?¿?à?¤?£?à?¤?¿?
?à?¤?¸?à?¤?ª?à?¤?µ?à?¤?¿?à?¤?¤?à?¥???à?¤?°?à?¤?•?à?¤?°?à?¥?‡?
?à?¤?¦?à?¥?‹?à?¤?·?à?¥?‹?à?¤?½?à?¤?­?à?¤?¿?à?¤?¹?à?¤?¿?à?¤?¤?à?¤?ƒ?
?à?¥?¤?

?à?¤?¸?à?¤?®?à?¤?¾?à?¤?²?à?¤?®?à?¥???à?¤?­?à?¤?¨?à?¤?®?à?¥??? ?=?
?à?¤?µ?à?¤?¿?à?¤?²?à?¥?‡?à?¤?ª?à?¤?¨?à?¤?®?à?¥??? ?à?¥?¤?

? ? ? ? ?à?¤?•?à?¥?‡?à?¤?š?à?¤?¿?à?¤?¤?à?¥???à?¤?¤?à?¥???
?à?¤?¬?à?¥???à?¤?°?à?¤?¾?à?¤?¹?à?¥???à?¤?®?à?¤?£?à?¤?¾?
?à?¤?\ldots{}?à?¤?ª?à?¤?¿?
?à?¤?¸?à?¥???à?¤?µ?à?¤?¾?à?¤?™?à?¥???à?¤?---?à?¤?µ?à?¤?¿?à?¤?²?à?¥?‡?à?¤?ª?à?¤?¨?à?¤?‚?
?à?¤?•?à?¥???à?¤?°?à?¥???à?¤?µ?à?¤?¾?à?¤?£?à?¤?¾?à?¤?ƒ?
?à?¤?ª?à?¤?µ?à?¤?¿?à?¤?¤?à?¥???à?¤?°?à?¤?¾?à?¤?£?à?¤?¿?
?à?¤?µ?à?¤?¿?à?¤?¨?à?¥???à?¤?¯?

?à?¤?¸?à?¥?‡?à?¤?¯?à?¥???à?¤?°?à?¤?¿?à?¤?¤?à?¥???à?¤?¯?à?¤?¾?à?¤?¹?à?¥???à?¤?ƒ?
?à?¥?¤?

? ? ? ? ?à?¤?\ldots{}?à?¤?¤?à?¥???à?¤?°?
?à?¤?µ?à?¤?¿?à?¤?¶?à?¥?‡?à?¤?·?à?¤?¾?à?¤?¨?à?¥???à?¤?¤?à?¤?°?à?¤?‚?
?à?¤?¦?à?¥?‡?à?¤?µ?à?¤?²?à?¤?¸?à?¥???à?¤?®?à?¥?ƒ?à?¤?¤?à?¤?¿?à?¤?­?à?¤?µ?à?¤?¿?à?¤?·?à?¥???à?¤?¯?à?¤?¤?à?¥???à?¤?ª?à?¥???à?¤?°?à?¤?¾?à?¤?£?à?¤?¯?à?¥?‹?à?¤?ƒ?
?à?¥?¤?

? ? ? ? ? ? ?
?à?¤?¯?à?¤?œ?à?¥???à?¤?ž?à?¥?‹?à?¤?ª?à?¤?µ?à?¥?€?à?¤?¤?à?¤?‚?
?à?¤?µ?à?¤?¿?à?¤?ª?à?¥???à?¤?°?à?¤?¾?à?¤?£?à?¤?¾?à?¤?‚?
?à?¤?¸?à?¥???à?¤?•?à?¤?¨?à?¥???à?¤?§?à?¤?¾?à?¤?¨?à?¥???à?¤?¨?à?¥?ˆ?à?¤?µ?à?¤?¾?à?¤?µ?à?¤?¤?à?¤?¾?à?¤?°?à?¤?¯?à?¥?‡?à?¤?¤?à?¥???
?à?¥?¤?

? ? ? ? ? ? ?
?à?¤?---?à?¤?¨?à?¥???à?¤?§?à?¤?¾?à?¤?¦?à?¤?¿?à?¤?ª?à?¥?‚?à?¤?œ?à?¤?¾?à?¤?¸?à?¤?¿?à?¤?¦?à?¥???à?¤?§?à?¥???à?¤?¯?à?¤?°?à?¥???à?¤?¥?à?¤?‚?
?à?¤?¦?à?¥?ˆ?à?¤?µ?à?¥?‡?
?à?¤?ª?à?¤?¿?à?¤?¤?à?¥???à?¤?°?à?¥???à?¤?¯?à?¥?‡? ?à?¤?š?
?à?¤?•?à?¤?°?à?¥???à?¤?®?à?¤?£?à?¤?¿? ?à?¥?¥? ?à?¤?‡?à?¤?¤?à?¤?¿?
?à?¥?¤?

?à?¤?¯?à?¤?œ?à?¤?®?à?¤?¾?à?¤?¨?à?¥?‹?
?à?¤?¨?à?¤?¾?à?¤?µ?à?¤?¤?à?¤?¾?à?¤?°?à?¤?¯?à?¥?‡?à?¤?¦?à?¤?¿?à?¤?¤?à?¥???à?¤?¯?à?¤?°?à?¥???à?¤?¥?à?¤?ƒ?
?à?¥?¤?

?à?¤?¶?à?¤?¾?à?¤?¤?à?¤?¾?à?¤?¤?à?¤?ª?à?¥?‹?à?¤?½?à?¤?ª?à?¤?¿?,?

? ? ? ? ? ?à?¤?•?à?¤?Ÿ?à?¥???à?¤?¯?à?¤?¾?à?¤?‚?
?à?¤?¤?à?¥???à?¤?°?à?¤?¿?à?¤?ª?à?¥???à?¤?·?à?¥???à?¤?•?à?¤?°?à?¤?‚?
?à?¤?•?à?¥?ƒ?à?¤?¤?à?¥???à?¤?µ?à?¤?¾?
?à?¤?---?à?¤?¨?à?¥???à?¤?§?à?¥?ˆ?à?¤?°?à?¥???à?¤?¯?à?¤?¸?à?¥???à?¤?¤?à?¥???
?à?¤?µ?à?¤?¿?à?¤?²?à?¤?¿?à?¤?®?à?¥???à?¤?ª?à?¤?¤?à?¤?¿? ?à?¥?¤?

? ? ? ? ?à?¤?ª?à?¤?¿?à?¤?¤?à?¥?ƒ?à?¤?¯?à?¤?œ?à?¥???à?¤?ž?à?¥?‡?
?à?¤?¨?à?¤?µ?à?¥?‡? ?à?¤?›?à?¤?¿?à?¤?¦?à?¥???à?¤?°?à?¤?‚?
?à?¤?¨?à?¤?¿?à?¤?°?à?¤?¾?à?¤?¶?à?¥?ˆ?à?¤?ƒ?
?à?¤?ª?à?¤?¿?à?¤?¤?à?¥?ƒ?à?¤?­?à?¤?¿?à?¤?°?à?¥???à?¤?---?à?¤?¤?à?¥?ˆ?à?¤?ƒ?
?à?¥?¥? ?à?¤?‡?à?¤?¤?à?¤?¿?
?à?¥?¤?\textless{}?/?s?p?a?n?\textgreater{}?\textless{}?/?p?\textgreater{}?\textless{}?/?b?o?d?y?\textgreater{}?\textless{}?/?h?t?m?l?\textgreater{}?
?

? ?b?o?d?y?\{? ?w?i?d?t?h?:? ?2?1?c?m?;? ?h?e?i?g?h?t?:? ?2?9?.?7?c?m?;?
?m?a?r?g?i?n?:? ?3?0?m?m? ?4?5?m?m? ?3?0?m?m? ?4?5?m?m?;? ?\}?
?\textless{}?/?s?t?y?l?e?\textgreater{}?\textless{}?!?D?O?C?T?Y?P?E?
?H?T?M?L? ?P?U?B?L?I?C? ?"?-?/?/?W?3?C?/?/?D?T?D? ?H?T?M?L?
?4?.?0?/?/?E?N?"?
?"?h?t?t?p?:?/?/?w?w?w?.?w?3?.?o?r?g?/?T?R?/?R?E?C?-?h?t?m?l?4?0?/?s?t?r?i?c?t?.?d?t?d?"?\textgreater{}?
?

?

?

?

? ?p?,? ?l?i? ?\{? ?w?h?i?t?e?-?s?p?a?c?e?:? ?p?r?e?-?w?r?a?p?;? ?\}?
?\textless{}?/?s?t?y?l?e?\textgreater{}?\textless{}?/?h?e?a?d?\textgreater{}?

? ?

?

? ? ? ? ? ? ? ? ? ? ? ? ? ? ? ? ? ?
?\textless{}?/?s?p?a?n?\textgreater{}?

?à?¤?---?à?¤?¨?à?¥???à?¤?§?à?¤?¾?à?¤?¦?à?¤?¿?à?¤?¦?à?¤?¾?à?¤?¨?à?¤?ª?à?¥???à?¤?°?à?¤?•?à?¤?¾?à?¤?°?à?¤?ƒ?
?à?¥?¤? ? ? ? ? ? ? ? ? ? ? ? ? ? ? ? ? ? ? ? ? ?
?à?¥?¨?à?¥?¨?à?¥?©?\textless{}?/?s?p?a?n?\textgreater{}?

?

?à?¤?¤?à?¥???à?¤?°?à?¤?¿?à?¤?ª?à?¥???à?¤?¸?à?¥???à?¤?•?à?¤?°?à?¤?®?à?¥???
?=? ?à?¤?‰?à?¤?ª?à?¤?µ?à?¥?€?à?¤?¤?à?¤?®?à?¥???
?\textless{}?/?s?p?a?n?\textgreater{}?

?à?¥?¤?\textless{}?/?s?p?a?n?\textgreater{}?

? ?à?¤?›?à?¤?¿?à?¤?¦?à?¥???à?¤?°? ?=? ?à?¤?µ?à?¤?¿?à?¤?µ?à?¤?°?à?¤?‚?
?à?¤?µ?à?¤?¿?à?¤?§?à?¥???à?¤?µ?à?¤?‚?à?¤?¸?à?¤?¨?à?¤?®?à?¤?¿?à?¤?¤?à?¥???à?¤?µ?à?¤?°?à?¥???à?¤?¥?à?¤?ƒ?
?à?¥?¤?
?à?¤?¤?à?¥???à?¤?°?à?¤?¿?à?¤?ª?à?¥???à?¤?·?à?¥???à?¤?•?à?¤?°?à?¤?---?à?¥???à?¤?°?

?à?¤?¹?à?¤?£?à?¤?®?à?¥???à?¤?¤?à?¥???à?¤?¤?à?¤?°?à?¥?€?à?¤?¯?à?¤?¸?à?¥???à?¤?¯?à?¤?¾?à?¤?ª?à?¥???à?¤?¯?à?¥???à?¤?ª?à?¤?²?à?¤?•?à?¥???à?¤?·?à?¤?•?à?¤?‚?,?
?à?¤?¤?à?¤?¦?à?¤?µ?à?¤?¤?à?¤?¾?à?¤?°?à?¤?£?à?¥?‡?
?à?¤?¹?à?¥???à?¤?¯?à?¥?‡?à?¤?•?à?¤?µ?à?¤?¸?à?¥???à?¤?¤?à?¥???à?¤?°?à?¤?¤?à?¤?¾?
?à?¤?¸?à?¥???à?¤?¯?à?¤?¾?à?¤?¤?à?¥??? ?à?¤?¸?à?¤?¾? ?à?¤?š?

?à?¤?ª?à?¥???à?¤?°?à?¤?¤?à?¤?¿?à?¤?·?à?¤?¿?à?¤?¦?à?¥???à?¤?§?à?¤?¾?
?à?¥?¤?

? ? ? ? ? ? ?
?à?¤?¸?à?¤?µ?à?¥???à?¤?¯?à?¤?¾?à?¤?¦?à?¤?‚?à?¤?¸?à?¤?¾?à?¤?¤?à?¥???
?à?¤?ª?à?¤?°?à?¤?¿?à?¤?­?à?¥???à?¤?°?à?¤?·?à?¥???à?¤?Ÿ?à?¤?®?à?¤?®?à?¥???à?¤?¬?à?¤?°?à?¤?‚?
?à?¤?¯?à?¤?¸?à?¥???à?¤?¤?à?¥???
?à?¤?§?à?¤?¾?à?¤?°?à?¤?¯?à?¥?‡?à?¤?¤?à?¥??? ?à?¥?¤?

? ? ? ? ? ? ? ?à?¤???à?¤?•?à?¤?µ?à?¤?¸?à?¥???à?¤?¤?à?¥???à?¤?°?à?¤?‚?
?à?¤?¤?à?¥??? ?à?¤?¤?à?¤?‚?
?à?¤?µ?à?¤?¿?à?¤?¦?à?¥???à?¤?¯?à?¤?¾?à?¤?¦?à?¥???à?¤?µ?à?¥?ˆ?à?¤?µ?à?¥?‡?
?à?¤?ª?à?¤?¿?à?¤?¤?à?¥???à?¤?°?à?¥???à?¤?¯?à?¥?‡? ?à?¤?š?
?à?¤?µ?à?¤?°?à?¥???à?¤?œ?à?¤?¯?à?¥?‡?à?¤?¤?à?¥??? ?à?¥?¤?

?à?¤?‡?à?¤?¤?à?¤?¿? ?à?¤?µ?à?¤?š?à?¤?¨?à?¥?‡?à?¤?¨? ?à?¥?¤?
?à?¤?¤?à?¤?¥?à?¤?¾?à?¤?¶?à?¤?™?à?¥???à?¤?--?à?¥?‡?à?¤?¨?à?¤?¾?à?¤?ª?à?¤?¿?
?à?¤?¦?à?¥?‹?à?¤?·? ?à?¤?‰?à?¤?•?à?¥???à?¤?¤?à?¤?ƒ? ?à?¥?¤?

? ? ? ? ? ? ?à?¤?‰?à?¤?ª?à?¤?µ?à?¥?€?à?¤?¤?à?¤?‚? ?à?¤?•?à?¤?Ÿ?à?¥?Œ?
?à?¤?•?à?¥?ƒ?à?¤?¤?à?¥???à?¤?µ?à?¤?¾?
?à?¤?•?à?¥???à?¤?°?à?¥???à?¤?¯?à?¤?¾?à?¤?¦?à?¥???à?¤?---?à?¤?¾?à?¤?¤?à?¥???à?¤?°?à?¤?¾?à?¤?¨?à?¥???à?¤?²?à?¥?‡?à?¤?ª?à?¤?¨?à?¤?®?à?¥???
?\textless{}?/?s?p?a?n?\textgreater{}?

?à?¥?¤?\textless{}?/?s?p?a?n?\textgreater{}?

?

? ? ? ? ? ? ?à?¤???à?¤?•?à?¤?µ?à?¤?¾?à?¤?¸?à?¤?¾?à?¤?¶?à?¥???à?¤?š?
?à?¤?¯?à?¥?‹?à?¤?½?à?¤?¶?à?¥???à?¤?¨?à?¥?€?à?¤?¯?à?¤?¾?à?¤?¨?à?¥???à?¤?¨?à?¤?¿?à?¤?°?à?¤?¾?à?¤?¶?à?¤?¾?à?¤?ƒ?
?à?¤?ª?à?¤?¿?à?¤?¤?à?¤?°?à?¥?‹? ?à?¤?---?à?¤?¤?à?¤?¾?à?¤?ƒ? ?à?¥?¥?
?à?¤?‡?à?¤?¤?à?¤?¿? ?à?¥?¤?

?à?¤?µ?à?¤?¿?à?¤?²?à?¥?‡?à?¤?ª?à?¤?¨?à?¤?‚?
?à?¤?š?à?¥?‹?à?¤?°?à?¥???à?¤?§?à?¥???à?¤?µ?à?¤?ª?à?¥???à?¤?£?à?¥???à?¤?¡?à?¥???à?¤?°?à?¤?¾?à?¤?•?à?¤?¾?à?¤?°?à?¤?‚?
?à?¤?•?à?¤?¾?à?¤?°?à?¥???à?¤?¯?à?¤?‚? ?à?¤?¨? ?à?¤?¤?à?¥???
?à?¤?µ?à?¤?°?à?¥???à?¤?¤?à?¥???à?¤?¤?à?¥???à?¤?²?à?¤?¾?à?¤?¦?à?¤?¿?à?¤?ª?à?¥???à?¤?£?à?¥???à?¤?¡?à?¥???à?¤?°?à?¤?¾?à?¤?•?à?¤?¾?à?¤?°?à?¤?®?à?¤?¿?à?¤?¤?à?¤?¿?

?à?¤?¹?à?¥?‡?à?¤?®?à?¤?¾?à?¤?¦?à?¥???à?¤?°?à?¤?¿?à?¤?ƒ? ?à?¥?¤?
?à?¤?®?à?¤?¾?à?¤?²?à?¥???à?¤?¯?à?¤?§?à?¤?¾?à?¤?°?à?¤?£?à?¤?ª?à?¥???à?¤?°?à?¤?¦?à?¥?‡?à?¤?¶?à?¤?¸?à?¥???à?¤?¤?à?¥???
?à?¤?µ?à?¥?ƒ?à?¤?¦?à?¥???à?¤?§?à?¤?®?à?¤?¨?à?¥???à?¤?¨?à?¥?‹?à?¤?•?à?¥???à?¤?¤?à?¤?ƒ?
?à?¥?¤?

? ? ? ?
?à?¤?‰?à?¤?ª?à?¤?µ?à?¥?€?à?¤?¤?à?¤?¿?à?¤?¤?à?¥???à?¤?µ?à?¤?®?à?¥???à?¤?¤?à?¥???à?¤?¸?à?¥?ƒ?à?¤?œ?à?¥???à?¤?¯?
?à?¤?•?à?¤?¾?à?¤?°?à?¤?¯?à?¥?‡?à?¤?¨?à?¥???à?¤?¨?à?¤?¾?à?¤?¨?à?¥???à?¤?²?à?¥?‡?à?¤?ª?à?¤?¨?à?¤?®?à?¥???
?à?¥?¤?

? ? ? ?à?¤?¨? ?à?¤?¨?à?¤?¿?à?¤?¯?à?¥???à?¤?•?à?¥???à?¤?¤?à?¤?ƒ?
?à?¤?¶?à?¤?¿?à?¤?--?à?¤?¾?à?¤?µ?à?¤?°?à?¥???à?¤?œ?à?¤?‚?
?à?¤?®?à?¤?¾?à?¤?²?à?¥???à?¤?¯?à?¤?‚? ?à?¤?¶?à?¤?¿?à?¤?°?à?¤?¸?à?¤?¿?
?à?¤?§?à?¤?¾?à?¤?°?à?¤?¯?à?¥?‡?à?¤?¤?à?¥???
?\textless{}?/?s?p?a?n?\textgreater{}?

?à?¥?¥?\textless{}?/?s?p?a?n?\textgreater{}?

? ?à?¤?‡?à?¤?¤?à?¤?¿? ?à?¥?¤?

?à?¤?¨?à?¤?¿?à?¤?¯?à?¥???à?¤?•?à?¥???à?¤?¤?à?¤?ƒ? ?=?
?à?¤?¶?à?¥???à?¤?°?à?¤?¾?à?¤?¦?à?¥???à?¤?§?à?¥?‡?
?à?¤?¨?à?¤?¿?à?¤?®?à?¤?¨?à?¥???à?¤?¤?à?¥???à?¤?°?à?¤?¿?à?¤?¤?à?¤?ƒ?
?à?¥?¤? ?à?¤?\ldots{}?à?¤?¤?à?¥???à?¤?°?
?à?¤?ª?à?¥???à?¤?·?à?¥???à?¤?ª?à?¤?¦?à?¤?¾?à?¤?¨?à?¥?‡?
?à?¤?®?à?¤?¨?à?¥???à?¤?¤?à?¥???à?¤?°? ?à?¤?‰?à?¤?•?à?¥???à?¤?¤?à?¥?‹?
?à?¤?µ?à?¤?¿?à?¤?·?à?¥???à?¤?£?à?¥???à?¤?¨?à?¤?¾? ?à?¥?¤?

?â?€?œ?
?à?¤?ª?à?¥???à?¤?·?à?¥???à?¤?ª?à?¤?µ?à?¤?¤?à?¥?€?à?¤?°?à?¤?¿?à?¤?¤?à?¤?¿?
?â?€??? ?à?¤?ª?à?¥???à?¤?·?à?¥???à?¤?ª?à?¤?®?à?¤?¿?à?¤?¤?à?¤?¿? ?à?¥?¤?
?à?¤?¦?à?¤?¦?à?¥???à?¤?¯?à?¤?¾?à?¤?¦?à?¤?¿?à?¤?¤?à?¥???à?¤?¯?à?¤?¨?à?¤?¨?à?¥???à?¤?¤?à?¤?°?à?¤?ª?à?¥?‚?à?¤?°?à?¥???à?¤?µ?à?¤?µ?à?¤?¾?à?¤?•?à?¥???à?¤?¯?à?¤?¾?à?¤?¦?à?¤?¨?à?¥???à?¤?·?à?¤?œ?à?¥???à?¤?¯?à?¤?¤?à?¥?‡?
?à?¥?¤?

?à?¤?ª?à?¥???à?¤?·?à?¥???à?¤?ª?à?¤?µ?à?¤?¤?à?¥?€?à?¤?°?à?¤?¿?à?¤?¤?à?¤?¿?
?à?¤?®?à?¤?¨?à?¥???à?¤?¤?à?¥???à?¤?°?à?¤?¾?à?¤?¦?à?¤?¿?à?¤?ª?à?¥???à?¤?°?à?¤?¤?à?¥?€?à?¤?•?à?¥?‹?à?¤?ª?à?¤?¾?à?¤?¤?à?¥???à?¤?¤?à?¥?‹?
?à?¤?®?à?¤?¨?à?¥???à?¤?¤?à?¥???à?¤?°?à?¤?ƒ?
?à?¤?•?à?¤?¸?à?¥???à?¤?®?à?¤?¿?à?¤?¨?à?¥???à?¤?¨?à?¤?ª?à?¤?¿?
?à?¤?µ?à?¥?‡?à?¤?¦?à?¥?‡? ?à?¤?¨? ?à?¤?²?à?¤?­?à?¥???à?¤?¯?à?¤?¤?à?¥?‡?

?à?¤?¤?à?¥?‡?à?¤?¨?
?à?¤?ª?à?¥???à?¤?·?à?¥???à?¤?ª?à?¤?µ?à?¤?¤?à?¥?€?à?¤?ª?à?¤?¦?à?¤?‚?
?à?¤?¯?à?¤?¸?à?¥???à?¤?®?à?¤?¿?à?¤?¨?à?¥???à?¤?®?à?¤?¨?à?¥???à?¤?¤?à?¥???à?¤?°?à?¤?®?à?¤?§?à?¥???à?¤?¯?à?¥?‡?
?à?¤?µ?à?¤?°?à?¥???à?¤?¤?à?¥???à?¤?¤?à?¤?¤?à?¥?‡? ?à?¤?¸?
?à?¤?®?à?¤?¨?à?¥???à?¤?¤?à?¥???à?¤?°?à?¤?ƒ?
?à?¤?ª?à?¥???à?¤?·?à?¥???à?¤?ª?à?¤?µ?à?¤?¤?à?¥?€?à?¤?¤?à?¤?¿?
?à?¤?ª?à?¥???à?¤?°?à?¤?¤?à?¥?€?-?

?à?¤?•?à?¥?‹?à?¤?ª?à?¤?¾?à?¤?¤?à?¥???à?¤?¤?à?¥?‹?
?à?¤?œ?à?¥???à?¤?ž?à?¥?‡?à?¤?¯?à?¤?ƒ? ?à?¥?¤? ?à?¤?¸? ?à?¤?š? ?â?€?œ?
?à?¤?¯?à?¤?¾? ?à?¤?''?à?¤?·?à?¤?§?à?¤?¯?à?¤?ƒ?
?à?¤?ª?à?¥???à?¤?°?à?¤?¤?à?¤?¿?à?¤?---?à?¥?ƒ?à?¤?¹?à?¥?€?à?¤?¤?
?à?¤?ª?à?¥???à?¤?·?à?¥???à?¤?ª?à?¤?µ?à?¤?¤?à?¥?€?
?à?¤?¸?à?¥???à?¤?ª?à?¤?¿?à?¤?ª?à?¥???à?¤?ª?

?à?¤?²?à?¤?¾?à?¤?ƒ? ?à?¥?¤? ?à?¤?\ldots{}?à?¤?¯?à?¤?‚? ?à?¤?µ?à?¥?‹?
?à?¤?---?à?¤?°?à?¥???à?¤?­? ?à?¤?‹?à?¤?¤?à?¥???à?¤?µ?à?¤?¿?à?¤?¯?à?¤?ƒ?
?\textless{}?/?s?p?a?n?\textgreater{}?

?à?¤?ª?à?¥???à?¤?°?à?¤?¤?à?¥???à?¤?¨?à?¤?¸?à?¤?‚?\textless{}?/?s?p?a?n?\textgreater{}?

?à?¤?˜?à?¤?¸?à?¥???à?¤?¥?à?¤?®?à?¤?¾?à?¤?¸?à?¤?¦?à?¤?µ?â?€???
?à?¤?‡?à?¤?¤?à?¤?¿? ?à?¥?¤?
?à?¤?§?à?¥?‚?à?¤?ª?à?¤?ª?à?¥???à?¤?°?à?¤?¦?à?¤?¾?à?¤?¨?à?¥?‡?

?à?¤?®?à?¤?¨?à?¥???à?¤?¤?à?¥???à?¤?°? ?à?¤?‰?à?¤?•?à?¥???à?¤?¤?à?¥?‹?
?à?¤?µ?à?¥???à?¤?¯?à?¤?¾?à?¤?¸?à?¥?‡?à?¤?¨?
?à?¤?§?à?¥?‚?à?¤?°?à?¤?¸?à?¥?€?à?¤?¤?à?¥???à?¤?¯?à?¥???à?¤?•?à?¥???à?¤?¤?à?¥???à?¤?µ?à?¥?‡?à?¤?¤?à?¤?¿?
?à?¤?§?à?¥?‚?à?¤?ª?à?¤?‚?
?à?¤?¦?à?¤?¦?à?¥???à?¤?¯?à?¤?¾?à?¤?¦?à?¤?¿?à?¤?¤?à?¥???à?¤?¯?à?¤?°?à?¥???à?¤?¥?à?¤?ƒ?
?à?¥?¤? ?à?¤?¯?à?¤?¦?à?¥???à?¤?¯?à?¤?ª?à?¥?€?à?¤?¦?à?¤?‚?
?à?¤?¯?à?¤?œ?à?¥???-?

?à?¤?¦?à?¤?°?à?¥???à?¤?¶?à?¤?ª?à?¥?‚?à?¤?°?à?¥???à?¤?£?à?¤?®?à?¤?¾?à?¤?²?à?¤?¯?à?¥?‹?à?¤?°?à?¥???à?¤?²?à?¤?¿?à?¤?™?à?¥???à?¤?---?à?¥?‡?à?¤?¨?à?¤?¾?à?¤?¨?à?¥?‹?à?¤?§?à?¥???à?¤?°?à?¤?­?à?¤?¿?à?¤?®?à?¤?°?à?¥???à?¤?¶?à?¤?¨?à?¥?‡?
?à?¤?µ?à?¤?¿?à?¤?¨?à?¤?¿?à?¤?¯?à?¥???à?¤?•?à?¥???à?¤?¤?à?¤?‚? ?,?
?à?¤?¤?à?¤?¥?à?¤?¾?à?¤?ª?à?¤?¿?
?à?¤?µ?à?¤?¿?à?¤?²?à?¤?™?à?¥???à?¤?---?à?¤?•?-?

?à?¤?®?à?¤?ª?à?¤?¿?
?à?¤?¸?à?¥???à?¤?®?à?¤?¾?à?¤?¤?à?¥???à?¤?¤?à?¥?‡?à?¤?¨?
?à?¤?µ?à?¤?š?à?¤?¨?à?¥?‡?à?¤?¨?
?à?¤?§?à?¥?‚?à?¤?ª?à?¤?¦?à?¤?¾?à?¤?¨?à?¥?‡?à?¤?½?à?¤?ª?à?¤?¿?
?à?¤?µ?à?¤?¿?à?¤?¨?à?¤?¿?à?¤?¯?à?¥???à?¤?•?à?¥???à?¤?¤?à?¤?®?à?¥???
?à?¥?¤? ?à?¤???à?¤?¨?à?¥???à?¤?¦?à?¥???à?¤?°?à?¤?¯?à?¤?¾?
?à?¤?---?à?¤?¾?à?¤?°?à?¥???à?¤?¹?à?¤?ª?à?¤?¤?à?¥???à?¤?¯?à?¤?®?à?¥???-?

?à?¤?ª?à?¤?¤?à?¤?¿?à?¤?·?à?¥???à?¤?~?à?¤?¤?
?à?¤?‡?à?¤?¤?à?¤?¿?à?¤?µ?à?¤?¤?à?¥???
?\textless{}?/?s?p?a?n?\textgreater{}?

?à?¥?¤?\textless{}?/?s?p?a?n?\textgreater{}?

?

?à?¤?¬?à?¥???à?¤?°?à?¤?¹?à?¥???à?¤?®?à?¤?¾?à?¤?£?à?¥???à?¤?¡?à?¤?ª?à?¥???à?¤?°?à?¤?¾?à?¤?£?à?¥?‡?
?à?¤?¤?à?¥???
?à?¤?®?à?¤?¨?à?¥???à?¤?¤?à?¥???à?¤?°?à?¤?¾?à?¤?¨?à?¥???à?¤?¤?à?¤?°?à?¤?®?à?¥???à?¤?•?à?¥???à?¤?¤?à?¤?®?à?¥???
?à?¥?¤?

?à?¤?µ?à?¤?¨?à?¤?¸?à?¥???à?¤?ª?à?¤?¤?à?¤?¿?à?¤?°?à?¤?¸?à?¥?‹?
?à?¤?¦?à?¤?¿?à?¤?µ?à?¥???à?¤?¯?à?¥?‹?
?à?¤?---?à?¤?¨?à?¥???à?¤?§?à?¤?¾?à?¤?¢?à?¥???à?¤?¯?à?¤?ƒ?
?à?¤?¸?à?¥???à?¤?®?à?¤?¨?à?¥?‹? ?à?¤?¹?à?¤?°?à?¤?ƒ? ?à?¥?¤?

?(?à?¥?§?)? ?à?¤?†?à?¤?¹?à?¤?¾?à?¤?°?à?¤?ƒ?
?à?¤?¸?à?¤?°?à?¥???à?¤?µ?à?¤?¦?à?¥?‡?à?¤?µ?à?¤?¾?à?¤?¨?à?¤?¾?à?¤?‚?
?à?¤?§?à?¥?‚?à?¤?ª?à?¥?‹?à?¤?½?à?¤?¯?à?¤?‚?
?à?¤?ª?à?¥???à?¤?°?à?¤?¤?à?¤?¿?à?¤?---?à?¥?ƒ?à?¤?¹?à?¥???à?¤?¯?à?¤?¤?à?¤?¾?à?¤?®?à?¥???
?à?¥?¥? ?à?¤?‡?à?¤?¤?à?¤?¿? ?à?¥?¤?

? ? ? ? ? ?à?¤?§?à?¥?‚?à?¤?ª?à?¤?¶?à?¥???à?¤?š?
?à?¤?¬?à?¥???à?¤?°?à?¤?¾?à?¤?¹?à?¥???à?¤?®?à?¤?£?à?¤?¸?à?¤?¾?à?¤?¨?à?¥???à?¤?¨?à?¤?¿?à?¤?§?à?¥???à?¤?¯?à?¥?‡?
?à?¤?µ?à?¥???à?¤?¯?à?¤?œ?à?¤?¨?à?¤?¾?à?¤?¦?à?¤?¿?à?¤?µ?à?¤?¾?à?¤?¤?à?¥?‡?à?¤?¨?
?à?¤?ª?à?¥???à?¤?°?à?¥?‡?à?¤?°?à?¤?£?à?¥?€?à?¤?¯?à?¥?‹? ?à?¤?¨?
?à?¤?¹?à?¤?¸?à?¥???à?¤?¤?à?¤?µ?à?¤?¾?à?¤?¤?à?¤?¨?à?¥?‡? ?à?¥?¤?

?à?¤?¤?à?¤?¥?à?¤?¾? ?à?¤?š? ?à?¤?¶?à?¤?¾?à?¤?¤?à?¤?¾?à?¤?¤?à?¤?ª?,?

? ? ? ? ?
?à?¤?¹?à?¤?¸?à?¥???à?¤?¤?à?¤?µ?à?¤?¾?à?¤?¤?à?¤?¾?à?¤?¹?à?¤?¤?à?¤?‚?
?à?¤?§?à?¥?‚?à?¤?ª?à?¤?‚? ?à?¤?¯?à?¥?‡?
?à?¤?ª?à?¤?¿?à?¤?¬?à?¤?¨?à?¥???à?¤?¤?à?¤?¿?
?à?¤?¦?à?¥???à?¤?µ?à?¤?¿?à?¤?œ?à?¥?‹?à?¤?¤?à?¥???à?¤?¤?à?¤?®?à?¤?¾?à?¤?ƒ?
?à?¥?¤?

? ? ? ? ?à?¤?µ?à?¥?ƒ?à?¤?¥?à?¤?¾? ?à?¤?­?à?¤?µ?à?¤?¤?à?¤?¿?
?à?¤?¤?à?¤?š?à?¥???à?¤?›?à?¥???à?¤?°?à?¤?¾?à?¤?¦?à?¥???à?¤?§?à?¤?‚?
?à?¤?¤?à?¤?¸?à?¥???à?¤?®?à?¤?¾?à?¤?¤?à?¥???à?¤?¤?à?¤?‚?
?à?¤?ª?à?¤?°?à?¤?¿?à?¤?µ?à?¤?°?à?¥???à?¤?œ?à?¤?¯?à?¥?‡?à?¤?¤?à?¥???
?à?¥?¥? ?à?¤?‡?à?¤?¤?à?¤?¿? ?à?¥?¤?

?à?¤?§?à?¥?‚?à?¤?ª?à?¤?¦?à?¤?¾?à?¤?¨?à?¤?¾?à?¤?¨?à?¤?¨?à?¥???à?¤?¤?à?¤?°?à?¤?‚?
?à?¤?¯?à?¤?¥?à?¥?‹?à?¤?•?à?¥???à?¤?¤?à?¤?˜?à?¥?ƒ?à?¤?¤?à?¤?¤?à?¥?ˆ?à?¤?²?à?¤?µ?à?¤?°?à?¥???à?¤?¤?à?¥???à?¤?¯?à?¤?¾?à?¤?¦?à?¤?¿?à?¤?¨?à?¤?¾?
?à?¤?ª?à?¥???à?¤?°?à?¤?µ?à?¤?°?à?¥???à?¤?¤?à?¥???à?¤?¤?à?¤?¿?à?¤?¤?à?¤?¦?à?¥?€?à?¤?ª?à?¥?‹?
?à?¤?¦?à?¥?‡?à?¤?¯?à?¤?ƒ?
?à?¥?¤?\textless{}?/?s?p?a?n?\textgreater{}?\textless{}?/?p?\textgreater{}?
?

? ?

?\textless{}?/?b?o?d?y?\textgreater{}?\textless{}?/?h?t?m?l?\textgreater{}?

{२२४ वीरमित्रोदयस्य श्राद्धप्रकाशे-}{\\
तत्र व्यासः -\\
इदं ज्योतिरिति प्रोच्य दीप दद्यात् समाहितः ।\\
एवं दीपं दत्वा वस्त्रं दद्यात् ।\\
तत्र वस्त्रदाने मन्त्रो, वस्त्राभावे प्रतिनिधित्वेन यशोपवीतदानमुक्त\\
शातातपेन ` युवासुवासा ' इति दद्यात्, तदभावे चौपर्वातकमिति । `` यु-\\
वासुवासाः परिवीत आगात्सउश्रेयान्भवति जायमानः । तं धीरा-\\
सः कवय उन्नयन्ति स्वाध्यो मनसा देवयन्त इति ।\\
अत्रिणा तु वस्त्रदाने मन्त्रान्तरमुक्तम् । `` युव वस्त्राणि मन्त्रेण\\
दद्याद्वस्त्राणि भक्तित इति । `` युवं वस्त्राणि पीवसा वसाथे युवोर.\\
च्छ्रिदामन्तवो ह सर्गाः । अवातिरतमभृतानि विश्वऋतेन मित्राव\\
रुणासवेथे इति ।\\
एतेषां च गन्धादिमन्त्राणां दैवपित्र्यसाधारण्यमुक्तम्-\\
भविष्योत्तरे,\\
गन्धद्वारामित्यनेन गन्धं दद्यात्प्रयत्नतः ।\\
पुष्पवत्या च पुष्पाणि धूरसीति च धूपकम् ॥\\
दीपं चेदं ज्योतिरिति दैवे पित्र्ये च कर्मणि ।\\
युवं वस्त्राणि मन्त्रेण दद्यात वासांसि शक्तितः । इति ।\\
विशेषस्तु आदित्यपुराणेऽभिहितः ।\\
अतोऽर्थं प्रदद्देद् धूपं पितॄनुदिश्य धर्मवित् ॥\\
सङ्कीर्त्यनामगोत्रादि प्रत्येकं च प्रकल्पयते । इति ।\\
यस्माद् धूपदानेन पि}{तॄ}{णामक्षया तृप्तिः प्रीतिर्जायते अत एतदर्थं\\
नामगोत्रानुच्चारणपूर्वकं पितॄनुद्दिश्य ब्राह्मणानां पुरतो घृतमधुसं-\\
युक्तं गुग्गुलाद्युक्त धूपं प्रदहेत् । तत्र पित्रादिभ्यः कल्पयेत्
पित्रा\\
द्युद्देशेन दद्यादित्यर्थः । धूपग्रहणं गन्धादीनामुपलक्षणार्थम् । तथाच\\
पैठीनसिः,\\
नामगोत्रे समुच्चार्य दद्याच्छ्रद्धासमन्वितः ।\\
पितॄनुद्दिश्य विप्रेभ्यो गन्धादीन् देवपूर्वकम् ॥ इति ।\\
आदिना पुष्पधूपदीपाच्छादनानि संगृह्यन्ते । अत एव\\
हारीतः,\\
गन्धान् पितृगोत्रनाम गृहीत्वाऽप उपस्पृश्यैषमेवेतरयोर्गन्धधूप-\\
दीपमास्याच्छादनमिति । `दद्यादिति शेषः । अत्र च गन्धादिदाने म\\


?

? ?b?o?d?y?\{? ?w?i?d?t?h?:? ?2?1?c?m?;? ?h?e?i?g?h?t?:? ?2?9?.?7?c?m?;?
?m?a?r?g?i?n?:? ?3?0?m?m? ?4?5?m?m? ?3?0?m?m? ?4?5?m?m?;? ?\}?
?\textless{}?/?s?t?y?l?e?\textgreater{}?\textless{}?!?D?O?C?T?Y?P?E?
?H?T?M?L? ?P?U?B?L?I?C? ?"?-?/?/?W?3?C?/?/?D?T?D? ?H?T?M?L?
?4?.?0?/?/?E?N?"?
?"?h?t?t?p?:?/?/?w?w?w?.?w?3?.?o?r?g?/?T?R?/?R?E?C?-?h?t?m?l?4?0?/?s?t?r?i?c?t?.?d?t?d?"?\textgreater{}?
?

?

?

?

? ?p?,? ?l?i? ?\{? ?w?h?i?t?e?-?s?p?a?c?e?:? ?p?r?e?-?w?r?a?p?;? ?\}?
?\textless{}?/?s?t?y?l?e?\textgreater{}?\textless{}?/?h?e?a?d?\textgreater{}?

? ?

?

? ? ? ? ? ? ? ? ? ? ? ?\textless{}?/?s?p?a?n?\textgreater{}?

?
?à?¤?---?à?¤?¨?à?¥???à?¤?§?à?¤?¾?à?¤?¦?à?¤?¿?à?¤?¦?à?¤?¾?à?¤?¨?à?¤?ª?à?¥???à?¤?°?à?¤?•?à?¤?¾?à?¤?°?à?¤?ƒ?
?à?¥?¤? ? ? ? ? ? ? ? ? ? ? ? ? ? ?à?¥?¨?à?¥?¨?à?¥?«?
?\textless{}?/?s?p?a?n?\textgreater{}?

?

?à?¤?¨?à?¥???à?¤?¤?à?¥???à?¤?°?à?¥?‹?à?¤?š?à?¥???à?¤?š?à?¤?¾?à?¤?°?à?¤?£?à?¤?¾?à?¤?¨?à?¤?¨?à?¥???à?¤?¤?à?¤?°?à?¤?‚?
?à?¤?---?à?¤?¨?à?¥???à?¤?§?à?¤?¾?à?¤?¦?à?¤?¿?à?¤?¦?à?¤?¾?à?¤?¨?à?¤?µ?à?¤?¾?à?¤?•?à?¥???à?¤?¯?à?¤?ª?à?¥???à?¤?°?à?¤?¯?à?¥?‹?à?¤?---?à?¥?‹?-?

?à?¤?¬?à?¥???à?¤?°?à?¤?¹?à?¥???à?¤?®?à?¤?ª?à?¥???à?¤?°?à?¤?¾?à?¤?£?à?¥?‡?,?

? ? ? ? ? ? ?à?¤?‡?à?¤?¦?à?¤?‚? ?à?¤?µ?à?¤?š?à?¤?ƒ?
?[?à?¤?µ?à?¤?¾?à?¤?¸?à?¤?ƒ?]?
?à?¤?ª?à?¤?¾?à?¤?¦?à?¥???à?¤?¯?à?¤?®?à?¤?°?à?¥???à?¤?¥?à?¥???à?¤?¯?à?¤?ª?à?¥???à?¤?·?à?¥???à?¤?ª?à?¤?§?à?¥?‚?à?¤?ª?à?¤?µ?à?¤?¿?à?¤?²?à?¥?‡?à?¤?ª?à?¤?¨?à?¤?®?à?¥???
?à?¥?¤?

? ? ? ? ? ? ?à?¤?\ldots{}?à?¤?¯?à?¤?‚? ?à?¤?¦?à?¥?€?à?¤?ª?à?¤?ƒ?
?à?¤?ª?à?¥???à?¤?°?à?¤?•?à?¤?¾?à?¤?¶?à?¤?¶?à?¥???à?¤?š?
?à?¤?µ?à?¤?¿?à?¤?¶?à?¥???à?¤?µ?à?¥?‡?à?¤?¦?à?¥?‡?à?¤?µ?à?¤?¾?à?¤?ƒ?
?à?¤?¸?à?¤?®?à?¤?°?à?¥???à?¤?ª?à?¥???à?¤?¯?à?¤?¤?à?¥?‡?
?à?¥?¥?\textless{}?/?s?p?a?n?\textgreater{}?\textless{}?/?p?\textgreater{}?
?

?

?à?¤?¤?à?¤?¥?à?¤?¾?,?

? ? ? ? ? ? ?à?¤?\ldots{}?à?¤?¯?à?¤?‚? ?à?¤?µ?à?¥?‹? ?à?¤?¦?à?¥?€?à?¤?ª?
?à?¤?‡?à?¤?¤?à?¥???à?¤?¯?à?¥???à?¤?•?à?¥???à?¤?¤?à?¥???à?¤?µ?à?¤?¾?
?à?¤?¦?à?¥?€?à?¤?ª?à?¤?‚? ?à?¤?¹?à?¥?ƒ?à?¤?¦?à?¥???à?¤?¯?à?¤?‚?
?à?¤?¨?à?¤?¿?à?¤?µ?à?¥?‡?à?¤?¦?à?¤?¯?à?¥?‡?à?¤?¤?à?¥??? ?à?¥?¤?

? ? ? ? ? ? ?à?¤?\ldots{}?à?¤?¨?à?¤?™?à?¥???à?¤?---?à?¤?²?à?¤?¨?à?¤?‚?
?à?¤?¸?à?¤?¦?à?¥???à?¤?µ?à?¤?¸?à?¥???à?¤?¤?à?¥???à?¤?°?à?¤?‚?
?à?¤?­?à?¤?µ?à?¥?‡?à?¤?¦?à?¥???à?¤?¯?à?¤?¤?à?¥???à?¤?¤?à?¤?¦?à?¥???à?¤?¯?à?¥???à?¤?---?à?¤?‚?
?à?¤?¶?à?¥???à?¤?­?à?¤?®?à?¥??? ?à?¥?¥?

? ? ? ? ? ? ? ?à?¤?‡?à?¤?¦?à?¤?‚? ?à?¤?µ?à?¥?‹?
?à?¤?µ?à?¤?¸?à?¥???à?¤?¤?à?¥???à?¤?°?à?¤?®?à?¤?¿?à?¤?¤?à?¥???à?¤?¯?à?¥???à?¤?•?à?¥???à?¤?¤?à?¥???à?¤?µ?à?¤?¾?
?à?¤?¤?à?¥???à?¤?°?à?¤?¿?à?¤?¤?à?¤?¯?à?¤?‚? ?à?¤?µ?à?¤?¾?
?à?¤?¨?à?¤?¿?à?¤?µ?à?¥?‡?à?¤?¦?à?¤?¯?à?¥?‡?à?¤?¤?à?¥??? ?à?¥?¤?

?à?¤?\ldots{}?à?¤?¤?à?¥???à?¤?°?
?à?¤?¦?à?¥?€?à?¤?¯?à?¤?®?à?¤?¾?à?¤?¨?à?¤?---?à?¤?¨?à?¥???à?¤?§?à?¤?¾?à?¤?¦?à?¤?¿?à?¤?¸?à?¥???à?¤?µ?à?¥?€?à?¤?•?à?¤?¾?à?¤?°?à?¤?•?à?¤?¾?à?¤?²?à?¥?‡?
?à?¤?¬?à?¥???à?¤?°?à?¤?¾?à?¤?¹?à?¥???à?¤?®?à?¤?£?à?¤?µ?à?¤?¾?à?¤?š?à?¥???à?¤?¯?à?¤?‚?
?à?¤?¦?à?¥?‡?à?¤?µ?à?¤?²?à?¤?¨?à?¥?‹?à?¤?•?à?¥???à?¤?¤?à?¤?®?à?¥???
?à?¥?¤?

?\&?q?u?o?t?;?à?¤?‡?à?¤?¦?à?¤?‚?
?à?¤?œ?à?¥???à?¤?¯?à?¥?‹?à?¤?¤?à?¤?¿?à?¤?°?à?¤?¿?à?¤?¤?à?¤?¿?à?¤?œ?à?¥???à?¤?¯?à?¥?‹?à?¤?¤?à?¤?¿?à?¤?ƒ?,?
?à?¤?¸?à?¥???à?¤?œ?à?¥???à?¤?¯?à?¥?‹?à?¤?¤?à?¤?¿?à?¤?°?à?¤?¿?à?¤?¤?à?¤?¿?
?à?¤?¤?à?¥?‡?à?¤?½?à?¤?ª?à?¤?¿? ?à?¤?š?à?¤?¤?à?¤?¿? ?à?¥?¤?

?à?¤?‡?à?¤?¦?à?¤?‚? ?à?¤?µ?à?¥?‹?
?à?¤?œ?à?¥???à?¤?¯?à?¥?‹?à?¤?¤?à?¤?¿?à?¤?°?à?¤?¿?à?¤?¤?à?¤?¿?
?à?¤?œ?à?¥???à?¤?¯?à?¥?‹?à?¤?¤?à?¤?¿?à?¤?ƒ?
?à?¤?¶?à?¥???à?¤?°?à?¤?¾?à?¤?¦?à?¥???à?¤?§?à?¤?•?à?¤?°?à?¥???à?¤?¤?à?¤?¾?
?à?¤?¨?à?¤?¿?à?¤?µ?à?¥?‡?à?¤?¦?à?¤?¯?à?¥?‡?à?¤?¤?à?¥??? ?à?¥?¤?
?à?¤?¤?à?¥?‡?à?¤?½?à?¤?ª?à?¤?¿? ?à?¤?š?à?¥?‡?à?¤?¤?à?¤?¿? ?à?¥?¤?

?à?¤?¤?à?¥?‡?à?¤?½?à?¤?ª?à?¤?¿?
?à?¤?µ?à?¤?¿?à?¤?ª?à?¥???à?¤?°?à?¤?¾?à?¤?ƒ?
?à?¤?¸?à?¥???à?¤?œ?à?¥???à?¤?¯?à?¥?‹?à?¤?¤?à?¤?¿?à?¤?°?à?¤?¿?à?¤?¤?à?¤?¿?
?à?¤?¬?à?¥???à?¤?°?à?¥?‚?à?¤?¯?à?¥???à?¤?°?à?¤?¿?à?¤?¤?à?¥???à?¤?¯?à?¤?°?à?¥???à?¤?¥?à?¤?ƒ?
?à?¥?¤? ?à?¤?\ldots{}?à?¤?¤?à?¥???à?¤?°?
?à?¤?š?à?¤?¶?à?¤?¬?à?¥???à?¤?¦?à?¥?‡?à?¤?¨? ?à?¤?†?à?¤?¸?

?à?¤?®?à?¤?¾?à?¤?¦?à?¤?¿?à?¤?¤?à?¤?¤?à?¥???à?¤?¤?à?¤?¤?à?¥???à?¤?ª?à?¤?¦?à?¤?¾?à?¤?°?à?¥???à?¤?¥?à?¤?¸?à?¥???à?¤?µ?à?¥?€?à?¤?•?à?¤?¾?à?¤?°?à?¤?•?à?¤?¾?à?¤?²?à?¥?‡?
?à?¤?¸?à?¥???à?¤?µ?à?¤?¾?à?¤?¸?à?¤?¨?à?¤?‚?
?à?¤?¸?à?¥???à?¤?µ?à?¤?°?à?¥???à?¤?˜?à?¥???à?¤?¯?à?¤?‚?
?à?¤?¸?à?¥???à?¤?---?à?¤?¨?à?¥???à?¤?§?à?¤?ƒ?
?à?¤?¸?à?¥???à?¤?ª?à?¥???à?¤?·?à?¥???à?¤?ª?à?¤?¾?à?¤?£?à?¤?¿?

?à?¤?¸?à?¥???à?¤?§?à?¥?‚?à?¤?ª?à?¤?ƒ?
?à?¤?¸?à?¥???à?¤?¦?à?¥?€?à?¤?ª?à?¤?ƒ?
?à?¤?¸?à?¥???à?¤?µ?à?¤?¾?à?¤?š?à?¥???à?¤?›?à?¤?¾?à?¤?¦?à?¤?¨?à?¤?®?à?¤?¿?à?¤?¤?à?¤?¿?
?à?¤?¤?à?¥?‡?
?à?¤?¬?à?¥???à?¤?°?à?¥?‚?à?¤?¯?à?¥???à?¤?°?à?¤?¿?à?¤?¤?à?¥???à?¤?¯?à?¥?‡?à?¤?¤?à?¤?¦?à?¥???à?¤?•?à?¥???à?¤?¤?à?¤?‚?
?à?¤?­?à?¤?µ?à?¤?¤?à?¥?‡?à?¤?¿? ?à?¥?¤?
?à?¤?\ldots{}?à?¤?¤?à?¥???à?¤?°?à?¤?¾?-?

?à?¤?š?à?¥???à?¤?›?à?¤?¾?à?¤?¦?à?¤?¨?à?¤?¦?à?¤?¾?à?¤?¨?à?¤?¯?à?¤?œ?à?¥???à?¤?ž?à?¥?‹?à?¤?ª?à?¤?µ?à?¥?€?à?¤?¤?à?¤?¦?à?¤?¾?à?¤?¨?à?¤?¾?à?¤?¨?à?¤?¨?à?¥???à?¤?¤?à?¤?°?à?¤?‚?
?à?¤?¯?à?¤?¥?à?¤?¾?à?¤?¶?à?¤?•?à?¥???à?¤?¤?à?¥???à?¤?¯?à?¤?²?à?¤?™?à?¥???à?¤?•?à?¤?°?à?¤?£?à?¤?›?à?¤?¤?à?¥???à?¤?°?à?¤?•?à?¤?®?à?¤?£?à?¥???à?¤?¡?

?à?¤?²?à?¥???à?¤?µ?à?¤?¾?à?¤?¦?à?¥?€?à?¤?¨?à?¤?¿?
?à?¤?¦?à?¥?‡?à?¤?¯?à?¤?¾?à?¤?¨?à?¥?€?à?¤?¤?à?¤?¿? ?à?¥?¤?
?à?¤?¤?à?¤?¦?à?¥???à?¤?¦?à?¤?¾?à?¤?¨?à?¥?‡? ?à?¤?¤?à?¥???
?à?¤?¸?à?¤?™?à?¥???à?¤?•?à?¤?²?à?¥???à?¤?ª?à?¤?®?à?¤?¾?à?¤?¤?à?¥???à?¤?°?à?¤?‚?
?à?¤?¤?à?¤?¦?à?¤?¾? ?à?¥?¤?
?à?¤?¸?à?¤?®?à?¥???à?¤?ª?à?¤?¾?à?¤?¦?à?¤?¨?à?¤?‚? ?à?¤?¤?à?¥???
?à?¤?¬?à?¥???à?¤?°?à?¤?¾?

?à?¤?¹?à?¥???à?¤?®?à?¤?£?à?¤?ª?à?¥???à?¤?°?à?¤?¸?à?¥???à?¤?¥?à?¤?¾?à?¤?ª?à?¤?¨?à?¤?•?à?¤?¾?à?¤?²?
?à?¤???à?¤?µ?
?à?¤?•?à?¤?°?à?¥???à?¤?¤?à?¥???à?¤?¤?à?¤?µ?à?¥???à?¤?¯?à?¤?®?à?¤?¿?à?¤?¤?à?¤?¿?
?à?¤?¹?à?¥?‡?à?¤?®?à?¤?¾?à?¤?¦?à?¥???à?¤?°?à?¤?¿?à?¤?ƒ? ?à?¥?¤?
?à?¤?¤?à?¤?¦?à?¥?‡?à?¤?µ?à?¤?‚?
?à?¤?---?à?¤?¨?à?¥???à?¤?§?à?¤?¾?à?¤?¦?à?¤?¿?à?¤?¦?à?¤?¾?à?¤?¨?à?¤?‚?
?à?¤?•?à?¥?ƒ?-?

?à?¤?¤?à?¥???à?¤?µ?à?¤?¾?
?à?¤?•?à?¥?ƒ?à?¤?¤?à?¤?¾?à?¤?ž?à?¥???à?¤?œ?à?¤?²?à?¤?¿?à?¤?ƒ?
?à?¤?¶?à?¥???à?¤?°?à?¤?¾?à?¤?¦?à?¥???à?¤?§?à?¤?•?à?¤?°?à?¥???à?¤?¤?à?¥???à?¤?¤?à?¤?¾?
?â?€?œ?à?¤?†?à?¤?¦?à?¤?¿?à?¤?¤?à?¥???à?¤?¯?à?¤?¾?
?à?¤?°?à?¥???à?¤?¦?à?¥???à?¤?°?à?¤?¾? ?à?¤?µ?à?¤?¸?à?¤?µ? ?â?€???
?à?¤?‡?à?¤?¤?à?¥???à?¤?¯?à?¥?‡?à?¤?¤?à?¤?¾?à?¤?®?à?¥?ƒ?à?¤?š?à?¤?‚?

?à?¤?œ?à?¤?ª?à?¥?‡?à?¤?¤?à?¥??? ?à?¥?¤? ?à?¤?¤?à?¤?¥?à?¤?¾? ?à?¤?š?
?à?¤?¬?à?¥???à?¤?°?à?¤?¹?à?¥???à?¤?®?à?¤?ª?à?¥???à?¤?°?à?¤?¾?à?¤?£?à?¥?‡?
?à?¤?²?à?¤?¿?à?¤?™?à?¥???à?¤?---?à?¤?¦?à?¤?°?à?¥???à?¤?¶?à?¤?¨?à?¤?®?à?¥???
?à?¥?¤?

? ? ? ? ?à?¤?¤?à?¤?¾?à?¤?¨?à?¤?¾?à?¤?°?à?¥???à?¤?š?à?¥???à?¤?¯?
?à?¤?­?à?¥?‚?à?¤?¯?à?¥?‹?
?à?¤?---?à?¤?¨?à?¥???à?¤?§?à?¤?¾?à?¤?¦?à?¥???à?¤?¯?à?¥?ˆ?à?¤?°?à?¥???à?¤?§?à?¥?‚?à?¤?ª?à?¤?‚?
?à?¤?¦?à?¤?¤?à?¥???à?¤?µ?à?¤?¾? ?à?¤?š?
?à?¤?­?à?¤?•?à?¥???à?¤?¤?à?¤?¿?à?¤?¤?à?¤?ƒ? ?à?¥?¤?

? ? ? ? ?à?¤?†?à?¤?¦?à?¤?¿?à?¤?¤?à?¥???à?¤?¯?à?¤?¾?
?à?¤?°?à?¥???à?¤?¦?à?¥???à?¤?°?à?¤?¾? ?à?¤?µ?à?¤?¸?à?¤?µ?
?à?¤?‡?à?¤?¤?à?¥???à?¤?¯?à?¥?‡?à?¤?¤?à?¤?¾?à?¤?®?à?¤?œ?à?¤?ª?à?¤?¤?à?¥???à?¤?ª?à?¥???à?¤?°?à?¤?­?à?¥???à?¤?ƒ?
?à?¥?¥? ?à?¤?‡?à?¤?¤?à?¤?¿? ?à?¥?¤?

?à?¤?†?à?¤?¦?à?¤?¿?à?¤?¤?à?¥???à?¤?¯?à?¤?¾?
?à?¤?°?à?¥???à?¤?¦?à?¥???à?¤?°?à?¤?¾? ?à?¤?µ?à?¤?¸?à?¤?µ?à?¤?ƒ?
?à?¤?¸?à?¥???à?¤?¨?à?¥?€?à?¤?¥?à?¤?¾?
?à?¤?¦?à?¥???à?¤?¯?à?¤?¾?à?¤?µ?à?¤?¾?à?¤?•?à?¥???à?¤?·?à?¤?¾?à?¤?®?à?¤?¾?
?à?¤?ª?à?¥?ƒ?à?¤?¥?à?¤?¿?à?¤?µ?à?¥?€?
?à?¤?\ldots{}?à?¤?¨?à?¥???à?¤?¤?à?¤?°?à?¤?¿?à?¤?•?à?¥???à?¤?·?à?¤?®?à?¥???
?à?¥?¤?

?à?¤?¸?à?¤?œ?à?¥?‹?à?¤?·?à?¤?¸?à?¥?‹?
?à?¤?¯?à?¤?œ?à?¥???à?¤?ž?à?¤?®?à?¤?µ?à?¤?¨?à?¥???à?¤?¤?à?¥???
?à?¤?¦?à?¥?‡?à?¤?µ?à?¤?¾? ?à?¤?Š?à?¤?°?à?¥???à?¤?§?à?¥???à?¤?µ?à?¤?‚?
?à?¤?•?à?¥?ƒ?à?¤?£?à?¥???à?¤?µ?à?¤?¨?à?¥???à?¤?¤?à?¥???à?¤?µ?à?¤?§?à?¥???à?¤?µ?à?¤?°?à?¤?¸?à?¥???à?¤?¯?
?à?¤?•?à?¥?‡?à?¤?¤?à?¥???à?¤?®?à?¤?¿?à?¤?¤?à?¤?¿? ?à?¥?¤?

?à?¤?µ?à?¤?¿?à?¤?·?à?¥???à?¤?£?à?¥???à?¤?§?à?¤?°?à?¥???à?¤?®?à?¥?‹?à?¤?¤?à?¥???à?¤?¤?à?¤?°?à?¥?‡?
?à?¤?¤?à?¥???
?à?¤?¬?à?¥???à?¤?°?à?¤?¾?à?¤?¹?à?¥???à?¤?®?à?¤?£?à?¤?¾?à?¤?¨?à?¤?µ?à?¤?²?à?¥?‹?à?¤?•?à?¤?¯?à?¤?¨?à?¥???à?¤?¤?à?¥???à?¤?°?à?¤?¿?à?¤?®?à?¤?¾?à?¤?®?à?¥?ƒ?à?¤?š?à?¤?‚?
?à?¤?œ?à?¤?ª?à?¥?‡?à?¤?¦?à?¤?¿?à?¤?¤?à?¥???à?¤?¯?à?¥???à?¤?•?à?¥???à?¤?¤?à?¤?®?à?¥???
?à?¥?¤?

? ? ? ? ?à?¤?¸?à?¤?®?à?¥???à?¤?ª?à?¥?‚?à?¤?œ?à?¥???à?¤?¯?
?à?¤?---?à?¤?¨?à?¥???à?¤?§?à?¤?ª?à?¥???à?¤?·?à?¥???à?¤?ª?à?¤?¾?à?¤?¦?à?¥???à?¤?¯?à?¥?ˆ?à?¤?°?à?¥???à?¤?¬?à?¥???à?¤?°?à?¤?¾?à?¤?¹?à?¥???à?¤?®?à?¤?£?à?¤?¾?à?¤?¨?à?¥???à?¤?ª?à?¥???à?¤?°?à?¤?¯?à?¤?¤?à?¤?ƒ?
?à?¤?¸?à?¤?¦?à?¤?¾? ?à?¥?¤?

? ? ? ?à?¤?†?à?¤?¦?à?¤?¿?à?¤?¤?à?¥???à?¤?¯?à?¤?¾?
?à?¤?°?à?¥???à?¤?¦?à?¥???à?¤?°?à?¤?¾? ?à?¤?µ?à?¤?¸?à?¤?µ?à?¥?‹?
?à?¤?¦?à?¥???à?¤?µ?à?¤?¿?à?¤?œ?à?¤?¾?à?¤?¨?à?¥???
?à?¤?µ?à?¤?•?à?¥???à?¤?·?à?¤?¿?à?¤?‚?à?¤?¸?à?¥???à?¤?¤?à?¤?¤?à?¥?‹?
?à?¤?œ?à?¤?ª?à?¥?‡?à?¤?¤?à?¥??? ?à?¥?¥?

?à?¤?¸?à?¤?®?à?¥???à?¤?ª?à?¥?‚?à?¤?œ?à?¥???à?¤?¯?
?à?¤?¤?à?¤?¤?à?¤?¸?à?¥???à?¤?¤?à?¤?¾?à?¤?¨?à?¥???
?à?¤?µ?à?¥?€?à?¤?•?à?¥???à?¤?·?à?¤?¯?à?¥?‡?à?¤?¤?à?¥??? ?à?¥?¤?
?à?¤?\ldots{}?à?¤?¤?à?¥???à?¤?°?
?à?¤?œ?à?¤?ª?à?¤?µ?à?¥?€?à?¤?•?à?¥???à?¤?·?à?¤?£?à?¤?¯?à?¥?‹?à?¤?°?à?¥?‡?à?¤?•?à?¤?•?à?¤?¾?à?¤?²?à?¤?¤?à?¥?ˆ?à?¤?µ?,?
?à?¤?µ?à?¥?€?.?

?à?¤?•?à?¥???à?¤?·?à?¤?‚?à?¤?¸?à?¥???à?¤?¤?à?¤?¤?à?¥?‹?
?à?¤?œ?à?¤?ª?à?¥?‡?à?¤?¦?à?¤?¿?à?¤?¤?à?¤?¿?
?à?¤?¶?à?¥???à?¤?°?à?¤?µ?à?¤?£?à?¤?¾?à?¤?¤?à?¥??? ?à?¥?¤?
?à?¤?µ?à?¤?¿?à?¤?·?à?¥???à?¤?£?à?¥???à?¤?¨?à?¤?¾?
?à?¤?¤?à?¥???à?¤?•?à?¤?°?à?¤?£?à?¤?¤?à?¥???à?¤?µ?à?¤?®?à?¤?¸?à?¥???à?¤?¯?
?à?¤?®?à?¤?¨?à?¥???à?¤?¤?à?¥???à?¤?°?à?¤?¸?à?¥???à?¤?¯?à?¥?‹?à?¤?•?à?¥???à?¤?¤?à?¤?®?à?¥???
?à?¥?¤?

?â?€?œ?à?¤?µ?à?¤?¿?à?¤?ª?à?¥???à?¤?°?à?¤?¾?à?¤?¨?à?¥???à?¤?¸?à?¤?®?à?¤?­?à?¥???à?¤?¯?à?¤?°?à?¥???à?¤?š?à?¥???à?¤?¯?à?¤?¾?à?¤?¦?à?¤?¿?à?¤?¤?à?¥???à?¤?¯?à?¤?¾?
?à?¤?°?à?¥???à?¤?¦?à?¥???à?¤?°?à?¤?¾? ?à?¤?µ?à?¤?¸?à?¤?µ?
?à?¤?‡?à?¤?¤?à?¤?¿?
?à?¤?µ?à?¥?€?à?¤?•?à?¥???à?¤?·?à?¥???à?¤?¯?à?¥?‡?à?¤?¤?à?¤?¿? ?à?¥?¤?
?à?¤?\ldots{}?à?¤?¸?à?¥???à?¤?®?à?¤?¿?à?¤?¨?à?¥???à?¤?ª?à?¤?•?à?¥???à?¤?·?à?¥?‡?

?à?¤?µ?à?¥?€?à?¤?•?à?¥???à?¤?·?à?¤?£?à?¤?¾?à?¤?™?à?¥???à?¤?---?à?¤?¤?à?¥???à?¤?µ?à?¤?¾?à?¤?¨?à?¥???à?¤?®?à?¤?¨?à?¥???à?¤?¤?à?¥???à?¤?°?à?¥?‹?à?¤?½?à?¤?ª?à?¤?¿?
?à?¤?ª?à?¥???à?¤?°?à?¤?¤?à?¤?¿?à?¤?¬?à?¥???à?¤?°?à?¤?¾?à?¤?¹?à?¥???à?¤?®?à?¤?£?à?¤?®?à?¤?¾?à?¤?µ?à?¤?°?à?¥???à?¤?¤?à?¥???à?¤?¤?à?¤?¤?à?¥?‡?
?à?¥?¤? ?à?¤?¨? ?à?¤?š?
?à?¤?µ?à?¥?€?à?¤?•?à?¥???à?¤?·?à?¤?£?à?¤?®?à?¤?ª?à?¤?¿?
?à?¤?¸?à?¤?•?à?¥?ƒ?

?à?¤?µ?à?¥?€?à?¥?¦? ?à?¤?®?à?¤?¿?
?à?¥?¨?à?¥?¬?\textless{}?/?s?p?a?n?\textgreater{}?\textless{}?/?p?\textgreater{}?\textless{}?/?b?o?d?y?\textgreater{}?\textless{}?/?h?t?m?l?\textgreater{}?

{२२६ वीरमित्रोदयस्य श्राद्धप्रकाशे-}{\\
देवास्तु इति वाच्यम् । विप्रसंस्कारार्थत्वाद्वक्षणस्य प्रतिप्रधानमा.\\
वृतेर्न्यायप्राप्तत्वात् । इति गन्धादिदानविधिः ।\\
अथ मण्डलकरणादयः पदार्थाः ।\\
हेमाद्रौ कालिकापुराणे ।\\
निर्वर्त्य ब्राह्मणादेशात् क्रियामेव यथाविधि ।\\
पुनर्भूमिं च संशोध्य पङ्केरन्तरमाचरेत् ॥\\
भाजनानि ततो दद्याद्धस्तशौचं पुनः क्रमात् ।\\
भूमिशोधनमर्चनप्रसङ्गपतितकुशाद्यपनयनम् । अन्तरं = भस्मादि-\\
भिः पात्रस्थापनार्थं परस्परम्भूमेर्मर्यादाकरणम् ।\\
अत्र विशेषो भृगुणोक्त' ।\\
भस्मना वारिणा वापि कारयेन्मण्डलं ततः ।\\
चतुष्कोणं द्विजाग्र्यस्य त्रिकोणं क्षत्रियस्य तु ॥\\
मण्डलाकृतिवैश्यस्य शूद्रस्याभ्युक्षणं स्मृतम् ।\\
बह्वृचपरिशिष्टे तु दैवे चतुरस्त्रं पित्र्ये वृत्तं मण्डलं कृत्वा
क्रमेण\\
सयवान्सतिलान्दर्भान्दद्यादित्युक्तम् । मण्डलसाधनानि-\\
ब्रह्मपुराणे,\\
मण्डलानि च कार्याणि नैवारैश्चूर्णकैः शुभैः ।\\
(१) गौरमृत्तिकया वापि प्रणीतेनाथ भस्मना ।\\
पाषाणचूर्णसङ्कीर्णं मारुतं च विसर्जयेत् ॥ इति ।\\
अत्र मण्डलकरणमेव मर्यादाकरणमिति हेमाद्रिः ।\\
प्रयोगपरिजातस्तु नीवारचूर्णादिना मण्डलानि कृत्वा पात्राणि\\
{[}सं{]} स्थाप्य भस्मना मर्यादां कुर्यादित्याह ।\\
ततः कृतान्तरायां भूमावेव भोजनपात्राणि स्थापयेत् । भूमावेव\\
निदध्यान्नोपरिपात्राणीति हारीतोक्तेः । उपरि = यन्त्रिकादौ । ततो हस्त\\
शुद्धिं कुर्यात् हस्तप्रक्षालनोदकं च शुचिदेशे प्रक्षिपेत् ।\\
प्रक्षाल्य हस्तपादादि पश्चादद्भिर्विधानवित् ।\\
प्रक्षालनजलं दर्भेस्तिलैर्मिश्र क्षिपेच्छुचौ ॥\\
एतच्च पादप्रक्षालनस्थाने क्षेप्यामेति शिष्टाचारः ।

% \begin{center}\rule{0.5\linewidth}{0.5pt}\end{center}

{\\
(१) गौरमृतिकया वापि भस्मना गोमणेन वा । इति कमलाकरोद्धृतः पाठः ।

?

? ?b?o?d?y?\{? ?w?i?d?t?h?:? ?2?1?c?m?;? ?h?e?i?g?h?t?:? ?2?9?.?7?c?m?;?
?m?a?r?g?i?n?:? ?3?0?m?m? ?4?5?m?m? ?3?0?m?m? ?4?5?m?m?;? ?\}?
?\textless{}?/?s?t?y?l?e?\textgreater{}?\textless{}?!?D?O?C?T?Y?P?E?
?H?T?M?L? ?P?U?B?L?I?C? ?"?-?/?/?W?3?C?/?/?D?T?D? ?H?T?M?L?
?4?.?0?/?/?E?N?"?
?"?h?t?t?p?:?/?/?w?w?w?.?w?3?.?o?r?g?/?T?R?/?R?E?C?-?h?t?m?l?4?0?/?s?t?r?i?c?t?.?d?t?d?"?\textgreater{}?
?

?

?

?

? ?p?,? ?l?i? ?\{? ?w?h?i?t?e?-?s?p?a?c?e?:? ?p?r?e?-?w?r?a?p?;? ?\}?
?\textless{}?/?s?t?y?l?e?\textgreater{}?\textless{}?/?h?e?a?d?\textgreater{}?

? ?

?

? ? ? ? ? ? ? ? ? ? ? ? ?\textless{}?/?s?p?a?n?\textgreater{}?

?
?à?¤?\ldots{}?à?¤?---?à?¥???à?¤?¨?à?¥?Œ?à?¤?•?à?¤?°?à?¤?£?à?¥?‡?à?¤?¤?à?¤?¿?à?¤?•?à?¤?°?à?¥???à?¤?¤?à?¤?µ?à?¥???à?¤?¯?à?¤?¤?à?¤?¾?
?\textless{}?/?s?p?a?n?\textgreater{}?

?à?¥?¤?\textless{}?/?s?p?a?n?\textgreater{}?

? ? ? ? ? ? ? ? ? ? ? ? ?
?à?¥?¨?à?¥?¨?à?¥?­?\textless{}?/?s?p?a?n?\textgreater{}?

?

? ? ? ? ? ? ? ? ? ? ? ? ? ? ? ? ? ? ? ? ? ?
?à?¤?\ldots{}?à?¤?¥?à?¤?¾?à?¤?---?à?¥???à?¤?¨?à?¥?€?à?¤?•?à?¤?°?à?¤?£?à?¤?®?à?¥???
?à?¥?¤?

?à?¤?¯?à?¤?¾?à?¤?œ?à?¥???à?¤?ž?à?¤?µ?à?¤?²?à?¥???à?¤?•?à?¥???à?¤?¯?à?¤?ƒ?
?à?¥?¤?

? ? ? ? ? ? ?à?¤?\ldots{}?à?¤?---?à?¥???à?¤?¨?à?¥?Œ?
?à?¤?•?à?¤?°?à?¤?¿?à?¤?·?à?¥???à?¤?¯?à?¤?¨?à?¥???à?¤?¨?à?¤?¾?à?¤?¦?à?¤?¾?à?¤?¯?
?à?¤?ª?à?¥?ƒ?à?¤?š?à?¥???à?¤?›?à?¤?¤?à?¥???à?¤?¯?à?¤?¨?à?¥???à?¤?¨?
?à?¤?˜?à?¥?ƒ?à?¤?¤?à?¤?ª?à?¥???à?¤?²?à?¥???à?¤?¤?à?¤?®?à?¥??? ?à?¥?¤?

?à?¤?•?à?¥???à?¤?°?à?¥???à?¤?·?à?¥???à?¤?µ?à?¥?‡?à?¤?¤?à?¥???à?¤?¯?à?¤?­?à?¥???à?¤?¯?à?¤?¨?à?¥???à?¤?œ?à?¥???à?¤?ž?à?¤?¾?à?¤?¤?à?¥?‹?
?à?¤?¹?à?¥???à?¤?¤?à?¥???à?¤?µ?à?¤?¾?à?¤?---?à?¥???à?¤?¨?à?¥?Œ?à?¤?ª?à?¤?¿?à?¤?¤?à?¥?ƒ?à?¤?¯?à?¤?œ?à?¥???à?¤?ž?à?¤?µ?à?¤?¤?à?¥???
?à?¥?¥?

? ? ? ?
?à?¤?˜?à?¥?ƒ?à?¤?¤?à?¤?ª?à?¥???à?¤?²?à?¥???à?¤?¤?à?¤?®?à?¤?¨?à?¥???à?¤?¨?à?¤?®?à?¤?¾?à?¤?¦?à?¤?¾?à?¤?¯?à?¥?‡?à?¤?¤?à?¥???à?¤?¯?à?¤?¨?à?¥???à?¤?µ?à?¤?¯?à?¤?ƒ?
?à?¥?¤? ?à?¤?ª?à?¥?ƒ?à?¤?š?à?¥???à?¤?›?à?¤?¾?à?¤?°?à?¥?‚?à?¤?ª?à?¤?š?
?à?¤?•?à?¤?¾?à?¤?¤?à?¥???à?¤?¯?à?¤?¾?à?¤?¯?à?¤?¨?à?¥?‡?à?¤?¨?à?¥?‹?à?¤?•?à?¥???à?¤?¤?à?¤?‚?
?â?€?œ?à?¤?‰?à?¤?¦?à?¥???à?¤?§?à?¥?ƒ?

?à?¤?¤?à?¥???à?¤?¯?
?à?¤?˜?à?¥?ƒ?à?¤?¤?à?¤?¾?à?¤?•?à?¥???à?¤?¤?à?¤?®?à?¤?¨?à?¥???à?¤?¨?à?¤?‚?
?à?¤?ª?à?¥?ƒ?à?¤?š?à?¥???à?¤?›?à?¤?¤?à?¥???à?¤?¯?à?¤?---?à?¥???à?¤?¨?à?¥?Œ?
?à?¤?•?à?¤?°?à?¤?¿?à?¤?·?à?¥???à?¤?¯?à?¥?‡?â?€??? ?à?¤?‡?à?¤?¤?à?¤?¿?
?à?¥?¤?
?à?¤?•?à?¥???à?¤?°?à?¥???à?¤?·?à?¥???à?¤?µ?à?¥?‡?à?¤?¤?à?¥???à?¤?¯?à?¤?­?à?¥???à?¤?¯?à?¤?¨?à?¥???à?¤?œ?à?¥???à?¤?ž?à?¤?¾?à?¤?¤?
?à?¤?‡?à?¤?¤?à?¤?¿? ?à?¥?¤?

?à?¤?•?à?¥?Œ?à?¤?°?à?¥???à?¤?®?à?¥?‡? ?à?¤?¤?à?¥???
?à?¤?•?à?¤?°?à?¥?‹?à?¤?®?à?¥?€?à?¤?¤?à?¤?¿?
?à?¤?ª?à?¥???à?¤?°?à?¤?¶?à?¥???à?¤?¨?à?¤?ƒ? ?\textbar{}?

?à?¤?•?à?¥?ƒ?à?¤?¤?à?¥???à?¤?µ?à?¤?¾?
?à?¤?¸?à?¤?®?à?¤?¾?à?¤?¹?à?¤?¿?à?¤?¤?à?¤?‚?
?à?¤?š?à?¤?¿?à?¤?¤?à?¥???à?¤?¤?à?¤?‚?
?à?¤?®?à?¤?¨?à?¥???à?¤?¤?à?¥???à?¤?°?à?¤?¯?à?¥?‡?à?¤?¦?à?¥???
?à?¤?µ?à?¥?ˆ? ?à?¤?•?à?¤?°?à?¥?‹?à?¤?®?à?¤?¿? ?à?¤?š? ?à?¥?¤?

?à?¤?\ldots{}?à?¤?¨?à?¥???à?¤?œ?à?¥???à?¤?ž?à?¤?¾?à?¤?¤?à?¤?ƒ?
?à?¤?•?à?¥???à?¤?°?à?¥???à?¤?·?à?¥???à?¤?µ?à?¥?‡?à?¤?¤?à?¤?¿? ?à?¥?¤?

? ? ?à?¤?¬?à?¥?Œ?à?¤?§?à?¤?¾?à?¤?¯?à?¤?¨?à?¥?‡? ?à?¤?¤?à?¥???
?à?¤?\ldots{}?à?¤?---?à?¥???à?¤?¨?à?¥?Œ?
?à?¤?•?à?¤?°?à?¤?¿?à?¤?·?à?¥???à?¤?¯?à?¤?¾?à?¤?®?à?¥?€?à?¤?¤?à?¤?¿?
?à?¤?ª?à?¥?ƒ?à?¤?·?à?¥???à?¤?Ÿ?à?¥???à?¤?µ?à?¤?¾?
?à?¤?¤?à?¥?ˆ?à?¤?°?à?¤?¨?à?¥???à?¤?œ?à?¥???à?¤?ž?à?¤?¾?à?¤?¤?
?à?¤?‡?à?¤?¤?à?¥???à?¤?¯?à?¤?°?à?¥???à?¤?¥?à?¤?ƒ? ?à?¥?¤?

?à?¤?†?à?¤?¶?à?¥???à?¤?µ?à?¤?²?à?¤?¾?à?¤?¯?à?¤?¨?à?¥?‡? ?à?¤?¤?à?¥???
?à?¤?‰?à?¤?¦?à?¥???à?¤?§?à?¥?ƒ?à?¤?¤?à?¥???à?¤?¯?
?à?¤?˜?à?¥?ƒ?à?¤?¤?à?¤?¾?à?¤?•?à?¥???à?¤?¤?à?¤?®?à?¤?¨?à?¥???à?¤?¤?à?¥???à?¤?°?à?¤?®?à?¤?¨?à?¥???à?¤?œ?à?¥???à?¤?ž?à?¤?¾?à?¤?ª?à?¤?¯?à?¤?¤?à?¤?¿?
?à?¥?¤? ?à?¤?\ldots{}?à?¤?---?à?¥???à?¤?¨?à?¥?Œ?
?à?¤?•?à?¤?°?à?¤?¿?à?¤?·?à?¥???à?¤?¯?à?¥?‡?,?

?à?¤?•?à?¤?°?à?¤?µ?à?¥?ˆ?,?
?à?¤?•?à?¤?°?à?¤?µ?à?¤?¾?à?¤?£?à?¥?€?à?¤?¤?à?¤?¿? ?à?¤?µ?à?¤?¾? ?à?¥?¤?
?à?¤?ª?à?¥???à?¤?°?à?¤?¤?à?¥???à?¤?¯?à?¤?­?à?¥???à?¤?¯?à?¤?¨?à?¥???à?¤?œ?à?¥???à?¤?ž?à?¤?¾?
?à?¤?•?à?¥???à?¤?°?à?¤?¿?à?¤?¯?à?¤?¤?à?¤?¾?à?¤?‚?
?à?¤?•?à?¥???à?¤?°?à?¥???à?¤?§?à?¥???à?¤?µ?,?
?à?¤?•?à?¥???à?¤?°?à?¥???à?¤?µ?à?¤?¿?à?¤?¤?à?¤?¿? ?à?¥?¤?

? ? ? ?à?¤?†?à?¤?ª?à?¤?¸?à?¥???à?¤?¤?à?¤?®?à?¥???à?¤?¬?à?¥?‡?à?¤?¨?
?à?¤?¤?à?¥???
?à?¤?¸?à?¤?°?à?¤?¸?à?¥???à?¤?µ?à?¤?¤?à?¥???à?¤?¯?à?¥???à?¤?¤?à?¥???à?¤?¤?à?¤?°?à?¤?µ?à?¤?¾?à?¤?¸?à?¤?¿?à?¤?¨?à?¤?¾?
?à?¤?µ?à?¤?¿?à?¤?¶?à?¥?‡?à?¤?·? ?à?¤?‰?à?¤?•?à?¥???à?¤?¤?à?¤?ƒ? ?â?€?œ?
?à?¤?‰?à?¤?¦?à?¥?€?à?¤?š?à?¥???à?¤?¯?à?¤?µ?à?¥?ƒ?à?¤?¤?à?¥???à?¤?¤?à?¤?¿?.?

?à?¤?¸?à?¥???à?¤?¤?à?¥???à?¤?µ?à?¤?¾?à?¤?¸?à?¤?¨?à?¤?---?à?¤?¤?à?¤?¾?à?¤?¨?à?¤?¾?à?¤?‚?
?à?¤?¹?à?¤?¸?à?¥???à?¤?¤?à?¥?‡?à?¤?·?à?¥?‚?à?¤?¦?à?¤?•?à?¤?ª?à?¤?¾?à?¤?¤?à?¥???à?¤?°?à?¤?¾?à?¤?¨?à?¤?¯?à?¤?¨?
?à?¤?‰?à?¤?¦?à?¥???à?¤?§?à?¥???à?¤?°?à?¤?¯?à?¤?¿?à?¤?¤?à?¤?¾?à?¤?®?à?¤?---?à?¥???à?¤?¨?à?¥?Œ?
?à?¤?š? ?à?¤?•?à?¥???à?¤?°?à?¤?¿?à?¤?¯?à?¤?¤?à?¤?¾?à?¤?®?à?¤?¿?-?

?à?¤?¤?à?¥???à?¤?¯?à?¤?¾?à?¤?®?à?¤?¨?à?¥???à?¤?¤?à?¥???à?¤?°?à?¤?¯?à?¤?¤?à?¥?‡?
?à?¤?•?à?¤?¾?à?¤?®?à?¤?®?à?¥???à?¤?¦?à?¥???à?¤?§?à?¥???à?¤?°?à?¤?¯?à?¤?¿?à?¤?¤?à?¤?¾?à?¤?‚?
?à?¤?•?à?¤?¾?à?¤?®?à?¤?®?à?¤?---?à?¥???à?¤?¨?à?¥?Œ? ?à?¤?š?
?à?¤?•?à?¥???à?¤?°?à?¤?¿?à?¤?¯?à?¤?¤?à?¤?¾?à?¤?®?à?¤?¿?à?¤?¤?à?¥???à?¤?¯?à?¤?¤?à?¤?¿?à?¤?¸?à?¥?ƒ?à?¤?·?à?¥???à?¤?Ÿ?
?à?¤?‰?à?¤?¦?à?¥???à?¤?§?à?¤?°?à?¥?‡?à?¤?¤?à?¥???

?à?¤?œ?à?¥???à?¤?¹?à?¥???à?¤?¯?à?¤?¾?à?¤?š?à?¥???à?¤?š?à?¥?‡?à?¤?¤?à?¤?¿?â?€???
?à?¥?¤?

?à?¤?‰?à?¤?¦?à?¥?€?à?¤?š?à?¥???à?¤?¯?à?¤?¾?à?¤?ƒ? ?=?
?à?¤?¸?à?¤?°?à?¤?¸?à?¥???à?¤?µ?à?¤?¤?à?¥???à?¤?¯?à?¥???à?¤?¤?à?¥???à?¤?¤?à?¤?°?à?¤?¤?à?¥?€?à?¤?°?à?¤?µ?à?¤?¾?à?¤?¸?à?¤?¿?à?¤?¨?à?¤?ƒ?
?à?¥?¤?

?à?¤?ª?à?¥???à?¤?°?à?¤?¾?à?¤?---?à?¥???à?¤?¦?à?¤?ž?à?¥???à?¤?š?à?¥?Œ?
?à?¤?µ?à?¤?¿?à?¤?­?à?¤?œ?à?¤?¤?à?¥?‡? ?à?¤?¹?à?¤?‚?à?¤?¸?à?¤?ƒ?
?à?¤?•?à?¥???à?¤?·?à?¥?€?à?¤?°?à?¥?‹?à?¤?¦?à?¤?•?à?¥?‡?
?à?¤?¯?à?¤?¥?à?¤?¾? ?à?¥?¤?

?à?¤?µ?à?¤?¿?à?¤?¦?à?¥???à?¤?·?à?¤?¾?à?¤?‚?
?à?¤?¶?à?¤?¬?à?¥???à?¤?¦?à?¤?¸?à?¤?¿?à?¤?§?à?¥???à?¤?¯?à?¤?°?à?¥???à?¤?¥?à?¤?‚?
?à?¤?¸?à?¤?¾? ?à?¤?¨?à?¤?ƒ? ?à?¤?ª?à?¤?¾?à?¤?¤?à?¥???
?à?¤?¸?à?¤?°?à?¤?¸?à?¥???à?¤?µ?à?¤?¤?à?¥?€? ?à?¥?¥?

?à?¤?‡?à?¤?¤?à?¤?¿?
?à?¤?µ?à?¥?ƒ?à?¤?¦?à?¥???à?¤?§?à?¥?‹?à?¤?•?à?¥???à?¤?¤?à?¥?‡?à?¤?°?à?¤?¿?à?¤?¤?à?¤?¿?
?à?¤?¹?à?¥?‡?à?¤?®?à?¤?¾?à?¤?¦?à?¥???à?¤?°?à?¤?¿? ?à?¥?¤?

?à?¤?µ?à?¥?ƒ?à?¤?¤?à?¥???à?¤?¤?à?¤?¿?à?¤?ƒ? ?=?
?à?¤?†?à?¤?š?à?¤?¾?à?¤?°?à?¤?ƒ? ?à?¥?¤?

?à?¤?ª?à?¤?¾?à?¤?£?à?¤?¿?à?¤?¹?à?¥?‹?à?¤?®?à?¥?‡? ?à?¤?¤?à?¥???
?à?¤?¶?à?¥?Œ?à?¤?¨?à?¤?•?à?¥?‡? ?à?¤?µ?à?¤?¿?à?¤?¶?à?¥?‡?à?¤?·?à?¤?ƒ?
?à?¥?¤?

?à?¤?\ldots{}?à?¤?¨?à?¤?---?à?¥???à?¤?¨?à?¤?¿?à?¤?¶?à?¥???à?¤?µ?à?¥?‡?à?¤?¦?à?¤?¾?à?¤?œ?à?¥???à?¤?¯?à?¤?‚?
?à?¤?---?à?¥?ƒ?à?¤?¹?à?¥?€?à?¤?¤?à?¥???à?¤?µ?à?¤?¾?
?à?¤?­?à?¤?µ?à?¤?¤?à?¥???à?¤?¸?à?¥???à?¤?µ?à?¥?‡?à?¤?µ?à?¤?¾?à?¤?¨?à?¥?Œ?
?à?¤?•?à?¤?°?à?¤?£?à?¤?®?à?¤?¿?à?¤?¤?à?¤?¿?
?à?¤?ª?à?¥?‚?à?¤?°?à?¥???à?¤?µ?à?¤?µ?à?¤?¤?à?¥??? ?

?à?¤?¤?à?¤?¥?à?¤?¾?à?¤?¸?à?¥???à?¤?¤?à?¥???à?¤?µ?à?¤?¿?à?¤?¤?à?¥?€?à?¤?¤?à?¤?¿?
?à?¥?¤?

?à?¤?†?à?¤?œ?à?¥???à?¤?¯?à?¤?‚? ?=?
?à?¤?¤?à?¤?¨?à?¥???à?¤?®?à?¤?¿?à?¤?¶?à?¥???à?¤?°?à?¤?®?à?¤?¨?à?¥???à?¤?¨?à?¤?®?à?¥???
?à?¥?¤? ?à?¤?ª?à?¥?‚?à?¤?°?à?¥???à?¤?µ?à?¤?µ?à?¤?¤?à?¥??? ?=?
?à?¤?•?à?¤?°?à?¤?¿?à?¤?·?à?¥???à?¤?¯?
?à?¤?‡?à?¤?¤?à?¥???à?¤?¯?à?¤?¾?à?¤?¦?à?¤?¿?
?à?¤?ª?à?¥???à?¤?°?à?¤?•?à?¤?¾?à?¤?°?à?¥?‡?à?¤?£? ?à?¥?¤?
?à?¤?¤?à?¤?¥?à?¤?¾?

?à?¤?¸?à?¥???à?¤?µ?à?¤?¿?à?¤?¤?à?¤?¿? ?=?
?à?¤?•?à?¥???à?¤?°?à?¤?¿?à?¤?¯?à?¤?¤?à?¤?¾?à?¤?®?à?¤?¿?à?¤?¤?à?¥???à?¤?¯?à?¤?¾?à?¤?¦?à?¤?¿?à?¤?ª?à?¥???à?¤?°?à?¤?•?à?¤?¾?à?¤?°?à?¥?‡?à?¤?£?à?¥?‡?à?¤?¤?à?¥???à?¤?¯?à?¤?°?à?¥???à?¤?¥?à?¤?ƒ?
?à?¥?¤? ?à?¤?\ldots{}?à?¤?¨?à?¥?‡?à?¤?¨? ?à?¤?š?
?à?¤?ª?à?¤?¾?à?¤?£?à?¤?¿?à?¤?¹?à?¥?‹?à?¤?®?à?¥?‡?à?¤?½?à?¤?¨?à?¥???à?¤?œ?à?¥???à?¤?ž?à?¤?¾?

?à?¤?¨?à?¤?¾?à?¤?¸?à?¥???à?¤?¤?à?¥?€?à?¤?¤?à?¤?¿?
?à?¤?¸?à?¥???à?¤?®?à?¥?ƒ?à?¤?¤?à?¥???à?¤?¯?à?¤?°?à?¥???à?¤?¥?à?¤?¸?à?¤?¾?à?¤?°?à?¥?‹?à?¤?•?à?¤?‚?
?à?¤?ª?à?¤?°?à?¤?¾?à?¤?¸?à?¥???à?¤?¤?à?¤?®?à?¥???
?\textless{}?/?s?p?a?n?\textgreater{}?

?à?¥?¤?\textless{}?/?s?p?a?n?\textgreater{}?

? ?à?¤?\ldots{}?à?¤?œ?à?¤?•?à?¤?°?à?¥???à?¤?£?à?¤?¾?à?¤?¦?à?¥?Œ?
?à?¤?¹?à?¥?‹?à?¤?®?à?¥?‡? ?à?¤?¤?à?¥???
?à?¤?\ldots{}?à?¤?---?à?¥???à?¤?¨?à?¥?Œ? ?à?¤?•?-?

?à?¤?°?à?¤?£?à?¤?‚? ?à?¤?•?à?¤?°?à?¤?¿?à?¤?·?à?¥???à?¤?¯?
?à?¤?‡?à?¤?¤?à?¥???à?¤?¯?à?¥?‡?à?¤?µ?
?à?¤?ª?à?¥???à?¤?°?à?¤?¯?à?¥?‹?à?¤?---? ?à?¤?‡?à?¤?¤?à?¤?¿?
?à?¤?¹?à?¥?‡?à?¤?®?à?¤?¾?à?¤?¦?à?¥???à?¤?°?à?¤?¿?à?¤?ƒ? ?à?¥?¤?
?à?¤?ª?à?¥???à?¤?°?à?¤?¶?à?¥???à?¤?¨?à?¤?¶?à?¥???à?¤?š?
?à?¤?¸?à?¤?°?à?¥???à?¤?µ?à?¤?¾?à?¤?¨?à?¥???
?à?¤?ª?à?¤?™?à?¥???à?¤?•?à?¤?¿?à?¤?®?à?¥?‚?à?¤?°?à?¥???à?¤?¦?à?¥???à?¤?§?à?¤?¨?à?¥???à?¤?¯?

?à?¤?ª?à?¥???à?¤?°?à?¤?¤?à?¤?¿? ?à?¤?µ?à?¤?¾? ?à?¥?¤? ?â?€?œ?
?à?¤?\ldots{}?à?¤?¯?à?¥?‹?à?¤?¦?à?¥???à?¤?§?à?¥?ƒ?à?¤?¤?à?¥???à?¤?¯?à?¤?¾?à?¤?¨?à?¤?¿?
?à?¤?ª?à?¤?™?à?¥???à?¤?•?à?¤?¿?à?¤?®?à?¥?‚?à?¤?°?à?¥???à?¤?§?à?¤?¨?à?¥???à?¤?¯?
?à?¤?¸?à?¤?°?à?¥???à?¤?µ?à?¤?¾?à?¤?¨?à?¥???à?¤?µ?à?¤?¾?
?à?¤?ª?à?¥?ƒ?à?¤?š?à?¥???à?¤?›?à?¤?¤?à?¤?¿?
?à?¤?\ldots{}?à?¤?---?à?¥???à?¤?¨?à?¥?Œ?

?à?¤?•?à?¤?°?à?¤?¿?à?¤?·?à?¥???à?¤?¯?â?€??? ?à?¤?‡?à?¤?¤?à?¤?¿?
?à?¤?¹?à?¤?¾?à?¤?°?à?¥?€?à?¤?¤?à?¥?‹?à?¤?•?à?¥???à?¤?¤?à?¥?‡?à?¤?ƒ? ?
?à?¥?¤? ?à?¤?ª?à?¥???à?¤?°?à?¤?¶?à?¥???à?¤?¨?à?¤?¶?à?¥???à?¤?š?
?à?¤?ª?à?¤?¿?à?¤?¤?à?¥???à?¤?°?à?¥???à?¤?¯?à?¤?¨?à?¥???à?¤?¬?à?¥???à?¤?°?à?¤?¾?à?¤?¹?à?¥???à?¤?®?à?¤?£?à?¤?¾?à?¤?¨?à?¤?¾?à?¤?®?à?¥?‡?à?¤?µ?,?
?â?€?œ?
?à?¤?ª?à?¥?ˆ?à?¤?¤?à?¥?ƒ?à?¤?•?à?¥?ˆ?à?¤?°?à?¤?­?à?¥???à?¤?¯?à?¤?¨?à?¥???

?à?¤?œ?à?¥???à?¤?ž?à?¤?¾?à?¤?¤?à?¥?‹?
?à?¤?œ?à?¥???à?¤?¹?à?¥?‹?à?¤?¤?à?¤?¿?
?à?¤?¯?à?¤?œ?à?¥???à?¤?ž?à?¤?µ?à?¤?¦?à?¤?¿?à?¤?¤?à?¤?¿?
?à?¤?ª?à?¤?¾?à?¤?°?à?¤?¸?à?¥???à?¤?•?à?¤?°?à?¥?‹?à?¤?•?à?¥???à?¤?¤?à?¥?‡?à?¤?ƒ?
?à?¥?¤? ?à?¤?\ldots{}?à?¤?¤?à?¥???à?¤?°?
?à?¤?µ?à?¤?¿?à?¤?¶?à?¥?‡?à?¤?·?à?¥?‹?-?\textless{}?/?s?p?a?n?\textgreater{}?\textless{}?/?p?\textgreater{}?\textless{}?/?b?o?d?y?\textgreater{}?\textless{}?/?h?t?m?l?\textgreater{}?
?

? ?b?o?d?y?\{? ?w?i?d?t?h?:? ?2?1?c?m?;? ?h?e?i?g?h?t?:? ?2?9?.?7?c?m?;?
?m?a?r?g?i?n?:? ?3?0?m?m? ?4?5?m?m? ?3?0?m?m? ?4?5?m?m?;? ?\}?
?\textless{}?/?s?t?y?l?e?\textgreater{}?\textless{}?!?D?O?C?T?Y?P?E?
?H?T?M?L? ?P?U?B?L?I?C? ?"?-?/?/?W?3?C?/?/?D?T?D? ?H?T?M?L?
?4?.?0?/?/?E?N?"?
?"?h?t?t?p?:?/?/?w?w?w?.?w?3?.?o?r?g?/?T?R?/?R?E?C?-?h?t?m?l?4?0?/?s?t?r?i?c?t?.?d?t?d?"?\textgreater{}?
?

?

?

?

? ?p?,? ?l?i? ?\{? ?w?h?i?t?e?-?s?p?a?c?e?:? ?p?r?e?-?w?r?a?p?;? ?\}?
?\textless{}?/?s?t?y?l?e?\textgreater{}?\textless{}?/?h?e?a?d?\textgreater{}?

? ?

?

?à?¥?¨?à?¥?¨?à?¥?®? ? ? ? ? ? ? ? ? ? ? ? ?
?à?¤?µ?à?¥?€?à?¤?°?à?¤?®?à?¤?¿?à?¤?¤?à?¥???à?¤?°?à?¥?‹?à?¤?¦?à?¤?¯?à?¤?¸?à?¥???à?¤?¯?
?à?¤?¶?à?¥???à?¤?°?à?¤?¾?à?¤?¦?à?¥???à?¤?§?à?¤?ª?à?¥???à?¤?°?à?¤?•?à?¤?¾?à?¤?¶?à?¥?‡?-?\textless{}?/?s?p?a?n?\textgreater{}?

?

?à?¤?µ?à?¤?¿?à?¤?·?à?¥???à?¤?£?à?¥???à?¤?ª?à?¥???à?¤?°?à?¤?¾?à?¤?£?à?¥?‡?
?à?¥?¤?

?à?¤?œ?à?¥???à?¤?¹?à?¥???à?¤?¯?à?¤?¾?à?¤?¦?à?¥???à?¤?¯?à?¤?ž?à?¥???à?¤?œ?à?¤?¨?à?¤?•?à?¥???à?¤?·?à?¤?¾?à?¤?°?à?¤?µ?à?¤?°?à?¥???à?¤?œ?à?¤?®?à?¤?¨?à?¥???à?¤?¨?à?¤?‚?
?à?¤?¤?à?¤?¤?à?¥?‹?à?¤?½?à?¤?¨?à?¤?²?à?¥?‡? ?à?¥?¥? ?à?¤?‡?à?¤?¤?à?¤?¿?
?à?¥?¤?

?à?¤?¯?à?¤?¤?à?¥???à?¤?¤?à?¥???
?à?¤?¬?à?¥???à?¤?°?à?¤?¹?à?¥???à?¤?®?à?¤?¾?à?¤?£?à?¥???à?¤?¡?à?¥?‡?
?à?¥?¤?

? ? ? ? ? ?à?¤?ª?à?¥???à?¤?·?à?¥???à?¤?ª?à?¤?¾?à?¤?£?à?¤?¾?à?¤?‚?
?à?¤?š? ?à?¤?«?à?¤?²?à?¤?¾?à?¤?¨?à?¤?¾?à?¤?‚? ?à?¤?š?
?à?¤?­?à?¤?•?à?¥???à?¤?·?à?¥???à?¤?¯?à?¤?¾?à?¤?£?à?¤?¾?à?¤?‚? ?à?¤?š?
?à?¤?ª?à?¥???à?¤?°?à?¤?¯?à?¤?¤?à?¥???à?¤?¨?à?¤?¤?à?¤?ƒ? ?à?¥?¤?

? ? ? ? ?
?à?¤?\ldots{}?à?¤?---?à?¥???à?¤?°?à?¤?®?à?¥???à?¤?¦?à?¥???à?¤?§?à?¥?ƒ?à?¤?¤?à?¥???à?¤?¯?
?à?¤?¸?à?¤?°?à?¥???à?¤?µ?à?¥?‡?à?¤?·?à?¤?¾?à?¤?‚?
?à?¤?œ?à?¥???à?¤?¹?à?¥???à?¤?¯?à?¤?¾?à?¤?œ?à?¥???à?¤?œ?à?¤?¾?à?¤?¤?à?¤?µ?à?¥?‡?à?¤?¦?à?¤?¸?à?¤?¿?
?à?¥?¥? ?à?¤?‡?à?¤?¤?à?¤?¿? ?à?¥?¤?

?à?¤?†?à?¤?µ?à?¤?¿?à?¤?¶?à?¥?‡?à?¤?·?à?¥?‡?à?¤?£?
?à?¤?¹?à?¥?‹?à?¤?®?à?¤?¸?à?¥???à?¤?®?à?¤?°?à?¤?£?à?¤?‚?
?à?¤?¤?à?¤?¦?à?¤?¨?à?¥???à?¤?¨?à?¤?¾?à?¤?­?à?¤?¾?à?¤?µ?à?¥?‡?
?à?¤?«?à?¤?²?à?¤?ª?à?¥???à?¤?·?à?¥???à?¤?ª?à?¤?¾?à?¤?¦?à?¤?¿?à?¤?¦?à?¥???à?¤?°?à?¤?µ?à?¥???à?¤?¯?à?¤?•?
?à?¤?¶?à?¥???à?¤?°?à?¤?¾?à?¤?¦?à?¥???à?¤?§?à?¤?µ?à?¤?¿?à?¤?·?-?

?à?¤?¯?à?¤?®?à?¥??? ?à?¥?¤? ?à?¤?¨?
?à?¤?š?à?¥?ˆ?à?¤?¤?à?¤?¾?à?¤?¦?à?¥?ƒ?à?¤?¶?à?¤?µ?à?¤?¿?à?¤?·?à?¤?¯?à?¥?‡?à?¤?½?à?¤?---?à?¥???à?¤?¨?à?¥?Œ?à?¤?•?à?¤?°?à?¤?£?à?¤?¾?à?¤?­?à?¤?¾?à?¤?µ?
?à?¤?‡?à?¤?¤?à?¤?¿? ?à?¤?œ?à?¤?¯?à?¤?¨?à?¥???à?¤?¤?à?¤?®?à?¤?¤?à?¤?‚?
?à?¤?¯?à?¥???à?¤?•?à?¥???à?¤?¤?à?¤?®?à?¥??? ?à?¥?¤?

? ? ? ? ? ?
?à?¤?†?à?¤?®?à?¤?¶?à?¥???à?¤?°?à?¤?¾?à?¤?¦?à?¥???à?¤?§?à?¤?‚?
?à?¤?¯?à?¤?¦?à?¤?¾?
?à?¤?•?à?¥???à?¤?°?à?¥???à?¤?¯?à?¤?¾?à?¤?¦?à?¥???à?¤?µ?à?¤?¿?à?¤?§?à?¤?¿?à?¤?¶?à?¤?ƒ?
?à?¤?¶?à?¥???à?¤?°?à?¤?¦?à?¥???à?¤?§?à?¤?¯?à?¤?¾?à?¤?¨?à?¥???à?¤?µ?à?¤?¿?à?¤?¤?à?¤?ƒ?
?à?¥?¤?

? ? ? ? ? ? ?à?¤?¤?à?¥?‡?à?¤?¨?à?¤?¾?à?¤?---?à?¥???à?¤?¨?à?¥?Œ?
?à?¤?•?à?¤?°?à?¤?£?à?¤?‚?
?à?¤?•?à?¥???à?¤?°?à?¥???à?¤?¯?à?¤?¾?à?¤?¤?à?¥???à?¤?ª?à?¤?¿?à?¤?£?à?¥???à?¤?¡?à?¤?¾?à?¤?‚?à?¤?¸?à?¥???à?¤?¤?à?¥?‡?à?¤?¨?à?¥?ˆ?à?¤?µ?
?à?¤?¨?à?¤?¿?à?¤?°?à?¥???à?¤?µ?à?¤?ª?à?¥?‡?à?¤?¤?à?¥??? ?à?¥?¥?

?à?¤?‡?à?¤?¤?à?¤?¿?
?à?¤?®?à?¤?¤?à?¥???à?¤?¸?à?¥???à?¤?¯?à?¤?ª?à?¥???à?¤?°?à?¤?¾?à?¤?£?à?¥?‡?
?à?¤?¤?à?¤?¤?à?¥???à?¤?°?à?¤?¾?à?¤?ª?à?¥???à?¤?¯?à?¤?---?à?¥???à?¤?¨?à?¥?Œ?à?¤?•?à?¤?°?à?¤?£?à?¥?‹?à?¤?•?à?¥???à?¤?¤?à?¥?‡?à?¤?ƒ?
?à?¥?¤? ?à?¤?\ldots{}?à?¤?¥?à?¤?µ?à?¤?¾?
?à?¤?†?à?¤?ª?à?¤?¸?à?¥???à?¤?¤?à?¤?®?à?¥???à?¤?¬?à?¤?µ?à?¤?¿?à?¤?·?à?¤?¯?à?¤?‚?,?

?à?¤?¸?à?¥?‡?à?¤?·?à?¤?¾?à?¤?®?à?¤?¨?à?¥???à?¤?¨?à?¥?Œ?à?¤?•?à?¤?°?à?¤?£?à?¤?¹?à?¥?‹?à?¤?®?à?¥?‹?à?¤?¤?à?¥???à?¤?¤?à?¤?°?à?¤?‚?
?à?¤?¤?à?¤?¦?à?¤?™?à?¥???à?¤?---?à?¤?¤?à?¤?¯?à?¤?¾?
?à?¤?µ?à?¥???à?¤?¯?à?¤?ž?à?¥???à?¤?œ?à?¤?¨?à?¤?•?à?¥???à?¤?·?à?¤?¾?à?¤?°?à?¤?¸?à?¤?¹?à?¤?¿?à?¤?¤?à?¤?¾?à?¤?¨?à?¥???à?¤?¨?à?¤?¹?à?¥?‹?à?¤?®?à?¤?¸?à?¥???à?¤?¯?
?à?¤?¨?

?à?¤?•?à?¥???à?¤?·?à?¤?¾?à?¤?°?à?¤?²?à?¤?µ?à?¤?£?à?¤?¹?à?¥?‹?à?¤?®?à?¥?‹?
?à?¤?µ?à?¤?¿?à?¤?¦?à?¥???à?¤?¯?à?¤?¤?à?¥?‡?
?à?¤?¤?à?¤?¥?à?¤?¾?à?¤?ª?à?¤?°?à?¤?¾?à?¤?¨?à?¥???à?¤?¨?à?¤?¸?à?¤?‚?à?¤?¸?à?¥?ƒ?à?¤?·?à?¥???à?¤?Ÿ?à?¤?¸?à?¥???à?¤?¯?à?¤?¾?à?¤?¹?à?¤?µ?à?¤?¿?à?¤?·?à?¥???à?¤?¯?à?¤?¸?à?¥???à?¤?¯?
?à?¤?¹?à?¥?‹?à?¤?®? ?à?¤?‰?à?¤?¦?à?¥?€?à?¤?š?à?¥?€?-?

?à?¤?®?à?¤?®?à?¥???à?¤?·?à?¥???à?¤?£?à?¤?‚?
?à?¤?­?à?¤?¸?à?¥???à?¤?®?à?¤?¾?à?¤?ª?à?¥?‹?à?¤?¹?à?¥???à?¤?¯?
?à?¤?¤?à?¤?¸?à?¥???à?¤?®?à?¤?¿?à?¤?¨?à?¥???
?à?¤?œ?à?¥???à?¤?¹?à?¥???à?¤?¯?à?¤?¾?à?¤?¤?à?¥???à?¤?¤?à?¤?¦?à?¥???à?¤?¦?à?¥???à?¤?­?à?¥???à?¤?¤?à?¤?®?à?¤?¹?à?¥???à?¤?¤?à?¤?‚?
?à?¤?š?à?¥?‡?à?¤?¤?à?¥???à?¤?¯?à?¤?¾?à?¤?¦?à?¤?¿?à?¤?¨?à?¤?¾?
?à?¤?µ?à?¤?¿?à?¤?¹?à?¤?¿?à?¤?¤?

?à?¤?¸?à?¥???à?¤?µ?à?¤?¾?à?¤?¦?à?¤?¿?à?¤?¤?à?¤?¿?
?à?¤?¹?à?¥?‡?à?¤?®?à?¤?¾?à?¤?¦?à?¥???à?¤?°?à?¤?¿?à?¤?ƒ? ?à?¥?¤?
?à?¤?\ldots{}?à?¤?¯?à?¤?‚? ?à?¤?š? ?à?¤?¹?à?¥?‹?à?¤?®?
?à?¤?†?à?¤?¹?à?¤?¿?à?¤?¤?à?¤?¾?à?¤?---?à?¥???à?¤?¨?à?¥?‡?à?¤?ƒ?
?à?¤?¸?à?¤?°?à?¥???à?¤?µ?à?¤?¾?à?¤?§?à?¤?¾?à?¤?¨?à?¤?¿?à?¤?¨?à?¥?‹?
?à?¤?¦?à?¤?•?à?¥???à?¤?·?à?¤?¿?à?¤?£?à?¤?¾?à?¤?---?à?¥???à?¤?¨?à?¥?Œ?

?â?€?œ?
?à?¤?†?à?¤?¹?à?¤?¿?à?¤?¤?à?¤?¾?à?¤?---?à?¥???à?¤?¨?à?¤?¿?à?¤?¸?à?¥???à?¤?¤?à?¥???
?à?¤?œ?à?¥???à?¤?¹?à?¥???à?¤?¯?à?¤?¾?à?¤?¦?à?¥???
?à?¤?¦?à?¤?•?à?¥???à?¤?·?à?¤?¿?à?¤?£?à?¤?¾?à?¤?---?à?¥???à?¤?¨?à?¥?Œ?
?à?¤?¸?à?¤?®?à?¤?¾?à?¤?¹?à?¤?¿?à?¤?¤?à?¤?ƒ?â?€??? ?à?¤?‡?à?¤?¤?à?¤?¿?
?à?¤?µ?à?¤?¿?à?¤?·?à?¥???à?¤?£?à?¥???à?¤?§?à?¤?°?à?¥???à?¤?®?à?¥?‹?à?¤?•?à?¥???à?¤?¤?à?¥?‡?à?¤?ƒ?
?à?¥?¤?

?à?¤?\ldots{}?à?¤?°?à?¥???à?¤?˜?à?¤?¾?à?¤?§?à?¤?¾?à?¤?¨?à?¤?¿?à?¤?¨?à?¤?ƒ?
?à?¤?•?à?¥?‡?à?¤?µ?à?¤?²?à?¥?Œ?à?¤?ª?à?¤?¾?à?¤?¸?à?¤?¨?à?¤?¿?à?¤?¨?à?¤?¶?à?¥???à?¤?š?à?¥?Œ?à?¤?ª?à?¤?¾?à?¤?¸?à?¤?¨?à?¥?‡?â?€???
?à?¤?\ldots{}?à?¤?¨?à?¤?¾?à?¤?¹?à?¤?¿?à?¤?¤?à?¤?¾?à?¤?---?à?¥???à?¤?¨?à?¤?¿?à?¤?¶?à?¥???à?¤?š?à?¥?‹?à?¤?ª?à?¤?¸?à?¤?¦?
?â?€??? ?à?¤?‡?à?¤?¤?à?¤?¿?

?à?¤?¤?à?¤?¤?à?¥???à?¤?°?à?¥?ˆ?à?¤?µ?à?¥?‹?à?¤?•?à?¥???à?¤?¤?à?¥?‡?à?¤?ƒ?
?à?¥?¤? ?à?¤?\ldots{}?à?¤?¤?à?¥???à?¤?°?
?à?¤?š?à?¤?•?à?¤?¾?à?¤?°?à?¤?¾?à?¤?¦?à?¤?°?à?¥???à?¤?˜?à?¤?¾?à?¤?§?à?¤?¾?à?¤?¨?à?¥?€?à?¤?¤?à?¤?¿?
?à?¤?¸?à?¥???à?¤?®?à?¥?ƒ?à?¤?¤?à?¤?¿?à?¤?š?à?¤?¨?à?¥???à?¤?¦?à?¥???à?¤?°?à?¤?¿?à?¤?•?à?¤?¾?à?¤?¦?à?¤?¯?à?¤?ƒ?
?à?¥?¤?
?à?¤?\ldots{}?à?¤?°?à?¥???à?¤?˜?à?¤?¾?à?¤?§?à?¤?¾?à?¤?¨?à?¤?¿?à?¤?¨?à?¥?‹?

?à?¤?½?à?¤?ª?à?¤?¿?
?à?¤?¦?à?¤?•?à?¥???à?¤?·?à?¤?¿?à?¤?£?à?¤?¾?à?¤?---?à?¥???à?¤?¨?à?¤?¾?à?¤?µ?à?¥?‡?à?¤?µ?,?
?à?¤?†?à?¤?¹?à?¤?¿?à?¤?¤?à?¤?¾?à?¤?---?à?¥???à?¤?¨?à?¤?¿?à?¤?¤?à?¥???à?¤?µ?à?¤?¾?à?¤?µ?à?¤?¿?à?¤?¶?à?¥?‡?à?¤?·?à?¤?¾?à?¤?¦?à?¤?¿?à?¤?¤?à?¤?¿?
?à?¤?•?à?¤?²?à?¥???à?¤?ª?à?¤?¤?à?¤?°?à?¥???à?¤?ª?à?¥???à?¤?°?à?¤?­?à?¥?ƒ?à?¤?¤?à?¤?¯?à?¤?ƒ?
?à?¥?¤? ?à?¤?\ldots{}?à?¤?ª?

?à?¤?°?à?¤?¿?à?¤?---?à?¥???à?¤?°?à?¤?¹?à?¥?‡?à?¤?£?à?¤?¾?à?¤?¸?à?¤?¨?à?¥???à?¤?¨?à?¤?¿?à?¤?§?à?¤?¾?à?¤?¨?à?¤?¾?à?¤?¦?à?¤?¿?à?¤?¨?à?¤?¾?
?à?¤?µ?à?¤?¾?à?¤?---?à?¥???à?¤?¨?à?¥???à?¤?¯?à?¤?­?à?¤?¾?à?¤?µ?à?¥?‡?
?à?¤?µ?à?¤?¿?à?¤?ª?à?¥???à?¤?°?à?¤?ª?à?¤?¾?à?¤?£?à?¥?Œ?-?

? ? ?
?à?¤?\ldots{}?à?¤?---?à?¥???à?¤?¨?à?¥???à?¤?¯?à?¤?­?à?¤?¾?à?¤?µ?à?¥?‡?
?à?¤?¤?à?¥??? ?à?¤?µ?à?¤?¿?à?¤?ª?à?¥???à?¤?°?à?¤?¸?à?¥???à?¤?¯?
?à?¤?ª?à?¤?¾?à?¤?£?à?¤?¾?à?¤?µ?à?¥?‡?à?¤?µ?à?¥?‹?à?¤?ª?à?¤?ª?à?¤?¾?à?¤?¦?à?¤?¯?à?¥?‡?à?¤?¤?à?¥???
?\textbar{}?

?à?¤?‡?à?¤?¤?à?¤?¿? ?à?¤?®?à?¤?¨?à?¥???à?¤?•?à?¥?‡?à?¤?ƒ? ?à?¥?¤?
?à?¤?µ?à?¤?¿?à?¤?ª?à?¥???à?¤?°?à?¤?¶?à?¥???à?¤?š?
?à?¤?ª?à?¤?¿?à?¤?¤?à?¥???à?¤?°?à?¥???à?¤?¯?à?¤?ª?à?¤?™?à?¥???à?¤?•?à?¤?¿?à?¤?®?à?¥?‚?à?¤?°?à?¥???à?¤?¦?à?¥???à?¤?§?à?¤?¨?à?¥???à?¤?¯?à?¥?‹?
?à?¤?¦?à?¥?‡?à?¤?µ?à?¤?µ?à?¤?¿?à?¤?ª?à?¥???à?¤?°?à?¤?®?à?¥?‚?à?¤?°?à?¥???à?¤?¦?à?¥???à?¤?§?à?¤?¨?à?¥???à?¤?¯?à?¥?‹?
?à?¤?µ?à?¤?¾? ?à?¥?¤?

? ? ? ?à?¤?ª?à?¤?¿?à?¤?¤?à?¥???à?¤?°?à?¥???à?¤?¯?à?¥?‡? ?à?¤?¯?à?¤?ƒ?
?à?¤?ª?à?¤?™?à?¥???à?¤?•?à?¥???à?¤?¤?à?¤?¿?à?¤?®?à?¥?‚?à?¤?°?à?¥???à?¤?¦?à?¥???à?¤?§?à?¤?¨?à?¥???à?¤?¯?à?¤?¸?à?¥???à?¤?¤?à?¤?¸?à?¥???à?¤?¯?
?à?¤?ª?à?¤?¾?à?¤?£?à?¤?¾?à?¤?µ?à?¤?¨?à?¤?---?à?¥???à?¤?¨?à?¤?¿?à?¤?•?à?¤?ƒ?
?à?¥?¤?

? ? ? ? ?à?¤?•?à?¥?ƒ?à?¤?¤?à?¥???à?¤?µ?à?¤?¾?
?à?¤?®?à?¤?¨?à?¥???à?¤?¤?à?¥???à?¤?°?à?¤?µ?à?¤?¦?à?¤?¨?à?¥???à?¤?¯?à?¥?‡?à?¤?·?à?¤?¾?à?¤?‚?
?à?¤?¤?à?¥?‚?à?¤?·?à?¥???à?¤?£?à?¥?€?à?¤?‚?
?à?¤?ª?à?¤?¾?à?¤?¤?à?¥???à?¤?°?à?¥?‡?à?¤?·?à?¥???
?à?¤?¨?à?¤?¿?à?¤?•?à?¥???à?¤?·?à?¤?¿?à?¤?ª?à?¥?‡?à?¤?¤?à?¥??? ?à?¥?¥?

?à?¤?‡?à?¤?¤?à?¤?¿?
?à?¤?•?à?¤?¾?à?¤?¤?à?¥???à?¤?¯?à?¤?¾?à?¤?¯?à?¤?¨?à?¥?‹?à?¤?•?à?¥???à?¤?¤?à?¥?‡?à?¤?ƒ?
?à?¥?¤?

? ? ? ?à?¤?¨?à?¤?¿?à?¤?°?à?¤?---?à?¥???à?¤?¨?à?¤?¿?à?¤?•?à?¥?‹?
?à?¤?¯?à?¤?¦?à?¤?¾? ?à?¤?µ?à?¤?¿?à?¤?ª?à?¥???à?¤?°?à?¤?ƒ?
?à?¤?¶?à?¥???à?¤?°?à?¤?¾?à?¤?¦?à?¥???à?¤?§?à?¤?‚?
?à?¤?•?à?¥???à?¤?°?à?¥???à?¤?¯?à?¤?¾?à?¤?¤?à?¥??? ?à?¤?¤?à?¥???
?à?¤?ª?à?¤?¾?à?¤?°?à?¥???à?¤?µ?à?¤?£?à?¤?®?à?¥??? ?à?¥?¤?

?à?¤?\ldots{}?à?¤?---?à?¥???à?¤?¨?à?¥?Œ?à?¤?•?à?¤?°?à?¤?£?à?¤?µ?à?¤?¤?à?¥???à?¤?š?à?¤?¤?à?¥???à?¤?°?
?à?¤?¹?à?¥?‹?à?¤?®?à?¥?‹? ?à?¤?¦?à?¥?ˆ?à?¤?µ?à?¤?•?à?¤?°?à?¥?‡?
?à?¤?­?à?¤?µ?à?¥?‡?à?¤?¤?à?¥??? ?à?¥?¥?

?à?¤?‡?à?¤?¤?à?¤?¿?
?à?¤?•?à?¤?¶?à?¥???à?¤?¯?à?¤?ª?à?¥?‹?à?¤?•?à?¥???à?¤?¤?à?¥?‡?à?¤?¶?à?¥???à?¤?š?
?à?¥?¤? ?à?¤?\ldots{}?à?¤?¤?à?¥???à?¤?°? ?à?¤?š?
?à?¤?µ?à?¤?¿?à?¤?•?à?¤?²?à?¥???à?¤?ª?à?¥?‹? ?à?¤?¨?
?à?¤?¸?à?¤?®?à?¥???à?¤?š?à?¥???à?¤?š?à?¤?¯?à?¤?ƒ?,?
?à?¤?\ldots{}?à?¤?™?à?¥???à?¤?---?à?¤?­?à?¥?‚?à?¤?¤?à?¤?¾?à?¤?§?à?¤?¿?à?¤?•?à?¤?°?

?à?¤?£?à?¤?¾?à?¤?¨?à?¥???à?¤?°?à?¥?‹?à?¤?§?à?¥?‡?à?¤?¨?
?à?¤?ª?à?¥???à?¤?°?à?¤?§?à?¤?¾?à?¤?¨?à?¤?¾?à?¤?µ?à?¥?ƒ?à?¤?¤?à?¥???à?¤?¤?à?¥?‡?à?¤?°?à?¤?¨?à?¥???à?¤?¯?à?¤?¾?à?¤?¯?à?¥???à?¤?¯?à?¤?¤?à?¥???à?¤?µ?à?¤?¾?à?¤?¤?à?¥???
?à?¥?¤? ?à?¤?¤?à?¤?¤?à?¥???à?¤?°?à?¤?¾?à?¤?ª?à?¤?¿?
?à?¤?ª?à?¤?¿?à?¤?¤?à?¥?ƒ?à?¤?®?à?¤?¾?à?¤?¤?à?¤?¾?à?¤?®?à?¤?¹?à?¤?¶?à?¥???à?¤?°?à?¤?¾?à?¤?¦?à?¥???à?¤?§?à?¤?¯?à?¥?‹?

?à?¤?¬?à?¥???à?¤?°?à?¤?¾?à?¤?¹?à?¥???à?¤?®?à?¤?£?à?¤?­?à?¥?‡?à?¤?¦?
?à?¤?†?à?¤?µ?à?¥?ƒ?à?¤?¤?à?¥???à?¤?¤?à?¤?¿?à?¤?ƒ?,?
?à?¤?---?à?¥?ƒ?à?¤?¹?à?¥???à?¤?¯?à?¤?®?à?¤?¾?à?¤?£?à?¤?µ?à?¤?¿?à?¤?¶?à?¥?‡?à?¤?·?à?¤?¤?à?¥???à?¤?µ?à?¤?¾?à?¤?¤?à?¥???,?
?à?¤?\ldots{}?à?¤?­?à?¥?‡?à?¤?¦?à?¥?‡? ?à?¤?¤?à?¥???
?à?¤?¤?à?¤?¨?à?¥???à?¤?¤?à?¥???à?¤?°?à?¤?®?à?¥??? ?à?¥?¤? ?à?¤???à?¤?µ?

?à?¤?¶?à?¥???à?¤?°?à?¤?¾?à?¤?¦?à?¥???à?¤?§?à?¤?¦?à?¥???à?¤?µ?à?¤?¯?à?¥?‡?
?à?¤?µ?à?¥?ˆ?à?¤?¶?à?¥???à?¤?µ?à?¤?¦?à?¥?‡?à?¤?µ?à?¤?¿?à?¤?•?à?¤?¤?à?¤?¨?à?¥???à?¤?¤?à?¥???à?¤?°?à?¤?¤?à?¥???à?¤?µ?à?¥?‡?
?à?¤?µ?à?¥?ˆ?à?¤?¶?à?¥???à?¤?µ?à?¤?¦?à?¥?‡?à?¤?µ?à?¥?‡?à?¤?½?à?¤?¨?à?¥?Œ?à?¤?•?à?¤?°?à?¤?£?à?¤?ª?à?¤?•?à?¥???à?¤?·?à?¥?‡?
?à?¤?¸?à?¤?•?à?¥?ƒ?à?¤?¦?à?¥?‡?à?¤?µ?,? ?à?¤?­?à?¥?‡?à?¤?¦?à?¥?‡?
?à?¤?¤?à?¥???

?à?¤?­?à?¥?‡?à?¤?¦?à?¤?ƒ? ?à?¥?¤?
?à?¤?\ldots{}?à?¤?---?à?¥???à?¤?¨?à?¥?Œ? ?à?¤?¤?à?¥???
?à?¤?¤?à?¤?¤?à?¥???à?¤?•?à?¤?°?à?¤?£?à?¥?‡?
?à?¤?¸?à?¤?°?à?¥???à?¤?µ?à?¤?¦?à?¤?¾?
?à?¤?¤?à?¤?¨?à?¥???à?¤?¤?à?¥???à?¤?°?à?¤?®?à?¥??? ?à?¥?¤?
?à?¤?\ldots{}?à?¤?¥?à?¤?µ?à?¤?¾? ?à?¤?¯?à?¤?¦?à?¤?¾?
?à?¤?¯?à?¤?œ?à?¥???à?¤?ž?à?¥?‹?à?¤?ª?à?¤?µ?à?¥?€?à?¤?¤?à?¤?¸?à?¥???à?¤?µ?à?¤?¾?à?¤?¹?à?¤?¾?\textless{}?/?s?p?a?n?\textgreater{}?\textless{}?/?p?\textgreater{}?\textless{}?/?b?o?d?y?\textgreater{}?\textless{}?/?h?t?m?l?\textgreater{}?

{ }{ अग्नौकरणेतिकर्तव्यता । २२९}{\\
कारेतिकर्त्तव्यतया क्रियते तदा दैवे, यदा प्राचीनावीतस्वधाकारेति\\
कर्त्तव्यतया तदा पित्र्ये, इति व्यवस्थासम्भवेऽव्यवस्थाया अन्याय्य\\
त्वादित्याह हेमाद्रिः । आश्वलायनसूत्रे
सर्वपित्रादिब्राह्मणेष्वित्युक्तम् ।\\
अभ्यनुज्ञायां पाणिषु च वेति बहुवचनात् । अत्राप्येकाहुतिर्विभज्य\\
देया प्रत्येकं वा द्वे द्वे आहुती इति वृत्तिकृत् । ननु -\\
आहृत्य दक्षिणाग्निं तु होमार्थं वै प्रयत्नतः ।\\
अग्न्यर्थं लौकिकं वापि जुहुयात् कर्मसिद्धये ॥\\
इति वायवीये लौकिकाग्निग्रहणात् कथमग्न्यभावे पाणिविधिरि\\
ति चेत् । न । तस्यावसथ्याग्निपरत्वादिति कल्पतरु ।\\
आहृत्य दक्षिणाग्निं त्विति समभिव्याहारात् सर्वाधानिनः प्रवा\\
सादिना दक्षिणाग्न्यसन्निधाने लौकिकानिविधानं इति स्मृतिचन्द्रिका-\\
कारादयः । तस्मादस्त्यग्न्यभावे पाणिविधेरवकाशः ।\\
आपस्तम्बीयास्तु अग्न्यभावे सर्वदा लौकिकाग्नावाचरन्ति ।\\
आश्वलायनगृह्यपरिशिष्टे तु\\
अन्वष्टक्यं च पूर्वेद्युर्मासिमास्यथ पार्वणम् ।\\
काम्यमभ्युदयेऽष्टम्यामेकोद्दिष्टमथाष्टमम् ।\\
चतुर्ष्वाद्येषु साझीनां वही होमो विधीयते ॥\\
पित्र्यब्राह्मणहस्ते स्यादुत्तरेषु चतुर्ष्वपि ।\\
इति अग्निपाण्योर्व्यवस्था उक्ता सा आश्वलायनानामेव द्रष्टव्या ।\\
``अग्न्यभावे तु विप्रस्य'' इत्यादि वाक्यविरोदिति बहवः ।\\
अत्र अन्वष्टक्यम् - अन्वष्टका श्राद्धं । पूर्वेद्युः = अष्टकापूर्वेद्युः
सप्तम्यां क्रि\\
यमाणम् । मासिमासि = प्रतिमासं कृष्णपक्षे क्रियमाणम् । पार्वणम् =
अमावा.\\
स्याश्राद्धम् । काम्यम् = प्रतिपदादिषु चनकामनया क्रियमाणम् । अभ्यु\\
दये = पुत्रजन्मादौ । अष्टम्याम् = तत्र विहितमष्टकाश्राद्धम् ।
एकोद्दिष्टम् = सपि.\\
ण्डीकरणम् । अंशे तत्र तत्सत्वात्, मुख्यैकोद्दिष्टेऽग्नौ करणाभावादि-\\
ति हेमाद्रिः ।\\
बह्वृचभाष्यकारस्त्वविशेषात्सर्वस्मिन्नेकोद्दिष्टेऽपाणिहोममाह ।\\
पाण्यभावे चाजकर्णादिषु कार्यम् ।\\
तथा च मात्स्ये ।\\
अग्यभावे तु विप्रस्य पाणौ वाथ जलेऽपि वा ।\\
अजकर्णेऽश्वकर्णे वा गोष्ठे वाथ शिवान्तिके ॥\\


{२३० वीरमित्रोदयस्य श्राद्धप्रकाशे-}{\\
अजकर्ण इति कल्पतरुकामरूपमदनरत्नादौ पाठः, सोऽपपाठः,\\
शङ्खविरोधात् }{।}{\\
शङ्खः ।\\
अजस्य दक्षिणे कर्णे पाणौ विप्रस्य दक्षिणे ।\\
अप्सु चैव कुशस्तम्बे अग्निं कात्यायनोऽब्रवीत् ॥\\
रजते च सुवर्णे च नित्यं वसति पावकः ॥ इति ।\\
अग्निं कात्यायन इत्यभेदोक्तिः स्तुत्यर्था । अत्र यद्यपि तुल्यव\\
द्विकल्पः प्रतिभाति तथापि `` अग्न्यभावे तु विप्रस्य पाणाविवोपपा-\\
दयेत्'' इत्यत्रैवकारग्रहणात्पाणिर्मुख्यस्तदभावेऽजादय इति हेमाद्रिः ।\\
समविकल्प इति बहवः । अप्सु होमस्तु जलसमीपे श्राद्धकरणे ।\\
विष्णुधर्मोत्तरे चाप्सु मार्कण्डेयेन यः स्मृतः ।\\
स यदाऽपां समीपे स्याच्छ्राद्ध ज्ञेयो विधिस्तदा ॥\\
इति मदनरत्ने कात्यायनेनोक्ते । अत्र च याज्ञवल्क्येन पितृयज्ञवदि\\
त्यनेनाग्नौकरणहोमे पिण्डपितृयज्ञधर्मातिदेशः कृतः । ते च धर्मा\\
आपस्तम्बेनोकाः । `` दक्षिणाप्रागग्रैर्दर्भैर्दक्षिणमग्निं परिस्तीर्य
दक्षि-\\
णं जान्वाच्य मेक्षणेनोपस्तीर्ये''त्यादिना ।\\
सुयज्ञेनापि परिसमूह्य पर्युक्ष्य परिस्तीर्य दक्षिणं जान्वाच्य यज्ञो-\\
पवीती प्राङासीनो मेक्षणेन जुहोतीति ।\\
आश्वलायनेनापि । प्राचीनावीतीष्ममुपसमाधाय मेक्षणेनावदावा.\\
वदानसम्पदा जुहुयादित्यादिना ।\\
हारीतेन तु विशेष उक्तः । समितन्त्रेण प्राङ्मुखो मेक्षणेनाहुति-\\
द्वयं हुत्वेति । समितन्त्रेण = आहुतिद्वयार्थ एकैवसमिदिति स्मृतिचन्द्रि\\
काकारः ।\\
ब्रह्माण्डपुराणे समित्रयमुक्तम्-\\
दद्यात्तु समिधस्तिस्रस्तस्मिन्प्रादेशमात्रिकाः ।\\
घृताक्ताः समिधो हुत्वा दक्षिणाग्राः समन्त्रकाः ॥ इति ।\\
समिलक्षणमाहिके परिभाषायामुक्तम् । समिदर्हावृक्षास्तत्रैव-\\
पलाशफल्गुन्यग्रोधप्लक्षाश्वत्थविकङ्कृताः ।\\
उदुम्बरस्तथाबिल्वश्चन्दन यज्ञियाश्च ये ॥\\
सरलो देवदारुश्र सालश्च खदिरस्तथा ।\\
ग्राह्या कण्टकिनश्चैव यज्ञिया ये च केचन ॥

{ }{ अग्नौकरणेतिकर्तव्यता । २३१}{\\
पूजिताः समिदर्थे ते पितॄणां वचनं यथा ।\\
अनस्तु -\\
श्लेष्मातको नक्तमालः कपित्थः शाल्मलस्तथा ।\\
नीपो विभीतकश्चैव श्राद्धकर्मणि गर्हिताः ॥\\
चिरबिल्वस्तथा टङ्कस्तिन्दुकाम्रातकौ तथा ।\\
बिल्वकः कोविदारश्च एते श्राद्धे विगर्हिताः ॥\\
निवासाश्चैव कीटानां गर्हिताः स्युरयज्ञियाः ।\\
अन्यांश्चैवंविधान् सर्वान् वर्जयेद्वै अयज्ञियान् ॥\\
फल्गु = काकोदुम्बरिका । टङ्कः =हिमारकः }{।}{\\
अथ देवतामन्त्रादय: ।\\
मनुबृहस्पती-\\
अग्नेः सोमयमाभ्यां च कृत्वाप्यायनमादितः ।\\
हविर्दानेन विधिवत् पश्चात्सन्तर्पयेत्पितृन् । इति ।\\
अग्न्यादीनां प्रथमं होमेनाप्यायनं कृत्वा पश्चात्पितॄन्यजेदित्यर्थः ।


{अत्र यद्यपि सोमयमाभ्यामिति द्वन्द्वोत्तरचतुर्थ्या व्यासक्तं देव\\
तात्वं प्रतीयते, तथापि न तद्विवक्षितं, प्रत्येकदेवतात्वावेदक मन्त्रा-\\
हुतिसंख्यादीनां बहुलं दर्शनात् । किञ्च षष्ठ्यर्थ एषा चतुर्थी , अन्य-\\
थाप्यायनपदानन्वयापत्तेः । दृश्यते च पाठान्तरमपि ``अग्निसोमय-\\
मानां चेति, तस्मात्प्रत्येकं देवतात्वम् । अतएव -\\
यमः-\/-\\
अग्नये चैव सोमाय यमाय जुहुयात्ततः ।\\
अग्नये हव्यवाहाय स्वाहेति जुहुयाद्धविः ॥\\
सोमाय च पितृमते यमायाङ्गिरसे तथा ॥ इति ।\\
मार्कण्डेयेन द्वे आहुती,\\
अग्नये कव्यवाहनाय स्वाहेति प्रथमाहुतिः ।\\
सोमाय वै पितृमते स्वाहेत्यन्या तथा भवेत् ॥ इति ।\\
वैशन्ब्दः पादपूरणार्थो न मन्त्रान्तर्गतः । एतदेवविपरीतमाह-\\
गोमिलः,\\
मेक्षणेनोपघातं जुहुयात्स्वाहा सोमाय पितृमत इति पूर्वं, स्वा\\
हाग्नये कव्यवाहनायेति द्वितीयम् । अत ऊर्ध्वं प्राचीनावीतीति ।\\
ऊर्ध्वमित्युक्तेः पूर्वं यज्ञोपवीतीति गम्यते ।

{२३२ वीरमित्रोदयस्य श्राद्धप्रकाशे-}{\\
ब्रह्माण्डपुराणे -\/-\\
अग्नये कव्यवाहनाय स्वधा नम इति ब्रुवन् ।\\
सोमाय च पितृमते स्वधा नम इति ब्रुवन् ॥\\
यमायाङ्गिरस्वते स्वधा नम इति ब्रुवन् ।\\
इत्येते होममन्त्रास्तु त्रयाणामनुपूर्वशः ॥\\
देशविशेषोऽपि तत्रैव-\\
दक्षिणतोऽग्नये नित्यं सोमायोत्तरतस्तथा ।\\
एतयोरन्तरे नित्यं जुहुयाद्वै विवस्वत इति ॥\\
आश्वलायनः ।\\
सोमाय पितृमते स्वधा नमः, अग्नये कव्यवाहनाय स्वधा नम\\
इति दक्षिणाग्नौ जुहोति यमायाङ्गिरस्वते पितृमते स्वधा नम इति\\
द्वितीयम् । अग्नये कव्यवाहनाय स्वधा नम इति तृतीयम् । न यमाय\\
जुहोतीत्येक इति ।\\
सांख्यायनः ।\\
अग्नये कव्यवाहनाय स्वाहा, सोमाय पितृमते स्वाहा, यमा-\\
याङ्गिरस्वते पितृमते स्वाहेति ।\\
बैजवापः }{।}{\\
अन्वाहार्यपत्रने मेक्षणेन द्वे आहुती जुहोत्यग्नये इति पूर्वो, सो-\\
मायेत्युत्तरामिति । अत्र केवलयोर्देवतात्वम् । यद्यप्येतदाश्वलाय\\
नादिभिः पिण्डपितृयज्ञमधिकृत्योक्तम् । तथाप्यतिदेशादग्नौ करणे\\
ऽपि कर्त्तव्यम् ।\\
आपस्तम्बेन तु त्रयोदशाहुतिकमन्नौकरणमुक्तम् । अन्नस्यो\\
त्तराभिजुहोत्याज्याहुतीरुत्तरा इति ।\\
अत्रान्नशब्देन भोजनान्नं विवक्षितम्, षष्ठ्या च सम्बन्धः , सौत्रो\\
पादानोपादेयभावः । मन्त्रप्रपाठकमध्ये पूर्वविनियुक्ताभ्य ऋग्म्य ऊर्ध्वं\\
पठिता ऋच उत्तरास्ताभिः । अत्राहुतीनामुत्तरा , पश्चाद्भाविनीः । भाज्या\\
हुतीर्जुहोतीत्यनुषङ्गः । लिङ्गविनियुक्तानामप्यनुष्ठानसौकार्याय विनि\\
योगं भाष्यकारादय आहुः । यन्मे माता प्रलुलोभ चरति यास्तिष्ठन्तीति\\
द्वाभ्याममुष्यै स्वाहेत्यन्ताभ्यां यन्मे पितामही प्रलुलोभ
चरत्यन्तर्दधे\\
पर्वतेरिति द्वाभ्यामृग्भ्याममुष्यै स्वाहेत्यन्ताभ्यां यन्मे प्रपितामही
प्रलु\\
लोभ चरन्त्यन्तर्दध ऋतुभिरिति द्वाभ्यामृग्भ्याममुष्यै स्वाहेत्यन्ता,

{ }{ अग्नौकरणेतिकर्तव्यता । २३३}{\\
भ्यामिति । अत्र सर्वत्र जुहोतीत्यनुषज्यते । अमुष्मा इत्यत्र चतुर्थ्य-\\
न्तं पितुर्नाम गृहीत्वा द्वे आहुती जुहुयात् । एवमुत्तरत्र द्वाभ्यां
द्वा-\\
भ्यामृग्भ्यां चतुर्थ्यन्त पितामहस्य प्रपितामहस्य च नाम गृहीत्वा द्वे\\
द्वे आहुती जुहुयात् । एवं महीनामपि पित्रादिपदस्थाने मा-\\
तामहादिपदोहेन, मात्राादिपदस्थाने मातामह्यादिपदोहेन चतुर्थ्यन्तं\\
मातामहादिनाम च गृहीत्वा द्वे द्वे आहुती जुहुयात् । ततो ये चेह\\
पितर इत्यृचान्नादेवैकामाहुः - जुहोति । ततः षडाज्याहुतीर्जुहोति ।\\
स्वाहा पित्रे पित्रे स्वाहेति मन्वाभ्यां द्वे आज्याहुती हुत्वा
पुनराभ्यामेव\\
तृतीयचतुर्थ्यौ हुत्वा स्वधास्वार्हति मन्त्रेण पञ्चमीम् , अग्नये
कव्यवाह\\
नायेति षष्ठीं जुहोति । एतच्च धर्मप्रश्नस्थालीपाकगृह्योक्तेतिकर्त्तव्य-\\
तासहितं कार्यम् ।\\
तथा च गृह्यभाष्यसङ्ग्रहङ्कारः ।\\
अग्नीन्धनादिप्रतिपद्यकर्म कृत्वाऽऽज्यभागं तथावदाय यन्मेति म\\
न्त्रैः प्रतिमन्त्रमन्नौकार्याः , तथा सप्तभिरन्नहोमाः
स्वाहादिमन्त्रैरापसर्पि-\\
षा स्युर्होमाः, ततः स्विष्टकृतं तु हुत्वा, भस्माद्यपोह्याहविरन्नहोमो-\\
र्त्तेपञ्चदर्व्योश्च समञ्जनादिशेषं च कृत्वा परिषेचनान्तं पात्रेषु
दद्याद्\\
हुतशेषमन्नमिति । सप्तभिरन्नहोम इत्यनहित मन्त्राभिप्रायम् ।\\
अग्नौकरणे च सव्यासव्ययोर्विकल्पः ।\\
अग्नौकरणहोमस्तु कर्त्तव्य उपवीतिना ।\\
अपसव्येन वा कार्यो दक्षिणाभिमुखेन च ॥\\
इति छन्दोगपरिशिष्टे कात्यायनोक्तेः । प्राचीनावीतीध्ममुपसमा\\
धाय मेक्षणेनावदायावदानसम्पदा जुहुयात्, सोमाय पितृमते स्वधा\\
नमः , अग्नये कव्यवाहनाय स्वधा नम इति, स्वाहाकारेण था, अग्निं पूर्वं\\
यज्ञोपवीतीत्याश्वलायनोकेश्च । तदेतत्सर्वे वैकल्पिकं पदार्थजातं\\
यथाचारं यथाकल्पसूत्रगृशं वा व्यवस्थितं ज्ञेयम् । अनग्म्यधिकरण.\\
होमपक्षे विशेषः ।\\
स्मृत्यर्थसारे,\\
अत्र मेक्षणेघ्माविप्रानुज्ञा न सन्ति परिसमूहनपर्युक्षणे स्त इति ।\\
स्मृतिचन्द्रिकाकारस्तु अनुज्ञाशाभावं नेच्छति उक्तशौनकविरोधात् ।\\
मेक्षणं तु मेक्षणकार्येऽन्यस्याविधानात्भवत्येव, परिसमूहनपर्युक्षणे तु\\
पांसुनिरसनदृष्टकार्यालोपान्नियमादृष्टमात्रस्यचाप्रयोजकत्वात् ने-\\
वी० मि ३०

{२३४ वीरमित्रोदयस्य श्राद्धप्रकाशे-}{\\
च्छति । अदृष्टार्थकत्वात्कार्ये इत्यन्ये । परिस्तरणं
त्वदृष्टार्थत्वाद्भवत्येव ।\\
अग्नीकरणवत् तत्र होमो दैवकरे भवेत् ।\\
पर्यस्तदर्भानास्तीर्य यतो ह्यग्निसमो हि सः ॥\\
इति यमोक्तेश्च ।\\
पर्यस्तदर्भान् = परितः सर्वतोन्यसनीयान् । ` पर्युक्ष्य
दर्भानास्तीर्ये'ति\\
क्वचित्पाठस्तदा पर्युक्षणमपि भवत्येव । इध्मस्तु समिन्धनार्थ •\\
त्वान्न भवत्येव ।\\
कर्कोपाध्यायास्तु -\\
परिस्तरणादीनि मेक्षणान्तानि अग्निहोमेऽपि न भवन्ति किं पुनः\\
पाणौ, पिण्डपितृयज्ञवदुपचार इति कात्यायनवचनेन सर्वेतिकर्त्तव्य-\\
तायाः प्राप्तौ पिण्डपितृयज्ञवद्भुत्वेति पुनर्वचनस्य परिस्तरणादिपरि-\\
संख्यानार्थत्वादित्याहु: । तन्मतानुसारिणामाचारोप्येवम् ।\\
अथ हुतावशिष्टप्रतिपत्तिः ।\\
यमः ।\\
दैवविप्रकरेऽनग्निः कृत्वाग्नौकरणं द्विजः ।\\
शेषयेत्पितृविप्रेभ्यः पिण्डार्थं शेषयेत् तथा ॥\\
अग्नौकरणशेषं तु पित्र्येषु प्रतिपादयेत् ।\\
हुतशेषं न दद्यात्तु कदाचिद्वैश्वदेविके ॥\\
अत्र कदाचिदित्यनेन यदापि वैश्वदेविकविप्रकरे होमस्तदापि न\\
तत्र हुतशेषप्रतिपत्तिरिति गम्यते । अत एव -\\
वायुपुराणे ।\\
हुत्वा दैवकरेऽनग्निः शेषं पित्र्ये निवेदयेत् ।\\
न हि स्मृताः शेषभाजो विश्वेदेवाः पुराणगैः ॥ इति ।\\
एतेन दैवविप्रकरेऽपि हुतशेषदानमिति गौड़निबन्धोक्तं परा-\\
स्तम् । इदं च यदा पाणिहोमस्तदा पङ्क्तिमूर्द्धन्यातिरिक्तेषु पितृपात्रेषु
।\\
पित्र्ये यः पङ्किमूर्द्धन्यस्तस्य पाणावनग्निकः ।\\
हुत्वा मन्त्रवदन्येषां तूष्णीं पात्रेषु निक्षिपेत् ॥\\
इति कात्यायनेनान्येषामित्युक्तोरिति प्रकाशकारः । अत्र चामन्त्रकः पा-\\
त्रेषु प्रक्षेप उक्तः ।\\
हुत्वा त्वग्नौ ततः सम्यग्विधिनानेन मन्त्रवित् ।\\
स्वधेत्येव हविःशेषमाददीत समीक्ष्य च ॥ इति ।

{ }{ परिवेषणप्रकारः । २३५}{\\
शेषमन्नं हस्तेन हस्तेषु पिण्डवत्प्रदायेति यमनिगमपरिशिष्टयोस्तु\\
स्वधेतिमन्त्रेण हस्तेषु प्रक्षेप उक्तः । अनयोश्च विकल्पः, स
यथाशास्त्रं\\
व्यवस्थितः । एतच्चापसव्येन कार्य ।\\
हुत्वाग्नौ परिशिष्टं तु पितृपात्रेष्वनन्तरम् ।\\
निवेद्यैवापसव्येनेति शौनकोक्तेः । यत्पाणी हुतं यच्च पात्रेषु\\
हस्तेषु वा प्रक्षिप्तं तद् द्वयमपि भोज्याश्वेन सह भोक्तव्यम् ।\\
यच्च पाणितले दत्तं यच्चान्यदुपकल्पितम् ।\\
एकीभावेन भोक्तव्यं पृथग्भावो न विद्यते । इति ।\\
तथा ।\\
यदन्नं दीयते पाणौ पात्रे वापि निधीयते ।\\
भुञ्जीरन् ब्राह्मणास्तत्तु पितृपङ्क्तौ निवेशिताः ॥\\
इति गृह्यपरिशिष्टगार्ग्ययोर्वचनादिति हेमाद्रिः । अत्र पात्रे वापि
निधी-\\
यत इत्यनेन हुतशेषेमुच्यते पितृपङ्क्तिविति समभिव्याहारात् । भो-\\
ज्यान्नस्य वैश्वदेविकेऽपि समानत्वात्तद्भोजनविधौ वैयर्थ्यात् ।\\
एतेन यदाश्वलायनव्याख्याकृतोक्तं यदि पाणिष्वाचान्तेष्वन्यदन्न-\\
मनुदिशति सूत्रव्याख्याने परिवेषणात्पूर्व विधीयमानमाचमनं पृथक्\\
पाणिहुतान्नभक्षणनिमित्तमिति तदपास्तम् ।\\
अन्नं पाणितले दत्तं पूर्वमश्नन्त्यबुद्धयः ।\\
पितरस्तेन तृप्यन्ति शेषान्नं न लभन्ति ते ।\\
इति बहुचगृह्यपरिशिष्टे निषेधाच्च । आचमनं त्वदृष्टार्थ भवि.\\
ष्यति वचनादिति । हरिहरोऽप्येवम् ।\\
बोधायनस्तु ।\\
हुतशेषं प्रकृत्य `` तस्मिंस्तु प्राशिते दद्याद्यदन्यत्प्रकृतं भवे''
दिति\\
हुतशेषस्य पृथग्भक्षणमाह । एतच्च बौधायनानामेवेति हेमाद्रिः । इति\\
हुत्वा वशिष्टप्रतिपतिः । इत्यग्नौकरणम् ।\\
अथ परिवेषणम् ।\\
प्रचेताः ।\\
हुतशेषं पितृभ्यस्तु दत्वान्नं परिवेषयेत् ।\\
अत्र हुत्तशेषप्रदानानन्तरं घृतेन दैवपूर्वकमाभासु पक्कमैरय इति\\
मन्त्रेण तूष्णीं वा पात्राभिघारणमाचरन्ति । परिवेषणेतिकर्त्त-\\
व्यतामाह ।

२३६ वीरमित्रोदयस्य श्राद्धेप्रकाशे -

{मनुः ।\\
पाणिभ्यामुपसंगृह्य स्वयमन्नस्य वर्द्धितम् ।\\
विप्रान्तिके पितॄन् ध्यायञ् शनकैरुपनिक्षिपेत् ॥\\
अन्नस्य वर्द्धितम् = पाकपात्रम् ।
पाकपात्रोद्धृतान्नपूर्णपात्रान्तरमिति\\
मेधातिथिः । स्वयमिति मुख्यः पक्षः । पत्न्यादिरपि वक्ष्यते । विप्रा\\
न्तिक उच्छिष्टस्पर्शादि यत्र न संभाव्यते तत्र । शनकैर्यथा पात्रभेदादि\\
न भवति तथा । उपनिक्षिपेत् परिवेषणार्थं स्थापयेत् ।\\
मनुः ।\\
उभयोर्हस्तयोर्मुक्तं यदन्नमुपनीयते ।\\
तद्धि प्रलुपन्त्यसुराः सहसा दुष्टचेतसः ।\\
मुक्तम् = आस्थितम् । सप्तमी वा तृतीयार्थे । परिवेषण व दैवपूर्वम् ।\\
`` विधिना दैवपूर्वं तु परिवेषणमाचरे '' दिति शौनकोक्तेः । एतच्चा\\
द्यपरिवेषणे, द्वितीयादौ त्वनियमः । `` यद्यद्रोचेत
विप्रेभ्यस्तत्तद्दद्याद\\
मत्सरी '' इति यमेन विप्रेच्छानुविधानोतेः। कर्त्तारो धर्मभविष्योत्तरयोः\\
फलस्यानन्तता प्रोक्ता स्वयं तु परिवेषणे इति ।\\
भार्यया श्राद्धकाले तु प्रशस्तं परिवेषणम् ॥ इति ।\\
अत्र पत्न्याः प्राशस्त्योक्तेरन्येऽपि पाककर्तृप्रकरणेअनुज्ञायन्ते ।\\
पत्नी च सवर्णा, द्विजातिभ्यः सवर्णाया नार्या हस्तेन दीयते इति\\
नारायणोक्तेः । कर्त्तुर्धर्मविशेष:-\\
पाद्ममात्स्ययोः ।\\
उभाभ्यामथ हस्ताभ्यामाहृत्य परिवेषयेत् ।\\
प्रशान्तचित्तः सतिलदर्भपाणिर्विशेषतः ॥ इति ।\\
मनुरपि ।\\
( १ ) उपनीय सर्वमेतच्छनकैः सुसमाहितः ।\\
परिवेषयेत्प्रयतोऽन्नगुणांस्तु प्रचोदयन् ॥ इति ॥\\
एतद् = भक्ष्यभोज्यादि । गुणान् = माधुर्यादीन् । पायसादिदाने मन्त्र-\\
विशेषा मानवमैत्रायणीयसूत्रे । `` पयः पृथिव्या '' मिति पायसं दद्यात् ।\\
मधुव्वाता ऋतायत'' इति मधु ``आयुदौ'' इति घृतं दद्यात् ।\\
कठसूत्रे \textbar{}\\
एषा वोर्खामासु पक्कमिति घृतं वामिच्येति । मन्त्रान्तराणि ।

% \begin{center}\rule{0.5\linewidth}{0.5pt}\end{center}

% \begin{center}\rule{0.5\linewidth}{0.5pt}\end{center}

  परिवेषणप्रकारः । २३७

{स्कान्दे ।\\
पायसं गुडसंयुक्तं हविष्यं गुडपूरितम् ।\\
नमो वः पितरो रसाय परिविषन्नभिमन्त्रयेत् }{॥}{\\
तेजोसि शुक्रमित्याज्यं दधिक्राव्णेति वै दधि ।\\
क्षीरमाप्यायमन्त्रेण व्यञ्जनानि च यानि तु ॥\\
भक्ष्यभोज्यानि सर्वाणि महा इन्द्रेण दापयेत् ।\\
संवत्सरोसि मन्त्रं तु जप्त्वा तेनोदकं द्विजः ॥ इति ।

{हारीतः ।\\
न पङ्क्यां विषमं दद्यादिति ।\\
परिवेषणं च दक्षिणहस्तेन कर्त्तव्यम् }{।}{\\
कर्मोपदिश्यते यत्र कर्त्तुरङ्गं न तूच्यते ।\\
दक्षिणस्तत्र विज्ञेयः कर्मणां पारगः करः ॥\\
इति छन्दोगपरिशिष्टादिति मिश्राः ।\\
उभाभ्यामपि हस्ताभ्यामाहृत्य परिवेषयेत् ॥\\
एकेन पाणिना दत्तं शुद्धदत्तं न भक्षयेत् ॥\\
इति वचनादुभाभ्यां हस्ताभ्यामिति गौढाः । हस्ताभ्यामित्यस्या.\\
हृत्येत्यनेनान्वयादेकेनेत्यस्य च केवलेनेत्यर्थान्मिश्रमतमेव श्रेयो\\
भाति । अत एव -\\
वशिष्ठः ।\\
हस्तदत्ताश्च ये स्नेहा लवणं व्यञ्जनानि च ।\\
दातारं नोपतिष्ठन्ति भोक्ता भुङ्क्ते च किल्विषम् ॥\\
तस्मादन्तरितं देयं पर्णेनाथ तृणेन वा ॥ इति ।\\
ब्राह्मणपरिवेषणसमये च पिण्डार्थमप्येकस्मिन् पात्रे परिवेषयेत् ।\\
तथा च ब्राह्मे }{।}{\\
ततोऽन्तं सुरसं स्वाद्वित्युपक्रम्य -\\
ब्राह्मणानां च प्रददौ पिण्डपात्रे वथैष च इति ।

{यमः ।\\
ब्राह्मणानं ददच्छ्रतः शूद्रान्नं ब्राह्मणो ददत् ।\\
तयोरजमभोज्यं तु भुक्त्वा चान्द्रायणं चरेत् ।\\
ददस्पात्र इति शेषः । उपनयन्निति मिश्राः । इति परिवेषणम् ।

२३८ वीरमित्रोदयस्य श्राद्धप्रकाशे- 

{ अथ पात्रालम्भजपाङ्गुष्ठनिवेशनानि ।\\
तत्र\\
कात्यायनः ।\\
हुतशेषं दत्त्वा पात्रमालभ्य जपति `` पृथिवी ते पात्र द्यौरपिधानं\\
ब्राह्मणस्य मुखे अमृते अमृतं जुहोमि स्वाहेति'' वैष्णव्यर्च्या यजु-\\
षा वाङ्गुष्ठमन्नेवगाह्य अपहता इति तिलान्प्रकीर्योष्णं स्विष्टमन्नं
दद्या-\\
दिति ।\\
इदं विष्णुरित्यृग् = वैष्णवी । विष्णो हव्यं रक्षेति यजुः । अन्ने=
पिज्य\\
पात्रनिहितेऽग्नौकरणशेषे दद्यात् = परिवेषयेत् ।\\
निगमे तु अङ्गुष्ठनिवेशनोत्तरं परिवेषणं ततः पात्रालम्भो जपमन्त्रे\\
पाठान्तरं चोक्तम् । शेषमन्नं हस्तेषु
पिण्डवत्प्रदायाङ्गुष्ठमन्नेऽवगाह्य\\
सोष्णमन्नं बहु च दद्यादमिमृश्यपात्र जपति पृथिवी ते पात्र द्योरपि-\\
धानं ब्राह्मणानां त्वा प्राणापानयोर्मध्येऽमृतं जुहोमि स्वाहेति ।
अत्रा-\\
ङ्गुष्ठनिवेशनं तूष्णीम् ।\\
मैत्रायणीयसूत्रे तु पात्रालम्भे मन्त्रान्तरमुक्तं ।\\
अवशिष्टेऽन्ने ब्राह्मणाङ्गुष्ठमुपयस्य द्यौ. पात्रं स्वधा पिधानं
ब्राह्म-\\
णस्य मुखे अमृते अमृतं जुहोमि स्वधेति ।\\
अवशिष्टेऽन्ने = नौकरण होमाविशिष्टे पितृपात्रदत्ते ।\\
बौधायनसूत्रे तु पित्रादिस्थानभेदेनाङ्गुष्ठनिवेशने मन्त्रविशेषः । अ-\\
थेतरद्ब्राह्मणेभ्यो निवेद्य ब्राह्मणस्याङ्गुष्ठेनानचेसेनानुदिशति ।
पृथिवी\\
सम तस्य तेग्निरुपद्रष्टा ऋचस्ते महिमादत्तस्याप्रमादाय पृथिवी पात्रं\\
द्यौरपिधानं ब्राह्मणस्य मुखे अमृतं जुहोमि ब्राह्मणाना त्वा विद्याव.\\
ता प्राणापानयोर्जुहोमि अक्षितमसि मा पितॄणा क्षेष्वा अमुत्रामु-\\
ष्मिन्लोक इति । द्वितीयमनुदिशति अन्तरिक्षसमं तस्य ते वायुरुप-\\
श्रोता यजूंषि ते महिमादत्तस्येत्यादि पूर्ववत् । पितॄणामिति शेषः ।\\
इतरत् = हुतावशिष्टम् । निवेद्य = पित्र्यविप्रपात्रेषु निक्षिप्य
\textbar{} अनुदिशति = }

{स्पर्शयति ।\\
प्रचेतसापि परिवेषणोत्तरं पात्रालम्भाद्युकं सर्वे च प्रकृतं दत्वा\\
पात्रमालभ्य सञ्जपेदिति ।\\
प्रकृतं = श्राद्धार्थतया सम्पादितम् । दत्वा =परिवेष्य \textbar{} सजपेत्
= पृथिवी.\\
तेपात्रमित्यादीति शेषः ।

{ }{ पात्रालम्भादिप्रकारः । २३९}{\\
दत्वान्नं पृथिवीपात्रमिति पात्राभिमन्त्रणम् ।\\
कृत्वेदं विष्णुरित्यन्ने द्विजाङ्गुष्ठं निवेशयेत् ॥\\
इति याज्ञवल्क्योक्तेः ।\\
पैठनिसिस्त्वसउदकेनाङ्गुष्ठनिवेशनमाह - पृथिवी ते पात्रं द्यौरपि-\\
धानं ब्राह्मणस्य मुखेऽमृते अमृतं जुहोमि स्वधा इदं विष्णुर्विच-\\
क्रम इत्यनेनाङ्गुष्ठमन्ने चोदके चावधायेति । उदकं चान्नवत्पानार्थं\\
परिवेषणकाल एवोपनीतं अत एव ।\\
यमः-\\
विष्णो हव्यं च कव्यं च ब्रूयाद्वक्षेति च क्रमात् ।\\
वारिष्वपि प्रदत्तेषु तमङ्गुष्ठं निवेशयेत् ॥ इति ।\\
प्रदत्तेषु = पानार्थं परिविष्टेषु । क्रमात् = दैवे हव्यं पित्र्ये
कव्यमिति ।\\
तम् = पूर्वमन्ने निवेशितम् । जलपरिवेषणं च-\\
ब्रह्मपुराणे ।\\
अथ दत्वा समग्रं तु जलान्तं भोजनं क्रमादिति ।\\
भोजनम् = भक्ष्यभोज्यादि । दत्वा = परिविष्य \textbar{} क्रमात् =
विश्वेदेवादि-\\
क्रमेण । प्रतिपादयेदित्युत्तरवाक्यस्थेनान्वयः । अङ्गुष्ठनिवेशनं च\\
जानु निपात्य कार्यं अङ्गुष्ठमुपयस्येदं विष्णुरिति जानु निपात्य भू-\\
माविति शङ्खलिखितोक्तेः । तच्च दैवे दक्षिणं पित्र्ये सव्यं `` दक्षिणं
पातये-\\
ज्जानु देवान् परिचरन्त्सदे''त्यादिपूर्वलिखित छन्दोगपरिशिष्टात् ।\\
कालिकापुराणे= अन्नेऽङ्गुष्ठभ्रामणमुक्तम्-\\
धृत्वाङ्गुष्ठं द्विजानां तु आवर्त्याज्यं मधुप्लुत इति ।\\
आवर्त्य = परिभ्राम्य । अङ्गुष्ठनिवेशनमावश्यकं `` निरङ्गुष्ठं च
यच्छ्राद्धं\\
न तत्प्रीणाति वै पितॄन् इति हारतोक्तेः । अङ्गुष्ठग्रहणे मन्त्रान्तरं\\
भ्रामणे च विशेषः-\\
पिप्पलादेसूत्रे }{। }{\\
अङ्गुष्ठमुपयमन्पात्रे प्रदक्षिणं दैवे, अपसव्यं पित्र्ये अतो देवा अव\\
न्तु नो यतोविष्णुरिति जपेज्जानुनी निपात्य भूमाविति ।\\
अपसव्यम् = अप्रदक्षिणम् । जानुनीति द्विवचनं दैवपित्र्याभिप्रायम् ।\\
अङ्गुष्ठमुपयम्य प्रदक्षिणं देवे, अपसव्यं पित्र्ये इदंविष्णुरिति
जपेज्जा-\\
नु निषद्य भूमाविति कौशिकेन दैवपित्र्ययोरेकजानु निपातनोक्तेः । अङ्गु-\\
ग्रहणं चानुत्तानस्य हस्तस्य अनुत्तानेन हस्तेन कार्यम् ।

{२४० वीरमित्रोदयस्य श्राद्धप्रकाशे-}{\\
परिवर्त्य न चाङ्गुष्ठं द्विजस्यान्ने निवेशयेत् ।\\
राक्षसं तद्भवेद्दैवं पितॄणां नोपतिष्ठते ॥\\
उत्तानेन तु हस्तेन द्विजाङ्गुष्ठनिवेशनम् ।\\
यः करोति द्विजो मोहात्तद्वै रक्षांसि गच्छति ॥\\
इति बौधायनेनोत्तानहस्ते दोषोक्तेः ।\\
परिवर्त्य = उप्तानीकृत्य । इति पात्रालम्भजपाङ्गुष्ठनिवेशनानि ।\\
अथान्नसङ्कल्पः ।\\
तत्र प्रभासखण्डे ।\\
पितृपात्रेषु दत्वान्नं कृत्स्नं सङ्कल्पमाचरेत् ।\\
पितृग्रहणं दैवस्याप्युपलक्षणम् । अत्र कृत्स्नमितिग्रहणादन्न\\
सङ्कल्पोत्तरं उच्छिष्टकाले परिवेषणम् । युक्तं चैतत् , दाने\\
सम्प्रदानविशेषणस्य शुचित्वस्यापेक्षितत्वादिति केचित् । तन्न ।\\
भोजनपदार्थस्य लौकिकप्रसिध्या तथैवावगतेः । परिवेक्ष्यमाण\\
स्यापि च सम्प्रदानशुचिताकाल एव त्यागोपपत्तेश्च । अ-\\
तएव परिविष्टं परिवेष्यमाणं चेति सङ्कल्पवाक्यं प्रयुञ्जते शिष्टाः ।\\
यद्यद्रोचेत विप्रेभ्यस्तत्तद्दद्यादमत्सरीतियमोक्तेश्चोच्छिष्टकालेऽपि\\
तदुक्तेश्च । सङ्कल्पप्रकारमाह -\\
विष्णुः ।\\
नमो विश्वेभ्यो देवेभ्य इत्यन्नमादौ प्राप्नुस्वर्योनिवेदयेत् । पि-\\
त्र्ये पितामहाय प्रपितामहाय नामगोत्राभ्यामुदङ्मुखेषु ।\\
नमो विश्वेभ्य इति च सतिलेनोदकेन च ।\\
प्राङ्मुखेषु च यद्दत्तं तदन्नमुपमन्त्रयेत् }{॥}{\\
उदङ्मुखेषु यद्दत्त नामगोत्रप्रकीर्त्तनैः ।\\
मन्त्रयेत्प्रयतः प्राज्ञः स्वधान्तैः सुसमाहितः ।\\
एवं च यदत्रिणा ।\\
हस्तेनामुक्तमन्नाद्यमिदमन्नमुदीरयेत् }{।}{\\
स्वाहेति च ततः कुर्यात्स्वसत्ताविनिवर्त्तनम् ? ॥\\
इति स्वाहान्तत्वमुक्तं, तद्वैश्वदेविकविषयम् । इदं च त्रैवर्णिक-\\
विषयं स्त्रीशूद्रयोस्तु नम इति प्रयोग इति शूलपाणिः । पात्रस्यादानं च\\
संध्येन । दक्षिणस्य त्यागे व्यापृतत्वात् ।

{ }{ सावित्रीजपादिप्रकारः । २४१}{\\
ब्रह्मपुराणे ।\\
एतद्वो अन्नमित्युक्त्वा विश्वान् देवांश्च संयजेत् ।\\
एतद्वोऽन्नमित्येतदिदमिदमन्नमित्यनेन सह विकल्पते तुल्यार्थ-\\
त्वात् । अत्र एतद्व इति निर्देशाच्चतुर्थीविभक्त्या देवतानिर्देशोऽत्र\\
विवक्षितः । अत एव कठसूत्रे पृथिवी ते पात्रमिति सङ्कल्पं कृत्वाऽ\\
मुष्मै स्वधा नमोऽमुष्मै स्वधा नम इति यथालिङ्गमनुमन्त्र्य भोजयेत्\\
इत्युक्तम् । इत्यन्नसङ्करूपः ।\\
अथ सावित्रीजपादि ।

{ तत्र पारस्करः ।\\
सङ्कल्प्य पितृदेवेभ्यः सावित्रीमधुमज्जपः ।\\
श्राद्ध निवेद्यापोशानं जुषध्वं प्रैषभोजनम् ॥\\
सावित्री - सवितृदेवत्या गायत्री; सा च प्रणवव्याहृतिपूर्विका }{ ।}{\\
ॐ भूर्भुवः स्वस्तत्सवितुरिति त्रिरुक्त्वेति मानवमैत्रायणीयोक्तेः ।
त्रिरि·\\
ति वैकल्पिकम् । `` अपोशानं प्रदायाथ जपेद्व्याहृतिपूर्वां गायत्रीं
त्रिः\\
सकृद्वेति कात्यायनेनोक्ते । मधुयज्ञ्जपः = मधुवातेस्यृक्त्रयजपः॥ ``
मधुब्बाता\\
इति तृचं मध्वित्येतत्त्रिकं तथे''ति प्रचेतःस्मृतेः । अत्र मधुमज्जपो
मधु.\\
दानाङ्गं ``मधुमन्त्रं ततो जप्त्वा अन्ते दद्याच वै मधु'' इति भविष्यो•\\
क्तेरितिगौडा रायमुकुटादयः पितृदयिता च । अन्ये तु भविष्यवचनं\\
कालसम्बन्धार्थमित्याहुः ।\\
श्राद्धनिवेदनप्रकारमाह-\/-}

{यमः-\\
( १ ) अन्नहीनं क्रियाहीनं मन्त्रहीनं च यद्भवेत् ।\\
सर्वमच्छिद्रमित्युक्त्वा ततो जलमपाशयेत् ॥ इति ।\\
अच्छिद्रमित्यस्यानन्तरं जायतामिति वाक्यशेषः ।\\
अपोज्ञान = तदर्थं जलम् । दद्यादिति शेषः । अत्र पारस्करवचने\\
सावित्रीजपाद्यनन्तरमपोशानदानमुकम् । उदाहृतकात्यायनेन तु\\
अपोशानदानोत्तरं सावित्रीजपाद्युक्तं तेन तयोर्विकल्पः , स च\\
यथाशास्त्रं व्यवस्थितो द्रष्टव्यः ।

% \begin{center}\rule{0.5\linewidth}{0.5pt}\end{center}

{\\
(१) मन्त्रहीनं क्रियाहीनं भक्तिहीनं द्विजोत्तमाः ।\\
श्राद्धं सम्पूर्णतां यातु प्रसादाद्भवतां मम ॥\\
इति श्राद्धकाशिकायां पाठः ।\\
वी.० मि ३१

{२४२ वीरमित्रोदयस्य श्राद्धप्रकाशे-}{\\
अपोशानदानात्पूर्व तिलादिविकरणमुक्तं-\\
ब्रह्मपुराणे ।\\
तिलयुक्तं च पानीयं सकुशं तेषु चाग्रतः ।\\
विकिरेत्पितृभ्यः स्वेभ्यो जपंश्चापहता इति ॥\\
तेभ्यो दद्यादपोशानं भवन्तः प्राशयन्त्विति ।\\
इदं पित्र्ये । दैवे तु यवोसीतिमन्त्रेण यवविकरणमिति मिश्राः । अ-\\
पहतेत्यनेनैव यवविकरणमिति तु पितृदयिता । अत्र ब्राह्मणैर्भूमौ\\
वलिदानं न कार्यम् ।\\
दत्ते वाप्यथवाऽदत्ते भूमौ यो निक्षिपेद्वलिम् ।\\
भोजनात्किञ्चदन्नाग्रं धर्मराजाय वै वलिम् ॥\\
दत्वाथ चित्रगुप्ताय विप्रश्चौर्यमवाप्नुयात् ।\\
इत्यत्रिणा बलिदाने दोषोक्तेः । अत्र भूमावित्युक्तेः प्राणाहुतयो\\
भवन्त्येव । जुषध्वं प्रैषभोजनम् = जुषध्वमितिप्रैषेण भोजनमित्यर्थः ।\\
तत्प्रकारमाह । मार्कण्डेयः-\\
यथा सुखं जुषध्वं भोरिति वाच्यमनिष्ठुरम् । इति ।\\
अश्नत्सु रक्षोघ्नादिजपो जप्यप्रकरणे द्रष्टव्यः ।\\
अथ विकिरदानादि ।\\
कात्यायनः ।\\
तृप्तान् ज्ञात्वान्नं प्रकीर्य सकृत्सकृदपो दत्वा पूर्ववद्वायत्रीं
जप्त्वा\\
मधुमतीर्मधुमध्विति च, तृप्ता स्थेति पृच्छति तृप्ताः स्मेत्यनुज्ञातः\\
शेषमन्नमनुज्ञाप्येति । एतड्याख्यास्यते ।\\
याज्ञवल्क्यः ।\\
आतृप्तेस्तु पवित्राणि जप्त्वा पूर्वजपं तथा ।\\
अन्नमादाय तृप्ताः स्थ शेषं चैवानुमान्य च }{॥}{\\
तदन्नं विकिरेद् भूमौ दद्याच्चापः सकृत्सकृत् ॥ इति ।\\
ब्रह्मपुराणे ।\\
दत्वामृतापिधाने च विप्रेभ्यश्च सकृत्सकृत् ।\\
किञ्चित्सम्पन्नमेतन्मे भो विप्रा इति तान्वदेत् ॥\\
ते च प्राहुः सुसम्पन्नं स च तान्पुनराह च ।\\
अन्नैर्भवन्तस्तृप्ता स्थ तृप्ताः स्मेति वदन्ति ते ॥\\
स तानाह पुनः शेषं क्व देयं चान्नमित्यपि ।

{ }{ विकिरदानप्रकार: । २४३}{\\
इष्टेभ्यो दीयतां चैव तदिदं प्रवदन्ति ते ॥\\
अथ तृप्तांस्तु तान् ज्ञात्वा भूमावेवान्नमुत्सृजेत् ॥ इति ।\\
एवं च पदार्थक्रमो यथाशास्त्रं व्यवस्थितो द्रष्टव्यः ।\\
विकिरेतिकर्त्तव्यतामाह -\\
मनुः ।\\
भुक्तवत्सु ततस्तेषां भोजनोपान्तिके नृप ।\\
सार्ववर्णिकमन्नाद्यं सन्नीयाप्लाव्य वारिणा ॥\\
समुत्सृजेद् भुक्तवतामग्रतो विकिरन् भुवीति ।\\
{[} अ० ३ श्लो० २४४ {]}\\
सार्ववार्णिकं = सर्वप्रकारम् \textbar{} आष्लाव्य = प्रोक्ष्य । अत्र
विशेषो -\\
विष्णुधर्मोत्तरे ।\\
अन्नं सतृणमभ्युक्ष्य मामेक्ष्येष्वेति मन्त्रतः । अत्र देशान्तरमाह-\/-\\
धूम्रः ।\\
कपित्थस्य प्रमाणेन पिण्डं दद्यात्समाहितः ।\\
तत्समं विकिरं दद्यात्पिण्डान्ते तु षडङ्गुले \textbar{}\\
भूप्रोक्षणं च-\\
वायवीये ।\\
प्रोक्ष्य भूमिमथोद्धुत्येति ।

{ब्राह्मे ।\\
उच्छिष्टे सतिलान्दर्भान्दक्षिणाग्रान्निधापयेत् ।\\
उच्छिष्टे तत्समीपे, तिलदक्षिणाग्रत्वे पित्र्ये ।\\
दैवविकिरे मन्त्र उक्तो - गोभिलेन ।\\
असोमपाश्च ये देवा यज्ञभागविवर्जिताः ।\\
तेषामन्नं प्रदास्यामि विकिरं वैश्वदेविकम् ॥\\
पित्र्ये कठसूत्रे ।\\
अग्निदग्धाश्च ये जीवा इत्यन्नं विकिरेद् भुवि ।\\
कात्यायनः ।\\
येऽनग्निदग्धा ये जीवा ये च जाताः कुले मम ।\\
भूमौ दन्तेन तृप्यन्तु तृप्ता यान्तु परां गतिम् ॥\\
विकिरप्रक्षेपानन्तरकर्त्तव्यमुक्तं -\\
ब्रह्मपुराणे ।\\
ततः प्रक्षाल्य हस्तौ च त्रिराचम्य हरिं स्मरेत् ।

{२४४ वीरमित्रोदयस्य श्राद्धप्रकाशे-}{\\
विकिरप्रतिपत्तिं चाह -\/-\\
गौतमः ।\\
विकिरमुच्छिष्टैः प्रतिपादयेत् }{। }{सहार्थे तृतीया ।\\
भार्गवस्तु प्रतिपत्यन्तरमाह-\/-\\
पिण्डवत्प्रतिपत्तिः स्याद्विकरस्येति तौल्वलिः ।\\
यदोच्छिष्टसन्निधौ विकिरदानं तदा गौतमोका प्रतिपत्तिः, यदा\\
तु पिण्डसन्निधौ तदा पिण्डवदिति व्यवस्थितं द्रष्टव्यम् ।\\
अथ पिण्डदानकालः ।\\
तत्र ब्राह्मणभोजनात्पूर्वं ब्राह्मणेष्वदत्सु वा ब्राह्मणभोजनादुत्त.\\
रकालं वा । यदापि पूर्वं तदापि ब्राह्मणार्चनानन्तरम्, अग्नौ\\
करणानन्तरं वा, यदा तु उत्तरं तदापि ब्राह्मणेष्वनाचान्तेषु आचा.\\
न्तेषु वा । यदापि अनाचान्तेषु तदापि विकिरं प्रक्षिप्य सकृद् गृही\\
तगण्डूषेषु ब्राह्मणेषु गायत्र्यादि जपं कृत्वा तृप्तिप्रश्नपूर्वकं
शेषाभ्यनु\\
ज्ञानन्तरं कर्त्तव्यम् । अथवाऽनाचान्तेषु विकिरदानम् । यदा तु आ-\\
चान्तेषु तदापि आचमनोत्तरकालम् }{।}{ अक्षय्यवाचनोत्तरकालं वा\\
इति पक्षाः । अत्र क्रमेण मूलवचनानि ।\\
साङ्ख्यायन गृह्ये }{।}{\\
भुक्तवत्सु पिण्डान् दद्यात्पुरस्तादेके ।\\
अदत्सु ब्राह्मणेष्वित्यनुवृत्तौ-\\
विष्णुः ।\\
उच्छिष्टसन्निधौ दक्षिणाग्नेषु पृथिवीदर्विरक्षितेत्येकं पिण्डं पित्र्ये\\
निदध्यादित्वादि ।\\
शङ्कः ।\\
उच्छिष्टसन्निधौ कार्यं पिण्डनिर्वपणं बुधैः ।\\
आदौ वापि ततः कुर्यादग्निकार्यं यथाविधि ।\\
मनुः । {[} अ० ३ श्लो० २१४ {]}\\
अपसव्यमग्नौ कृत्वा सर्वमावृत्य विक्रमम् \textbar{}\\
अपसव्येन हस्तेन निर्वपेदुदकं शुचिः ॥\\
र्त्रीस्तु तस्माद्धविःशेषात्पिण्डान्कृत्वा समाहितः ।\\
[ अ० ३ श्लो० २१५ ]

{ }{ पिण्डदानकालविचार: । २४६}{\\
हविःशेषात् = अनौकरणशेषात् ।\\
कात्यायनः ।\\
तृप्ताञ्ज्ञात्वान्नं प्रकीर्य सकृत्सकृदपो दत्वा पूर्ववद्गायत्रीं जपि-\\
त्वा मधुमतीः, `` मधुमध्वि''ति च तृप्ताः स्थेति पृच्छति, तृप्ताः स्म\\
इत्यनुज्ञातः । तथा च-\/-\\
मनुः -\\
समुत्सृजेद् भुक्तवतामग्रतो विकिरन् भुवि । इति ।\\
पूर्ववत् इति = प्रणवव्याहृतित्रित्व स कृत्वादिप्राप्त्यर्थम् । मधुमतीः =
मधु .\\
बाता इति तिस्र ऋचः । मधुमध्विति चेति प्रावर्गिकमन्त्रस्य प्रतीके-\\
न मधुमधुमधु इति त्रिरुच्चार्य । तृप्ता स्थेति बहुवचनोपदेशात्सर्वेषाम्
।\\
यत्तु पङ्किमुर्द्धन्यं पृच्छतीति वचनं तदबहुवचनागतेषु अग्नौ करिष्ये\\
इत्येवमादिषु द्रष्टव्यम् ।\\
केचित्तु तृतिप्रश्नस्य दृष्टार्थत्वात्सर्वार्थत्व न बहुवचनोपदेशा\\
दित्याहुः । तन्न । अदृष्टार्थत्वात्प्रश्नस्य सकृत्सकृदपां दानेन
उन्मुक्त-\\
पात्रत्वेन तृप्तिप्रश्नस्य प्रयोजनाभावात् । अतो बहुवचनबलादेव स.\\
र्वेषां प्रश्न इति युक्तम् । तृप्ताः स्म इत्यनुज्ञापि सर्वेर्दातव्या
प्रश्नस्य\\
सर्वार्थत्वादिति कर्कः । शेषान्नानुशाप्रकारस्तु शेषमन्नं किं क्रियतां
?\\
इष्टैः सह भुज्यतामित्येवं कार्य इति मदनरत्नः । सर्वशब्दः प्रकृता\\
पेक्षः तेन प्रकृतश्राद्धोपयोगिसाधितद्रव्यात्किञ्चिदादायेत्यर्थः ।\\
उच्छिष्टसमीपे शुचिदेश इति यावत्, उच्छिष्टस्य प्रतिषिद्धत्वात् ।
दर्भेषु\\
सकृदाच्छिन्नेषु पिण्डपितृयज्ञवदुपचार इत्यनेन पितृपिण्डयज्ञधर्मा-\\
तिदेशात् । अवनेज्य = अवनेजनं कृत्वेत्यर्थः । यद्यपि चावनेजनं पिण्डपि\\
तृयज्ञातिदेशादेव प्राप्तं तथापि तस्य प्रकृतौ रेखायां विहितत्वादत्र\\
दर्भेषु प्राप्त्यर्थे पुनर्वचनमिति वाचस्पतिमिश्राः }{।}{\\
केवित्तु ``पित्रादिक्रमतो दद्यादेखायामवनेजन'' मिति भविष्यव-\\
वनात् रेखायामेवावनेजनं छन्दोगभिन्नानामाहुः । `` पिण्डासनं समा-\\
स्तीर्ण छन्दोगा अवनेजनम्'' इति तत्परिशिष्टशत् । त्रींस्त्रीनिति वीप्सा\\
मातामहविषया ।\\
याज्ञवल्क्यः ।\\
अन्नमादाय तृप्ता स्थ शेषं चैवानुमान्य च ।\\
तदनं विकिरेद् भूमौ दद्यादापः सकृच्छकृत् }{॥}{\\
( अ० १ श्राद्धप्र० श्लो० २४१ )

{२४६ वीरमित्रोदयस्य श्राद्धप्रकाशे-}{\\
सर्वमन्नमुपादाय सतिलं दक्षिणामुखः ।\\
उच्छिष्टसन्निधौ पिण्डान्दद्याद्वै पितृयज्ञवत् ॥\\
मातामहानामप्येवं दद्यादाचमनं ततः ।

{ ( अ० १ श्राद्धप्र० श्लो० २४३ - २४४ )\\
अत्र = सर्वमिति विज्ञानेश्वरः ।\\
शङ्खलिखितौ ।\\
तृप्तान् ज्ञात्वा स्वदितमिति पृष्ट्वा शेषमन्नमनुज्ञाप्य प्रकृतादन्ना-\\
द्विकिरं कुर्यात् स्वधां वाचयित्वा
विष्टरांस्त्रींन्निदध्यात्त्रीण्येवोदपा-\\
त्राणि सतिलानि सपवित्राणि मृन्मयाश्ममयौदुम्बराणि वा । धूपगन्ध-\\
माल्यादर्शप्रदीपाञ्जनादीनि चोपहरेत् । सर्वान्नप्रकारमादाय पिण्डा-\\
न्निदध्यात् । स्वधावाचनप्रकारस्तु वक्ष्यते । विष्टरनिधानं पिण्डा\\
धारत्वेन । विष्टरः पञ्चविंशतिदर्भपिञ्जूलात्मकः । उदपात्रत्रयं अवनेज-\\
नार्थं । हेमाद्रौ बृहस्पतिस्तु अनाचान्तेषु `` तत्समीपे प्रकुर्याच्च
पिण्ड-\\
निर्वपणं तत'' इत्यादिना पिण्डदानमुक्त्वोपसृष्टोदकानां तु इत्यादिना\\
विकिरदानमाह ।\\
कात्यायनः ।\\
आचान्तेष्वित्येके । आचान्तेष्वित्यत्राचमनोत्तरकालता गम्यते\\
न चात्र सामान्योकेर्वक्ष्यमाणयमवचनानुसारादक्षय्यदानोत्तरकाल-\\
त्वेनोपसंहारः किं न स्यात् । कात्यायनसूत्रे अक्षय्योदकदानादेरेत.\\
त्सूत्रोत्तरकालविहितत्वात् । एकग्रहणादस्य पक्षस्य परमतत्वमिति\\
हेमाद्रिः । उभयशास्त्रत्वाद्विकल्प इति कर्कः ।\\
यमः ।\\
स्वधेतिवत्प्रवक्तव्यं प्रीयन्तां पितरस्तथा ।\\
अक्षय्यमन्नदानं तु वाच्यं प्रीतैर्द्विजातिभिः ॥\\
ततो निर्वापण कुर्यात्पिण्डानां तदनन्तरम् ।\\
हारीतस्तु ।\\
वाजे वाजे इत्यनुव्रज्येत्यनुवजनमुक्त्वा शेषस्य पिण्डान्पितृयज्ञ-\\
वन्निदध्यादिति ।\\
एते च कालास्तत्तच्छाखाभेदेन व्यवस्थिताः । भोजनात्पूर्वोत्तर-\\
कालयोर्व्यवस्थानन्तरमाह -\\
लौगाक्षिः ।\\
अप्रशस्तेषु यागेषु पूर्वं पिण्डावनेजनम् ।

{ }{ पिण्डदानदेशविचारः । २४७}{\\
भोजनस्य प्रशस्ते तु पश्चादेवोपकल्पयेत् ॥\\
अप्रशस्तेषु = सपिण्डीकरणात्पूर्वभाविषु । पिण्डावनेजनम् = अवाचीनपाणि.\\
ना पिण्डनिर्वपणमिति स्मृतिचन्द्रिकाकारः ।\\
हेमाद्रिस्तु पिण्डनिर्वपणारम्भपदार्थेन अवनेजनाख्येन पिण्डदानं\\
लक्ष्यते इत्याह । भोजनस्य पूर्वं भोजनात्पूर्वमित्यर्थ, इयं च व्यव-\\
स्था यस्यां शाखायां पार्वणादिश्राद्धप्रकरणे पूर्वकालता नोक्ता तद्वि\\
षयेति हेमाद्रिः ।\\
स्मृतिचन्द्रिकाकारस्तु पार्वणादिश्राद्धे पश्चादेवेत्याह । येषां तु
गृह्यादौ\\
पिण्डदानकालो नोक्तः तेषां सौकर्यादाचान्तेष्वित्येव पक्षो ग्राह्य\\
इति बहवः । इति पिण्डदानकालः ।\\
अथ पिण्डदानदेशा ।\\
तत्र साग्निकेन तावदग्निसद्भावे अग्निसन्निधौ पिण्डदानं कार्यम् ।\\
`` पिण्डपितृयज्ञवदुपचारः पित्र्ये'' इत्यनेन श्राद्धे
पिण्डपितृयज्ञधर्मा-\\
तिदेशात् । पिण्डपितृयज्ञवद्दक्षिणेनोल्लिखति अपरेण वेत्यनेनाग्नि\\
सन्निधानस्य विहितत्वात् । अत एव -\\
देवलः ।\\
हुत्वैवमग्निं पिण्डानां सन्निधौ तदनन्तरम् ।\\
पक्वान्नेन बलिमेभ्यः पिण्डेभ्यो दापयेद्विजः ॥\\
पिण्डानां सन्निधौ एवम् = उक्तेन प्रकारेण । अग्निं हुत्वा अनन्तरं\\
एभ्यः पक्वान्नेन बलिं = नैवेद्यं दद्यादित्यर्थः । एवञ्च पिण्डसन्निधान.\\
मग्निहोमस्य वदता पिण्डानामग्निसान्निध्यमुक्तं भवति । तस्मादग्नि-\\
सद्भावे तत्सन्निधावेव साग्निकेन पिण्डदानं कार्ये अग्न्यभावे तु का\\
त्यायनवचनादुच्छिष्टसन्निधाविति हेमाद्रिः ।\\
अपरार्कस्तु ।\\
अतिदेशप्राप्तस्यापि अग्निसन्निधानस्योपदेशिकेन उच्छिष्टसन्नि-\\
धानेन बाधोपपत्तेः सर्वैरपि उच्छिष्टसन्निधावेव कार्यमित्याह । न\\
चाधिकारिभेदेन अबाधोपपत्तौ बाधो न युक्तः । शरैरपि कुशानाम-\\
बाधापत्तेः तेषामपि कुशाभावे प्रतिनिधित्वेन विधानोपपत्तेः । यद-\\
पि च देवलवचः, तदपि अन्नौकरणहोमात् पूर्वं पिण्डदानकरणे उ-\\
च्छिष्टसन्निधानस्याभावादग्निसन्निधानज्ञापकमिति न कश्चिद्विरोध ।\\
उच्छिष्टसन्निधिस्तत्समीपे सद्देशो ग्राह्यः । तथा च-

२४८ वीरमित्रोदयस्य श्राद्धप्रकाशे-

{व्यासः ।\\
अरत्निमात्रमुत्सृज्य पिण्डांस्तत्र प्रदापयेत् ।\\
यत्रोपस्पृशतां वापि प्राप्नुवन्ति न बिन्दवः ॥\\
अरत्निमात्रमिति नारत्निनियमार्थे, किन्तु समीपे शुचिदेशोपलक्षणा\\
र्थं यत्रोपस्पृशतामिति वाक्यशेषात्, अत एवात्रिणाऽरत्नित्रयमुक्तम्,\\
`पितॄणामासनस्थानादग्रतस्त्रिष्वरत्निषु ' । उच्छिष्टसन्निधान तत्रोच्छि\\
ष्टासनसन्निधानम् । यत् त्रिषु अरत्निषु स्थानं तदेवोच्छिष्टसन्निधान\\
न तु उच्छिष्टासनसन्निधानमेव पिण्डस्थानमित्यर्थः । अप्रतः = पुर\\
स्तात्, न तु पश्चात्पार्श्वयोर्वा, अत एव -\\
देवलः ।\\
पुरस्तादुपविश्यैषां पिण्डावापं निवेदवेत् ।\\
ततस्तैरभ्यनुज्ञातो दक्षिणां दिशमेत्य सः ॥\\
उपलिप्ते शुचौ देशे स्थानं कुर्वीत सैकतम् ।\\
मण्डलं चतुरस्त्नं वा दक्षिणावनतं महत् ॥\\
पिण्डावापः = पिण्डदानम् । तस्य निवेदनं च पिण्डदानमहं करिष्य\\
इति प्रकारेण । अभ्यनुज्ञानमपि चैवं कुरुष्वेत्येवं कार्यम् ।
दक्षिणादिक्\\
श्राद्धकर्त्रपेक्षयेति कल्पतरुः ।
श्राद्धदेशोत्तरस्थदक्षिणाग्न्यपेक्षवेति\\
वाचस्पतिः । सैकतम् = वालुकानिर्मितम् । यत्तु -\\
ब्रह्मपुराणे ।\\
ततो दक्षिणपूर्वस्यां कार्या वेदी यथाविधि ।\\
इति दिगन्तरावधानं तच्छास्त्रान्तरविषयायति स्मार्त्तभट्टाचार्यः ।\\
पिण्डदाने देशविशेषमाह-\\
देवल. ।\\
छायायां हस्तिनश्चैव वस्तदौहित्रसन्निधौ ।\\
बस्तः = छागः । न चेयं हस्तिच्छाया कालविशेषरूपा, देशप्रस्तावे\\
पाठात् । इदं च देशविधानं फलातिशयार्थे न पुनर्नियमार्थं हस्ति\\
च्छायाया: -\\
भोजयेत्तु कुलेऽस्माकं छायायां कुञ्जरस्य च ।\\
आकल्पकालिकी तृप्तिस्तेनास्माकं भवेदिति ॥\\
वायुपुराणादौ प्रशस्ततरत्वाभिधानात् । अत एव देवलेन भोज\\
नात्पूर्वमेव पिण्डदानस्य विहितत्वान्न तत्रैवायं दशः । किन्तु
स्मृत्यन्तरे

{ }{ पिण्डदानेतिकर्तव्यता । २४९}{\\
विहितभोजनोत्तरकालीनपिण्डदानेऽपि स भवत्येव, प्रशस्ततरत्वबो-\\
धकवाक्ये पिण्डदानसामान्यस्यैव प्रकरणादुपस्थितत्वात् भोजनपू\\
र्वकालीनस्य पिण्डदानस्यानुपस्थितत्वात् । तस्मात्प्राशस्त्यातिश-\\
यार्थमेवेद देशविधानमिति हेमाद्रिः । इति पिण्डदानदेशाः ।\\
अथ पिण्डदानेतिकर्तव्यता ।\\
तत्र पिण्डदानानुज्ञाग्रहणे विशेषमाह -\\
देवलः ।\\
अथ संगृह्य कलशं सदर्भं पूर्णमम्भसा ।\\
पुरस्तादुपविश्यैषां पिण्डावापं निवेदयेत् ॥\\
सदर्भम् = सपवित्रम् । अम्भसा पूर्णं कलश संगृह्य गृहीत्वा ।
ब्राह्मणानां\\
पुरस्तादुपविश्य पिण्डदाननिवेदन कुर्यादित्यर्थः । निवेदनप्रकारस्तु\\
प्राग्दर्शितः । इदं च कलशग्रहणं येषां तजलेन `` ततः पानीयकुम्भेन\\
तर्पयेत्प्रयतः पितृन्'' इत्यादिना किञ्चित्तर्पणादि कर्म विहित तेषामेव
।\\
अन्येषां तु तद्ग्रहणं विनापि अनुज्ञाग्रहणं भवत्येव । अत एव कलश-\\
ग्रहणमनुक्त्वैवानुज्ञाग्रहणमाह -\\
शालङ्कायनः ।\\
पिण्डावापमनुज्ञाप्य यतवाक्कायमानसः ।\\
सतिलेन ततोऽन्नेन पिण्डान् सर्वेण निर्वपेत् । इति\\
पिण्डदाने आधारविशेषमाह-\/-\\
देवलः ।\\
उपलिप्ते शुचौ देशे स्थानं कुर्वीत सैकतम् ।\\
मण्डले चतुरस्रं वा दक्षिणावनतं महत् ॥\\
स्थानं पिण्डाधारभूतं स्थलम् । मण्डल = वृत्तम् । चतुरस्त्र = चतुष्कोणम्
।\\
वेदिपरिमाणं च-\/-\\
ब्राझे-\\
हस्तमात्रा तथा भूमेश्चतुरङ्गुलमुच्छ्रिता ।\\
पिण्डनिर्वपणार्थाय रमणीया विशेषतः ॥\\
मातामहपार्वणे भिन्ना वेदिरिति श्राद्धकल्पः । इदं च वेदिकरणं\\
छन्दोगयजुर्वेदिव्यतिरिक्तविषयम् । तेषां वेद्यश्रवणादुच्छिष्टसमीप\\
एव पिण्डदानश्रवणाश्चेति केचित् । तन । उच्छिष्टसमीपेऽपि वेदेः कर्तुं\\
वी० मि ३२

{२५० वीरमित्रोदयस्य श्राद्धप्रकाशे-}{\\
शक्यत्वात् । अत एव -\\
भविष्ये ।\\
मण्डलं चतुरस्रं वा निर्मायोच्छिष्टसन्निधौ ।\\
ॐ निहन्मीतिमन्त्रेण ततोऽप्यपहता इति ॥\\
पठन् रेखां दक्षिणाग्नां कुशमूलेन वै लिखेत् ।\\
अपरार्कसकलमैथिलस्वरसोप्येवं । हेमाद्रिस्तु वेदिकरणं केषाञ्चि\\
देव शाखिनां `` सुपलिप्ते महीपृष्ठे गोशकृन्मूत्रवारिणे'' ति मत्स्यपुरा-\\
णे महीपृष्ठस्याऽधारत्वविधानादित्याह । पिण्डसंस्कारमाह-\\
देवलः ।\\
एकदर्भेण तन्मध्यमुल्लिखेत्विश्च तं त्यजेत् ।\\
एकदर्भः = एकदर्भशिखेति हेमाद्रिः । एकदर्भेणेति सामगातिरिक्त .\\
विषयम् ।\\
सामगानां तु -\\
पिञ्जुल्याद्यभिसंगृह्य दक्षिणेनेतरात्करात् ।\\
अन्वारभ्य च सव्येन कुर्यादुल्लेखनादिकम् }{॥}{\\
इति छन्दोगपरिशिष्टोक्तपि}{ञ्जु}{ल्या रेखाकरणं बोध्यम् ।\\
पि}{ञ्जु}{ली= पवित्रम् ।\\
ब्रह्माण्डपुराणे तु त्रिभिः कुशैः स्फेनापि वा लेखाकरणमुक्तम् ।\\
वज्रेण वा कुशैर्वापि उल्लिखेत् तन्महीं द्विजः ।\\
वज्रो वै स्फ्य इति श्रुतेर्वज्रः }{स्फ्यः}{ । इदं च देशप्राप्तस्यापि\\
वज्रस्य कुशैर्बाधे प्रतिप्रसवार्थे पुनः श्रवणं, एवं चोभयोर्विकल्पः ।
स्मृ-\\
त्यर्थसारेऽप्येवं । हेमाद्रिस्तु स्फ्याभावे कुशविधिमाह -
स्फ्यादिग्रहणं\\
च वामेन कार्ये, उल्लेखनन्तु वामान्वारब्धेन दक्षिणेन कार्यम् अत एव\\
छन्दोगपरिशिष्टे ।\\
परिग्रहणमात्रं हि सव्येन स्यादिति ध्रुवम् ।\\
पिञ्जुल्याद्यभिसंगृह्य दक्षिणेनेतरात्करात् ॥\\
अन्वारभ्य च सव्येन कुर्यादुल्लेखनादिकम् । इति ॥\\
कल्पतरौ तु सव्येन गृहीत्वा उभाभ्यां हस्ताभ्यामुल्लिखेदित्युक्तम् ।\\
अत एव पारस्करः ।\\
कराभ्यामुल्लिखेत् स्फ्येन कुशैर्वापि महीं द्विजः ।\\
तच्च यथा रेखायां क्रियमाणायां दक्षिणमुष्टेरुपरि काममुष्टिर्भ-

{ }{ पिण्डदानेतिकर्तव्यता २५१.}{\\
वति तथा कुर्यात् । अत एव -\\
वायुपुराणे ।\\
सव्योत्तराभ्यां पाणिभ्यां कुर्यादुल्लेखनं द्विजः।\\
तच्चोल्लेखनं आग्नेय्याभिमुख्येन पराङ्मुखतया कर्त्तव्यम् ।\\
तथा च पिण्डपितृयज्ञे ।\\
आपस्तम्बः ।\\
दक्षिणाप्राचीं पराचीं वेदिमुद्धृत्येति ।\\
दक्षिणाप्राचीम् = आग्नेयीम् । पराचीम् = पराङ्मुखीम् । सर्वे च आग्ने-\\
याभिमुखोऽवनेजनादिक कुर्यादिति हेमाद्रिः ।\\
अन्ये तु दक्षिणादिक् पितॄणामिति श्रुतेर्दक्षिणामुखत्वेनैवेति ।\\
रेखाकरणं च त्रिवारम् ।\\
एकदर्भेण तन्मध्यमुल्लिखोत्त्रिश्च तं त्यजेत् ।\\
इति देवलवचनात् । इदं च कातीयभिन्नानां, तेषां सकृदेव ``अथ\\
दक्षिणेनान्वाहार्यपचनं सकृदुल्लिखति'' इति श्रुतेः । अत एव हला-\\
युधेन सकृदेवेत्युक्तम् हेमाद्रिस्तु त्रिर्ग्रहणं पिण्डत्रयाभिप्रायेण
वर्गत्रया-\\
भिप्रायेण वा द्रष्टव्यम् । एवं च यत्र वर्गद्वयं तत्र द्वे रेखे इत्याह ।
रे-\\
खाकरणे मन्त्र उक्तो-\/-\\
ब्रह्मपुराणे ।\\
ॐ निहम्नि सर्वं यदमेध्यवद्भवेत् हताश्च सर्वेऽसुरदानवा मया ।\\
रक्षांसि यक्षाश्च पिशाचसङ्घा हता मया यातुधानाश्च सर्वे ॥\\
अनेन मन्त्रेण सुसंयतात्मा वेदीं च सर्वां सकृदुल्लिखेच्च ।\\
पिण्डपितृयज्ञे कात्यायनेन तु मन्त्रान्तरमुक्तं `` तद्दक्षिणेनोल्लिख\\
त्वपहता इत्यादिना । अनयोश्च समुच्चयः ।\\
नच दृष्टार्थानां कथं समुचयः प्रत्युतातिदेशतः प्राप्तस्यापहता\\
इत्यस्य बाध एव स्यादिति वाच्यं ।\\
ॐ निहन्मीति मन्त्रेण ततोप्यपहता इति ।\\
पठन् रेखां दक्षिणाग्रां कुशमूलेन संल्लिखेत् ॥\\
इति भविष्यपुराणवचनात् समुच्चयः ।\\
रेखाकरणामन्तरमभ्युक्षणमुक्तमाश्वलायनेन ``तामभ्युक्ष्य सकृदा.\\
छिन्नैरवस्तीर्ये'' त्यादिना ।\\
अभ्युक्षणानन्तरमभिमन्त्रणमुक्तमापस्तम्बेन । `` अवक्षायन्तु पितरो

२५२ वीरमित्रोदयस्यप्रकाशे-  

{मनीयवस इत्यभिमन्त्र्येति ।\\
उल्लेखनानन्तरमुल्मुकनिधाने-\/-\\
कात्यायन. ।\\
उल्मुकं परस्तात्करोति ``ये रुपाणी ''ति । परस्तात् = रेखापरस्ता-\\
त्पुरोभागे । इदं च निरग्नेरुल्मुकनिधानं न भवति अग्नेः प्रतिनिध्य\\
भावेन लौकिकाङ्गारस्य ग्रहीतुमशक्यत्वादिति यज्ञपतिपशुपतिप्रभृ-\\
तयः । सति निरग्नेः श्राद्धाधिकारे निषादस्थपतीष्टिवल्लाकिका-\\
ग्न्युपादानस्याविरुद्धत्वमिति भानूपाध्यायादयः । अत्र दर्भास्तरणं\\
कार्यम् ।\\
तथा च देवलः ।\\
तस्मिँस्थाने ततो दर्भानेकमूलाञ्छिवान् बहून् ।\\
दक्षिणाग्रानुदक्पादान् सर्वांस्तांस्तृणुयात्समम् ॥\\
तस्मिन् = पिण्डस्थाने । दर्भान् = कुशान् । तदभावे काशादि । एकमू-\\
लान् = सलूँलग्नबहुशिखान् । शिवान् = साग्रत्वादिगुणयुक्तान् । ते च\\
समूला: सकृदाच्छिन्ना इत्याह-\/-\\
कात्यायनः ।\\
उपमूलं सकृदाच्छिन्नानि रेखग्यां कृत्वेति ।\\
उपमू = समूलमिति हेमाद्रिः । अत्र लुनातीति शेषः । अत्र च दर्भा-\\
निति बहुवचन कपिञ्जलवत्त्रित्वपरमिति न वाच्यम् । दर्भास्तरणस्य\\
दृष्टार्थत्वात्पिण्डाधारत्वेन बहूनामेव प्राप्तेः, अत एव पृथक् बहूनिति\\
ग्रहणमेव कल्पत इति हेमाद्रि । त्रय एवेति पितृभक्तिः । उदक्पादान् =\\
उदङ्मुखान् । स्तृणुयत् = विस्तारयेत् । समं = समाग्रतया समूलतया\\
वा । इदं च दर्भास्तरणं यदि रेखा दक्षिणाग्रा तदा दक्षिणाग्रं क-\\
र्त्तव्यं यदा तु आग्नेयी दिशमाश्रित्य लेखा क्रियते तदा तदभिमुख-\\
मेव कुर्यात् । अत एव वायुपुराणे आग्नेयाभिमुखत्वमुक्तं ।\\
प्राग्दक्षिणाग्रान्नियतो दद्यात् पिण्डाननन्तरम् ।\\
प्राग्दक्षिणा = आग्नेयीति हेमाद्रि ।\\
दर्भास्तरणे मन्त्र उक्त आपस्तम्बेन ।\\
सकृदाच्छिन्नेबर्हितृर्णामृदुस्योनं पितृभ्यस्त्वा स्तृणाम्यहम् }{।}{\\
अस्मिन्सीदन्तु मे पितरः सोम्याः पितामहाः प्रपितामहाञ्चानु-\\
गैः सहेति सकृदाछिन्नेन वर्हिषा वेदिं स्तृणाति ।

{ }{ पिण्डदानेतिकर्तव्यता । २५३}{\\
शङ्खलिखिताभ्यां तु पिण्डभूमौ विष्टरत्रयनिधानमुक्तं-\/-\\
विष्टंराँस्त्रीन्निदध्यात् इति ।\\
अत्र च संख्यासास्यादेकैकपिण्डाधारत्वेनैकैको विष्टरो निधेय\\
इति हेमाद्रिः । विष्टरनिधाने नामगोत्राद्युच्चारणपूर्वत्वमुक्तं यमेन ।\\
विष्टरांस्त्रीन् वपेत् तत्र नामगोत्रसमन्वितान् ।\\
अद्भिरभ्युक्ष्य विधिवत्तिलैरभ्यवकीर्य च ॥\\
ततो दर्भेषु तं हस्तं निर्मृजेल्लेपभागिनम् ।\\
नामगोत्रसमन्वितानिति विष्टरविशेषणमिति ।\\
हेमाद्रिः ।\\
वस्तुतस्तु निर्मजेल्लेपभागिनमिति उपसंहारात् पिण्डदानविध्यभावा-\\
पत्तेः विष्टरांस्त्रीन्निदध्यादित्यध्याहारस, वपेत् पिण्डानितिचाघ्याहार\\
इति यमवचनं व्याख्येयम् । कल्पतरुरप्येवम् ।\\
दर्भास्तरणानन्तरं आवाहनपूर्वकं पितॄणां स्थानकल्पनामाह-\\
देवलः ।\\
अथ साञ्जलिरुत्थाय स्थित्वा चावाहयेत् पितॄन् ।\\
पितरो मे प्रसीदन्तु प्रयान्तु च पितामहाः ॥\\
इति सङ्कीर्त्तयंस्तूष्णीं तिष्ठेत् क्षणमनुच्छृसन ।\\
आवाहयित्वा दर्भाग्रैस्तेषां स्थानानि कल्पयेत् ॥\\
तेषां पितॄणां स्थानानि कल्पयेत् । आस्तृतभूमाविति शेषः । करे\\
दर्भान्गृहीत्वा पित्रे इदं स्थानमित्येवं क्रमेण दर्भाग्रैर्विनिर्दिशेत्
।\\
स्थानकल्पनानन्तरं तत्र मार्जनं तिलविकिरणं चाह स एव ।\\
तेष्वासीनेषु पात्रेण प्रयच्छेन्मार्जनोदकम् ।\\
प्रक्षाल्य विकिरेत्तत्र नानावर्णोस्तिलानपि ॥\\
मार्जनोदकं = अवनेजनोदकमिति कल्पतरु । अत्रावनेजनसाधनं\\
पात्र उक्तम्, आपस्तम्बेन तु मञ्जलिरुक्तः , मार्जयन्तां मम पितरो मा-\\
र्जयन्तां मम पितामहाः मार्जयन्तां मम प्रपितामहा इत्येकरेखायां\\
त्रीनुदकाञ्जलीभिनयति । आश्वलायनेन तु मन्त्रान्तरमुक्तं प्राची-\\
नावती रेखां त्रिरुदकेनोपनयेत् शुन्धन्तां पितरः, शुन्धन्तां पिता-\\
महाः शुन्धन्तां प्रपितामहा इति ।\\
{[} कात्यायनः ।\\
उदपात्रेणावनेजयनित्यपसव्यं सव्यमाद्धरणं सामर्थ्यात् । ]

{२५४ वीरमित्रोदयस्य श्राद्धप्रकाशे-\\
}{लेखां सकृदाच्छिन्नोपस्तीर्णां अवनेजयति अपसव्यम् = दक्षिणेन\\
हस्तेन जलं ददातीत्यर्थः । अपसव्यं तु दक्षिणमित्यमरः । हेमाद्रिरपि\\
मनुवाक्यं अपसव्येन दक्षिणेनेति व्याख्यातवान् । सव्येनोद्धरणम्\\
वामहस्तेन तस्यान्नस्योद्धरणं कार्यं । अत्र हेतुः सामर्थ्यादिति । वाम-\\
स्य हि उद्धरणे सामर्थ्यं दक्षिणस्यावनेजने व्यापृतत्वात् । हलायुधेन तु\\
सव्येन वोद्धरणसामर्थ्यादिति पाठं लिखित्वा एवं व्याख्यातमपसव्यं\\
वामहस्तेन यथा स्यात्तथा सव्येन दक्षिणेन तत्र हेतुः , उद्धरणसामर्थ्यां\\
दिति, सहि कर उद्धरणे समर्थः । तथा च दक्षिणवामयोर्विकल्पः ।\\
स च शाखाभेदात् व्यवस्थितः । विशेषान्तरमाह -\\
कात्यायनः ।\\
असाववनेनिक्ष्वेति यजमानस्य पितृप्रभृतित्रीनिति ।\\
असौ इत्युपलक्षणं सम्बन्धनामगोत्रादेः ।

{तथा च व्यासः ।\\
पिण्डोदकप्रदानं तु नित्यनैमित्तिकष्वपि ।\\
{[} आलव्य नामगोत्राभ्यां कर्तव्यं सर्वदैव हि ॥ {]}\\
पिण्डदानेषु पिण्डोदकं अवनेजनोदकं नामगोत्रेणालप्य सम्बो-\\
ध्य कर्त्तव्यमित्यर्थः । यजमानग्रहणं पिण्डपितृयज्ञस्य
अध्वर्युकर्त्तृक-\\
त्वात् । अवनेजनोदकं सतिलं सपुष्पं च कर्त्तव्यं । तथा च-\\
उशनाः ।\\
तिलोन्मिश्रितेनोदकेनासिच्य दर्भास्तीर्णायां भूमौ पिण्डान्नि\\
वेदयेत् ।\\
ब्रह्मपुराणे ।\\
सपुष्पं जलमादाय तेषां पृष्ठे पृथक् पृथक् ।\\
अप्रदक्षिणं तु निर्णिज्यात् गोत्रनामानुमन्त्रितम् ।\\
अत्र कात्यायनमते अमुकगोत्र अस्मत्पितः अमुकशर्म्मन्नवनेनि-\\
क्ष्वेति प्रयोगवाक्यं द्रष्टव्यम् आश्वलायनमते तु शुन्धन्ता पितर इत्यादि
।\\
आपस्तम्बेनापि मार्जयन्तां मम पितर इत्याद्युक्तम् । अत्राश्वावनेजना-\\
त्प्राक् देवताभ्यः इति त्रिः पठनीयम् ।\\
आद्यावसाने श्राद्धस्य त्रिरावृत्या जपेत्सदा ।\\
पिण्डनिर्वपणे चैव जपेदेतत्समाहितः ॥\\
इति ब्रह्मपुराणात् ।\\
न च पिण्डनिर्वपणे चेति वाक्यात्पिण्डनिर्वापान्पूर्वमेवायं अपः

{ }{ पिण्डदानेऽन्नविशेषः । २५५}{\\
स्यान्नावनेजनात्पूर्वमिति वाच्यं । अवनेज्य दद्यादिति कात्यायनसूत्र-\\
विरोधादिति मैथिलाः ।\\
अन्ये तु यथाश्रुतवाक्यात्पिण्डदानात्प्रागेवेत्याहुः ।\\
अवनेजनानन्तरं पिण्डाः कार्या इत्याह-\\
जातूकर्ण्यः ।\\
कृत्वावनेजनं कुर्यात्त्रीन्पिण्डांस्तु यथाविधि ।\\
त्रीनिति पित्राद्येकवर्गापेक्षया कात्यायनादिवाक्ये त्रींस्त्रीनिति\\
वीप्साश्रवणात्, श्राद्धे मातामहवर्गेऽपि पिण्डनिर्वापस्थावगमात् ।\\
अथ पिण्डदाने अन्नविशेषः ।\\
तत्र यदान्नौकरणात्पूर्वं पिण्डदानं तदाऽग्नौकरणार्थकचरुणा\\
कर्तव्यमित्याह -\\
देवलः ।\\
ततश्चरुमुपादाय स पवित्रेण पाणिनेत्यादिना । पिण्डार्थं विभजे-\\
दिति वक्ष्यमाणेन सम्बन्धः । अग्नौकरणोत्तरकालं पिण्डदाने तच्छे-\\
षेण पिण्डाः कार्या इत्याह-\\
मनुः ।\\
त्रींस्तु तस्माद्धविशेषात्पिण्डान्कृत्वा समाहितः ।\\
{[} अ० ३ श्लो० २१५ {]}\\
हविःशेषात् = अग्नौकरणशेषात् । भोजनोत्तरकालीनेषु अन्नमाह -\/-\\
कात्यायनः ।\\
सर्वमन्नमेकत उद्धृत्य उच्छिष्टसमीपे दर्भेषु त्रींस्त्रीन्पिण्डानवनेज्य\\
दद्यात् ।\\
व्याख्यातं चेदं वचनं प्राक् । एतञ्च सर्वप्रकारमन्नमग्नौकरणाव\\
शिष्टेन चरुणा मिश्रणीयमाह-\/-\\
आश्वलायनः ।\\
यदन्नमुपभुक्तं तत्स्थालीपाकेन सह पिण्डार्थमुद्धत्येति ।\\
उपभुक्तादन्नात्किञ्चिदुद्धृत्य स्थालीपाकेनाग्नौकरणशिष्टेन सह\\
संयोज्य पिण्डान् दद्यादित्यर्थः । इदं च यत्र पिण्डपितृयज्ञकल्पो\\
विहितः । यथा-\/-\\
अन्वष्टक्यं च पूर्वेधुर्मासिमास्यथ पार्वणम् ।\\
काम्यमभ्युदयेऽष्टम्यामेकोद्दिष्टं तथाष्टमम् ॥\\
इति वाक्योक्तेष्वाद्येषु चतुर्षु तत्रैव तेन संयोजनं अन्यत्र तु के-\\
वलेनैव भुक्तशिष्टेन कार्यमिति वृतिः । अत्र च स्थालीपाकेन सहेत्य

{२५६ वीरमित्रोदयस्य श्राद्धप्रकाशे-}{\\
मिधानात्केवलावसथ्याग्निमता पिण्डपितृयज्ञार्थं स्थालीपाकं श्रप-\\
यित्वा आच्छादनदानात्तं श्राद्धं निर्वत्य स्थालीपाकस्यान्नेनाग्नोकरणं\\
होमं कृत्वा ब्राह्मणाम्भोजयित्वा श्राद्धशेषेण स्थालीपाकान्नमेकीकृत्य\\
पिण्डप्रदानं कर्त्तव्यमित्यमावास्याश्राद्धे व्यतिषङ्गः फलितो भव\\
तीति कल्पतरुप्रभृतयः । एतच्चानं मघ्वाज्यतिलयुक्तं कर्त्तव्यमुक्तं-\/-\\
ब्राह्मे -\/-\\
मध्वाज्यतिलसंयुक्तं सर्वव्यञ्जनसयुतम् ।\\
उष्णमादाय पिण्डन्तु कृत्वा बिल्वफलोपमम् ॥\\
दद्यात्पितामहादिभ्यो दर्भमूलाद्यथाक्रमम् ।\\
पितामहपदं पितृपरमिति श्राद्धचन्द्रिका । अत्र च मध्वादिभिः पि-\\
ण्डकरणं न तु अन्नकृतस्य पिण्डस्य मध्वादियोगः ।\\
अतश्च-\\
मधुसर्पिस्तिलैर्युक्तान् त्रीन्पिण्डान्निर्वपेद्बुधः ।\\
इति देवलवाक्यमपि एवमेव व्याख्येयम् । मूलभूतश्रुत्यन्तरक-\\
ल्पनागौरवात् । श्राद्धकल्पे तु मधुसंयुक्तान्ननिर्मितान् पिण्डान्मध्वा-\\
दियुक्तान्कृत्वा दद्यादित्युकम् । एतच्च मध्वादित्रयदानं न नियमार्थं\\
किन्तु फलातिशयार्थं , स्मृत्यन्तरे द्वयोरेकस्यापि वा श्रवणादिति\\
हेमाद्रिः । मधुदानं च कलियुगातिरिक्तविषयं `` श्राद्धं मांसं तथा मधु ''\\
इति श्राद्धप्रयोगे कलिवर्ज्येषु वर्जनात् । पिण्डप्रमाणमाह -\\
व्यासः ।\\
द्विहायनस्य वत्सस्य विंशत्यास्ये यथासुखम् ।\\
तथा कुर्यात्प्रमाणं तु पिण्डानां व्यासभाषितम् ॥\\
द्विहायनो = द्विवर्षवयस्कः । प्रमाणान्तरमुक्तं -\\
ब्रह्माण्डे ।\\
त्रीन्पिण्डानानुपूर्वेण साङ्गुष्टमुष्टिवर्द्धनात् ।\\
साङ्गुष्टो मुष्टिर्यावान् तावद्वर्द्धन पुष्टिर्येषां ते इति विग्रहः ।
इदं\\
च परिमाणं मातामहपिण्डविषयम् ।\\
पृथक् मातामहानां च केचिदिच्छन्ति मानवाः ।\\
त्रीन्पिण्डानानुपूर्वेण साङ्गुष्ठान्मुष्टिवर्द्धनान् }{॥}{\\
इति वायुपुराणे तत्पिण्ड एव एतत्परिमाणस्यो-\\
क्तत्वादिति पितृभक्तिः ।

  पिण्डदानेतिकर्तव्यता । २५७

{अङ्गिरा अपि ।\\
कपित्थबिल्वमात्रान् वा पिण्डान्दद्याद्विधानतः ।\\
कुक्कुटाण्डप्रमाणान्वा यदि वामलकैः समान् ।\\
बदरेण समान् वापि दद्याच्छ्रद्धा समन्वितः ।\\
एषां च शक्तिभेदेन व्यवस्था वेदितव्या ।\\
अत्र व्यवस्थान्तरयुक्तानि कानि चित्परिमाणान्याह\\
मरीचिः ।\\
आर्द्रामलकयुक्तांस्तु पिण्डान्कुर्वीत पार्वणे ।\\
एकोद्दिष्टे बिल्वमात्रं पिण्डमेकं तु निर्वपेत् ॥\\
नवश्राद्धे स्थूलतमं तस्मादपि तु निर्वपेत् ।\\
तस्मादपि स्थूलतरमाशौचे प्रतिवासरम् }{॥}{\\
अत्र प्रकरणादेव पार्वण सम्बन्धे सिद्धे पार्वणग्रहणंव्यवस्थार्थम् ।\\
अत्र चामलकमात्रानेव पार्वण कुर्वीतेति न व्याक्येयम्, किन्तु\\
आमलकमात्रान्पार्वण एवेति ।\\
अतश्च पार्वणेऽप्यामलकाधिकपरिमाणपिण्डनिर्वापः ।\\
अत एवाचारोऽप्यवम् । बिल्वमात्रमित्यनेन बिल्वन्यूनमामलका-\\
दिपरिमाणमेव निवर्त्यते नत्वधिकं तथैवाचारात् । नवश्राद्धं = आशौच\\
मध्ये प्रथमतृतीयादिविषमदिनेषु विहितं श्राद्धम् । आशौचमध्ये श्रा-\\
द्धमन्तरेणैवावयपिण्डदानं ? प्रतिदिवसं यत्क्रियते तस्मिन् ।\\
पिण्डत्रयस्योत्तरोत्तरं आधिक्यमुक्तं मैत्रायणीयसूत्रे ।\\
पितामहस्य नाम्ना स्थवीयांसं मध्यमं, प्रपितामहस्य नाम्ना\\
स्थविष्ठं दक्षिणम् ।\\
स्थवीयासं = पितृपिण्डापेक्षया ।\\
अत्र सव्यजानुनिपातनमुक्तम् - ब्राह्मे ।\\
मधुसर्पिस्तिलयुतांस्त्रीन्पिण्डान्निर्वपेद् बुधः ।\\
जानु कृत्वा तथा सव्यं भूमौ पितृपरायणः ।\\
अत्र च श्रुतिक्रमेण पाठक्रमं बाधित्वा पिण्डनिर्वापात्पूर्वे जानु\\
निपातनं कार्यम् । पिण्डदाने पात्रमाह-\\
मरीचिः ।\\
पात्राणां खङ्गपात्रेण पिण्डदानं विधीयते ।\\
राजतौदुम्बराभ्यां वा हस्तेनैवाथ वा पुनः ॥\\
वी० मि ३३

{२५८ वीरमित्रोदयस्य श्राद्धप्रकाशे-}{\\
औदुम्बरं = ताम्रमयम् । पिण्डदाने पितृतर्थमुक्तं षट्त्रिंशन्मते -\\
निर्वपेत्पितृतीर्थेन स्वधाकारमनुस्मरन् ।\\
पिण्डदाने मन्त्रमाह-\\
कात्यायनः ।\\
यथावन्निर्णिकं पिण्डान्ददाति असावेतत्त इति । ये च त्वाम-\\
न्विति चैक इलि ।\\
यथावन्निर्णिक्त = यत्र यत्र येन क्रमेण यस्यावनेजनं कृतं तत्र तत्र
तेनैव\\
क्रमेण तस्य पिण्डं दद्याव । असाविति गोत्र सम्बन्धनाम्नामभिधानं\\
विवक्षितम् । अत एव पारस्करः ।\\
अर्धदाने च सङ्कल्पे पिण्डदाने तथाक्षये ।\\
गोत्रसम्बन्धनामानि यथावत्प्रतिपादयेत् }{॥}{\\
सङ्कल्पे = अन्नत्यागे । अत्र च गोत्रादीनां सम्बोधनान्तानामुच्चारणं\\
कार्यम् । असाविति प्रथमानिर्देशात्, असम्बोधने अवनेनिक्ष्वेत्यस्य\\
वैयधिकरण्यापत्तेश्च । अत एव -\\
बौधायनः ।\\
एतन्ते ततासौ ये च त्वामन्त्रान्विति ।\\
तत शब्दस्य सम्बोधनान्तत्वमाह । एतदिति = निर्वप्समानपिण्डनि\\
र्देशः । न च पुंलिङ्गस्य पिण्डशब्दस्य कथं नपुंसकलिङ्गेन निर्देश\\
इति वाच्यम् । पिण्डशब्दस्य कुडन्यस्य पिण्डं पततीत्यादौ महाभा\\
ष्ये नपुंसकलिङ्गेनापि प्रयोगदर्शनात् । अत एव पिण्डशब्दस्य पुंल्लि\\
ङ्गत्वाद्यजुर्वेदिनामपि एष ते पिण्ड इत्येव प्रयोग इति मैथिलमतमपा-\\
स्तम्, पिण्डशब्दस्य नपुंसकलिङ्गेऽपि प्रयोगात् । तेनैतच्छब्देनैव\\
पिण्डस्योपस्थितत्वादध्याहारं विनैव प्रयोगोपपत्तेरिति स्मृतिचन्द्रिकाका\\
रादयः ।\\
हेमाद्रिस्तु `` इषे त्वे '' त्यादौ छिनद्मीत्यध्याहारवदत्रापि वाक्यपू-\\
रणाय अन्नपदं पिण्डपदं वाध्याहार्यं अत एव कर्कभाष्ये अन्नशब्द उ\\
पलक्षणार्थः । तयोरपि च व्यवस्थोक्ता लौगाक्षिणा ।\\
महालये गयाश्राद्धे प्रेतश्राद्धे दशाहिके ।\\
पिण्डशब्दप्रयोगः स्यादन्नमन्यत्र कीर्त्तयेत् ॥ इति ।\\
एतद्वचनोपत्तिषु श्राद्धेषु एतत्ते पिण्डमिति प्रयोगः । अन्यत्र\\
तु एतते अन्नमिति प्रयोग इत्याह ।

{ }{ पिण्डदानेतिकर्तव्यता २५९}{\\
अन्ये तु इषे त्वेत्यादौ क्रियाविशेषस्यापेक्षितत्वाद्युक्तः छिन.\\
द्मोत्यध्याहारः, प्रकृते तु द्रव्यविशेषस्यैतच्छब्देनैवसमर्पणादग्नय इदं\\
न मम इतिवन्नाध्याहारः । तेन प्रकृतौ एष ते पिण्ड इति गोभिलवा-\\
क्यात् गौभिलीयानामेष ते पिण्ड इत्येष प्रयोगः । वाजसनेयिनामपि\\
` एतत्ते पिण्डमित्युक्त्वा दद्युर्वाजसनेयिन'' इति भविष्यपुराणवचनादेत-\\
त्ते पिण्डमित्येव प्रयोगः । एतस्यापि वचनस्य प्रकृतावेवाम्नातत्वात् ।\\
अन्येषां तु प्रागुक्तलौगाक्षिवचनाद्व्यवस्थेत्याहुः । शाखापरतया वा\\
व्यवस्थेति दिक् । अत्र च स्वधानमःशब्दौ प्रयोज्यावित्याह-\\
शाठ्यायनिः ।\\
असावेतत्त इत्युक्त्वा तदन्ते च स्वधानमः ।\\
अत्र च नमः शब्दान्तं मन्त्रमुचार्य पुनः पित्रादीन् चतुर्थ्यन्तपदे-\\
नोद्दिश्येदं न ममेति उच्चार्य, शिष्टाचारादिति हेमाद्रिः । अतश्चैवं
प्रयो\\
गवाक्यं कार्तायानाम्, अमुकगोत्रास्मत्पितरमुकशर्मन् एतत्तेऽन्न पिण्ड\\
वा स्वधा नमः, इदममुकगोत्रायास्मत्पित्रेऽमुकशर्मणे न ममेति । अत्र\\
आचारावगतवाक्यात्पूर्वं ये च त्वामन्विति मन्त्रः पठनीय इत्याह-\\
कात्यायनः ।\\
ये च त्वामन्विति चैके ।\\
एकग्रहणं पक्षप्राप्त्यर्थमिति स्मृतिचन्द्रिकाकारः ।\\
हेमाद्रिस्तु एकग्रहणं परमतत्वसूचनार्थे तेन वाजसनेयिनामयं न.\\
भवत्येव । अत एव शतपथे ये च त्वामन्वित्युहैक आहुः तदु तथा\\
न ब्रूयादिति । हलायुधस्वरसोप्येवम् ।\\
बौधायनस्तु मन्त्रान्तरमाह -\\
एतत्ते तत असौ ये त्वामत्रान्विति ।\\
गोभिलोऽपि ।\\
असावेष ते पिण्डो ये चात्र त्वानुतेभ्यश्च स्वधेत्यनुषजेत् ।\\
विष्णुरपि ।\\
उच्छिष्टसन्निधौ दक्षिणाग्रेषु दर्भेषु पृथिवीदर्विरक्षितेत्यकं पि-\\
ण्ड पित्र्ये निदध्यात्, अन्तरिक्षं दर्वि रक्षितेति द्वितीयं पितामहाय,\\
द्यौद्धिरक्षितेति तृतीयं प्रपितामहाय । एतन्मन्त्रान्तर्गतासौशब्दे ना.\\
मगोत्रोच्चारणं कर्त्तव्यमिति हेमाद्रिः । शौनकाथर्वणश्राद्धकल्पे
त्वैते-\\
रेव मन्त्रैः पिण्डार्थान्त्रोद्धरणं उक्त्वा एतत्ते
प्रपितामहेत्यादिमन्त्रैः

{२६० वीरमित्रोदयस्य श्राद्धप्रकाशे-}{\\
प्रपितामहादारभ्य पितृपर्यन्तं पिण्डदानं उक्तम् । द्यौदर्विरक्षितेति
ति-\\
सृभिः सर्वान्नप्रकारमुद्धृत्याज्येन सन्नीय त्रीन्पिण्डान् संहतान्निद-\\
धात्येतत्ते प्रपितामहेतीत्यादिना । पिप्पलादाथर्वणश्राद्धकल्पे तु\\
विपरीतः क्रमो नोक्तः । एषां च पक्षाणां तत्तच्छाखानुसारेण व्यव-\\
स्था दृष्टव्या ।\\
अत्र च पित्रादीनां नामाज्ञाने -\\
गोभिलः ।\\
यदि नामानि न विद्यात् स्वधा पितृभ्यः पृथिवषिद्भ्य इति\\
प्रथमं निदध्यात् स्वधा पितृभ्योऽन्तरिक्षसद्भ्य इति द्वितीयं स्वधा\\
पितृभ्यो दिविषद्भ्य इति तृतीयं निधायेति ।\\
गोत्राज्ञाने काश्यपगोत्रमाह -\\
व्याघ्रः ।\\
गोत्रनाशे तु काश्यपः, इति ।\\
नाशो = अज्ञानम् । अत्र षड्दैवत्यश्राद्धादौ मातामहवर्गस्यापि\\
गोत्रसम्बन्धनामानि प्रयुज्य प्रयोगोऽनुष्ठेयः । ``योज्यः पित्रादिश-\\
ब्दानां स्थाने मातामहादिक'' इत्यापस्तम्बवचनात् । एव नवदैवत्येऽपि\\
मातृवर्गे स्त्रीलिङ्गानि गोत्रसम्बन्धनामानि प्रयुज्य प्रयोगः कार्यः ।\\
अत्र चावनेजनमन्त्रे आश्वलायनानां शुन्धन्तां मातामहाः, शुन्ध\\
न्तां मातर इत्यादि ऊहो मातामहयोर्ज्ञेयः । अवनेजनमन्त्रे पित्रादि-\\
शब्दानां प्रकृत्यंशे समवेतार्थकत्वात् । मातृवर्गे ये च त्वामत्रान्धि-\\
त्यूह इति हेमाद्रिः । तन्न । एतत्पिण्डरूपमन्नं तुभ्यं, ये चान्ये
अत्र\\
त्वामनुयान्ति तेभ्यवश्चेत्ययं खलु तस्यार्थः । न चानुयाविनः पुरु-\\
षस्य पुरुषा एव स्त्रियो वा स्त्रीणामिति नियमे प्रमाणमस्ति, तेन\\
स्त्रीणामपि पुंसो अनुयामिनः सम्भवन्तीति नाहसिद्धिः । आपस्तम्बेन\\
तु एतत्ते मातरसौ याश्च त्वामत्रान्वित्येव मन्त्रः पठित इति तेषां\\
तादृश एव । अत्र च नवदैवत्ये सपत्नमात्रादीनामपि श्राद्धं भवत्येव\\
`` पितृपत्न्यः सर्वा मातरः'' इति सुमन्तुना तास्वपि मातृत्वाद्यतिदे.\\
शात् । तत्रापि च जनन्यादिस्थानीयब्राह्मणे तत्पिण्डे वा तासामुद्देशो\\
न तु पृथक् ।\\
अनेका मातरो यस्य भाद्धे चापरपक्षिके ।\\
अर्घ्यदानं पृथक्कुर्यात्पिण्डमेकं तु निर्वपेत् ॥

{ }{ पिण्डदानेतिकर्तव्यता । २६१}{\\
इति गालववचनात् । एवं च पिण्डदाने एतत्तेऽसौ इति मन्त्रे\\
असौशब्दस्थाने वामिति ब्रूयात् । त्वाशब्दस्थाने च युष्मानिति\\
ब्रूयात् । नारायणवृत्तिस्वरसोऽप्येवम् । इदं च मातृश्राद्ध वृद्धिव्य\\
रिक्तश्राद्धेषु तीर्थमहालयान्वष्टकादिषु पितृपार्वणानन्तरमेव ।\\
क्षयाहे केवलाः कार्या वृद्धावादौ प्रकीर्त्तिताः ।\\
सर्वत्रैव हि मध्यस्था नान्त्याः कार्यास्तु मातरः ॥\\
इति छागलेयवचनादिति हेमाद्रिः । शूलपाणि {[}स्तुपूर्व{]} मेव ।\\
क्षयाहे केवलाः कार्या वृद्धावादौ प्रकीर्तिताः ।\\
अष्टकासु च कर्त्तव्यं श्राद्धं हैमन्तिकासु वै ।\\
अन्वष्टकासु क्रमशो मातृपूर्वं तदिष्यते ।\\
इति ब्रह्मपुराणवचनादन्वष्टकासु मातृश्राद्धं पूर्वमेवेत्याह ।\\
वस्तुतस्तु -\\
पितृभ्यः प्रथमं दद्यान्मातृभ्यस्तदनन्तरम् ।\\
ततो मातामहेभ्यश्चेत्यान्वष्टक्ये क्रमः स्मृतः ॥\\
इति हेमाद्रयुदाहृतब्रह्माण्डपुराणवचने मध्येऽप्युक्तत्वाद्विकल्पः ।\\
एवं `` पित्रादिनवदैवत्यं तथा द्वादशदैवत'' मित्यग्निपुराणवचनाद्यत्र\\
वैकल्पिकं तीर्थमहालयादौ द्वादशदैवत्यं तत्रापि मातामह्यादीनाम\\
वनेजनमन्त्रादावूद्दो बोध्यः । निवेशस्तु तासां सर्वान्ते `` आगन्तुनाम-\\
न्ते निवेश'' इति न्यायात्, मातृपार्वणे तु वचनादादौ मध्ये वा\\
निवेश: ।\\
स्मृतिरमावलीकारादयस्तु षड्दैवत्यमेव गयाव्यतिरिक्ततीर्थमहाल-\\
वादावित्याहुः ।\\
अमावास्यायान्तु केषां चित् षड्दैवतं केषाञ्चिच्छाखिनां नव-\\
दैवत्यं वा तेषां शाखानुसारेण व्यवस्था । अत्र च नवदैवत्ये द्वाद-\\
शदैवत्ये वा पार्वणानां पदार्थानुक्रम एव , तत्र पदार्थप्रत्यासत्तिला-\\
भात् । एवं च वेदिकरणादयः पदार्था एकस्य कृत्वा अपरस्यापि\\
कर्त्तव्याः । तत्र वेदिकरणं मातृमातामह्यादीनां न भिन्नं पश्चिमेन त-\\
त्पत्नीनां किञ्चिदन्तर्धायेति सांख्यायनगृह्ये किञ्चिदन्तर्धानोक्त्या\\
वेद्यभेदप्रतीतेः । मातामहपार्वणस्य तु भिन्ना वेदिरितिश्राद्धकल्पः ।\\
लेखा तु मातापित्रोर्भिन्नेव अत एवान्वष्टकावामाश्वलायनः ।\\
कर्पूध्वेके द्वयोः षट्सु व पूर्वासु पितृभ्यो दद्यादपरासुस्त्रीभ्यः ।

{२६२ वीरमित्रोदयस्य श्राद्धप्रकाशे-}{\\
कर्पू = अवटः एकप्रहणमतिदेशप्राप्ताया लेखाया अबाधार्थम् । तेन\\
लेखया सह विकल्पः । यदा द्वे कर्ष्वौ तदा आयते भवतः , यदा षट् तदा\\
परिमण्डलाः । पूर्वासु = पूर्वा च पूर्वाश्च ताः पूर्वाः तासु । तथा
चायम-\\
र्थः । पूर्वस्यां लेखायां पूर्वस्यां कर्ष्वां पूर्वकर्षषु पितृभ्यो
दद्यात् ।\\
पक्षत्रयेऽपि पूर्वस्यामेव पितृभ्यो दद्यादित्यर्थः । अपरासु = अत्रापि
पूर्व-\\
वदेकशेषः तेन कर्षूर्वा प्रतिपार्वणे लेखा मातापित्रोर्भिन्ना । यदा तु\\
कर्पूषट्कं तदा प्रतिदैवत कर्षूभेदः । मातामहपार्वणे तु लेखा भिन्नैव\\
वेदेरेव भिन्नत्वात् । एवं मातामह्यादीनामपि लेखाभेदः । दक्षिणाप्रा-\\
या मातामहलेखायाः पश्चिमेन मातामह्यादीनां पिण्डदानविधानात् ।\\
अतश्च दर्भास्तरणावनेजनादिकमपि भिन्नमेव गृह्यमाणविशेषत्वात् ।\\
एवं उन्मुकनिधानमपि भिन्नं, रेखापुरोभागस्य भिन्नत्वेन गृह्यमाण-\\
विशेषत्वात् । इदं च नवदैवत्यश्राद्धादौ मातुः पृथक् पिण्डदान\\
यदि माता अकृतसहगमना तदा बोध्यम् । यदा तु सा कृतसहगम-\\
ना तदा भर्तारं तां चोद्दिश्य {[} एक {]} एव पिण्डो देयः ।\\
मृताहनि समासेन पिण्डनिर्वपणं पृथक् ।\\
नवश्राद्धं च दम्पत्योरन्वारोहण एव तु ॥\\
इति लौगाक्षिवचनात् । मृताहनि = मृततिथ्येकत्वे । तेन तिथिभेदे\\
पृथगेव श्राद्धम् । समासेन = एकस्मिन्पिण्डे द्वयोरुपलक्षणरूपेण । पृथक्\\
नवश्राद्धं च = चस्त्वर्थो नवश्राद्धं तु पृथगित्यर्थः । पृथक्त्वं च
पूर्वोक्त-\\
समासविपर्ययः तेन प्रधानभेदमात्रम् , अङ्गतन्त्रता न भवत्येव ।\\
अत एव भृगुः । नवश्राद्ध युगपत्तु समापयेदिति । समापनं आर-\\
म्भस्योपलक्षणम् । अत्र च नवश्राद्ध एवं पृथक्त्वविधानादन्यत्र ती-\\
र्थमहालयादौ पृथक्त्वपरिसंख्यायाः समासोऽवगम्यते । एवं चो\\
पक्रमगतो मृताहशब्दस्तीर्थमहालयाद्युपलक्षणार्थ: । एवं च-\\
एकचित्यां समारूढौ दम्पती निधनं गतौ ।\\
पृथक् श्राद्धं तयोः कुर्यादोदनं च पृथक्पृथक् ॥\\
इति स्मृत्यन्तरं नवश्राद्धविषयमिति मदनपारिजातादयः । श्राद्धं\\
ब्राह्मणभोजनात्मकं प्रधानं । ओदनं पिण्डरूपं । हेमाद्रिप्रभृतयस्तु
मृताह\\
नीति विशेषश्रवणात्सावत्सरिकमात्रविषयमेतत् तेन तीर्थादौ पृथगेव\\
देयमित्याहु: । न च नवश्राद्धे पृथक्त्वविध्यानर्थक्यं मृताहमात्र एव\\
समासविधानात् अन्यत्र समासाप्राप्तेरिति वाच्यं । अनुवादत्वात्पृ

{ }{ पिण्डदानेतिकर्तव्यता । २६३}{\\
थक्त्वश्रवणस्य । अत एव नवश्राद्धग्रहणमितरश्राद्धोपलक्षणार्थे । अथ\\
वाऽङ्गतन्त्रताप्राप्त्यर्थं नवश्राद्धे पृथक् ग्रहणं भृगुवचनैकवाक्यत्वात्
।\\
पितृपार्वणे पिण्डदानानन्तरं प्रपितामहात्परांस्त्रीनुद्दिश्य तूष्णीं\\
चतुर्थं पिण्डं फलभूमार्थं दद्यादित्यापस्तम्बः ।\\
तूष्णीं चतुर्थ स कृताकृतः प्रपितामहप्रभृतीन् । उद्दिश्येति शेषः ।\\
तूष्णीम् = अमन्त्रकम् कृताकृतो = वैकल्पिकः । अतश्च करणं फलभूमार्थं,\\
अन्यथा अकरणेनैव साङ्गत्वसिद्धौ करणविध्यानर्थक्यापत्तेः । प्रपिता-\\
महप्रभृतीनित्यतद्गुणसंविज्ञानो बहुव्रीहिः । तेन
प्रपितामहात्परांस्त्री-\\
नुद्दिश्येत्ययमर्थो लभ्यते । देवलेन तु मातृपार्वणपिण्डदानानन्तरं ज्ञा-\\
तिवर्गस्य सामान्यपिण्डो देयं इति उक्तम् ।\\
हविःशेषं ततो मुष्टिमादायैकैकमादितः ।\\
क्रमशः पितृपत्नीनां पिण्डनिर्वपणं चरेत् ॥\\
ततः पिण्डमुपादाय हविषः संस्कृतं महत् ।\\
ज्ञातिवर्गस्य सर्वस्य सामान्यमिति निर्वपेत् ।\\
पिण्डदानानन्तरं तदन्तिके तच्छेषविकिरणमुक्तं गारुडे-\\
पिण्डशेषविकिरणं च पिण्डान्तिकं इति गद्यरूपेण ।\\
अत्र च पिण्डदानानन्तरं प्रपितामहात्परांस्त्रीनुद्दिश्य तेषामयं भागो\\
ऽस्तु इति ब्रुवन् दर्भमूले करावघर्षणं कुर्यात् । तथा च-\\
मनुः ।\\
न्युप्य पिण्डान्पितृभ्यश्च प्रयतो विधिपूर्वकम् ।\\
तेषु दर्भेषु तं हस्तं निर्मृज्याल्लेपभागिनाम् }{॥}{\\
लेपभागिनश्च मत्स्यपुराणे दर्शिताः । {[} अ० ३ श्लो० २१६ {]}\\
लेपभाजश्चतुर्थाद्याः पित्राद्याः पिण्डभागिनः ।\\
पिण्डदः सप्तमस्तेषां सापिण्ड्यं साप्तपूरुषम् ॥\\
निर्मार्जनं च मूलप्रदेशे । अत एव विष्णु, दर्भमूले करावघर्षणम् ।\\
अत्र मन्त्र उक्तः मैत्रायणीयसूत्रे बर्हिषि हस्तं निर्माष्टि पात्रपितरः
स्व\\
धातया यूयं यथाभागं मादायध्वमिति अत्र पितरोमादायध्वमिति वा ।\\
मानवमैत्रायणीये तु पूर्वमन्त्रस्य जपमात्रमुक्तं दक्षिणां दिशमन्वींक्ष-\\
माणो जपतीति । इदं च पित्रादित्रिकदर्भमूल एव करसंमार्जनं वृद्ध-\\
प्रपितामहादीनामेव लेपभागित्वात् । अत्र च षडदैवत्यश्राद्धादौ स-\\
र्वेषां पार्वणानामन्ते करसंमार्जनं कार्यं लाघवात् ।\\


{२६४ वीरमित्रोदयस्य श्राद्धप्रकाशे-}{\\
यत्तु विष्णुपुराणम्-

{ स्वपित्र्ये प्रथमं पिण्डं दद्यादुच्छिष्टसन्निधौ ।\\
पितामहाय चैवाथ तत्पित्र्ये च ततः परम् ।\\
दर्भमूले लेपभुजं प्रीणयेल्लेपघर्षणैः ॥\\
पिण्डैर्माता महांस्तद्वद् गन्धमाल्यादिसंयुतैः ।\\
प्रीणयित्वा द्विजाग्य्याणां दद्यादाचमनं ततः ॥\\
इति, तन्न क्रमकरूपकं, किन्तु दर्भमूलविधायकं, अन्यथा प्रत्यव-\\
नेजनादेः प्रागपि तदुक्तब्राह्मणाचमनापचेरिति स्मार्त्तमहाचार्यः । अत\\
एव पितृदयिताश्राद्धकल्पप्रभृतिभिः षट्पिण्डानन्तरमेव करसंमार्जनं लि-\\
खितम् ।\\
रायमुकुटादयस्तु यथाश्रुतविष्णुपुराणवचनात् ।\\
त्रींस्तु तस्माद्धविःशेषात्पिण्डान्कृत्वा समाहितः ।\\
औदकेनैव विधिना निर्वपेद् दक्षिणामुखः ॥\\
{[} अ० ३ श्लो० २१५ {]}\\
न्युप्य पिण्डांस्ततस्तांस्तु प्रयतो विधिपूर्वकम् ।\\
तेषु दर्भेषु तं हस्तं निर्मृज्याल्लेपभागिनाम् ।।\\
इति मनुवचनाच्च पितृपार्वणे पिण्डदानानन्तरं करसंमार्जनं का-\\
र्यमित्याहुः । न चैतस्य विष्णुवचनस्य क्रमकल्पकत्वे प्रत्यवने जनस्या-\\
पि ब्राह्मणाचमनोत्तरकालत्वापत्तिः । विष्णुपुराणे प्रत्यवनेजनस्यानुक्त-\\
त्वात् । अत्र चैकोद्दिष्टे लेपभुजामसंभवेऽपि निरुद्देश्यककरसंमार्ज-\\
नादिकं कर्त्तव्यमेव, साधारणप्रवृत्तप्रागुक्तब्रह्मपुराणवचनादिति
स्मार्त्त-\\
भट्टाचार्यः । अत्र च हस्तोन्मार्जनस्य दृष्टार्थत्वाद्यदि हस्तलेपो
नास्ति\\
तदा नोन्मार्जनं कर्त्तव्यं शकृल्लोहितनिरसनवत् । मेधातिथिहरिहरप्र-\\
भृतयस्तु यद्यपि हस्ते लेपो न भवेत् तथापि उन्मार्जन कर्त्तव्यमेव,\\
नह्यस्य प्रतिपत्तिकर्मत्वं किं तर्हि हस्तोन्मार्जनं अदृष्टार्थम् । अत
एव\\
विष्णुस्मृतौ करावघर्षणमात्रं श्रूयते अतश्च तादृशस्य लेपाभावेऽपि\\
कर्त्तुं शक्यत्वाद्भवत्येवोन्मार्जनं । न च लेपभागिनामित्यस्यानुपपत्तिः\\
लेपशब्दस्यात्रान्नरसोष्मादिपरत्वात् । पिण्डदाने हि क्रियमाणे अन्नर-\\
सोष्मादिकं किञ्चिद्धस्ते संक्रामत्येव अतस्तादृशसंक्रान्तान्नरस\\
एव लेपभागिनामयं भागोऽस्त्विति ब्रुवन् हस्तं निमृज्यादित्याहुः ।\\


{ }{ पिण्डदानेति कर्तव्यता । २६५\\
}{पिण्डदानानन्तरम्-\\
मनुः ।\\
आचम्योदक् परावृत्य त्रिरायम्य शनैरसुन् ।\\
षडप्यृतून्नमस्कुर्यात्पितॄनेव च मन्त्रवत् । {[}अ०३श्लो०२१७{]}\\
आचमनं च हस्तप्रक्षालनपूर्वकं कर्त्तव्यमित्युक्तम्-\\
ब्रह्मपुराणे,\\
ततो दर्भेषु तं हस्तं समृज्य च करौ पुनः ।\\
प्रक्षाल्य च जलेनाथ त्रिराचम्य हारे स्मरेत् ॥\\
त्रिरिति च `` निर्वर्त्य पितृकर्माणि सकृदाचमनं चरेत्'' इत्यस्यापवा\\
दार्थमिति हेमाद्रिः । ``त्रिः पिबेद्वीक्षितं तोय'' मित्यादिना
यत्सामान्यतो\\
विहितं त्रिराचमनं तदत्र त्रिरावर्त्तत इति श्राद्धचिन्तामणिः । एकाचमने\\
त्रिस्तोयपानं यद्विहित तस्यैवायमनुवाद इति श्राद्धदीपिका । उदक्परावृ\\
स्य = उदीचीं दिशं परावृत्य, परिवर्त्तनेनोत्तराभिमुखो भूत्वेत्यर्थः ।\\
इदं चोदक्परावर्त्तनं पिण्डानामनुमन्त्रणपूर्वकं कर्त्तव्यमित्याह -\\
आश्वलायनः ।\\
निपृताननुमन्त्रयेतात्रपितरो मादयध्व यथाभागमावृषायध्व-\\
मिति सव्यावृदुदगावृत्येति ।\\
निपृतान् = विधिपूर्वकं दर्भेषु निहितान् । सव्यावृत् - सव्यप्रकारमप्र-\\
दक्षिणमित्यर्थः । उदगावृत्य = उदगन्तमेवावृत्य ।\\
छन्दोगपरिशिष्टे विशेष उक्तः ।\\
वामेनावर्त्तनं केचिदुदगन्तं प्रचक्षते ।\\
सर्वं गौतमशाण्डिल्यौ शाण्डिल्यायन एव च ॥ इति ।\\
वामेन = वाममङ्गं पुरस्कृत्य । उद्गन्तम् = यावदुदङ्मुखो भवति ता-\\
वदित्यर्थः । केचित् = गौतमादिव्यतिरिक्ताः । ते तु सर्वं सव्येनापसव्येन\\
वा परावर्त्तनं प्रचक्षत इत्यर्थः ।\\
उदक्परावर्त्तनानन्तरं च उत्तराभिमुखेन श्वासनियमनं कार्यं\\
प्रागुक्तमनुवचनात् । अत्र च मनुवचने त्रिरिति श्रवणादुक्तलक्षणं प्रा\\
णायामं त्रिः कृत्वेत्यर्थ इति मेधातिथिः । विनैव मन्त्रेन प्राणानां निय\\
मनमिति तु कर्कादिबहुसंमतोऽर्थः । श्वासं नियम्य षट् ऋत्तून्पितॄंश्च\\
नमस्कुर्यादित्याह -\\
स एव ।\\
षट्ऋतूंस्तु नमस्कुर्यात्पि}{तॄ}{नेव च मन्त्रवत् ।\\
वी० मि ३४

{२६६ वीरमित्रोदयस्य श्राद्धप्रकाशे-}{\\
तत्र ऋतुनमस्कारे मन्त्र उक्तो -\\
ब्रह्मपुराणे ।\\
वसन्ताय नमस्तुभ्यं ग्रीष्माय च नमो नमः ।\\
वर्षाभ्यञ्च शरत्संज्ञऋतवे च नमः सदा ॥\\
हेमन्ताय नमस्तुभ्यं नमस्ते शिशिराय च ।\\
माससंवत्सरेभ्यश्च दिवसेभ्यो नमो नमः ॥ इति ।\\
पितृनमस्कारे तु मन्त्रो `` नमो वः पितर इषे '' इत्यादिर्द्रष्टव्य इति\\
हेमाद्रिः ।\\
अन्ये तु मनुवचनमेवं व्याचक्षते ऋत्पितॄनेव पितृतया ध्याय-\\
न्नेव नमस्कुर्यात् , मन्त्रवदिति = `` वसन्ताय नमस्तुभ्य"
मित्यादिमन्त्रे-\\
णेत्यर्थः ।\\
यत्तु कात्यायनसूत्रे ``नमो वः पितर'' इति मन्त्रेण षडखलिकर.\\
णमुकं, तक्ष ऋतुनमस्काराद् भिन्नमेव, तत्र ऋतुनमस्कारस्यानुक.\\
त्वात् । अतश्च कास्यायनैरपि मनुक्त ऋतुनमस्कारः पितृनमस्कारश्च\\
भिन्नतया कार्य इत्याहुः ।\\
अयं च ऋतुनमस्कार उदङ्मुखेनैव कर्त्तव्य इत्युक्तं मारुडे-\\
वामेन परावृत्योदङ्मुखः प्राणान्नियम्य षड्भ्यो नम इति नम-\\
स्कृत्य वामेनैव परावृत्य दक्षिणामुखोऽमीमदन्तेति जपति ।\\
स्मार्त्तभट्टाचार्यस्तु एवमाह, न वसन्ताय नम इत्यनेन ऋतुनम-\\
स्कार इत्येवं मनुवचनं व्याख्येयं किन्तु याजुर्वैदिकेन नमो वः पितर\\
इति मन्त्रेण रसादिशब्दप्रतिपाद्यान्वसन्तादीन् ब्राह्मणसर्वस्वे हला\\
युधव्याख्यातान् षडृतून्नमस्कुर्यादित्यर्थ इति व्याख्येयम् । अतश्च\\
कात्यायनादर्शनां नमो व इत्यनेनैव ऋतुनमस्कारस्य सिद्धत्वात् न\\
पृथक् सोऽनुष्टयः प्राप्नोति येषां तु सामगादीनां न कथञ्चिद् ऋतु-\\
नमस्कारप्राप्तिस्तेषां ब्रह्मपुराणवचनात्तत्प्राप्तिः तथा च
ब्रह्मपुराणं-\/-\\
तेभ्यः संस्रवपात्रेभ्यो जलेनैवावनेजनम् ।\\
दत्वात्र पितरश्चेति जपेच्चोदङ्मुखस्ततः ॥\\
सञ्चिन्तयन् पितॄन् तुष्टान्सर्वान् भास्करमूर्त्तिकान् ।\\
अमीमदन्तपितर इति पश्यन् पितॄन्पठेत् }{ ॥}{\\
नीवीं विस्रस्य च जपेन्नमो वः पितरस्त्विति ।\\
अत्र षट्पुरुषान्तं च मन्त्रं जप्त्वा कृताञ्जलिः ॥

{ }{ पिण्डदानेतिकर्तव्यता २६७}{\\
एतद्वः पितरो वास इति जल्पन् पृथक् पृथक् ।\\
अमुकामुकगोत्रैतत्तुभ्यं वासः पठेत्ततः ॥\\
दद्यात्क्रमेण वासांसि श्वेतवस्त्रभवा दशाः ।\\
गते वयसि वृद्धानि स्वानि लोमान्यथापि वा ॥\\
क्षौम सूत्रं नव दद्यात् शाणं कार्पासमेव च ।\\
कृष्णोर्णा नीलरक्तान्यकौशेयानि विवर्जयेत् ॥\\
मधु चाज्यं जलं चार्घ्यं पुष्पं धूपं विलेपनम् ।\\
वलिं दद्यात्तु विधिवत्पिण्डोऽष्टाङ्गो भवेद्यथा ॥\\
सम्पूजयित्वा विप्रांस्तु पितॄँश्च प्रणमेदृतृन् ।\\
वसन्ताय नमस्तुभ्यं ग्रीष्माय च नमो नमः ॥\\
वर्षाभ्यश्च शरत्संज्ञॠतवे च नमो नमः ।\\
हेमन्ताय नमस्तुभ्यं नमस्ते शिशिराय च ॥\\
माससंवत्सरेभ्यश्च दिवसेभ्यो नमो नमः ॥ इति ।\\
एवं च तेषां एतद्वचनानुसारात् ऋतुनमस्कारः पिण्डपूजान-\\
न्तरमेव भवति । अत एव श्राद्धचिन्तामणिरपि ब्रह्मपुराणोक्तऋतुनम-\\
स्कारस्य सामगपरत्वं तत्सूत्रसंवादादित्याह । क्षौमं = अतसीप्रभवम् ।\\
अतसी स्थादुमा क्षमेत्यमरात् । इदं च ब्रह्मपुराणोक्तं ऋतुनमस्करणं\\
दक्षिणाभिमुखेनैव कर्त्तव्यम् । एवं च उदक्परावृत्य श्वासं निरुध्य\\
पुनस्तेनैव पथा परावर्त्येति स्मार्त्तमते अनुष्ठानक्रमः सिद्धो भवति ।\\
उदङ्मुखावस्थानस्यावधिमाह -\\
कात्यायनः ।\\
उदङ्ङनस्ते आदमनात् । तमन = ग्लानि, तदवधि । अतश्चोदङ्मु-\\
खावस्थानमात्रेण ग्लानेरसम्भवात् कात्यायनानुक्तमपि श्वासनिरो-\\
धात्मकं शास्त्रान्तरोपदिष्टं ग्लानिकारणं अवश्यस्वीकर्त्तव्यम् । श्वा-\\
सनिरोधानन्तरं-\\
स एव ।\\
आवृत्यामीमदन्तेति जपति ।\\
आवर्त्तनं तेनैव मार्गेणेति कर्कः ।\\
अत एव छन्दोगपरिशिष्टे ।\\
जपंस्तेनैव चावृत्य ततः प्राणं विमोचयेत् ।\\
जपन् = अमीमदन्तेति मन्त्रमिति शेषः । प्राणं विमोचयेत् = उच्छ्वसे-

{२६८ वीरमित्रोदयस्य श्राद्धप्रकाशे-}{\\
दित्यर्थः । अत्र च जपन्निति श्रवणाज्जपतोऽभ्यावृत्तिः प्रतीयते । का-\\
त्यायनादीनां तु अभ्यावृत्युत्तरकालं जप इत्येतावान्विशेषः । अत\\
एव तेषामस्मिन् जपे दक्षिणामुखत्वं प्रागुक्तगरुडपुराणैकवाक्य-\\
त्वात् । छन्दोगानां तु {[}न{]} दक्षिणामुखत्वं परावर्त्तनसमय एव जपवि\\
धानात् । अयं च मन्त्रजप उपांशु कर्त्तव्यः जपत्वात् श्वासनिरोधेन\\
व्यक्तजपस्य कर्त्तुमशक्यत्वादिति स्मार्त्तः ।\\
आपस्तम्बेन तु पिण्डविधानानन्तरं मन्त्रान्तरेणोपस्थानमुक्त्वा\\
अत्र पितर इत्यनयोर्मन्त्रयोः पाठान्तरेण विनियोगो दर्शितः ।\\
यन्मे माता प्रलुलोभ यच्चचाराननुब्रतम् ।\\
तन्मे रेतः पिता वृक्तां माभ्युरन्योपपद्यताम् }{॥}{\\
पितृभ्यः स्वधायिभ्यः स्वधा नमः पितामहेभ्यः स्वधायिभ्यः\\
स्वधानमः प्रपितामहेभ्यः स्वधायिभ्यः स्वधानम इति उपस्थायात्र\\
पितरो यथाभागं मादयध्वमित्युक्त्वा परागावर्त्तते ओष्मणो व्यावृ-\\
त्त उपास्तेऽमीमदन्तपितरः सोम्यास इति ।\\
आऊष्मणः = पिण्डोष्मापगमं यावत् । अमीमदन्तेत्यस्यानन्तरं पि-\\
ण्डावशिष्टान्नाघ्राणमुक्त्वा प्रत्यवनेजनमुक्तम्-\\
आश्वनायनसूत्रे ।\\
चरोः प्राणभक्षं भक्षयेत् ।\\
चरोः = पिण्डावशिष्टस्य । प्राणभक्षं = अवघ्राणं यथाकृतं भवति तथा\\
भक्षयेत् = अवजिघ्रेत् ।\\
आपस्तम्बेन तु पिण्डावशेषस्य आघ्राणमुक्त्वा तस्य समन्त्रकं\\
काम्यं भक्षणमुक्तम् । यः स्थाल्यां शेषस्तमवजिघ्रति-\\
ये समानाः सुमनसः पितरो यमराज्ये ।\\
तेषां लोकः स्वधा नमा यज्ञो देवेषु कल्पताम् । वीरं मे दत्तपि-\\
तर इति आमयाविना प्राश्यो अन्नाद्यकामेन प्राश्यो योऽलमन्नाद्याय\\
तेन वा प्राश्यः । आमयावी = रोगी अस्य चौचित्यानिवृत्तिशेषत्वेनैव\\
काम्यत्वम् । अन्नाद्यं = अन्नं तत्कामोऽपि भक्षयेत् । यो वा
अन्नाद्यायालं-\\
समर्थः सन अरुच्यादिना नाद्यात्स तदुत्पत्यर्थं भक्षयेत् ।\\
कात्यायनस्तु अवत्राणमनुक्त्वा अमीमदन्तेत्यस्य जपस्यानन्तरं\\
कर्त्तव्यमाह ।\\
अवनेज्य पूर्ववन्नीवीं विस्रस्य नमो व इत्यञ्जलिं करोति । इदं

{ }{ पिण्डदानेतिकर्तव्यता । २६९\\
}{चावनेजनं पूर्वावनेजनजलशेषजलेन । `` तेभ्यः संस्रवपात्रेभ्यो जलेनै-\\
वावनेजनम्'' इति ब्रह्मपुराणवचनात् । संस्रवपात्रेभ्यः पूर्वदत्तावनेज\\
नपात्रेभ्यः । मनुवचनेऽपि ।\\
उदकं निनयेच्छेषं शनैः पिण्डान्तिके पुनः ।\\
{[} अ० ३ श्लो० २१८ {]}\\
इदं च सामगभिन्नवाजसनेयादिपरं, तेषां तु पिण्डपात्रक्षालन-\\
जलेन `` तत्पात्रक्षालनेनाथ पुनरप्यवनेजयेत्'' इति छन्दोगपरिशिष्ठात् ।\\
केचित्तु वाजसनेयिनामपि पिण्डपात्रक्षालनजलेनैवेच्छन्ति । अत्र\\
चावनेजनवाक्यशेषशब्द प्रतिपत्तिकर्मत्वम् । एवं च दैवादस्य जलस्य\\
नाशे नेदं प्रत्यवनेजनं कार्यमिति मेधातिथिः ।\\
वस्तुतस्तु आश्वलायनसूत्रे नित्यं निनयनमिति नित्यग्रहणात्\\
कर्तव्यमेव । हेमाद्रिरण्येवम् ।\\
अवनेजनानन्तरं पिण्डपात्रस्य न्युब्जीकरणमुक्तम्-\\
गारुडे ।\\
कृत्वावनेजनं कुर्यात्पिण्डपात्रमधोमुखम् ।\\
नीवीं विस्रस्येति । वामकट्यां वस्त्रदशासङ्गोपनं नीविरिति हेमाद्रिः ।\\
तामुन्मुच्य । नाविविस्रंसनानन्तरं आचमनं कार्यं नीवीं विस्रस्य\\
परिधायोपस्पृशेदिति बौधायनवचनात् ।\\
केचित्तु नीविविस्रंसनस्य द्विराचमनप्रकरणे कात्यायनेन पाठाद्\\
द्विराचमनमेव कार्यमित्याहुः ।\\
अन्ये तु न तु कर्मस्थ आचामेद्दक्षिणं श्रवणं स्पृशेदिति वचना-\\
त्कर्णस्पर्शमाहुः । किमपि न कार्यमिति पितृभक्तिः । नमो व इति अञ्जलिं\\
करोतीति । अञ्जलिः = करसम्पुटः । अञ्जलिमावघ्नन्पिण्डाभिमुखः पितृ-\\
भ्यो नमस्कारान्कुर्यादित्यर्थः । अत्र च प्रतिमन्त्रं षट्कृत्वो नमस्कारः
।\\
`` आषट्कृत्वो नमस्करोती''ति श्रुतेः । अतश्च प्रतिनमस्कारमञ्जलि-\\
करणमावर्त्तत इति कर्कः । ते च नमो वः पितरो रसायेत्यादि पित-\\
रो नमो व इत्यन्ताः । अत्र च तत्तच्छाखान्तरे तत्तत्सुत्रान्तरे च ये\\
मन्त्राः पठितास्ते तत्रैष द्रष्टव्याः । मैत्रायणीयपरिशिष्टेषु तु
अञ्जलेः\\
पिण्डोपरि करणमुक्तम् ।\\
निह्णुवतेऽञ्जलिं कृत्वा नमो व इत्यादि ।\\
 निह्णुवनं= पिण्डोपरि पाण्योः करणम् । अध प्रस्तरे निह्णुवत
इत्या-\\
दिषु तथा प्रमितेः। गोभिलस्तु निह्णुवने विशेषमाह । अथ निह्णुवते

{२७० वीरमित्रोदयस्य श्राद्धप्रकाशे-}{\\
पूर्वस्यां दक्षिणोत्तानौ पाणी कृत्वा नमो वः पितरो जीवाय नमो वः\\
पितरः शोषावेति मध्यमायां सव्योत्तानौ नमो वः पितरो नमो व\\
इति दक्षिणोत्तानाविति । दक्षिणपाणिमुत्तानं कृत्वा तदुत्तरमुक्तम् ।\\
कात्यायनैस्तु नमस्कारानन्तरं वासोदानं कार्यं तथा च-\\
स एव ।\\
एतद्व इत्यपास्यति सूत्राणि प्रतिपिण्डं ऊर्णां दशां वा वयस्युत्तरे\\
यजमानस्य रोमाणि वा ।\\
सूत्राणीति बहुवचनात्त्रीणि त्रीणीति कर्क । ऊर्णा = मेषरोमाणि । इयं\\
ऊर्णा श्वेताया ग्राह्या, कृष्णाया: `` कृष्णोर्णा नीलरक्तानि अकौशेयानि\\
वर्जये'' दिति ब्रह्मपुराणे निषेधात् । दशा = वस्त्राञ्चलसूत्राणि ।
सूत्राणीत्यने-\\
नैव दशाया अपि सिद्धेः पुनर्दशाग्रहणं दशाया अनुकल्पत्वसूचनार्थम् ।\\
अत एव वायुपुराणे-\/-\\
वर्जयेत्तु दशां प्राशो यद्यप्यहतकोद्भवामिति निषेध इति हेमाद्रिः ।\\
अत एव श्वेतवस्त्रभवादशा इति ब्रह्मपुराणवचनं पित्र्यनुकल्पत्वपरमिति\\
व्याख्येयम् । न चात्र श्वेनदशाया विधानाद्वायुपुराणीयदशानिषेधः\\
कृष्णदशाविषयोऽस्त्विति वाच्यम् । तत्राहतकोद्भवामिति श्रवणादहत\\
कस्य च शुक्लत्वेन तस्या अपि निषेधप्रतीतेः । भानूपाभ्येयोऽप्येवम् ।
अन्ये\\
त्वहतस्वाहतं यन्त्रनिर्मुक्तमिति वचनान्नूतनपर्यायस्वावगमान्नूतनकृ\\
ष्णनिषेधार्थ वायुपुराणवचनम् । अत एव शुक्लवस्त्रदशाभवं सूत्रमिति पि\\
तृदयिता, वयस्युत्तरेयजमानरोमाणि चेति । शतवर्षस्यायुषः समं भा.\\
गद्वयं प्रकल्प्य द्वे वयसी कल्पते तत्रोत्तर उपरितने वयसि वर्त्तमानस्य\\
यजमानस्य लोमानि तानि च प्रशस्ताङ्गकत्वादुरस्थितान्येव ग्राह्याणि ।\\
उत्तरे वयस्पुरोलोमानीति सूत्रान्तरदर्शनात् । वाशब्दादस्मिन्नपि व-\\
यसि सूत्रादिना विकल्पो लोम्नामिति दर्शितमिति हेमाद्रिः । एतच्च सू-\\
त्रादिवस्त्राभावे वेदितव्यम् । अथ वस्त्रमभावे दशामूर्णां वेति विष्णुव-\\
चनात् । अत्र च प्रतिपिण्डमित्यभिधानादेतद्व इति मन्त्रः प्रतिपिण्ड\\
मावर्त्तते । बहुवचनं प्रयोगसाधुत्वार्थम् । यत्वत्र वाचस्पतिनान्यैरपि\\
( १ ) गौडेरुक्तमत एव द्विपाशिकाया विकृतौ बहुवचनान्तमेव पाशपदं\\
प्रयोक्तुमुचित मित्युक्तं (२) पाशाधिकरण इति, तन्मीमांसानवबोध-

% \begin{center}\rule{0.5\linewidth}{0.5pt}\end{center}

% \begin{center}\rule{0.5\linewidth}{0.5pt}\end{center}

{ }{पिण्डदानेतिकर्तव्यता । २७१}{\\
विलसितम् । अत्रापि च तत्तच्छाखाभेदेन सूत्रभेदेन च ये मन्त्रा\\
उक्तास्ते तत्रैव ज्ञेयाः ।\\
मन्त्रपाठोत्तरं वाक्यप्रयोग उक्तो ब्राह्मे -\\
एतद्वः पितरो वास इति जल्पन्पृथक् पृथक् ।\\
अमुकामुकगोत्रैतत्तुभ्यं वासः पठेत्ततः ।\\
अयं च वाक्यप्रयोगो मन्त्रपाठोत्तरकालीनत्वात्प्रतिपिण्डमाव\\
र्त्तते । एतद्वः पितरो वास इत्युक्त्वा वाक्यपूर्वकम् । सम्बोध्य
प्रतिपि-\\
ण्डे तु वस्त्रं वापि विनिःक्षिपेत् '' इति भविष्यवचनाच्च । एतेन
चामुकगोत्रा\\
अमुकशर्माण एतानि वः पिण्डेषु वासांसि स्वधेत्युक्त्वा एतद्वः पि\\
तरो वास इति प्रतिपिण्डं दद्यादिति रायमुकुटकारमतं परास्तम्, पूर्वो-\\
क्तवाक्यविरोधात् । अत्र च नवदवत्यश्राद्धादौ मातामहपार्वणे वा\\
नास्य मन्त्रस्योहः प्रकृतो पिण्डपितृयज्ञे पितृशब्दस्य सपिण्डीकर\\
णसंस्कारवन्त्रनत्वात् पितामहप्रपितामहयोरपि मन्त्रप्रयोगात् । अत्र\\
च वस्त्रादीनामभ्युक्षणं कृत्वा दानमुक्तमानवमैत्रायर्णायसूत्रे ``वास\\
ऊर्णां दशां वाभ्युक्ष्य पिण्डदेशे निदधाति '' ।\\
अत्र कैश्चित्सूत्रकारैर्वासोदानात्पूर्वं
गृहाद्यवेक्षणाञ्जनदानाम्युक्ता\\
नि । तत्र गोभिलस्तावद् गृहानवेक्षेत गृहान्नः पितरो दत्तेति पिण्डा-\\
नवेक्षेत सतो वः पितरो देष्मेति । अत्र गृहशब्देन पत्न्युच्यते गृहाः\\
पत्नीति तेनैवोक्तत्वादिति स्मात्तः । अञ्जनाभ्यञ्जने त्वाश्वलायनेनाक्ते
।\\
असावभ्यङ्क्ष्वासावङ्क्ष्वेति पिण्डेष्वभ्यञ्जनाञ्जने वासो दद्यादिति ।\\
अभ्यञ्जनं तैलम् । अञ्जनं त्रैककुदादि । तदुक्तम्-\\
ब्रह्मपुराणे ।\\
श्रेष्टमाहुस्त्रिककुदमञ्जनं नित्यमेव च ।\\
तैलं कृष्णतिलोत्थं तु यत्नात्सुपरिरक्षितम् ।\\
श्रेष्टमाहुरित्यभिधानात्तदभावे कज्जलाद्यभ्यनुज्ञा गम्यते । वासः\\
सूत्रदानानन्तरम् । कात्यायनः ।\\
ऊर्जमित्यपोऽभिषिञ्चति ।\\
ऊर्जमितिप्रतीकेन सम्पूर्णो मन्त्रो गृह्यते ।\\
अनेन मन्त्रेण पिण्डानामुपरि दक्षिणाग्रां जलधारां दद्यादित्यर्थः ।\\
पिण्डेषु गन्धादिदानमाह-\\
विष्णुः ।\\
अर्धपुष्पधूपानुलेपनान्नाद्यभक्ष्यभोज्यं निवेदयेत् । उदपात्रं मधु-

{२७२ वीरमित्रोदयस्य श्राद्धप्रकाशे-}{\\
तिलाभ्यां संयुतं च । इदं च गन्धादिदानं तूष्णीं कार्यम्, `` गन्धादि\\
निक्षिपेत्तूष्णीं तत आचामयेद् द्विजान् '' इति परिशिष्टवचनादिति\\
वाचस्पतिमिश्राः ।\\
हेमाद्रिस्तु ।\\
एतद्वः पितरो देवा देवाश्च पितरः पुनः ।\\
पुष्पगन्धादिधूपानामेवं मन्त्रमुदाहृतम् }{॥}{\\
इति ब्रह्मवैवर्तवचनान्मन्त्रपूर्वकमित्याह - कल्पतरौ ब्रह्मपुराणना-\\
म्ना पठितमिद वचनम् । गन्धदानोत्तरं ब्राह्मणानाचामयेत्प्रागुक्तछ-\\
न्दोगपरिशिष्टवचनात् ।\\
कात्यायनः ।\\
अवधायावजिघ्रति यजमानः ।\\
अवधाय = पिण्डपात्रे पिण्डान् निधाय । अवधानं कृत्वेति भानूपाण्याय }{।}{\\
अवजिघ्रति यजमानः, तत्संस्कारकं चेदं, स्तत्संस्कारकत्वं च `` स यजमान-\\
भाग'' इति वचनात् । अतश्चाध्वर्युकर्तृके पिण्डपितृयज्ञे श्राद्धे
चाप्रतिनि\\
धिकर्तृके यजमानस्यैवेदम् । न चावधानस्याध्वर्य्वादिकर्तृकत्वेन भि-\\
न्नकर्त्तृकत्वात् क्त्वाप्रत्ययानुपपत्तिः । पूर्वकालतामात्र एव
क्त्वाप्रत्य\\
योपपत्तेः । निरूप्याजं प्रातर्दोहनमितिवदिति कर्कः । वस्तुतस्तु अवधाने\\
यजमानस्य प्रयोजककर्तृत्वेन समानकर्तृकत्वस्याप्युपपत्तिः । मनुना तु\\
प्रत्यवनेजनानन्तरं भूमिस्थानामेव पिण्डानां क्रमेणावघ्राणमुक्तम्-\\
उदकं निनयच्छेषं शनैः पिण्डान्तिके पुनः ।\\
अवजिघ्रेच्च तान्पिण्डान्यथान्युप्तान् समाहितः ॥\\
यथान्युप्तानित्यनेन निर्वापक्रमेण पिडानामवत्राणं दर्शितम् ।\\
अवत्राणानन्तर कात्यायनः ।\\
उल्मुकसकृदाछिन्नान्यनौ ।\\
आदघातीति शेषः । अत्र च सकृदाच्छिन्नं पूर्वं क्षिपेत् सकृदा-\\
च्छिन्नान्यन्नावभ्याददाति पुनरुल्मुकमपिसर्जति इति शतपथश्रुतेः ।\\
सूत्रे त्वजाद्यदन्तत्वादल्पाच्तरत्वाद्वा पूर्वनिपातः । आपस्तम्बेन तु
उ-\\
ल्मुकापिसर्जने मन्त्र उक्तः , अभून्नोद्वतो हविषे । जातवेदाः प्रवाह-\\
व्यानि सुरभीणि कृत्वा । प्रादात्पितृभ्यः स्वधया ते अक्षन्प्रजानन्नभे\\
पुनरप्येतु देवानित्येकोल्मुकं प्रत्यपिसृजति । गोभिलेन तु अनेनैव
मन्त्रे\\
णोल्मुकाभ्युक्षणमुकम् । इति पिण्डदानम् ।

{ }{ अक्षय्योदकदानादिविचारः । २७३}{\\
अथाक्षय्योदकदानम् ।\\
विष्णुः ।\\
ततः सुप्रोक्षितमिति श्राद्धदेशं प्रोक्ष्य दर्भपाणिः सर्वं कुर्यात् ।\\
ततः = पैतृकद्विजपुरःसरमाचमनोदकं दत्वा । तत्पुरःसरत्वं च देव-\\
लेनोक्तम् । उदङ्मुखेषु घनमादौ दत्वा प्राङ्मुखेषु दद्यात् । सर्वं =
वक्ष्य\\
माणम् । आचमनोत्तरं -\\
कात्यायनः ।\\
आचान्तेषूदकपुष्पाक्षतानक्षय्योदकं च दद्यात् ।\\
अक्षता = यवाः ।\\
आचान्तेषूदक दद्यात्पुष्पाणि सयवानि च ।\\
यवोऽसीति पठन्मन्त्रं श्रद्धाभक्तिसमन्वितः ॥\\
इति ब्रह्मपुराणवचनात् । श्राद्धकल्पोऽप्येवम् । इदं च यवदानं\\
देवकरे, पित्रे तु तिलदानम् ।\\
तथा तत्रैव ।\\
सतिलाम्बु पितृष्वादौ दत्वा दैवेषु साक्षतम् ।\\
आदावित्यस्य दैवेष्वित्यनेनान्वयः ।\\
आदौ साक्षतमम्बु दैवेषु दत्वा सतिलाम्बु पितृषु दद्यादित्यर्थः ।\\
अत एव कर्केणापि देवपूर्वकत्वमुदकदानादेरुक्तम् ।\\
उदकादिदाने मन्त्राः ।\\
छन्दोगपरिशिष्टे ।\\
शिवा आपः सन्त्विति च युग्मानेवोदकेन वा ।\\
सौमनस्यमस्त्विति च पुष्पदानमनन्तरम् ॥\\
अक्षतं चारिष्टमस्त्वित्वक्षतान् प्रतिपादयेत् ।\\
युग्मानिति विध्यभिप्रायेणेति हलायुधः । शातातपेन तु मन्त्रान्तरा-\\
ण्युक्तानि ।\\
अपां मध्ये स्थिता देवाः सर्वमप्सु प्रतिष्ठितम् ।\\
ब्राह्मणस्य करे न्यस्ताः शिवा आपो भवन्तु नः ।\\
लक्ष्मीर्वसतु पुष्पेषु लक्ष्मीर्वसतु पुष्करे ।\\
लक्ष्मीर्वसेत्सदा सोमे सौमनस्यं सदास्तु मे ॥\\
अक्षतं चास्तु मे पुण्यं शान्तिः पुष्टिर्धृतिश्च मे ।\\
यद्यच्छ्रेयस्करं लोके तत्तदस्तु सदा मम ।।\\
अन्नोदकदानादौ यथालिङ्गं मन्त्राणां विभज्य विनियोगः । अत्र\\
वी० मि १५

{२७४ वीररामत्रोदयस्य श्राद्धप्रकाशे-}{\\
च परिशिष्टशातातपोक्तानां मन्त्राणामैच्छिको विकल्प इति मिश्रा-\\
भिप्रायः । उदकादिदानं च पित्र्ये अपसव्येनेति कर्कः ।\\
शङ्खधरस्तु ।\\
`` ततः पुष्पाणि सव्येन सोदकानि पृथक् पृथक् '' इति शातात.\\
पवचनात्सव्येनैवेत्याह । न चैतस्य देवपरत्वमिति वाच्यम् । आन\\
र्थक्यापत्तेः । पृथक् पृथगित्यस्यानुपपत्तेश्च ।\\
अक्षय्योदकं च दद्यादिति । दत्तानामन्नादीनामक्षयार्थमुदकमक्षय्यो-\\
दकम् । इदं चाक्षय्योदकदानं (च) नामगोत्राभ्याम्, `` उदङ्मुखेभ्यो\\
दत्वा विश्वेदेवाः प्रीयन्तामिति प्राङ्मुखेभ्यः'' इति वचनात्प्रथ.\\
मतः पित्र्ये दत्वा पश्चादैवेऽपि अक्षय्योदक देयमित्याहु: । अत एव\\
च नामगोत्रोच्चारणपूर्वकत्वमपि भवति । अत्र गोत्रादिसम्बुद्धिस्थाने\\
षष्ठी उता-\\
छन्दोगपरिशिष्टे ।\\
अक्षय्योदकदानं तु अर्घदानवदिष्यते ।\\
षष्ठयैव नित्यं तत्कुर्यान्न चतुर्थ्या कदाचन ॥ इति ।\\
अर्घदानवत्प्रत्येकमक्षय्योदकदानं कुर्यादित्यर्थ इति हलायुधः ।\\
स्मार्त्तस्तु तन्त्रतानिवृत्तेः ; `` अर्धेऽक्षय्यादक चैव'' इत्यनेनेव
सिद्धेर्ज्ये-\\
ष्ठोत्तरकरत्व प्राप्त्यर्थोऽतिदेश इत्याह । एवं च अमुकशर्मणोऽस्मत्पि-\\
तुर्दत्तान्नादेरक्षय्यमस्त्विति पित्र्ये वाक्यं बोध्यम् । दैवे तु
पुरूरवार्द्रव-\\
संज्ञका विश्वदेवा दत्तैतदन्नपानादिना प्रीयन्तामिति । अत्र ब्राह्मण-\\
हस्ते दत्तानामुदकादीनां धारणप्रयोजकभाविकार्याभावाच्छुचौ देशे\\
प्रक्षेपः कार्य इति हेमाद्रिः ।\\
पुनश्च तेषामुदकादिदानं कार्यमित्युक्तं -\\
मत्स्यपुराणे ।\\
आचान्तेषु पुनर्दद्याज्जलपुष्पाक्षतोदकम् ।\\
दत्वाशीः प्रतिगृह्णीयाद् द्विजेभ्यः प्राङ्मुखो बुधः ।\\
स्वस्तिवाचनकं कुर्यात्पिण्डानुद्धृत्य भक्तितः ॥\\
पुनः शब्देन चात्रेदं द्वितीयं दानमिति सूचितमिति हेमाद्रिस्मृतिच-\\
न्द्रिकाकारप्रभृतयः । आशीर्ग्रहणप्रकारश्च कात्यायनेनोक्तः अघोराः पि-\\
तरः सन्तु, सन्त्वित्युक्ते गोत्रं नो वर्द्धतां वर्द्धतामित्युक्ते,\\
दातारो नोऽभिवर्द्धन्तां वेदाः सन्ततिरेव च ।\\
श्रद्धा च नो मा व्यगमत् बहु देयं च नोऽस्त्विति ॥

{ }{ वरयाचनादिविचारः । २७५}{\\
आशिषः प्रतिगृह्येति । क्वचित्तु वेदैः सन्ततिरेव चेति पाठः ।\\
अत्र बहु देयं च नोऽस्त्वीत्मन्तं मन्त्रं पठित्वा ।\\
अन्नं च नो बहु भवेदिति श्रीश्च (१) लभेमहि ।\\
याचितारश्च नः सन्तु मा च याचिष्म कं चन ॥\\
इत्यधिकं बौधायनेन पठितम् । अत्र च वरयाचने क्रियमाणे दा-\\
तार इत्येवमादावपि प्रतिवरं ब्राह्मणैर्यथालिङ्गं प्रत्युत्तरं देयम् ।\\
प्रार्थनासु प्रतिप्रोक्ते सर्वास्वेव द्विजोत्तमः ।\\
इति वचनादिति हेमादिः । अन्यैस्त्वत्र प्रतिवचनं न लिखितम् ।\\
यत्तु स्मृतिचन्द्रिकाकारेण कात्यायनगृह्ये दातार इत्यादिकं नोक्तमि\\
त्युक्तं तत् (कर्म) कल्पतरुहलायुधदिसकलनिबन्धेषु पाठदर्शनादयुक्तम् ।\\
वरग्रहणे कालविशेषः प्रकारविशेषश्चोक्तो मनुना ।\\
विसृज्य ब्राह्मणास्तांस्तु नियतो वाग्यतः शुचिः ।\\
दक्षिणां दिशमाकाङ्क्षन् याचेतमान् वरान्पितृन् ॥\\
{[} अ० ३ श्लो० २७८ {]}\\
विसृज्य = पीठेभ्यो मन्त्रेणोत्थाप्य । आकाङ्क्षन् = अवलोकयन् इति\\
टीका । अत्र च पितृनिति श्रवणात्पितृब्राह्मण एव वरयाचनं कार्यमि-\\
ति हलायुधः । इदं च दक्षिणाभिमुखत्वं ब्राह्मणविसर्जनोत्तरकालत्वं च ।\\
प्रागुक्तमत्स्यपुराणवचनोक्तप्राङ्मुखश्वब्राह्मणधि सर्जन पूर्व
कालत्वाम्यां\\
विकल्पते । सच विकल्पः शाखाभेदेन व्यवस्थितः । अथवा ``वि-\\
सृज्य ब्राह्मणांस्तांस्तु तेषां कृत्वा प्रदक्षिणम् । दक्षिणां
दिशमाकाङ्क्षन् ''\\
इति मनुवचने आकांक्षन् = अवलोकयन् इति टीकाकुद्व्याख्यानात् प्रागु-\\
दङ्मुखेनैव दक्षिणां दिशमालोचयता याचनीयमित्याहुः । अत्र चा-\\
शीः प्रतिग्रहानन्तरं यजमानस्य पुत्रपौत्रादिभिः पिण्डनमस्कारः का\\
र्यः, आचारादिति हेमाद्रिः । यद्यपि चाशीर्ग्रहणानन्तरं प्रागुदाहृतम\\
त्स्यपुराणवचनात् पिण्डोद्धारणं कृत्वा स्वस्तिवाचनं कार्यमिति प्र.\\
तिभाति, तथापि पिण्डोद्धारणानन्तरं पात्रचालनं कृत्वा तत्कार्यम् ।\\
पात्रचालनमकृत्वा स्वस्तिवाचनस्य निषिद्धत्वात् । तथा च वृद्ध\\
बृहस्पतिः ।\\
भाजनेषु च तिष्ठत्सु स्वस्ति कुर्वन्ति ये द्विजा. ।\\
तदन्नमसुरैर्भुक्तं निराशैः पितृभिर्गतैः ॥

% \begin{center}\rule{0.5\linewidth}{0.5pt}\end{center}

(१) अतिथींश्च लभेमहीति अन्यत्र पाठः ।

{२७६ वीरमित्रोदयस्य श्राद्धप्रकाशे-}{\\
पात्रचालने कर्त्तार उक्ताः -\\
प्रचेतसा ।\\
स्वयं पुत्रोऽथवा यश्च वाञ्छेदभ्युदयं परम् ।\\
न स्त्रियो न च बालश्च नान्यजातिर्न चाव्रतः ॥\\
एतेन यत् श्रद्धे भोजनपात्राणि स्वयं न चालयेत् इति हेमाद्रि-\\
णोक्तं तन्निरस्तम् । अव्रतो=ऽनुपनीतः । एवं चोपनीतस्याप्यज्ञस्य नि-\\
वृत्यर्थं बालग्रहणम् ।\\
स्वस्तिवाचनप्रकारस्तु - पारस्करेणोक्तः, ``स्वस्ति भवान् ब्रूही ''ति ।\\
याज्ञवल्क्येन तु अक्षय्यदानात्पूर्वं स्वस्तिवाचनमुक्तं , तथा च
क्रमभेदः\\
शाखाभेदेन व्यवस्थितो द्रष्टव्यः ।\\
कात्यायनः ।\\
स्वधावाचनीयान्सपवित्रान्कुशानास्तीर्य स्वधां वाचयिष्य इति\\
पृच्छति, वाच्यतामित्यनुज्ञातः पितृभ्यः पितामहेभ्यः प्रपितामहेभ्यो\\
मातामहेभ्यः प्रमातामहेभ्यो वृद्धप्रमातामहेभ्यः स्वधोच्यताम्, अस्तु\\
स्ववेत्युच्यमाने स्वधावाचनीयेष्यपो निषिञ्चति ।\\
स्वधावाचनीयसंज्ञका ये कुशास्ते सपवित्राः साग्रत्वादिगुण-\\
विशिष्टाः कार्या इति कर्कः । हलायुधहेमाद्रिशूलपाणिप्रभृतयस्तु सपवित्रान्
=\\
अर्धसम्बन्धिपवित्रसहितानित्याहुः । मैथिलास्तु नार्धसम्बन्धिपवित्र\\
ग्रहणं, विना वचनं विनियुक्तविनियोगस्यायुक्तत्वात्, अन्यथा पिण्ड\\
रेखायामपि अर्धसम्बन्धिपवित्रस्य ग्रहणापत्तेः, तस्मात्पवित्रान्तरमे\\
वात्र ग्राह्यमित्याहुः । आस्तीर्य = भूमावितिशेषः । पिण्डानामुत्तरतो\\
दक्षिणाप्रानिति हेमाद्रिः । अत्र पितृभ्य इत्यादौ एकत्वेऽपि बहुवचनं\\
प्रयोगमात्रं ( १ ) पाशवत् । प्रकृत्यंशस्य तु समवेतार्थकत्वा नवदैवत्या-\\
दावूहसिद्धेः । स्वघोच्यतामिति च प्रत्येकमनुषञ्जनीयम् । ``स्वभाव-\\
मन एव च'' इति छन्दोगपरिशिष्टे तन्त्रतानिषेधात् । अत एव अस्तु स्वधेति\\
प्रत्युत्तरमपि प्रत्येकमेवोकम् ।\\
भविष्ये ।\\
अस्तु स्वधेति षड्वारान्कुर्युर्विप्राः सुसम्भ्रमा इति ।\\
मैथिलास्तु यथाश्रुतसुत्रानुसारात्सकृदेव स्वधोच्यतामस्तु स्व\\
धेति वक्तव्यमित्याहुः ।\\
अणं निषेचनं ``ऊर्जे वहन्तीत्यनेन मन्त्रेण कार्यमित्युक्तम् ।

% \begin{center}\rule{0.5\linewidth}{0.5pt}\end{center}

% \begin{center}\rule{0.5\linewidth}{0.5pt}\end{center}

{ }{दक्षिणादानविचारः । २७७}{\\
ब्रह्मपुराणे ।\\
सपवित्रान्कुशान्साग्रानास्तीर्य सतिलांस्ततः ।\\
ततः सम्भवपात्रेभ्यो जलमादाय चार्चितः ॥\\
ऊर्जं वहन्तीति जपन्पिण्डांश्चाप्यवसिञ्चति ।\\
पिण्डांवेति । चकारो दर्भसमुच्चयार्थः । अतश्चेदं शाखिविशेषव्य-\\
वस्थितम् । वाजसनेयिनां तु दर्भेष्वेव, कर्कहेमाद्रिरायमुकुटादिस्वरसो\\
ऽप्येवम् । मैथिलास्तु तेषां पिण्डेष्वेवेत्याहुः । छन्दोगानां तु
पिण्डेषु\\
पवित्राच्छादितेषुः `` पवित्रान्तर्हितान्पिण्डान्
सिञ्चेदुत्तानपात्रकृत्''\\
इति छन्दोगपरिशिष्टात् । इदं च निषेचनं प्रागुक्तब्रह्मवचनादर्घसंस्रव\\
जलेन कार्यमिति हेमाद्रिः । मिताक्षराकारस्तु कमण्डलु जलेनेत्याह ।
निषे}{\\
घ}{नानन्तरम् ।\\
कात्यायनः ।\\
उत्तानं पात्रं कृत्वा यथाशक्ति दक्षिणां दद्यात् ब्राह्मणेभ्यः ।\\
न्युब्जीकृतं पितुरर्धपात्रमुत्तानीकृत्य दक्षिणां दद्यात् । एवं च\\
``पवित्रान्तर्हितान्पिण्डान्सिञ्चेदुत्तानपात्रकृत्'' इति उत्तरत योज्यम्
।\\
अत एव ।\\
गोभिलोऽपि ।\\
अपो निषेचनानन्तरमुत्तानं पात्रं कृत्वा यथाशक्ति दक्षिणां\\
दद्यादित्याह ।\\
भट्टनारायणस्मार्त्तभट्टाचार्यादयस्तु उत्तानपात्रक्कत्वेदिति परिशि-\\
ष्टवचनं व्याख्यायोत्तानं पात्रं कृत्वा निषेचनं कुर्यादित्याहुः ।\\
अत्र च येषां मातामहपात्रस्यापि पूर्वं न्युब्जीकरणं कृतं तेषां मते\\
तस्यापि उत्तानता कार्या । यदा तु शूलपाण्यादिमतमाश्रित्य केवलं\\
पितृपात्रस्यैव म्युजीकरणं कृतं तदा तस्यैवोत्तानत्वमिति विवेकः ।\\
एवं (च ) नवदैवत्यादावपि बोध्यम् । केचित्तु त्रयाणामपि पात्राणां\\
न्युब्जीकरणमिच्छन्ति तन्मते सर्वेषामप्युधानत्वम् ।\\
अत्र च पित्र्युद्देशेन दक्षिणादाने कृतेऽपि `` ब्राह्मणान्भोजयित्वा तु\\
दद्याच्छत्त्वा च दक्षिणाम्'' इति वचनात्कर्मकरत्वाच्च ब्राह्मणानां\\
स्वत्वमुत्पद्यते, यथा देवताप्रतिष्ठादौ `` प्रतिष्ठाप्य देवतायै
सङ्कलिते\\
वस्त्राभरणादौ सर्वमुपकरणमाचार्वाय दद्यात्'' इति वचनात् कर्मक.\\
रत्वाच्चाचार्यस्य स्वत्वमुत्पद्यते एवमिहापीति स्मृतिचन्द्रिका । माधवमद-

{२७८ वीरमित्रोदयस्य श्राद्धप्रकाशे-}{\\
नरत्नादयस्तु ब्राह्मणोद्देशेन वा दक्षिणादानं पित्र्युद्देशेन वा । अत
एव\\
देवलपारस्कराभ्यां पक्षद्वयमुक्तं `` आचान्तेभ्यो द्विजेभ्यस्तु
प्रयच्छेद्द\\
क्षिणाम्'' इति, `` हिरण्यं विश्वेभ्यो देवेभ्यो रजतं पितृभ्य इति च'\\
तत्र यदा ब्राह्मणोद्देशेन तदा यज्ञोपवीतिना कार्यं `` सर्वे
कर्मापसव्येन\\
दक्षिणा दानवर्जितम् '' इति जमदग्निवचनात् । यत्तु तेनैवोक्तं अप-\\
सव्यं तत्रापि, `` मत्स्यो हि भगवान्मेने '' इति, तत्पितॄनुदिश्य
दानपक्षे\\
इत्याहुः । एतच्च दक्षिणादानम् ।\\
यत्र यत्क्रियते कर्म पैतृकं ब्राह्मणान्प्रति ।\\
तत्सर्वं तत्र कर्त्तव्यं वैश्वदेवस्यपूर्वकम् }{॥}{\\
इति देवलवचनाद्देवपूर्वकं कार्यमिति वाचस्पतिः । तन्न । ``दक्षिणां\\
पितृविप्रेभ्यो दद्यात्पूर्वं ततो द्वयोः'' इति देवलेनैव
दक्षिणापुरस्कारेण\\
विपरीतक्रमविधानात् ।\\
पितृभ्यः प्रथमं भक्त्या तन्मनस्को नरेश्वरः ।\\
सुस्ववेत्याशिषा युक्तां दद्याच्छक्त्या च दक्षिणाम् ॥\\
इति विष्णुस्मरणाच्च । न च देवलवचने पितृविप्रेभ्य इति श्र-\\
वणात् ब्राह्मणोद्देश्यकत्वावगतिः, तदुद्देश्यकदक्षिणादानपक्षेऽपि एत-\\
स्योकत्वात् तस्मादुभयथापि पित्राद्यैव देया, शूलपाणिपितृदयितादिष्वेवम्
।\\
अत्र च दक्षिणाद्रव्यपरिमाणादिकं द्रव्यनिर्णयप्रकरणे प्रागुक्तं वेदि-\\
तव्यम् । अत्र च ब्राह्मणोद्देशेन दक्षिणादानपक्षे अयं प्रयोगक्रमः, इदं\\
हिरण्यमग्निदैवतममुकगोत्रायामुकशर्मणे ब्राह्मणाय सम्प्रददे इति\\
पित्रादिसम्बन्धिब्राह्मणेभ्यः प्रत्येकं दद्यात् । `` दद्याच्चैव
द्विजातिभ्यः\\
प्रत्येकं वाक्यपूर्वक'' मिति भविष्योक्तेः, यदा तु पित्र्युद्देशेन
दक्षिणादानं\\
तदा अमुकगोत्रायास्मत्पित्रेऽमुकशर्मण इदं रजतं चन्द्रदैवतं स्वधे-\\
त्यादि वाक्यं बोध्यम् ।\\
दक्षिणादानानन्तरं -\\
कात्यायनः ।\\
विश्वेदेवाः प्रीयन्तामिति दैवे वाचयित्वा वाजे वाजेऽवत इति\\
विसृज्य आमावाजस्येत्यनुव्रज्य प्रदक्षिणीकृत्योपविशेत् ।\\
वाचनं च विश्वदेवाः प्रीयन्तामिति ब्रूहीति देवब्राह्मणप्रेरणरूपं\\
कार्यमिति भानूपाध्यायहलायुधप्रभृतयः ।\\
श्रीयन्तामिति तैश्चोक्ते देवताभ्य इति त्रिः पठनीयम् । आद्यावसान\\
इति प्रागुक्तब्रह्मपुराणवचनात् ।

{ }{ पिण्डप्रतिपत्तिनिर्णय: । २७९\\
}{भविष्येऽपि ।\\
ततो देवद्विजाभ्यां तु प्रार्थयेद् वचनं त्विह ।\\
विश्वेदेवाः प्रीयन्तामिति ब्रूतां युवामिति }{॥}{\\
तौ वदेतां प्रीयन्तामोमित्येव तु सकृद्वचः ।\\
ततः पठेद्देवताभ्यः पितृभ्यश्चेत्यमुं त्रिशः ॥\\
स्मार्त्तपितृदयितादयोऽप्येवम् ।\\
विसर्जनात्पूर्वकृत्यमाह -\\
याज्ञवल्क्यः ।\\
इत्युक्त्वोक्त्वा प्रिया वाचः प्रणिपत्य विसर्जयेत् ।\\
{[} अ० १ श्राद्धप्र० श्लो० २४७ {]}\\
भवद्भिरहं कृतार्थीकृत इत्यादि प्रिया वाचश्चोक्त्वा प्रणिपत्य\\
विसर्जयेदित्यर्थः ।\\
विसर्जनं च पितृपूर्वं कार्यमित्याह -\\
स एव ।\\
वाजे बाज इति प्रीतः पितृपूर्वं विसर्जयेत् । इति ।\\
अत्र च विसर्जनमुदकपात्रं गृहीत्वा कार्यमित्युकं -\\
मत्स्यपुराणे ।\\
ततस्तानप्रतो स्थित्वा प्रतिगृह्योदपात्रिकम् ।\\
वाजे बाज इति जपन् कुशाग्रेण विसर्जयेत् ॥\\
अत्र प्रतिब्राह्मणं मन्त्रावृति । कुशाग्रस्पर्शपूर्वकविसर्गस्य स\\
र्वविप्रविषवस्य युगपत्कर्तुमशक्यत्वादिति अपरार्क: । इति विसर्जनं
{[}च{]}\\
ब्राह्मणस्थपितॄणां न ब्राह्मणानां बेषामावाहनं तेषामेव विसर्जनस्य\\
युक्तत्वात्, अतश्चापात्रकश्राद्धेऽपि विसर्जनं भवस्येवेति स्मार्त्तः ।
इति\\
विसर्जनम् ।\\
तत्र वायुपुराणे ।\\
अथ पिण्डप्रतिपत्तिः ।\\
पिण्डमग्नौ सदा दद्याद् गोत्रार्थी सततं नरः ।\\
पत्न्यै प्रजार्थी दद्यातु मध्यमं मन्त्रपूर्वकम् ॥\\
उत्तमां गतिमन्विच्छन् गोभ्यो नित्यं प्रयच्छति ।\\
आज्ञां प्रज्ञां यशः कीर्त्तिमप्सु नित्यं निधापयेत् ॥\\
प्रार्थयेदीर्घमायुश्च वायसेभ्यः प्रयच्छति ।

{२८० वीरमित्रोदयस्य श्राद्धप्रकाशे-}{\\
आकाशं गमयेद् दिक्षु स्थितो वा दक्षिणामुखः ॥\\
पितॄणां स्थानमाकाशं दक्षिणादिक् तथैव च ।\\
गोत्र = कुलम् । मन्त्रश्च अपां त्वौषधीनां रसं प्राशयामि भूतकृतं\\
गर्भं वृत्स्वेति पितृभक्तौ लिखितः । पिण्डानां मध्यमं पत्नीं प्राशयेत्
।\\
`` आधत्त पितरो गर्भं कुमारं पुष्करस्रजम् । यथायमरपा असदित्या\\
श्ववलायनोक्तो वा ।\\
यमः\\
अप्स्वेकं प्लावयेत्पिण्डमेकं पत्न्यै निवेदयेत् ।\\
एकं च जुहुयादग्नौ त्रयः पिण्डाः प्रकीर्त्तिताः ।\\
पिण्डस्तु गोजविप्रेभ्यो दद्यादग्नौ जलेऽपि वा ।\\
विप्रान्ते वाथ विकिरेद् वयोभिरथवाऽऽशयेत् ॥\\
तीर्थश्राद्धे तु विशेषो-\\
विष्णुधर्मोत्तरे ।\\
तीर्थश्राद्धे सदा पिण्डान्क्षिपेत्तीर्थे समाहितः ।\\
अथोछिष्टोद्वासनम् ।\\
तत्र मनुः ।\\
उच्छेषणं भूमिगतमजिह्वस्याशठस्य च ।\\
दासवर्गस्य तत्पित्र्ये भागधेयं प्रचक्षते ॥

{ }{{[} अ० ३ श्लोक २४६ {]}}{\\
दासा मृताः ।

{ब्राह्मे ।}{\\
अस्तङ्गते ततः सूर्ये विप्रपात्राणि चाम्भसि ।\\
निक्षिपेत्प्रयतो भूत्वा सर्वाण्यधोमुखान्यपि ॥\\
द्वितीयेऽहनि सर्वेषां भाण्डानां क्षालनं तथा ।\\
परदिने क्षालनं च पात्रान्तरसत्वे । अस्तं गत = इत्यपि गृहान्त-\\
रसत्वे ।\\
याज्ञवल्क्यः ।\\
सत्सु विप्रेषु द्विजोच्छिष्टं न मार्जयेत् ।\\
{[} अ० १ श्राद्धप्र० श्लो० २५७ {]}\\
ब्रह्माण्डे ।\\
शूद्राय चानुपेताय श्राद्धोच्छिष्टं न दापयेत् ।\\
कामं दद्याच्च सर्वं तु शिष्याय च सुताय च ।\\
जातूकर्ण्यः ।

{ }{ श्राद्धोत्तरकर्मनिरूपणम् । २८१}{\\
द्विजभुक्तावशिष्टं तु सर्वमेकत्र संहरेत् ।\\
शुचिभूमौ प्रयत्नेन निखन्याच्छादयेद् बुध ।\\
अथ श्राद्धोत्तरं कर्म ।\\
तत्र - देवलः ।\\
निवृत्ते पितृमेधे तु दीपं प्रच्छाद्य पाणिना ।\\
आचम्य पाणिं प्रक्षाल्य ज्ञातीञ्छेषेण तोषयेत् ॥\\
प्रच्छाय = निर्वाप्य । अत्र विशेषो -\\
मात्स्ये ।\\
निवृत्य प्रणिपत्याथ पर्युक्ष्याग्निं समन्त्रवित् ।\\
वैश्वदेवं प्रकुर्वीत नैत्यक बलिमेव च ।\\
इदं च श्राद्धोत्तरं वैश्वदेवकरणं निरग्निविषयम् ।\\
पक्षान्तं कर्म निर्वर्त्य वैश्वदेवं च साग्निकः ।\\
पितृयज्ञं ततः कुर्यात्ततोऽन्वाहार्यकं बुधः ॥\\
इति लौगाक्षिणा साग्निकस्य श्राद्धात्पूर्वमेव वैश्वदेवस्य विहितत्वात्\\
पक्षान्त - अन्वाधानम् । अन्वाहार्य = दर्शश्राद्धम् । अत्र च साग्निकपदं
श्रौता-\\
ग्निपरमिति हेमाद्रिः । साग्निकस्यापि क्वचिदपवाद उक्तः -\/-\\
परिशिष्टे ।\\
सम्प्राप्ते पार्वणे श्राद्ध एकोदिष्टे तथैव च ।\\
अग्रतो वैश्वदेवः स्यात्पश्चादेकादशेऽहनि }{॥}{\\
यद्यपि चात्र न साग्निकश्रवणे, तथापि निरग्निकस्य सर्वदैव\\
श्राद्धोत्तरं वैश्वदेविकविधानादिदं साग्निकपरमेव युक्तम् ।\\
निरग्निकस्य वैश्वदेवे कालान्तरमुकम् -\\
ब्रह्माण्डे ।\\
(१) वैश्वदेवं हुते त्वग्नौ अर्वाग् ब्राह्मणभोजनात् ।\\
जुहुयाद् भूतयज्ञादि श्राद्धं कृत्वा तु तत्स्मृतम् ॥\\
हुते अग्नौ = अग्नौ करणोत्तरमित्यर्थः ।\\
भविष्येऽपि ।\\
पितॄन्सन्तर्प्य विधिवद् बलिं दद्याद्विधानतः ।\\
वैश्वदेव ततः कुर्यात् पश्चाद् ब्राह्मणवाचनम् ।\\
बलि = विकरबलिः । अस्मिन्नपि पक्षे वैश्वदेवमात्रं तस्मिन्काले,

% \begin{center}\rule{0.5\linewidth}{0.5pt}\end{center}

% \begin{center}\rule{0.5\linewidth}{0.5pt}\end{center}

{२८२ वीरमित्रोदयस्य श्राद्धप्रकाशे-}{\\
भूतयज्ञादिकं श्राद्धान्त एवेति हेमाद्रिः । यद्यपि चास्मिन्पक्षद्वये
निरग्नि-\\
कग्रहणं नास्ति, तथापि साग्निकस्य लौगाक्षिणा पक्षान्तरस्योक्तत्वा\\
न्निरग्निकविषयमेवेति हेमाद्रिप्रभृतयः }{।}{\\
मार्कण्डेयपुराणे ।\\
नित्यक्रियां पितॄणां हि केचिदिच्छन्ति सत्तमाः ।\\
नित्याकया = नित्यश्राद्धम् । अयं च विकल्पो यत्र षट्पुरुषश्राद्धं\\
तत्र न भवति, यत्र तु एकोद्दिष्टादौ ततो न्यूनं, तत्र भवतीति व्यव-\\
स्थितो ज्ञेयः । अत एव\\
चमत्कारखण्डे ।\\
नित्यश्राद्धं न कुर्वीत प्रसङ्गाद्यत्र सिध्यति }{॥}{\\
श्राद्धान्तरे कृते ऽन्यत्र ( १ ) नित्यत्वात्तन्न द्वापयेत् ।\\
मैथिला अप्येवम् । गौड़ास्तु षट्पुरुषतृप्तेर्जातत्वादकरणे दोषाभा-\\
वमात्रं, करणे तु अभ्युदयः , षोडशिग्रहणवदित्याहुः ।\\
मात्स्ये ।\\
ततस्तु वैश्वदेवान्ते सभृत्यसुतबान्धवः ।\\
भुञ्जीतातिथिसंयुक्तः (२) सर्व पितृनिषेवितम् ॥\\
आतातपोऽपि ।\\
शेषमन्नमनुज्ञातो भुञ्जीत तदनन्तरम् ।\\
अत्र च यदि ते अभ्यनुज्ञानं प्रयच्छन्ति तदा तत्तेभ्यो दत्वाऽन्ना\\
न्तरेण स्वयं भुञ्जीतेति हेमाद्रिः । अतश्चैकादश्यादौ भोजनाभावेऽपि\\
न वैगुण्यमिति चिन्तामणिः ।\\
गौड़ास्तु रागतः प्राप्तभोजनस्यैवायं नियमोनार्यनुगमनवत् । तस्य\\
च प्रकरणात् श्राद्धाङ्गभूतस्योपकाराकाङ्क्षायां प्रतिपत्यर्हकृतार्थ-\\
श्राद्धीयद्रव्यार्थता विज्ञायते । अतश्च यस्यैव प्रतिपत्तिर्न क्रियते\\
तस्यैव वैगुण्यापत्तेः सर्वमेव भोक्तव्यम् । मांसेषु निमन्त्रितवत्\\
यजमानस्य न नियमः । `` देवान्पितॄन्समभ्यर्च्य खादन्मांसं न दो-\\
षभाक्'' इत्युक्तत्वात् । कृतान्वाधाने तु व्रताविरोधेन भोक्तव्यं\\
तस्य श्रौतत्वादिति हेमाद्रिः । एवं फलार्थेन तत्तद्वस्तुत्यागरूपेण

% \begin{center}\rule{0.5\linewidth}{0.5pt}\end{center}

% \begin{center}\rule{0.5\linewidth}{0.5pt}\end{center}

{ }{ आमहेमश्राद्धनिरूपणम् । २८३}{\\
व्रतेनाध्येतस्य शेषभोजनस्य बाधो ज्ञेयः । एकादश्यां तु ``आघ्राय\\
पितृसेवित'' मिति वचनादाघ्राणमित्याहुः ।\\
अन्ये तु शेषगुणकमर्थकर्मैवेद भोजन, न तु प्रतिपतिकर्म, तथा\\
सति ब्राह्मणानुमति वैयर्थ्यापत्तेः । अन्यदयिस्यान्यत्र विनियोगे हि\\
अनुमतिरपेक्षते न तु तेष्वेव विनियोगोपयोगिनि संस्कारे, तस्माद-\\
र्थकर्मैवैतदित्याहुः ।\\
इति श्रीमच्छकल सामन्तचक्रचूडामणिमरीचिमञ्जरीनीराजितचरण\\
कमलश्रीमन्महाराजाधिराजप्रतापरुद्रतनूजश्रीमहाराजमधुकर-\\
साहसूनुचतुरुदधिवलयवसुन्धराहृदयपुण्डरीकविकासदि-\\
नकरश्रीमन्महाराजाधिराजश्रीवीरसिंहोद्योजितहंसप-\\
ण्डितात्मजपरशुराममिश्रसुनुसकलविद्यापारावार.\\
पारीणधुरीणजगद्दारिद्र्यमहागजपारीन्द्रविद्व-\\
ज्जनजीवातुश्रीमन्मित्रमिश्रकृते विरमित्रोद\\
याभिधाननिबन्धे श्राद्धप्रकाशे प्रकृति-\\
भूतपार्वणश्राद्धनिर्णयः ॥\\
अथ तद्नुकल्पः ।\\
तत्रामहेमश्राद्धम् ॥

{कात्यायनः ।\\
आपद्यनग्नौ तीर्थे च प्रवासे पुत्रजन्मनि ।\\
आमश्राद्धं द्विजैः कार्यं शूद्रेण तु सदैव हि ॥

{तथा-\\
हेमश्राद्धं प्रकुर्वीत भार्यारजसि संक्रमे ।\\
बौधायनोऽपि ।\\
संक्रमेऽन्नद्विजाभावे प्रवासे पुत्रजन्मनि ।\\
हेमश्राद्धं संग्रहे च द्विजः शूद्रः सदा चरेत् ॥\\
अत्र सर्वत्र हेमविधिरामाभावे ज्ञेयः । अत्र प्रकृतित्वेनाऽऽमस्या-\\
न्तरङ्गत्वात् । `` आमान्नस्याप्यभावे तु श्राद्धं कुर्वीत बुद्धिमान् ।\\
धान्याच्चतुर्गुणेनैव हिरण्येन सुरोचिषा'' इति मरीचिवचनाच्च ।\\
पुत्रजन्मनि तु विशेषः संवर्तेनोक्तः ।\\


{२८४ वीरमित्रोदयस्य श्राद्धप्रकाशे-}{\\
पुत्रजन्मनि कुर्वीत श्राद्धं हेम्नैव बुद्धिमान् ।\\
न पक्केन न चाप्तेन कल्याणान्यभिकामयन् ॥\\
अत एव पुत्रजन्मनि हेम्नोऽनुकल्पत्वम् । ग्रहणप्रसवादिषु तु\\
अन्नाद्यभाव एतद्विधिर्ज्ञेयः । संक्रमाग्न्यभावयोस्तु स्वतन्त्रयोरेव\\
तन्निमित्तत्वम् । केचित्तु अग्न्यभावस्यापि पाकासम्भवकृतमेव निमि\\
त्तत्वमित्याहुः ।\\
आमपरिमाणमाह धर्म ।\\
आमं तु द्विगुणं प्रोक्तं हेम तद्वच्चतुर्गुणम् ।\\
अन्नाभावे द्विजातीनां ब्राह्मणस्य विशेषतः ॥\\
व्यासोऽपि ।\\
आमं ददतु कौन्तेय तदामं द्विगुणं भवेत् ।\\
त्रिगुणं चतुर्गुणं वापि नत्वेकगुणमर्पयेत् }{ ॥}{\\
अत्र हेम्नश्चतुर्गुणत्वमामापेक्षयेति हेमाद्रिः । अन्नापेक्षयेत्यन्ये ।
स्मृ\\
त्वर्थसारे तु सममपि आमादि देयमित्युक्तम् । आमश्राद्धे च पिण्डदाना\\
द्यपि आमेनैव कार्यम्, `` तेनाग्नौकरणं कुर्यात्पिण्डांस्तेनैव
निर्वपेदि-\\
ति'' शातातपवचनादिति गौडाः ।\\
अन्ये तु ।\\
आमश्राद्धं यदा कुर्यात्पिण्डदानं कथं भवेत् ।\\
गृहपाकात्समुद्धृत्य सक्तुभिः पायसेन वा ॥\\
पिण्डदानं प्रकुर्वीत हेमश्राद्धे कृते सति ॥\\
इति भविष्योत्तरादामहेमश्राद्धे भवत्येव द्रव्यान्तरेण पिण्डदानम् ।\\
यत्तु शातातपवचनं तत् तत्राप्यत्राभावे वेदितव्यमित्याहुः ।\\
एवं शूद्रकर्तृकेऽप्यामश्राद्धेअन्नेनैव पिण्डदानं कार्यम् ।\\
शूद्रस्तु गृहपाकेन तत्पिण्डान्निर्वपेत्तथा।\\
सक्तुर्मूलं फलं तस्य पायसं वा भवेत्स्मृतम् ॥\\
इति हेमाद्रिधृतभविष्यवचनात् । अत्र पिण्डदानग्रहणाद्विकिरादा-\\
वाममेवेति केचित् ।\\
वस्तुतस्तु ।\\
नामन्त्रणं नाग्नौकरणं विकिरो नैव विद्यते ।\\
तृप्तिः प्रश्नोऽपि नैवात्र कर्तव्यः केनचिद्भवेत् ॥

{ }{ ब्राह्मणानुकल्प निरूपणम् । २८५}{\\
तु विकल्पो विहितप्रतिषिद्धत्वाज्ञयेः । अत्र ब्राह्मणाल्लुब्धमामं
स्वगृहे\\
पक्त्वा स्वयमेव भोक्तव्यं न तु कार्यान्तर उपयोक्तव्यम् । क्षत्रिया-\\
दिलब्धं तु यथेष्टं विनियोज्यं, शुद्रलब्धस्य तु भोजन एव स्वीये प-\\
रकीये वा विनियोगो न तु यथेष्टम् । तथा च व्यासः षट्त्रिंशत्मते ।\\
हिरण्यामं तु श्राद्धीयं लब्ध्वं यत्क्षत्रियादितः ।\\
यथेष्टं विनियोज्यं स्याद् भुञ्जीयाद्ब्राह्मणः स्वयम् }{॥}{\\
आमं शूद्रस्य यत्किञ्चिच्छ्राद्धिकं प्रतिगृह्यते ।\\
तत्सर्वे भोजनायालं नित्यनैमित्तिके न तु ॥\\
आमादिश्राद्धे च प्राणाहुत्यादिकं लुप्यते ।\\
आमश्राद्धे मन्त्रेषु केषु केन चिदूहो मरीचिनोक्तः ।\\
आवाहने स्वधाकारे मन्त्रा ऊह्या विसर्जने ।\\
अन्यकर्मण्यनूह्याः स्युररामश्राद्धविधिः स्मृतः ॥ इति ।\\
आवाहनमन्त्रे पितॄन्हविषे अत्तव इत्यत्र स्वीकर्तव इत्यूहः }{।}{\\
विसर्जन मन्त्रे तृप्ता यातेत्यत्र तर्प्स्वतेत्यूहः।\\
स्वधाकारमन्त्रे `` नमोवः पितर इष इत्यत्रामायेत्यूह इति हेमाद्रिः ।\\
अन्ये तु ।\\
रसादिपदवदेतस्याशास्यान्नप्रतिपादकत्वान्न प्रदेयान्नप्रतिपाद\\
कत्वं तेन स्वधाकारपदेन त्यागमन्त्रो गृह्यते । तत्र चान्नपदस्याने\\
धान्यादिपदेनोहः कार्य इत्याहुः ।\\
इदं चामश्राद्धं ग्रहणादन्यत्र मृताहे द्विजानां न भवति ।\\
श्राद्धविघ्ने द्विजातीनामामश्राद्धं प्रकीर्त्तितम् ।\\
अमावास्यादिनियतं माससंवत्सरादृते ॥\\
इति हारीतवचनात् ।\\
दर्शे रविग्रहे पित्रोः प्रत्याब्दिकमुपस्थितम् ।\\
अन्नेनासम्भवे हेम्ना कुर्यादामेन वा सुतः }{॥}{\\
इति गोभिलेन ग्रहणे आमादेः प्रतिप्रसूतत्वाच्च । अत्र दर्शादिपदानि\\
उपलक्षणं पौर्णमास्यादेः । इत्यामादिश्राद्धम् ।\\
अथ ब्राह्मणानुकल्प. ।\\
तत्र ब्राह्मणाभावे अपात्रकं श्राद्धमिति मैथिलाः । तन्न ।\\
निधाय वा दर्भबटूनासनेषु समाहितः ।\\
प्रैषामुप्रैषसंयुक्तं विधानं प्रतिपादयेत् ॥

{૨८६ वीरमित्रोदयस्य श्राद्धप्रकाशे-}{\\
इति देवलवचनविरोधात् ।\\
ब्राह्मणानामसंपत्तौ कृत्वा दर्भमयान् द्विजान् ।\\
श्राद्धं कृत्वा विधानेन पश्चाद्विप्रे प्रदापयेत् ॥\\
इति समुद्रकरधृतभविष्यवचनाच्च । प्रैषानुप्रैषम् = उत्तरप्रत्युत्तरम् ।
कुश-\\
बहुप्रमाणं चोक्तं व्यासेन ।\\
पञ्चाशद्भिर्भवेद् ब्रह्मा तदर्द्धेन तु विष्टरः ।\\
तदर्द्धेनोपनयनं तदर्द्धेन द्विजः स्मृतः ।\\
राजमुकुटनिबन्धे ।\\
नवभिः सप्तभिर्वापि कुशपत्रैर्विनिर्मितः ।\\
सावित्र्यावर्त्तितग्रन्थिः कुशब्राह्मण उच्यते ।\\
ॐकारेण तु बध्नीयाद् द्विजः कुर्यात्कुशद्विजम् ।\\
यत्तु छन्दोगपरिशिष्टे ।\\
यज्ञवास्तुनि मुष्टयां च स्तम्बे दर्भबटौ तथा ।\\
दर्भसंख्या न विहिता विष्टरास्तरणेषु च }{॥}{\\
इत्यनियमः, सोऽपि विभवे सति पक्षान्तरमिति परिशिष्टप्रकाशः ।\\
कुशब्राह्मणे च निमन्त्रणं लुप्यत इति शूलपाणिः तत्तु तदभावे कुश-\\
मयं स्नापयित्वा निमन्त्रयेदिति गौडनिबन्धोदाहृतभविष्यविरोधा-\\
दुपेक्षणीयम् । इति ब्राह्मणानुकल्पः ।\\
अथ सांकल्पिकश्राद्धम् ।\\
संवर्तः ।\\
समग्रं यस्तु शक्नोति कर्तुं नैवेह पार्वणम् ।\\
अपि संकल्पविधिना काले तस्य विधीयते ॥\\
पात्रभोज्यस्य चान्नस्य त्यागः सङ्कल्प उच्यते ।\\
व्यासः ।\\
सांकल्पिकं यदा कुर्यान्न कुर्यात्पात्रपूरणम् ।\\
नावाहनं नाग्नौकरणं पिण्डं चैव न दापयेत् ।\\
पात्र = अर्घ्यार्थम् ।\\
तथाङ्गनुष्ठानासमर्थे प्रत्याह-\\
व्याघ्रः ।\\
अङ्गानि पितृयज्ञस्य यदा कर्त्तुं न शक्नुयात् ।

{\\
विकृतिश्राद्धनिर्णयः । २८७

{अनुकल्पान्तरमाह देवलः । }{\\
पिण्डमात्रं प्रदातव्यमलाभे द्रव्यविप्रयोः ।\\
श्राद्धेऽहनि तु संप्राप्ते भवेन्निरशनोऽपि वा ।\\
अद्भिर्वा चिप्रसंवादात्पितॄणां तृप्तिमावहेत् ।\\
हारीतः ।\\
अपि मूलफलैर्वापि तथाप्युदकतर्पणैः ।\\
अविद्यमाने कुर्वीत न तु प्राप्तं विलङ्घयेत् । प्राप्तं = कालम् ।

{भविष्ये ।\\
किञ्चिद्दद्यादशक्तश्चेदुदकुम्भादिकं द्विजे ।\\
अग्निना वा दहेत्कक्षं श्राद्धकाले समागते ।\\
तस्मिन्नोपवसेदहि जपेद्वा श्राद्धसंहिताम् ।\\
तिलैः सप्ताष्टभिर्वापि समवेतं जलाञ्जलिम् ॥\\
यतः कुतश्चित्संप्राप्य गोभ्यो वापि गवाह्निकम् ।\\
सर्वाभावे वनं गत्वा कक्षामूलप्रदर्शकः ।\\
पूर्वादिलोकपालानामिदमुच्चैः पठन्बुधः }{॥}{\\
न मेऽस्ति वित्तं न धनं न चान्यत्\\
श्राद्धोपयोग्यं स्वपि}{तॄ}{न्नतोऽस्मि }{॥}{\\
तृप्यन्तु भक्त्या पितरौ मयैतो\\
कृत्तौ भुजौ वर्त्मनि मारुतस्य ॥ इति । इत्यनुकरूपाः ।\\
अथ विकृतिश्राद्धनिर्णयः ।\\
तत्र श्राद्धं तावद् द्विविधं पावर्णमेकोद्दिष्टं च । तत्र पावर्णं\\
भेदास्तावद् याज्ञवल्क्येन-\\
अमावास्याष्टका वृद्धिः कृष्णपक्षोऽयनद्वयम् ।\\
द्रव्यं ब्राह्मणसंपत्तिर्विषुवत्सूर्य संक्रमः }{॥}{\\
व्यतीपातो गजच्छाया ग्रहणं चन्द्रसूर्ययोः ।\\
श्राद्धं प्रति रुचिश्चैव श्राद्धकालाः प्रकीर्तिताः ॥\\
इत्यनेनोक्ताः । अत्र चामावस्याश्राद्धस्येतरश्राद्धप्रकृतित्वं तत्रैव\\
प्रायशो बहूनामङ्गानामुपदिष्टत्वात् `` पार्वणेन विधानेन तदप्युक्तं ख\\
गाधिप'' इत्यादौ वचनातिदेशाच्च नावाहनमित्यादि लिङ्गाच्च\\
ज्ञेयम् । अष्टकाः = अमावास्यान्तमार्गशीर्षादिमासचतुष्टयभाद्रपद.\\
कृष्णाष्ठम्यः । यद्यपि चाश्वलायनेन भाद्रपदकृष्णाष्टम्यां हेमन्तशि-

{२८८ वीरमित्रोदयस्य श्राद्धप्रकाशे-}{\\
शिरयोश्चतुर्णामपरपक्षाणामष्टमीष्वष्टका इति , एतेन माध्यावर्षं\\
व्याख्यातमित्यादिना संज्ञान्तरं कृतं, तथापि ``प्रोष्ठपद्यष्टका भूयः\\
पितृलोके भविष्यतीत्यादीना तस्या अपि अष्टकात्वसिद्धिः ।\\
अत्र द्रव्यविशेष उक्तो ब्रह्मपुराणे ।\\
ऐन्द्रयां तु प्रथमायां च शाकैः संतर्प्ययेत्पितृम् ।\\
प्राजापत्यां द्वितीयायां मांसैः संतर्पयेत्पितॄन् }{॥}{\\
वैश्यदेव्यां तृतीयायामपूपैश्च यथाक्रमम् ।\\
वर्षासु मेध्यशाकैश्च चतुर्थ्यां चैव सर्वदा ।\\
एतस्य च द्रव्यस्य प्राधान्यमात्रं न तु निरपेक्षसाधनत्वं\\
शाकादीनां केवलानामभोज्यत्वात् घृतश्राद्धाङ्गवृतवत् । अत्र यत्सा\\
ग्निकं प्रति आश्वलायनादिभिः पशुना स्थालीपाकेन वेत्यादिना\\
होमप्रकार उक्तः, स संस्कारप्रकाश उक्तः । अत्र च कामकालौ विश्वे-\\
देवौ । ``अष्टम्यां कामकालौ'' इति शङ्खोक्तेः । अष्टकापूर्वोत्तरदिनयो-\\
रपि च श्राद्धं कार्यं, पूर्वेद्युः पितृभ्यो
दद्यादपरेद्युरन्ववयमित्याश्वलायन\\
वचनात् । अन्वष्टकासु विशेष उक्तः कात्यायनेन ।\\
अन्वष्टकासु नवभिः पिण्डैः श्राद्धमुदाहृतम् ।\\
पित्रादि मातृमध्यं च ततो मातामहान्तकम् ॥\\
यत्तु ब्रह्मपुराणे ।\\
`` अन्वष्टकासु क्रमशो मातृपूर्वं तदिष्यते'' इति मात्रादित्वमुक्तं,\\
तच्छाखाभेदेन व्यवस्थितिमिति पृथ्वीचन्द्रोदयः । इदं च जीवत्पितृ-\\
केणापि सपिण्डमेव कार्ये `` अम्बष्टकासु च स्त्रीणां श्राद्धं कार्यं
तथैव\\
च'' इत्युपक्रम्य - -\\
पिण्डनिर्वपणं कार्यं तस्यामपि नृसत्तम ।\\
इति हेमाद्रौ विष्णुधर्मोत्तरे पुनः पिण्डदानविधानस्य श्राद्धाविधिनैव\\
पिण्डदानप्राप्तौ जीवत्पित्तृकत्वादिना पिण्डनिवृत्तिप्रसक्तौ पिण्डदान\\
प्राप्त्यर्थत्वात्, अन्यथाऽऽनर्थक्यापत्तेरिति ।\\
अत्र सुवासिन्यपि भोजनीया।\\
भर्तुरग्ने मृता नारी सह दाहेन वा मृता ।\\
तस्या स्थाने नियुञ्जीति विप्रैः सह सुवासिनीम् ॥\\
इति अम्बष्टकां प्रक्रम्य वचनात् । अत्र च सपत्नमातुरपि श्राद्धं

{ }{ वृद्धिश्राद्धनिरूपणम् । २८९}{\\
पिण्डश्च, नामैक्ये तु द्विवचनादिप्रयोग इति नारायणवृत्तिः । एतच्च जी\\
वति भर्तरि कार्यम्, मृते तु तस्मिन् लुप्यते `` श्राद्धं नवभ्यां
कुर्यात्\\
तन्मृते भर्तेरि लुप्यत'' इति वचनादिति दाक्षिणात्याः । तदसत् । अस्य\\
वाक्यस्य क्वापि महानिबन्धेऽदर्शनात् । एतच्चानुपनीतोऽपि\\
कुर्यात् । तथाचेदमेव प्रक्रम्य -\\
मात्स्ये ।\\
एतच्चानुपनीतोऽपि कुर्यात्सर्वेषु पर्वसु । इत्यष्टकान्वष्टका
श्राद्धनिर्णयः ।\\
अथ वृद्धिश्राद्धम् ।\\
तन्निमितान्याह वृद्धगार्ग्यः ।\\
अग्न्याधानाभिषेकादाविष्टापूर्ते स्त्रिया ऋतौ ।\\
वृद्धिश्राद्धं प्रकुर्वीत आश्रमग्रहणे तथा ॥\\
ऋतावाद्य एव ।\\
गार्ग्योऽपि ।\\
पुत्रोत्पत्तिप्रतिष्ठासु तन्मौञ्जीत्यागबन्धने ।\\
चूड़ायां च विवाहे च वृद्धिश्राद्धं विधीयते ।\\
विष्णुपुराणेऽपि ।\\
कन्यापुत्रविवाहेषु प्रवेशे नववेश्मनः ।\\
नामकर्माणि बालानां चूड़ाकर्मादिके तथा ॥\\
सीमन्तोन्नयने चैव पुत्रादिमुखदर्शने ।\\
नान्दीमुखान्पितृगणान्पूजयेत्प्रयतो गृही ॥ इति ।\\
अत्र चूड़ाकर्मादिक इत्यादिपदात्, अनुक्तानां गर्भाधानोपनयना-\\
दीनामपि ग्रहणम् । पुत्रस्यादिमं मुखदर्शनं जन्म पुत्रजन्मेति यावत् ।\\
पूर्वोक्तगार्ग्यादिवचनेषु साक्षात्पुत्रजन्मन एवोपादानात् ।\\
वसिष्ठः ।\\
पुत्रजन्मविवाहादौ वृद्धिश्राद्धमुदाहृतम् । इति ।

{जाबालिः ।}{\\
यज्ञोद्वाहप्रतिष्ठासु मेखलाबन्धमोक्षयोः ।\\
पुत्रजन्मवृषोत्सर्गे वृद्धिश्राद्धं समाचरेत् ॥ इति ।\\
ब्रह्मपुराणे ।\\
कर्मण्यथाभ्युदयिके माङ्गल्ये चातिशोभने ।\\
जन्मन्यथोपनयने विवाहे पुत्रकस्य च ॥

{२९० वीरमित्रोदयस्य श्राद्धप्रकाशे-}{\\
पि}{तॄ}{न्नान्दीमुखानेव तर्पयेद्विधिपूर्वकम् ॥ इति ।\\
आभ्युदयिकं = स्वाभ्युदयार्थकर्म, राज्याभिषेकनवग्रहमखमहादानादि ।\\
माङ्गल्यं = गर्भाधानसीमन्तोन्नयनादि । अतिशोभन इति माङ्गल्यविशेष-\\
णं, तेन ब्राह्मण्यश्च वृद्धा जीवत्पत्यो जीवत्प्रजा
यद्यदुपदिशेयुस्तत्त-\\
त्कुर्युः, अथ खलूच्चावचा जनपदधर्मा ग्रामधर्माश्च तान् विवाहे प्रती-\\
यादित्याश्वलायनसूत्रानुमतस्याचारप्राप्तस्य फलवस्त्रादिभिर्गर्भिणीपू.\\
जनोत्सवादेर्विवाहाङ्गहरिद्रावन्दनादेश्च व्यावृत्तिः ।\\
मत्स्य पुराणेऽपि ।\\
उत्सवानन्दसंताने यज्ञोद्वाहादिमङ्गले ।\\
मातरः प्रथमं पूज्याः पितरस्तदनन्तरम् }{॥}{\\
ततो मातामहा राजन्विश्वेदेवास्तथैव च }{॥}{\\
उत्सवः = पुत्र जन्म । आनन्दः = पुंसवनादिः । यज्ञो = ज्योतिष्टोमादिः ।\\
मातरः प्रथममिति मातॄणां पित्रादिश्राद्धात्पूर्वं कर्त्तव्यता ।
विश्वेदेवा इति\\
न क्रमपरम् । इदं चोपलक्षणं वैदिककर्ममात्रस्य, अत एवोक्तक तत्रैव,\\
`` नानिष्टा तु पितॄन् श्राद्धे कर्म वैदिकमाचरेत्'' इति । उपलक्षणत्वे\\
ऽपि च पूर्वोक्तगार्ग्यवचने पुत्रोत्पत्तीति ग्रहणात्पुत्रोत्पत्तिरेव
निमित्तं\\
न तु जातकर्म, तत्र -\\
नाष्टकासु भवेच्छ्राद्धं न श्राद्धे श्राद्धमिष्यते ।\\
न सोष्यन्तीजातकर्मप्रेषितागतकर्मसु ॥\\
इति छन्दोगपरिशिष्टे तन्निषेधात् । अतश्च ग्रहणवनैमित्तिकं श्राद्धं\\
पुत्रजन्मन्यपि भवत्येव न तु कर्माङ्गम् । अत एव -\\
निषेककाले सोमे च सीमन्तोन्नयने तथा ।\\
ज्ञेयं पुंसवने श्राद्धं कर्माङ्ग वृद्धिवत्कृतम् }{॥}{\\
इति पारस्करवचने कर्माङ्गश्राद्धविषयत्वेन जातकर्म नोपाप्तं किन्तु\\
वृद्धिवदिति पुत्रजन्मनिमित्तं श्राद्धं दृष्टान्तत्वेन पृथगेवोपात्तम् ।
अतएव-\\
यज्ञोद्वाहप्रतिष्ठासु मेखलाबन्धमोक्षयोः ।\\
पुत्रजन्मवृषोत्सर्गे-\\
इति जाबालिवचनेऽपि पुत्रजन्मैव गृहीतम् ।\\
अत्राधिकारी विष्णुपुराणे ।\\
जातस्य जातकर्मादि क्रियाकाण्डमशेषतः ।\\
पुत्रस्य कुर्वीत पिता श्राद्धं चाभ्युदयात्मकम् ॥

{ }{ वृद्धिश्राद्धनिरूपणम् । २९१\\
कात्यायनः ।\\
स्वपितृभ्यः पिता दद्यात् सुतसंस्कारकर्मसु ।\\
पिण्डानोद्वाहहनात्तेषां तस्याभावे तु तत्क्रमात् ॥ इति ।\\
तेषां = सुतानाम् \textbar{} ओद्वाहनात् = प्रथमविवाहपर्यन्तम् । पिता
स्वपितृभ्य-\/-\\
पिण्डान् = तदुपलक्षितं वृद्धिश्राद्धं कुर्यात् । तस्य पितुरभावे तु तस्य
सं-\/-\\
स्कार्य[स्य] पितॄणां यः क्रमः तेन क्रमेण पितृव्याचार्यमातुलादिः
श्राद्धं\\
दद्यात्, न स्वपितृभ्य- इति हेमाद्रिणा व्याख्यातम् । तत्र कालमाह -\\
वसिष्ठः ।\\
पूर्वेद्युर्मातृकं श्राद्धं कर्माद्दे पैतृकं तथा ।\\
उत्तरेद्युः प्रकुर्वीत मातामहगणस्य तु ॥ इति ।\\
अत्रिः ।\\
पूर्वाह्णे वै भवेद् वृद्धिर्विना जन्मनिमित्तकम् \textbar{}\\
पुत्रजन्मनि कुर्वीत श्राद्धे तात्कालिकं बुधः ॥ इति ।\\
अग्न्याधाननिमित्ते श्राद्धे विशेषकाल उक्तो गालवेन-\\
पार्वणं चापराद्धे तु वृद्धिश्राद्धं तथाग्निकम् । इति ।\\
आग्निकम् = अग्न्याधाननिमित्तं वृद्धिश्राद्धमपराहे कुर्यादिति । एक-\\
स्मिन्नपि दिने क्रियमाणानां कालभेदेनानुष्ठानं कार्यमित्याह\\
शातातपः ।\\
पूर्वाह्णे मातृकं श्राद्धमपराह्णे च पैतृकम् ।\\
ततो मातामहानां च वृद्धौ श्राद्धत्रयं स्मृतम् ॥\\
एवंविधस्यापि कालभेदस्यासम्भवे आह-\\
वृद्धमनुः ।\\
अलाभे भिन्नकालानां नान्दीश्राद्धत्रयं बुधः ।\\
पूर्वेद्युर्वै प्रकुर्वीत पूर्वाह्णे मातृपूर्वकम् ॥ इति ।\\
बौधायनसूत्रे \textbar{}\\
वेदकर्माणि प्रयोज्यन्पूर्वेद्युरेव युग्मान् ब्राह्मणान् भोजये.\\
दिति । नान्दीमुखा नवैता उक्ता भवन्ति, नैकाहेनैव दैवं पित्र्यं च\\
कुर्वीत यस्यैकाह्णा दैवं पित्र्य च कुर्वीत प्रजा अस्य प्रमायुका भव\\
न्ति तस्मात्पितृभ्यः पूर्वेद्युः करोति पितृभ्य एव तद्यज्ञं निःक्रीय\\
यजमानः प्रतनुत इति ब्राह्मणम् ।\\
अत्र ब्राह्मणवाक्येऽपि वृद्धिश्राद्धस्यास्य प्रजाः प्रमायु का अल्प.

\end{document}